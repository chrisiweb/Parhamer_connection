\section{31 - MAT - WS 1.1, WS 1.3 - Hallenbad - Matura 2013/14 Haupttermin}

\begin{langesbeispiel} \item[0] %PUNKTE DES BEISPIELS
				Das �rtliche Hallenbad einer kleinen Gemeinde ver�ffentlicht Anfang 2008 in der Gemeindezeitschrift eine Statistik �ber die j�hrlichen Besucherzahlen und die Anzahl der offenen Tage f�r die letzten acht Jahre:
					
					\resizebox{1\linewidth}{!}{\begin{tabular}{|l|c|c|c|c|c|c|c|c|}\hline
					Jahr&2000&2001&2002&2003&2004&2005&2006&2007\\ \hline
					Besucherzahlen&33\,200&32\,500&34\,000&33\,500&33\,200&33\,100&32\,900&32\,500\\ \hline
					offene Tage&197&192&200&195&193&190&186&180\\ \hline
					\end{tabular}}

Das Hallenbad bedarf einer Renovierung. Im Gemeinderat steht nun die Entscheidung an, ob Geld in das Hallenbad investiert oder das Hallenbad geschlossen werden soll. Im Vorfeld der Entscheidung ver�ffentlichen zwei �rtliche Gemeinderatsparteien - Partei $A$ und Partei $B$ - folgende Diagramme in ihren Parteizeitschriften:

\begin{center}
\textbf{Besucherzahlen 2000-2007}\leer

\resizebox{0.5\linewidth}{!}{\psset{xunit=1.0cm,yunit=0.15cm,algebraic=true,dimen=middle,dotstyle=o,dotsize=5pt 0,linewidth=0.8pt,arrowsize=3pt 2,arrowinset=0.25}
\begin{pspicture*}(-1.6848,-4.089123900293161)(8.8938,44.10412206744805)
\multips(0,0)(0,5.0){10}{\psline[linestyle=dashed,linecap=1,dash=1.5pt 1.5pt,linewidth=0.4pt,linecolor=lightgray]{c-c}(0,0)(8.8938,0)}
\multips(0,0)(1.0,0){11}{\psline[linestyle=dashed,linecap=1,dash=1.5pt 1.5pt,linewidth=0.4pt,linecolor=lightgray]{c-c}(0,0)(0,44.10412206744805)}
\psaxes[labelFontSize=\scriptstyle,xAxis=true,yAxis=true,showorigin=false,ylabelFactor={\,000}, Ox=1999,Dx=1.,Dy=5.,ticksize=-2pt 0,subticks=2]{->}(0,0)(0.,0.)(8.8938,44.10412206744805)
\psline(1.,33.2)(2.,32.5)
\psline(2.,32.5)(3.,34.)
\psline(3.,34.)(4.,33.5)
\psline(4.,33.5)(5.,33.2)
\psline(5.,33.2)(6.,33.1)
\psline(6.,33.1)(7.,32.9)
\psline(7.,32.9)(8.,32.5)
\begin{scriptsize}
\psdots[dotsize=3pt 0,dotstyle=square*,dotangle=45](1.,33.2)
\psdots[dotsize=3pt 0,dotstyle=square*,dotangle=45](2.,32.5)
\psdots[dotsize=3pt 0,dotstyle=square*,dotangle=45](3.,34.)
\psdots[dotsize=3pt 0,dotstyle=square*,dotangle=45](4.,33.5)
\psdots[dotsize=3pt 0,dotstyle=square*,dotangle=45](5.,33.2)
\psdots[dotsize=3pt 0,dotstyle=square*,dotangle=45](6.,33.1)
\psdots[dotsize=3pt 0,dotstyle=square*,dotangle=45](7.,32.9)
\psdots[dotsize=3pt 0,dotstyle=square*,dotangle=45](8.,32.5)
\end{scriptsize}
\end{pspicture*}}

\textit{Partei $A$}
\end{center}

\begin{center}
\textbf{Besucherzahlen 2000-2007}\leer

\resizebox{0.5\linewidth}{!}{\psset{xunit=1.0cm,yunit=2cm,algebraic=true,dimen=middle,dotstyle=o,dotsize=5pt 0,linewidth=0.8pt,arrowsize=3pt 2,arrowinset=0.25}
\begin{pspicture*}(0,31.9)(8.8938,34.4)
\multips(0,32.2)(0,0.4){10}{\psline[linestyle=dashed,linecap=1,dash=1.5pt 1.5pt,linewidth=0.4pt,linecolor=lightgray]{c-c}(2,0)(8.8938,0)}
\multips(2,0)(1.0,0){11}{\psline[linestyle=dashed,linecap=1,dash=1.5pt 1.5pt,linewidth=0.4pt,linecolor=lightgray]{c-c}(0,32.2)(0,44.10412206744805)}
\psaxes[labelFontSize=\scriptstyle,xAxis=true,yAxis=true,showorigin=false,labels=x,Ox=1999,Dx=1.,Dy=0.4,ticksize=-2pt 0]{->}(2,32.2)(2.,32.45)(8.8938,44.10412206744805)
\pszigzag[coilarm=0.125,coilwidth=0.3,coilheight=0.5](2,32.2)(2,32.5)
\psline(3.,34.)(4.,33.5)
\psline(4.,33.5)(5.,33.2)
\psline(5.,33.2)(6.,33.1)
\psline(6.,33.1)(7.,32.9)
\psline(7.,32.9)(8.,32.5)
\begin{scriptsize}
\psdots[dotsize=3pt 0,dotstyle=square*,dotangle=45](3.,34.)
\psdots[dotsize=3pt 0,dotstyle=square*,dotangle=45](4.,33.5)
\psdots[dotsize=3pt 0,dotstyle=square*,dotangle=45](5.,33.2)
\psdots[dotsize=3pt 0,dotstyle=square*,dotangle=45](6.,33.1)
\psdots[dotsize=3pt 0,dotstyle=square*,dotangle=45](7.,32.9)
\psdots[dotsize=3pt 0,dotstyle=square*,dotangle=45](8.,32.5)
\rput[tl](1,32.65){32\,400}
\rput[tl](1,33.05){32\,800}
\rput[tl](1,33.45){33\,200}
\rput[tl](1,33.85){33\,600}
\rput[tl](1,34.25){34\,000}
\end{scriptsize}
\end{pspicture*}}

\textit{Partei $B$}
\end{center}


\subsection{Aufgabenstellung:}
\begin{enumerate}
	\item Gib f�r jede Partei eine passende Botschaft an, wie mit dem jeweiligen Diagramm bezogen auf die Entwicklung der Besucherzahlen transportiert werden soll!\leer
		
		Partei $A$: \rule{5cm}{0.3pt}\leer
		
		Partei $B$: \rule{5cm}{0.3pt}\leer
		
		\item Partei $B$ hat bei der grafischen Darstellung verschiedene Manipulationen eingesetzt, um die Entwicklung der Besucherzahlen aus ihrer Sicht darzustellen.
		
 Beschreibe zwei dieser angewandten Manipulationen!\leer

\item \framebox[5mm][c]{A} Ermittle die Besucherzahlen pro �ffnungstag (gerundet auf ein eine Nachkommastelle) f�r die entsprechenden Jahre!\leer

\resizebox{1\linewidth}{!}{\begin{tabular}{|l|c|c|c|c|c|c|c|c|}\hline
Jahr&2000&2001&2002&2003&2004&2005&2006&2007\\ \hline
BesucherInnen&&&&&&&&\\
pro Tag&&&&&&&&\\ \hline
\end{tabular}}\leer

Formuliere eine dazu passende Aussage in Bezug auf die bevorstehende Entscheidung im Gemeinderat!

						\end{enumerate}\leer
				
\antwort{
\begin{enumerate}
	\item \subsection{L�sungserwartung:} 
	
	Partei $A$:  Das Hallenbad muss renoviert werden, da die Besucherzahlen �ber die letzten Jahre ann�hernd konstant geblieben sind.
	
	Partei $B$:   Das Hallenbad soll nicht renoviert werden, da die Besucherzahlen in den letzten Jahren stark abgenommen haben.
	
	\subsection{L�sungsschl�ssel:}
	\begin{itemize}
		\item  Ein Punkt f�r eine richtige Aussage zu Partei $A$. 
		\item Ein Punkt f�r eine richtige Aussage zu Partei $B$.
		
		Zul�ssig sind auch andere Formulierungen, die den Kern der Aussagen treffen.
	\end{itemize}
	
	\item \subsection{L�sungserwartung:}
	\begin{itemize}
		\item Die senkrechte Achse beginnt bei null, allerdings ist der erste Abschnitt bis zum ersten  Skalierungswert verk�rzt dargestellt. 
		\item �nderung/"'Verfeinerung"' der Skalierung
		\item Die x-Achsen-Skala beginnt mit 2002, daher f�llt "`der ansteigende Teil"' in der Graphik weg (vgl. Anstieg der Besucherzahlen lt. Partei $A$).
	\end{itemize}
	
	\subsection{L�sungsschl�ssel:}
	Zwei Punkte: je ein Punkt f�r eine (sinngem��) korrekt angef�hrte Manipulation.
	
	\item \subsection{L�sungserwartung:} 
	\resizebox{1\linewidth}{!}{\begin{tabular}{|l|c|c|c|c|c|c|c|c|}\hline
Jahr&2000&2001&2002&2003&2004&2005&2006&2007\\ \hline
BesucherInnen&\multirow{2}{*}{168,5}&\multirow{2}{*}{169,3}&\multirow{2}{*}{170,0}&\multirow{2}{*}{171,8}&\multirow{2}{*}{172,0}&\multirow{2}{*}{174,2}&\multirow{2}{*}{176,9}&\multirow{2}{*}{180,6}\\
pro Tag&&&&&&&&\\ \hline
\end{tabular}}\leer

Investitionen in das Hallenbad lohnen sich, denn in den letzten acht Jahren stieg die Zahl der t�glichen Besucher/innen jedes Jahr an.

\subsection{L�sungsschl�ssel:}
\begin{itemize}
	\item Ein Ausgleichspunkt f�r die richtigen Werte in der Tabelle. Die Angabe einer Null nach dem Komma (z. B.: 170,0) kann entfallen.
	\item Ein Punkt f�r eine (sinngem��) korrekte Antwort
\end{itemize}
\end{enumerate}}
		\end{langesbeispiel}