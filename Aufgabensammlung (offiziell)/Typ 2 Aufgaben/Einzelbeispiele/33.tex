\section{33 - MAT - AN 1.2, AN 1.3. FA 5.3, FA 5.5  - Chemische Reaktionsgeschwindigkeit - Matura 2013/14 Haupttermin}

\begin{langesbeispiel} \item[0] %PUNKTE DES BEISPIELS
				Die Reaktionsgleichung $A\rightarrow B+D$ beschreibt, dass ein Ausgangsstoff $A$ zu den Endstoffen $B$ und $D$ reagiert, wobei aus einem Molek�l des Stoffes $A$ jeweils ein Molek�l der Stoffe $B$ und $D$ gebildet wird.\\
Die Konzentration eines chemischen Stoffes in einer L�sung wird in Mol pro Liter (mol/L) angegeben. Die Geschwindigkeit einer chemischen Reaktion ist als Konzentrations�nderung eines Stoffes pro Zeiteinheit definiert.\\
Die unten stehende Abbildung zeigt den Konzentrationsverlauf der Stoffe $A$ und $B$ bei der gegebenen chemischen Reaktion in Abh�ngigkeit von der Zeit $t$.\\
$c_A(t)$ beschreibt die Konzentration des Stoffes $A$, $c_B(t)$ die Konzentration des Stoffes $B$. Die Zeit $t$ wird in Minuten angegeben.\leer

\begin{center}
	\resizebox{0.5\linewidth}{!}{\psset{xunit=1.0cm,yunit=1.0cm,algebraic=true,dimen=middle,dotstyle=o,dotsize=5pt 0,linewidth=0.8pt,arrowsize=3pt 2,arrowinset=0.25}
\begin{pspicture*}(-0.54,-0.48)(6.64,5.68)
\multips(0,0)(0,1.0){7}{\psline[linestyle=dashed,linecap=1,dash=1.5pt 1.5pt,linewidth=0.4pt,linecolor=lightgray]{c-c}(0,0)(6.64,0)}
\multips(0,0)(1.0,0){8}{\psline[linestyle=dashed,linecap=1,dash=1.5pt 1.5pt,linewidth=0.4pt,linecolor=lightgray]{c-c}(0,0)(0,5.68)}
\psaxes[labelFontSize=\scriptstyle,xAxis=true,yAxis=true,Dx=1.,Dy=1.,ticksize=-2pt 0,subticks=2]{->}(0,0)(0.,0.)(6.64,5.68)[$t$,140] [\mbox{$c_A(t),c_B(t)$},-40]
\psplot[linewidth=1.2pt,plotpoints=200]{0}{6.640000000000011}{2.0^(-(x-2.0))}
\psplot[linewidth=1.2pt,plotpoints=200]{0}{6.640000000000011}{-2.0^(-(x-2.0))+4.0}
\rput[tl](0.3,3.9){$c_A$}
\rput[tl](3.3,3.48){$c_B$}
\antwort{\psline(1.,2.)(3.,2.)
\psline(3.,2.)(3.,3.5)
\pscustom[fillstyle=solid,opacity=0.8]{
\parametricplot{1.5707963267948966}{3.141592653589793}{0.6*cos(t)+3.|0.6*sin(t)+2.}
\lineto(3.,2.)\closepath}
\psellipse*[fillstyle=solid,opacity=1](2.750432900757689,2.249567099242311)(0.04,0.04)
\rput[tl](3.1,2.78){1,5}
\rput[tl](2.1,1.9){2}
\psline(1.,2.)(3.,3.5)}
\end{pspicture*}}
\end{center}

\subsection{Aufgabenstellung:}
\begin{enumerate}
	\item  \fbox{A} Ermittle anhand der Abbildung die durchschnittliche Reaktionsgeschwindigkeit des Stoffes $B$ im Zeitintervall $[1; 3]$! 
	
	F�r die gegebene Reaktion gilt die Gleichung $c_A'(t)=-c_B'(t)$. Interpretiere diese Gleichung im Hinblick auf den Reaktionsverlauf!
 
\item  Bei der gegebenen Reaktion kann die Konzentration $c_A(t)$ des Stoffes $A$ in Abh�ngigkeit von der Zeit $t$ durch eine Funktion mit der Gleichung $c_A(t)=c_0\cdot e^{k\cdot t}$ beschrieben werden.

Gib die Bedeutung der Konstante $c_0$ an! Argumentiere anhand des Verlaufs des Graphen von $c_A$, ob der Parameter $k$ positiv oder negativ ist! 

Leite eine Formel f�r jene Zeit $\tau$ her, nach der sich die Konzentration des Ausgangsstoffes halbiert hat! Gib auch den entsprechenden Ansatz an!

						\end{enumerate}\leer
				
\antwort{
\begin{enumerate}
	\item \subsection{L�sungserwartung:} 
	
	$\frac{1,5}{2}=0,75\,\frac{\text{Mol}}{\text{Liter}\cdot\text{Minute}}$
	
	\textit{M�gliche Interpretation:}
	
	Die Konzentration von $A$ nimmt zu jedem Zeitpunkt gleich stark ab, wie die Konzentration von $B$ zu diesem Zeitpunkt zunimmt.\\ 
Oder:\\ 
Die beiden Reaktionsgeschwindigkeiten sind zu jedem Zeitpunkt betragsm��ig gleich gro�.
 	
	\subsection{L�sungsschl�ssel:}
	\begin{itemize}
		\item Ein Ausgleichspunkt f�r das Ermitteln der durchschnittlichen Reaktionsgeschwindigkeit. Jedes Ergebnis, das im Intervall $[0,7; 0,8]$ liegt, ist als richtig zu werten. Die Einheit muss nicht angegeben werden. Falls ein richtiges Ergebnis mit einer falschen Einheit angegeben ist, so ist die Aufgabe als richtig gel�st zu werten. 
		\item Ein Punkt f�r eine (sinngem��) richtige Deutung der momentanen �nderungsraten der Konzentrationen der Stoffe $A$ und $B$.
	\end{itemize}
	
	\item \subsection{L�sungserwartung:}
		$c_0$ ist die Anfangskonzentration des Stoffes zum Zeitpunkt $t=0$. F�r den Ausgangsstoff $A$ ist der Graph (bzw. die Funktion) streng monoton fallend, d.h., der Parameter $k$ im Exponenten der Exponentialfunktion muss negativ sein.   
		
		\textit{M�glicher Ansatz:}
		
		$\frac{c_0}{2}=c_0\cdot e^{k\cdot\tau}$
		
		und/oder
		
		$\frac{1}{2}=e^{k\cdot\tau}$
		
		$\ln(\frac{1}{2})=k\cdot\tau$
		
		$\tau=\dfrac{\ln(\frac{1}{2})}{k}$ oder $\tau=\frac{-\ln(2)}{k}$

	\subsection{L�sungsschl�ssel:}
	
\begin{itemize}
	\item  Ein Punkt f�r die Deutung von $c_0$ und eine (sinngem��) korrekte Argumentation, warum der Parameter $k$ negativ ist.
	\item Ein Punkt f�r den richtigen Ansatz und das richtige Ergebnis f�r $\tau$.
	
	Auch der Ansatz $\frac{c_0}{2}=c_0\cdot e^{-k\cdot\tau}$ mit der L�sung $\tau=\frac{\ln(2)}{k}$ ist als richtig zu werten.
\end{itemize}
\end{enumerate}}
		\end{langesbeispiel}