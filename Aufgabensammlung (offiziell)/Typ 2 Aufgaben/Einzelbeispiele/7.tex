\section{07 - MAT - FA 1.4, WS 3.1 - Gl�cksrad - BIFIE Aufgabensammlung}

\begin{langesbeispiel} \item[0] %PUNKTE DES BEISPIELS
				Auf einem Jahrmarkt werden nach dem Drehen eines Gl�cksrades \EUR{0}, \EUR{1}, \EUR{2} oder \EUR{4} ausbezahlt.Jedes Mal, bevor das Rad gedreht wird, ist eine Spielgeb�hr $e$ (in \euro) zu entrichten.
				
Der Spielbetreiber hat f�r mathematisch Interessierte den Graphen einer sogenannten kumulativen Verteilungsfunktion $F$ mit $F(x)=P(X \leq x)$ angegeben. Die Zufallsvariable $X$ gibt dabei die Gr��e des auszuzahlenden Betrags an. Aus der Abbildung lassen sich die Wahrscheinlichkeiten f�r die einzelnen Auszahlungsbetr�ge ermitteln, wobei die Variable $x$ angibt, welche Werte die Zufallsvariable $X$ annimmt, d.h. wie gro� die einzelnen auszuzahlenden Betr�ge sind.

Bei diesem Spiel sind sie, wie oben angegeben, \EUR{0}, \EUR{1}, \EUR{2} oder \EUR{4}. 
				\leer
				\begin{center}
				\resizebox{0.8\linewidth}{!}{\psset{xunit=2cm,yunit=10cm,algebraic=true,dimen=middle,dotstyle=o,dotsize=5pt 0,linewidth=0.8pt,arrowsize=3pt 2,arrowinset=0.25}
\begin{pspicture*}(-0.7907949790794977,-0.15)(5.2,1.0757839248873753)
\psaxes[labelFontSize=\scriptstyle,xAxis=true,yAxis=true,Dx=1.,Dy=0.1,ticksize=-2pt 0,subticks=2]{->}(0,0)(-0.7907949790794977,-0.15)(5.2,1.0757839248873753)[x,140] [y,-40]
\psline[linewidth=2.pt](0.,0.2)(1.,0.2)
\psline[linewidth=2.pt](1.,0.5)(2.,0.5)
\psline[linewidth=2.pt](2.,0.9)(4.,0.9)
\psline[linewidth=2.pt](4.,1.)(6.,1.)
\psline[linewidth=2.pt](0.,0.)(-2.,0.)
\begin{scriptsize}
\psdots[dotstyle=*](0.,0.2)
\psdots(1.,0.2)
\psdots[dotstyle=*](1.,0.5)
\psdots(2.,0.5)
\psdots[dotstyle=*](2.,0.9)
\psdots(4.,0.9)
\psdots[dotstyle=*](4.,1.)
\psdots(6.,1.)
\psdots(0.,0.)
\psdots[dotstyle=*,linecolor=blue](-2.,0.)
\end{scriptsize}
\end{pspicture*}}
				\end{center}
				
\subsection{Aufgabenstellung:}
\begin{enumerate}
	\item Ermittle mithilfe der Graphik die Wahrscheinlichkeitsverteilung der Zufallsvariablen $X$ und trag die entsprechenden Werte in der nachstehenden Tabelle ein!
	\leer
	
	\begin{tabular}{|l|C{2cm}|C{2cm}|C{2cm}|C{2cm}|}\hline
	Ausbezahlungsbetrag $x$ in \euro&0&1&2&4\\ \hline
	P(X=x) &\antwort{0,2}&\antwort{0,3}&\antwort{0,4}&\antwort{0,1}\\ \hline
	\end{tabular}

Begr�nde, warum die Funktion $F$ monoton steigend ist und warum das Maximum von $F$ immer 1 sein muss!

\item Der Erwartungswert von $X$ betr�gt bei diesem Spiel \EUR{1,50}, d. h., im Mittel betr�gt der auszuzahlende Betrag \EUR{1,50}.

Versetze dich in die Lage des Spielbetreibers. Wie gro� w�hlst du den Betrag der Spielgeb�hr $e$ pro Drehung mindestens, wenn du die Gr��e des Erwartungswerts von $X$ kennst? Gib eine Begr�ndung f�r deine Wahl an!

Die Zufallsvariable $Y$ gibt die H�he des (tats�chlichen) Gewinns aus der Sicht 
der Spielerin/des Spielers an. 

Welche Werte $y$ wird der Gewinn in Abh�ngigkeit von $e$ bei den bekannten Auszahlungsbetr�gen \EUR{0}, \EUR{1}, \EUR{2} oder \EUR{4} annehmen?
 
Vervollst�ndige die Tabelle und gib die Wahrscheinlichkeitsverteilung von 
$Y$ an!
\leer

\begin{tabular}{|l|C{2cm}|C{2cm}|C{2cm}|C{2cm}|}\hline
	Gewinn $y$ in \euro&\antwort{$0-e$}&\antwort{$1-e$}&\antwort{$2-e$}&\antwort{$4-e$} \\ \hline
	$P(Y=y)$ &\antwort{0,2}&\antwort{0,3}&\antwort{0,4}&\antwort{0,1}\\ \hline
	\end{tabular}
\end{enumerate}
	
\antwort{\subsection{L�sungserwartung:}
\begin{enumerate}
	\item Tabelle: siehe oben!
	
	Eine kumulierte Verteilungsfunktion entsteht durch Summenbildung der Einzelwahrscheinlichkeiten. Da die Einzelwahrscheinlichkeiten definitionsgem�� immer gr��er oder gleich 0 sein m�ssen, ist die Funktion $F$ monoton steigend. Das Maximum muss 1 sein, da die Summe aller Einzelwahrscheinlichkeiten einer Wahrscheinlichkeitsverteilung 1 ergeben muss.
	
	\item Tabelle: siehe oben!
	
	$e$ muss gr��er als \EUR{1,50} sein, da der Spielbetreiber auf lange Sicht einen Gewinn und keinen Verlust erzielen m�chte, der sich aus $e-1,5>0$ errechnet.
\end{enumerate}}
\end{langesbeispiel}