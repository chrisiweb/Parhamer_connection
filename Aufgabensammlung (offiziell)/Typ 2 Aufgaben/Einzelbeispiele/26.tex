\section{26 - MAT - AG 2.1, FA 1.8, FA 1.7, FA 1.8, AG 2.1, FA 1.2 - Hohlspiegel - BIFIE Aufgabensammlung}

\begin{langesbeispiel} \item[0] %PUNKTE DES BEISPIELS
				In der Physik spricht man von einem kugelf�rmigen Hohlspiegel, wenn er Teil einer innenverspiegelten Kugel ist. Charakteristische Punkte beim Hohlspiegel sind der Mittelpunkt $M$ der Kugel, der Scheitelpunkt $S$ und der Brennpunkt $F$ des Spiegels.
				
Es gelten folgende Relationen (siehe untenstehende Abbildungen):

Brennweite f des Spiegels: $f=\overline{FS}=\frac{\overline{MS}}{2} (f>0)$\\
Radius der Kugel: $\overline{MS}=2\cdot f$

Die Entfernung eines Gegenstands $G$ (mit der H�he $G$) vom Scheitelpunkt $S$ wird mit $g\,(g>0)$ bezeichnet, die Entfernung des nach Reflexion der Strahlen am Spiegel entstehenden Bildes $B$ (mit der H�he $B$) vom Scheitel $S$ mit $b$.

Das Vorzeichen von b hat dabei die folgenden Bedeutungen:
\begin{itemize}
 \item $b>0:$ Es entsteht ein reelles Bild "`vor"' dem Spiegel, das auf einem Schirm aufgefangen werden kann.
 \item $b<0:$ Es entsteht ein virtuelles Bild "`hinter"' dem Spiegel.
\end{itemize}

Skizzen des Querschnitts:
\begin{itemize}
 \item linke Grafik: reelles Bild $B$ eines Gegenstandes $G$ $(b>0)$
 \item rechte Grafik: virtuelles Bild $B$ eines Gegenstandes $G$ $(b<0)$
\end{itemize}\leer

\meinlr{\vspace*{0,25cm}\resizebox{0.9\linewidth}{!}{\psset{xunit=0.05cm,yunit=0.05cm,algebraic=true,dimen=middle,dotstyle=o,dotsize=5pt 0,linewidth=0.8pt,arrowsize=3pt 2,arrowinset=0.25}
\begin{pspicture*}(-83.69299915175435,-53.047985659680364)(42.58989979676596,42.28322237008523)
\psplot[linewidth=1.2pt,linestyle=dashed,dash=7pt 3pt 5pt 3pt ]{-83.69299915175435}{33.36}{(--28.-0.*x)/4.}
\psline{->}(-26.64,7.)(-26.64,27.)
\psline{->}(-66.64,7.)(-66.64,-33.)
\psline(-26.64,-36.45787932722763)(33.36,-36.45787932722763)
\psline(33.36,-44.62912572977896)(-66.64,-44.62912572977896)
\parametricplot{-0.8409172341495976}{0.8916365003859895}{1.*40.*cos(t)+0.*40.*sin(t)+-6.64|0.*40.*cos(t)+1.*40.*sin(t)+7.}
\begin{scriptsize}
\rput[tl](0.24798661990915494,-39){g}
\rput[tl](-20,-45.61957983917915){b}
\rput[tl](-74.03607158510279,-8.972777791373158){B}
\rput[tl](-50,3.903125630828949){M}
\rput[tl](35.40910750361481,12.569599088080366){S}
\rput[tl](-9,3.903125630828949){F}
\rput[tl](-32.970409607395545,17){G}
\psdots[dotsize=3pt 0,dotstyle=*](-6.64,7.)
\psdots[dotsize=3pt 0,dotstyle=*](33.36,7.)
\psdots[dotsize=3pt 0,dotstyle=*](-46.64,7.)
\psdots[dotsize=3pt 0,dotstyle=*](33.36,7.)
\psdots[dotsize=3pt 0,dotstyle=triangle*,dotangle=270](33.36,-36.45787932722763)
\psdots[dotsize=3pt 0,dotstyle=triangle*,dotangle=270](33.36,-44.62912572977896)
\psdots[dotsize=3pt 0,dotstyle=triangle*,dotangle=90](-26.64,-36.45787932722763)
\psdots[dotsize=3pt 0,dotstyle=triangle*,dotangle=90](-66.64,-44.62912572977896)
\end{scriptsize}
\end{pspicture*}}}{\resizebox{1\linewidth}{!}{\psset{xunit=0.05cm,yunit=0.05cm,algebraic=true,dimen=middle,dotstyle=o,dotsize=5pt 0,linewidth=0.8pt,arrowsize=3pt 2,arrowinset=0.25}
\begin{pspicture*}(-55.71700837110291,-25.63433586625919)(90.1985553149613,49.44626045612646)
\psplot[linewidth=1.2pt,linestyle=dashed,dash=7pt 3pt 5pt 3pt ]{-55.71700837110291}{90.1985553149613}{(--28.-0.*x)/4.}
\parametricplot{-0.8409172341495976}{0.8916365003859895}{1.*40.*cos(t)+0.*40.*sin(t)+-6.64|0.*40.*cos(t)+1.*40.*sin(t)+7.}
\psline{->}(13.36,7.)(13.36,27.)
\psline{->}(73.36,7.)(73.36,47.)
\psline(13.36,0.)(32.5,0.)
\psline(33.5,0.)(73.36,0.)
\begin{scriptsize}
\rput[tl](35.40910750361481,12.569599088080366){S}
\rput[tl](-9,3.903125630828949){F}
\rput[tl](8,16.151591253639968){G}
\rput[tl](67,29.782294752644503){B}
\rput[tl](-50,3.903125630828949){M}
\rput[tl](20.704312885610655,-1){g}
\rput[tl](52.658257153768666,-1){b}
\psdots[dotsize=3pt 0,dotstyle=*](-6.64,7.)
\psdots[dotsize=3pt 0,dotstyle=*,linecolor=darkgray](-46.64,7.)
\psdots[dotsize=3pt 0,dotstyle=*,linecolor=darkgray](33.36,7.)
\psdots[dotsize=3pt 0,dotstyle=triangle*,dotangle=270](-66.64,-44.62912572977896)
\psdots[dotsize=3pt 0,dotstyle=triangle*,dotangle=90](13.36,0.)
\psdots[dotsize=3pt 0,dotstyle=triangle*,dotangle=270](32.5,0.)
\psdots[dotsize=3pt 0,dotstyle=triangle*,dotangle=270](73.36,0.)
\psdots[dotsize=3pt 0,dotstyle=triangle*,dotangle=90](33.5,0.)
\end{scriptsize}
\end{pspicture*}}}

Aufgrund physikalischer �berlegungen gelten unter bestimmten Bedingungen die Beziehungen $\frac{G}{B}=\frac{g}{b}$ und $\frac{G}{B}=\frac{g-f}{f}$. Daraus ergibt sich der Zusammenhang $\frac{1}{g}+\frac{1}{b}=\frac{1}{f}$.

Der Quotient $\frac{B}{G}$ bestimmt den Vergr��erungsfaktor; er ist bei einem reellen Bild positiv $(g>0 \text{ und } b>0)$ und bei einem virtuellen Bild negativ $(g>0 \text{ und } b<0)$.

\subsection{Aufgabenstellung:}
\begin{enumerate}
	\item Gib den Vergr��erungsfaktor $\frac{B}{G}$ f�r $f=40\,\text{cm}$ und $g=50\,\text{cm}$ an!
	
Gib ein Intervall f�r die Gegenstandsweite $g$ an, damit ein virtuelles Bild entsteht!

Begr�nde deine Antwort durch eine mathematische Argumentation!

\item Stellen Sie die Bildweite $b$ als Funktion der Gegenstandsweite $g$ bei konstanter Brennweite $f$ dar! Betrachte die F�lle $g = 2f$ sowie $g = f$ und gib die jeweilige Auswirkung f�r $b$ an!

Was kann mithilfe dieser Funktion �ber den Grenzwert von $b$ ausgesagt werden, wenn
$g>f$ ist und sich $g$ der Brennweite $f$ ann�hert? T�tige eine entsprechende Aussage
und begr�nde diese durch Betrachtung von Z�hler und Nenner!

\item Leite aus den gegebenen Beziehungen $\frac{G}{B}$ die oben angef�hrte Formel $\frac{1}{g}+\frac{1}{b}=\frac{1}{f}$ her!\\
Gib die notwendigen Umformungsschritte an!

Der Ausdruck $\frac{1}{b}$ kann als Funktion in Abh�ngigkeit von $g$ der Form $\frac{1}{b}(g)=a\cdot g^k+c$ betrachtet werden. Gib die Werte der Parameter $a$ und $c$ sowie des Exponenten $k$ f�r diesen Fall an!
	
						\end{enumerate}\leer
				
\antwort{\subsection{L�sungserwartung:}
\begin{enumerate}
	\item $\frac{1}{b}=\frac{1}{40}-\frac{1}{50}=\frac{1}{200}\rightarrow$ Bildweite $200\,\text{cm}=2\,\text{m}$
	
	$\frac{B}{G}=\frac{200}{50}=4\rightarrow$ vierfache Vergr��erung
	
	Bildweite negativ:
	
	Intervall f�r $g$: $(0;f)$ bzw. Angabe des Intervalls durch: $0<g<f$
	
	Akzeptiert wird auch der Bezug zur ersten Fragestellung mit $f=40$.
	
	Intervall f�r $g$: $(0;40)$ bzw. $0<g<40$
	
	Begr�ndung 1: Aus $b=\frac{g\cdot f}{g-f}$ folgt $g<f$.
	
	Begr�ndung 2: Aus $\frac{1}{b}=\frac{1}{f}-\frac{1}{g}$ folgt $g<f$, da der Kehrwert von $b$ dann gr��er ist als der Kehrwert von $f$.
	
	\item Funktion: $b(g)=\frac{f\cdot g}{g-f}$
	
	$b(2f)=2f$; Bildweite und Gegenstandsweite sind gleich gro� und entsprechen dem Radius
der Kugel. Erweiterung: Auch $G$ und $B$ sind gleich gro�. $b(f)$ existiert nicht; der Nenner hat den Wert 0.

(Auch die Form $b(g)=\dfrac{1}{\frac{1}{f}-\frac{1}{g}}$ ist als richtig zu werten.)

Ann�herung von $g$ an $f$ mit $g>f$:

Der Ausdruck $\frac{f\cdot g}{g-f}$ ist positiv; der Z�hler ist eine positive Zahl (auch: n�hert sich dem Wert $f�$), der Nenner ist positiv und n�hert sich dem Wert 0, daher wird $b$ immer gr��er (der Grenzwert ist unendlich - oder �hnliche Aussagen).

Anmerkung: Wenn die Form $b(g)=\dfrac{1}{\frac{1}{f}-\frac{1}{g}}$ verwendet wird, sind auch umgangssprachliche Formulierungen wie "`oben steht die positive Zahl 1, unten steht etwas Positives, das gegen 0 geht, daher ist der Grenzwert $+1$"' als richtig zu werten. Auch Argumente, bei denen teilweise oder immer "`oben"' statt "`Z�hler"' und "`unten"' statt "`Nenner"' (oder �hnliches) verwendet wird, sind als richtig zu werten.

\item Zwei m�gliche Umformungen werden angef�hrt:

Variante 1:

$\frac{g}{b}=\frac{g-f}{f} \rightarrow \frac{g}{b}=\frac{g}{f}-1$ $|:g$

$\frac{1}{b}=\frac{1}{f}-\frac{1}{g}$

Variante 2:

$\frac{g}{b}=\frac{g-f}{f}$ $|\cdot (b\cdot f)$

$g\cdot f=b\cdot g-f\cdot b$ $|:(b\cdot g\cdot f)$

$\frac{1}{b}=\frac{1}{f}-\frac{1}{g}$

Daraus ergibt sich direkt der angegebene Zusammenhang.

$\frac{1}{b}(g)=\frac{1}{f}-\frac{1}{g}\Rightarrow a=-1,k=-1,c=\frac{1}{f}$
	
			\end{enumerate}}
		\end{langesbeispiel}