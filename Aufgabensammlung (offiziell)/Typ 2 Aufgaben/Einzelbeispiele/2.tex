\section{02 - MAT - WS 2.3, WS 3.2, WS 3.3 - Aufnahmetest - BIFIE Aufgabensammlung}

\begin{langesbeispiel} \item[0] %PUNKTE DES BEISPIELS
Eine Universit�t f�hrt f�r die angemeldeten Bewerber/innen einen Aufnahmetest durch. Dabei werden zehn Multiple-Choice-Fragen gestellt, wobei jede Frage vier Antwortm�glichkeiten hat.
Nur eine davon ist richtig. Wer mindestens acht Fragen richtig beantwortet, wird sicher aufgenommen. Wer alle zehn Fragen richtig beantwortet, erh�lt zus�tzlich ein Leistungsstipendium.
Die Ersteller/innen dieses Tests geben die Wahrscheinlichkeit, bei zuf�lligem Ankreuzen aller Fragen aufgenommen zu werden, mit 0,04158\,\% an. Nimm an, dass Kandidat $K$ alle
Antworten v�llig zuf�llig ankreuzt.

\subsection{Aufgabenstellung:}

\begin{enumerate}
	\item Nenne zwei Gr�nde, warum die Anzahl der richtig beantworteten Fragen unter den vorliegenden Angaben binomialverteilt ist! 
	
Gib einen m�glichen Grund an, warum in der Realit�t das Modell der Binomialverteilung hier eigentlich nicht anwendbar ist!
	
	\item Geben Sie die Wahrscheinlichkeit an, dass Kandidat $K$ nicht aufgenommen wird! Berechnen Sie die Wahrscheinlichkeit, dass Kandidat $K$ ein Leistungsstipendium erh�lt!
\end{enumerate}

\antwort{\subsection{L�sungserwartung:}

\begin{enumerate}
	\item Dieser Aufgabenteil ist durch sinngem��es Angeben von mindestens zwei der vier angef�hrten Gr�nde richtig gel�st: \leer
	
Die Anzahl der richtig beantworteten Fragen ist unter den vorliegenden Angaben binomialverteilt, weil

\begin{itemize}
	\item es nur die beiden Ausg�nge "`richtig beantwortet"' und "`falsch beantwortet"' gibt
\item das Experiment unabh�ngig mit $n = 10$ Mal wiederholt wird
\item die Erfolgswahrscheinlichkeit dabei konstant bleibt
\item es sich dabei um ein "`Bernoulli-Experiment"' handelt
\end{itemize} 

Der zweite Aufgabenteil ist korrekt gel�st, wenn ein Grund (sinngem��) angef�hrt wird, z. B.: \leer

\begin{itemize}
	\item Eine Bewerberin/ein Bewerber, die/der sich f�r ein Studium interessiert, wird sicher
nicht beim Aufnahmetest zuf�llig ankreuzen.
\item Sobald Kandidat $K$ auch nur eine Antwortm�glichkeit einer Frage ausschlie�en kann, w�re die Voraussetzung f�r die Binomialverteilung verletzt. Genau aus diesem Grund
wird die Universit�t mit zehn Multiple-Choice-Fragen nicht das Auslangen finden, da die Erfolgswahrscheinlichkeit f�r kompetenzbasiertes Antworten sicher wesentlich h�her ist als 0,25.

\item Die Unabh�ngigkeit der Wiederholung des Zufallsexperiments ist sicher dadurch verletzt, dass die einzelnen Kandidatinnen und Kandidaten aufgrund ihrer Vorbildung
unterschiedliche Erfolgswahrscheinlichkeiten f�r die Beantwortung der einzelnen Fragen aufweisen. Somit kann unter diesen Voraussetzungen niemals von einer unabh�ngigen
Wiederholung mit Z�hlen der Anzahl der Erfolge im Sinne eines Bernoulli-Experiments gesprochen werden.
\end{itemize}

Es sind auch weitere eigenst�ndige L�sungen denkbar. \leer

\item F�r die L�sung ist keine Binomialverteilung n�tig, da das gesuchte Ereignis das Gegenereignis zur "`Aufnahme"' darstellt. Somit betr�gt die (von den Testautorinnen und Testautoren) angegebene Wahrscheinlichkeit:

\[P(\text{Ablehnung})=1-P(\text{Aufnahme})=1-0,0004158=0,9995842\]

Die Ablehnung des Kandidaten $K$ ist somit praktisch sicher.\leer

Auch hier ist keine Binomialverteilung n�tig, da ein Zufallsexperiment mit einer Erfolgswahrscheinlichkeit
von 0,25 zehnmal unabh�ngig wiederholt wird, wobei bei jeder Wiederholung ein "`Erfolg"' eintritt. \leer


Die Wahrscheinlichkeit betr�gt somit $P(\text{Leistungsstipendium})=0,25^{10}\approx 0$.
\end{enumerate}
}

				
\end{langesbeispiel}