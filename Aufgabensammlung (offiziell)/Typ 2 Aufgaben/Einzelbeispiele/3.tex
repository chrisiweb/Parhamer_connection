\section{03 - MAT - WS 1.1, WS 1.3, WS 3.1, WS 3.2, WS 3.3 - Section Control - BIFIE Aufgabensammlung}

\begin{langesbeispiel} \item[0] %PUNKTE DES BEISPIELS
				Der Begriff Section Control (Abschnittskontrolle) bezeichnet ein System zur �berwachung von Tempolimits im Stra�enverkehr, bei dem nicht die Geschwindigkeit an einem bestimmten Punkt gemessen wird, sondern die Durchschnittsgeschwindigkeit �ber eine l�ngere Strecke. Dies geschieht mithilfe von zwei �berkopfkontrollpunkten, die mit Kameras ausgestattet sind. 
Das Fahrzeug wird sowohl beim ersten als auch beim zweiten Kontrollpunkt fotografiert.
 
Die zul�ssige H�chstgeschwindigkeit bei einer bestimmten Abschnittskontrolle betr�gt $100\,km/h$. Da die Polizei eine Toleranz kleiner $3\,km/h$ gew�hrt, l�st die Section Control bei $103\,km/h$ aus. Lenker/innen von Fahrzeugen, die dieses Limit erreichen oder �berschreiten, machen sich strafbar und werden im Folgenden als "`Tempos�nder"' bezeichnet.

Eine Stichprobe der Durchschnittsgeschwindigkeiten von zehn Fahrzeugen ist in der nachfolgenden Tabelle aufgelistet und im abgebildeten Boxplot dargestellt.
				\leer
				
				\begin{tabular}{|c|c|c|c|c|c|c|c|c|c|c|} \hline
				$v$ in $km/h$&88&113&93&98&121&98&90&98&105&129 \\ \hline				
				\end{tabular}
				\leer
				
				\begin{center}\newrgbcolor{zzttqq}{0.6 0.2 0.}
\psset{xunit=0.3cm,yunit=0.3cm,algebraic=true,dimen=middle,dotstyle=o,dotsize=5pt 0,linewidth=0.8pt,arrowsize=3pt 2,arrowinset=0.25}
\begin{pspicture*}(86.08,-2.)(131.,7.765)
\multips(86,0)(1.0,0){45}{\psline[linestyle=dashed,linecap=1,dash=1.5pt 1.5pt,linewidth=0.4pt,linecolor=lightgray]{c-c}(0,-2.)(0,7.765)}
\psaxes[labelFontSize=\scriptstyle,xAxis=true,yAxis=true,Dx=2.,Dy=5.,ticksize=-2pt 0,subticks=2]{->}(0,0)(86.08,-2.)(131.,7.765)
\psframe[linecolor=zzttqq,fillcolor=zzttqq,fillstyle=solid,opacity=0.1](93.,1.0)(113.,5.)
\psline[linecolor=zzttqq,fillcolor=zzttqq,fillstyle=solid,opacity=0.1](88.,1.)(88.,5.)
\psline[linecolor=zzttqq,fillcolor=zzttqq,fillstyle=solid,opacity=0.1](129.,1.)(129.,5.)
\psline[linecolor=zzttqq,fillcolor=zzttqq,fillstyle=solid,opacity=0.1](98.,1.)(98.,5.)
\psline[linecolor=zzttqq,fillcolor=zzttqq,fillstyle=solid,opacity=0.1](88.,3.)(93.,3.)
\psline[linecolor=zzttqq,fillcolor=zzttqq,fillstyle=solid,opacity=0.1](113.,3.)(129.,3.)
\end{pspicture*}\end{center}

\subsection{Aufgabenstellung:}
\begin{enumerate}
	\item Bestimme den arithmetischen Mittelwert $\overline{x}$ und die empirische Standardabweichung $s$ der Durchschnittsgeschwindigkeiten in der Stichprobe!
	
	Kreuze die zutreffende(n) Aussage(n) zur Standardabweichung an!
	
	\multiplechoice[5]{  %Anzahl der Antwortmoeglichkeiten, Standard: 5
					L1={Die Standardabweichung ist ein Ma� f�r die Streuung um den arithmetischen Mittelwert.},   %1. Antwortmoeglichkeit 
					L2={Die Standardabweichung ist immer ca. ein Zehntel des arithmetischen Mittelwerts.},   %2. Antwortmoeglichkeit
					L3={Die Varianz ist die quadrierte Standardabweichung.},   %3. Antwortmoeglichkeit
					L4={Im Intervall $[\overline{x}-s;\overline{x}+2]$ der obigen Stichprobe liegen ca. $60\,\%$ bis $80\,\%$ der Werte.},   %4. Antwortmoeglichkeit
					L5={Die Standardabweichung ist der arithmetische Mittelwert der Abweichung von 	$\overline{x}$.},	 %5. Antwortmoeglichkeit
					L6={},	 %6. Antwortmoeglichkeit
					L7={},	 %7. Antwortmoeglichkeit
					L8={},	 %8. Antwortmoeglichkeit
					L9={},	 %9. Antwortmoeglichkeit
					%% LOESUNG: %%
					A1=1,  % 1. Antwort
					A2=3,	 % 2. Antwort
					A3=4,  % 3. Antwort
					A4=0,  % 4. Antwort
					A5=0,  % 5. Antwort
					}
		\item Bestimme aus dem Boxplot (Kastenschaubild) der Stichprobe den Median sowie das obere und untere Quartil! Gib an, welche zwei Streuma�e aus dem Boxplot ablesbar sind! Bestimme auf deren Werte!
		\item Die Erfahrung zeigt, dass die Wahrscheinlichkeit, ein zuf�llig ausgew�hltes Fahrzeug mit einer Durchschnittsgeschwindigkeit von mindestens $103\,km/h$ zu erfassen, $14\,\%$ betr�gt. Berechne den Erwartungswert $\mu$ und die Standardabweichung $\sigma$ der Tempos�nder unter f�nfzig zuf�llig ausgew�hlen Fahrzeuglenkern! Berechne, wie gro� die Wahrscheinlichkeit ist, dass die Anzahl der Tempos�nder unter f�nfzig Fahrzeuglenkern innerhalb der einfachen Standardabweichung um den Erwartungswert, d.h. im Intervall $[\mu-\sigma;\mu+\sigma]$ liegt!
\end{enumerate}

\antwort{\subsection{L�sungserwartung:}
\begin{enumerate}
	\item Richtige L�sungen siehe oben. Zus�tzliche Information:
	
	$$\overline{x}=\frac{1}{10}\cdot\sum_{i=1}^{10}{x_i=103,3}\,km/h$$
	$$s=\sqrt{\frac{1}{9}\cdot\sum_{i=1}^{10}{(x_i-\overline{x})�}}=13,6\,km/h$$
	\item Daten aus dem Boxplot:
		 \subitem Median ... $98\,km/h$
		 \subitem unteres Quartil ... $93\,km/h$
		\subitem oberes Quartil ... $113\,km/h$
		\subitem Spannweite ... $41\,km/h$
		\subitem Quartilsabstand ... $20\,km/h$
	\item L�sung mittels Binomialverteilung
	\subitem $\mu=n\cdot p=50\cdot 0,14=7$
	\subitem $\sigma=\sqrt{\mu\cdot (1-p)}=2,45$
	\subitem $P(\mu-\sigma<X<\mu+\sigma)=P(5\leq X\leq 9)=P(X=5)+P(X=6)+P(X=7)+P(X=8)+P(X=9)=0,1286+0,1570+0,1606+0,1406+0,1068=0,6936=69,36\,\%$
\end{enumerate}}
\end{langesbeispiel}