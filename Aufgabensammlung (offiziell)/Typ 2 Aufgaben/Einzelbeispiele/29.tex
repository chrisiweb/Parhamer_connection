\section{29 - MAT - AG 2.1, FA 1.2, AN 2.1, AN 3.3 - Aufnahme einer Substanz ins Blut - Saturn-V-Rakete - BIFIE Aufgabensammlung}

\begin{langesbeispiel} \item[0] %PUNKTE DES BEISPIELS
				Wenn bei einer medizinischen Behandlung eine Substanz verabreicht wird, kann die Konzentration der Substanz im Blut (kurz: Blutkonzentration) in Abh�ngigkeit von der Zeit $t$ in manchen F�llen durch eine sogenannte Bateman-Funktion c mit der Funktionsgleichung $c(t)=d\cdot (\textbf{\textit{e}}^{-a\cdot t}-\textbf{\textit{e}}^{-b\cdot t})$ mit den Parametern $a,b,d\in\mathbb{R}$ und $a,b,d>0$ modelliert werden. Die Zeit $t$ wird in Stunden gemessen, $t=0$ entspricht dem Zeitpunkt der Verabreichung der Substanz.\\
Die Bioverf�gbarkeit $f$ gibt den Anteil der verabreichten Substanz an, der unver�ndert in den
Blutkreislauf gelangt. Bei einer intraven�sen Verabreichung betr�gt der Wert der Bioverf�gbarkeit 1.\\
Das Verteilungsvolumen $V$ beschreibt, in welchem Ausma� sich die Substanz aus dem Blut ins Gewebe verteilt.\\
Der Parameter $d$ ist direkt proportional zur verabreichten Dosis $D$ und zur Bioverf�gbarkeit $f$, au�erdem ist d indirekt proportional zum Verteilungsvolumen $V$.\\
Die nachstehende Abbildung zeigt exemplarisch den zeitlichen Verlauf der Blutkonzentration in
Nanogramm pro Milliliter (ng/ml) f�r den Fall der Einnahme einer bestimmten Dosis der Substanz Lysergs�urediethylamid.

\begin{center}
\resizebox{0.8\linewidth}{!}{\psset{xunit=1.0cm,yunit=0.5cm,algebraic=true,dimen=middle,dotstyle=o,dotsize=5pt 0,linewidth=0.8pt,arrowsize=3pt 2,arrowinset=0.25}
\begin{pspicture*}(-0.5083674461248454,-1.6869948801997758)(9.20426795498609,9.811469816975478)
\multips(0,0)(0,2.0){6}{\psline[linestyle=dashed,linecap=1,dash=1.5pt 1.5pt,linewidth=0.4pt,linecolor=lightgray]{c-c}(0,0)(9.20426795498609,0)}
\multips(0,0)(1.0,0){10}{\psline[linestyle=dashed,linecap=1,dash=1.5pt 1.5pt,linewidth=0.4pt,linecolor=lightgray]{c-c}(0,0)(0,9.811469816975478)}
\psaxes[labelFontSize=\scriptstyle,xAxis=true,yAxis=true,Dx=1.,Dy=2.,ticksize=-2pt 0,subticks=2]{->}(0,0)(0.,0.)(9.20426795498609,9.811469816975478)
\psplot[linewidth=1.2pt,plotpoints=200]{0}{9.20426795498609}{19.5*(EXP(-0.4*x)-EXP(-1.3*x))}
\begin{scriptsize}
\rput[tl](0.25,9.355737984465483){Blutkonzentration $c(t)$ (in ng/ml)}
\rput[tl](6.5,-0.9508126892220918){Zeit $t$ (in Stunden)}
\rput[tl](4,4.4){c}
\end{scriptsize}
\end{pspicture*}}
\end{center}

Dieser zeitliche Verlauf wird durch die Bateman-Funktion $c$ mit den Parametern $d=19,5,
 a=0,4$ und $b=1,3$ beschrieben.\\
Der Graph der Bateman-Funktion n�hert sich f�r gro�e Zeiten $t$ wie eine Exponentialfunktion
asymptotisch der Zeitachse an.

\subsection{Aufgabenstellung:}
\begin{enumerate}
	\item Berechne f�r die in der Einleitung angegebene Bateman-Funktion denjenigen Zeitpunkt, zu dem die maximale Blutkonzentration erreicht wird! Gib dazu die Gleichung der entsprechenden Ableitungsfunktion und den Ansatz in Form einer Gleichung an!
	
Begr�nde allgemein, warum der Wert des Parameters $d$ in der Bateman-Funktion nur
die Gr��e der maximalen Blutkonzentration beeinflusst, aber nicht den Zeitpunkt, zu dem
diese erreicht wird!

\item Kreuze diejenige Formel an, die den Zusammenhang zwischen dem Parameter $d$ der
Bateman-Funktion und den in der Einleitung beschriebenen Gr��en $V$, $D$ und $f$ korrekt beschreibt! Der Parameter $\lambda$ ist dabei ein allgemeiner Proportionalit�tsfaktor.\leer

\multiplechoice[6]{  %Anzahl der Antwortmoeglichkeiten, Standard: 5
				L1={$d=\lambda\cdot\frac{D}{V\cdot f}$},   %1. Antwortmoeglichkeit 
				L2={$d=\lambda\cdot\frac{D\cdot V}{f}$},   %2. Antwortmoeglichkeit
				L3={$d=\lambda\cdot\frac{V\cdot f}{D}$},   %3. Antwortmoeglichkeit
				L4={$d=\lambda\cdot\frac{D\cdot f}{V}$},   %4. Antwortmoeglichkeit
				L5={$d=\lambda\cdot\frac{V}{D\cdot f}$},	 %5. Antwortmoeglichkeit
				L6={$d=\lambda\cdot\frac{f}{V\cdot D}$},	 %6. Antwortmoeglichkeit
				L7={},	 %7. Antwortmoeglichkeit
				L8={},	 %8. Antwortmoeglichkeit
				L9={},	 %9. Antwortmoeglichkeit
				%% LOESUNG: %%
				A1=4,  % 1. Antwort
				A2=0,	 % 2. Antwort
				A3=0,  % 3. Antwort
				A4=0,  % 4. Antwort
				A5=0,  % 5. Antwort
				}\leer
				
				Bei einem konstanten Wert des Parameters $d$ und der Bioverf�gbarkeit $f$ kann man die
verabreichte Dosis $D$ als Funktion des Verteilungsvolumens $V$ auffassen. Beziehe dich auf die von dir angekreuzte Formel und gib f�r die in der Einleitung dargestellte Bateman-Funktion und f�r den Fall einer intraven�sen Verabreichung die Funktionsgleichung $D(V)$ der Funktion $D$ an! Gib an, um welchen Funktionstyp es sich bei $D$ handelt!	
						\end{enumerate}\leer
				
\antwort{\subsection{L�sungserwartung:}
\begin{enumerate}
	\item $c(t)=19,5\cdot(\textbf{\textit{e}}^{-0,4\cdot t}-\textbf{\textit{e}}^{-1,3\cdot t})$\\
	$c'(t)=19,5\cdot(-0,4\cdot\textbf{\textit{e}}^{-0,4\cdot t}+1,3\cdot\textbf{\textit{e}}^{-1,3\cdot t})=0\Rightarrow t\approx 1,31$ Stunde
	
	M�gliche Begr�ndung:
	
	F�r die Berechnung des Zeitpunktes der maximalen Blutkonzentration muss die Gleichung
$c'(t)=0$ nach $t$ gel�st werden. Da der Parameter $d$ in der Funktion $c$ eine multiplikative
Konstante ist und bei der Berechnung der ersten Ableitung von $c$ unver�ndert bleibt, beeinflusst er nicht die L�sungsmenge dieser Gleichung. F�r die Berechnung der Gr��e der
maximalen Blutkonzentration muss in die Funktion $c$ eingesetzt werden, und deshalb beeinflusst der Wert von $d$ dieses Ergebnis.

oder:

$c'(t)=d\cdot(-a\cdot\textbf{\textit{e}}^{-a\cdot t}+b\cdot\textbf{\textit{e}}^{-b\cdot t})=0\Rightarrow t=\frac{\ln(a)-\ln(b)}{a-b}$
	
Der Parameter $d$ tritt in dieser Formel nicht auf, $t$ ist also von $d$ unabh�ngig.

\item L�sung Multiple Choice siehe oben

Die Funktionsgleichung lautet: $D(V)=\frac{19,5}{\lambda}\cdot V$; dabei handelt es sich um eine lineare Funktion.
\end{enumerate}}
		\end{langesbeispiel}