\section{67 - MAT - AN 1.1, FA 5.2 - Ebola - Matura 2015/16 2. Nebentermin}

\begin{langesbeispiel} \item[0] %PUNKTE DES BEISPIELS
	
 Ebola ist eine durch Viren ausgel�ste, ansteckende Krankheit. Die Ebola-Epidemie, die 2014 in mehreren L�ndern Westafrikas ausbrach, gilt nach der Weltgesundheitsorganisation WHO als die bisher schwerste Ebola-Epidemie. Der Verlauf der Epidemie wurde von der WHO genau beobachtet und dokumentiert.

In der nachstehenden Tabelle ist ein Auszug der Dokumentation der WHO f�r die Staaten Guinea, Liberia und Sierra Leone f�r drei Tage im September 2014 dargestellt. Angef�hrt ist jeweils die Gesamtanzahl der Erkrankten.

\begin{center}
	\begin{tabular}{|l|c|c|c|}\hline
	\cellcolor[gray]{0.9}Datum&6. September&13. September&20. September\\ \hline
	\cellcolor[gray]{0.9}Gesamtzahl der Erkrankten&4\,269&4\,963&5\,843\\ \hline	
	\end{tabular}
\end{center}
\begin{scriptsize}Datenquelle: http://www.who.int/csr/disease/ebola/situation-reports/en/ [20.09.2014].\end{scriptsize}

\subsection{Aufgabenstellung:}
\begin{enumerate}
	\item Gib die Bedeutung der Ausdr�cke $4\,963-4\,269$ und $\frac{4\,963-4\,269}{4\,269}$ im gegebenen Kontext an!
	
Mithilfe dieser Ausdr�cke kann auf Basis der Anzahl der Erkrankungen vom 6. September 2014 und vom 13. September 2014 die Anzahl der Erkrankungen vom 20. September 2014 vorhergesagt werden, wenn man ein lineares oder ein exponentielles Wachstumsmodell zugrunde legt. 

 Ermittle die Werte beider Wachstumsmodelle f�r den 20. September 2014, vergleiche sie mit den tats�chlichen Daten und gib an, welches der beiden Modelle zur Modellierung der Anzahl der Erkrankungen im betrachteten Zeitraum eher angemessen ist!
	
	\item  Mitte September 2014 zitierte die New York Times die Behauptung von Wissenschaftlern, die Epidemie k�nne 12 bis 18 Monate dauern; es k�nne allein bis Mitte Oktober 2014 bereits 20\,000 Infektionsf�lle geben.
	
	\begin{scriptsize}\begin{singlespace}Datenquelle: http://www.nytimes.com/2014/09/13/world/africa/us-scientists-see-long-fight-against-ebola.html [29.06.2016].\end{singlespace}\end{scriptsize}
	
	Die zeitliche Entwicklung der Anzahl von Erkrankungsf�llen bei einer Epidemie kann f�r einen beschr�nkten Zeitraum durch eine Exponentialfunktion beschrieben werden. Auf Basis der Anzahl der Erkrankten vom 6. September 2014 und vom 20. September 2014 soll die Anzahl der Erkrankungsf�lle in Form einer Exponentialfunktion $f$ mit $f(t)=a\cdot b^t$ modelliert werden. 
	
	Die Zeit $t$ wird dabei in Tagen ab dem 6. September 2014 gemessen, der Zeitpunkt $t=0$ entspricht dem 6.�September 2014
	
	\fbox{A} Gib den Wert von $b$ an!
	
	Ermittle, wie viele Tage nach dem 6. September 2014 die Anzahl der Erkrankten gem�� der Modellfunktion die Zahl 20\,000 �berschreitet, und vergleiche dein Resultat mit der Aussage der Wissenschaftler!  
	
\end{enumerate}
\antwort{
\begin{enumerate}
	\item \subsection{L�sungserwartung:} 

$4\,963-4\,269$ gibt die absolute Zunahme der Erkrankungen in dieser Woche an.

$\frac{4\,963-4\,269}{4\,269}$ gibt die relative Zunahme der Erkrankungen in dieser Woche an.\leer

prognostizierte Erkrankungen f�r den 20. September 2014:

lineares Modell: $4\,963+(4\,963-4\,269)=5\,657$

exponentielles Modell: $4\,963\cdot \left(\frac{4\,963-4\,269}{4\,269}+1\right)\approx 5\,770$\leer

Das exponentielle Modell ist eher angemessen, da es n�her beim tats�chlichen Wert von 5\,843 Erkrankungen liegt.

 
	\subsection{L�sungsschl�ssel:}
	\begin{itemize}
		\item Ein Punkt f�r eine (sinngem��) korrekte Deutung beider Ausdr�cke.
		\item Ein Punkt f�r die Angabe der beiden korrekten Werte und die Angabe der entsprechenden angemessenen Modellierung. 
		
		Toleranzintervall f�r den exponentiellen Wert: $[5\,450; 5\,960]$

	\end{itemize}
	
	\item \subsection{L�sungserwartung:}
			
M�gliche Vorgehensweise:

$f(0)=4\,269$

$f(14)=5\,843=4\,269\cdot b^{14}$

$b=\sqrt[14]{\frac{5\,843}{4\,269}}\approx 1,0227$\leer

$t=\dfrac{\ln\left(\frac{20\,000}{4\,269}\right)}{\ln(1,0227)}\approx 68,80$, also am 69. Tag nach dem 6. September 2014. Dieser Zeitpunkt ist Mitte November.

Die Aussage der Wissenschaftler, es k�nne bis Mitte Oktober 2014 bereits 20\,000 Erkrankungsf�lle geben, erscheint daher (nach vorliegendem Modell) nicht gerechtfertigt.

	
	\subsection{L�sungsschl�ssel:}
	
\begin{itemize}
	\item Ein Ausgleichspunkt f�r die richtige L�sung. 
	
	Toleranzintervall: $[1,02; 1,03]$  
	
	Die Aufgabe ist auch dann als richtig gel�st zu werten, wenn bei korrektem Ansatz das Ergebnis aufgrund eines Rechenfehlers nicht richtig ist. 
	\item  Ein Punkt f�r die richtige L�sung und einen (sinngem��) korrekten Vergleich. 
	
	Toleranzintervall: $[68; 70]$ 
	
	Die Aufgabe ist auch dann als richtig gel�st zu werten, wenn bei korrektem Ansatz das Ergebnis aufgrund eines Rechenfehlers nicht richtig ist.  
\end{itemize}

\end{enumerate}}
		\end{langesbeispiel}