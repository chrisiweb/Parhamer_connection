\section{WS 3.1 - 2 Testung - MC - BIFIE}

\begin{beispiel}[WS 3.1]{1} %PUNKTE DES BEISPIELS
Es werden zwei Tests $T_X$ und $T_Y$, bei denen man jeweils maximal zehn Punkte erwerben kann, auf ihre L�sungsh�ufigkeit untersucht. Bei mehr als f�nf Punkten gilt der jeweilige Test als bestanden. Die Zufallsvariablen $X$ und $Y$ beschreiben die Anzahl der erreichten Punkte. Die beiden untenstehenden Abbildungen zeigen jeweils die Verteilungen der beiden Variablen $X$ und $Y$.


\meinlr{\resizebox{0.9\linewidth}{!}{\psset{xunit=1.0cm,yunit=20.0cm,algebraic=true,dimen=middle,dotstyle=o,dotsize=5pt 0,linewidth=0.8pt,arrowsize=3pt 2,arrowinset=0.25}
\begin{pspicture*}(-0.6358431442285014,-0.03578357435278022)(10.593099860999192,0.3261276822185875)
\multips(0,0)(0,0.05){8}{\psline[linestyle=dashed,linecap=1,dash=1.5pt 1.5pt,linewidth=0.4pt,linecolor=lightgray]{c-c}(-0.6358431442285014,0)(10.593099860999192,0)}
\multips(0,0)(1.0,0){12}{\psline[linestyle=dashed,linecap=1,dash=1.5pt 1.5pt,linewidth=0.4pt,linecolor=lightgray]{c-c}(0,-0.03578357435278022)(0,0.3261276822185875)}
\psaxes[labelFontSize=\scriptstyle,xAxis=true,yAxis=true,Dx=1.,Dy=0.1,ticksize=-2pt 0,subticks=2]{->}(0,0)(0.,0.)(10.593099860999192,0.3261276822185875)[$x$,140] [$P(X=x)$,-40]
\psline[linewidth=1.2pt](1.,0.2)(1.,0.)
\psline[linewidth=1.2pt](2.,0.2)(2.,0.)
\psline[linewidth=1.2pt](3.,0.1)(3.,0.)
\psline[linewidth=1.2pt](8.,0.)(8.,0.15)
\psline[linewidth=1.2pt](9.,0.)(9.,0.2)
\psline[linewidth=1.2pt](10.,0.)(10.,0.2)
\end{pspicture*}}}{
\resizebox{0.9\linewidth}{!}{\psset{xunit=1.0cm,yunit=20.0cm,algebraic=true,dimen=middle,dotstyle=o,dotsize=5pt 0,linewidth=0.8pt,arrowsize=3pt 2,arrowinset=0.25}
\begin{pspicture*}(-0.6358431442285014,-0.03578357435278022)(10.593099860999192,0.3261276822185875)
\multips(0,0)(0,0.05){8}{\psline[linestyle=dashed,linecap=1,dash=1.5pt 1.5pt,linewidth=0.4pt,linecolor=lightgray]{c-c}(-0.6358431442285014,0)(10.593099860999192,0)}
\multips(0,0)(1.0,0){12}{\psline[linestyle=dashed,linecap=1,dash=1.5pt 1.5pt,linewidth=0.4pt,linecolor=lightgray]{c-c}(0,-0.03578357435278022)(0,0.3261276822185875)}
\psaxes[labelFontSize=\scriptstyle,xAxis=true,yAxis=true,Dx=1.,Dy=0.1,ticksize=-2pt 0,subticks=2]{->}(0,0)(0.,0.)(10.593099860999192,0.3261276822185875)[$y$,140] [$P(Y=y)$,-40]
\psline[linewidth=1.2pt](3,0.05)(3.,0.)
\psline[linewidth=1.2pt](4.,0.2)(4.,0.)
\psline[linewidth=1.2pt](5.,0.25)(5.,0.)
\psline[linewidth=1.2pt](6.,0.)(6.,0.25)
\psline[linewidth=1.2pt](7.,0.)(7.,0.2)
\psline[linewidth=1.2pt](8.,0.)(8.,0.05)
\end{pspicture*}}}
\leer


Kreuze diejenigen zwei Aussagen an, die aus den gegebenen Informationen ablesbar sind.


\multiplechoice[5]{  %Anzahl der Antwortmoeglichkeiten, Standard: 5
				L1={Mit Test $T_Y$ werden mehr Kandidatinnen/Kandidaten den Test bestehen
als mit Test $T_X$.},   %1. Antwortmoeglichkeit 
				L2={Beide Zufallsvariablen $X$ und $Y$ sind binomialverteilt.},   %2. Antwortmoeglichkeit
				L3={Die Erwartungswerte sind gleich: $E(X) = E(Y)$.},   %3. Antwortmoeglichkeit
				L4={Die Standardabweichungen sind gleich: $\sigma_X = \sigma_Y$.},   %4. Antwortmoeglichkeit
				L5={Der Test $T_X$ unterscheidet besser zwischen Kandidatinnen/Kandidaten
mit schlechteren und besseren Testergebnissen.},	 %5. Antwortmoeglichkeit
				L6={},	 %6. Antwortmoeglichkeit
				L7={},	 %7. Antwortmoeglichkeit
				L8={},	 %8. Antwortmoeglichkeit
				L9={},	 %9. Antwortmoeglichkeit
				%% LOESUNG: %%
				A1=3,  % 1. Antwort
				A2=5,	 % 2. Antwort
				A3=0,  % 3. Antwort
				A4=0,  % 4. Antwort
				A5=0,  % 5. Antwort
				}


\end{beispiel}