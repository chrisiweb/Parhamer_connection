\section{WS 1.2 - 3 Histogramm erstellen - OA - BIFIE}

\begin{beispiel}[WS 1.2]{1} %PUNKTE DES BEISPIELS
				Bei einer LKW-Kontrolle wurde bei 500 Fahrzeugen eine �berladung festgestellt. Zur Festlegung
des Strafrahmens wurde die �berladung der einzelnen Fahrzeuge in der folgenden Tabelle
festgehalten.\\

\begin{center}
\begin{tabular}{|c|c|c|}\hline
\multicolumn{2}{|c|}{�berladung (in kg)} & Anzahl der\\
von & bis & LKW\\ \hline
 & <1000 & 140 \\ \hline
1000 & <2000 & 240 \\ \hline
2000 & <3000 & 80 \\ \hline
3000 & <4000 & 40 \\ \hline
\end{tabular}
\end{center}


Zeichne ein Histogramm der Daten im vorgegebenen Koordinatensystem.\\

\psset{xunit=0.002cm,yunit=12.5cm,algebraic=true,dimen=middle,dotstyle=o,dotsize=5pt 0,linewidth=0.8pt,arrowsize=3pt 2,arrowinset=0.25}
\begin{pspicture*}(-921.7768661422257,-0.06294899804841189)(5511.515027925262,0.638065421069014)
\psaxes[labelFontSize=\scriptstyle,xAxis=true,yAxis=true,Dx=500.,Dy=0.04,ticksize=-2pt 0,subticks=2]{->}(0,0)(0.,0.)(5491.515027925262,0.638065421069014)[\footnotesize{�berladung (in kg)},130] [\footnotesize{relativer Anteil},-40]
\antwort{\psline[linewidth=1.6pt](0.,0.)(0.,0.28)
\psline[linecolor=red,linewidth=1.6pt](0.,0.28)(1000.,0.28)
\psline[linecolor=red,linewidth=1.6pt](1000.,0.28)(1000.,0.)
\psline[linecolor=red,linewidth=1.6pt](1000.,0.28)(1000.,0.48)
\psline[linecolor=red,linewidth=1.6pt](1000.,0.48)(2000.,0.48)
\psline[linecolor=red,linewidth=1.6pt](2000.,0.48)(2000.,0.16)
\psline[linecolor=red,linewidth=1.6pt](2000.,0.16)(2000.,0.)
\psline[linecolor=red,linewidth=1.6pt](2000.,0.16)(3000.,0.16)
\psline[linecolor=red,linewidth=1.6pt](3000.,0.16)(3000.,0.)
\psline[linecolor=red,linewidth=1.6pt](3000.,0.08)(4000.,0.08)
\psline[linecolor=red,linewidth=1.6pt](4000.,0.08)(4000.,0.)}
\end{pspicture*}
\end{beispiel}