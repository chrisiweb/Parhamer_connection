\section{WS 1.2 - 4 S�ulendiagramm - OA - BIFIE}

\begin{beispiel}[WS 1.2]{1} %PUNKTE DES BEISPIELS
				Bei einer Umfrage werden die 480 Sch�ler/innen einer Schule befragt, mit welchem Verkehrsmittel
sie zur Schule kommen. Die Antwortm�glichkeiten waren "`�ffentliche Verkehrsmittel"' (A),
"`mit dem Auto / von den Eltern gebracht"' (B) sowie "`mit dem Rad / zu Fu�"' (C). Folgendes
Kreisdiagramm zeigt die Ergebnisse:\\

\begin{center}
\kreisdiagramm\begin{tikzpicture}
\pie[color={black!10 ,black!20 , black!30, black!40}, %Farbe
text=pin %Format: inside,pin, legend
]
{33.3/C , 50/A , 16.7/B} %Werte
\end{tikzpicture}
\end{center}


Vervollst�ndige das folgende S�ulendiagramm anhand der Werte aus dem obenstehenden Kreisdiagramm.\\

\newrgbcolor{cqcqcq}{0.7529411764705882 0.7529411764705882 0.7529411764705882}
\newrgbcolor{uuuuuu}{0.26666666666666666 0.26666666666666666 0.26666666666666666}
\psset{xunit=0.005cm,yunit=0.03cm,algebraic=true,dimen=middle,dotstyle=o,dotsize=5pt 0,linewidth=0.8pt}
\begin{pspicture*}(-200,-37.22464384593907)(2883.698063496676,288.43425453542056)
\multips(0,0)(0,10.0){33}{\psline[linestyle=dashed,linecap=1,dash=1.5pt 1.5pt,linewidth=0.4pt,linecolor=lightgray]{c-c}(0,0)(2883.698063496676,0)}
\multips(0,0)(5000.0,0){1}{\psline[linestyle=dashed,linecap=1,dash=1.5pt 1.5pt,linewidth=0.4pt,linecolor=lightgray]{c-c}(0,0)(0,288.43425453542056)}
\psaxes[labelFontSize=\scriptstyle,xAxis=true,yAxis=true,Dx=5000.,Dy=50.,ticksize=-2pt 0,subticks=2](0,0)(0.,0.)(2883.698063496676,288.43425453542056)
\psline[linewidth=1.6pt,linestyle=dotted](300.,30.)(300.,0.)
\psline[linewidth=1.6pt,linestyle=dotted](600.,30.)(600.,0.)
\psline[linewidth=1.6pt,linestyle=dotted](1200.,30.)(1200.,0.)
\psline[linewidth=1.6pt,linestyle=dotted](1500.,30.)(1500.,0.)
\psline[linewidth=1.6pt,linestyle=dotted](2100.,30.)(2100.,0.)
\psline[linewidth=1.6pt,linestyle=dotted](2400.,30.)(2400.,0.)
\rput[tl](402.76936491868076,22.091350006265078){?}
\rput[tl](1314.614280386488,22.683457094231187){?}
\rput[tl](2202.7749123356507,22.683457094231187){?}
\rput[tl](373.16401052037526,-12.1061114800064538){A}
\rput[tl](1307.2457133491991,-12.6982185679725623){B}
\rput[tl](2185.0116996966676,-12.6982185679725623){C}
\antwort{\pspolygon[linecolor=darkgray,fillcolor=darkgray,fillstyle=solid,opacity=0.1](300.,0.)(300.,240.)(600.,240.)(600.,0.)
\pspolygon[linecolor=darkgray,fillcolor=darkgray,fillstyle=solid,opacity=0.1](1200.,0.)(1200.,80.)(1500.,80.)(1500.,0.)
\pspolygon[linecolor=darkgray,fillcolor=darkgray,fillstyle=solid,opacity=0.1](2100.,0.)(2100.,160.)(2400.,160.)(2400.,0.)
\rput[tl](373.16401052037526,253.49993634542017){240}
\rput[tl](1302.7721386271658,91.26259424270647){80}
\rput[tl](2173.1695579373454,171.1970511181311){160}}
\end{pspicture*}
\end{beispiel}