\section{WS 3.2 - 6 Wahrscheinlichkeitsverteilung - OA - Matura 2015/16 - Haupttermin}

\begin{beispiel}[WS 3.2]{1} %PUNKTE DES BEISPIELS
Der Wertebereich einer Zufallsvariablen X besteht aus den Werten $x_1, x_2, x_3$.
Man kennt die Wahrscheinlichkeit $P(X = x_1) = 0,4$. Au�erdem wei� man, dass $x_3$ doppelt so wahrscheinlich wie $x_2$ ist. \leer

Berechne $P(X=x_2)$ und $P(X=x_3)$. \leer

$P(X=x_2)=$ \rule{5cm}{0.3pt} \leer

$P(X=x_3)=$ \rule{5cm}{0.3pt} \leer

\antwort{
$P(X=x_2)= 0,2$ 

$P(X=x_3)=0,4$}

\end{beispiel}