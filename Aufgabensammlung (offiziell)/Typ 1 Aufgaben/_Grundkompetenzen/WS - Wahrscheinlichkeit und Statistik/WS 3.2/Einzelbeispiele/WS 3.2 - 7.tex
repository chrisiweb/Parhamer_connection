\section{WS 3.2 - 7 Erwartungswert des Gewinns - OA - Matura 2014/15 - Haupttermin}

\begin{beispiel}[WS 3.2]{1} %PUNKTE DES BEISPIELS
Bei einem Gewinnspiel gibt es 100 Lose. Der Lospreis betr�gt \euro\,5. F�r den Haupttreffer werden \euro\,100 ausgezahlt, f�r zwei weitere Treffer werden je \euro\,50 ausgezahlt und f�r f�nf weitere Treffer werden je \euro\,20 ausgezahlt. F�r alle weiteren Lose wird nichts ausgezahlt. Unter \textit{Gewinn} versteht man \textit{Auszahlung minus Lospreis}. \leer

Berechne den Erwartungswert des Gewinns aus der Sicht einer Person, die ein Los kauft.

\antwort{$E=\frac{1}{100}\cdot 100 + \frac{2}{100} \cdot 50 + \frac{5}{100} \cdot 20 -5 =-2$ \leer

$E=\frac{92}{100}\cdot (-5) + \frac{5}{100} \cdot 15 +\frac{2}{100}\cdot 45 + \frac{1}{100} \cdot 95 = -2$ \leer

Der Erwartungswert des Gewinns betr�gt \euro\,-2 \leer

L�sungsschl�ssel:\\
Ein Punkt f�r die richtige L�sung, wobei die Einheit Euro nicht angef�hrt werden muss. Der Wert
$E = 2$ ist nur dann als richtig zu werten, wenn aus der Antwort klar hervorgeht, dass es sich dabei um einen Verlust von \euro\,2 aus Sicht der Person, die ein Los kauft, handelt. Die Aufgabe ist auch dann als richtig gel�st zu werten, wenn bei korrektem Ansatz das Ergebnis aufgrund eines Rechenfehlers nicht richtig ist.}
\end{beispiel}