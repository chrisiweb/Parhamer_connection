\section{WS 3.2 - 9 Gewinn beim Gl�cksrad - OA - Matura 2014/15 - Nebentermin 1}

\begin{beispiel}[WS 3.2]{1}
Das unten abgebildete Gl�cksrad ist in acht gleich gro�e Sektoren unterteilt, die mit gleicher Wahrscheinlichkeit auftreten. F�r einmaliges Drehen des Gl�cksrades muss ein Einsatz von 5\,\euro gezahlt werden. Die Gewinne, die ausbezahlt werden, wenn das Gl�cksrad im entsprechenden
Sektor stehen bleibt, sind auf dem Gl�cksrad abgebildet.\leer

\begin{center}
\psset{xunit=1.0cm,yunit=1.0cm,algebraic=true,dimen=middle,dotstyle=o,dotsize=5pt 0,linewidth=0.8pt,arrowsize=3pt 2,arrowinset=0.25}
\begin{pspicture*}(0.7823214592356473,0.7419987008660722)(7.117101769544686,7.19630316797339)
\pscircle(4.,4.){3.}
\psline(4.,4.)(1.8786796564403576,6.121320343559642)
\psline(4.,4.)(4.,7.)
\psline(4.,4.)(6.121320343559642,6.121320343559642)
\psline(4.,4.)(7.,4.)
\psline(4.,4.)(6.121320343559642,1.878679656440358)
\psline(4.,4.)(4.,1.)
\psline(4.,4.)(1.8786796564403576,1.8786796564403576)
\psline(4.,4.)(1.,4.)
\rput[tl](4.527411705581872,6){$5$\,\euro}
\rput[tl](5.563287731166997,5){$0$\,\euro}
\rput[tl](5.503525652767855,3.5){$10$\,\euro}
\rput[tl](4.447728934383016,2.594623131239469){$0$\,\euro}
\rput[tl](3,2.614543824039183){$5$\,\euro}
\rput[tl](1.82,3.5){$0$\,\euro}
\rput[tl](1.82,4.9){$15$\,\euro}
\rput[tl](2.953676974404469,6){$0$\,\euro}
\end{pspicture*}
\end{center}
\leer

Das Gl�cksrad wird einmal gedreht. Berechne den entsprechenden Erwartungswert des Reingewinns $G$ (in Euro) aus der Sicht des Betreibers des Gl�cksrades. Der Reingewinn ist die
Differenz aus Einsatz und Auszahlungsbetrag.

\antwort{
$G=5-\left( \frac{1}{4}\cdot 5 + \frac{1}{8} \cdot 10 + \frac{1}{8} \cdot 15\right) = \frac{5}{8} \Rightarrow G \approx$ \euro\, 0,63 \leer

L�sungsschl�ssel:\\
Ein Punkt f�r die richtige L�sung, wobei die Einheit nicht angef�hrt sein muss.\\
Toleranzintervall: $[0,62; 0,63]$}
\end{beispiel}