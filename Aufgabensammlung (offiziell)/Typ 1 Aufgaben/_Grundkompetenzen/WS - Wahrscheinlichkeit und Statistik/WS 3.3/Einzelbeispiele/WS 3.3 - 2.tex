\section{WS 3.3 - 2 Binomialverteilung - MC - BIFIE}
\begin{beispiel}[WS 3.3]{1} %PUNKTE DES BEISPIELS
Einige der unten angef�hrten Situationen k�nnen mit einer Binomialverteilung modelliert werden.\leer

Kreuze diejenige(n) Situation(en) an, bei der/denen die Zufallsvariable $X$ binomialverteilt ist.

\multiplechoice[5]{  %Anzahl der Antwortmoeglichkeiten, Standard: 5
				L1={Aus einer Urne mit vier blauen, zwei gr�nen und drei wei�en
Kugeln werden drei Kugeln mit Zur�cklegen gezogen.
($X$ = Anzahl der gr�nen Kugeln)},   %1. Antwortmoeglichkeit 
				L2={In einer Gruppe mit 25 Kindern sind sieben Linksh�nder. Es
werden drei Kinder zuf�llig ausgew�hlt.
($X$ = Anzahl der Linksh�nder)},   %2. Antwortmoeglichkeit
				L3={In einem U-Bahn-Waggon sitzen 35 Personen. Vier haben keinen
Fahrschein. Drei werden kontrolliert.
($X$ = Anzahl der Personen ohne Fahrschein)
},   %3. Antwortmoeglichkeit
				L4={Bei einem Multiple-Choice-Test sind pro Aufgabe drei von f�nf
Wahlm�glichkeiten richtig. Die Antworten werden nach dem
Zufallsprinzip angekreuzt. Sieben Aufgaben werden gestellt.
($X$ = Anzahl der richtig gel�sten Aufgaben).},   %4. Antwortmoeglichkeit
				L5={Die Wahrscheinlichkeit f�r die Geburt eines M�dchens liegt bei
52\,\%. Eine Familie hat drei Kinder.
($X$ = Anzahl der M�dchen)
},	 %5. Antwortmoeglichkeit
				L6={},	 %6. Antwortmoeglichkeit
				L7={},	 %7. Antwortmoeglichkeit
				L8={},	 %8. Antwortmoeglichkeit
				L9={},	 %9. Antwortmoeglichkeit
				%% LOESUNG: %%
				A1=1,  % 1. Antwort
				A2=4,	 % 2. Antwort
				A3=5,  % 3. Antwort
				A4=0,  % 4. Antwort
				A5=0,  % 5. Antwort
				} 
\end{beispiel} 