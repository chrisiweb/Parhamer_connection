\section{WS 4.1 - 12 - MAT - Sicherheit eines Konfidenzintervalls - OA - Matura 2016/17 2. NT}

\begin{beispiel}[WS 4.1]{1} %PUNKTE DES BEISPIELS
Die Abf�llanlagen eines Betriebes m�ssen in bestimmten Zeitabst�nden �berpr�ft und eventuell neu eingestellt werden.

Nach der Einstellung einer Abf�llanlage sind von 1\,000 �berpr�ften Packungen 30 nicht ordnungsgem�� bef�llt. F�r den unbekannten relativen Anteil $p$ der nicht ordnungsgem�� bef�llten Packungen wird vom Betrieb das symmetrische Konfidenzintervall $[0,02; 0,04]$ angegeben.

Ermittle unter Verwendung einer die Binomialverteilung approximierenden Normalverteilung die Sicherheit dieses Konfidenzintervalls!\leer

\antwort{M�gliche Vorgehensweise:

$n=1\,000$, $h=\frac{30}{1\,000}=0,03$ Intervallbreite des Konfidenzintervalls\,=\,0,02

aus $z\cdot\sqrt{\frac{h\cdot(1-h)}{n}}=0,01$ folgt: $z\approx 1,85$ mit $\Phi(1,85)\approx 0,9678$

$\Rightarrow \gamma=2\cdot\Phi(1,85)-1\approx 0,9356$

Somit liegt die Sicherheit dieses Konfidenzintervalls bei ca. 93,56\,\%.

Toleranzintervall: $[93\,\%;\,94\,\%]$}
\end{beispiel}