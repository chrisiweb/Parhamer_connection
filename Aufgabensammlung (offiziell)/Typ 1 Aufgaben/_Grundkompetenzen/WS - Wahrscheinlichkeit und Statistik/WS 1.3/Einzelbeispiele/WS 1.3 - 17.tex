\section{WS 1.3 - 17 Spenden - ZO - Matura 17/18}

\begin{beispiel}[WS 1.3]{1} %PUNKTE DES BEISPIELS
F�r einen guten Zweck spenden 20 Personen Geld, wobei jede Person einen anderen Betrag spendet. Diese 20 Geldbetr�ge (in Euro) bilden den Datensatz $x_1, x_2,...,x_{20}$. Von diesem Datensatz ermittelt man Minimum, Maximum, arithmetisches Mittel, Median sowie unteres (erstes) und oberes (drittes) Quartil.

Frau M�ller ist eine dieser 20 Personen und spendet 50 Euro.

Jede der vier Fragen in der linken Tabelle kann unter Kenntnis einer der statistischen Kennzahlen aus der rechten Tabelle korrekt beantwortet werden.
Ordnen den vier Fragen jeweils die entsprechende statistische Kennzahl (aus A bis F) zu!

\zuordnen[0.15]{
				R1={Ist die Spende von Frau M�ller eine
der f�nf gr��ten Spenden?},				% Response 1
				R2={Ist die Spende von Frau M�ller eine
der zehn gr��ten Spenden?},				% Response 2
				R3={Ist die Spende von Frau M�ller die
kleinste Spende?},				% Response 3
				R4={Wie viel Euro spenden die 20�Perso-
nen insgesamt?},				% Response 4
				%% Moegliche Zuordnungen: %%
				A={Minimum}, 				%Moeglichkeit A  
				B={Maximum}, 				%Moeglichkeit B  
				C={arithmetisches Mittel}, 				%Moeglichkeit C  
				D={Median}, 				%Moeglichkeit D  
				E={unteres Quartil}, 				%Moeglichkeit E  
				F={oberes Quartil}, 				%Moeglichkeit F  
				%% LOESUNG: %%
				A1={F},				% 1. richtige Zuordnung
				A2={D},				% 2. richtige Zuordnung
				A3={A},				% 3. richtige Zuordnung
				A4={C},				% 4. richtige Zuordnung
				}
\end{beispiel}