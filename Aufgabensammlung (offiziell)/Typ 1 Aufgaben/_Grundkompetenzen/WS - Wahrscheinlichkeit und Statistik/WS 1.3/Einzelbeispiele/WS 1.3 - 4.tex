\section{WS 1.3 - 4 Datenreihe - MC - BIFIE}

\begin{beispiel}[WS 1.3]{1} %PUNKTE DES BEISPIELS
				Der arithmetische Mittelwert $\overline{x}$ der Datenreihe $x_{1},x_{2},...,x_{10}$ ist $\overline{x}=20$. Die Standardabweichung $\sigma$ der Datenreihe ist $\sigma=5$.

Die Datenreihe wird um die beiden Werte $x_{11}=19$ und $x_{12}=21$ erg�nzt.

Kreuze die beiden zutreffenden Aussagen an.

\multiplechoice[5]{  %Anzahl der Antwortmoeglichkeiten, Standard: 5
				L1={Das Maximum der neuen Datenreihe $x_{1},...,x_{12}$ ist gr��er als das Maximum der urspr�nglichen Datenreihe $x_{1},...,x_{10}$.},   %1. Antwortmoeglichkeit 
				L2={Die Spannweite der neuen Datenreihe $x_{1},...,x_{12}$ ist um 2 gr��er als die Spannweite der urspr�nglichen Datenreihe $x_{1},...,x_{10}$.},   %2. Antwortmoeglichkeit
				L3={Der Median der neuen Datenreihe $x_{1},...,x_{12}$ stimmt immer mit dem Median der urspr�nglichen Datenreihe $x_{1},...,x_{10}$ �berein.},   %3. Antwortmoeglichkeit
				L4={Die Standardabweichung der neuen Datenreihe $x_{1},...,x_{12}$ ist kleiner als die Standardabweichung der urspr�nglichen Datenreihe $x_{1},...,x_{10}$.},   %4. Antwortmoeglichkeit
				L5={Der arithmetische Mittelwert der neuen Datenreihe $x_{1},...,x_{12}$ stimmt mit dem arithmetischen Mittelwert der urspr�nglichen Datenreihe $x_{1},...,x_{10}$ �berein.},	 %5. Antwortmoeglichkeit
				L6={},	 %6. Antwortmoeglichkeit
				L7={},	 %7. Antwortmoeglichkeit
				L8={},	 %8. Antwortmoeglichkeit
				L9={},	 %9. Antwortmoeglichkeit
				%% LOESUNG: %%
				A1=4,  % 1. Antwort
				A2=5,	 % 2. Antwort
				A3=0,  % 3. Antwort
				A4=0,  % 4. Antwort
				A5=0,  % 5. Antwort
				}
				\end{beispiel}