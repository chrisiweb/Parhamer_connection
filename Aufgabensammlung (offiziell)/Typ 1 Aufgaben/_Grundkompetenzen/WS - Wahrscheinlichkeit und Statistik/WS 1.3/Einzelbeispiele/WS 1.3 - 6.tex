\section{WS 1.3 - 6 Geordnete Urliste - MC - BIFIE}

\begin{beispiel}[WS 1.3]{1} %PUNKTE DES BEISPIELS
				9 Kinder wurden dahingehend befragt, wie viele Stunden sie am Wochenende fernsehen. Die nachstehende Tabelle gibt ihre Antworten wieder.

\begin{center}
\begin{tabular}{|l|c|}\hline
Kind&Fernsehstunden\\ \hline 
Fritz&2\\ \hline
Susi&2\\ \hline
Michael&3\\ \hline
Martin&3\\ \hline
Angelika&4\\ \hline
Paula&5\\ \hline
Max&5\\ \hline
Hubert&5\\ \hline
Lisa&8\\ \hline
\end{tabular}
\end{center}

Kreuze die beiden zutreffenden Aussagen an.

\multiplechoice[5]{  %Anzahl der Antwortmoeglichkeiten, Standard: 5
				L1={Der Median w�rde sich erh�hen, wenn Fritz um eine
Stunde mehr fernsehen w�rde.},   %1. Antwortmoeglichkeit 
				L2={Der Median ist kleiner als das arithmetische Mittel der Fernsehstunden.},   %2. Antwortmoeglichkeit
				L3={Die Spannweite der Fernsehstunden betr�gt 3.},   %3. Antwortmoeglichkeit
				L4={Das arithmetische Mittel w�rde sich erh�hen, wenn Lisa anstelle von 8 Stunden 10 Stunden fernsehen w�rde.},   %4. Antwortmoeglichkeit
				L5={Der Modus ist 8.},	 %5. Antwortmoeglichkeit
				L6={},	 %6. Antwortmoeglichkeit
				L7={},	 %7. Antwortmoeglichkeit
				L8={},	 %8. Antwortmoeglichkeit
				L9={},	 %9. Antwortmoeglichkeit
				%% LOESUNG: %%
				A1=2,  % 1. Antwort
				A2=4,	 % 2. Antwort
				A3=0,  % 3. Antwort
				A4=0,  % 4. Antwort
				A5=0,  % 5. Antwort
				}
				\end{beispiel}