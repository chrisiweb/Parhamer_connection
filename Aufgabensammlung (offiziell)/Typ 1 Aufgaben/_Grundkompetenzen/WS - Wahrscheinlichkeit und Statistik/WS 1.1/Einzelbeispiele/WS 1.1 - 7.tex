\section{WS 1.1 - 7 Entwicklung der Landwirtschaft in �sterreich - MC - Matura 2014/15 - Nebentermin 1}

\begin{beispiel}[WS 1.1]{1}
Der Website der Statistik Austria kann man folgende Tabelle �ber die Entwicklung der Agrarstruktur in �sterreich entnehmen:

\begin{longtable}{|p{4cm}|c|c|c|} \hline
Jahr & 1995 & 1999 & 2010 \\ \hline
Anzahl der land- und forstwirtschaftlichen Betriebe insgesamt & 239\,099 & 217\,508 & 173\,317 \\ \hline
durchschnittliche Betriebsgr��e in Hektar & 31,5 & 34,6 & 42,4 \\ \hline
\end{longtable}
\begin{flushleft}
\tiny Datenquelle: http://www.statistik.at/web\_de/statistiken/land\_und\_forstwirtschaft/index.html
\end{flushleft}

\leer

Kreuze die beiden zutreffenden Aussagen an. 

\multiplechoice[5]{  %Anzahl der Antwortmoeglichkeiten, Standard: 5
				L1={Die Anzahl der land- und forstwirtschaftlichen Betriebe ist im Zeitraum von 1995 bis 2010 in jedem Jahr um die gleiche Zahl gesunken.},   %1. Antwortmoeglichkeit 
				L2={Die durchschnittliche Betriebsgr��e hat von 1995 bis 1999 im Jahresdurchschnitt um mehr Hektar zugenommen als von
1999 bis 2010.},   %2. Antwortmoeglichkeit
				L3={Die durchschnittliche Betriebsgr��e hat von 1995 bis 1999 um durchschnittlich 0,5 ha pro Jahr abgenommen.},   %3. Antwortmoeglichkeit
				L4={Die Gesamtgr��e der land- und forstwirtschaftlich genutzten Fl�che hat von 1995 bis 2010 abgenommen.},   %4. Antwortmoeglichkeit
				L5={Die Anzahl der land- und forstwirtschaftlichen Betriebe ist im Zeitraum von 1995 bis 2010 um mehr als ein Drittel gesunken.},	 %5. Antwortmoeglichkeit
				L6={},	 %6. Antwortmoeglichkeit
				L7={},	 %7. Antwortmoeglichkeit
				L8={},	 %8. Antwortmoeglichkeit
				L9={},	 %9. Antwortmoeglichkeit
				%% LOESUNG: %%
				A1=2,  % 1. Antwort
				A2=4,	 % 2. Antwort
				A3=0,  % 3. Antwort
				A4=0,  % 4. Antwort
				A5=0,  % 5. Antwort
				}

\end{beispiel}