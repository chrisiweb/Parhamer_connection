\section{WS 1.1 - 8 Anzahl der Heizungstage - MC - Matura 2014/15 - Nebentermin 2}

\begin{beispiel}[WS 1.1]{1} %PUNKTE DES BEISPIELS
				Die K�rpergr��en der $450$ Sch�lerInnen einer Schulstufe einer Gemeinde wurden in Zentimetern gemessen und deren Verteilung wurde in einem Kastenschaubild (Boxplot) grafisch dargestellt.
				\begin{center}
					\resizebox{1\linewidth}{!}{\newrgbcolor{cqcqcq}{0.7529411764705882 0.7529411764705882 0.7529411764705882}
\newrgbcolor{uuuuuu}{0.26666666666666666 0.26666666666666666 0.26666666666666666}
\psset{xunit=0.7cm,yunit=0.7cm,algebraic=true,dimen=middle,dotstyle=o,dotsize=5pt 0,linewidth=0.8pt,arrowsize=3pt 2,arrowinset=0.25}
\begin{pspicture*}(162.71202623579816,-2.7882401108569232)(187.56415568828166,3.3920920819306586)
\multips(0,0)(0,10.0){1}{\psline[linestyle=dashed,linecap=1,dash=1.5pt 1.5pt,linewidth=0.4pt,linecolor=darkgray]{c-c}(162.71202623579816,0)(187.56415568828166,0)}
\multips(162,0)(1.0,0){25}{\psline[linestyle=dashed,linecap=1,dash=1.5pt 1.5pt,linewidth=0.4pt,linecolor=darkgray]{c-c}(0,0)(0,3.3920920819306586)}
\psaxes[labelFontSize=\scriptstyle,xAxis=true,yAxis=true,Dx=2.,Dy=2.,ticksize=-2pt 0,subticks=2]{}(0,0)(162.71202623579816,-2.7882401108569232)(187.56415568828166,3.3920920819306586)
\psframe[linecolor=darkgray,fillcolor=darkgray,fillstyle=solid,opacity=0.1](164.,0.5)(178.,1.5)
\psline[linecolor=darkgray,fillcolor=darkgray,fillstyle=solid,opacity=0.1](164.,0.5)(164.,1.5)
\psline[linecolor=darkgray,fillcolor=darkgray,fillstyle=solid,opacity=0.1](185.,0.5)(185.,1.5)
\psline[linecolor=darkgray,fillcolor=darkgray,fillstyle=solid,opacity=0.1](170.,0.5)(170.,1.5)
\psline[linecolor=darkgray,fillcolor=darkgray,fillstyle=solid,opacity=0.1](164.,1.)(164.,1.)
\psline[linecolor=darkgray,fillcolor=darkgray,fillstyle=solid,opacity=0.1](178.,1.)(185.,1.)
\rput[tl](182.16862758346616,-1.382132786571918){K�rpergr��en in cm}
\end{pspicture*}}
				\end{center}
				
				Zur Interpretation dieses Kastenschaubilds werden verschiedene Aussagen get�tigt. Kreuze die beiden zutreffenden Aussagen an.\\
				\multiplechoice[5]{  %Anzahl der Antwortmoeglichkeiten, Standard: 5
								L1={$60\,\%$ der Sch�lerInnen sind genau $172\,cm$ gro�.},   %1. Antwortmoeglichkeit 
								L2={Mindestens eine Sch�lerin bzw. ein Sch�ler ist genau $185\,cm$ gro�.},   %2. Antwortmoeglichkeit
								L3={H�chstens $50\,\%$ der Sch�lerInnen sind kleiner als $170\,cm$.},   %3. Antwortmoeglichkeit
								L4={Mindestens $75\,\%$ der Sch�lerInnen sind gr��er als $178\,cm$.},   %4. Antwortmoeglichkeit
								L5={H�chstens $50\,\%$ der Sch�lerInnen sind mindestens $164\,cm$ und h�chstens $178\,cm$ gro�.},	 %5. Antwortmoeglichkeit
								L6={},	 %6. Antwortmoeglichkeit
								L7={},	 %7. Antwortmoeglichkeit
								L8={},	 %8. Antwortmoeglichkeit
								L9={},	 %9. Antwortmoeglichkeit
								%% LOESUNG: %%
								A1=2,  % 1. Antwort
								A2=3,	 % 2. Antwort
								A3=0,  % 3. Antwort
								A4=0,  % 4. Antwort
								A5=0,  % 5. Antwort
								}
\end{beispiel}