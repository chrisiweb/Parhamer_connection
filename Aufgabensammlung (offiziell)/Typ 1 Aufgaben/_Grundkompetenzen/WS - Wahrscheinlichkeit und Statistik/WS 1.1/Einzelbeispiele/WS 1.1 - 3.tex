\section{WS 1.1 - 3 Boxplot - MC - BIFIE}

\begin{beispiel}[WS 1.1]{1} %PUNKTE DES BEISPIELS
				Die Nettogeh�lter von 44 Angestellten einer Firmenabteilung werden durch folgendes Kastenschaubild (Boxplot) dargestellt:

\resizebox{0.9\linewidth}{!}{\newrgbcolor{uuuuuu}{0.26666666666666666 0.26666666666666666 0.26666666666666666}
\psset{xunit=1.0cm,yunit=1.0cm,algebraic=true,dimen=middle,dotstyle=o,dotsize=5pt 0,linewidth=0.8pt,arrowsize=3pt 2,arrowinset=0.25}
\begin{pspicture*}(-0.94951,-1.4361208549209195)(22.18000034031045,2.257139279924421)
\psaxes[labelFontSize=\scriptstyle,xAxis=true,yAxis=false,labels=y,Dx=1.,Dy=1.,ticksize=-4pt 0,subticks=1]{}(0,0)(-0.94951,-1.4361208549209195)(22.18000034031045,2.257139279924421)[,140] [,-40]
\psframe[linecolor=darkgray,fillcolor=darkgray,fillstyle=solid,opacity=0.1](4.,0.29999999999999993)(11.,1.1)
\psline[linecolor=darkgray,fillcolor=darkgray,fillstyle=solid,opacity=0.1](1.,0.3)(1.,1.1)
\psline[linecolor=darkgray,fillcolor=darkgray,fillstyle=solid,opacity=0.1](21.,0.3)(21.,1.1)
\psline[linecolor=darkgray,fillcolor=darkgray,fillstyle=solid,opacity=0.1](7.,0.3)(7.,1.1)
\psline[linecolor=darkgray,fillcolor=darkgray,fillstyle=solid,opacity=0.1](1.,0.7)(4.,0.7)
\psline[linecolor=darkgray,fillcolor=darkgray,fillstyle=solid,opacity=0.1](11.,0.7)(21.,0.7)
\rput[tl](-0.7681628309196053,-0.25){\EUR{1.000}}
\rput[tl](9.2675772280847,-0.25){\EUR{2.000}}
\rput[tl](19.190883277314667,-0.25){\EUR{3.000}}
\end{pspicture*}}

Kreuze die beiden zutreffenden Aussagen an.

\multiplechoice[5]{  %Anzahl der Antwortmoeglichkeiten, Standard: 5
				L1={22 Angestellte verdienen mehr als \EUR{2.400}.},   %1. Antwortmoeglichkeit 
				L2={Drei Viertel der Angestellten verdienen \EUR{2.100} oder mehr.},   %2. Antwortmoeglichkeit
				L3={Ein Viertel aller Angestellten verdient \EUR{1.400} oder weniger.},   %3. Antwortmoeglichkeit
				L4={Es gibt Angestellte, die mehr als \EUR{3.300} verdienen.},   %4. Antwortmoeglichkeit
				L5={Das Nettogehalt der H�lfte aller Angestellten liegt im Bereich $\left[ \text{\euro 1.400} ; \text{\euro 2.100} \right]$.},	 %5. Antwortmoeglichkeit
				L6={},	 %6. Antwortmoeglichkeit
				L7={},	 %7. Antwortmoeglichkeit
				L8={},	 %8. Antwortmoeglichkeit
				L9={},	 %9. Antwortmoeglichkeit
				%% LOESUNG: %%
				A1=3,  % 1. Antwort
				A2=5,	 % 2. Antwort
				A3=0,  % 3. Antwort
				A4=0,  % 4. Antwort
				A5=0,  % 5. Antwort
				}
				\end{beispiel}