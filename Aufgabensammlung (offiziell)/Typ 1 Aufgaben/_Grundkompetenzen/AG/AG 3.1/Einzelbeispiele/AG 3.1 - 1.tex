\section{AG 3.1 - 1 Energiesparlampen - OA - BIFIE}

\begin{beispiel}[AG 3.1]{1} %PUNKTE DES BEISPIELS
Ein H�ndler handelt mit 7 verschiedenen Typen von Energiesparlampen. In der Buchhaltung verwendet er folgende 7-dimensionale Vektoren (die Werte in den Vektoren beziehen sich auf einen bestimmten Tag):
\begin{itemize}
	\item Lagerhaltungsvektor $L_{1}$ f�r Lager 1 zu Beginn des Tages
	\item Lagerhaltungsvektor $L_{2}$ f�r Lager 2 zu Beginn des Tages
	\item Vektor $P$ der Verkaufspreise
	\item Vektor $B$, der die Anzahl der an diesem Tag ausgelieferten Lampen angibt
\end{itemize}

Gib die Bedeutung des Ausdrucks $(L_{1}+L_{2}-B)\cdot P$ in diesem Zusammenhang an!

\antwort{Die Zahl $(L_{1}+L_{2}-B)\cdot P$ gibt den Lagerwert der am Ende des Tages in den beiden Lagern noch vorhandenen Lampen an.}
\end{beispiel}