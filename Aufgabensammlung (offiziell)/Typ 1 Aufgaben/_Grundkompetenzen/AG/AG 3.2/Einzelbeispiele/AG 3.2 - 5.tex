\section{AG 3.2 - 5 Quader mit quadratischer Grundfl�che - OA - Matura 2016/17 - Haupttermin}

\begin{beispiel}[AG 3.2]{1} %PUNKTE DES BEISPIELS

Die nachstehende Abbildung zeigt einen Quader, dessen quadratische Grundfl�che in der
$xy$-Ebene liegt. Die L�nge einer Grundkante betr�gt 5 L�ngeneinheiten, die K�rperh�he betr�gt
10 L�ngeneinheiten. Der Eckpunkt $D$ liegt im Koordinatenursprung, der Eckpunkt $C$ liegt auf der positiven $y$-Achse. \leer

Der Eckpunkt $E$ hat somit die Koordinaten $E = (5|0|10)$.

\begin{center}
\psset{xunit=1.0cm,yunit=1.0cm,algebraic=true,dimen=middle,dotstyle=o,dotsize=5pt 0,linewidth=0.8pt,arrowsize=3pt 2,arrowinset=0.25}
\begin{pspicture*}(-2.15576670160615,-2.102265733431046)(4.180087497445945,6.529688216231206)
\pspolygon[fillcolor=black,fillstyle=solid,opacity=0.1](-0.991931202159403,4.008068797840597)(1.5162773563746335,4.008068797840597)(1.5162773563746335,-0.9919312021594031)(-0.991931202159403,-0.991931202159403)
\pspolygon[fillcolor=black,fillstyle=solid,opacity=0.1](1.5162773563746335,4.008068797840597)(2.5082085585340366,5.)(2.5082085585340366,0.)(1.5162773563746335,-0.9919312021594031)
\pspolygon[fillcolor=black,fillstyle=solid,opacity=0.1](2.5082085585340366,5.)(0.,5.)(-0.991931202159403,4.008068797840597)(1.5162773563746335,4.008068797840597)
\psline(-0.991931202159403,4.008068797840597)(1.5162773563746335,4.008068797840597)
\psline(1.5162773563746335,4.008068797840597)(1.5162773563746335,-0.9919312021594031)
\psline(1.5162773563746335,-0.9919312021594031)(-0.991931202159403,-0.991931202159403)
\psline(-0.991931202159403,-0.991931202159403)(-0.991931202159403,4.008068797840597)
\psline(1.5162773563746335,4.008068797840597)(2.5082085585340366,5.)
\psline(2.5082085585340366,5.)(2.5082085585340366,0.)
\psline(2.5082085585340366,0.)(1.5162773563746335,-0.9919312021594031)
\psline(1.5162773563746335,-0.9919312021594031)(1.5162773563746335,4.008068797840597)
\psline(2.5082085585340366,5.)(0.,5.)
\psline(0.,5.)(-0.991931202159403,4.008068797840597)
\psline(-0.991931202159403,4.008068797840597)(1.5162773563746335,4.008068797840597)
\psline(1.5162773563746335,4.008068797840597)(2.5082085585340366,5.)
\psline[linestyle=dashed,dash=4pt 4pt](0.,0.)(-0.991931202159403,-0.991931202159403)
\psline[linestyle=dashed,dash=4pt 4pt](0.,0.)(2.5082085585340366,0.)
\psline[linestyle=dashed,dash=4pt 4pt](0.,0.)(0.,5.)
\psline{->}(2.5082085585340366,0.)(4.,0.)
\psline{->}(-0.991931202159403,-0.991931202159403)(-1.9132173687770402,-1.9132173687770402)
\psline{->}(0.,5.)(0.,6.357049137237967)
\rput[tl](3.8348093394594547,0.4182648198703315){$y$}
\rput[tl](0.1748608648026591,6.218937874043365){$z$}
\rput[tl](-1.62058555672709,-1.7915153912432054){$x$}
\begin{scriptsize}
\psdots[dotsize=1pt 0,dotstyle=*](0.,0.)
\rput[bl](0.08854132530603656,0.14204229348113936){$D$}
\psdots[dotsize=1pt 0,dotstyle=*](2.5082085585340366,0.)
\rput[bl](2.59180797070809,0.1765701092797884){$C$}
\psdots[dotsize=1pt 0,dotstyle=*](0.,5.)
\rput[bl](0.12306914110468557,5.148575584285246){$H$}
\psdots[dotsize=1pt 0,dotstyle=*](1.5162773563746335,-0.9919312021594031)
\rput[bl](1.4178622335540239,-1.2908620621627946){$B$}
\psdots[dotsize=1pt 0,dotstyle=*](1.5162773563746335,4.008068797840597)
\rput[bl](1.3488066019567257,4.095477202426451){$F$}
\psdots[dotsize=1pt 0,dotstyle=*](-0.991931202159403,-0.991931202159403)
\rput[bl](-1.0336126881500565,-1.27359815426347){$A$}
\psdots[dotsize=1pt 0,dotstyle=*](-0.991931202159403,4.008068797840597)
\rput[bl](-1.119932227646679,4.147268926124425){$E$}
\psdots[dotsize=1pt 0,dotstyle=*](2.5082085585340366,5.)
\rput[bl](2.3328493522182225,5.096783860587273){$G$}
\end{scriptsize}
\end{pspicture*}
\end{center}

Gib die Koordinaten (Komponenten) des Vektors $\vek{HB}$ an!

\antwort{\leer

$\vek{HB}=\left(\begin{array}{c}5\\5\\-10\end{array}\right)$}

\end{beispiel}