\section{AG 4.1 - 10 Sonnenh�he - OA - Matura 2014/15 - Nebentermin 1}

\begin{beispiel}[AG 4.1]{1} %PUNKTE DES BEISPIELS
Unter der Sonnenh�he $\varphi$ versteht man denjenigen spitzen Winkel, den die einfallenden Sonnenstrahlen mit einer horizontalen Ebene einschlie�en. Die Schattenl�nge $s$ eines Geb�udes der H�he $h$ h�ngt von der Sonnenh�he $\varphi$ ab ($s$, $h$ in Metern). \leer

Gib eine Formel an, mit der die Schattenl�nge $s$ eines Geb�udes der H�he $h$ mithilfe der Sonnenh�he $\varphi$ berechnet werden kann. \leer

$s=$ \antwort[\rule{3cm}{0.3pt}km]{$\frac{h}{\tan(\varphi)}$ mit $\varphi \in (0^\circ, 90^\circ)$ bzw. $\varphi \in \left( 0; \frac{\pi}{2}\right)$}
\end{beispiel}