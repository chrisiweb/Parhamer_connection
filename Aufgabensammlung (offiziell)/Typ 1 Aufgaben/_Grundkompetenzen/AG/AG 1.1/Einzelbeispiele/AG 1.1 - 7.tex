\section{AG 1.1 - 7 Anetas Behauptungen - MC - MK}

\begin{beispiel}[AG 1.1]{1} %PUNKTE DES BEISPIELS
				Sherif und Aneta haben beim �ben f�r die Schularbeit f�nf Behauptungen �ber die verschiedenen Zahlenmengen aufgestellt, leider sind nicht alle richtig. Kreuze die beiden richtigen Aussagen an.\\

\multiplechoice[5]{  %Anzahl der Antwortmoeglichkeiten, Standard: 5
				L1={Jede nat�rliche Zahl kann auch als Bruchzahl dargestellt werden.},   %1. Antwortmoeglichkeit 
				L2={Jede Dezimalzahl kann auch als Bruchzahl dargestellt werden.},   %2. Antwortmoeglichkeit
				L3={Die Zahl $\pi$ ist eine rationale Zahl.},   %3. Antwortmoeglichkeit
				L4={Jede nichtnegative ganze Zahl ist auch eine nat�rliche Zahl.},   %4. Antwortmoeglichkeit
				L5={Die rationalen Zahlen bestehen ausschlie�lich aus positiven Zahlen.},	 %5. Antwortmoeglichkeit
				L6={},	 %6. Antwortmoeglichkeit
				L7={},	 %7. Antwortmoeglichkeit
				L8={},	 %8. Antwortmoeglichkeit
				L9={},	 %9. Antwortmoeglichkeit
				%% LOESUNG: %%
				A1=1,  % 1. Antwort
				A2=4,	 % 2. Antwort
				A3=0,  % 3. Antwort
				A4=0,  % 4. Antwort
				A5=0,  % 5. Antwort
				}
\end{beispiel}