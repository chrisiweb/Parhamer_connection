\section{AG 2.3 - 16 L�sungsf�lle quadratischer Gleichungen - OA - Matura 17/18}

\begin{beispiel}[AG 2.3]{1} %PUNKTE DES BEISPIELS
Gegeben ist eine quadratische Gleichung der Form $r\cdot x^2+s\cdot x+t=0$ in der Variablen $x$ mit den Koeffizienten $r,s,t\in\mathbb{R}\backslash\{0\}$.

Die Anzahl der reellen L�sungen der Gleichung h�ngt von $r,s$ und $t$ ab.

Gib die Anzahl der reellen L�sungen der gegebenen Gleichung an, wenn $r$ und $t$ verschiedene Vorzeichen haben, und begr�nde deine Antwort allgemein!

\antwort{Wenn $r$ und $t$ verschiedene Vorzeichen haben, dann hat die gegebene Gleichung genau zwei (verschiedene) reelle L�sungen.

\textit{M�gliche Begr�ndung:}

L�sungen der Gleichung $x_{1,2}=\dfrac{-s\pm\sqrt{s^2-4\cdot r\cdot t}}{2\cdot r}$

Haben $r$ und $t$ verschiedene Vorzeichen, dann ist $-4\cdot r\cdot t$ in jedem Fall positiv und es gilt $s^2-4\cdot r\cdot t>0$.

Daraus ergeben sich zwei verschiedene reelle L�sungen.}
\end{beispiel}