\section{AG 3.3 - 16 Vektorkonstruktion - OA - Matura 2013/14 Haupttermin}

\begin{beispiel}[AG 3.3]{1} %PUNKTE DES BEISPIELS
				Die Abbildung zeigt zwei als Pfeile dargestellte Vektoren $\vec{a}$ und $\vec{b}$ und einen Punkt $P$.

Erg�nze die unten stehende Abbildung um einen Pfeil, der vom Punkt $P$ ausgeht und den Vektor $\vec{a}-\vec{b}$ darstellt!

\begin{center}
\resizebox{0.7\linewidth}{!}{\psset{xunit=1.0cm,yunit=1.0cm,algebraic=true,dimen=middle,dotstyle=o,dotsize=5pt 0,linewidth=0.8pt,arrowsize=3pt 2,arrowinset=0.25}
\begin{pspicture*}(-11.24,3.52)(-0.56,11.22)
\multips(0,3)(0,1.0){8}{\psline[linestyle=dashed,linecap=1,dash=1.5pt 1.5pt,linewidth=0.4pt,linecolor=lightgray]{c-c}(-11.24,0)(-0.56,0)}
\multips(-11,0)(1.0,0){11}{\psline[linestyle=dashed,linecap=1,dash=1.5pt 1.5pt,linewidth=0.4pt,linecolor=lightgray]{c-c}(0,3.52)(0,11.22)}
\psline{->}(-10.,7.)(-7.,10.)
\antwort{\psline{->}(-10.,7.)(-5.,8.)}
\antwort{\psline{->}(-7.,10.)(-5.,8.)}
\psline{->}(-3.,8.)(-5.,10.)
\antwort{\psline{->}(-8.,5.)(-3.,6.)}
\begin{scriptsize}
\rput[tl](-4,9.5){$\vec{b}$}
\antwort{\rput[tl](-6,9.48){$-\vec{b}$}}
\antwort{\rput[tl](-7.76,7.28){$\vec{a}-\vec{b}$}
\rput[tl](-6.26,5.3){$\vec{a}-\vec{b}$}}
\rput[tl](-8.6,9){$\vec{a}$}
\psdots[dotsize=3pt 0,dotstyle=*](-8.,5.)
\rput[bl](-8.18,5.16){$P$}
\end{scriptsize}
\end{pspicture*}}
\end{center}

\end{beispiel}