\section{AG 3.3 - 18 W�rstelstand - OA - Matura NT 1 16/17}

\begin{beispiel}[AG 3.3]{1} %PUNKTE DES BEISPIELS
Ein W�rstelstandbesitzer f�hrt Aufzeichnungen �ber die Anzahl der t�glich verkauften W�rstel. Die Aufzeichnung eines bestimmten Tages ist nachstehend angegeben:

\begin{tabular}{|c|C{3.5cm}|C{3.5cm}|C{3.5cm}|}\cline{2-4}
\multicolumn{1}{c|}{}&\cellcolor[gray]{0.9}Anzahl der verkauften Portionen&\cellcolor[gray]{0.9}Verkaufspreis pro Portion (in Euro)&\cellcolor[gray]{0.9}Einkaufspreis pro Portion (in Euro)\\ \hline
\cellcolor[gray]{0.9}Frankfurter&24&2,70& 0,90\\ \hline
\cellcolor[gray]{0.9}Debreziner&14&3,00& 1,20\\ \hline
\cellcolor[gray]{0.9}Burenwurst&11&2,80& 1,00\\ \hline
\cellcolor[gray]{0.9}K�sekrainer&19&3,20& 1,40\\ \hline
\cellcolor[gray]{0.9}Bratwurst&18&3,20& 1,20\\ \hline
\end{tabular}

Die mit Zahlenwerten ausgef�llten Spalten der Tabelle k�nnen als Vektoren abgeschrieben werden. Dabei gibt der Vektor $A$ die Anzahl der verkauften Portionen, der Vektor $B$ die Verkaufspreise pro Portion (in Euro) und der Vektor $C$ die Einkaufspreise pro Portion (in Euro) an.

Gib einen Ausdruck mithilfe der Vektoren $A,B$ und $C$ an, der den an diesem Tag erzielten Gesamtgewinn des W�rstelstandbesitzers bezogen auf den Verkauf der W�rstel beschreibt!

Gesamtgewinn = \antwort[\rule{3cm}{0.3pt}]{$A\cdot (B-C)$}
\end{beispiel}