\section{AG 3.3 - 6 Rechenoperationen bei Vektoren - MC - BIFIE}

\begin{beispiel}[AG 3.3]{1} %PUNKTE DES BEISPIELS
Gegeben sind die Vektoren $\vek{a}$ und $\vek{b}$ sowie ein Skalar $r\in\mathbb{R}$.

Welche der folgenden Rechenoperationen liefert/liefern als Ergebnis wieder einen Vektor? Kreuze die zutreffende(n) Antwort(en) an!
\multiplechoice[5]{  %Anzahl der Antwortmoeglichkeiten, Standard: 5
				L1={$\vek{a}+r\cdot\vek{b}$},   %1. Antwortmoeglichkeit 
				L2={$\vek{a}+r$},   %2. Antwortmoeglichkeit
				L3={$\vek{a}\cdot\vek{b}$},   %3. Antwortmoeglichkeit
				L4={$r\cdot\vek{b}$},   %4. Antwortmoeglichkeit
				L5={$\vek{b}-\vek{a}$},	 %5. Antwortmoeglichkeit
				L6={},	 %6. Antwortmoeglichkeit
				L7={},	 %7. Antwortmoeglichkeit
				L8={},	 %8. Antwortmoeglichkeit
				L9={},	 %9. Antwortmoeglichkeit
				%% LOESUNG: %%
				A1=1,  % 1. Antwort
				A2=4,	 % 2. Antwort
				A3=5,  % 3. Antwort
				A4=0,  % 4. Antwort
				A5=0,  % 5. Antwort
				}
\end{beispiel}