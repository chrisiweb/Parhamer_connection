\section{AG 1.2 - 6 Definitionsmengen - ZO - Matura 2013/14 1. Nebentermin}

\begin{beispiel}[AG 1.2]{1} %PUNKTE DES BEISPIELS
				Es sind vier Terme und sechs Mengen (A bis F) gegeben.
				
				Ordne den vier Termen jeweils die entsprechende gr��tm�gliche Definitionsmenge $D_A, D_B, ... , D_F$ in der Menge der reellen Zahlen zu!\leer
				
				\zuordnen{
								R1={$\ln(x+1)$},				% Response 1
								R2={$\sqrt{1-x}$},				% Response 2
								R3={$\frac{2x}{x\cdot(x+1)^2}$},				% Response 3
								R4={$\frac{2x}{x^2+1}$},				% Response 4
								%% Moegliche Zuordnungen: %%
								A={$D_A=\mathbb{R}$}, 				%Moeglichkeit A  
								B={$D_B=(1;\infty)$}, 				%Moeglichkeit B  
								C={$D_C=(-1;\infty)$}, 				%Moeglichkeit C  
								D={$D_D=\mathbb{R}\backslash\left\{-1;0\right\}$}, 				%Moeglichkeit D  
								E={$D_E=(-\infty;1)$}, 				%Moeglichkeit E  
								F={$D_F=(-\infty;1]$}, 				%Moeglichkeit F  
								%% LOESUNG: %%
								A1={C},				% 1. richtige Zuordnung
								A2={F},				% 2. richtige Zuordnung
								A3={D},				% 3. richtige Zuordnung
								A4={A},				% 4. richtige Zuordnung
								}
\end{beispiel}