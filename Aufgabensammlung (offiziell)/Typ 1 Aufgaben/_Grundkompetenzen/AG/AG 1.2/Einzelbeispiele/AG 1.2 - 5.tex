\section{AG 1.2 - 5 Punktladungen - OA - Matura 2013/14 Haupttermin}

\begin{beispiel}[AG 1.2]{1} %PUNKTE DES BEISPIELS
				Der Betrag $F$ der Kraft zwischen Punktladungen $q_1$ und $q_2$ im Abstand $r$ wird beschrieben durch die Gleichung $F=C\cdot \frac{q_1\cdot q_2}{r�}$ ($C$ ... physikalische Konstante).
				
				Gib an, um welchen Faktor sich der Betrag $F$ �ndert, wenn der Betrag der Punktlandungen $q_1$ und $q_2$ jeweils verdoppelt und der Abstand $r$ zwischen diesen beiden Punktlandungen halbiert wird.\leer
				
\antwort{$F=C\cdot\dfrac{2\cdot q_1\cdot 2\cdot q_2}{\left(\frac{r}{2}\right)�}=C\cdot \dfrac{16\cdot q_1\cdot q_2}{r�}$

Der Betrag der Kraft $F$ wird 16-mal so gro�.

\textit{Ein Punkt f�r die richtige L�sung. Weder die Rechnung noch ein Antwortsatz m�ssen angegeben werden. Die Angabe des Faktors 16 ist ausreichend.}}
\end{beispiel}