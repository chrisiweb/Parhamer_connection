\section{AG 3.5 - 2 Normalvektor aufstellen - OA - BIFIE}

\begin{beispiel}[AG 3.5]{1} %PUNKTE DES BEISPIELS
Der gegebene Pfeil veranschaulicht einen Vektor $\vek{a}$.

Der zugrunde gelegte Raster legt dabei die Einheit fest.

\newrgbcolor{cqcqcq}{0.7529411764705882 0.7529411764705882 0.7529411764705882}
\psset{xunit=1.0cm,yunit=1.0cm,algebraic=true,dimen=middle,dotstyle=o,dotsize=5pt 0,linewidth=0.8pt,arrowsize=3pt 2,arrowinset=0.25}
\begin{pspicture*}(-4.3,-0.2)(3.52,6.3)
\multips(0,0)(0,1.0){7}{\psline[linestyle=dashed,linecap=1,dash=1.5pt 1.5pt,linewidth=0.4pt,linecolor=lightgray]{c-c}(-4.3,0)(3.52,0)}
\multips(-4,0)(1.0,0){8}{\psline[linestyle=dashed,linecap=1,dash=1.5pt 1.5pt,linewidth=0.4pt,linecolor=lightgray]{c-c}(0,-0.2)(0,6.3)}
\psline{->}(-3.,4.)(2.,2.)
\rput[tl](-0.56,3.98){$\overrightarrow{a}$}
\end{pspicture*}

Gib die Koordinaten eines Vektors $\vek{b}$ an, der auf $\vek{a}$ normal steht und gleich lang ist!
\leer

$\vek{b}=$ \antwort[\rule{3cm}{0.3pt}]{$\Vek{2}{5}{}$ bzw. $\vek{b}=\Vek{-2}{-5}{}$}
\end{beispiel}