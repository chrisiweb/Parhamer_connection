\section{FA 2.6 - 1 Zusammenhang - LT - BIFIE}

\begin{beispiel}[FA 2.6]{1} %PUNKTE DES BEISPIELS
Gegeben ist eine lineare Funktion $f$ mit der Gleichung $f(x)=k\cdot x+d$ $(\text{mit }k\in\mathbb{R^+} \text{ und }d\in\mathbb{R})$.

\lueckentext{
				text={$f$ beschreibt immer dann auch einen \gap Zusammenhang, wenn \gap gilt.}, 	%Lueckentext Luecke=\gap
				L1={direkt proportionalen}, 		%1.Moeglichkeit links  
				L2={indirekt proportionalen}, 		%2.Moeglichkeit links
				L3={exponentiellen}, 		%3.Moeglichkeit links
				R1={$k=-d$}, 		%1.Moeglichkeit rechts 
				R2={$k=\frac{1}{d}$}, 		%2.Moeglichkeit rechts
				R3={$d=0$}, 		%3.Moeglichkeit rechts
				%% LOESUNG: %%
				A1=1,   % Antwort links
				A2=3		% Antwort rechts 
				}
\end{beispiel}