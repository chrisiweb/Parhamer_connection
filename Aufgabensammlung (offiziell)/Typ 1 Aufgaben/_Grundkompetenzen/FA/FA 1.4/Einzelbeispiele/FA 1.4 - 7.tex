\section{FA 1.4 - 7 Funktionsgraphen - MC - BIFIE}

\begin{beispiel}[FA 1.4]{1} %PUNKTE DES BEISPIELS
Gegeben sind die Graphen der Funktionen $f$, $g$ und $h$.

\begin{center}
\resizebox{0.7\linewidth}{!}{\psset{xunit=1.0cm,yunit=1.0cm,algebraic=true,dimen=middle,dotstyle=o,dotsize=5pt 0,linewidth=0.8pt,arrowsize=3pt 2,arrowinset=0.25}
\begin{pspicture*}(-0.8534760248867364,-0.704367597661392)(7.9923785887858365,7.5352290674105715)
\multips(0,0)(0,1.0){9}{\psline[linestyle=dashed,linecap=1,dash=1.5pt 1.5pt,linewidth=0.4pt,linecolor=lightgray]{c-c}(-0.8534760248867364,0)(7.9923785887858365,0)}
\multips(0,0)(1.0,0){9}{\psline[linestyle=dashed,linecap=1,dash=1.5pt 1.5pt,linewidth=0.4pt,linecolor=lightgray]{c-c}(0,-0.704367597661392)(0,7.5352290674105715)}
\psaxes[labelFontSize=\scriptstyle,xAxis=true,yAxis=true,Dx=1.,Dy=1.,ticksize=-2pt 0,subticks=2]{->}(0,0)(0.,0.)(7.9923785887858365,7.5352290674105715)
\psplot[plotpoints=200]{0.0001}{7.9923785887858365}{1.0/x}
\rput[tl](-0.35,7.5){$y$}
\rput[tl](7.6,-0.2){$x$}
\psplot[plotpoints=200]{-0.8534760248867364}{7.9923785887858365}{x}
\rput[tl](6.090933204538461,7){$g$}
\psplot[plotpoints=200]{-0.8534760248867364}{7.9923785887858365}{-x+4.0}
\rput[tl](2.9494147436080147,1.6){$h$}
\begin{scriptsize}
\rput[bl](0.3314826928326426,6.3502703496911925){\normalsize{$f$}}
\end{scriptsize}
\end{pspicture*}}
\end{center}

Kreuze die beiden zutreffenden Aussagen an.

\multiplechoice[5]{  %Anzahl der Antwortmoeglichkeiten, Standard: 5
				L1={$g(1)>g(3)$},   %1. Antwortmoeglichkeit 
				L2={$h(1)>h(3)$},   %2. Antwortmoeglichkeit
				L3={$f(1)=g(1)$},   %3. Antwortmoeglichkeit
				L4={$h(1)=g(1)$},   %4. Antwortmoeglichkeit
				L5={$f(1)<f(3)$},	 %5. Antwortmoeglichkeit
				L6={},	 %6. Antwortmoeglichkeit
				L7={},	 %7. Antwortmoeglichkeit
				L8={},	 %8. Antwortmoeglichkeit
				L9={},	 %9. Antwortmoeglichkeit
				%% LOESUNG: %%
				A1=2,  % 1. Antwort
				A2=3,	 % 2. Antwort
				A3=0,  % 3. Antwort
				A4=0,  % 4. Antwort
				A5=0,  % 5. Antwort
				}

\end{beispiel}