\section{FA 1.4 - 13 Zerfallsprozess - MC - Matura 2013/14 Haupttermin}

\begin{beispiel}[FA 1.4]{1} %PUNKTE DES BEISPIELS
				Der unten abgebildete Graph einer Funktion $N$ stellt einen exponentiellen Zerfallsprozess dar; dabei bezeichnet $t$ die Zeit und $N(t)$ die zum Zeitpunkt $t$ vorhandene Menge des zerfallenden Stoffes.  F�r die zum Zeitpunkt $t=0$ vorhandene Menge gilt: $N(0) = 800$.
				
				\begin{center}
				\resizebox{0.8\linewidth}{!}{\psset{xunit=1.3cm,yunit=0.5cm,algebraic=true,dimen=middle,dotstyle=o,dotsize=5pt 0,linewidth=0.8pt,arrowsize=3pt 2,arrowinset=0.25}
\begin{pspicture*}(-0.6328888888888822,-0.8651252701880512)(10.469037037037031,11.545255598912444)
\multips(0,0)(0,1.0){13}{\psline[linestyle=dashed,linecap=1,dash=1.5pt 1.5pt,linewidth=0.4pt,linecolor=darkgray]{c-c}(-0.6328888888888822,0)(10.469037037037031,0)}
\multips(0,0)(1.0,0){12}{\psline[linestyle=dashed,linecap=1,dash=1.5pt 1.5pt,linewidth=0.4pt,linecolor=darkgray]{c-c}(0,-0.8651252701880512)(0,11.545255598912444)}
\psaxes[labelFontSize=\scriptstyle,xAxis=true,yAxis=true,labels=x,Dx=1.,Dy=1.,ticksize=-2pt 0,subticks=2]{->}(0,0)(-0.6328888888888822,-0.8651252701880512)(10.469037037037031,11.545255598912444)
\psplot[linewidth=1.2pt,plotpoints=200]{0}{10.469037037037031}{8.000000000000002*EXP(-0.2310490601866484*x)}
\begin{scriptsize}
\rput[tl](9.401037037037034,0.7){t}
\rput[tl](0.21096296296296882,11.14689769447218){N(t)}
\rput[tl](2.386518518518522,5.63117286376085){N}
\end{scriptsize}
\end{pspicture*}}
\end{center}

Mit $t_H$ ist diejenige Zeitspanne gemeint, nach deren Ablauf die urspr�ngliche Menge des zerfallenden Stoffes auf die H�lfte gesunken ist.

Kreuze die beiden zutreffenden Aussagen an!
\leer

\multiplechoice[5]{  %Anzahl der Antwortmoeglichkeiten, Standard: 5
				L1={$t_H=6$},   %1. Antwortmoeglichkeit 
				L2={$t_H=2$},   %2. Antwortmoeglichkeit
				L3={$t_H=3$},   %3. Antwortmoeglichkeit
				L4={$N(t_H)=400$},   %4. Antwortmoeglichkeit
				L5={$N(t_H)=500$},	 %5. Antwortmoeglichkeit
				L6={},	 %6. Antwortmoeglichkeit
				L7={},	 %7. Antwortmoeglichkeit
				L8={},	 %8. Antwortmoeglichkeit
				L9={},	 %9. Antwortmoeglichkeit
				%% LOESUNG: %%
				A1=3,  % 1. Antwort
				A2=4,	 % 2. Antwort
				A3=0,  % 3. Antwort
				A4=0,  % 4. Antwort
				A5=0,  % 5. Antwort
				}
\end{beispiel}