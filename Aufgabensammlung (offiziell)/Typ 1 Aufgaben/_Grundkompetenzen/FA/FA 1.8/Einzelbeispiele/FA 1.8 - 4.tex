\section{FA 1.8 - 4 Quadratische Pyramide - MC - Matura 17/18}

\begin{beispiel}[FA 1.8]{1} %PUNKTE DES BEISPIELS
Die Oberfl�che einer regelm��igen quadratischen Pyramide kann als Funktion $O$ in Abh�ngigkeit von der L�nge der Grundkante $a$ und der H�he der Seitenfl�che $h_1$ aufgefasst werden.

Es gilt: $O(a,h_1)=a^2+2\cdot a\cdot h_1$, wobei $a\in\mathbb{R}^+$ und $h_1>\frac{a}{2}$.

Gegeben sind sechs Aussagen zur Oberfl�che von regelm��igen quadratischen Pyramiden. Kreuze die zutreffende Aussage an.\leer

\multiplechoice[6]{  %Anzahl der Antwortmoeglichkeiten, Standard: 5
				L1={Ist $h_1$ konstant, dann ist die Oberfl�che direkt proportional zu $a$.},   %1. Antwortmoeglichkeit 
				L2={Ist $a$ konstant, dann ist die Oberfl�che direkt proportional zu $h_1$.},   %2. Antwortmoeglichkeit
				L3={F�r $a=1$\,cm ist die Oberfl�che sicher gr��er als 2\,cm$^2$.},   %3. Antwortmoeglichkeit
				L4={F�r $a=1$ ist die Oberfl�che sicher kleiner als 10\,cm$^2$.},   %4. Antwortmoeglichkeit
				L5={Werden sowohl $a$ als auch $h_1$ verdoppelt, so wird die Oberfl�che verdoppelt.},	 %5. Antwortmoeglichkeit
				L6={Ist $h_1=a^2$, dann kann die Oberfl�che durch eine Exponentialfunktion in Abh�ngigkeit von $a$ beschrieben werden.},	 %6. Antwortmoeglichkeit
				L7={},	 %7. Antwortmoeglichkeit
				L8={},	 %8. Antwortmoeglichkeit
				L9={},	 %9. Antwortmoeglichkeit
				%% LOESUNG: %%
				A1=3,  % 1. Antwort
				A2=0,	 % 2. Antwort
				A3=0,  % 3. Antwort
				A4=0,  % 4. Antwort
				A5=0,  % 5. Antwort
				}
\end{beispiel}