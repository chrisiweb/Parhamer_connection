\section{FA 2.3 - 4 Lineare Funktion - OA - BIFIE}

\begin{beispiel}[FA 2.3]{1} %PUNKTE DES BEISPIELS
Die Gerade $g$ ist sowohl durch ihren Graphen als auch durch ihre Gleichung $y=\frac{3}{2}\cdot x-3$ festgelegt. Au�erdem ist ein Steigungsdreieck eingezeichnet, allerdings fehlt die x-Achse.

\begin{center}
\psset{xunit=1.0cm,yunit=1.0cm,algebraic=true,dimen=middle,dotstyle=o,dotsize=5pt 0,linewidth=0.8pt,arrowsize=3pt 2,arrowinset=0.25}
\begin{pspicture*}(-5.492806343293466,-4.508969377686958)(2.2961544532774445,3.6516638220398074)
\psline[linewidth=1.6pt](-4.991856582518158,3.393109106800941)(-5.,2.)
\psline[linewidth=1.6pt](-5.,2.)(-3.6667636669189574,2.003377512392025)
\psline[linewidth=1.6pt](-4.994154701026502,2.9999658313124113)(-4.,3.)
\psline[linewidth=1.6pt](-4.,3.)(-4.,2.)
\psline[linewidth=1.6pt](-1.,-4.)(-1.,3.)
\psline[linewidth=1.6pt](-1.,-1.)(1.,-1.)
\psline[linewidth=1.6pt](1.,-1.)(1.,2.)
\psline[linewidth=1.6pt,linecolor=blue](1.7011940734961382,3.0517911102442072)(-3.,-4.)
\begin{scriptsize}
\rput[bl](-5.234251628054598,2.407369254952756){1}
\rput[bl](-4.539385830850139,1.6963437880458694){1}
\psdots[dotsize=3pt 0,dotstyle=triangle*](-1.,3.)
\rput[bl](-0.8388214689938348,2.7305626490013406){y}
\rput[bl](0.13075871315192172,-1.2770354372011103){2}
\rput[bl](1.1811372438098247,0.2581331845296673){3}
\rput[bl](-0.24091369000395163,0.6944442664952567){\blue{g}}
\antwort{\psline[linewidth=1.6pt](-3.,2.)(2.,2.)
\psdots[dotsize=3pt 0,dotstyle=triangle*,dotangle=270](2.,2.)
\rput[bl](1.7952046925021372,1.7125034577482987){x}}
\end{scriptsize}
\end{pspicture*}
\end{center}

Zeichne die x-Achse so ein, dass die dargestellte Gerade die gegebene Gleichung hat!
\end{beispiel}