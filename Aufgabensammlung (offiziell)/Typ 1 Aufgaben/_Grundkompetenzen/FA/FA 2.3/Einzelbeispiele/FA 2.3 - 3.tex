\section{FA 2.3 - 3 Zeit-Weg-Diagramm, Geschwindigkeiten - ZO - BIFIE}

\begin{beispiel}[FA 2.3]{1} %PUNKTE DES BEISPIELS
Das folgende Zeit-Weg-Diagramm stellt eine Bewegung dar. Der Weg wird in Metern $(m)$, die Zeit in Sekunden $(s)$ gemessen. Zur Beschreibung dieser Bewegung sind zudam verschiedene Geschwindigkeiten $(v_x)$ gegeben.

\begin{center}
\resizebox{0.5\linewidth}{!}{\psset{xunit=1.0cm,yunit=0.05cm,algebraic=true,dimen=middle,dotstyle=o,dotsize=5pt 0,linewidth=0.8pt,arrowsize=3pt 2,arrowinset=0.25}
\begin{pspicture*}(-1.6826568265682644,-29.614778390105094)(6.999852398523986,120.14167968844762)
\multips(0,0)(0,10.0){15}{\psline[linestyle=dashed,linecap=1,dash=1.5pt 1.5pt,linewidth=0.4pt,linecolor=lightgray]{c-c}(0,0)(6.999852398523986,0)}
\multips(0,0)(0.5,0){18}{\psline[linestyle=dashed,linecap=1,dash=1.5pt 1.5pt,linewidth=0.4pt,linecolor=lightgray]{c-c}(0,0)(0,120.14167968844762)}
\psaxes[labelFontSize=\scriptstyle,xAxis=true,yAxis=true,Dx=1.,Dy=20.,ticksize=-2pt 0,subticks=2]{->}(0,0)(0.,0.)(6.999852398523986,120.14167968844762)
\psline[linewidth=1.6pt](0.,0.)(1.5,30.)
\psline[linewidth=1.6pt](1.5,30.)(3.,30.)
\psline[linewidth=1.6pt](3.,30.)(4.,80.)
\psline[linewidth=1.6pt](4.,80.)(6.,100.)
\rput[tl](0.413431734317342,113.0745209926058){Weg (in m)}
\rput[tl](5.2051660516605175,10.957582737299049){Zeit (in s)}
\end{pspicture*}}
\end{center}

Ordne jeweils jedem Zeitintervall jene Geschwindigkeit zu, die der Bewegung in diesem Intervall entspricht!

\zuordnen{
				title1={Geschwindigkeit}, 		%Titel Antwortmoeglichkeiten
				A={$v_A=0\,m/s$}, 				%Moeglichkeit A  
				B={$v_B=5\,m/s$}, 				%Moeglichkeit B  
				C={$v_C=10\,m/s$}, 				%Moeglichkeit C  
				D={$v_D=20\,m/s$}, 				%Moeglichkeit D  
				E={$v_E=25\,m/s$}, 				%Moeglichkeit E  
				F={$v_F=50\,m/s$}, 				%Moeglichkeit F  
				title2={Zeitintervall},		%Titel Zuordnung
				R1={$[0; 1,5]$},				%1. Antwort rechts
				R2={$[1,5; 3]$},				%2. Antwort rechts
				R3={$[3; 4]$},				%3. Antwort rechts
				R4={$[4; 6]$},				%4. Antwort rechts
				%% LOESUNG: %%
				A1={D},				% 1. richtige Zuordnung
				A2={A},				% 2. richtige Zuordnung
				A3={F},				% 3. richtige Zuordnung
				A4={C},				% 4. richtige Zuordnung
				}
\end{beispiel}