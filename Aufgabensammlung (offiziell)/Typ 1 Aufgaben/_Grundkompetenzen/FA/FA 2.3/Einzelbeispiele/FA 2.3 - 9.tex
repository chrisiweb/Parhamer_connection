\section{FA 2.3 - 9 Lineare Funktionen - ZO - Matura 2016/17 - Haupttermin}

\begin{beispiel}[FA 2.3]{1} %PUNKTE DES BEISPIELS
Gegeben sind die Graphen von vier verschiedenen linearen Funktionen $f$ mit $f(x) = k \cdot x + d$, wobei $k, d \in \mathbb{R}$. \leer

Ordne den vier Graphen jeweils die entsprechende Aussage �ber die Parameter $k$ und $d$ (aus A bis F) zu!

\zuordnen[0.05]{
				R1={\resizebox{0.8\linewidth}{!}{\psset{xunit=1.0cm,yunit=1.0cm,algebraic=true,dimen=middle,dotstyle=o,dotsize=5pt 0,linewidth=0.8pt,arrowsize=3pt 2,arrowinset=0.25}
\begin{pspicture*}(-3.5,-3.5)(2.5,2.5)
\multips(0,-3)(0,1.0){7}{\psline[linestyle=dashed,linecap=1,dash=1.5pt 1.5pt,linewidth=0.4pt,linecolor=black!70]{c-c}(-3.5624604195235996,0)(2.6262646545420583,0)}
\multips(-3,0)(1.0,0){7}{\psline[linestyle=dashed,linecap=1,dash=1.5pt 1.5pt,linewidth=0.4pt,linecolor=black!70]{c-c}(0,-3.5612490417004694)(0,2.5725465198734816)}
\psaxes[labelFontSize=\scriptstyle,xAxis=true,yAxis=true,Dx=1.,Dy=1.,ticksize=-2pt 0,subticks=2]{->}(0,0)(-3.5,-3.5)(2.5,2.5)[$x$,140] [$f(x)$,-40]
\psplot[plotpoints=200]{-3.5624604195235996}{2.6262646545420583}{1.0/3.0*x-2.0}
\begin{scriptsize}
\rput[bl](-2.75682756964523,-2.7189965168276284){$f$}
\end{scriptsize}
\end{pspicture*}}},				% Response 1
				R2={\resizebox{0.8\linewidth}{!}{\psset{xunit=1.0cm,yunit=1.0cm,algebraic=true,dimen=middle,dotstyle=o,dotsize=5pt 0,linewidth=0.8pt,arrowsize=3pt 2,arrowinset=0.25}
\begin{pspicture*}(-3.5,-3.5)(2.5,2.5)
\multips(0,-3)(0,1.0){7}{\psline[linestyle=dashed,linecap=1,dash=1.5pt 1.5pt,linewidth=0.4pt,linecolor=black!70]{c-c}(-3.5624604195235996,0)(2.6262646545420583,0)}
\multips(-3,0)(1.0,0){7}{\psline[linestyle=dashed,linecap=1,dash=1.5pt 1.5pt,linewidth=0.4pt,linecolor=black!70]{c-c}(0,-3.5612490417004694)(0,2.5725465198734816)}
\psaxes[labelFontSize=\scriptstyle,xAxis=true,yAxis=true,Dx=1.,Dy=1.,ticksize=-2pt 0,subticks=2]{->}(0,0)(-3.5,-3.5)(2.5,2.5)[$x$,140] [$f(x)$,-40]
\psplot[plotpoints=200]{-3.5624604195235996}{2.6262646545420583}{-3.0/2.0*x+1.0}
\begin{scriptsize}
\rput[bl](-0.6,1.2){$f$}
\end{scriptsize}
\end{pspicture*}}},				% Response 2
				R3={\resizebox{0.8\linewidth}{!}{\psset{xunit=1.0cm,yunit=1.0cm,algebraic=true,dimen=middle,dotstyle=o,dotsize=5pt 0,linewidth=0.8pt,arrowsize=3pt 2,arrowinset=0.25}
\begin{pspicture*}(-3.5,-3.5)(2.5,2.5)
\multips(0,-3)(0,1.0){7}{\psline[linestyle=dashed,linecap=1,dash=1.5pt 1.5pt,linewidth=0.4pt,linecolor=black!70]{c-c}(-3.5624604195235996,0)(2.6262646545420583,0)}
\multips(-3,0)(1.0,0){7}{\psline[linestyle=dashed,linecap=1,dash=1.5pt 1.5pt,linewidth=0.4pt,linecolor=black!70]{c-c}(0,-3.5612490417004694)(0,2.5725465198734816)}
\psaxes[labelFontSize=\scriptstyle,xAxis=true,yAxis=true,Dx=1.,Dy=1.,ticksize=-2pt 0,subticks=2]{->}(0,0)(-3.5,-3.5)(2.5,2.5)[$x$,140] [$f(x)$,-40]
\psplot[plotpoints=200]{-3.5624604195235996}{2.6262646545420583}{2}
\begin{scriptsize}
\rput[bl](-2.75682756964523,1.5){$f$}
\end{scriptsize}
\end{pspicture*}}},				% Response 3
				R4={\resizebox{0.8\linewidth}{!}{\psset{xunit=1.0cm,yunit=1.0cm,algebraic=true,dimen=middle,dotstyle=o,dotsize=5pt 0,linewidth=0.8pt,arrowsize=3pt 2,arrowinset=0.25}
\begin{pspicture*}(-3.5,-3.5)(2.5,2.5)
\multips(0,-3)(0,1.0){7}{\psline[linestyle=dashed,linecap=1,dash=1.5pt 1.5pt,linewidth=0.4pt,linecolor=black!70]{c-c}(-3.5624604195235996,0)(2.6262646545420583,0)}
\multips(-3,0)(1.0,0){7}{\psline[linestyle=dashed,linecap=1,dash=1.5pt 1.5pt,linewidth=0.4pt,linecolor=black!70]{c-c}(0,-3.5612490417004694)(0,2.5725465198734816)}
\psaxes[labelFontSize=\scriptstyle,xAxis=true,yAxis=true,Dx=1.,Dy=1.,ticksize=-2pt 0,subticks=2]{->}(0,0)(-3.5,-3.5)(2.5,2.5)[$x$,140] [$f(x)$,-40]
\psplot[plotpoints=200]{-3.5624604195235996}{2.6262646545420583}{-0.45*x-1}
\begin{scriptsize}
\rput[bl](-2.75682756964523,-2.7189965168276284){$f$}
\end{scriptsize}
\end{pspicture*}}},				% Response 4
				%% Moegliche Zuordnungen: %%
				A={$k=0, d<0$}, 				%Moeglichkeit A  
				B={$k>0, d>0$}, 				%Moeglichkeit B  
				C={$k=0, d>0$}, 				%Moeglichkeit C  
				D={$k<0, d<0$}, 				%Moeglichkeit D  
				E={$k>0, d<0$}, 				%Moeglichkeit E  
				F={$k<0, d>0$}, 				%Moeglichkeit F  
				%% LOESUNG: %%
				A1={E},				% 1. richtige Zuordnung
				A2={F},				% 2. richtige Zuordnung
				A3={C},				% 3. richtige Zuordnung
				A4={D},				% 4. richtige Zuordnung
				}
\end{beispiel}