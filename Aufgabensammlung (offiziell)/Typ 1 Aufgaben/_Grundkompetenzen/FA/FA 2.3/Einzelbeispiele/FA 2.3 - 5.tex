\section{FA 2.3 - 5 Produktionskosten - OA - Matura 2014/15 - Haupttermin}

\begin{beispiel}[FA 2.3]{1} %PUNKTE DES BEISPIELS
Ein Betrieb gibt f�r die Absch�tzung der Gesamtkosten $K(x)$ f�r $x$ produzierte St�ck einer Ware folgende Gleichung an: $K(x) = 25x + 12\,000$. \leer

Interpretiere die beiden Zahlenwerte 25 und 12\,000 in diesem Kontext.

\antwort{
25 \ldots

\ldots der Kostenzuwachs f�r die Produktion eines weiteren St�cks \\
\ldots zus�tzliche (variable) Kosten, die pro St�ck f�r die Produktion anfallen \\


12\,000 \ldots 

\ldots Fixkosten \\
\ldots jene Kosten, die unabh�ngig von der produzierten St�ckzahl anfallen}

\end{beispiel}