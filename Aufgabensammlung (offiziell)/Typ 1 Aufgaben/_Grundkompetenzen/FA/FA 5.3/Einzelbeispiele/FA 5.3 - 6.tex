\section{FA 5.3 - 6 Parameter von Exponentialfunktionen - LT - Matura 2015/16 - Haupttermin}

\begin{beispiel}[FA 5.3]{1} %PUNKTE DES BEISPIELS
Die nachstehende Abbildung zeigt die Graphen zweier Exponentialfunktionen $f$ und $g$ mit den
Funktionsgleichungen $f(x) = c \cdot a^x$ und $g(x) = d \cdot b^x$ mit $a, b, c, d \in \mathbb{R}^+$.\leer


\begin{center}
\resizebox{0.7\linewidth}{!}{
\psset{xunit=1.0cm,yunit=1.0cm,algebraic=true,dimen=middle,dotstyle=o,dotsize=5pt 0,linewidth=0.8pt,arrowsize=3pt 2,arrowinset=0.25}
\begin{pspicture*}(-0.475913492966801,-0.5409316646749739)(6.276768452132614,6.363651101275364)
\psaxes[labelFontSize=\scriptstyle,xAxis=true,yAxis=true,labels=none,Dx=1.,Dy=1.,ticksize=-3pt 0,subticks=0]{->}(0,0)(-0.475913492966801,-0.5409316646749739)(6.276768452132614,6.363651101275364)[$x$,140] [\text{$f(x)$, $g(x)$},-40]
\psplot[linewidth=1.2pt,plotpoints=200]{-0.475913492966801}{6.276768452132614}{1.5*1.3^(x)}
\psplot[linewidth=1.2pt,plotpoints=200]{-0.475913492966801}{6.276768452132614}{2.5*1.19^(x)}
\rput[tl](1.0845222121379718,3.588008829363328){$f$}
\rput[tl](1.5540338402225937,2.1){$g$}
\end{pspicture*}}
\end{center}

\lueckentext{
				text={F�r die Parameter $a, b, c, d$ der beiden gegebenen Exponentialfunktionen gelten die Beziehungen \gap und \gap.}, 	%Lueckentext Luecke=\gap
				L1={$c<d$}, 		%1.Moeglichkeit links  
				L2={$c=d$}, 		%2.Moeglichkeit links
				L3={$c>d$}, 		%3.Moeglichkeit links
				R1={$a<b$}, 		%1.Moeglichkeit rechts 
				R2={$a=b$}, 		%2.Moeglichkeit rechts
				R3={$a>b$}, 		%3.Moeglichkeit rechts
				%% LOESUNG: %%
				A1=3,   % Antwort links
				A2=1		% Antwort rechts 
				}
\end{beispiel}