\section{FA 5.3 - 4 Exponentialfunktionen vergleichen - MC - BIFIE}

\begin{beispiel}[FA 5.3]{1} %PUNKTE DES BEISPIELS
Gegeben sind die zwei Exponentialfunktionen $f$ und $h$ mit $f(x)=a\cdot b^x$ und $h(x)=c\cdot d^x$. Dabei gilt: $a,b,c,d \in \mathbb{R}^+$.
\leer

\begin{center}
\psset{xunit=1.0cm,yunit=1.0cm,algebraic=true,dimen=middle,dotstyle=o,dotsize=5pt 0,linewidth=0.8pt,arrowsize=3pt 2,arrowinset=0.25}
\begin{pspicture*}(-3.6400723423008503,-0.6922911807133295)(4.587173783981014,5.662409822395419)
\multips(0,0)(0,1.0){7}{\psline[linestyle=dashed,linecap=1,dash=1.5pt 1.5pt,linewidth=0.4pt,linecolor=darkgray]{c-c}(-3.6400723423008503,0)(4.587173783981014,0)}
\multips(-3,0)(1.0,0){9}{\psline[linestyle=dashed,linecap=1,dash=1.5pt 1.5pt,linewidth=0.4pt,linecolor=darkgray]{c-c}(0,-0.6922911807133295)(0,5.662409822395419)}
\psaxes[labelFontSize=\scriptstyle,xAxis=true,yAxis=true,Dx=1.,Dy=1.,ticksize=-2pt 0,subticks=2]{->}(0,0)(-3.6400723423008503,-0.6922911807133295)(4.587173783981014,5.662409822395419)
\psplot[plotpoints=200]{-3.6400723423008503}{4.587173783981014}{1.5*2.0^(x)}
\psplot[plotpoints=200]{-3.6400723423008503}{4.587173783981014}{2.0*1.5^(x)}
\begin{scriptsize}
\rput[bl](-3.1818964079074292,0.22406068807351195){$f$}
\rput[bl](-3.381103335904569,0.6224745440677909){$h$}
\end{scriptsize}
\end{pspicture*}
\end{center}

\leer

Welche der nachstehenden Aussagen �ber die Parameter $a, b, c$ und $d$ sind zutreffend? 
Kreuze die beiden zutreffenden Aussagen an.

\multiplechoice[5]{  %Anzahl der Antwortmoeglichkeiten, Standard: 5
				L1={$a>c$},   %1. Antwortmoeglichkeit 
				L2={$b>d$},   %2. Antwortmoeglichkeit
				L3={$a<c$},   %3. Antwortmoeglichkeit
				L4={$b<d$},   %4. Antwortmoeglichkeit
				L5={$a=c$},	 %5. Antwortmoeglichkeit
				L6={},	 %6. Antwortmoeglichkeit
				L7={},	 %7. Antwortmoeglichkeit
				L8={},	 %8. Antwortmoeglichkeit
				L9={},	 %9. Antwortmoeglichkeit
				%% LOESUNG: %%
				A1=2,  % 1. Antwort
				A2=3,	 % 2. Antwort
				A3=0,  % 3. Antwort
				A4=0,  % 4. Antwort
				A5=0,  % 5. Antwort
				}
\end{beispiel}