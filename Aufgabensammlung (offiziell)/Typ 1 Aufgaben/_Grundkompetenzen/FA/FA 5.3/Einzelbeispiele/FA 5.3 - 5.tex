\section{FA 5.3 - 5 Bakterienkolonie - OA - BIFIE}

\begin{beispiel}[FA 5.3]{1} %PUNKTE DES BEISPIELS
Das Wachstum einer Bakterienkolonie in Abh�ngigkeit von der Zeit $t$ (in Stunden) kann n�herungsweise
durch die Funktionsgleichung $A = 2 \cdot 1,35^t$
 beschrieben werden, wobei $A(t)$ die
zum Zeitpunkt $t$ besiedelte Fl�che (in $\text{mm}�$) angibt. 

\leer

Interpretiere die in der Funktionsgleichung vorkommenden Werte $2$ und $1,35$ im Hinblick auf den Wachstumsprozess.


\antwort{Zum Zeitpunkt $t = 0$ betr�gt der Inhalt der besiedelten Fl�che $2\,\text{mm}^2$. Die Bakterienkolonie
w�chst pro Stunde um 35\%.\\

L�sungsschl�ssel:

Die Aufgabe ist als richtig gel�st zu werten, wenn die Interpretation beider Werte sinngem�� richtig ist. Die Einheit muss nicht angegeben sein. }
\end{beispiel}