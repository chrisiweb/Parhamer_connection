\section{FA 1.2 - 3 Quadratisches Prisma - OA - BIFIE}

\begin{beispiel}[FA 1.2]{1} %PUNKTE DES BEISPIELS
Das Volumen $V$ eines geraden quadratischen Prismas h�ngt von der Seitenl�nge $a$ der quadratischen Grundfl�che und von der H�he $h$ ab. Es wird durch die Formel $V=a�\cdot h$ beschrieben.

Stelle die Abh�ngigkeit des Volumens $V(a)$ in $cm�$ eines geraden quadratischen Prismas von der Seitenl�nge $a$ in cm bei konstanter H�he $h=5\,cm$ durch einen entsprechenden Funktionsgraphen im Intervall $[0;4]$ dar!
\leer

\begin{center}
\resizebox{0.6\linewidth}{!}{
\psset{xunit=1.0cm,yunit=0.1cm,algebraic=true,dimen=middle,dotstyle=o,dotsize=5pt 0,linewidth=0.8pt,arrowsize=3pt 2,arrowinset=0.25}
\begin{pspicture*}(-0.7275874070558441,-6.4824687309738005)(5.1325823973,90.30221723643636)
\multips(0,0)(0,5.0){20}{\psline[linestyle=dashed,linecap=1,dash=1.5pt 1.5pt,linewidth=0.4pt,linecolor=lightgray]{c-c}(0,0)(5,0)}
\multips(0,0)(1.0,0){6}{\psline[linestyle=dashed,linecap=1,dash=1.5pt 1.5pt,linewidth=0.4pt,linecolor=lightgray]{c-c}(0,0)(0,90.30221723643636)}
\psaxes[labelFontSize=\scriptstyle,xAxis=true,yAxis=true,Dx=1.,Dy=5.,ticksize=-2pt 0,subticks=2]{->}(0,0)(0.,0.)(5.1325823973,90.30221723643636)[\scriptsize a in cm,140] [\scriptsize V(a) in cm$^3$,-40]
\antwort{\psplot[linewidth=1.2pt,linecolor=red,plotpoints=200]{0}{4}{5.0*x^(2.0)-3.0E-50*x}
\rput[tl](3.4990855877234583,53.95691322335828){\red{v}}}
\end{pspicture*}}
\end{center}
\leer

\antwort{Ein Punkt ist genau dann zu geben, wenn der darestellte Graph als Parabel erkennbar ist (bzw. links gekr�mmt ist) und die Punkte (1/5), (2/20), (3/45) sowie (4/80) enth�lt.}
\end{beispiel}