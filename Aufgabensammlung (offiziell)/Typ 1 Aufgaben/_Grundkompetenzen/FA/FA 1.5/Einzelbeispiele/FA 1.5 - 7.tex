\section{FA 1.5 - 7 Nullstellen einer Funktion - OA - BIFIE}

\begin{beispiel}[FA 1.5]{1} %PUNKTE DES BEISPIELS
Eine Funktion ist durch die Gleichung $f(x)=x\cdot (x-1) \cdot (x+1)$ gegeben.

\leer

Kennzeichne im gegebenen Koordinatensystem alle Nullstellen des Funktionsgraphen durch Punkte.
\leer

\begin{center}
\resizebox{0.8\linewidth}{!}{\psset{xunit=1.0cm,yunit=1.0cm,algebraic=true,dimen=middle,dotstyle=o,dotsize=5pt 0,linewidth=0.8pt,arrowsize=3pt 2,arrowinset=0.25}
\begin{pspicture*}(-3.4807067999031385,-1.5488809711010314)(5.543367038367279,2.8535921376357503)
\multips(0,-1)(0,1.0){5}{\psline[linestyle=dashed,linecap=1,dash=1.5pt 1.5pt,linewidth=0.4pt,linecolor=lightgray]{c-c}(-3.4807067999031385,0)(5.543367038367279,0)}
\multips(-3,0)(1.0,0){10}{\psline[linestyle=dashed,linecap=1,dash=1.5pt 1.5pt,linewidth=0.4pt,linecolor=lightgray]{c-c}(0,-1.5488809711010314)(0,2.8535921376357503)}
\psaxes[labelFontSize=\scriptstyle,xAxis=true,yAxis=true,Dx=1.,Dy=1.,ticksize=-2pt 0,subticks=2]{->}(0,0)(-3.4807067999031385,-1.5488809711010314)(5.543367038367279,2.8535921376357503)[x,140] [f(x),-40]
\antwort{\begin{scriptsize}
\psdots[dotstyle=*,linecolor=red](0.,0.)
\rput[bl](0.08509721124565758,0.283822766472651){$N_2$}
\psdots[dotstyle=*,linecolor=red](-1.,0.)
\rput[bl](-1.3691133631334604,0.3037434592723649){$N_1$}
\psdots[dotstyle=*,linecolor=red](1.,0.)
\rput[bl](1.1010525440310688,0.22406068807350912){$N_3$}
\end{scriptsize}}
\end{pspicture*}}
\end{center}


\end{beispiel}