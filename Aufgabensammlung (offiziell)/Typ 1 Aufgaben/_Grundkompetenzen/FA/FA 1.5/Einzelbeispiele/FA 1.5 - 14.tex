\section{FA 1.5 - 14 Quadratische Funktion und ihre Nullstellen - OA - Matura 2014/15 - Kompensationspr�fung}

\begin{beispiel}[FA 1.5]{1} %PUNKTE DES BEISPIELS
				Skizziere den Graphen einer m�glichen quadratischen Funktion, die in $P=(0|-1)$ ein lokales Minimum (einen Tiefpunkt) hat, und gib die Anzahl der Nullstellen dieser Funktion an.
				\begin{center}
					\resizebox{0.7\linewidth}{!}{\psset{xunit=1.0cm,yunit=1.0cm,algebraic=true,dimen=middle,dotstyle=o,dotsize=5pt 0,linewidth=0.8pt,arrowsize=3pt 2,arrowinset=0.25}
\begin{pspicture*}(-5.62,-5.5)(6.62,5.76)
\multips(0,-5)(0,1.0){12}{\psline[linestyle=dashed,linecap=1,dash=1.5pt 1.5pt,linewidth=0.4pt,linecolor=lightgray]{c-c}(-5.62,0)(6.62,0)}
\multips(-5,0)(1.0,0){13}{\psline[linestyle=dashed,linecap=1,dash=1.5pt 1.5pt,linewidth=0.4pt,linecolor=lightgray]{c-c}(0,-5.5)(0,5.76)}
\psaxes[labelFontSize=\scriptstyle,xAxis=true,yAxis=true,Dx=1.,Dy=1.,ticksize=-2pt 0,subticks=2]{->}(0,0)(-5.62,-5.5)(6.62,5.76)[x,140] [f(x),-40]
\antwort{\psplot[linewidth=1.2pt,plotpoints=200]{-5.620000000000001}{6.620000000000004}{x^(2.0)-1.0}
\begin{scriptsize}
\rput[bl](-2.72,4.28){$f$}
\end{scriptsize}}
\end{pspicture*}}
				\end{center}
				
				\antwort{Diese Funktion hat jedenfalls zwei Nullstellen.}
\end{beispiel}