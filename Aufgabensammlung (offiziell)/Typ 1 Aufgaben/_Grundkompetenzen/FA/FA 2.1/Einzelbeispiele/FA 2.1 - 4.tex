\section{FA 2.1 - 4 Lineare Gleichung - lineare Funktion - OA - BIFIE}

\begin{beispiel}[FA 2.1]{1} %PUNKTE DES BEISPIELS
Eine lineare Funktion $y=f(x)$ kann durch eine Gleichung $a\cdot x+b\cdot y=0$ mit $a,b\in\mathbb{R^+}$ festgelegt werden.

Gib einen Funktionsterm von $f$ an und skizziere, wie der Graph aussehen k�nnte!

\begin{center}
\psset{xunit=1.0cm,yunit=1.0cm,algebraic=true,dimen=middle,dotstyle=o,dotsize=5pt 0,linewidth=0.8pt,arrowsize=3pt 2,arrowinset=0.25}
\begin{pspicture*}(-3.8085226694288887,-2.5260519073983163)(5.741163459241477,4.745379081231383)
\multips(0,-2)(0,1.0){8}{\psline[linestyle=dashed,linecap=1,dash=1.5pt 1.5pt,linewidth=0.4pt,linecolor=lightgray]{c-c}(-3.8085226694288887,0)(5.741163459241477,0)}
\multips(-3,0)(1.0,0){10}{\psline[linestyle=dashed,linecap=1,dash=1.5pt 1.5pt,linewidth=0.4pt,linecolor=lightgray]{c-c}(0,-2.5260519073983163)(0,4.745379081231383)}
\psaxes[labelFontSize=\scriptstyle,xAxis=true,yAxis=true,Dx=1.,Dy=1.,ticksize=-2pt 0,subticks=2]{->}(0,0)(-3.8085226694288887,-2.5260519073983163)(5.741163459241477,4.745379081231383)[x,140] [f(x),-40]
\antwort{\psplot[linewidth=1.2pt,plotpoints=200]{-3.8085226694288887}{5.741163459241477}{-2.0*x/3.0}
\begin{scriptsize}
\rput[bl](-3.219973424918389,2.353211184188715){$f$}
\end{scriptsize}}
\end{pspicture*}
\end{center}

$f(x)=\rule{5cm}{0.3pt}$
\leer

\antwort{$f(x)=-\frac{a}{b}\cdot x$

Der Graph muss als Gerade erkennbar sein, durch den Ursprung gehen und monoton fallend sein.}
\end{beispiel}