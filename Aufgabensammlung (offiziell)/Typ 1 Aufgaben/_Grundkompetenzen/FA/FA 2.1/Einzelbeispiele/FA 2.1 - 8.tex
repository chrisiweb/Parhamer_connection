\section{FA 2.1 - 8 - Lineare Zusammenh�nge - MC - Matura - 1. NT 2017/18}

\begin{beispiel}[FA 2.1]{1}
Verbal gegebene Zusammenh�nge k�nnen in bestimmten F�llen als lineare Funktionen betrachtet werden.

Welche der folgenden Zusammenh�nge lassen sich mittels einer linearen Funktion beschreiben? Kreuze die beiden zutreffenden Zusammenh�nge an!

\multiplechoice[5]{  %Anzahl der Antwortmoeglichkeiten, Standard: 5
				L1={Die Wohnungskosten steigen j�hrlich um 10\,\% des aktuellen Wertes.},   %1. Antwortmoeglichkeit 
				L2={Der Fl�cheninhalt eines quadratischen Grundst�cks w�chst mit zunehmender Seitenl�nge.},   %2. Antwortmoeglichkeit
				L3={Der Umfang eines Kreises w�chst mit zunehmenden Radius.},   %3. Antwortmoeglichkeit
				L4={Die L�nge einer 17\,cm hohen Kerze nimmt nach dem Anz�nden in jeder Minute um 8\,mm ab.},   %4. Antwortmoeglichkeit
				L5={In einer Bakterienkultur verdoppelt sich st�ndlich die Anzahl der Bakterien.},	 %5. Antwortmoeglichkeit
				L6={},	 %6. Antwortmoeglichkeit
				L7={},	 %7. Antwortmoeglichkeit
				L8={},	 %8. Antwortmoeglichkeit
				L9={},	 %9. Antwortmoeglichkeit
				%% LOESUNG: %%
				A1=3,  % 1. Antwort
				A2=4,	 % 2. Antwort
				A3=0,  % 3. Antwort
				A4=0,  % 4. Antwort
				A5=0,  % 5. Antwort
				}
\end{beispiel}