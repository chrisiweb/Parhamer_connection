\section{FA 2.2 - 6 Steigung des Graphen einer linearen Funktion - OA - Matura 2013/14 1. Nebentermin}

\begin{beispiel}[FA 2.2]{1} %PUNKTE DES BEISPIELS
				Gegeben ist eine Gleichung einer Geraden $g$ in der Ebene: $3\cdot x+5\cdot y=15$.
				
				Gib die Steigung des Graphen der dieser Gleichung zugeordneten linearen Funktion an!\leer
				
				\antwort{Die Steigung der zugeordneten linearen Funktion betr�gt $-\frac{3}{5}$
				
				Ein Punkt f�r die richtige L�sung. Wird die Steigung der linearen Funktion z.B. mit $k$ oder mit $f'(x)$ bezeichnet, so ist dies als richtig zu werten. Jede korrekte Schreibweise des Ergebnisses (als �quivalenter Bruch oder als Dezimalzahl) ist als richtig zu werten.
}
\end{beispiel}