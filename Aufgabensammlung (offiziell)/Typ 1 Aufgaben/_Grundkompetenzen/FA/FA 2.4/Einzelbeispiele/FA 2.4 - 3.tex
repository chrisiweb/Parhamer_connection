\section{FA 2.4 - 3 Eigenschaften linearer Funktionen - OA - BIFIE}

\begin{beispiel}[FA 2.4]{1} %PUNKTE DES BEISPIELS
Gegeben ist eine lineare Funktion $f$ mit der Gleichung $f(x)=4x-2$.

W�hle zwei Argumente $x_1$ und $x_2$ mit $x_2=x_1+1$ und zeige, dass die Differenz $f(x_2)-f(x_1)$ gleich dem Wert der Steigung $k$ der gegebenen linearen Funktion $f$ ist!
\leer

\antwort{$f(x)=4x-2\rightarrow k=4$

$x_1=3$ und $f(x_1)=10$

$x_2=4$ und $f(x_2)=14$

$\rightarrow f(x_2)-f(x_1)=14-10=4=k$

Es k�nnen beliebige Argumente gew�hlt werden, die sich um 1 unterscheiden! Jedoch muss die Argumentation in jedem Fall korrekt wiedergegebenen werden!}
\end{beispiel}