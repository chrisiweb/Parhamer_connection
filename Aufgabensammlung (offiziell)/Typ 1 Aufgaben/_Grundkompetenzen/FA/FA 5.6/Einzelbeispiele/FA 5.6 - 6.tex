\section{FA 5.6 - 6 Zerfallsprozess - MC - BIFIE}

\begin{beispiel}[FA 5.6]{1} %PUNKTE DES BEISPIELS
Die Population $P$ einer vom Aussterben bedrohten Tierart sinkt jedes Jahr um ein Drittel der Population des vorangegangenen Jahres. $P_0$ gibt die Anzahl der urspr�nglich vorhandenen Tiere an.

\leer

Welche der nachstehend angef�hrten Gleichungen beschreibt die Population $P$ in Abh�ngigkeit von der Anzahl der abgelaufenen Jahre $t$? Kreuze die zutreffende Gleichung an.

\multiplechoice[6]{  %Anzahl der Antwortmoeglichkeiten, Standard: 5
				L1={$P(t)=P_0\cdot \left( \frac{1}{3}\right)^t$},   %1. Antwortmoeglichkeit 
				L2={$P(t)=P_0\cdot \left( \frac{2}{3}\right)^t$},   %2. Antwortmoeglichkeit
				L3={$P(t)=P_0\cdot \left(1-\frac{1}{3}\cdot t\right)$},   %3. Antwortmoeglichkeit
				L4={$P(t)=\dfrac{P_0}{3 \cdot t}$},   %4. Antwortmoeglichkeit
				L5={$P(t)=\dfrac{2 \cdot P_0}{3} \cdot t$},	 %5. Antwortmoeglichkeit
				L6={$P(t)=\left(P_0 - \frac{1}{3}\right)^t$},	 %6. Antwortmoeglichkeit
				L7={},	 %7. Antwortmoeglichkeit
				L8={},	 %8. Antwortmoeglichkeit
				L9={},	 %9. Antwortmoeglichkeit
				%% LOESUNG: %%
				A1=2,  % 1. Antwort
				A2=0,	 % 2. Antwort
				A3=0,  % 3. Antwort
				A4=0,  % 4. Antwort
				A5=0,  % 5. Antwort
				}
\end{beispiel}