\section{FA 6.2 - 2 Luftvolumen - OA - BIFIE}

\begin{beispiel}[FA 6.2]{1} %PUNKTE DES BEISPIELS
				Der Luftstrom beim Ein- und Ausatmen einer Person im Ruhezustand �ndert sich in Abh�ngigkeit von der Zeit nach einer Funktion $f$. Zum Zeitpunkt $t=0$ beginnt ein Atemzyklus.

$f(t)$ ist die bewegte Luftmenge in Litern pro Sekunde zum Zeitpunkt $t$ in Sekunden.

$F(t)$ beschreibt das zum Zeitpunkt $t$ in der Lunge vorhandene Luftvolumen, abgesehen vom Restvolumen.
\leer

\newrgbcolor{qqttcc}{0. 0.2 0.8}
\newrgbcolor{qqwuqq}{0. 0.39215686274509803 0.}
\psset{xunit=1.0cm,yunit=5.0cm,algebraic=true,dimen=middle,dotstyle=o,dotsize=5pt 0,linewidth=0.8pt,arrowsize=3pt 2,arrowinset=0.25}
\begin{pspicture*}(-1,-0.6658314108501197)(10.929713083858253,0.9271596640918052)
\multips(0,0)(0,0.2){12}{\psline[linestyle=dashed,linecap=1,dash=1.5pt 1.5pt,linewidth=0.4pt,linecolor=lightgray]{c-c}(0,-1)(11,-1)}
\multips(0,0)(1.0,0){12}{\psline[linestyle=dashed,linecap=1,dash=1.5pt 1.5pt,linewidth=0.4pt,linecolor=lightgray]{c-c}(0,-0.6)(0,0.9271596640918052)}
\psaxes[labelFontSize=\scriptstyle,xAxis=true,yAxis=true,Dx=1.,Dy=0.2,ticksize=-2pt 0,subticks=2]{->}(0,0)(0.,-0.65)(10.929713083858253,0.9271596640918052)[t,140] [f(t); F(t),-40]
\psplot[linewidth=1.2pt,linecolor=qqttcc,plotpoints=200]{0}{10.929713083858253}{SIN(1.256637*x)/2.0}
\psplot[linewidth=1.2pt,linestyle=dashed,dash=1pt 1pt,linecolor=qqwuqq,plotpoints=200]{0}{10.929713083858253}{(-COS(1.25664*x))/(2.0*1.25664)+0.4}
\rput[tl](3.127307818836539,0.7871164926683393){$F$}
\rput[tl](2.4896112346761106,-0.24320112566144572){$f$}
\end{pspicture*}

\begin{tiny}(Quelle: Timschl, W. (1995). Biomathematik: Eine Einf�hrung f�r Biologen und Mediziner. 2. Auflage. Wien u.A.: Springer.)\end{tiny}

\leer

Bestimme $F(2,5)$ und interpretiere den Wert.
\leer

\antwort{$F(2,5)=0,8$

Das insgesamt eingeatmete Luftvolumen betr�gt nach 2,5 Sekunden 0,8 Liter.}
\end{beispiel}