\subsection{FA 6.2 - 1 Trigonometrische Funktion skalieren - OA - BIFIE}

\begin{beispiel}[FA 6.2]{1} %PUNKTE DES BEISPIELS
				Gegeben ist der Graph der Funktion $f(x)=\sin(x+\frac{\pi}{2})$.

Erg�nze in der nachstehenden Zeichnung die Skalierung in den vorgegebenen f�nf K�stchen!
\leer


\begin{center}
\resizebox{0.9\linewidth}{!}{\winkelfunktion\psset{xunit=1.5cm,yunit=1.0cm,trigLabels,algebraic=true,dimen=middle,dotstyle=o,dotsize=5pt 0,linewidth=0.8pt,arrowsize=3pt 2,arrowinset=0.25}
\begin{pspicture*}(-1.6878444479288832,-1.9)(4.5,1.9)
\multips(0,-4)(0,1.0){10}{\psline[linestyle=dashed,linecap=1,dash=1.5pt 1.5pt,linewidth=0.4pt,linecolor=lightgray]{c-c}(-10,0)(10,0)}
\multips(-2,0)(1,0){12}{\psline[linestyle=dashed,linecap=1,dash=1.5pt 1.5pt,linewidth=0.4pt,linecolor=lightgray]{c-c}(0,-1)(0,1)}
\psaxes[labelFontSize=\scriptstyle,trigLabelBase=2,xAxis=true,yAxis=true,labels=none,Dx=1,Dy=1.,ticksize=-5pt 0,subticks=0]{->}(0,0)(-1.6878444479288832,-1.9)(4.5,1.9)[x,140] [f(x),-40]
\psplot[xunit=0.95493cm,linewidth=1.2pt,plotpoints=200]{-10}{6.28318}{sin(x+3.14159/2)}
\psline(-0.75,-1.75)(-1.25,-1.75)
\psline(-0.75,-1)(-1.25,-1)
\psline(-0.75,-1.75)(-0.75,-1)
\psline(-1.25,-1.75)(-1.25,-1)
\psline(-1,-1)(-1,-0.5)
\antwort{\rput[bl](-1.2,-1.6){$-\frac{\pi}{2}$}}

\psline(1.25,-1.75)(0.75,-1.75)
\psline(1.25,-1)(0.75,-1)
\psline(1.25,-1.75)(1.25,-1)
\psline(0.75,-1.75)(0.75,-1)
\psline(1,-1)(1,-0.5)
\antwort{\rput[bl](0.9,-1.6){$\frac{\pi}{2}$}}


\psline(2.25,1.75)(1.75,1.75)
\psline(2.25,1)(1.75,1)
\psline(2.25,1.75)(2.25,1)
\psline(1.75,1.75)(1.75,1)
\psline(2,1)(2,0.5)
\antwort{\rput[bl](1.9,1.3){$\pi$}}

\psline(3.25,-1.75)(2.75,-1.75)
\psline(3.25,-1)(2.75,-1)
\psline(3.25,-1.75)(3.25,-1)
\psline(2.75,-1.75)(2.75,-1)
\psline(3,-1)(3,-0.5)
\antwort{\rput[bl](2.85,-1.6){$\frac{3\pi}{2}$}}


\psline(4.25,-1.75)(3.75,-1.75)
\psline(4.25,-1)(3.75,-1)
\psline(4.25,-1.75)(4.25,-1)
\psline(3.75,-1.75)(3.75,-1)
\psline(4,-1)(4,-0.5)
\antwort{\rput[bl](3.85,-1.5){$2\pi$}}

\begin{scriptsize}
\rput[bl](-1.5882409839303135,-1.2899119647047494){$f$}
\end{scriptsize}
\end{pspicture*}}
\end{center}

\antwort{Alle f�nf Werte m�ssen korrekt angegeben sein. Auch die Angabe als Dezimalzahl ist richtig zu werten -- vorausgesetzt, es ist mindestens eine Nachkommastelle angegeben.}
\end{beispiel}