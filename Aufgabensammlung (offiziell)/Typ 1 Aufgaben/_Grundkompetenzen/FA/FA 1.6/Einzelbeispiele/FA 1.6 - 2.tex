\section{FA 1.6 - 2 Kosten- und Erl�sfunktion - OA - BIFIE}

\begin{beispiel}[FA 1.6]{1} %PUNKTE DES BEISPIELS
Die Herstellungskosten eines Produkts k�nnen ann�hernd durch eine lineare Funktion $K$ mit $K(x) = 392 + 30x$ beschrieben werden. \leer

Beim Verkauf dieses Produkts wird ein Erl�s erzielt, der ann�hernd durch die quadratische Funktion $E$ mit $E(x) = -2x^2 + 100x$ angegeben werden kann.\leer

$x$ gibt die Anzahl der produzierten und verkauften Einheiten des Produkts an. \leer


Ermittle die x-Koordinaten der Schnittpunkte dieser Funktionsgraphen und interpretiere diese im gegebenen Zusammenhang.


\antwort{$x_1=7$, $x_2=28$ \leer

Bei der Herstellung und dem Verkauf von 7 (bzw. 28) St�ck des Produkts sind die Herstellungskosten genauso hoch wie der Erl�s. Das hei�t, in diesen F�llen wird kein Gewinn/Verlust erzielt.}
\end{beispiel}