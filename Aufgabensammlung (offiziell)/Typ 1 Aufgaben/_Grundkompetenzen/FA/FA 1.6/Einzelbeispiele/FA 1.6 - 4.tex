\section{FA 1.6 - 4 - MAT - Schnittpunkte - OA - Matura 2016/17 2. NT}

\begin{beispiel}{1} %PUNKTE DES BEISPIELS
In der nachstehenden Abbildung sind der Graph der Funktion $f$ mit $f(x)=x^2-4\cdot x - 2$ und der Graph der Funktion $g$ mit $g(x)=x-6$ darstellt sowie deren Schnittpunkte $A$ und $B$ gekennzeichnet.

\begin{center}
\resizebox{0.5\linewidth}{!}{
\newrgbcolor{qqwuqq}{0. 0.39215686274509803 0.}
\newrgbcolor{ccqqqq}{0.8 0. 0.}
\psset{xunit=1.0cm,yunit=1.0cm,algebraic=true,dimen=middle,dotstyle=o,dotsize=5pt 0,linewidth=1.6pt,arrowsize=3pt 2,arrowinset=0.25}
\begin{pspicture*}(-2.1332139426114285,-6.707995882777694)(7.596118995592024,5.659219577403467)
\psaxes[labelFontSize=\scriptstyle,xAxis=true,yAxis=true,labels=none,Dx=2.,Dy=2.,ticksize=-2pt 0,subticks=2]{->}(0,0)(-2.1332139426114285,-6.707995882777694)(7.596118995592024,5.659219577403467)[x,140] [\text{f(x), g(x)},-40]
\psplot[linewidth=2.pt,linecolor=qqwuqq,plotpoints=200]{-2.1332139426114285}{7.596118995592024}{x^(2.0)-4.0*x-2.0}
\psplot[linewidth=2.pt,linecolor=ccqqqq,plotpoints=200]{-2.1332139426114285}{7.596118995592024}{x-6.0}
\begin{scriptsize}
\rput[bl](-1.0712092909061222,5.145346358836383){\qqwuqq{$f$}}
\rput[bl](-0.8999182180504277,-6.125606235068332){\ccqqqq{$g$}}
\psdots[dotsize=4pt 0,dotstyle=*](1.,-5.)
\rput[bl](0.26486107736829545,-5.029343368791886){\darkgray{$A$}}
\psdots[dotsize=4pt 0,dotstyle=*](4.,-2.)
\rput[bl](4.341588611333827,-2.3229444176719087){\darkgray{$B$}}
\end{scriptsize}
\end{pspicture*}}
\end{center}

Bestimme die Koeffizienten $a$ und $b$ der quadratischen Gleichung $x^2+a\cdot x + b = 0$ so, dass die beiden L�sungen dieser Gleichungen die $x$-Koordinaten der Schnittpunkte $A$ und $B$ sind.

\antwort{$x^2-4\cdot x -2 = x-6$

$x^2-5\cdot x + 4 =0 \Rightarrow a=-5,~b=4$}
\end{beispiel}