\section{FA 3.2 - 2 Punkte einer Wurzelfunktion  - MC - BIFIE}


\begin{beispiel}[FA 3.2]{1} %PUNKTE DES BEISPIELS
Eine Wurzelfunktion kann durch die Funktionsgleichung $f(x)=a\cdot \sqrt{x}+b$ mit $a,b \in \mathbb{R}$ festgelegt werden.
\leer

Welche der nachstehenden Punkte liegen jedenfalls (bei beliebiger Wahl von $a$ und $b$) auf dem Graphen der Funktion $f$? \\
Kreuze die beiden entsprechenden Punkte an.

\multiplechoice[5]{  %Anzahl der Antwortmoeglichkeiten, Standard: 5
				L1={$P_1=(-1|a)$},   %1. Antwortmoeglichkeit 
				L2={$P_2=(0|b)$},   %2. Antwortmoeglichkeit
				L3={$P_3=(a|b)$},   %3. Antwortmoeglichkeit
				L4={$P_4=(b|a\cdot b)$},   %4. Antwortmoeglichkeit
				L5={$P_5=(1|a+b)$},	 %5. Antwortmoeglichkeit
				L6={},	 %6. Antwortmoeglichkeit
				L7={},	 %7. Antwortmoeglichkeit
				L8={},	 %8. Antwortmoeglichkeit
				L9={},	 %9. Antwortmoeglichkeit
				%% LOESUNG: %%
				A1=2,  % 1. Antwort
				A2=5,	 % 2. Antwort
				A3=0,  % 3. Antwort
				A4=0,  % 4. Antwort
				A5=0,  % 5. Antwort
				}
\end{beispiel}