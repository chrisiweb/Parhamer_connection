\section{FA 6.3 - 2 Trigonometrische Funktion - OA - BIFIE}

\begin{beispiel}[FA 6.3]{1} %PUNKTE DES BEISPIELS
				Gegeben ist der Graph der Funktion $f(x)=\sin(x)$.
\leer

\psset{xunit=1.0cm,yunit=1.0cm,algebraic=true,dimen=middle,dotstyle=o,dotsize=5pt 0,linewidth=0.8pt,arrowsize=3pt 2,arrowinset=0.25}
\begin{pspicture*}(-7.643890198234905,-2.7107030224215674)(7.560199111086908,3.025180699201213)
\multips(0,-2)(0,1.0){6}{\psline[linestyle=dashed,linecap=1,dash=1.5pt 1.5pt,linewidth=0.4pt,linecolor=lightgray]{c-c}(-7.643890198234905,0)(7.560199111086908,0)}
\multips(-7,0)(1.5707963267948966,0){10}{\psline[linestyle=dashed,linecap=1,dash=1.5pt 1.5pt,linewidth=0.4pt,linecolor=lightgray]{c-c}(0,-2.7107030224215674)(0,3.025180699201213)}
\psaxes[labelFontSize=\scriptstyle,xAxis=true,yAxis=true,labels=y,Dx=3.141592653589793,Dy=2.,ticksize=-2pt 0,subticks=2]{->}(0,0)(-7.643890198234905,-2.7107030224215674)(7.560199111086908,3.025180699201213)[x,140] [f(x),-30]
\psplot[linewidth=1.2pt,plotpoints=200]{-7.643890198234905}{7.560199111086908}{SIN(x)}
\rput[tl](2.942660876403987,-0.09293520922364816){$\pi$}
\rput[tl](6.321347389586612,-0.13151738672335295){$2\pi$}
\rput[tl](-3.1389748473247376,-0.15723883838982283){$-\pi$}
\rput[tl](-6.217333670446685,-0.11865666089011802){$-2\pi$}
\antwort{\psplot[linewidth=1.2pt,linestyle=dashed,dash=2pt 2pt,linecolor=red,plotpoints=200]{-7.643890198234905}{7.560199111086908}{2.0*SIN(x)}}
\begin{scriptsize}
\rput[bl](-7.13708722125751,-0.28002583994016494){f(x)}
\end{scriptsize}
\end{pspicture*}
\leer

Zeichne in die gegebenen Abbildung den Graphen der Funktion $g(x)=2\cdot\sin(x)$ ein.
\end{beispiel}