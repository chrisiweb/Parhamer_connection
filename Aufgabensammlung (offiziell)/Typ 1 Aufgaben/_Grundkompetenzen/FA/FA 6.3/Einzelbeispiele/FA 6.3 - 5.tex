\section{FA 6.3 - 5 Parameter Sinus - OA - BIFIE - Kompetenzcheck 2016}

\begin{beispiel}[FA 6.3]{1} %PUNKTE DES BEISPIELS
				Die nachstehende Abbildung zeigt den Graphen der Funktion $s$ mit der Gleichung $s(x)=c\cdot sin(d\cdot x)$ mit $c,d\in\mathbb{R}^{+}$ im Intervall $\left[-2\pi;2\pi\right]$.\\

\resizebox{1\linewidth}{!}{\winkelfunktion\psset{xunit=1.0cm,yunit=1.0cm,trigLabels,algebraic=true,dimen=middle,dotstyle=o,dotsize=5pt 0,linewidth=0.8pt,arrowsize=3pt 2,arrowinset=0.25}
\begin{pspicture*}(-4.6878444479288832,-3.0827743166790045)(4.676070775940966,4.028913012818873)
\multips(0,-4)(0,1.0){10}{\psline[linestyle=dashed,linecap=1,dash=1.5pt 1.5pt,linewidth=0.4pt,linecolor=lightgray]{c-c}(-10,0)(10,0)}
\multips(-4,0)(1,0){12}{\psline[linestyle=dashed,linecap=1,dash=1.5pt 1.5pt,linewidth=0.4pt,linecolor=lightgray]{c-c}(0,-5)(0,5)}
\psaxes[labelFontSize=\scriptstyle,trigLabelBase=2,xAxis=true,yAxis=true,Dx=1,Dy=1.,ticksize=-2pt 0,subticks=2]{->}(0,0)(-4.6878444479288832,-3.0827743166790045)(4.676070775940966,4.028913012818873)[,140] [\tiny{$s(x)\text{, }s_{1}(x)$},-40]
\psplot[xunit=0.63661977cm,linewidth=1.2pt,plotpoints=200]{-10}{20}{SIN(0.5*x)}
\antwort{\psplot[xunit=0.63661977cm,linewidth=1.2pt,plotpoints=200]{-10}{20}{2*SIN(x)}}
\begin{scriptsize}
\rput[bl](-1.5882409839303135,-1.2899119647047494){$s$}
\antwort{\rput[bl](-1.5882409839303135,-1.2899119647047494){$s_{1}$}}
\end{scriptsize}
\end{pspicture*}}\\

Erstelle im obigen Koordinatensystem eine Skizze eines m�glichen Funktionsgraphen der Funktion $s_{1}$ mit $s_{1}(x)=2c\cdot sin(2d\cdot x)$ im Intervall $\left[-2\pi;2\pi\right]$.
\end{beispiel}	