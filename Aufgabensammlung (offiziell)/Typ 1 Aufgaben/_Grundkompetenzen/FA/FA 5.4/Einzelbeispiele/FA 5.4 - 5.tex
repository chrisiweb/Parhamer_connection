\section{FA 5.4 - 5 Exponentialfunktion - MC - Matura 2013/14 Haupttermin}

\begin{beispiel}{1} %PUNKTE DES BEISPIELS
			Eine relle Funktion $f$ mit der Gleichung $f(x)=c\cdot a^x$ ist eine Exponentialfunktion, f�r deren reelle Parameter $c$ und $a$ gilt: $c\neq 0,a>1$.
			
			Kreuze jene beiden Aussagen an, die auf diese Exponentialfunktion $f$ und alle Werte $k,h\in\mathbb{R},k>1$ zutreffen!\leer
			
			\multiplechoice[5]{  %Anzahl der Antwortmoeglichkeiten, Standard: 5
							L1={$f(k\cdot x)=k\cdot f(x)$},   %1. Antwortmoeglichkeit 
							L2={$\frac{f(x+h)}{f(x)}=a^h$},   %2. Antwortmoeglichkeit
							L3={$f(x+1)=a\cdot f(x)$},   %3. Antwortmoeglichkeit
							L4={$f(0)=0$},   %4. Antwortmoeglichkeit
							L5={$f(x+h)=f(x)+f(h)$},	 %5. Antwortmoeglichkeit
							L6={},	 %6. Antwortmoeglichkeit
							L7={},	 %7. Antwortmoeglichkeit
							L8={},	 %8. Antwortmoeglichkeit
							L9={},	 %9. Antwortmoeglichkeit
							%% LOESUNG: %%
							A1=2,  % 1. Antwort
							A2=3,	 % 2. Antwort
							A3=0,  % 3. Antwort
							A4=0,  % 4. Antwort
							A5=0,  % 5. Antwort
							}
\end{beispiel}