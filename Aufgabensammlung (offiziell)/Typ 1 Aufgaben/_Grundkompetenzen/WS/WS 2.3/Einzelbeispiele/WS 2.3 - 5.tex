\section{WS 2.3 - 5 Laplace-Experiment - MC - BIFIE}

\begin{beispiel}[WS 2.3]{1}
In einer Schachtel befinden sich rote, blaue und gelbe Wachsmalstifte. Ein Stift wird zuf�llig entnommen, dessen Farbe notiert und der Stift danach zur�ckgelegt. Dann wird das Experiment wiederholt.

Beobachtet wird, wie oft bei zweimaligem Ziehen ein gelber Stift entnommen wurde. Die Werte der Zufallsvariablen $X$ beschreiben die Anzahl $x$ der gezogenen gelben Stifte.

Die nachstehende Tabelle stellt die Wahrscheinlichkeitsverteilung der Zufallsvariablen $X$ dar.

\renewcommand{\arraystretch}{1.4}
\begin{center}
\begin{tabular}{|c|c|} \hline
$x$ & $P(X=x)$ \\ \hline
$0$ & $\frac{4}{9}$ \\ \hline
$1$ & $\frac{4}{9}$ \\ \hline
$2$ & $\frac{1}{9}$ \\ \hline
\end{tabular}
\end{center}

\leer

Kreuze die beiden zutreffenden Aussagen an.

\multiplechoice[5]{  %Anzahl der Antwortmoeglichkeiten, Standard: 5
				L1={Die Wahrscheinlichkeit, mindestens einen gelben Stift zu ziehen, ist $\frac{4}{9}$.},   %1. Antwortmoeglichkeit 
				L2={Die Wahrscheinlichkeit, h�chstens einen gelben Stift zu ziehen, ist $\frac{4}{9}$.},   %2. Antwortmoeglichkeit
				L3={Die Wahrscheinlichkeit, nur rote oder blaue Stifte zu ziehen, ist $\frac{4}{9}$.},   %3. Antwortmoeglichkeit
				L4={Die Wahrscheinlichkeit, keinen oder einen gelben Stift zu ziehen, ist $\frac{4}{9}$.},   %4. Antwortmoeglichkeit
				L5={Die Wahrscheinlichkeit, dass mehr als ein gelber Stift gezogen wird, ist gr��er als 10\,\%.
},	 %5. Antwortmoeglichkeit
				L6={},	 %6. Antwortmoeglichkeit
				L7={},	 %7. Antwortmoeglichkeit
				L8={},	 %8. Antwortmoeglichkeit
				L9={},	 %9. Antwortmoeglichkeit
				%% LOESUNG: %%
				A1=3,  % 1. Antwort
				A2=5,	 % 2. Antwort
				A3=0,  % 3. Antwort
				A4=0,  % 4. Antwort
				A5=0,  % 5. Antwort
				}
\end{beispiel}