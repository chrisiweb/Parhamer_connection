\section{WS 2.3 - 15 Haus�bungskontrolle - OA- Matura 2013/14 Haupttermin}

\begin{beispiel}[WS 2.3]{1} %PUNKTE DES BEISPIELS
				Eine Lehrerin w�hlt am Beginn der Mathematikstunde nach dem Zufallsprinzip 3 Sch�ler/innen aus, die an der Tafel die L�sungsans�tze der Haus�bungsaufgaben erkl�ren m�ssen. Es sind 12�Burschen und 8 M�dchen anwesend. 
				
Berechne die Wahrscheinlichkeit, dass f�r das Erkl�ren der L�sungsans�tze 2 Burschen und 1 M�dchen ausgew�hlt werden!\leer

\antwort{$P(\text{"`2 Burschen, 1 M�dchen"'})=\frac{12}{20}\cdot \frac{11}{19}\cdot \frac{8}{18}\cdot 3=\frac{44}{95}\approx 0,46=46\,\%$

Toleranzintervall: $[0,46; 0,47]$ bzw. $[46\,\%;47\,\%]$. Sollte als L�sungsmethode die hypergeometrische Verteilung gew�hlt werden ist dies auch als richtig zu werten:

$P(E)=\dfrac{\binom{12}{2}\cdot \binom{8}{1}}{\binom{20}{3}}$ }

\end{beispiel}