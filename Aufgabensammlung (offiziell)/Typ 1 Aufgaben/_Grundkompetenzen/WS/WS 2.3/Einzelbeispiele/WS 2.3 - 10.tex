\section{WS 2.3 - 10 Maturaball - OA - BIFIE - Kompetenzcheck 2016}

\begin{beispiel}[WS 2.3]{1} %PUNKTE DES BEISPIELS
				Die nachstehende Abbildung zeigt die Wahrscheinlichkeitsverteilung einer Zufallsvariablen X, die die Werte $k=1,2,3,4,5$ annehmen kann.\\
				
				\resizebox{0.8\linewidth}{!}{\newrgbcolor{cqcqcq}{0.7529411764705882 0.7529411764705882 0.7529411764705882}
\newrgbcolor{uuuuuu}{0.26666666666666666 0.26666666666666666 0.26666666666666666}
\psset{xunit=1.0cm,yunit=10cm,algebraic=true,dimen=middle,dotstyle=o,dotsize=5pt 0,linewidth=0.8pt,arrowsize=3pt 2,arrowinset=0.25}
\begin{pspicture*}(-1.320617181087312,-0.09858644140495518)(5.687647281722601,0.5339054534246019)
\multips(0,0)(0,0.05){13}{\psline[linestyle=dashed,linecap=1,dash=1.5pt 1.5pt,linewidth=0.4pt,linecolor=lightgray]{c-c}(0,0)(5.687647281722601,0)}
\multips(0,0)(20.0,0){1}{\psline[linestyle=dashed,linecap=1,dash=1.5pt 1.5pt,linewidth=0.4pt,linecolor=lightgray]{c-c}(0,0)(0,0.5339054534246019)}
\psaxes[labelFontSize=\scriptstyle,xAxis=true,yAxis=true,labels=y,Dx=20.,Dy=0.05,ticksize=-2pt 0,subticks=2]{}(0,0)(0.,0.)(5.687647281722601,0.5339054534246019)
\pspolygon[linecolor=darkgray,fillcolor=darkgray,fillstyle=solid,opacity=0.1](0.25,0.)(0.25,0.1)(0.75,0.1)(0.75,0.)
\pspolygon[linecolor=darkgray,fillcolor=darkgray,fillstyle=solid,opacity=0.1](1.25,0.)(1.25,0.3)(1.75,0.3)(1.75,0.)
\pspolygon[linecolor=darkgray,fillcolor=darkgray,fillstyle=solid,opacity=0.1](2.25,0.)(2.25,0.4)(2.75,0.4)(2.75,0.)
\pspolygon[linecolor=darkgray,fillcolor=darkgray,fillstyle=solid,opacity=0.1](3.25,0.)(3.25,0.1)(3.75,0.1)(3.75,0.)
\pspolygon[linecolor=darkgray,fillcolor=darkgray,fillstyle=solid,opacity=0.1](4.25,0.)(4.25,0.1)(4.75,0.1)(4.75,0.)
\psline[linecolor=darkgray](0.25,0.)(0.25,0.1)
\psline[linecolor=darkgray](0.25,0.1)(0.75,0.1)
\psline[linecolor=darkgray](0.75,0.1)(0.75,0.)
\psline[linecolor=darkgray](0.75,0.)(0.25,0.)
\psline[linecolor=darkgray](1.25,0.)(1.25,0.3)
\psline[linecolor=darkgray](1.25,0.3)(1.75,0.3)
\psline[linecolor=darkgray](1.75,0.3)(1.75,0.)
\psline[linecolor=darkgray](1.75,0.)(1.25,0.)
\psline[linecolor=darkgray](2.25,0.)(2.25,0.4)
\psline[linecolor=darkgray](2.25,0.4)(2.75,0.4)
\psline[linecolor=darkgray](2.75,0.4)(2.75,0.)
\psline[linecolor=darkgray](2.75,0.)(2.25,0.)
\psline[linecolor=darkgray](3.25,0.)(3.25,0.1)
\psline[linecolor=darkgray](3.25,0.1)(3.75,0.1)
\psline[linecolor=darkgray](3.75,0.1)(3.75,0.)
\psline[linecolor=darkgray](3.75,0.)(3.25,0.)
\psline[linecolor=darkgray](4.25,0.)(4.25,0.1)
\psline[linecolor=darkgray](4.25,0.1)(4.75,0.1)
\psline[linecolor=darkgray](4.75,0.1)(4.75,0.)
\psline[linecolor=darkgray](4.75,0.)(4.25,0.)
\rput[tl](0.4274759955772923,-0.017){\scriptsize{1}}
\rput[tl](1.396873120818573,-0.017){\scriptsize{2}}
\rput[tl](2.4298372706658387,-0.017){\scriptsize{3}}
\rput[tl](3.4469096643616086,-0.017){\scriptsize{4}}
\rput[tl](4.448090301905882,-0.017){\scriptsize{5}}
\rput[tl](2.5728630760293063,-0.067){\scriptsize{k}}
\rput[tl](-1.229916107269355,.327){\scriptsize{$\rotatebox{90}{P(X=k)}$}}
\end{pspicture*}}\\

Ermittle den Erwartungswert $E(X)$.\\

\antwort{$E(X)=2,8$ - Toleranzintervall: $\left[2,65;2,95\right]$}
\end{beispiel}