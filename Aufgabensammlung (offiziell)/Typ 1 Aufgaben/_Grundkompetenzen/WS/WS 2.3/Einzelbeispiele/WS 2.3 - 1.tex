\section{WS 2.3 - 1 Kugelschreiber - ZO - BIFIE}

\begin{beispiel}[WS 2.3]{1}
Ein Kugelschreiber besteht aus zwei Bauteilen, der Mine (M) und dem Geh�use mit dem Mechanismus (G). Bei der Qualit�tskontrolle werden die Kugelschreiber einzeln entnommen und auf ihre Funktionst�chtigkeit hin getestet. Ein Kugelschreiber gilt als defekt, wenn mindestens ein Bauteil fehlerhaft ist.\leer

Im nachstehenden Baumdiagramm sind alle m�glichen F�lle f�r defekte und nicht defekte Kugelschreiber
aufgelistet.


\begin{center}
\resizebox{1.1\linewidth}{!}{\Huge
\psset{xunit=1.0cm,yunit=1.0cm,algebraic=true,dimen=middle,dotstyle=o,dotsize=5pt 0,linewidth=0.8pt,arrowsize=3pt 2,arrowinset=0.25}
\begin{pspicture*}(-16.570738647589398,5.491882395363719)(10.343075289784963,16.418622163615545)
\rput[tl](-4.1,15){\fbox{Start}}
\rput[tl](-11.6,11.5){\fbox{M defekt}}
\rput[tl](2.2,11.5){\fbox{M ist o.k.}}
\rput[tl](-15.5,7.6){\fbox{G defekt}}
\rput[tl](-7.8,7.6){\fbox{G ist o.k.}}
\rput[tl](-1.3,7.6){\fbox{G defekt}}
\rput[tl](5.909684973977872,7.6){\fbox{G ist o.k.}}
\psline{->}(-4.524455870295381,14.04519098444609)(-10.044781720361687,12.)
\psline{->}(-1.6136440467856739,14.04519098444609)(4.,12.)
\psline{->}(-12.047784891058932,10.328308194426004)(-14.,8.)
\psline{->}(-7.927866617783654,10.373089914787691)(-6.,8.)
\psline{->}(1.8793301414259747,10.283526474064315)(0.,8.)
\psline{->}(6.088811855424628,10.283526474064315)(8.,8.)
\rput[tl](-8.19655693995378,14.089972704807778){0,05}
\rput[tl](0.6702236916604013,14.04519098444609){0,95}
\rput[tl](-14.510779510951759,9.9700544315325){0,08}
\rput[tl](-6.7187601680180835,9.835709270447436){0,92}
\rput[tl](-0.44931931738179337,9.9700544315325){0,08}
\rput[tl](7.253136584828506,9.79092755008575){0,92}
\end{pspicture*}}
\normalsize
\end{center}

Ordnen den Ereignissen $E_1$, $E_2$, $E_3$ bzw. $E_4$ die entsprechende Wahrscheinlichkeit $p_1$, $p_2$, $p_3$,
$p_4$, $p_5$ oder $p_6$ zu.

\leer

\zuordnen[0.014]{
				R1={$E_1$: Eine Mine ist defekt und das Geh�use ist in Ordnung.},				% Response 1
				R2={$E_2$: Ein Kugelschreiber ist defekt.},				% Response 2
				R3={$E_3$: H�chstens ein Teil ist defekt.},				% Response 3
				R4={$E_4$: Ein Kugelschreiber ist nicht defekt.},				% Response 4
				%% Moegliche Zuordnungen: %%
				A={$p_1=0,95\cdot 0,92$}, 				%Moeglichkeit A  
				B={\small $p_2=0,05 \cdot 0,08 + 0,95 \cdot 0,08$}, 				%Moeglichkeit B  
				C={$p_3=0,05 +0,92$}, 				%Moeglichkeit C  
				D={$p_4=0,05+0,95\cdot 0,08$}, 				%Moeglichkeit D  
				E={$p_5=0,05\cdot 0,92$}, 				%Moeglichkeit E  
				F={$p_6=1-0,05\cdot 0,08$}, 				%Moeglichkeit F  
				%% LOESUNG: %%
				A1={E},				% 1. richtige Zuordnung
				A2={D},				% 2. richtige Zuordnung
				A3={F},				% 3. richtige Zuordnung
				A4={A},				% 4. richtige Zuordnung
				}

\end{beispiel}