\section{WS 2.3 - 11 Mehrere Wahrscheinlichkeiten - MC - Matura 2014/15 - Haupttermin}

\begin{beispiel}[WS 2.3]{1} %PUNKTE DES BEISPIELS
In einer Unterrichtsstunde sind 15 Sch�lerinnen und 10 Sch�ler anwesend. Die Lehrperson w�hlt
f�r �berpr�fungen nacheinander zuf�llig drei verschiedene Personen aus dieser Schulklasse aus.
Jeder Pr�fling wird nur einmal befragt. \leer

Kreuze die beiden zutreffenden Aussagen an.

\multiplechoice[5]{  %Anzahl der Antwortmoeglichkeiten, Standard: 5
				L1={Die Wahrscheinlichkeit, dass die Lehrperson drei Sch�lerinnen
ausw�hlt, kann mittels $\frac{15}{25}\cdot \frac{14}{25} \cdot \frac{13}{25}$ berechnet werden.},   %1. Antwortmoeglichkeit 
				L2={Die Wahrscheinlichkeit, dass die Lehrperson als erste Person
einen Sch�ler ausw�hlt, ist $\frac{10}{25}$.},   %2. Antwortmoeglichkeit
				L3={Die Wahrscheinlichkeit, dass die Lehrperson bei der Wahl von drei
Pr�flingen als zweite Person eine Sch�lerin ausw�hlt, ist $\frac{24}{25}$.},   %3. Antwortmoeglichkeit
				L4={Die Wahrscheinlichkeit, dass die Lehrperson drei Sch�ler ausw�hlt,
kann mittels $\frac{10}{25}\cdot \frac{9}{24} \cdot \frac{8}{23}$ berechnet werden.},   %4. Antwortmoeglichkeit
				L5={Die Wahrscheinlichkeit, dass sich unter den von der Lehrperson
ausgew�hlten Personen genau zwei Sch�lerinnen befinden, kann
mittels $\frac{15}{25}\cdot \frac{14}{24} \cdot \frac{23}{23}$ berechnet werden.},	 %5. Antwortmoeglichkeit
				L6={},	 %6. Antwortmoeglichkeit
				L7={},	 %7. Antwortmoeglichkeit
				L8={},	 %8. Antwortmoeglichkeit
				L9={},	 %9. Antwortmoeglichkeit
				%% LOESUNG: %%
				A1=2,  % 1. Antwort
				A2=4,	 % 2. Antwort
				A3=0,  % 3. Antwort
				A4=0,  % 4. Antwort
				A5=0,  % 5. Antwort
				}
\end{beispiel}