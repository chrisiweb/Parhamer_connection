\section{WS 3.2 - 1 Binomialverteilung - MC - BIFIE}

\begin{beispiel}[WS 3.2]{1} %PUNKTE DES BEISPIELS
Die Zufallsvariable $X$ sei binomialverteilt mit $n = 25$ und $p = 0,15$.
Es soll die Wahrscheinlichkeit bestimmt werden, sodass die Zufallsvariable $X$ h�chstens den
Wert 2 annimmt.\leer

Kreuze den zutreffenden Term an.

\multiplechoice[6]{  %Anzahl der Antwortmoeglichkeiten, Standard: 5
				L1={$\begin{pmatrix}25\\2\\\end{pmatrix}\cdot 0,15^2 \cdot 0,85^{23}$},   %1. Antwortmoeglichkeit 
				L2={$0,85^{25}+\begin{pmatrix}25\\1\\\end{pmatrix}\cdot 0,15^1 \cdot 0,85^{24}+\begin{pmatrix}25\\2\\\end{pmatrix}\cdot 0,15^2\cdot 0,85^{23}$},   %2. Antwortmoeglichkeit
				L3={$\begin{pmatrix}25\\1\\\end{pmatrix}\cdot 0,15^1\cdot 0,85^{24}+\begin{pmatrix}25\\2\\\end{pmatrix}\cdot0,15^2\cdot0,85^{23}$},   %3. Antwortmoeglichkeit
				L4={$1-\begin{pmatrix}25\\2\\\end{pmatrix}\cdot0,15^2\cdot0,85^{23}$},   %4. Antwortmoeglichkeit
				L5={$1-\left[ 0,85^{25} + \begin{pmatrix}25\\1\\\end{pmatrix} \cdot 0,15^1 \cdot 0,85^{24} + \begin{pmatrix}25\\2\\\end{pmatrix}\cdot 0,15^2 \cdot 0,85^{23}\right]$},	 %5. Antwortmoeglichkeit
				L6={$\begin{pmatrix}25\\2\\\end{pmatrix}\cdot 0,85^2\cdot 0,15^{23}$},	 %6. Antwortmoeglichkeit
				L7={},	 %7. Antwortmoeglichkeit
				L8={},	 %8. Antwortmoeglichkeit
				L9={},	 %9. Antwortmoeglichkeit
				%% LOESUNG: %%
				A1=2,  % 1. Antwort
				A2=0,	 % 2. Antwort
				A3=0,  % 3. Antwort
				A4=0,  % 4. Antwort
				A5=0,  % 5. Antwort
				}
\end{beispiel}