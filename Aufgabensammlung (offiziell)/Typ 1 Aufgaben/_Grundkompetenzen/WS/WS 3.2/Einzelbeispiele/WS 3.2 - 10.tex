\section{WS 3.2 - 10 Parameter einer Binomialverteilung - OA - Matura 2015/16 - Nebentermin 1}

\begin{beispiel}[WS 3.2]{1} %PUNKTE DES BEISPIELS
Ein Zufallsexperiment wird durch eine binomialverteilte Zufallsvariable $X$ beschrieben. Diese hat
die Erfolgswahrscheinlichkeit $p = 0,36$ und die Standardabweichung $\sigma = 7,2$.\leer

Berechneden zugeh�rigen Parameter $n$ (Anzahl der Versuche).\leer

n=\rule{8cm}{0.3pt}

\antwort{$n\cdot 0,36\cdot (1-0,36)=7,2^2$ 

$n=225$}

\end{beispiel}