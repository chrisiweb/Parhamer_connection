\section{WS 4.1 - 1 Wahl - OA - BIFIE}

\begin{beispiel}[WS 4.1]{1}
Bei einer Befragung von 2,000 zuf�llig ausgew�hlten wahlberechtigten Personen geben 14\,\%
an, dass sie bei der n�chsten Wahl f�r die Partei "`Alternatives Leben"' stimmen werden. Aufgrund dieses Ergebnisses gibt ein Meinungsforschungsinstitut an, dass die Partei mit 12\,\% bis 16\,\% der Stimmen rechnen kann.  

\leer

Mit welcher Sicherheit kann man diese Behauptung aufstellen? \\

\antwort{Konfidenzintervall: $[0,12 ; 0,16]$

$\mu=n\cdot p = 2\,000 \cdot 0,14=280$ 

$\sigma=\sqrt{n\cdot p \cdot (1-p)}=15,5$

$0,16 \cdot 2\,000 =320$

$320=280 + z\cdot 15,5 \rightarrow z=2,58 \rightarrow\Theta(z)=0,995$
 
$2 \cdot \Theta(z)-1=0,99$

Die Behauptung kann mit 99\,\%iger Sicherheit aufgestellt werden.}
\end{beispiel}