\section{WS 4.1 - 5 Essgewohnheiten - OA - BIFIE}

\begin{beispiel}[WS 4.1]{1} %PUNKTE DES BEISPIELS
				Um die Essgewohnheiten von Jugendlichen zu untersuchen, wurden 400 Jugendliche eines Bezirks zuf�llig ausgew�hlt und befragt.

Dabei gaben 240 der befragten Jugendlichen an, t�glich zu fr�hst�cken.

Berechne aufgrund des in der Umfrage erhobenen Stichprobenergebnisses ein 99-\%-Konfidenzintervall f�r den tats�chlichen (relativen) Anteil $p$ derjenigen Jugendlichen dieses Bezirks, die t�glich fr�hst�cken.\\

\antwort{Die Zufallsvariable $X$ gibt die Anzahl der Jugendlichen, die t�glich fr�hst�cken, an.\\
$h=\frac{240}{400}=0,6$\\
$2\cdot \Theta(z)-1=D(z)=0,99\Rightarrow z\approx2,58$\\
$p_{1,2}=0,6\pm 2,58\cdot \sqrt{\dfrac{0,6\cdot 0,4}{400}}\Rightarrow p_{1}\approx 0,536; p_{2}\approx0,664$\\
99-\%-Konfidenzintervall: $\left[0,536;0,664\right]$ bzw. $0,6\pm 0,064$\\

Ein Punkt ist genau dann zu geben, wenn das Konfidenzintervall richtig berechnet wurde.\\
Toleranzintervall f�r die untere Grenze: $\left[0,53;0,54\right]$\\
Toleranzintervall f�r die obere Grenze: $\left[0,66;0,67\right]$}	
\end{beispiel}	