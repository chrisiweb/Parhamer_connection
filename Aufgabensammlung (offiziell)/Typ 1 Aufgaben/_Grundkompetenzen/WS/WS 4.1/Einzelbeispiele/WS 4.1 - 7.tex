\section{WS 4.1 - 7 Meinungsbefragung - MC - BIFIE - Kompetenzcheck 2016}

\begin{beispiel}[WS 4.1]{1} %PUNKTE DES BEISPIELS
				Bei einer Meinungsbefragung wurden 500 zuf�llig ausgew�hlte BewohnerInnen einer Stadt zu ihrer Meinung bez�glich der Einrichtung einer Fu�g�ngerzone im Stadtzentrum befragt. Es sprachen sich 60\,\% der Befragten f�r die Einrichtung einer solchen Fu�g�ngerzone aus, 40\,\% sprachen sich dagegen aus.

Als 95-\%-Konfidenzintervall f�r den Anteil der BewohnerInnen dieser Stadt, die die Einrichtung einer Fu�g�ngerzone im Stadtzentrum bef�rworten, erh�lt man mit Normalapproximation das Intervall $\left[55,7\,\%;64,3\,\%\right]$.\\

Kreuze die beiden zutreffenden Aussagen an.

\multiplechoice[5]{  %Anzahl der Antwortmoeglichkeiten, Standard: 5
				L1={Das Konfidenzintervall w�re breiter, wenn man einen gr��eren Stichprobenumfang gew�hlt h�tte und der relative Anteil der Bef�rworterInnen gleich gro� geblieben w�re.},   %1. Antwortmoeglichkeit 
				L2={Das Konfidenzintervall w�re breiter, wenn man ein h�heres Konfidenzniveau (eine h�here Sicherheit) gew�hlt h�tte.},   %2. Antwortmoeglichkeit
				L3={Das Konfidenzintervall w�re breiter, wenn man die Befragung in einer gr��eren Stadt durchgef�hrt h�tte.},   %3. Antwortmoeglichkeit
				L4={Das Konfidenzintervall w�re breiter, wenn der Anteil der Bef�rworterInnen in der Stichprobe gr��er gewesen w�re.},   %4. Antwortmoeglichkeit
				L5={Das Konfidenzintervall w�re breiter, wenn der Anteil der Bef�rworterInnen und der Anteil der GegnerInnen in der Stichprobe gleich gro� gewesen w�ren.},	 %5. Antwortmoeglichkeit
				L6={},	 %6. Antwortmoeglichkeit
				L7={},	 %7. Antwortmoeglichkeit
				L8={},	 %8. Antwortmoeglichkeit
				L9={},	 %9. Antwortmoeglichkeit
				%% LOESUNG: %%
				A1=2,  % 1. Antwort
				A2=5,	 % 2. Antwort
				A3=0,  % 3. Antwort
				A4=0,  % 4. Antwort
				A5=0,  % 5. Antwort
				}
\end{beispiel}