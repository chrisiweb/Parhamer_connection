\section{WS 1.2 - 6 Beladung von LKW - OA - Matura 2015/16 - Haupttermin}

\begin{beispiel}[WS 1.2]{1} %PUNKTE DES BEISPIELS
Bei einer Verkehrskontrolle wurde die Beladung von LKW �berpr�ft. 140 der �berpr�ften LKW
waren �berladen. Details der Kontrolle sind in der nachstehenden Tabelle zusammengefasst.

\begin{center}
	\begin{tabular}{|l|c|c|c|} \hline
	�berladung � in Tonnen & $\text{�}<1\,t$ & $1\,t \leq \text{�} < 3\,t$ & $3\,t \leq \text{�} < 6\,t$ \\ \hline
	Anzahl der LKW & 30 & 50 & 60 \\ \hline
	\end{tabular}
\end{center} \leer


Stelle die Daten der obigen Tabelle durch ein Histogramm dar. Dabei sollen die absoluten
H�ufigkeiten als Fl�cheninhalte von Rechtecken abgebildet werden.

\begin{center}
\resizebox{0.8\linewidth}{!}{
\psset{xunit=1.0cm,yunit=0.1cm,algebraic=true,dimen=middle,dotstyle=o,dotsize=5pt 0,linewidth=0.8pt,arrowsize=3pt 2,arrowinset=0.25}
\begin{pspicture*}(-0.66,-10.2)(6.72,63.6)
\multips(0,0)(0,10.0){8}{\psline[linestyle=dashed,linecap=1,dash=1.5pt 1.5pt,linewidth=0.4pt,linecolor=lightgray]{c-c}(0,0)(6.72,0)}
\multips(0,0)(1.0,0){8}{\psline[linestyle=dashed,linecap=1,dash=1.5pt 1.5pt,linewidth=0.4pt,linecolor=lightgray]{c-c}(0,0)(0,63.6)}
\psaxes[labelFontSize=\scriptstyle,xAxis=true,yAxis=true,Dx=1.,Dy=10.,ticksize=-2pt 0,subticks=2]{->}(0,0)(0.,0.)(6.72,63.6)
\rput[tl](3.5,-4.8){\scriptsize �berladung (in Tonnen)}
\antwort{\pspolygon[linewidth=1.2pt,linecolor=red,fillcolor=red,fillstyle=solid,opacity=0.15](0.,30.)(1.,30.)(1.,0.)(0.,0.)
\pspolygon[linewidth=1.2pt,linecolor=red,fillcolor=red,fillstyle=solid,opacity=0.15](1.,25.)(3.,25.)(3.,0.)(1.,0.)
\pspolygon[linewidth=1.2pt,linecolor=red,fillcolor=red,fillstyle=solid,opacity=0.15](3.,20.)(6.,20.)(6.,0.)(3.,0.)
\psline[linewidth=1.2pt,linecolor=red](0.,30.)(1.,30.)
\psline[linewidth=1.2pt,linecolor=red](1.,30.)(1.,0.)
\psline[linewidth=1.2pt,linecolor=red](1.,0.)(0.,0.)
\psline[linewidth=1.2pt,linecolor=red](0.,0.)(0.,30.)
\psline[linewidth=1.2pt,linecolor=red](1.,25.)(3.,25.)
\psline[linewidth=1.2pt,linecolor=red](3.,25.)(3.,0.)
\psline[linewidth=1.2pt,linecolor=red](3.,0.)(1.,0.)
\psline[linewidth=1.2pt,linecolor=red](1.,0.)(1.,25.)
\psline[linewidth=1.2pt,linecolor=red](3.,20.)(6.,20.)
\psline[linewidth=1.2pt,linecolor=red](6.,20.)(6.,0.)
\psline[linewidth=1.2pt,linecolor=red](6.,0.)(3.,0.)
\psline[linewidth=1.2pt,linecolor=red](3.,0.)(3.,20.)}
\end{pspicture*}}
\end{center}


\end{beispiel}