\section{WS 1.2 - 7 - MAT - Statistische Darstellungen - OA - Matura 2016/17 2. NT}

\begin{beispiel}[WS 1.2]{1} %PUNKTE DES BEISPIELS
Bei einer meteorologischen Messstelle wurden die Tagesh�chsttemperaturen f�r den Zeitraum von einem Monat in einem sehr hei�en Sommer aufgezeichnet. Die Messwerte in Grad Celsius k�nnen dem nachstehenden St�ngel-Blatt-Diagramm entnommen werden.

\begin{center}
	\begin{tabular}{|c|l|}\hline
	1&9\\ \hline
	2&2\,\,2\,\,3\,\,3\,\,3\\ \hline
	2&5\,\,6\,\,6\,\,6\,\,6\,\,6\,\,7\,\,7\,\,7\,\,7\,\,7\,\,7\,\,7\\ \hline
	3&1\,\,1\,\,1\,\,2\,\,3\,\,3\,\,3\,\,4\,\,4\,\,4\\ \hline
	3&8\\ \hline
	4&0\,\,0\\ \hline
	\end{tabular}
\end{center}\leer

Stelle die aufgezeichneten Tagesh�chsttemperaturen in einem Kastenschaubild (Boxplot) dar!

\begin{center}
	\resizebox{0.9\linewidth}{!}{\psset{xunit=0.6cm,yunit=0.9cm,algebraic=true,dimen=middle,dotstyle=o,dotsize=5pt 0,linewidth=1.6pt,arrowsize=3pt 2,arrowinset=0.25}
\begin{pspicture*}(13.5,-1.46)(41.66,5.14)
\multips(0,0)(0,10.0){1}{\psline[linestyle=dashed,linecap=1,dash=1.5pt 1.5pt,linewidth=0.4pt,linecolor=darkgray]{c-c}(13.5,0)(41.66,0)}
\multips(13,0)(1.0,0){29}{\psline[linestyle=dashed,linecap=1,dash=1.5pt 1.5pt,linewidth=0.4pt,linecolor=darkgray]{c-c}(0,0)(0,5.14)}
\psaxes[labelFontSize=\scriptstyle,xAxis=true,yAxis=true,Dx=1.,Dy=1.,ticksize=-2pt 0,subticks=2](0,0)(13.5,-1.46)(41.66,5.14)
\antwort{\psframe[linewidth=0.8pt,fillcolor=black,fillstyle=solid,opacity=0.1](26.,1.0)(33.,3.)
\psline[linewidth=0.8pt,fillcolor=black,fillstyle=solid,opacity=0.1](19.,1.)(19.,3.)
\psline[linewidth=0.8pt,fillcolor=black,fillstyle=solid,opacity=0.1](40.,1.)(40.,3.)
\psline[linewidth=0.8pt,fillcolor=black,fillstyle=solid,opacity=0.1](27.,1.)(27.,3.)
\psline[linewidth=0.8pt,fillcolor=black,fillstyle=solid,opacity=0.1](19.,2.)(26.,2.)
\psline[linewidth=0.8pt,fillcolor=black,fillstyle=solid,opacity=0.1](33.,2.)(40.,2.)}
\rput[tl](35,-0.7){Temperatur in $^\circ$C}
\end{pspicture*}}
\end{center}
\end{beispiel}