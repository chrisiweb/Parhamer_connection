\section{WS 1.4 - 2 Monatsnettoeinkommen - OA - BIFIE}

\begin{beispiel}[WS 1.4]{1} %PUNKTE DES BEISPIELS
				Die nachstehende Tabelle zeigt Daten zum Monatsnettoeinkommen unselbstst�ndig Erwerbst�tiger in �sterreich (im Jahresdurchschnitt 2010) in Abh�ngigkeit vom Alter.
				
\begin{scriptsize}
\begin{center}
\begin{longtable}{|C{2cm}|C{2cm}|C{1.3cm}|C{1.3cm}|C{1.3cm}|C{1.3cm}|C{1.3cm}|C{1.3cm}|}\hline
\multirow{3}{2cm}{Merkmale}&\multirow{2}{2cm}{Unselbstst�ndig Erwerbst�tige}&\multirow{3}{1.3cm}{arithmet-\newline isches Mittel}&\multirow{2}{1.3cm}{10\%}&\multicolumn{3}{|c|}{Quartile}&\multirow{2}{1.3cm}{90\%}\\ \cline{5-7}
&&&&25\%&50\% Median&75\%& \\ \cline{2-2} \cline{4-8}
&in 1.000&&\multicolumn{5}{|c|}{verdienen weniger oder gleichviel als ... } \\ \hline
\end{longtable}

\begin{longtable}{C{2cm}C{2cm}C{1.3cm}C{1.3cm}C{1.3cm}C{1.3cm}C{1.3cm}C{1.3cm}}
&&&\multicolumn{4}{c}{\textbf{Insgesamt}}&\\
\textbf{Insgesamt}&3.407,9&1.872,8&665.0&1.188,0&1.707,0&2.303,0&3.122,0\\
\textbf{Alter (in Jahren)}&&&&&&&\\
15-19 Jahre&173,5&799,4&399,0&531,0&730,0&1.020,0&1.315,0\\
20-29 Jahre&705,1&1.487,0&598,0&1.114,0&1.506,0&1.843,0&2.175,0\\
30-39 Jahre&803,1&1.885,7&770,0&1.252,0&1.778,0&2.306,0&2.997,0\\
40-49 Jahre&1.020,4&2.086,1&863,0&1.338,0&1892,0&2.556,0&3.442,0\\
50-59 Jahre&632,8&2.205,0&893,0&1.394,0&1.977,0&2.779,0&3.710,0\\
60+ Jahre&73,0&2.144,7&258,0&420,0&1.681,0&3.254,0&4.808,0\\

\end{longtable}
\end{center}
\end{scriptsize}

Wie viel Euro verdienen genau 25\% der 30-39 J�hrigen mindestens? Gib an, welche statistische Kennzahl du zur Beantwortung dieser Frage ben�tigst, und ermittle die entsprechende Verdienstuntergrenze.\\

\antwort{3. Quartil: EUR 2.306}
				\end{beispiel}