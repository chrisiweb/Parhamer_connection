\section{WS 2.1 - 2 Sch�lerinnenbefragung - MC - BIFIE}

\begin{beispiel}[WS 2.1]{1} %PUNKTE DES BEISPIELS
In einer Schule wird unter den M�dchen eine Umfrage durchgef�hrt. Dazu werden pro Klasse
zwei Sch�lerinnen zuf�llig f�r ein Interview ausgew�hlt. Eva und Sonja gehen in die 1A.
F�r das Ereignis $E_1$ gilt: Eva und Sonja werden f�r das Interview ausgew�hlt. \leer

Welche der nachstehenden Aussagen beschreibt das Gegenereignis $E_2$? (Das Gegenereignis
$E_2$ enth�lt diejenigen Elemente des Grundraums, die nicht Elemente von $E_1$ sind.)
Kreuze die zutreffende Aussage an.

\multiplechoice[6]{  %Anzahl der Antwortmoeglichkeiten, Standard: 5
				L1={Nur Eva wird ausgew�hlt.},   %1. Antwortmoeglichkeit 
				L2={Keines der beiden M�dchen wird ausgew�hlt.},   %2. Antwortmoeglichkeit
				L3={Mindestens eines der beiden M�dchen wird ausgew�hlt.},   %3. Antwortmoeglichkeit
				L4={Nur Sonja wird ausgew�hlt.},   %4. Antwortmoeglichkeit
				L5={H�chstens eines der beiden M�dchen wird ausgew�hlt.},	 %5. Antwortmoeglichkeit
				L6={Genau eines der beiden M�dchen wird ausgew�hlt.},	 %6. Antwortmoeglichkeit
				L7={},	 %7. Antwortmoeglichkeit
				L8={},	 %8. Antwortmoeglichkeit
				L9={},	 %9. Antwortmoeglichkeit
				%% LOESUNG: %%
				A1=5,  % 1. Antwort
				A2=0,	 % 2. Antwort
				A3=0,  % 3. Antwort
				A4=0,  % 4. Antwort
				A5=0,  % 5. Antwort
				} 
\end{beispiel}