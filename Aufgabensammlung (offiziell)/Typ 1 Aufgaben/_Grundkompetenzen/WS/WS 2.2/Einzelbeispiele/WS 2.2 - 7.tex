\section{WS 2.2 - 7 Online-Gl�cksspiel - MC - Matura NT 2 15/16}

\begin{beispiel}{1} %PUNKTE DES BEISPIELS
Ein Mann spielt �ber einen l�ngeren Zeitraum regelm��ig dasselbe Online-Gl�cksspiel mit konstanter Gewinnwahrscheinlichkeit. Von 768 Spielen gewinnt er 162. \leer

Mit welcher ungef�hren Wahrscheinlichkeit wird er das n�chste Spiel gewinnen?

Kreuze den zutreffenden Sch�tzwert f�r diese Wahrscheinlichkeit an. \leer

\multiplechoice[6]{  %Anzahl der Antwortmoeglichkeiten, Standard: 5
				L1={$0,162\,\%$},   %1. Antwortmoeglichkeit 
				L2={$4,74\,\%$},   %2. Antwortmoeglichkeit
				L3={$16,2\,\%$},   %3. Antwortmoeglichkeit
				L4={$21,1\,\%$},   %4. Antwortmoeglichkeit
				L5={$7,68\,\%$},	 %5. Antwortmoeglichkeit
				L6={$76,6\,\%$},	 %6. Antwortmoeglichkeit
				L7={},	 %7. Antwortmoeglichkeit
				L8={},	 %8. Antwortmoeglichkeit
				L9={},	 %9. Antwortmoeglichkeit
				%% LOESUNG: %%
				A1=4,  % 1. Antwort
				A2=0,	 % 2. Antwort
				A3=0,  % 3. Antwort
				A4=0,  % 4. Antwort
				A5=0,  % 5. Antwort
				}
\end{beispiel}