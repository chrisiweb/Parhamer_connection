\section{WS 2.2 - 1 W�rfelergebnisse - LT - BIFIE}

\begin{beispiel}[WS 2.2]{1}
Zwei Spielw�rfel (6 Seiten, beschriftet mit 1 bis 6 Augen) werden geworfen und die Augensumme wird ermittelt.

\lueckentext[-0.2]{
				text={Die Wahrscheinlichkeit, das Ereignis "`Augensumme 6"' zu w�rfeln, ist \gap Wahrscheinlichkeit, das Ereignis "`Augensumme 9"' zu w�rfeln, weil \gap.}, 	%Lueckentext Luecke=\gap
				L1={gr��er als die}, 		%1.Moeglichkeit links  
				L2={kleiner als die}, 		%2.Moeglichkeit links
				L3={gleich der}, 		%3.Moeglichkeit links
				R1={6 kleiner als 9 ist und das Ereignis "`Augensumme 6"' somit seltener eintritt}, 		%1.Moeglichkeit rechts 
				R2={die Wahrscheinlichkeit beide Male $\frac{5}{36}$ betr�gt}, 		%2.Moeglichkeit rechts
				R3={es nur vier M�glichkeiten gibt, die Augen-
summe "`9"' zu w�rfeln, aber f�nf M�glichkeiten, die Augensumme "`6"' zu w�rfeln}, 		%3.Moeglichkeit rechts
				%% LOESUNG: %%
				A1=1,   % Antwort links
				A2=3		% Antwort rechts 
				}
\end{beispiel}