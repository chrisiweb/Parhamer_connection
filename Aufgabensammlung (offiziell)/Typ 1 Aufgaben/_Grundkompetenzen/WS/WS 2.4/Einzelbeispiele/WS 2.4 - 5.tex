\section{WS 2.4 - 5 Binomialkoeffizient - MC - Matura 2013/14 1. Nebentermin}

\begin{beispiel}[WS 2.4]{1} %PUNKTE DES BEISPIELS
				Betrachtet wird der Binomialkoeffizient $\binom{6}{2}$.
				
				Kreuze die beiden Aufgabenstellungen an, die mit der Rechnung $\binom{6}{2}=15$ gel�st werden k�nnen!\leer
				
				\multiplechoice[5]{  %Anzahl der Antwortmoeglichkeiten, Standard: 5
								L1={Gegeben sind sechs verschiedene Punkte einer Ebene, von denen nie mehr als zwei auf einer Geraden liegen. Wie viele M�glichkeiten gibt es, zwei Punkte auszuw�hlen, um jeweils eine Gerade durchzulegen?},   %1. Antwortmoeglichkeit 
								L2={An einem Wettrennen nehmen sechs Personen teil. Wie viele M�glichkeiten gibt es f�r den Zieleinlauf, wenn nur die ersten beiden Pl�tze relevant sind?},   %2. Antwortmoeglichkeit
								L3={Von sechs Kugeln sind vier rot und zwei blau. Sie unterscheiden sich nur durch ihre Farbe. Wie viele M�glichkeiten gibt es, die Kugeln in einer Reihe anzuordnen?},   %3. Antwortmoeglichkeit
								L4={Sechs M�dchen einer Schulklasse kandidieren f�r das Amt der Klassensprecherin. Die Siegerin der Wahl soll Klassensprecherin werden, die Zweitplatzierte deren Stellvertreterin. Wie viele M�glichkeiten gibt es f�r die Vergabe der beiden �mter?},   %4. Antwortmoeglichkeit
								L5={Wie viele sechsstellige Zahlen k�nnen aus den Ziffern 6 und 2 gebildet werden?},	 %5. Antwortmoeglichkeit
								L6={},	 %6. Antwortmoeglichkeit
								L7={},	 %7. Antwortmoeglichkeit
								L8={},	 %8. Antwortmoeglichkeit
								L9={},	 %9. Antwortmoeglichkeit
								%% LOESUNG: %%
								A1=1,  % 1. Antwort
								A2=3,	 % 2. Antwort
								A3=0,  % 3. Antwort
								A4=0,  % 4. Antwort
								A5=0,  % 5. Antwort
								}
\end{beispiel}