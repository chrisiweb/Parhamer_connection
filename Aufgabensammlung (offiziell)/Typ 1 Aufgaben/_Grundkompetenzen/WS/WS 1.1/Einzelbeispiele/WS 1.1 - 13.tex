\section{WS 1.1 - 13 St�ngel-Blatt-Diagramme - MC - Matura NT 1 16/17}

\begin{beispiel}[WS 1.1]{1} %PUNKTE DES BEISPIELS
Die nachstehenden St�ngel-Blatt-Diagramme zeigen die Anzahl der Kinobesucher/innen je Vorstellung der Filme $A$ und $B$ im Lauf einer Woche. In diesen Diagrammen ist die Einheit des St�ngels 10, die des Blatts 1.

\meinlr{\hspace{2cm}
\begin{tabular}{|r|l|}\hline
\multicolumn{2}{|c|}{\cellcolor[gray]{0.9}Film $A$}\\ \hline
2&0,3,8\\ \hline
3&6,7\\ \hline
4&1,1,5,6\\ \hline
5&2,6,8,9\\ \hline
6&1,8\\ \hline
\end{tabular}}
{\begin{tabular}{|r|l|}\hline
\multicolumn{2}{|c|}{\cellcolor[gray]{0.9}Film $B$}\\ \hline
2&1\\ \hline
3&1,4,5\\ \hline
4&4,5,8\\ \hline
5&0,5,7,7\\ \hline
6&1,2\\ \hline
7&0\\ \hline
\end{tabular}}

Kreuze diejenige(n) Aussage(n) an, die bezogen auf die dargestellten St�ngel-Blatt-Diagramme mit Sicherheit zutrifft/zutreffen!

\multiplechoice[5]{  %Anzahl der Antwortmoeglichkeiten, Standard: 5
				L1={Es gab in dieser Woche mehr Vorstellungen des Films $A$ als der Films $B$.},   %1. Antwortmoeglichkeit 
				L2={Der Median der Anzahl der Besucher/innen ist bei Film $A$ gr��er als bei Film $B$.},   %2. Antwortmoeglichkeit
				L3={Die Spannweite der Anzahl der Besucher/innen ist bei Film $A$ kleiner als bei Film $B$.},   %3. Antwortmoeglichkeit
				L4={Die Gesamtanzahl der Besucher/innen in dieser Woche war bei Film $A$ gr��er als bei Film $B$.},   %4. Antwortmoeglichkeit
				L5={In einer Vorstellung des Films $B$ waren mehr Besucher/innen als in jeder einzelnen Vorstellung des Films $A$.},	 %5. Antwortmoeglichkeit
				L6={},	 %6. Antwortmoeglichkeit
				L7={},	 %7. Antwortmoeglichkeit
				L8={},	 %8. Antwortmoeglichkeit
				L9={},	 %9. Antwortmoeglichkeit
				%% LOESUNG: %%
				A1=1,  % 1. Antwort
				A2=3,	 % 2. Antwort
				A3=4,  % 3. Antwort
				A4=5,  % 4. Antwort
				A5=0,  % 5. Antwort
				}
\end{beispiel}