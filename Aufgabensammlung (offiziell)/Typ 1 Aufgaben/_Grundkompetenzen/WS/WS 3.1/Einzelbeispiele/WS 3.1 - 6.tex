\section{WS 3.1 - 6 Zufallsvariable - OA - Matura 2015/16 - Nebentermin 1}

\begin{beispiel}[WS 3.1]{1} %PUNKTE DES BEISPIELS

Nachstehend sind die sechs Seitenfl�chen eines fairen Spielw�rfels abgebildet. Auf jeder Seitenfl�che sind drei Symbole dargestellt. (Ein W�rfel ist "`fair"', wenn die Wahrscheinlichkeit, nach einem Wurf nach oben zu zeigen, f�r alle sechs Seitenfl�chen gleich gro� ist.) \leer


\meinlcr{\resizebox{0.8\linewidth}{!}{
\psset{xunit=1.0cm,yunit=1.0cm,algebraic=true,dimen=middle,dotstyle=o,dotsize=5pt 0,linewidth=0.8pt,arrowsize=3pt 2,arrowinset=0.25}
\begin{pspicture*}(1.8350070501313624,3.3391980762417046)(3.6321796193372906,5.126708427333627)
\rput[tl](2.0475758486395907,4.93){$\bigstar$}
\psline[linewidth=0.6pt](2.,5.)(3.5,5.)
\psline[linewidth=0.6pt](3.5,3.5)(2.,3.5)
\psline[linewidth=0.6pt](2.,5.)(2.,3.5)
\psline[linewidth=0.6pt](3.5,5.)(3.5,3.5)
\rput[tl](2.5693356267961502,4.43){$\Delta$}
\rput[tl](3.07,3.88){\LARGE $\circ$}
\psline[linewidth=0.6pt,linestyle=dashed,dash=1pt 1pt](2.5,5.)(2.5,3.5)
\psline[linewidth=0.6pt,linestyle=dashed,dash=1pt 1pt](3.,5.)(3.,3.5)
\psline[linewidth=0.6pt,linestyle=dashed,dash=1pt 1pt](2.,4.5)(3.5,4.5)
\psline[linewidth=0.6pt,linestyle=dashed,dash=1pt 1pt](2.,4.)(3.5,4.)
\end{pspicture*}}}
{\resizebox{0.8\linewidth}{!}{
\psset{xunit=1.0cm,yunit=1.0cm,algebraic=true,dimen=middle,dotstyle=o,dotsize=5pt 0,linewidth=0.8pt,arrowsize=3pt 2,arrowinset=0.25}
\begin{pspicture*}(1.8350070501313624,3.3391980762417046)(3.6321796193372906,5.126708427333627)
\psline[linewidth=0.6pt](2.,5.)(3.5,5.)
\psline[linewidth=0.6pt](3.5,3.5)(2.,3.5)
\psline[linewidth=0.6pt](2.,5.)(2.,3.5)
\psline[linewidth=0.6pt](3.5,5.)(3.5,3.5)
\rput[tl](2.07,4.9){\LARGE $\circ$}
\rput[tl](2.5693356267961502,4.43){$\bigstar$}
\rput[tl](3.07,3.88){\LARGE $\circ$}
\psline[linewidth=0.6pt,linestyle=dashed,dash=1pt 1pt](2.5,5.)(2.5,3.5)
\psline[linewidth=0.6pt,linestyle=dashed,dash=1pt 1pt](3.,5.)(3.,3.5)
\psline[linewidth=0.6pt,linestyle=dashed,dash=1pt 1pt](2.,4.5)(3.5,4.5)
\psline[linewidth=0.6pt,linestyle=dashed,dash=1pt 1pt](2.,4.)(3.5,4.)
\end{pspicture*}}}{
\resizebox{0.8\linewidth}{!}{
\psset{xunit=1.0cm,yunit=1.0cm,algebraic=true,dimen=middle,dotstyle=o,dotsize=5pt 0,linewidth=0.8pt,arrowsize=3pt 2,arrowinset=0.25}
\begin{pspicture*}(1.8350070501313624,3.3391980762417046)(3.6321796193372906,5.126708427333627)
\psline[linewidth=0.6pt](2.,5.)(3.5,5.)
\psline[linewidth=0.6pt](3.5,3.5)(2.,3.5)
\psline[linewidth=0.6pt](2.,5.)(2.,3.5)
\psline[linewidth=0.6pt](3.5,5.)(3.5,3.5)
\rput[tl](2.07,4.9){$\Delta$}
\rput[tl](2.5693356267961502,4.38){\LARGE $\circ$}
\rput[tl](3.05,3.93){$\bigstar$}
\psline[linewidth=0.6pt,linestyle=dashed,dash=1pt 1pt](2.5,5.)(2.5,3.5)
\psline[linewidth=0.6pt,linestyle=dashed,dash=1pt 1pt](3.,5.)(3.,3.5)
\psline[linewidth=0.6pt,linestyle=dashed,dash=1pt 1pt](2.,4.5)(3.5,4.5)
\psline[linewidth=0.6pt,linestyle=dashed,dash=1pt 1pt](2.,4.)(3.5,4.)
\end{pspicture*}}}

\meinlcr{\resizebox{0.8\linewidth}{!}{
\psset{xunit=1.0cm,yunit=1.0cm,algebraic=true,dimen=middle,dotstyle=o,dotsize=5pt 0,linewidth=0.8pt,arrowsize=3pt 2,arrowinset=0.25}
\begin{pspicture*}(1.8350070501313624,3.3391980762417046)(3.6321796193372906,5.126708427333627)
\rput[tl](2.0475758486395907,4.93){$\bigstar$}
\psline[linewidth=0.6pt](2.,5.)(3.5,5.)
\psline[linewidth=0.6pt](3.5,3.5)(2.,3.5)
\psline[linewidth=0.6pt](2.,5.)(2.,3.5)
\psline[linewidth=0.6pt](3.5,5.)(3.5,3.5)
\rput[tl](2.5693356267961502,4.38){\LARGE $\circ$}
\rput[tl](3.05,3.93){$\bigstar$}
\psline[linewidth=0.6pt,linestyle=dashed,dash=1pt 1pt](2.5,5.)(2.5,3.5)
\psline[linewidth=0.6pt,linestyle=dashed,dash=1pt 1pt](3.,5.)(3.,3.5)
\psline[linewidth=0.6pt,linestyle=dashed,dash=1pt 1pt](2.,4.5)(3.5,4.5)
\psline[linewidth=0.6pt,linestyle=dashed,dash=1pt 1pt](2.,4.)(3.5,4.)
\end{pspicture*}}}
{\resizebox{0.8\linewidth}{!}{
\psset{xunit=1.0cm,yunit=1.0cm,algebraic=true,dimen=middle,dotstyle=o,dotsize=5pt 0,linewidth=0.8pt,arrowsize=3pt 2,arrowinset=0.25}
\begin{pspicture*}(1.8350070501313624,3.3391980762417046)(3.6321796193372906,5.126708427333627)
\psline[linewidth=0.6pt](2.,5.)(3.5,5.)
\psline[linewidth=0.6pt](3.5,3.5)(2.,3.5)
\psline[linewidth=0.6pt](2.,5.)(2.,3.5)
\psline[linewidth=0.6pt](3.5,5.)(3.5,3.5)
\rput[tl](2.07,4.9){$\Delta$}
\rput[tl](2.5693356267961502,4.38){\LARGE $\circ$}
\rput[tl](3.08,3.88){$\Delta$}
\psline[linewidth=0.6pt,linestyle=dashed,dash=1pt 1pt](2.5,5.)(2.5,3.5)
\psline[linewidth=0.6pt,linestyle=dashed,dash=1pt 1pt](3.,5.)(3.,3.5)
\psline[linewidth=0.6pt,linestyle=dashed,dash=1pt 1pt](2.,4.5)(3.5,4.5)
\psline[linewidth=0.6pt,linestyle=dashed,dash=1pt 1pt](2.,4.)(3.5,4.)
\end{pspicture*}}}{
\resizebox{0.8\linewidth}{!}{
\psset{xunit=1.0cm,yunit=1.0cm,algebraic=true,dimen=middle,dotstyle=o,dotsize=5pt 0,linewidth=0.8pt,arrowsize=3pt 2,arrowinset=0.25}
\begin{pspicture*}(1.8350070501313624,3.3391980762417046)(3.6321796193372906,5.126708427333627)
\psline[linewidth=0.6pt](2.,5.)(3.5,5.)
\psline[linewidth=0.6pt](3.5,3.5)(2.,3.5)
\psline[linewidth=0.6pt](2.,5.)(2.,3.5)
\psline[linewidth=0.6pt](3.5,5.)(3.5,3.5)
\rput[tl](2.07,4.9){\LARGE $\circ$}
\rput[tl](2.5693356267961502,4.43){$\bigstar$}
\rput[tl](3.05,3.93){$\bigstar$}
\psline[linewidth=0.6pt,linestyle=dashed,dash=1pt 1pt](2.5,5.)(2.5,3.5)
\psline[linewidth=0.6pt,linestyle=dashed,dash=1pt 1pt](3.,5.)(3.,3.5)
\psline[linewidth=0.6pt,linestyle=dashed,dash=1pt 1pt](2.,4.5)(3.5,4.5)
\psline[linewidth=0.6pt,linestyle=dashed,dash=1pt 1pt](2.,4.)(3.5,4.)
\end{pspicture*}}} \leer

Bei einem Zufallsversuch wird der W�rfel einmal geworfen. Die Zufallsvariable $X$ beschreibt die
Anzahl der Sterne auf der nach oben zeigenden Seitenfl�che. \leer

Gib die Wahrscheinlichkeitsverteilung von $X$ an, d.h. die m�glichen Werte von $X$ samt zugeh�riger
Wahrscheinlichkeiten.

\antwort{Die Zufallsvariable $X$ kann die Werte $x_1= 0$, $x_2= 1$ und $x_3=2$ annehmen.

Es gilt:

$P(X=0)=\frac{1}{6}$, $P(X=1)=\frac{3}{6}$, $P(X=2)=\frac{2}{6}$ \leer

L�sungsschl�ssel:

Ein Punkt f�r die korrekte Angabe aller m�glichen Werte, die die Zufallsvariable $X$ annehmen
kann, und der jeweils zugeh�rigen Wahrscheinlichkeit. Andere Schreibweisen der Ergebnisse sind
ebenfalls als richtig zu werten. Eine korrekte grafische Darstellung der Wahrscheinlichkeitsverteilung ist ebenfalls als richtig zu werten.}

\end{beispiel}