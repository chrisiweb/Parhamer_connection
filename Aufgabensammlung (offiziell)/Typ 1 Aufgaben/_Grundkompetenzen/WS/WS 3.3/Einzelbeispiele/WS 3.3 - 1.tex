\section{WS 3.3 - 1 Aufnahmetest - MC - BIFIE}
\begin{beispiel}[WS 3.3]{1} %PUNKTE DES BEISPIELS
Eine Universit�t f�hrt einen Aufnahmetest durch. Dabei werden zehn Multiple-Choice-Fragen gestellt,
wobei jede Frage vier Antwortm�glichkeiten hat. Nur eine davon ist richtig. In den letzten Jahren wurden durchschnittlich 40 Bewerber/innen aufgenommen. Dabei traten etwa 95\,\% der angemeldeten Kandidatinnen und Kandidaten tats�chlich zum Aufnahmetest an. Heuer treten 122 Bewerber/innen zu diesem Aufnahmetest an.\leer

Nimm an, dass Kandidat $K$ alle Antworten v�llig zuf�llig ankreuzt. \leer

Kreuze die zutreffende(n) Aussage(n) an.

\multiplechoice[5]{  %Anzahl der Antwortmoeglichkeiten, Standard: 5
				L1={Die Anzahl der angemeldeten Kandidatinnen und Kandidaten, die tats�chlich
zum Aufnahmetest erscheinen, ist binomialverteilt mit $n = 122$
und $p = 0,40$.},   %1. Antwortmoeglichkeit 
				L2={Die Anzahl der richtig beantworteten Fragen des Aufnahmetests des
Kandidaten $K$ ist binomialverteilt mit $n = 10$ und $p = 0,25$.},   %2. Antwortmoeglichkeit
				L3={Die durchschnittliche Anzahl der richtig beantworteten Fragen aller angetretenen
Kandidatinnen und Kandidaten ist binomialverteilt mit $n = 122$
und $p = 0,40$.},   %3. Antwortmoeglichkeit
				L4={Die Anzahl der zuf�llig ankreuzenden Kandidatinnen und Kandidaten, die
aufgenommen werden, ist binomialverteilt mit $n = 40$ und $p = 0,25$.},   %4. Antwortmoeglichkeit
				L5={Die Anzahl der falsch beantworteten Fragen des Aufnahmetests des
Kandidaten $K$ ist binomialverteilt mit $n = 10$ und $p = 0,75$.},	 %5. Antwortmoeglichkeit
				L6={},	 %6. Antwortmoeglichkeit
				L7={},	 %7. Antwortmoeglichkeit
				L8={},	 %8. Antwortmoeglichkeit
				L9={},	 %9. Antwortmoeglichkeit
				%% LOESUNG: %%
				A1=2,  % 1. Antwort
				A2=5,	 % 2. Antwort
				A3=0,  % 3. Antwort
				A4=0,  % 4. Antwort
				A5=0,  % 5. Antwort
				}



\end{beispiel} 