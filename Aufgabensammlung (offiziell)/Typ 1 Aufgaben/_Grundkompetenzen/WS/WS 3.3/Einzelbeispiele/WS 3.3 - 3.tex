\section{WS 3.3 - 3 Modellierung mit Binomialverteilung - MC - BIFIE}

\begin{beispiel}[WS 3.3]{1} %PUNKTE DES BEISPIELS
Gegeben sind f�nf Situationen, bei denen nach einer Wahrscheinlichkeit gefragt wird. \leer

Kreuze  diejenige(n) Situation(en) an, die mithilfe der Binomialverteilung modelliert werden
kann/k�nnen.

\multiplechoice[5]{  %Anzahl der Antwortmoeglichkeiten, Standard: 5
				L1={In der Kantine eines Betriebes essen 80 Personen. Am Montag werden ein
vegetarisches Gericht und drei weitere Men�s angeboten. Erfahrungsgem��
w�hlt jede vierte Person das vegetarische Gericht. Es werden 20 vegetarische
Gerichte vorbereitet. \\
Wie gro� ist die Wahrscheinlichkeit, dass diese nicht ausreichen? },   %1. Antwortmoeglichkeit 
				L2={Bei einer Lieferung von 20 Smartphones sind f�nf defekt. Es werden nacheinander
drei Ger�te entnommen, getestet und nicht zur�ckgelegt.
Mit welcher Wahrscheinlichkeit sind mindestens zwei davon defekt? },   %2. Antwortmoeglichkeit
				L3={In einer Klasse m�ssen die Sch�ler/innen bei der �berpr�fung der Bildungsstandards
auf einem anonymen Fragebogen ihr Geschlecht (m, w) ankreuzen.
In der Klasse sind 16 Sch�lerinnen und 12 Sch�ler. F�nf Personen haben auf
dem Fragebogen das Geschlecht nicht angekreuzt. \\
Mit welcher Wahrscheinlichkeit befinden sich drei Sch�ler unter den f�nf Personen? },   %3. Antwortmoeglichkeit
				L4={Ein Gro�h�ndler erh�lt eine Lieferung von 2 000 Smartphones, von denen
erfahrungsgem�� 5\,\% defekt sind.\\
Mit welcher Wahrscheinlichkeit befinden sich 80 bis 90 defekte Ger�te in der
Lieferung? },   %4. Antwortmoeglichkeit
				L5={In einer Klinik werden 500 kranke Personen mit einem bestimmten Medikament
behandelt. Die Wahrscheinlichkeit, dass schwere Nebenwirkungen auftreten,
betr�gt 0,001. \\
Wie gro� ist die Wahrscheinlichkeit, dass bei mehr als zwei Personen schwere
Nebenwirkungen auftreten? },	 %5. Antwortmoeglichkeit
				L6={},	 %6. Antwortmoeglichkeit
				L7={},	 %7. Antwortmoeglichkeit
				L8={},	 %8. Antwortmoeglichkeit
				L9={},	 %9. Antwortmoeglichkeit
				%% LOESUNG: %%
				A1=1,  % 1. Antwort
				A2=4,	 % 2. Antwort
				A3=5,  % 3. Antwort
				A4=0,  % 4. Antwort
				A5=0,  % 5. Antwort
				} 

\end{beispiel}