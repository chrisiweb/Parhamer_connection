\section{AG 2.3 - 15 - MAT - L�sungen einer quadratischen Gleichung - LT - Matura 2016/17 2. NT}

\begin{beispiel}[AG 2.3]{1} %PUNKTE DES BEISPIELS
Eine Gleichung, die man auf die Form $a\cdot x^2+b\cdot x+c=0$ mit $a,b\in\mathbb{R}$ und $a\neq 0$ umformen kann, nennt man quadratische Gleichung in der Variablen $x$ mit den Koeffizienten $a,b,c$.

\lueckentext{
				text={Eine quadratische Gleichung der Form $a\cdot x^2+b\cdot x+c=0$ mit \gap hat in jedem Fall \gap.}, 	%Lueckentext Luecke=\gap
				L1={$a>0$ und $c>0$}, 		%1.Moeglichkeit links  
				L2={$a>0$ und $c<0$}, 		%2.Moeglichkeit links
				L3={$a<0$ und $c<0$}, 		%3.Moeglichkeit links
				R1={zwei verschiedene reelle L�sungen}, 		%1.Moeglichkeit rechts 
				R2={genau eine reelle L�sung}, 		%2.Moeglichkeit rechts
				R3={keine reelle L�sung}, 		%3.Moeglichkeit rechts
				%% LOESUNG: %%
				A1=2,   % Antwort links
				A2=1		% Antwort rechts 
				}
\end{beispiel}