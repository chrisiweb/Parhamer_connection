\section{AG 2.3 - 14 L�sung einer quadratischen Gleichung - LT - Matura NT 1 16/17}

\begin{beispiel}[AG 2.3]{1} %PUNKTE DES BEISPIELS
Gegeben ist eine quadratische Gleichung $x^2+p\cdot x+3=0$ mit $p\in\mathbb{R}$.

\lueckentext[0.12]{
				text={Diese Gleichung hat \gap, wenn \gap gilt.}, 	%Lueckentext Luecke=\gap
				L1={unendlich viele reelle L�sungen}, 		%1.Moeglichkeit links  
				L2={genau eine reelle L�sung}, 		%2.Moeglichkeit links
				L3={keine reelle L�sung}, 		%3.Moeglichkeit links
				R1={$\frac{p^2}{4}-3>0$}, 		%1.Moeglichkeit rechts 
				R2={$\frac{p^2}{4}-3<0$}, 		%2.Moeglichkeit rechts
				R3={$\frac{p^2}{4}-3>1$}, 		%3.Moeglichkeit rechts
				%% LOESUNG: %%
				A1=3,   % Antwort links
				A2=2		% Antwort rechts 
				}
\end{beispiel}