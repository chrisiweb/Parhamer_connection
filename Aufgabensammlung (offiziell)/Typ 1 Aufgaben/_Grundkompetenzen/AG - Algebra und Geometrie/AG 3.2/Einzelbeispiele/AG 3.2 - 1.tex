\section{AG 3.2 - 1 Kr�fte - OA - BIFIE}

\begin{beispiel}[AG 3.2]{1} %PUNKTE DES BEISPIELS
Zwei an einem Punkt $P$ eines K�rpers angreifende Kr�fte $\vek{F_{1}}$ und $\vek{F_{2}}$ lassen sich durch eine einzige am selben Punkt angreifende resultierende Kraft $\vek{F}$ ersetzen, die allein diesselbe Wirkung aus�bt wie $\vek{F_{1}}$ und $\vek{F_{2}}$ zusammen.

Gegeben sind zwei an einem Punkt $P$ angreifenden Kr�fte $\vek{F_{1}}$ und $\vek{F_{2}}$. Ermittle grafisch die resultierende Kraft $\vek{F}$ als Summe der Kr�fte $\vek{F_{1}}$ und $\vek{F_{2}}$!
\leer

\psset{xunit=1.0cm,yunit=1.0cm,algebraic=true,dimen=middle,dotstyle=o,dotsize=5pt 0,linewidth=0.8pt,arrowsize=3pt 2,arrowinset=0.25}
\begin{pspicture*}(-2.04,-3.42)(13.06,7.9)
\multips(0,-3)(0,1.0){12}{\psline[linestyle=dashed,linecap=1,dash=1.5pt 1.5pt,linewidth=0.4pt,linecolor=lightgray]{c-c}(-2.04,0)(13.06,0)}
\multips(-2,0)(1.0,0){16}{\psline[linestyle=dashed,linecap=1,dash=1.5pt 1.5pt,linewidth=0.4pt,linecolor=lightgray]{c-c}(0,-3.42)(0,7.9)}
\psaxes[labelFontSize=\scriptstyle,xAxis=false,yAxis=false,Dx=1.,Dy=1.,ticksize=-2pt 0,subticks=2]{->}(0,0)(-2.04,-3.42)(13.06,7.9)
\psline(0.,0.)(6.,-1.)
\psline(0.,0.)(2.,3.)
\rput[tl](5.06,-1.2){$\overrightarrow{F_{2}}$}
\rput[tl](1.06,3.18){$\overrightarrow{F_{1}}$}
\antwort{\psline[linecolor=red](2.,3.)(8.,2.)
\psline[linecolor=red](6.,-1.)(8.,2.)
\psline[linestyle=dashed,dash=1pt 3pt 5pt 3pt ,linecolor=red](0.,0.)(8.,2.)
\rput[tl](3.94,2.02){$\red{\overrightarrow{F}}$}}
\begin{scriptsize}
\psdots[dotsize=3pt 0,dotstyle=*](0.,0.)
\rput[bl](-0.36,-0.32){P}
\psdots[dotsize=3pt 0,dotstyle=triangle*,dotangle=270](6.,-1.)
\psdots[dotsize=3pt 0,dotstyle=triangle*,dotangle=90](2.,3.)
\psdots[dotsize=3pt 0,dotstyle=triangle*,dotangle=180,linecolor=darkgray](8.,2.)
\end{scriptsize}
\end{pspicture*}
\end{beispiel}