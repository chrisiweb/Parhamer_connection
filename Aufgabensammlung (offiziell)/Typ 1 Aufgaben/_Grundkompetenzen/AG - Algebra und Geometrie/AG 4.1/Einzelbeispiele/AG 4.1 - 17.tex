\section{AG 4.1 - 17 - MAT - Gef�lle einer Regenrinne - OA - Matura 2016/17 2. NT}

\begin{beispiel}[AG 4.1]{1} %PUNKTE DES BEISPIELS
Eine Regenrinne hat eine bestimmte L�nge $l$ (in Metern). Damit das Wasser gut abrinnt, muss die Regenrinne unter einem Winkel von mindestens $\alpha$ zur Horizontalen geneigt sein. Dadurch ergibt sich ein H�henunterschied von mindestens $h$ Metern zwischen den beiden Endpunkten der Regen rinne.

Gib eine Formel zur Berechnung von $h$ in Abh�ngigkeit von $l$ und $\alpha$ an!\leer

$h=$\,\antwort[\rule{3cm}{0.3pt}]{$l\cdot\sin(\alpha)$}
\end{beispiel}