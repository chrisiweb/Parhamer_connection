\section{AG 4.1 - 8 Vermessung einer unzug�nglichen Steilwand - OA - Matura 2015/16 - Haupttermin}

\begin{beispiel}[AG 4.1]{1} %PUNKTE DES BEISPIELS
Ein Steilwandst�ck $CD$ mit der H�he $h = \overline{CD}$ ist unzug�nglich. Um $h$ bestimmen zu k�nnen, werden die Entfernung $e = 6$ Meter und zwei Winkel $\alpha = 24^\circ$ und $\beta = 38^\circ$ gemessen. Der Sachverhalt wird durch die nachstehende (nicht ma�stabgetreue) Abbildung veranschaulicht. 

\begin{center}
\resizebox{0.5\linewidth}{!}{
\newrgbcolor{sqsqsq}{0.12549019607843137 0.12549019607843137 0.12549019607843137}
\newrgbcolor{wqwqwq}{0.3764705882352941 0.3764705882352941 0.3764705882352941}
\psset{xunit=1.0cm,yunit=1.0cm,algebraic=true,dimen=middle,dotstyle=o,dotsize=5pt 0,linewidth=0.8pt,arrowsize=3pt 2,arrowinset=0.25}
\begin{pspicture*}(-3.66,-2.9)(5.9,3.48)
\psline(-3.,-2.)(5.,-2.)
\psline(5.,-2.)(5.,3.)
\psline(-3.,-2.)(5.,1.)
\psline(-3.,-2.)(5.,3.)
\psline[linewidth=2.pt](5.,3.)(5.,1.)
\pscustom[linecolor=sqsqsq,fillcolor=sqsqsq,fillstyle=solid,opacity=0.1]{
\parametricplot{0.0}{0.3587706702705722}{1.1*cos(t)+-3.|1.1*sin(t)+-2.}
\lineto(-3.,-2.)\closepath}
\pscustom[linecolor=wqwqwq,fillcolor=wqwqwq,fillstyle=solid,opacity=0.1]{
\parametricplot{0.0}{0.5585993153435624}{1.6*cos(t)+-3.|1.6*sin(t)+-2.}
\lineto(-3.,-2.)\closepath}
\parametricplot{7.853981633974483}{9.42477796076938}{0.6*cos(t)+5.|0.6*sin(t)+-2.}
\psellipse*[fillcolor=black,fillstyle=solid,opacity=1](4.750432900757689,-1.7504329007576886)(0.04,0.04)
\begin{scriptsize}
\psdots[dotstyle=*,linecolor=gray](-3.,-2.)
\rput[bl](-3.45,-2.5){\large $A$}
\psdots[dotstyle=*,linecolor=gray](5.,-2.)
\rput[bl](5.08,-2.5){\large$B$}
\rput[bl](0.98,-2.32){\large$e$}
\psdots[dotstyle=*,linecolor=gray](5.,3.)
\rput[bl](5.22,2.96){\large$D$}
\psdots[dotstyle=*,linecolor=gray](5.,1.)
\rput[bl](5.26,0.88){\large$C$}
\rput[bl](5.18,1.8){\large$h$}
\rput[bl](-2.3,-1.95){\large $\alpha$}
\rput[bl](-1.8,-1.98){\large $\beta$}
\end{scriptsize}
\end{pspicture*}}
\end{center}


Berechne die H�he $h$ des des unzug�nglichen Steilwandst�cks in Metern.

\antwort{\leer

M�gliche Vorgehensweise: \\
$\tan(\alpha)=\dfrac{\overline{BC}}{e} \Rightarrow \overline{BC} \approx 2,67\,m$ 

$\tan(\beta)=\dfrac{\overline{BD}}{e} \Rightarrow \overline{BD} \approx 4,69\,m$ 

$h=\overline{BD} - \overline{BC} \approx 2,02 \,m$ \leer

L�sungsschl�ssel:

Ein Punkt f�r die richtige L�sung, wobei die Einheit "`m"' nicht angegeben sein muss.
Die Aufgabe ist auch dann als richtig gel�st zu werten, wenn bei korrektem Ansatz das Ergebnis
aufgrund eines Rechenfehlers nicht richtig ist.
Toleranzintervall: $[2\,m;~2,1\,m]$



} 
\end{beispiel}