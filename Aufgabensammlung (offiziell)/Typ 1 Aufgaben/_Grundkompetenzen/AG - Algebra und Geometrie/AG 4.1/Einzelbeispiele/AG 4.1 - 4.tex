\section{AG 4.1 - 4 Dennis Tito - OA - BIFIE}

\begin{beispiel}[AG 4.1]{1} %PUNKTE DES BEISPIELS
Dennis Tito, der 2001 als erster Weltraumtourist unterwegs war, sah die Erdoberfl�che unter einem Sehwinkel von $142^\circ$.

\begin{center}
\newrgbcolor{ttqqqq}{0.2 0. 0.}
\newrgbcolor{qqwuqq}{0. 0.39215686274509803 0.}
\psset{xunit=1.0cm,yunit=1.0cm,algebraic=true,dimen=middle,dotstyle=o,dotsize=5pt 0,linewidth=0.8pt,arrowsize=3pt 2,arrowinset=0.25}
\begin{pspicture*}(-2.560695078849563,-0.016827320585059646)(4.4037428643082865,6.077055879678049)
\psline(1.,5.)(1.,1.)
\pscustom[linecolor=qqwuqq,fillcolor=qqwuqq,fillstyle=solid,opacity=0.1]{
\parametricplot{-1.5707963267948966}{-0.6806784082777886}{0.46636861226503457*cos(t)+1.|0.46636861226503457*sin(t)+5.}
\lineto(1.,5.)\closepath}
\psline(1.,5.)(4.108583845827884,2.48271843580065)
\psline(1.,5.)(-2.1085838458278836,2.48271843580065)
\psline(1.1547081988618701,4.693495663291773)(1.7920786356240843,5.299774859236317)
\psline[linewidth=2.8pt](1.,4.109485010682683)(1.,5.)
\psline(1.,1.)(3.0144696775797484,3.368714619740426)
\psline(-0.9562952014676114,3.4158233816355184)(1.,1.)
\parametricplot{0.3945323246430148}{2.6865242036312766}{1.*3.109485010682683*cos(t)+0.*3.109485010682683*sin(t)+1.|0.*3.109485010682683*cos(t)+1.*3.109485010682683*sin(t)+1.}
\pscustom[linecolor=qqwuqq,fillcolor=qqwuqq,fillstyle=solid,opacity=0.1]{
\parametricplot{2.4609142453120048}{4.031710572106901}{0.46636861226503457*cos(t)+3.0144696775797484|0.46636861226503457*sin(t)+3.368714619740426}
\lineto(3.0144696775797484,3.368714619740426)\closepath}
\psellipse*[linecolor=qqwuqq,fillcolor=qqwuqq,fillstyle=solid,opacity=1](2.7416380327502816,3.3400388583543483)(0.03109124081766897,0.03109124081766897)
\pscustom[linecolor=qqwuqq,fillcolor=qqwuqq,fillstyle=solid,opacity=0.1]{
\parametricplot{-0.890117918517108}{0.6806784082777886}{0.46636861226503457*cos(t)+-0.9562952014676114|0.46636861226503457*sin(t)+3.4158233816355184}
\lineto(-0.9562952014676114,3.4158233816355184)\closepath}
\psellipse*[linecolor=qqwuqq,fillcolor=qqwuqq,fillstyle=solid,opacity=1](-0.6834635566381446,3.387147620249441)(0.03109124081766897,0.03109124081766897)
\rput[tl](1.9008979784859343,5.797234712319029){$71^\circ$}
\begin{scriptsize}
\psdots[dotsize=3pt 0,dotstyle=*,linecolor=ttqqqq](1.,5.)
\psdots[dotsize=3pt 0,dotstyle=*,linecolor=ttqqqq](1.,1.)
\rput[bl](0.9059782723205274,0.6205431161771533){\ttqqqq{$M$}}
\psdots[dotsize=3pt 0,dotstyle=*,linecolor=darkgray](1.,4.109485010682683)
\rput[bl](0.7816133090498515,4.33594639388859){$h$}
\rput[bl](2.1962647662537895,2.0351945733810894){$r$}
\end{scriptsize}
\end{pspicture*}
\end{center}

Berechne, wie hoch $(h)$ �ber der Erdoberfl�che sich Dennis Tito befand, wenn vereinfacht die Erde als Kugel mit einem Radius $r=6\,370\,km$ angenommen wird.
Gib das Ergebnis auf ganze Kilometer gerundet an!
\leer

\antwort{$sin(71^\circ)=\dfrac{r}{r+h}$

$r+h=\dfrac{r}{sin(71^\circ)}$

$h=\dfrac{r}{sin(71^\circ)}-r$

$h=6\,737,004-6\,370=367,044$

Dennis Tito befand sich (in diesem Augenblick rund $367\,km$ �ber der Erdoberfl�che.

L�sungsintervall: $[367;368]$}
\end{beispiel}