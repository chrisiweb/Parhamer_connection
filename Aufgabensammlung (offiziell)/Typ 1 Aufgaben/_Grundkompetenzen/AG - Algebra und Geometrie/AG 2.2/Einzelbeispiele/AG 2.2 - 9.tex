\section{AG 2.2 - 9 - MAT - Fahrtzeit - OA - Matura 2016/17 2. NT}

\begin{beispiel}{1} %PUNKTE DES BEISPIELS
Um 8:00 Uhr f�hrt ein G�terzug von Salzburg in Richtung Linz ab. Vom 124\,km entfernten Bahnhof Linz f�hrt eine halbe Stunde sp�ter ein Schnellzug Richtung Salzburg ab. Der G�terzug bewegt sich mit einer mittleren Geschwindigkeit von 100\,km/h, die mittlere Geschwindigkeit des Schnellzugs ist 150\,km/h. \leer

Mit $t$ wird die Fahrzeit des G�terzugs in Stunden bezeichnet, die bis zur Begegnung der beiden Z�ge vergeht. Gib eine Gleichung f�r die Berechnung der Fahrzeit $t$ des G�terzugs an und berechne diese Fahrzeit!

\antwort{M�gliche Gleichung:

$100\cdot t +150 \cdot (t-0,5)=124$ \\
$t=0,796 \Rightarrow t \approx 0,8 h$\leer

Toleranzintervall: [0,7 h; 0,8 h]}				
\end{beispiel}