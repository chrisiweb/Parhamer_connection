\section{AG 3.4 - 17 Parallele Geraden - OA - Matura 2013/14 Haupttermin}

\begin{beispiel}[AG 3.4]{1} %PUNKTE DES BEISPIELS
				Gegeben sind Gleichungen der Geraden $g$ und $h$. Die beiden Geraden sind nicht ident.
				
				$g$: $y=-\frac{x}{4}+8$
				
				$h$: $X=\Vek{4}{3}{}+s\cdot \Vek{4}{-1}{}$ mit $s\in\mathbb{R}$
				
				Begr�nde, warum diese beiden Gerade parallel zueinander liegen!\leer
				
				\antwort{Parallele Geraden haben die gleiche Steigung bzw. parallele Richtungvektoren.
				
				$k_g=-\frac{1}{4}$
				
				$\vec{a}_h=\Vek{4}{-1}{}||\Vek{1}{-\frac{1}{4}}{}$ und aus $\vec{a}=\Vek{1}{k}{}$ folgt $k_h=k_g$
				
				oder:
				
				$g:$ $X=\Vek{4}{7}{}+t\Vek{4}{-1}{}$, $t\in\mathbb{R}$
				
				$\Vek{4}{-1}{}=\Vek{4}{-1}{}$ \hspace*{1cm} Somit ist $\vec{a}_g=\vec{a}_h$.
				
				Oder:
				
				Auch eine Begr�ndung mit Normalvektoren ist m�glich.
				
				$g$: $x+4y=32$\\
				$h$: $x+4y=16$
				
				Somit ist $\vec{n}_g||\vec{n}_h$.
				
				oder:
				
				$\vec{n}_g\cdot \vec{a}_h=0$}
\end{beispiel}