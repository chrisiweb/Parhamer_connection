\section{AG 3.4 - 18 Parameterdarstellung von Geraden - MC - Matura 2013/14 1. Nebentermin}

\begin{beispiel}[AG 3.4]{1} %PUNKTE DES BEISPIELS
				Gegeben ist eine Gerade $g$:
				
				$g$: $X=\Vek{4}{1}{2}+s\cdot\Vek{2}{-3}{1}$ mit $s\in\mathbb{R}$
				
				Welche der folgenden Geraden $h_i\,(i=1,2,...,5)$ mit $t_i\in\mathbb{R}\,(i=1,2,...,5)$ sind parallel zu $g$?\\
				Kreuze die beiden zutreffenden Antworten an!\leer
				
				\multiplechoice[5]{  %Anzahl der Antwortmoeglichkeiten, Standard: 5
								L1={$h_1$: $X=\Vek{8}{2}{4}+t_1\cdot\Vek{-3}{1}{2}$},   %1. Antwortmoeglichkeit 
								L2={$h_2$: $X=\Vek{3}{4}{-7}+t_2\cdot\Vek{4}{-6}{2}$},   %2. Antwortmoeglichkeit
								L3={$h_3$: $X=\Vek{4}{1}{2}+t_3\cdot\Vek{-2}{1}{-2}$},   %3. Antwortmoeglichkeit
								L4={$h_4$: $X=\Vek{3}{5}{-1}+t_4\cdot\Vek{-2}{3}{-1}$},   %4. Antwortmoeglichkeit
								L5={$h_5$: $X=\Vek{1}{2}{4}+t_5\cdot\Vek{1}{2}{-3}$},	 %5. Antwortmoeglichkeit
								L6={},	 %6. Antwortmoeglichkeit
								L7={},	 %7. Antwortmoeglichkeit
								L8={},	 %8. Antwortmoeglichkeit
								L9={},	 %9. Antwortmoeglichkeit
								%% LOESUNG: %%
								A1=2,  % 1. Antwort
								A2=4,	 % 2. Antwort
								A3=0,  % 3. Antwort
								A4=0,  % 4. Antwort
								A5=0,  % 5. Antwort
								}
\end{beispiel}