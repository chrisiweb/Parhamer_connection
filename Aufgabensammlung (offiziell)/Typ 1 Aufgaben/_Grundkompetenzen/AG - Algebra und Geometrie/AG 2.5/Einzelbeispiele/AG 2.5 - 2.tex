\section{AG 2.5 - 2 Gleichungssysteme - MC - BIFIE}

\begin{beispiel}[AG 2.5]{1} %PUNKTE DES BEISPIELS
		Gegeben sind Aussagen �ber die L�sbarkeit von verschiedenen linearen Gleichungssystemen mit zwei Unbekannten $x$ und $y$.	
			\leer
			
Kreuze die zutreffende(n) Aussage(n) an!

\multiplechoice[5]{  %Anzahl der Antwortmoeglichkeiten, Standard: 5
				L1={\begin{tabular}{ccc}
				\multirow{2}{*}{Das Gleichungssystem} & I: $x+y=2$ & \multirow{2}{*}{hat genau eine L�sung.} \\
				&II: $x-4y=2$& \\
				\end{tabular}
				},   %1. Antwortmoeglichkeit 
				L2={\begin{tabular}{clr}
				\multirow{2}{*}{Das Gleichungssystem} & I: $-x+4y=-2$ & hat unendlich viele \\
				&II: $x-4y=2$& L�sungen. \\
				\end{tabular}},   %2. Antwortmoeglichkeit
				L3={\begin{tabular}{clc}
				\multirow{2}{*}{Das Gleichungssystem} & I: $x+y=62$ & \multirow{2}{*}{hat genau zwei L�sungen.} \\
				&II: $x-4y=-43$& \\
				\end{tabular}},   %3. Antwortmoeglichkeit
				L4={\begin{tabular}{clc}
				\multirow{2}{*}{Das Gleichungssystem} & I: $x-y=1$ & \multirow{2}{*}{hat genau eine L�sung.} \\
				&II: $-x+y=2$& \\
				\end{tabular}},   %4. Antwortmoeglichkeit
				L5={\begin{tabular}{clc}
				\multirow{2}{*}{Das Gleichungssystem} & I: $x+y=62$ & \multirow{2}{*}{hat keine L�sung.} \\
				&II: $x+y=-43$& \\
				\end{tabular}},	 %5. Antwortmoeglichkeit
				L6={},	 %6. Antwortmoeglichkeit
				L7={},	 %7. Antwortmoeglichkeit
				L8={},	 %8. Antwortmoeglichkeit
				L9={},	 %9. Antwortmoeglichkeit
				%% LOESUNG: %%
				A1=1,  % 1. Antwort
				A2=2,	 % 2. Antwort
				A3=5,  % 3. Antwort
				A4=0,  % 4. Antwort
				A5=0,  % 5. Antwort
				}			
\end{beispiel}