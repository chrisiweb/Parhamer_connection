\section{AG 2.5 - 5 Gleichungssystem  - LT - Matura 2014/15 - Nebentermin 1}

\begin{beispiel}[AG 2.5]{1} %PUNKTE DES BEISPIELS
Eine Teilmenge der L�sungsmenge einer linearen Gleichung wird durch die nachstehende Abbildung dargestellt. Die durch die Gleichung beschriebene Gerade $g$ verl�uft durch die Punkte $P_1$
und $P_2$, deren Koordinaten jeweils ganzzahlig sind.

\begin{center}
\resizebox{0.6\linewidth}{!}{\psset{xunit=1.0cm,yunit=1.0cm,algebraic=true,dimen=middle,dotstyle=o,dotsize=5pt 0,linewidth=0.8pt,arrowsize=3pt 2,arrowinset=0.25}
\begin{pspicture*}(-2.3850686959188705,-1.269991271905034)(7.4557535471398175,6.5189996127831185)
\multips(0,-1)(0,1.0){8}{\psline[linestyle=dashed,linecap=1,dash=1.5pt 1.5pt,linewidth=0.4pt,linecolor=lightgray]{c-c}(-2.3850686959188705,0)(7.4557535471398175,0)}
\multips(-2,0)(1.0,0){10}{\psline[linestyle=dashed,linecap=1,dash=1.5pt 1.5pt,linewidth=0.4pt,linecolor=lightgray]{c-c}(0,-1.269991271905034)(0,6.5189996127831185)}
\psaxes[labelFontSize=\scriptstyle,xAxis=true,yAxis=true,Dx=1.,Dy=1.,ticksize=-2pt 0,subticks=2]{->}(0,0)(-2.3850686959188705,-1.269991271905034)(7.4557535471398175,6.5189996127831185)[x,140] [y,-40]
\psplot[linewidth=1.2pt,plotpoints=200]{-2.3850686959188705}{7.4557535471398175}{-0.5*x+3.0}
\begin{scriptsize}
\rput[bl](-1.887051375926022,4.248040633615728){$g$}
\psdots[dotsize=3pt 0,dotstyle=*,linecolor=darkgray](0.,3.)
\rput[bl](0.08509721124565872,3.1125611440320338){\darkgray{$P_1$}}
\psdots[dotsize=3pt 0,dotstyle=*,linecolor=darkgray](6.,0.)
\rput[bl](6.041384358360127,0.22406068807351195){\darkgray{$P_2$}}
\end{scriptsize}
\end{pspicture*}}
\end{center}\leer

Die lineare Gleichung und eine zweite Gleichung bilden ein lineares Gleichungssystem.

\lueckentext{
				text={Hat die zweite lineare Gleichung die Form \gap, so \gap.}, 	%Lueckentext Luecke=\gap
				L1={$2x+y=1$}, 		%1.Moeglichkeit links  
				L2={$x+2y=8$}, 		%2.Moeglichkeit links
				L3={$y=5$}, 		%3.Moeglichkeit links
				R1={hat das Gleichungssystem unendlich viele L�sungen}, 		%1.Moeglichkeit rechts 
				R2={ist die L�sungsmenge des Gleichungssystem $L=\{(-2|4)\}$}, 		%2.Moeglichkeit rechts
				R3={hat das Gleichungssystem keine L�sung}, 		%3.Moeglichkeit rechts
				%% LOESUNG: %%
				A1=2,   % Antwort links
				A2=3		% Antwort rechts 
				}
\end{beispiel}