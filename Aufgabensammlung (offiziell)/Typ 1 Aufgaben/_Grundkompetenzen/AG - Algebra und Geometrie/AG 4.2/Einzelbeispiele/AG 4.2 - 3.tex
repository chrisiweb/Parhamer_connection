\section{AG 4.2 - 3 Cosinus im Einheitskreis - OA - BIFIE}

\begin{beispiel}[AG 4.2]{1} %PUNKTE DES BEISPIELS
				Zeichne im Einheitskreis alle Winkel aus $\left[0^\circ; 360^\circ\right]$ ein, f�r die $\cos(\beta)=0,4$ gilt!
Achte auf die Kennzeichnung der Winkel durch Winkelb�gen.
\leer

\resizebox{1\linewidth}{!}{\psset{xunit=1.0cm,yunit=1.0cm,algebraic=true,dimen=middle,dotstyle=o,dotsize=5pt 0,linewidth=0.8pt,arrowsize=3pt 2,arrowinset=0.25}
\begin{pspicture*}(-13.126446036243554,-12.112437968495003)(12.316524990288347,12.23340155238512)
\multips(0,-12)(0,1.0){25}{\psline[linestyle=dashed,linecap=1,dash=1.5pt 1.5pt,linewidth=0.4pt,linecolor=lightgray]{c-c}(-13.126446036243554,0)(12.316524990288347,0)}
\multips(-13,0)(1.0,0){26}{\psline[linestyle=dashed,linecap=1,dash=1.5pt 1.5pt,linewidth=0.4pt,linecolor=lightgray]{c-c}(0,-12.112437968495003)(0,12.23340155238512)}
\psaxes[labelFontSize=\scriptstyle,xAxis=true,yAxis=true,labels=none,Dx=1.,Dy=1.,ticksize=-2pt 0,subticks=2]{}(0,0)(-13.126446036243554,-12.112437968495003)(12.316524990288347,12.23340155238512)
\pscircle(0.,0.){10.}
\rput[tl](0.4570703228905204,10.66607282786494){1}
\rput[tl](10.17450894904028,-0.35747253459365624){1}
\rput[tl](0.24809314813461159,-10.12266633580143){-1}
\rput[tl](-10.984429994995487,0.6351689909357908){-1}
\rput[tl](0.4048260292015432,5.3371551644963295){0,5}
\rput[tl](4.479880936941766,-0.4619611162283349){0,5}
\rput[tl](0.5615589102684748,-4.641504381615481){-0,5}
\rput[tl](-5.864489213475721,-0.35747253459365624){-0,5}
\antwort{
\psline[linestyle=dashed,dash=6pt 6pt](4.,-9.16515138991168)(4.,9.16515138991168)
\psline(0.,0.)(4.,9.16515138991168)
\psline(0.,0.)(4.,-9.16515138991168)
\parametricplot{0.0}{1.1592794807274085}{1.*3.*cos(t)+0.*3.*sin(t)+0.|0.*3.*cos(t)+1.*3.*sin(t)+0.}
\parametricplot{0.0}{5.1239058264521775}{1.*4.*cos(t)+0.*4.*sin(t)+0.|0.*4.*cos(t)+1.*4.*sin(t)+0.}}
\begin{scriptsize}
\psdots[dotsize=3pt 0,dotstyle=*](0.,0.)
\end{scriptsize}
\end{pspicture*}}
\end{beispiel}