\section{AG 3.3 - 15 Trapez - OA - Matura NT 2 15/16}

\begin{beispiel}[AG 3.3]{1} %PUNKTE DES BEISPIELS
Von einem Trapez $ABCD$ sind die Koordinaten der Eckpunkte gegeben:

$A=(2/-6), B=(10/-2), C=(9/2), D=(3/y)$.

Die Seiten $a=AB$ und $c=CD$ sind zueinander parallel.

\newrgbcolor{xdxdff}{0.49019607843137253 0.49019607843137253 1.}
\psset{xunit=1.0cm,yunit=1.0cm,algebraic=true,dimen=middle,dotstyle=o,dotsize=5pt 0,linewidth=0.8pt,arrowsize=3pt 2,arrowinset=0.25}
\begin{pspicture*}(-2.64,1.04)(4.66,5.88)
\psline(-2.,2.)(4.,2.)
\psline(4.,2.)(3.,5.)
\psline(3.,5.)(0.,5.)
\psline(0.,5.)(-2.,2.)
\begin{scriptsize}
\psdots[dotsize=3pt 0,dotstyle=*,linecolor=blue](-2.,2.)
\rput[bl](-2.28,1.52){\blue{$A$}}
\psdots[dotsize=3pt 0,dotstyle=*,linecolor=blue](4.,2.)
\rput[bl](4.18,1.56){\blue{$B$}}
\rput[bl](1.,1.7){a}
\psdots[dotsize=3pt 0,dotstyle=*,linecolor=blue](3.,5.)
\rput[bl](3.1,5.04){\blue{$C$}}
\rput[bl](3.64,3.26){b}
\psdots[dotsize=3pt 0,dotstyle=*,linecolor=xdxdff](0.,5.)
\rput[bl](-0.38,5.08){\xdxdff{$D$}}
\rput[bl](1.46,5.12){c}
\rput[bl](-1.28,3.68){d}
\end{scriptsize}
\end{pspicture*}

Gib den Wert der Koordinate $y$ des Punkts $D$ an!

$y=$ \antwort[\rule{3cm}{0.3pt}]{-1}
\end{beispiel}