\section{FA 5.5 - 3 Halbwertszeit eines Isotops - MC - BIFIE}

\begin{beispiel}[FA 5.5]{1} %PUNKTE DES BEISPIELS
Der radioaktive Zerfall des Iod-Isotops ${131}^\text{I}$ verh�lt sich gem�� der Funktion $N$ mit $N(t)=N(0)\cdot e^{-0,086\cdot t}$ mit $t$ in Tagen. 

\leer

Kreuze diejenige(n) Gleichung(en) an, mit der/denen die Halbwertszeit des Isotops in Tagen berechnet werden kann.

\multiplechoice[5]{  %Anzahl der Antwortmoeglichkeiten, Standard: 5
				L1={$\ln\left(\frac{1}{2}\right)=-0,086 \cdot t \cdot \ln e$},   %1. Antwortmoeglichkeit 
				L2={$2=e^{-0,086\cdot t}$},   %2. Antwortmoeglichkeit
				L3={$N(0)=\frac{N(0)}{2}\cdot e^{-0,086\cdot t}$},   %3. Antwortmoeglichkeit
				L4={$\ln\left(\frac{1}{2}\right)=-\ln 0,086 \cdot t \cdot e$},   %4. Antwortmoeglichkeit
				L5={$\frac{1}{2}=1 \cdot e^{-0,086 \cdot t}$},	 %5. Antwortmoeglichkeit
				L6={},	 %6. Antwortmoeglichkeit
				L7={},	 %7. Antwortmoeglichkeit
				L8={},	 %8. Antwortmoeglichkeit
				L9={},	 %9. Antwortmoeglichkeit
				%% LOESUNG: %%
				A1=1,  % 1. Antwort
				A2=5,	 % 2. Antwort
				A3=0,  % 3. Antwort
				A4=0,  % 4. Antwort
				A5=0,  % 5. Antwort
				}
\end{beispiel}