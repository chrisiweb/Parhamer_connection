\section{FA 5.5 - 2 Halbwertszeit von Felbamat - OA - BIFIE}

\begin{beispiel}[FA 5.5]{1} %PUNKTE DES BEISPIELS
Zur Behandlung von Epilepsie wird oft der Arzneistoff Felbamat eingesetzt. Nach der Einnahme einer Ausgangsdosis $D_0$ nimmt die Konzentration $D$ von Felbamat im K�rper n�herungsweise exponentiell mit der Zeit ab. \leer

F�r $D$ gilt folgender funktionaler Zusammenhang: $D(t) = D_0 \cdot 0,9659^t$.
Dabei wird die Zeit $t$ in Stunden gemessen. 

\leer

Berechne die Halbwertszeit von Felbamat! Gib die L�sung auf Stunden gerundet an.

\leer

\antwort{$\frac{D_0}{2}=D_0 \cdot 0,9659^t$

$\frac{1}{2}=0,9659^t$ 

$\ln(0,5)=t\cdot \ln(0,9659)$ 

$\Rightarrow \dfrac{\ln(0,5)}{\ln(0,9659)} \approx$ 20 Stunden
}
\end{beispiel}