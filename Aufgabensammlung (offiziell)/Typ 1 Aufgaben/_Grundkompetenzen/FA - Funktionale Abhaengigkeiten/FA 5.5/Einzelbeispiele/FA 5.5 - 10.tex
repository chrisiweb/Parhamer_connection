\section{FA 5.5 - 10 - Halbwertszeit - OA - Matura - 1. NT 2017/18}

\begin{beispiel}[FA 5.5]{1}
Die Masse $m(t)$ einer radioaktiven Substanz kann durch eine Exponentialfunktion $m$ in Abh�ngigkeit von der Zeit $t$ beschrieben werden.\\
Zu Beginn einer Messung sind 100\,mg der Substanz vorhanden, nach vier Stunden misst man noch 75\,mg dieser Substanz.

Bestimme die Halbwertszeit $t_H$ dieser radioaktiven Substanz in Stunden!

\antwort{$t_H\approx 9,64$\,Stunden}
\end{beispiel}