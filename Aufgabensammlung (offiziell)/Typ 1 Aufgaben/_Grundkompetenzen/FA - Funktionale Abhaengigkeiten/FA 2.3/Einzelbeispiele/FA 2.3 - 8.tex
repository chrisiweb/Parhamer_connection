\section{FA 2.3 - 8 Steigung des Graphen einer linearen Funktion - MC - Matura 2013/14 1. Nebentermin}

\begin{beispiel}[FA 2.3]{1} %PUNKTE DES BEISPIELS
				In der untenstehenden Graphik sind drei Geraden $g_1,g_2$ und $g_3$ dargestellt. Es gilt:
				
				$g_1$: $y=k_1\cdot x+d_1$\\
				$g_2$: $y=k_2\cdot x+d_2$\\
				$g_3$: $y=k_3\cdot x+d_3$
				
				\begin{center}\resizebox{0.7\linewidth}{!}{\psset{xunit=1.0cm,yunit=1.0cm,algebraic=true,dimen=middle,dotstyle=o,dotsize=5pt 0,linewidth=0.8pt,arrowsize=3pt 2,arrowinset=0.25}
\begin{pspicture*}(-3.7,-3.6)(5.94,4.74)
\multips(0,-3)(0,1.0){9}{\psline[linestyle=dashed,linecap=1,dash=1.5pt 1.5pt,linewidth=0.4pt,linecolor=lightgray]{c-c}(-3.7,0)(5.94,0)}
\multips(-3,0)(1.0,0){10}{\psline[linestyle=dashed,linecap=1,dash=1.5pt 1.5pt,linewidth=0.4pt,linecolor=lightgray]{c-c}(0,-3.6)(0,4.74)}
\psaxes[labelFontSize=\scriptstyle,xAxis=true,yAxis=true,Dx=1.,Dy=1.,ticksize=-2pt 0,subticks=2]{->}(0,0)(-3.7,-3.6)(5.94,4.74)[x,140] [y,-40]
\psplot{-3.7}{5.94}{(--3.--1.*x)/3.}
\psplot{-3.7}{5.94}{(-0.-4.*x)/-3.}
\psplot{-3.7}{5.94}{(-2.-1.*x)/2.}
\rput[tl](3.12,4.04){$g_1$}
\rput[tl](4.68,2.46){$g_2$}
\rput[tl](4.2,-2.6){$g_3$}
\end{pspicture*}}\end{center}\leer

Kreuze die beiden zutreffenden Aussagen an!\leer

\multiplechoice[5]{  %Anzahl der Antwortmoeglichkeiten, Standard: 5
				L1={$k_1<k_2$},   %1. Antwortmoeglichkeit 
				L2={$d_3>d_2$},   %2. Antwortmoeglichkeit
				L3={$k_2>k_3$},   %3. Antwortmoeglichkeit
				L4={$k_3<k_1$},   %4. Antwortmoeglichkeit
				L5={$d_1<d_3$},	 %5. Antwortmoeglichkeit
				L6={},	 %6. Antwortmoeglichkeit
				L7={},	 %7. Antwortmoeglichkeit
				L8={},	 %8. Antwortmoeglichkeit
				L9={},	 %9. Antwortmoeglichkeit
				%% LOESUNG: %%
				A1=3,  % 1. Antwort
				A2=4,	 % 2. Antwort
				A3=0,  % 3. Antwort
				A4=0,  % 4. Antwort
				A5=0,  % 5. Antwort
				}
\end{beispiel}