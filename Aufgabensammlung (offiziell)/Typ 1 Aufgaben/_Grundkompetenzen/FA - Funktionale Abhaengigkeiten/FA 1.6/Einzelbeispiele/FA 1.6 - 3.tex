\section{FA 1.6 - 3 Kosten, Erl�s und Gewinn - OA - Matura 2015/16 - Haupttermin}

\begin{beispiel}[FA 1.6]{1} %PUNKTE DES BEISPIELS
Die Funktion $E$ beschreibt den Erl�s (in \euro) beim Absatz von $x$ Mengeneinheiten eines Produkts. Die Funktion $G$ beschreibt den dabei erzielten Gewinn in \euro. Dieser ist definiert als Differenz "`Erl�s - Kosten"'.\leer

Erg�nze die nachstehende Abbildung durch den Graphen der zugeh�rigen Kostenfunktion $K$!
Nehmen Sie dabei $K$ als linear an! (Die L�sung der Aufgabe beruht auf der Annahme, dass alle
produzierten Mengeneinheiten des Produkts verkauft werden.)
\leer


\begin{center}
\psset{xunit=1.0cm,yunit=1.0cm,algebraic=true,dimen=middle,dotstyle=o,dotsize=5pt 0,linewidth=0.8pt,arrowsize=3pt 2,arrowinset=0.25}
\begin{pspicture*}(-0.78,-2.3)(11.98,8.62)
\psaxes[labelFontSize=\scriptstyle,xAxis=true,yAxis=true,labels=none,Dx=1.,Dy=1.,ticksize=0pt 0,subticks=0]{->}(0,0)(-0.78,-2.3)(11.98,8.62)
\psplot[plotpoints=200]{0}{11.979999999999976}{-0.05333333333333334*x^(2.0)+0.6933333333333334*x-1.6}
\psplot[linewidth=1.2pt,plotpoints=200]{0}{11.979999999999976}{-0.05707476384525304*x^(2.0)+1.2870410129151044*x}
\antwort{\psline[linewidth=2.pt,linestyle=dashed,dash=5pt 5pt,linecolor=red](2.981292969413772,3.32975967117698)(3.,0.)
\psline[linewidth=2.pt,linestyle=dashed,dash=5pt 5pt,linecolor=red](9.98,7.16)(10.,0.)
\psplot[linewidth=2.pt,linecolor=red]{0}{11.98}{(--11.88494385734365--3.8302403288230202*x)/6.998707030586228}}
\rput[tl](5.96,6.58){$E$}
\antwort{\rput[tl](10.68,8.46){$\red{K}$}}
\rput[tl](6.36,1.26){$G$}
\rput[bl](0.2,8){$E(x), G(x), K(x)$}
\rput[bl](11.5,0.2){$x$}
\antwort{
\psdots[dotstyle=*,linecolor=red](3.,0.)
\rput[bl](2.92,-0.64){\red{$X_1$}}
\psdots[dotstyle=*,linecolor=red](10.,0.)
\rput[bl](9.96,-0.58){\red{$X_2$}}
\psdots[dotstyle=*,linecolor=red](9.98,7.16)
\psdots[dotstyle=*,linecolor=red](2.981292969413772,3.32975967117698)}
\end{pspicture*}
\end{center}

\antwort{L�sungsschl�ssel:\\

Ein Punkt ist genau dann zu geben, wenn der Graph einer linearen Kostenfunktion skizziert wurde
und dieser den Graphen der Erl�sfunktion $E$ an den Stellen $x_1$ und $x_2$ schneidet.}

\end{beispiel}