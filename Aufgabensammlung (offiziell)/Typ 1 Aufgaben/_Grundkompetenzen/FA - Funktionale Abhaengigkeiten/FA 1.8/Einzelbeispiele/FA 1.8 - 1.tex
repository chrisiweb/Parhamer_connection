\section{FA 1.8 - 1 Masse - OA - BIFIE}

\begin{beispiel}[FA 1.8]{1} %PUNKTE DES BEISPIELS
Die Masse eines Drehzylinders in Abh�ngigkeit von seinen Abmessungen $r$ und $h$ und seiner Dichte $\rho$ kann durch die Funktion $M$ mit $M(r,h,\rho)=\pi \cdot r�\cdot h\cdot \rho$ beschrieben werden.

Ein aus Fichtenholz geschnitzter Drehzylinder hat den Durchmesser $d=8\,cm$ und die H�he $h=6\,dm$. Die Dichte von Fichtenholz betr�gt ca. $0,5\,g/cm�$.

Gib die Masse des in der Angabe beschriebenen Drehzylinders in Kilogramm an!
\leer

\antwort{$M(4,60,0,5)\approx1\,507,96$

Die Masse des Drehzylinders betr�gt ca. $1,5\,kg$.

Toleranzintervall: $[1,5;1,51]$.}
\end{beispiel}