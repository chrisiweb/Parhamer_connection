\section{FA 1.2 - 4 - MAT - Stefan-Boltzmann-Gesetz - LT - Matura 2016/17 2. NT}

\begin{beispiel}[FA 1.2]{1} %PUNKTE DES BEISPIELS
Die Leuchtkraft $L$ eines Sterns wird durch folgende Formel beschrieben:

$L=4\cdot\pi\cdot R^2\cdot T^4\cdot\sigma$

Dabei ist $R$ der Sternradius und $T$ die Oberfl�chentemperatur des Sterns; $\sigma$ ist eine Konstante (die sogenannte Stefan-Boltzmann-Konstante).

\lueckentext{
				text={F�r verschiedene Sterne mit gleichem, bekanntem Sternradius $R$ ist die Leuchtkraft $L$ eine Funktion \gap; es handelt sich dabei um eine \gap.}, 	%Lueckentext Luecke=\gap
				L1={des Sternradius $R$}, 		%1.Moeglichkeit links  
				L2={der Obenfl�chentemperatur $T$}, 		%2.Moeglichkeit links
				L3={der Konstanten $\sigma$}, 		%3.Moeglichkeit links
				R1={lineare Funktion}, 		%1.Moeglichkeit rechts 
				R2={Potenzfunktion}, 		%2.Moeglichkeit rechts
				R3={Exponentialfunktion}, 		%3.Moeglichkeit rechts
				%% LOESUNG: %%
				A1=2,   % Antwort links
				A2=2		% Antwort rechts 
				}
\end{beispiel}