\section{FA 4.4 - 6 Polynomfunktion - OA - Matura 17/18}

\begin{beispiel}[FA 4.4]{1} %PUNKTE DES BEISPIELS
Die nachstehende Abbildung zeigt den Graphen einer Polynomfunktion $f$.

\begin{center}
	\resizebox{0.8\linewidth}{!}{\psset{xunit=1.0cm,yunit=1.0cm,algebraic=true,dimen=middle,dotstyle=o,dotsize=5pt 0,linewidth=1pt,arrowsize=3pt 2,arrowinset=0.25}
\begin{pspicture*}(-8.799681992451319,-3.6763413598116923)(7.626474677092015,5.572489304922304)
\psaxes[labelFontSize=\scriptstyle,xAxis=true,yAxis=true,Dx=1.,Dy=1.,ticksize=-2pt 0,subticks=0]{->}(0,0)(-8.799681992451319,-3.6763413598116923)(7.626474677092015,5.572489304922304)[x,140] [f(x),-40]
\psplot[linewidth=1.pt,plotpoints=200]{-8.799681992451319}{7.626474677092015}{0.006*(x+7.1)*x^(2.0)*(x-5.1)}
\rput[tl](5.6,2.538172619962255){f}
\end{pspicture*}}
\end{center}

Begr�nde, warum es sich bei der dargestellten Funktion nicht um eine Polynomfunktion dritten Grades handeln kann!\leer

\antwort{M�gliche Begr�ndungen:

Eine Polynomfunktion dritten Grades hat h�chsten zwei lokale Extremstellen. (Die dargestellte Funktion $f$ hat aber mindestens drei lokale Extremstellen.)

oder:

Eine Polynomfunktion dritten Grades hat genau eine Wendestelle. (Die dargestellte Funktion $f$ hat aber mindestens zwei Wendestellen.)

oder:

Die dargestellte Funktion hat bei $x_1\approx -7$ und bei $x_2\approx 5$ jeweils eine Nullstelle und bei $x_3\approx 0$ eine Nullstelle, die auch lokale Extremstelle ist. Damit kann im dargestellten Intervall die Funktionsgleichung in der Form $f(x)=a\cdot(x-x_1)\cdot(x-x_2)\cdot(x-x_3)^2$ mit $a\in\mathbb{R}^+$ angegeben werden. Der Grad von $f$ w�re somit zumindest vier.}

\end{beispiel}