\section{FA 1.5 - 18 Kr�mmungsverhalten einer Polynomfunktion - MC - Matura 2016/17 - Haupttermin}

\begin{beispiel}[FA 1.5]{1} %PUNKTE DES BEISPIELS
Der Graph einer Polynomfunktion dritten Grades hat im Punkt $T = (-3|1)$ ein lokales Minimum, in $H=(-1|3)$ ein lokales Maximum und in $W = (-2|2)$ einen Wendepunkt. \leer

In welchem Intervall ist diese Funktion linksgekr�mmt (positiv gekr�mmt)? 

Kreuze das zutreffende Intervall an! \leer

\multiplechoice[6]{  %Anzahl der Antwortmoeglichkeiten, Standard: 5
				L1={$(-\infty; 2)$},   %1. Antwortmoeglichkeit 
				L2={$(-\infty; -2)$},   %2. Antwortmoeglichkeit
				L3={$(-3; -1)$},   %3. Antwortmoeglichkeit
				L4={$(-2; 2)$},   %4. Antwortmoeglichkeit
				L5={$(-2; \infty)$},	 %5. Antwortmoeglichkeit
				L6={$(3; \infty)$},	 %6. Antwortmoeglichkeit
				L7={},	 %7. Antwortmoeglichkeit
				L8={},	 %8. Antwortmoeglichkeit
				L9={},	 %9. Antwortmoeglichkeit
				%% LOESUNG: %%
				A1=2,  % 1. Antwort
				A2=0,	 % 2. Antwort
				A3=0,  % 3. Antwort
				A4=0,  % 4. Antwort
				A5=0,  % 5. Antwort
				} 
\end{beispiel}