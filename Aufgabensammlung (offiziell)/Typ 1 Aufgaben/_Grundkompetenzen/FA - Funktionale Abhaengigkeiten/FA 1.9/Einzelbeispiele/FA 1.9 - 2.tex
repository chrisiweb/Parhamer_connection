\section{FA 1.9 - 2 Typen mathematischer Funktionen - LT - BIFIE}

\begin{beispiel}[FA 1.9]{1} %PUNKTE DES BEISPIELS
Die nachstehende Tabelle zeigt die Abh�ngigkeit der Gr��e $y$ von $x$.

\begin{longtable}{|c|c|}
\hline
x&y\\
\hline
1&3\\
\hline
2&5\\
\hline
4&9\\
\hline
6&13\\
\hline
\end{longtable}
\leer

\lueckentext{
				text={Die angegebenen Werte k�nnten Funktionswerte einer \gap sein, wie sie eine Gleichung des Typs \gap erf�llen.}, 	%Lueckentext Luecke=\gap
				L1={Potenzfunktion}, 		%1.Moeglichkeit links  
				L2={Exponentialfunktion}, 		%2.Moeglichkeit links
				L3={linearen Funktion}, 		%3.Moeglichkeit links
				R1={$f(x)=k\cdot x+d$}, 		%1.Moeglichkeit rechts 
				R2={$f(x)=a\cdot b^x$}, 		%2.Moeglichkeit rechts
				R3={$f(x)=a\cdot x^{-1}$}, 		%3.Moeglichkeit rechts
				%% LOESUNG: %%
				A1=3,   % Antwort links
				A2=1		% Antwort rechts 
				}
\end{beispiel}