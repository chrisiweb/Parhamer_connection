\section{FA 5.4 - 2 Exponentielles Wachstum - MC - BIFIE}

\begin{beispiel}[FA 5.4]{1} %PUNKTE DES BEISPIELS
Die Funktion $f$ mit $f(x) = 100 \cdot 2^x$ beschreibt einen exponentiellen Wachstumsprozess. Wie ver�ndert sich der Funktionswert, wenn $x$ um $1$ erh�ht wird?

\leer

Kreuze die beiden zutreffenden Aussagen an.

\leer

Der Funktionswert $f(x+1)$ ist \ldots

\multiplechoice[5]{  %Anzahl der Antwortmoeglichkeiten, Standard: 5
				L1={um 1 gr��er als $f(x)$.},   %1. Antwortmoeglichkeit 
				L2={doppelt so gro� wie $f(x)$.},   %2. Antwortmoeglichkeit
				L3={um 100 gr��er als $f(x)$.},   %3. Antwortmoeglichkeit
				L4={um 200 gr��er als $f(x)$.},   %4. Antwortmoeglichkeit
				L5={um 100\% gr��er als $f(x)$.},	 %5. Antwortmoeglichkeit
				L6={},	 %6. Antwortmoeglichkeit
				L7={},	 %7. Antwortmoeglichkeit
				L8={},	 %8. Antwortmoeglichkeit
				L9={},	 %9. Antwortmoeglichkeit
				%% LOESUNG: %%
				A1=2,  % 1. Antwort
				A2=5,	 % 2. Antwort
				A3=0,  % 3. Antwort
				A4=0,  % 4. Antwort
				A5=0,  % 5. Antwort
				} 

\end{beispiel}