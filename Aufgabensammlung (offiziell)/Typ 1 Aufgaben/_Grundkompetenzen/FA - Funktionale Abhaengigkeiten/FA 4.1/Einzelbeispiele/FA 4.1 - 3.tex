\section{FA 4.1 - 3 Parabel - MC - BIFIE}

\begin{beispiel}[FA 4.1]{1} %PUNKTE DES BEISPIELS
				Der Graph einer Polynomfunktion zweiten Grades mit $f(x)=ax�+bx+c$ ist eine Parabel.

\meinlr{Welche Bedingungen m�ssen die Koeffizienten $a,b$ und $c$ jedenfalls erf�llen, damit die Parabel (so wie in der nebenstehenden Skizze) nach unten offen ist und ihren Scheitel auf der y-Achse hat?)}{\begin{center}
	\resizebox{0.7\linewidth}{!}{\psset{xunit=1.0cm,yunit=1.0cm,algebraic=true,dimen=middle,dotstyle=o,dotsize=5pt 0,linewidth=0.8pt,arrowsize=3pt 2,arrowinset=0.25}
\begin{pspicture*}(-3.4926224553042267,-0.8239989411748612)(3.6095622780309817,3.4618021910101917)
\psaxes[labelFontSize=\scriptstyle,xAxis=true,yAxis=true,labels=none,Dx=1.,Dy=1.,ticksize=-2pt 0,subticks=2]{->}(0,0)(-3.4926224553042267,-0.8239989411748612)(3.6095622780309817,3.4618021910101917)[x,140] [y,-40]
\psplot[linewidth=1.2pt,linecolor=green,plotpoints=200]{-3.4926224553042267}{3.6095622780309817}{-x^(2.0)+3.0}
\psplot[linewidth=1.2pt,linecolor=blue,plotpoints=200]{-3.4926224553042267}{3.6095622780309817}{-2.5*x^(2.0)+3.0}
\psplot[linewidth=1.2pt,linecolor=blue,plotpoints=200]{-3.4926224553042267}{3.6095622780309817}{-0.5*x^(2.0)+3.0}
\end{pspicture*}}\end{center}}

Kreuze die beiden zutreffenden Aussagen an!
\multiplechoice[5]{  %Anzahl der Antwortmoeglichkeiten, Standard: 5
				L1={$a<0$},   %1. Antwortmoeglichkeit 
				L2={$a>0$},   %2. Antwortmoeglichkeit
				L3={$b=0$},   %3. Antwortmoeglichkeit
				L4={$b<0$},   %4. Antwortmoeglichkeit
				L5={$c=0$},	 %5. Antwortmoeglichkeit
				L6={},	 %6. Antwortmoeglichkeit
				L7={},	 %7. Antwortmoeglichkeit
				L8={},	 %8. Antwortmoeglichkeit
				L9={},	 %9. Antwortmoeglichkeit
				%% LOESUNG: %%
				A1=1,  % 1. Antwort
				A2=3,	 % 2. Antwort
				A3=0,  % 3. Antwort
				A4=0,  % 4. Antwort
				A5=0,  % 5. Antwort
				}
\end{beispiel}