\section{FA 4.1 - 1 Quadratische Funktion - ZO - BIFIE}

\begin{beispiel}[FA 4.1]{1} %PUNKTE DES BEISPIELS
				Eine quadratische Funktion hat die Funktionsgleichung \begin{center}
	$f(x)=ax�+bx+c$ mit $a,b,c\in\mathbb{R}$ und $a\neq 0$.
\end{center}
Ihr Graph ist eine Parabel.

Ordne den vorgegebenen Bedingungen f�r $a,b$ und $c$ die daraus jedenfalls resultierende Eigenschaft zu!

\zuordnen[-0.27]{
				title1={Eigenschaften}, 		%Titel Antwortmoeglichkeiten
				A={Der Funktionsgraph hat keine Nullstelle.}, 				%Moeglichkeit A  
				B={Der Graph hat mindestens einen Schnittpunkt mit der x-Achse.}, 				%Moeglichkeit B  
				C={Der Scheitelpunkt der Parabel ist ein Hochpunkt.}, 				%Moeglichkeit C  
				D={Der Scheitelpunkt der Parabel ist ein Tiefpunkt.}, 				%Moeglichkeit D  
				E={Der Graph der Funktion ist symmetrisch zur x-Achse.}, 				%Moeglichkeit E  
				F={Der Graph der Funktion ist symmetrisch zur y-Achse.}, 				%Moeglichkeit F  
				title2={Bedingungen},		%Titel Zuordnung
				R1={$a<0$},				%1. Antwort rechts
				R2={$a>0$},				%2. Antwort rechts
				R3={$c=0$},				%3. Antwort rechts
				R4={$b=0$},				%4. Antwort rechts
				%% LOESUNG: %%
				A1={C},				% 1. richtige Zuordnung
				A2={D},				% 2. richtige Zuordnung
				A3={B},				% 3. richtige Zuordnung
				A4={F},				% 4. richtige Zuordnung
				}
\end{beispiel}