\section{FA 2.5 - 1 Modellierung mittels linearer Funktionen - MC - BIFIE}

\begin{beispiel}[FA 2.5]{1} %PUNKTE DES BEISPIELS
Reale Sachverhalte k�nnen durch eine lineare Funktion $f(x)=k\cdot x+d$ mathematisch modelliert werden.

In welchem Sachverhalt ist eine Modellierung mittels einer linearen Funktion sinnvoll m�glich? Kreuze die beiden zutreffenden Sachverhalte an!

\multiplechoice[5]{  %Anzahl der Antwortmoeglichkeiten, Standard: 5
				L1={der zur�ckgelegte Weg in Abh�ngigkeit von der Zeit bei einer gleichbleibenden Geschwindigkeit von $30\,km/h$},   %1. Antwortmoeglichkeit 
				L2={die Einwohnerzahl einer Stadt in Abh�ngigkeit von der Zeit, wenn die Anzahl der Einwohner/innen in einem bestimmten Zeitraum j�hrlich um $3\,\%$ w�chst},   %2. Antwortmoeglichkeit
				L3={Der Fl�cheninhalt eines Quadrates in Abh�ngigkeit von der Seitenl�nge},   %3. Antwortmoeglichkeit
				L4={Die Stromkosten in Abh�ngigkeit von der verbrauchten Energie (in kWh) bei einer monatlichen Grundgeb�hr von \euro $12$ und Kosten von \euro $0,4$ pro kWh},   %4. Antwortmoeglichkeit
				L5={die Fahrzeit in Abh�ngigkeit von der Geschwindigkeit f�r eine bestimmte Entfernung},	 %5. Antwortmoeglichkeit
				L6={},	 %6. Antwortmoeglichkeit
				L7={},	 %7. Antwortmoeglichkeit
				L8={},	 %8. Antwortmoeglichkeit
				L9={},	 %9. Antwortmoeglichkeit
				%% LOESUNG: %%
				A1=1,  % 1. Antwort
				A2=4,	 % 2. Antwort
				A3=0,  % 3. Antwort
				A4=0,  % 4. Antwort
				A5=0,  % 5. Antwort
				}
\end{beispiel}