\section{FA 2.1 - 7 Gleichungssysteme und ihre L�sungsf�lle - OA - Matura 2014/15 - Kompensationspr�fung}

\begin{beispiel}[FA 2.2]{1} %PUNKTE DES BEISPIELS
				Gegeben ist folgende grafische Darstellung:
				\begin{center}
					\resizebox{0.7\linewidth}{!}{\psset{xunit=1.0cm,yunit=1.0cm,algebraic=true,dimen=middle,dotstyle=o,dotsize=5pt 0,linewidth=0.8pt,arrowsize=3pt 2,arrowinset=0.25}
\begin{pspicture*}(-1.42,-2.6)(9.58,4.64)
\multips(0,-2)(0,1.0){8}{\psline[linestyle=dashed,linecap=1,dash=1.5pt 1.5pt,linewidth=0.4pt,linecolor=lightgray]{c-c}(-1.42,0)(9.58,0)}
\multips(-1,0)(1.0,0){12}{\psline[linestyle=dashed,linecap=1,dash=1.5pt 1.5pt,linewidth=0.4pt,linecolor=lightgray]{c-c}(0,-2.6)(0,4.64)}
\psaxes[labelFontSize=\scriptstyle,xAxis=true,yAxis=true,Dx=1.,Dy=1.,ticksize=-2pt 0,subticks=2]{->}(0,0)(-1.42,-2.6)(9.58,4.64)[x,140] [y,-40]
\psplot{-1.42}{9.58}{(--16.-4.*x)/4.}
\psplot{-1.42}{9.58}{(-10.--1.*x)/5.}
\begin{scriptsize}
\psdots[dotsize=3pt 0,dotstyle=*](5.,-1.)
\end{scriptsize}
\end{pspicture*}}
				\end{center}
				
				Gib ein dieser Grafik entsprechendes lineares Gleichungssystem mit den Variablen $x$ und $y$ an.
				
				\antwort{$I:y=-x+4$\\
				$II:y=\frac{1}{5}x-2$\\
				oder\\
				$I:x+y=4$\\
				$II:x-5y=10$}
\end{beispiel}