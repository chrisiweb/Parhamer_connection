\section{FA 5.1 - 4 Exponentialfunktion - OA - Matura 2014/15 - Kompensationspr�fung}

\begin{beispiel}[FA 5.1]{1} %PUNKTE DES BEISPIELS
Die nachstehende Abbildung zeigt den Graphen einer Exponentialfunktion $f$ mit $f(x)=a^{x}$ mit $a\in\mathbb{R}^{+}\backslash\left\{1\right\}$.

\begin{center}
	\resizebox{0.7\linewidth}{!}{\psset{xunit=1.0cm,yunit=1.0cm,algebraic=true,dimen=middle,dotstyle=o,dotsize=5pt 0,linewidth=0.8pt,arrowsize=3pt 2,arrowinset=0.25}
\begin{pspicture*}(-3.72,-1.3)(4.72,7.42)
\multips(0,-1)(0,1.0){9}{\psline[linestyle=dashed,linecap=1,dash=1.5pt 1.5pt,linewidth=0.4pt,linecolor=darkgray]{c-c}(-3.72,0)(4.72,0)}
\multips(-3,0)(1.0,0){9}{\psline[linestyle=dashed,linecap=1,dash=1.5pt 1.5pt,linewidth=0.4pt,linecolor=darkgray]{c-c}(0,-1.3)(0,7.42)}
\psaxes[labelFontSize=\scriptstyle,xAxis=true,yAxis=true,Dx=1.,Dy=1.,ticksize=-2pt 0,subticks=2]{->}(0,0)(-3.72,-1.3)(4.72,7.42)[x,140] [y,-40]
\psplot[linewidth=1.2pt,plotpoints=200]{-3.72}{4.720000000000003}{0.5^(x)}
\begin{scriptsize}
\rput[bl](-2.38,6.24){$f$}
\end{scriptsize}
\end{pspicture*}}
\end{center}

Bestimme den Parameter $a$.\leer

$a=$ \antwort[\rule{5cm}{0.3pt}]{0,5}
\end{beispiel}