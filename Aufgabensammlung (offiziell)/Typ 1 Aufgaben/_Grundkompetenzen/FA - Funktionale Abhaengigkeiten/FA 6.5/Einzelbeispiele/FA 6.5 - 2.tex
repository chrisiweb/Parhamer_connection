\section{FA 6.5 - 2 Zusammenhang zwischen Sinus- und Cosinusfunktion - MC - BIFIE}

\begin{beispiel}[FA 6.5]{1} %PUNKTE DES BEISPIELS
				Die Funktion $\cos(x)$ kann auch durch eine allgemeine Sinusfunktion beschrieben werden.

Welche der nachstehend angef�hrten Sinusfunktionen beschreiben die Funktion $\cos(x)$
Kreuze die beiden zutreffenden Funktionen an!

\multiplechoice[5]{  %Anzahl der Antwortmoeglichkeiten, Standard: 5
				L1={$sin(x+2\pi$},   %1. Antwortmoeglichkeit 
				L2={$sin(x+\frac{\pi}{2})$},   %2. Antwortmoeglichkeit
				L3={$sin\left(\frac{x}{2}-\pi\right)$},   %3. Antwortmoeglichkeit
				L4={$sin\left(\dfrac{x-\pi}{2}\right)$},   %4. Antwortmoeglichkeit
				L5={$sin\left(x-\dfrac{3\pi}{2}\right)$},	 %5. Antwortmoeglichkeit
				L6={},	 %6. Antwortmoeglichkeit
				L7={},	 %7. Antwortmoeglichkeit
				L8={},	 %8. Antwortmoeglichkeit
				L9={},	 %9. Antwortmoeglichkeit
				%% LOESUNG: %%
				A1=2,  % 1. Antwort
				A2=5,	 % 2. Antwort
				A3=0,  % 3. Antwort
				A4=0,  % 4. Antwort
				A5=0,  % 5. Antwort
				}
\end{beispiel}