\section{FA 3.2 - 9 Graphen quadratische Funktionen - OA - Matura 17/18}

\begin{beispiel}[FA 3.2]{1} %PUNKTE DES BEISPIELS
Die nachstehende Abbildung zeigt die Graphen quadratischer Funktionen $f_1,f_2$ und $f_3$ mit den Gleichungen $f_i(x)=a_i\cdot x^2+b_i$, wobei gilt: $a_i,b_i\in\mathbb{R},i\in\{1,2,3\}$.

\begin{center}
\resizebox{0.6\linewidth}{!}{\psset{xunit=1.7cm,yunit=1.7cm,algebraic=true,dimen=middle,dotstyle=o,dotsize=5pt 0,linewidth=1.6pt,arrowsize=3pt 2,arrowinset=0.25}
\begin{pspicture*}(-3.48,-3.52)(3.62,3.64)
\psaxes[labelFontSize=\scriptstyle,xAxis=true,yAxis=true,labels=none,Dx=1.,Dy=1.,ticksize=-4pt 0,subticks=0]{->}(0,0)(-3.48,-3.52)(3.62,3.64)[$x$,140] [$f_i(x)$,-40]
\psplot[linewidth=2.pt,linestyle=dashed,dash=2pt 2pt,plotpoints=200]{-3.4800000000000013}{3.6200000000000014}{2.0*x^(2.0)+0.3}
\rput[tl](0.7,1.){$f_2$}
\psplot[linewidth=2.pt,linestyle=dashed,dash=1pt 2pt 5pt 2pt ,plotpoints=200]{-3.4800000000000013}{3.6200000000000014}{-0.5*x^(2.0)-0.5}
\rput[tl](1.6,-1.26){$f_3$}
\psplot[linewidth=2.pt,plotpoints=200]{-3.4800000000000013}{3.6200000000000014}{0.2*x^(2.0)+1.5}
\rput[tl](2.5,2.7){$f_1$}
\end{pspicture*}}
\end{center}

Ordne die Parameterwerte $a_i$ und $b_i$ jeweils der Gr��e nach, beginnend mit dem kleinsten!\leer

Parameterwerte $a_i$: \antwort[\rule{1.5cm}{0.3pt}\,<\,\rule{1.5cm}{0.3pt}\,<\,\rule{1.5cm}{0.3pt}]{$a_3<a_1<a_2$}\leer

Parameterwerte $b_i$: \antwort[\rule{1.5cm}{0.3pt}\,<\,\rule{1.5cm}{0.3pt}\,<\,\rule{1.5cm}{0.3pt}]{$b_3<b_2<b_1$}
\end{beispiel}