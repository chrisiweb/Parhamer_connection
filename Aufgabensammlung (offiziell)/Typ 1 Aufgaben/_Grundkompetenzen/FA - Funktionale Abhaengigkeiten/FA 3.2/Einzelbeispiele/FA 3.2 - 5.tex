\section{FA 3.2 - 5 Gleichung einer quadratischen Funktion - OA - Matura 2013/14 Haupttermin}

\begin{beispiel}{1} %PUNKTE DES BEISPIELS
			Im nachfolgenden Koordinatensystem ist der Graph einer quadratischen Funktion $f$ mit der Gleichung $f(x)=a\cdot x�+b$ $(a,b\in\mathbb{R})$ dargestellt.
					
			\begin{center}
			\resizebox{0.8\linewidth}{!}{\psset{xunit=1.0cm,yunit=1.0cm,algebraic=true,dimen=middle,dotstyle=o,dotsize=5pt 0,linewidth=0.8pt,arrowsize=3pt 2,arrowinset=0.25}
\begin{pspicture*}(-5.36,-0.84)(5.84,8.54)
\multips(0,0)(0,1.0){10}{\psline[linestyle=dashed,linecap=1,dash=1.5pt 1.5pt,linewidth=0.4pt,linecolor=lightgray]{c-c}(-5.36,0)(5.84,0)}
\multips(-5,0)(1.0,0){12}{\psline[linestyle=dashed,linecap=1,dash=1.5pt 1.5pt,linewidth=0.4pt,linecolor=lightgray]{c-c}(0,-0.84)(0,8.54)}
\psaxes[labelFontSize=\scriptstyle,xAxis=true,yAxis=true,Dx=1.,Dy=1.,ticksize=-2pt 0,subticks=2]{->}(0,0)(-5.36,-0.84)(5.84,8.54)[x,140] [f(x),-40]
\rput[tl](-3.38,5.48){f}
\psplot[linewidth=1.2pt,plotpoints=200]{-5.360000000000003}{5.840000000000001}{0.25*x^(2.0)+2.0}
\end{pspicture*}}
\end{center}

Erg�nze die Werte der Parameter $a$ und $b$! Die f�r die Berechnung relevante Punkte mit ganzzahligen Koordinaten k�nnen dem Diagramm entnommen werden.\leer

$a=$ \antwort[\rule{5cm}{0.3pt}]{$a=\frac{1}{4}$ oder $a=0,25$}\leer

$b=$ \antwort[\rule{5cm}{0.3pt}]{2}
\end{beispiel}