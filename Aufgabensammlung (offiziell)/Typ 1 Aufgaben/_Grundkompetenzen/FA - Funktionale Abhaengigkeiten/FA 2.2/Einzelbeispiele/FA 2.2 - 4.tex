\section{FA 2.2 - 4 Erw�rmung von Wasser - OA - Matura 2015/16 - Haupttermin}

\begin{beispiel}[FA 2.2]{1} %PUNKTE DES BEISPIELS
Bei einem Versuch ist eine bestimmte Wassermenge f�r eine Zeit $t$ auf konstanter Energiestufe in einem Mikrowellenger�t zu erw�rmen. Die Ausgangstemperatur des Wassers und die Temperatur
des Wassers nach 30 Sekunden werden gemessen.

\begin{center}
\begin{tabular}{|l|c|c|} \hline
\cellcolor{gray!30}Zeit (in Sekunden) & $t=0$ & $t=30$ \\ \hline
\cellcolor{gray!30}Temperatur (in �C) & 35,6 & 41,3 \\ \hline
\end{tabular}
\end{center} \leer

Erg�nze die Gleichung der zugeh�rigen linearen Funktion, die die Temperatur $T(t)$ zum Zeitpunkt
$t$ beschreibt. \leer

$T(t) = \rule{4cm}{0.3pt} \cdot t + 35,6$

\antwort{\leer

$T(t) = 0,19 \cdot t + 35,6$}
\end{beispiel}