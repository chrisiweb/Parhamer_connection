\section{FA 2.2 - 3 Steigung einer Geraden - OA - BIFIE}

\begin{beispiel}[FA 2.2]{1} %PUNKTE DES BEISPIELS
Die Gerade $g$ ist durch ihren Graphen dargestellt. Zus�tzlich ist ein Steigungsdreieck eingezeichnet.

\begin{center}
\resizebox{0.5\linewidth}{!}{\psset{xunit=1.0cm,yunit=1.0cm,algebraic=true,dimen=middle,dotstyle=o,dotsize=5pt 0,linewidth=0.8pt,arrowsize=3pt 2,arrowinset=0.25}
\begin{pspicture*}(-11.957973257443935,-5.964568386012204)(1.7389479904028735,9.444468017815433)
\psline[linewidth=1.6pt](-2.,-2.)(-8.,-2.)
\psline[linewidth=1.6pt](-8.,-2.)(-8.,6.)
\psline[linewidth=1.6pt,linecolor=blue](-9.991472453803379,8.655296605071172)(-0.33514297407139093,-4.219809367904812)
\begin{scriptsize}
\rput[bl](-5.560069253515492,-2.990894694045467){\large{b}}
\rput[bl](-8.8941882414782,1.514671505904135){\large{a}}
\rput[bl](-4.749067337524562,2.8663413658890153){\large{\blue{g}}}
\end{scriptsize}
\end{pspicture*}}
\end{center}
Ermittle einen Ausdruck in Abh�ngigkeit von $a$ und $b$ zur Berechnung des Anstiegs $k$!
\leer

$k=\rule{5cm}{0.3pt}$
\leer

\antwort{$k=-\frac{a}{b}$}
\end{beispiel}