\section{FA 2.2 - 8 Radfahrer - MC - Matura 17/18}

\begin{beispiel}[FA 2.2]{1} %PUNKTE DES BEISPIELS
Zwei Radfahrer $A$ und $B$ fahren mit Elektrofahrr�dern vom gleichen Startpunkt aus mit jeweils konstanter Geschwindigkeit auf einer geradlinigen Stra�e in dieselbe Richtung.

In der nachstehenden Abbildung sind die Graphen der Funktionen $s_A$ und $s_B$ dargestellt, die den von den Radfahrern zur�ckgelegt Weg in Abh�ngigkeit von der Fahrzeit beschreiben. Die Markierten Punkte haben die Koordinaten $(0\mid 0), (2\mid 0)$ bzw. $(8\mid 2\,400)$.

\begin{center}
	\resizebox{0.6\linewidth}{!}{\psset{xunit=1.0cm,yunit=0.0025cm,algebraic=true,dimen=middle,dotstyle=o,dotsize=5pt 0,linewidth=1.6pt,arrowsize=3pt 2,arrowinset=0.25}
\begin{pspicture*}(-1.08,-289.5620280474871)(11.78,3928.391513844178)
\multips(0,0)(0,400.0){11}{\psline[linestyle=dashed,linecap=1,dash=1.5pt 1.5pt,linewidth=0.4pt,linecolor=darkgray]{c-c}(0,0)(11.78,0)}
\multips(0,0)(1.0,0){13}{\psline[linestyle=dashed,linecap=1,dash=1.5pt 1.5pt,linewidth=0.4pt,linecolor=darkgray]{c-c}(0,0)(0,3928.391513844178)}
\psaxes[labelFontSize=\scriptstyle,xAxis=true,yAxis=true,Dx=1.,Dy=400.,ticksize=-2pt 0,subticks=2]{->}(0,0)(0.,0.)(11.78,3928.391513844178)
\rput[tl](0.3,3745.0022294141063){$s$ in Metern}
\rput[tl](9.2,221.9975548363996){$t$ in Minuten}
\psplot[linewidth=2.pt]{0.}{11.78}{(-0.--2400.*x)/8.}
\psplot[linewidth=2.pt]{2.}{11.78}{(-4800.--2400.*x)/6.}
\rput[tl](3.56,1438.158072635827){$s_A$}
\rput[tl](5.3,1264.4208558073374){$s_B$}
\begin{scriptsize}
\psdots[dotstyle=*](8.,2400.)
\psdots[dotstyle=*](0.,0.)
\psdots[dotstyle=*](2.,0.)
\end{scriptsize}
\end{pspicture*}}
\end{center}

Kreuze die beiden Aussagen an, die der obigen Abbildung entnommen werden k�nnen!

\multiplechoice[5]{  %Anzahl der Antwortmoeglichkeiten, Standard: 5
				L1={Der Radfahrer $B$ startet zwei Minuten sp�ter als der Radfahrer $A$.},   %1. Antwortmoeglichkeit 
				L2={Die Geschwindigkeit des Radfahrers $A$ betr�gt 200 Meter pro Minute.},   %2. Antwortmoeglichkeit
				L3={Der Radfahrer $B$ holt den Radfahrer $A$ nach einer Fahrstrecke von 2,4 Kilometern ein.},   %3. Antwortmoeglichkeit
				L4={Acht Minuten nach dem Start von Radfahrer $B$ sind die beiden Radfahrer gleich weit vom Startpunkt entfernt.},   %4. Antwortmoeglichkeit
				L5={Vier Minuten nach der Abfahrt des Radfahrers $A$ sind die beiden Radfahrer 200 Meter voneinander entfernt.},	 %5. Antwortmoeglichkeit
				L6={},	 %6. Antwortmoeglichkeit
				L7={},	 %7. Antwortmoeglichkeit
				L8={},	 %8. Antwortmoeglichkeit
				L9={},	 %9. Antwortmoeglichkeit
				%% LOESUNG: %%
				A1=1,  % 1. Antwort
				A2=3,	 % 2. Antwort
				A3=0,  % 3. Antwort
				A4=0,  % 4. Antwort
				A5=0,  % 5. Antwort
				}
\end{beispiel}