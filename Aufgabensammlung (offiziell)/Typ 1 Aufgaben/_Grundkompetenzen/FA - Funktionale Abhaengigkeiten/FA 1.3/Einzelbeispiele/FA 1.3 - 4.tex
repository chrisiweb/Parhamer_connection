\section{FA 1.3 - 4 Zylindervolumen - OA - Matura 2016/17 - Haupttermin}

\begin{beispiel}[FA 1.3]{1} %PUNKTE DES BEISPIELS
Bei einem Drehzylinder wird der Radius des Grundkreises mit $r$ und die H�he des Zylinders mit $h$ bezeichnet. Ist die H�he des Zylinders konstant, dann beschreibt die Funktion $V$ mit
$V(r) =�r^2 \cdot \pi \cdot h$ die Abh�ngigkeit des Zylindervolumens vom Radius. \leer

Im nachstehenden Koordinatensystem ist der Punkt $P=(r_1|V(r_1))$ eingezeichnet. Erg�nze in
diesem Koordinatensystem den Punkt $Q=(3\cdot r_1|V(3\cdot r_1))$. 

\begin{center}
\resizebox{0.65\linewidth}{!}{\psset{xunit=1.0cm,yunit=1.0cm,algebraic=true,dimen=middle,dotstyle=o,dotsize=5pt 0,linewidth=0.8pt,arrowsize=3pt 2,arrowinset=0.25}
\begin{pspicture*}(-1.26,-0.72)(9.76,12.5)
\multips(0,0)(0,1.0){14}{\psline[linestyle=dashed,linecap=1,dash=1.5pt 1.5pt,linewidth=0.4pt,linecolor=black!70]{c-c}(0,0)(9.76,0)}
\multips(0,0)(1.0,0){12}{\psline[linestyle=dashed,linecap=1,dash=1.5pt 1.5pt,linewidth=0.4pt,linecolor=black!70]{c-c}(0,0)(0,12.5)}
\psaxes[labelFontSize=\scriptstyle,xAxis=true,yAxis=true,labels=none,Dx=1.,Dy=1.,ticksize=-3pt 0,subticks=0]{->}(0,0)(0.,0.)(9.76,12.5)[$r$,140] [$V(r)$,-40]
\psline[linestyle=dashed,dash=5pt 5pt](0.,1.)(2.,1.)
\psline[linestyle=dashed,dash=5pt 5pt](2.,1.)(2.,0.)
\rput[tl](-1.2,1.25){$V(r_1)$}
\rput[tl](1.82,-0.35){$r_1$}
\psdots[dotsize=3pt 0,dotstyle=*](2.,1.)
\rput[bl](2.08,1.12){$P$}
\antwort{\psdots[dotsize=3pt 0,dotstyle=*,linecolor=red](6.,9.)
\rput[bl](6.08,9.12){$Q$}}
\rput[bl](-0.4,-0.1){0}
\rput[bl](-0.1,-0.5){0}
\end{pspicture*}}
\end{center}
\end{beispiel}