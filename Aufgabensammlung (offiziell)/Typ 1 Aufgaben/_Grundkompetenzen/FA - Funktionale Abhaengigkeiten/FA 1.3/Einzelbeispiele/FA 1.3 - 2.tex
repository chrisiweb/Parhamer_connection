\section{FA 1.3 - 2 Funktionswerte - OA - BIFIE}

\begin{beispiel}[FA 1.3]{1} %PUNKTE DES BEISPIELS
Die nachstehende Abbildung zeigt den Graphen einer Funktion $f$.
\leer

\begin{center}
\newrgbcolor{zzttff}{0.6 0.2 1.}
\psset{xunit=0.2cm,yunit=0.02cm,algebraic=true,dimen=middle,dotstyle=o,dotsize=5pt 0,linewidth=0.8pt,arrowsize=3pt 2,arrowinset=0.25}
\begin{pspicture*}(-13.740576233315798,-177.73573657644127)(24.377393463682914,185.19946207703413)
\multips(0,-170)(0,10.0){37}{\psline[linestyle=dashed,linecap=1,dash=1.5pt 1.5pt,linewidth=0.4pt,linecolor=lightgray]{c-c}(-13.740576233315798,0)(24.377393463682914,0)}
\multips(-15,0)(5.0,0){8}{\psline[linestyle=dashed,linecap=1,dash=1.5pt 1.5pt,linewidth=0.4pt,linecolor=lightgray]{c-c}(0,-177.73573657644127)(0,185.19946207703413)}
\psaxes[labelFontSize=\scriptstyle,xAxis=true,yAxis=true,Dx=5.,Dy=50.,ticksize=-2pt 0,subticks=2]{->}(0,0)(-13.740576233315798,-177.73573657644127)(24.377393463682914,185.19946207703413)[x,140] [f(x),-40]
\psplot[linewidth=0.8pt,linecolor=black,plotpoints=200]{-13.740576233315798}{24.377393463682914}{0.1*x^(3.0)-1.5*x^(2.0)-15.0*x+100.0}
\rput[tl](-8.527862428598025,99.84127352478049){$f$}
\end{pspicture*}
\end{center}
\leer

Erstelle aus dem Graphen von $f$ eine Wertetabelle f�r $-10\leq x\leq20$ mit der Schrittweite 5!
\leer

\antwort{Wertetabelle:

\begin{longtable}{|c|c|}\hline
x&y\\ \hline
-10&0\\ \hline
-5&125\\ \hline
0&100\\ \hline
5&0\\ \hline
10&-100\\ \hline
15&-125\\ \hline
20&0\\ \hline
\end{longtable}

Toleranz f�r die Ablesegenauigkeit: $\pm 1$.}
\end{beispiel}