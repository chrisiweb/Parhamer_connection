\section{AN 1.1 - 10 - MAT - Radioaktiver Zerfall - MC - Matura 2016/17 2. NT}

\begin{beispiel}[AN 1.1]{1} %PUNKTE DES BEISPIELS
Der Wert $m(t)$ bezeichnet die nach $t$ Tagen vorhandene Menge eines radioaktiven Stoffes.

Einer der nachstehend angef�hrten Ausdr�cke beschreibt die relative �nderung der Menge des radioaktiven Stoffes innerhalb der ersten drei Tage.

Kreuze den zutreffenden Ausdruck an!\leer

\multiplechoice[6]{  %Anzahl der Antwortmoeglichkeiten, Standard: 5
				L1={$m(3)-m(0)$},   %1. Antwortmoeglichkeit 
				L2={$\frac{m(3)-m(0)}{3}$},   %2. Antwortmoeglichkeit
				L3={$\frac{m(0)}{m(3)}$},   %3. Antwortmoeglichkeit
				L4={$\frac{m(3)-m(0)}{m(0)}$},   %4. Antwortmoeglichkeit
				L5={$\frac{m(3)-m(0)}{m(0)-m(3)}$},	 %5. Antwortmoeglichkeit
				L6={$m'(3)$},	 %6. Antwortmoeglichkeit
				L7={},	 %7. Antwortmoeglichkeit
				L8={},	 %8. Antwortmoeglichkeit
				L9={},	 %9. Antwortmoeglichkeit
				%% LOESUNG: %%
				A1=4,  % 1. Antwort
				A2=0,	 % 2. Antwort
				A3=0,  % 3. Antwort
				A4=0,  % 4. Antwort
				A5=0,  % 5. Antwort
				}
\end{beispiel}