\section{AN 1.1 - 6 Fertilit�t - OA - Matura NT 2 15/16}

\begin{beispiel}[AN 1.1]{1} %PUNKTE DES BEISPIELS
Auf der Website der Statistik Austria findet man unter dem Begriff \textit{Fertilit�t} (Fruchbarkeit) folgende Information:

"`Die Gesamtfertilit�tsrate lag 2014 bei 1,46 Kindern je Frau, d.h., dass bei zuk�nftiger Konstanz der altersspezifischen Fertilit�tsraten eine heute 15-j�hrige Frau in �sterreich bis zu ihrem 50. Geburtstag statistisch gesehen 1,46 Kinder zur Welt bringen wird. Dieser Mittelwert liegt damit deutlich unter dem "`Bestanderhaltungsniveau"' von etwa 2 Kindern pro Frau."'

Berechne, um welchen Prozentsatz die f�r das Jahr 2014 g�ltige Gesamtfertilit�tsrate von 1,46 Kindern je Frau ansteigen m�sste, um das "`Bestanderhaltungsniveau"' zu erreichen.

prozentuelle Zunahme: \antwort[\rule{3cm}{0.3pt}\,\%]{36,99\,\% Toleranzintervall: $\left[36\,\%;37\,\%\right]$}
\end{beispiel}