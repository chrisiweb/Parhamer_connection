\section{AN 3.3 - 12 Lokale Eigenschaften einer Funktion - ZO - BIFIE}

\begin{beispiel}[AN 3.3]{1} %PUNKTE DES BEISPIELS
Gegeben ist der Graph einer Funktion $f$.

Die eingezeichneten Punkte $A, B, C, D, E, F, G, H$ und $I$ liegen auf dem Funktionsgraphen; weiters sind die Tangenten in A$, C, E$ und $G$ eingetragen; in $B, D, H$ und $I$ ist die Tangente
horizontal (waagrecht).
\leer

\begin{center}
\newrgbcolor{uququq}{0 0 0}
\psset{xunit=1.0cm,yunit=1.0cm,algebraic=true,dimen=middle,dotstyle=o,dotsize=5pt 0,linewidth=0.8pt,arrowsize=3pt 2,arrowinset=0.25}
\begin{pspicture*}(-3.6355667060082197,-3.269920448394353)(9.820220603907824,4.1054274076684125)
\multips(0,-3)(0,1.0){8}{\psline[linestyle=dashed,linecap=1,dash=1.5pt 1.5pt,linewidth=0.4pt,linecolor=gray]{c-c}(-3.6355667060082197,0)(9.820220603907824,0)}
\multips(-3,0)(1.0,0){14}{\psline[linestyle=dashed,linecap=1,dash=1.5pt 1.5pt,linewidth=0.4pt,linecolor=gray]{c-c}(0,-3.269920448394353)(0,4.1054274076684125)}
\psaxes[labelFontSize=\scriptstyle,xAxis=true,yAxis=true,labels=none,Dx=1.,Dy=1.,ticksize=-2pt 0,subticks=0]{->}(0,0)(-3.6355667060082197,-3.269920448394353)(9.820220603907824,4.1054274076684125)[x,140] [y,-40]
\psplot[linewidth=1.2pt,plotpoints=200]{2}{9.820220603907824}{-5.555110798399976E-4*x^(6.0)+0.019850289694066662*x^(5.0)-0.2687826512955311*x^(4.0)+1.703233896493089*x^(3.0)-4.815576026222835*x^(2.0)+3.5949090632028415*x+2.147480666169065}
\psplot[linewidth=1.2pt,plotpoints=200]{-3.6355667060082197}{2}{-0.007933237741692101*x^(5.0)-0.021919040655537094*x^(4.0)-0.13556938065418442*x^(3.0)-0.22315688536656297*x^(2.0)+0.6692093264838113*x+1.2433321919548455}
\psplot{-3.5}{-2.5}{(-6.453782600701254-2.4979275335670845*x)/1.}
\psplot{-1.8}{0.6}{(--1.2707402119495215--0.803770489476355*x)/1.}
\psplot{1}{2.1}{(--3.8502573771817783-1.8049947678049099*x)/1.}
\psplot{2}{3.5}{(--2.745588972279339-1.5273264011153778*x)/1.}
\begin{scriptsize}
\psdots[dotsize=3pt 0,dotstyle=*,linecolor=uququq](11.03459734931723,3.5937446663710273)
\rput[bl](11.434164409560237,2.416416448264726){\uququq{$J$}}
\psdots[dotsize=3pt 0,dotstyle=*,linecolor=uququq](7.956520007776573,-1.3551210036873933)
\rput[bl](7.9,-1.2431072971099286){\uququq{$I$}}
\psdots[dotsize=3pt 0,dotstyle=*,linecolor=uququq](3.9822374799802778,-2.2722631254865395)
\rput[bl](3.9,-2.1626799305630473){\uququq{$H$}}
\psdots[dotsize=3pt 0,dotstyle=*,linecolor=black](2.6982385094614747,-1.3755019397273744)
\rput[bl](2.782674939725793,-1.2618740855477475){\uququq{$G$}}
\psdots[dotsize=3pt 0,dotstyle=*,linecolor=black](1.6320303478522467,0.9044511384096461)
\rput[bl](1.712967998770124,1.0089073154283204){\uququq{$E$}}
\psdots[dotsize=3pt 0,dotstyle=*,linecolor=black](2.,0.)
\rput[bl](2.06953697908868,0.10810147041302073){\uququq{$F$}}
\psdots[dotsize=3pt 0,dotstyle=*,linecolor=uququq](-3.,1.04)
\rput[bl](-2.8848951684954693,1.2341087766821452){\uququq{$A$}}
\psdots[dotsize=3pt 0,dotstyle=*,linecolor=black](-0.678770016914573,0.7251649032122212)
\rput[bl](-0.8580820172110446,0.9150733732392267){\uququq{$C$}}
\rput[bl](-3.5,3){\normalsize{$f$}}
\psdots[dotsize=3pt 0,dotstyle=*,linecolor=black](-2.,0.)
\rput[bl](-2.1,-0.3){\uququq{$B$}}
\psdots[dotsize=3pt 0,dotstyle=*,linecolor=black](0.7909989050562288,1.5549192179289262)
\rput[bl](0.8,1.6657449107519764){\uququq{$D$}}
\end{scriptsize}
\end{pspicture*}
\end{center}

Ordne den angegebenen Eigenschaften jeweils einen der markierten Punkte zu.

\zuordnen[0.22]{
				title1={Punkte}, 		%Titel Antwortmoeglichkeiten
				A={A}, 				%Moeglichkeit A  
				B={B}, 				%Moeglichkeit B  
				C={C}, 				%Moeglichkeit C  
				D={D}, 				%Moeglichkeit D  
				E={E}, 				%Moeglichkeit E  
				F={F}, 				%Moeglichkeit F  
				title2={Eigenschaften},		%Titel Zuordnung
				R1={\mbox{$f(x)<0$}, \mbox{$f'(x)=0$}, \mbox{$f''(x)<0$}},				%1. Antwort rechts
				R2={\mbox{$f(x)>0$}, \mbox{$f'(x)>0$}, \mbox{$f''(x)=0$}},				%2. Antwort rechts
				R3={\mbox{$f(x)=0$}, \mbox{$f'(x)=0$}, \mbox{$f''(x)>0$}},				%3. Antwort rechts
				R4={\mbox{$f(x)>0$}, \mbox{$f'(x)<0$}, \mbox{$f''(x)>0$}},				%4. Antwort rechts
				%% LOESUNG: %%
				A1={D},				% 1. richtige Zuordnung
				A2={C},				% 2. richtige Zuordnung
				A3={B},				% 3. richtige Zuordnung
				A4={A},				% 4. richtige Zuordnung
				}

\end{beispiel}