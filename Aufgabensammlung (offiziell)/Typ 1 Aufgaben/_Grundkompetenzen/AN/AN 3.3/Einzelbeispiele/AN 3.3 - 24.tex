\section{AN 3.3 - 24 Graph einer Ableitungsfunktion - MC - Matura 2014/15 - Nebentermin 1}

\begin{beispiel}[AN 3.3]{1} %PUNKTE DES BEISPIELS
Die nachstehende Abbildung zeigt den Graphen der Ableitungsfunktion $f'$ einer Funktion $f$. Die Funktion $f'$ ist eine Polynomfunktion zweiten Grades.


\begin{center}
\resizebox{1\linewidth}{!}{\psset{xunit=1.0cm,yunit=1.0cm,algebraic=true,dimen=middle,dotstyle=o,dotsize=5pt 0,linewidth=0.8pt,arrowsize=3pt 2,arrowinset=0.25}
\begin{pspicture*}(-5.4329346942751044,-3.4811881726732863)(10.60322300949462,5.54288566559713)
\multips(0,-3)(0,1.0){10}{\psline[linestyle=dashed,linecap=1,dash=1.5pt 1.5pt,linewidth=0.4pt,linecolor=gray]{c-c}(-5.4329346942751044,0)(10.60322300949462,0)}
\multips(-5,0)(1.0,0){17}{\psline[linestyle=dashed,linecap=1,dash=1.5pt 1.5pt,linewidth=0.4pt,linecolor=gray]{c-c}(0,-3.4811881726732863)(0,5.54288566559713)}
\psaxes[labelFontSize=\scriptstyle,xAxis=true,yAxis=true,Dx=1.,Dy=1.,ticksize=-2pt 0,subticks=2]{->}(0,0)(-5.4329346942751044,-3.4811881726732863)(10.60322300949462,5.54288566559713)[$x$,140] [$f'(x)$,-40]
\psplot[linewidth=1.2pt,plotpoints=200]{-5.4329346942751044}{10.60322300949462}{0.5*x^(2.0)-3.0}
\rput[tl](-4.177931047893126,3.650419849624306){$f'$}
\end{pspicture*}}
\end{center}

\leer

Kreuze die beiden zutreffenden Aussagen an. 

\multiplechoice[5]{  %Anzahl der Antwortmoeglichkeiten, Standard: 5
				L1={Die Funktion $f$ ist eine Polynomfunktion dritten Grades.},   %1. Antwortmoeglichkeit 
				L2={Die Funktion $f$ ist im Intervall $[0; 4]$ streng monoton steigend},   %2. Antwortmoeglichkeit
				L3={Die Funktion $f$ ist im Intervall $[-4; -3]$ streng monoton fallend.},   %3. Antwortmoeglichkeit
				L4={Die Funktion $f$ hat an der Stelle $x = 0$ eine Wendestelle},   %4. Antwortmoeglichkeit
				L5={Die Funktion $f$ ist im Intervall $[-4; 4]$ links gekr�mmt.},	 %5. Antwortmoeglichkeit
				L6={},	 %6. Antwortmoeglichkeit
				L7={},	 %7. Antwortmoeglichkeit
				L8={},	 %8. Antwortmoeglichkeit
				L9={},	 %9. Antwortmoeglichkeit
				%% LOESUNG: %%
				A1=1,  % 1. Antwort
				A2=4,	 % 2. Antwort
				A3=0,  % 3. Antwort
				A4=0,  % 4. Antwort
				A5=0,  % 5. Antwort
				}
\end{beispiel}