\section{AN 3.2 - 7 Funktionen und Ableitungsfunktionen - ZO - Matura 2015/16 - Haupttermin}

\begin{beispiel}[AN 3.2]{1} %PUNKTE DES BEISPIELS
Links sind die Graphen von vier Polynomfunktionen $(f_1, f_2, f_3, f_4)$ abgebildet, rechts die Graphen sechs weiterer Funktionen $(g_1, g_2, g_3, g_4, g_5, g_6)$.

Ordnen Sie den Polynomfunktionen $f_1$ bis $f_4$ ihre jeweilige Ableitungsfunktion aus den Funktionen $g_1$ bis $g_6$ (aus A bis F) zu.


\zuordnen[0.043]{
				R1={\resizebox{0.5\linewidth}{!}{\psset{xunit=1.0cm,yunit=1.0cm,algebraic=true,dimen=middle,dotstyle=o,dotsize=5pt 0,linewidth=0.8pt,arrowsize=3pt 2,arrowinset=0.25}
\begin{pspicture*}(-2.64,-1.32)(3,6.52)
\multips(0,-1)(0,1.0){8}{\psline[linestyle=dashed,linecap=1,dash=1.5pt 1.5pt,linewidth=0.4pt,linecolor=lightgray]{c-c}(-2.64,0)(3,0)}
\multips(-2,0)(1.0,0){6}{\psline[linestyle=dashed,linecap=1,dash=1.5pt 1.5pt,linewidth=0.4pt,linecolor=lightgray]{c-c}(0,-1.32)(0,6.52)}
\psaxes[labelFontSize=\scriptstyle,xAxis=true,yAxis=true,Dx=1.,Dy=1.,ticksize=-2pt 0,subticks=2]{->}(0,0)(-2.64,-1.32)(3,6.52)[$x$,140] [$f_1(x)$,-40]
\psplot[linewidth=1.2pt,plotpoints=200]{-2.640000000000001}{3}{x^(2.0)+1.0}
\begin{scriptsize}
\rput[bl](-1.84,5.36){$f_1$}
\end{scriptsize}
\end{pspicture*}}},				% Response 1
				R2={\resizebox{0.5\linewidth}{!}{\psset{xunit=1.0cm,yunit=1.0cm,algebraic=true,dimen=middle,dotstyle=o,dotsize=5pt 0,linewidth=0.8pt,arrowsize=3pt 2,arrowinset=0.25}
\begin{pspicture*}(-2.64,-1.32)(2.56,6.52)
\multips(0,-1)(0,1.0){8}{\psline[linestyle=dashed,linecap=1,dash=1.5pt 1.5pt,linewidth=0.4pt,linecolor=lightgray]{c-c}(-2.64,0)(2.56,0)}
\multips(-2,0)(1.0,0){6}{\psline[linestyle=dashed,linecap=1,dash=1.5pt 1.5pt,linewidth=0.4pt,linecolor=lightgray]{c-c}(0,-1.32)(0,6.52)}
\psaxes[labelFontSize=\scriptstyle,xAxis=true,yAxis=true,Dx=1.,Dy=1.,ticksize=-2pt 0,subticks=2]{->}(0,0)(-2.64,-1.32)(2.56,6.52)[$x$,140] [$f_2(x)$,-40]
\psplot[linewidth=1.2pt,plotpoints=200]{-2.640000000000001}{3}{x^(3.0)+x+3.0}
\begin{scriptsize}
\rput[bl](1.3,5.36){$f_2$}
\end{scriptsize}
\end{pspicture*}}},				% Response 2
				R3={\resizebox{0.5\linewidth}{!}{\psset{xunit=1.0cm,yunit=1.0cm,algebraic=true,dimen=middle,dotstyle=o,dotsize=5pt 0,linewidth=0.8pt,arrowsize=3pt 2,arrowinset=0.25}
\begin{pspicture*}(-2.64,-1.32)(3,6.52)
\multips(0,-1)(0,1.0){8}{\psline[linestyle=dashed,linecap=1,dash=1.5pt 1.5pt,linewidth=0.4pt,linecolor=lightgray]{c-c}(-2.64,0)(3,0)}
\multips(-2,0)(1.0,0){6}{\psline[linestyle=dashed,linecap=1,dash=1.5pt 1.5pt,linewidth=0.4pt,linecolor=lightgray]{c-c}(0,-1.32)(0,6.52)}
\psaxes[labelFontSize=\scriptstyle,xAxis=true,yAxis=true,Dx=1.,Dy=1.,ticksize=-2pt 0,subticks=2]{->}(0,0)(-2.64,-1.32)(3,6.52)[$x$,140] [$f_3(x)$,-40]
\psplot[linewidth=1.2pt,plotpoints=200]{-2.640000000000001}{3}{x^4-2*x^2+2}
\begin{scriptsize}
\rput[bl](-1.7,5.36){$f_3$}
\end{scriptsize}
\end{pspicture*}}},				% Response 3
				R4={\resizebox{0.5\linewidth}{!}{\psset{xunit=1.0cm,yunit=1.0cm,algebraic=true,dimen=middle,dotstyle=o,dotsize=5pt 0,linewidth=0.8pt,arrowsize=3pt 2,arrowinset=0.25}
\begin{pspicture*}(-2.64,-1.32)(3,6.52)
\multips(0,-1)(0,1.0){8}{\psline[linestyle=dashed,linecap=1,dash=1.5pt 1.5pt,linewidth=0.4pt,linecolor=lightgray]{c-c}(-2.64,0)(3,0)}
\multips(-2,0)(1.0,0){6}{\psline[linestyle=dashed,linecap=1,dash=1.5pt 1.5pt,linewidth=0.4pt,linecolor=lightgray]{c-c}(0,-1.32)(0,6.52)}
\psaxes[labelFontSize=\scriptstyle,xAxis=true,yAxis=true,Dx=1.,Dy=1.,ticksize=-2pt 0,subticks=2]{->}(0,0)(-2.64,-1.32)(3,6.52)[$x$,140] [$f_4(x)$,-40]
\psplot[linewidth=1.2pt,plotpoints=200]{-2.640000000000001}{3}{-x^(3.0)+3*x^2}
\begin{scriptsize}
\rput[bl](-1.7,5.36){$f_4$}
\end{scriptsize}
\end{pspicture*}}},				% Response 4
				%% Moegliche Zuordnungen: %%
				A={\resizebox{0.4\linewidth}{!}{\psset{xunit=1.0cm,yunit=1.0cm,algebraic=true,dimen=middle,dotstyle=o,dotsize=5pt 0,linewidth=0.8pt,arrowsize=3pt 2,arrowinset=0.25}
\begin{pspicture*}(-2.64,-1.32)(3,6.52)
\multips(0,-1)(0,1.0){8}{\psline[linestyle=dashed,linecap=1,dash=1.5pt 1.5pt,linewidth=0.4pt,linecolor=lightgray]{c-c}(-2.64,0)(3,0)}
\multips(-2,0)(1.0,0){6}{\psline[linestyle=dashed,linecap=1,dash=1.5pt 1.5pt,linewidth=0.4pt,linecolor=lightgray]{c-c}(0,-1.32)(0,6.52)}
\psaxes[labelFontSize=\scriptstyle,xAxis=true,yAxis=true,Dx=1.,Dy=1.,ticksize=-2pt 0,subticks=2]{->}(0,0)(-2.64,-1.32)(3,6.52)[$x$,140] [$g_1(x)$,-40]
\psplot[linewidth=1.2pt,plotpoints=200]{-2.640000000000001}{3}{3*x^(2.0)+1}
\begin{scriptsize}
\rput[bl](-1.7,5.36){$g_1$}
\end{scriptsize}
\end{pspicture*}}}, 				%Moeglichkeit A  
				B={\resizebox{0.4\linewidth}{!}{\psset{xunit=1.0cm,yunit=1.0cm,algebraic=true,dimen=middle,dotstyle=o,dotsize=5pt 0,linewidth=0.8pt,arrowsize=3pt 2,arrowinset=0.25}
\begin{pspicture*}(-2.64,-1.32)(3,6.52)
\multips(0,-1)(0,1.0){8}{\psline[linestyle=dashed,linecap=1,dash=1.5pt 1.5pt,linewidth=0.4pt,linecolor=lightgray]{c-c}(-2.64,0)(3,0)}
\multips(-2,0)(1.0,0){6}{\psline[linestyle=dashed,linecap=1,dash=1.5pt 1.5pt,linewidth=0.4pt,linecolor=lightgray]{c-c}(0,-1.32)(0,6.52)}
\psaxes[labelFontSize=\scriptstyle,xAxis=true,yAxis=true,Dx=1.,Dy=1.,ticksize=-2pt 0,subticks=2]{->}(0,0)(-2.64,-1.32)(3,6.52)[$x$,140] [$g_2(x)$,-40]
\psplot[linewidth=1.2pt,plotpoints=200]{-2.640000000000001}{3}{-3*x^(2.0)+6*x+2}
\begin{scriptsize}
\rput[bl](1.6,4.5){$g_2$}
\end{scriptsize}
\end{pspicture*}}}, 				%Moeglichkeit B  
				C={\resizebox{0.4\linewidth}{!}{\psset{xunit=1.0cm,yunit=1.0cm,algebraic=true,dimen=middle,dotstyle=o,dotsize=5pt 0,linewidth=0.8pt,arrowsize=3pt 2,arrowinset=0.25}
\begin{pspicture*}(-2.64,-1.8)(3,6.52)
\multips(0,-1)(0,1.0){8}{\psline[linestyle=dashed,linecap=1,dash=1.5pt 1.5pt,linewidth=0.4pt,linecolor=lightgray]{c-c}(-2.64,0)(3,0)}
\multips(-2,0)(1.0,0){6}{\psline[linestyle=dashed,linecap=1,dash=1.5pt 1.5pt,linewidth=0.4pt,linecolor=lightgray]{c-c}(0,-1.8)(0,6.52)}
\psaxes[labelFontSize=\scriptstyle,xAxis=true,yAxis=true,Dx=1.,Dy=1.,ticksize=-2pt 0,subticks=2]{->}(0,0)(-2.64,-1.8)(3,6.52)[$x$,140] [$g_3(x)$,-40]
\psplot[linewidth=1.2pt,plotpoints=200]{-2.640000000000001}{3}{-4*x^3+4*x}
\begin{scriptsize}
\rput[bl](-1.84,5.36){$g_3$}
\end{scriptsize}
\end{pspicture*}}}, 				%Moeglichkeit C  
				D={\resizebox{0.4\linewidth}{!}{\psset{xunit=1.0cm,yunit=1.0cm,algebraic=true,dimen=middle,dotstyle=o,dotsize=5pt 0,linewidth=0.8pt,arrowsize=3pt 2,arrowinset=0.25}
\begin{pspicture*}(-2.64,-1.32)(3,6.52)
\multips(0,-1)(0,1.0){8}{\psline[linestyle=dashed,linecap=1,dash=1.5pt 1.5pt,linewidth=0.4pt,linecolor=lightgray]{c-c}(-2.64,0)(3,0)}
\multips(-2,0)(1.0,0){6}{\psline[linestyle=dashed,linecap=1,dash=1.5pt 1.5pt,linewidth=0.4pt,linecolor=lightgray]{c-c}(0,-1.32)(0,6.52)}
\psaxes[labelFontSize=\scriptstyle,xAxis=true,yAxis=true,Dx=1.,Dy=1.,ticksize=-2pt 0,subticks=2]{->}(0,0)(-2.64,-1.32)(3,6.52)[$x$,140] [$g_4(x)$,-40]
\psplot[linewidth=1.2pt,plotpoints=200]{-2.640000000000001}{3}{-3*x^(2.0)+6*x}
\begin{scriptsize}
\rput[bl](1.5,2.6){$g_4$}
\end{scriptsize}
\end{pspicture*}}}, 				%Moeglichkeit D  
				E={\resizebox{0.4\linewidth}{!}{\psset{xunit=1.0cm,yunit=1.0cm,algebraic=true,dimen=middle,dotstyle=o,dotsize=5pt 0,linewidth=0.8pt,arrowsize=3pt 2,arrowinset=0.25}
\begin{pspicture*}(-2.64,-1.32)(3,6.52)
\multips(0,-1)(0,1.0){8}{\psline[linestyle=dashed,linecap=1,dash=1.5pt 1.5pt,linewidth=0.4pt,linecolor=lightgray]{c-c}(-2.64,0)(3,0)}
\multips(-2,0)(1.0,0){6}{\psline[linestyle=dashed,linecap=1,dash=1.5pt 1.5pt,linewidth=0.4pt,linecolor=lightgray]{c-c}(0,-1.32)(0,6.52)}
\psaxes[labelFontSize=\scriptstyle,xAxis=true,yAxis=true,Dx=1.,Dy=1.,ticksize=-2pt 0,subticks=2]{->}(0,0)(-2.64,-1.32)(3,6.52)[$x$,140] [$g_5(x)$,-40]
\psplot[linewidth=1.2pt,plotpoints=200]{-2.640000000000001}{3}{2*x}
\begin{scriptsize}
\rput[bl](1.5,2.36){$g_5$}
\end{scriptsize}
\end{pspicture*}}}, 				%Moeglichkeit E  
				F={\resizebox{0.4\linewidth}{!}{\psset{xunit=1.0cm,yunit=1.0cm,algebraic=true,dimen=middle,dotstyle=o,dotsize=5pt 0,linewidth=0.8pt,arrowsize=3pt 2,arrowinset=0.25}
\begin{pspicture*}(-2.64,-1.8)(3,6.52)
\multips(0,-1)(0,1.0){8}{\psline[linestyle=dashed,linecap=1,dash=1.5pt 1.5pt,linewidth=0.4pt,linecolor=lightgray]{c-c}(-2.64,0)(3,0)}
\multips(-2,0)(1.0,0){6}{\psline[linestyle=dashed,linecap=1,dash=1.5pt 1.5pt,linewidth=0.4pt,linecolor=lightgray]{c-c}(0,-1.8)(0,6.52)}
\psaxes[labelFontSize=\scriptstyle,xAxis=true,yAxis=true,Dx=1.,Dy=1.,ticksize=-2pt 0,subticks=2]{->}(0,0)(-2.64,-1.8)(3,6.52)[$x$,140] [$g_6(x)$,-40]
\psplot[linewidth=1.2pt,plotpoints=200]{-2.640000000000001}{3}{4*x^3-4*x}
\begin{scriptsize}
\rput[bl](1.5,2.5){$g_6$}
\end{scriptsize}
\end{pspicture*}}}, 				%Moeglichkeit F  
				%% LOESUNG: %%
				A1={E},				% 1. richtige Zuordnung
				A2={A},				% 2. richtige Zuordnung
				A3={F},				% 3. richtige Zuordnung
				A4={D},				% 4. richtige Zuordnung
				}

\end{beispiel}