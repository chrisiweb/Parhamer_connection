\section{AN 4.3 - 8 Integral - OA - Matura 2015/16 - Haupttermin}

\begin{beispiel}[AN 4.3]{1} %PUNKTE DES BEISPIELS
Gegeben ist die Potenzfunktion $f$ mit $f(x) = x^3$.\leer

Gin eine Bedingung f�r die Integrationsgrenzen $b$ und $c$ $(b \neq c)$ so an, dass

$$\int_{b}^{c}\! f(x)\, \mathrm{d}x=0 \quad \text{gilt.}$$ 


\antwort{$b=-c$ \\

L�sungsschl�ssel:

Ein Punkt f�r die Angabe einer korrekten Relation zwischen $b$ und $c$. �quivalente Relationen sind als richtig zu werten, ebenso konkrete Beispiele wie $b = -5$ und $c=5$.}
\end{beispiel}