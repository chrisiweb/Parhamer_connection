\section{AN 4.2 - 8 Fl�cheninhalt - OA - Matura NT 2 15/16}

\begin{beispiel}[AN 4.2]{1} %PUNKTE DES BEISPIELS
Abgebildet ist ein Ausschnitt des Graphen der Polynomfunktion $f$ mit \mbox{$f(x)=-\frac{x�}{8}+2\cdot x$}.

Die Fl�che zwischen dem Graphen der Funktion $f$ und der x-Achse im Intervall $[-2;2]$ ist grau markiert.

\begin{center}
	\resizebox{0.7\linewidth}{!}{\newrgbcolor{sqsqsq}{0.12549019607843137 0.12549019607843137 0.12549019607843137}
\psset{xunit=1.0cm,yunit=1.0cm,algebraic=true,dimen=middle,dotstyle=o,dotsize=5pt 0,linewidth=0.8pt,arrowsize=3pt 2,arrowinset=0.25}
\begin{pspicture*}(-5.72,-5.86)(5.96,5.7)
\multips(0,-5)(0,1.0){12}{\psline[linestyle=dashed,linecap=1,dash=1.5pt 1.5pt,linewidth=0.4pt,linecolor=darkgray]{c-c}(-5.72,0)(5.96,0)}
\multips(-5,0)(1.0,0){12}{\psline[linestyle=dashed,linecap=1,dash=1.5pt 1.5pt,linewidth=0.4pt,linecolor=darkgray]{c-c}(0,-5.86)(0,5.7)}
\psaxes[labelFontSize=\scriptstyle,xAxis=true,yAxis=true,Dx=1.,Dy=1.,ticksize=-2pt 0,subticks=2]{->}(0,0)(-5.72,-5.86)(5.96,5.7)[x,140] [f(x),-40]
\pscustom[linecolor=sqsqsq,fillcolor=sqsqsq,fillstyle=solid,opacity=0.25]{\psplot{-2.}{2.}{(-x^(3.0))/8.0+2.0*x}\lineto(2.,0)\lineto(-2.,0)\closepath}
\psplot[linewidth=2.pt,plotpoints=200]{-5.720000000000001}{5.959999999999994}{(-x^(3.0))/8.0+2.0*x}
\rput[bl](-4.56,4.02){$f$}
\rput[bl](-0.12,0.24){\sqsqsq{$0$}}
\end{pspicture*}}
\end{center}

Berechne den Inhalt der grau markierten Fl�che!
\leer

\antwort{$$2\cdot\int^2_0{f(x)}dx=7$$}
\end{beispiel}