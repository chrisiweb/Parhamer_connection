\section{AN 2.1 - 10 Ableitungsregeln - MC - Matura 2015/16 - Nebentermin 1}

\begin{beispiel}[AN 2.1]{1} %PUNKTE DES BEISPIELS
�ber zwei Polynomfunktionen $f$ und $g$ ist bekannt, dass f�r alle $x\in \mathbb{R}$ gilt: 

$g(x)=3\cdot f(x)-2$ \leer

Welche der nachstehenden Aussagen ist jedenfalls f�r alle $x\in \mathbb{R}$ wahr? Kreuze die zutreffende Aussage an. 

\multiplechoice[6]{  %Anzahl der Antwortmoeglichkeiten, Standard: 5
				L1={$g'(x)=f'(x)$},   %1. Antwortmoeglichkeit 
				L2={$g'(x)=f'(x)-2$},   %2. Antwortmoeglichkeit
				L3={$g'(x)=3\cdot f'(x)$},   %3. Antwortmoeglichkeit
				L4={$g'(x)=3\cdot f'(x) -2$},   %4. Antwortmoeglichkeit
				L5={$g'(x)=3\cdot f'(x)-2\cdot x$},	 %5. Antwortmoeglichkeit
				L6={$g'(x)=-2\cdot f'(x)$},	 %6. Antwortmoeglichkeit
				L7={},	 %7. Antwortmoeglichkeit
				L8={},	 %8. Antwortmoeglichkeit
				L9={},	 %9. Antwortmoeglichkeit
				%% LOESUNG: %%
				A1=3,  % 1. Antwort
				A2=0,	 % 2. Antwort
				A3=0,  % 3. Antwort
				A4=0,  % 4. Antwort
				A5=0,  % 5. Antwort
				}
\end{beispiel}