\section{AN 1.3 - 7 Temperaturverlauf - MC - BIFIE}

\begin{beispiel}[AN 1.3]{1} %PUNKTE DES BEISPIELS
Aus dem nachstehend dargestellten Graphen der Funktion T l�sst sich der Temperaturverlauf in �C in einem Reagenzglas w�hrend eines chemischen Versuchs f�r die ersten 7 Minuten ablesen.

\begin{center}
\resizebox{0.8\linewidth}{!}{\psset{xunit=1.0cm,yunit=.1cm,algebraic=true,dimen=middle,dotstyle=o,dotsize=5pt 0,linewidth=0.8pt,arrowsize=3pt 2,arrowinset=0.25}
\begin{pspicture*}(-0.5293830818778249,-4.754367060689084)(8.366226176972997,54.438845951477234)
\multips(0,0)(0,5.0){12}{\psline[linestyle=dashed,linecap=1,dash=1.5pt 1.5pt,linewidth=0.4pt,linecolor=lightgray]{c-c}(0,0)(8.366226176972997,0)}
\multips(0,0)(0.5,0){18}{\psline[linestyle=dashed,linecap=1,dash=1.5pt 1.5pt,linewidth=0.4pt,linecolor=lightgray]{c-c}(0,0)(0,54.438845951477234)}
\psaxes[labelFontSize=\scriptstyle,xAxis=true,yAxis=true,Dx=1.,Dy=10.,ticksize=-2pt 0,subticks=2]{->}(0,0)(-0.5293830818778249,-4.754367060689084)(8.366226176972997,54.438845951477234)[t (in min),140] [Temperatur (in �C),-40]
\psplot[linewidth=1.2pt,plotpoints=200]{0}{7}{-0.008980203035689095*x^(5.0)+0.13719708284602655*x^(4.0)+0.3561102134564625*x^(3.0)-7.997836965184824*x^(2.0)+16.530060508880833*x+23.84460102384072}
\rput[tl](3.045487928688394,26.504970147983016){$T$}
\end{pspicture*}}
\end{center}

\leer

Kreuze die auf den Temperaturverlauf zutreffende(n) Aussage(n) an.

\multiplechoice[5]{  %Anzahl der Antwortmoeglichkeiten, Standard: 5
				L1={Im Intervall $[3; 6]$ ist die mittlere �nderungsrate ann�hernd 0\,�C/min. },   %1. Antwortmoeglichkeit 
				L2={Im Intervall $[0,5; 1,5]$ ist der Differenzenquotient gr��er als 25\,�C/min.},   %2. Antwortmoeglichkeit
				L3={Im Intervall $[0; 2]$ gibt es einen Zeitpunkt, in dem die momentane �nderungsrate
0\,�C/min betr�gt. },   %3. Antwortmoeglichkeit
				L4={Der Differenzialquotient zum Zeitpunkt $t = 3$ ist ann�hernd -10\,�C/min.},   %4. Antwortmoeglichkeit
				L5={Der Differenzenquotient ist im Intervall $[2; t]$ mit $2 < t < 6$ immer kleiner als
0\,�C/min. },	 %5. Antwortmoeglichkeit
				L6={},	 %6. Antwortmoeglichkeit
				L7={},	 %7. Antwortmoeglichkeit
				L8={},	 %8. Antwortmoeglichkeit
				L9={},	 %9. Antwortmoeglichkeit
				%% LOESUNG: %%
				A1=1,  % 1. Antwort
				A2=3,	 % 2. Antwort
				A3=4,  % 3. Antwort
				A4=5,  % 4. Antwort
				A5=0,  % 5. Antwort
				}

\end{beispiel}