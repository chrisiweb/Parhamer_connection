\section{AN 1.3 - 13 Schwimmbad - OA - Matura NT 1 16/17}

\begin{beispiel}[AN 1.3]{1} %PUNKTE DES BEISPIELS
In ein Schwimmbad wird ab dem Zeitpunkt $t=0$ Wasser eingelassen.

Die Funktion $h$ beschreibt die H�he des Wasserspiegels zum Zeitpunkt $t$. Die H�he $h(t)$ wird dabei in dm gemessen, die Zeit $t$ in Stunden.

Interpretiere das Ergebnis der folgenden Berechnung im gegebenen Kontext!

$\frac{h(5)-h(2)}{5-2}=4$\leer

\antwort{Die Wasserh�he nimmt im Zeitintervall $[2;5]$ um durchschnittlich 4\,dm pro Stunde zu.}
\end{beispiel}