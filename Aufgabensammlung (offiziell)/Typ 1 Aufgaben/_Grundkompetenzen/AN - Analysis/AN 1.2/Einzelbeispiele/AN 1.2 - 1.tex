\section{AN 1.2 - 1 Luftwiderstand - OA - BIFIE}


\begin{beispiel}[AN 1.2]{1} %PUNKTE DES BEISPIELS
Der Luftwiderstand $F_L$ eines bestimmten PKWs in Abh�ngigkeit von der Fahrtgeschwindigkeit $v$ l�sst sich durch folgende Funktionsgleichung beschreiben: $F_L(v) = 0,4 \cdot v^2$. Der Luftwiderstand ist dabei in Newton (N) und die Geschwindigkeit in Metern pro Sekunde (m/s)
angegeben.

\leer

Berechne die mittlere Zunahme des Luftwiderstandes in $\dfrac{\text{N}}{\text{m/s}}$ bei einer Erh�hung der
Fahrtgeschwindigkeit von 20\,m/s auf 30\,m/s.

\antwort{$\dfrac{F_L(30)-F_L(20)}{30-20}=\dfrac{360-160}{10}=20\,\dfrac{\text{N}}{\text{m/s}}$}
\end{beispiel}