\section{AN 2.1 - 11 Beschleunigungsfunktion bestimmen - OA - Matura 2013/14 1. Nebentermin}

\begin{beispiel}[AN 2.1]{1} %PUNKTE DES BEISPIELS
				Der Weg $s(t)$, den ein K�rper in der Zeit $t$ zur�cklegt, wird in einem bestimmten Zeitintervall durch
				\begin{center}$s(t)=\frac{t�}{6}+5\cdot t�+5\cdot t$\end{center}
				beschrieben ($s(t)$ in Metern, $t$ in Sekunden)
				
				Gib die Funktion $a$ an, die die Beschleunigung dieses K�rpers in Abh�ngigkeit von der Zeit $t$ beschreibt!\leer
				
				$a(t)=$ \antwort[\rule{3cm}{0.3pt}]{$t+10$}
\end{beispiel}