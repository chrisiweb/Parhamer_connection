\section{AN 2.1 - 13 Tiefe eines Gerinnes - OA - Matura 2016/17 - Haupttermin}

\begin{beispiel}[AN 2.1]{1} %PUNKTE DES BEISPIELS
Zur Vorbeugung vor Hochw�ssern wurde in einer Stadt ein Gerinne (Wasserlauf) angelegt. \leer

Die Funktion $f$ beschreibt die Wassertiefe dieses Gerinnes bei einer Hochwasserentwicklung in
Abh�ngigkeit von der Zeit $t$ an einer bestimmten Messstelle f�r das Zeitintervall $[0; 2]$. \leer

Die Gleichung der Funktion $f$ lautet $f(t)=t^3+6\cdot t^2+12\cdot t +8$ mit $t\in [0; 2]$. \leer

Dabei wird $f(t)$ in dm und $t$ in Tagen gemessen. \leer

Gib eine Gleichung der Funktion $g$ an, die die momentane �nderungsrate der Wassertiefe
des Gerinnes (in dm pro Tag) in Abh�ngigkeit von der Zeit $t$ beschreibt! \leer

$g(t)=$ \antwort[\rule{5cm}{0.3pt}]{$3\cdot t^2 + 12\cdot t +12$

oder: $g(t)=f'(t)$} 


\end{beispiel}