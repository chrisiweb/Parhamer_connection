\section{AN 4.3 - 13 Pflanzenwachstum - OA - Matura 2013/14 Haupttermin}

\begin{beispiel}[AN 4.3]{1} %PUNKTE DES BEISPIELS
				Die unten stehende Abbildung beschreibt n�herungsweise das Wachstum einer Schnellw�chsigen Pflanze. Sie zeigt die Wachstumsgeschwindigkeit $v$ in Abh�ngigkeit von der Zeit $t$ w�hrend eines Zeitraums von 60 Tagen.
				
				\begin{center}\resizebox{1\linewidth}{!}{\psset{xunit=0.2cm,yunit=0.5cm,algebraic=true,dimen=middle,dotstyle=o,dotsize=5pt 0,linewidth=0.8pt,arrowsize=3pt 2,arrowinset=0.25}
\begin{pspicture*}(-3.322727272727277,-1.6690909090908792)(71.04772727272733,12.96)
\multips(0,-2)(0,2.0){8}{\psline[linestyle=dashed,linecap=1,dash=1.5pt 1.5pt,linewidth=0.4pt,linecolor=lightgray]{c-c}(-3.322727272727277,0)(71.04772727272733,0)}
\multips(-5,0)(5.0,0){15}{\psline[linestyle=dashed,linecap=1,dash=1.5pt 1.5pt,linewidth=0.4pt,linecolor=lightgray]{c-c}(0,-1.6690909090908792)(0,12.96)}
\begin{scriptsize}
\psaxes[labelFontSize=\scriptstyle,xAxis=true,yAxis=true,Dx=5.,Dy=2.,ticksize=-2pt 0,subticks=2]{->}(0,0)(-3.322727272727277,-1.6690909090908792)(71.04772727272733,12.96)[$t$ (in Tagen),140] [$v(t)$ (in cm/Tag),-40]
\end{scriptsize}
\psline(0.,0.)(40.,4.)
\psline(40.,4.)(50.,4.)
\psline(50.,4.)(60.,0.)
\begin{scriptsize}
\rput[bl](20.718181818181833,2.5036363636363954){$v$}
\end{scriptsize}
\end{pspicture*}}\end{center}\leer
				
				Gib an, um wie viel cm die Pflanze in diesem Zeitraum insgesamt gewachsen ist!
				
				\antwort{$\frac{40\cdot 4}{2}+10\cdot 4+\frac{10\cdot 4}{2}=140$
				
				Die Pflanze w�chst in diesen 60 Tagen 140 cm.
				
				Ein weiterer (sehr aufwendiger) L�sungsweg w�re die Berechnung der Funktionsgleichung in den einzelnen Wachstumsabschnitten sowie die Berechnung der entsprechenden bestimmten Integrale.}
\end{beispiel}