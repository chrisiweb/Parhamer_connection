\section{AN 4.3 - 1 Bestimmte Integrale - ZO - BIFIE}

\begin{beispiel}[AN 4.3]{1} %PUNKTE DES BEISPIELS
				Gegeben ist die Funktion $f(x)=-x�+2x$.
				
				Die nachstehende Tabelle zeigt Graphen der Funktion mit unterschiedlich schraffierten Fl�chenst�cken. Beurteile, ob die nachstehend angef�hrten Integrale den Fl�cheninhalt einer der markierten Fl�chen ergeben und ordne entsprechend zu!
				\leer
				
				
				\zuordnen{
								title1={Integrale}, 		%Titel Antwortmoeglichkeiten
								A={$$2\cdot\int_1^2{(-x�+2x)}dx$$}, 				%Moeglichkeit A  
								B={$$\int_1^3{(-x�+2x)}dx$$}, 				%Moeglichkeit B  
								C={$\int_1^2{(-x�+2x)}dx+\left|\int_2^3{(-x�+2x)}dx\right|$}, 				%Moeglichkeit C  
								D={$\int_0^1{(-x�+2x)}dx-\int_1^2{(-x�+2x)}dx$}, 				%Moeglichkeit D  
								E={$$\left|\int_2^3{(-x�+2x)}dx\right|$$}, 				%Moeglichkeit E  
								F={$$\int_1^2{(-x�+2x)}dx$$}, 				%Moeglichkeit F  
								title2={Markierte Fl�chen},		%Titel Zuordnung
								R1={\resizebox{0.6\linewidth}{!}{\psset{xunit=1.0cm,yunit=1.0cm,algebraic=true,dimen=middle,dotstyle=o,dotsize=5pt 0,linewidth=0.8pt,arrowsize=3pt 2,arrowinset=0.25}
\begin{pspicture*}(-2.740390687361422,-4.354125270739658)(4.601600654629923,2.934144183383457)
\multips(0,-4)(0,1.0){8}{\psline[linestyle=dashed,linecap=1,dash=1.5pt 1.5pt,linewidth=0.4pt,linecolor=darkgray]{c-c}(-2.740390687361422,0)(4.601600654629923,0)}
\multips(-2,0)(1.0,0){8}{\psline[linestyle=dashed,linecap=1,dash=1.5pt 1.5pt,linewidth=0.4pt,linecolor=darkgray]{c-c}(0,-4.354125270739658)(0,2.934144183383457)}
\psaxes[labelFontSize=\scriptstyle,xAxis=true,yAxis=true,Dx=1.,Dy=1.,ticksize=-2pt 0,subticks=2]{->}(0,0)(-2.740390687361422,-4.354125270739658)(4.601600654629923,2.934144183383457)[x,140] [y,-40]
\pscustom[fillcolor=black,fillstyle=solid,opacity=0.1]{\psplot{1.}{2.}{-x^(2.0)+2.0*x}\lineto(2.,0)\lineto(1.,0)\closepath}
\psplot[linewidth=1.2pt,plotpoints=200]{-2.740390687361422}{4.601600654629923}{-x^(2.0)+2.0*x}
\end{pspicture*}}},				%1. Antwort rechts
								R2={\resizebox{0.6\linewidth}{!}{\psset{xunit=1.0cm,yunit=1.0cm,algebraic=true,dimen=middle,dotstyle=o,dotsize=5pt 0,linewidth=0.8pt,arrowsize=3pt 2,arrowinset=0.25}
\begin{pspicture*}(-2.740390687361422,-4.354125270739658)(4.601600654629923,2.934144183383457)
\multips(0,-4)(0,1.0){8}{\psline[linestyle=dashed,linecap=1,dash=1.5pt 1.5pt,linewidth=0.4pt,linecolor=darkgray]{c-c}(-2.740390687361422,0)(4.601600654629923,0)}
\multips(-2,0)(1.0,0){8}{\psline[linestyle=dashed,linecap=1,dash=1.5pt 1.5pt,linewidth=0.4pt,linecolor=darkgray]{c-c}(0,-4.354125270739658)(0,2.934144183383457)}
\psaxes[labelFontSize=\scriptstyle,xAxis=true,yAxis=true,Dx=1.,Dy=1.,ticksize=-2pt 0,subticks=2]{->}(0,0)(-2.740390687361422,-4.354125270739658)(4.601600654629923,2.934144183383457)[x,140] [y,-40]
\pscustom[fillcolor=black,fillstyle=solid,opacity=0.1]{\psplot{2.}{3.}{-x^(2.0)+2.0*x}\lineto(3.,0)\lineto(2.,0)\closepath}
\psplot[linewidth=1.2pt,plotpoints=200]{-2.740390687361422}{4.601600654629923}{-x^(2.0)+2.0*x}
\end{pspicture*}}},				%2. Antwort rechts
								R3={\resizebox{0.6\linewidth}{!}{\psset{xunit=1.0cm,yunit=1.0cm,algebraic=true,dimen=middle,dotstyle=o,dotsize=5pt 0,linewidth=0.8pt,arrowsize=3pt 2,arrowinset=0.25}
\begin{pspicture*}(-2.740390687361422,-4.354125270739658)(4.583693358673846,2.934144183383457)
\multips(0,-4)(0,1.0){8}{\psline[linestyle=dashed,linecap=1,dash=1.5pt 1.5pt,linewidth=0.4pt,linecolor=darkgray]{c-c}(-2.740390687361422,0)(4.583693358673846,0)}
\multips(-2,0)(1.0,0){8}{\psline[linestyle=dashed,linecap=1,dash=1.5pt 1.5pt,linewidth=0.4pt,linecolor=darkgray]{c-c}(0,-4.354125270739658)(0,2.934144183383457)}
\psaxes[labelFontSize=\scriptstyle,xAxis=true,yAxis=true,Dx=1.,Dy=1.,ticksize=-2pt 0,subticks=2]{->}(0,0)(-2.740390687361422,-4.354125270739658)(4.583693358673846,2.934144183383457)[x,140] [y,-40]
\pscustom[fillcolor=black,fillstyle=solid,opacity=0.1]{\psplot{2.}{3.}{-x^(2.0)+2.0*x}\lineto(3.,0)\lineto(2.,0)\closepath}
\pscustom[fillcolor=black,fillstyle=solid,opacity=0.1]{\psplot{1.}{2.}{-x^(2.0)+2.0*x}\lineto(2.,0)\lineto(1.,0)\closepath}
\psplot[linewidth=1.2pt,plotpoints=200]{-2.740390687361422}{4.583693358673846}{-x^(2.0)+2.0*x}
\end{pspicture*}}},				%3. Antwort rechts
								R4={\resizebox{0.6\linewidth}{!}{\psset{xunit=1.0cm,yunit=1.0cm,algebraic=true,dimen=middle,dotstyle=o,dotsize=5pt 0,linewidth=0.8pt,arrowsize=3pt 2,arrowinset=0.25}
\begin{pspicture*}(-2.740390687361422,-4.354125270739658)(4.547878766761693,2.934144183383457)
\multips(0,-4)(0,1.0){8}{\psline[linestyle=dashed,linecap=1,dash=1.5pt 1.5pt,linewidth=0.4pt,linecolor=darkgray]{c-c}(-2.740390687361422,0)(4.547878766761693,0)}
\multips(-2,0)(1.0,0){8}{\psline[linestyle=dashed,linecap=1,dash=1.5pt 1.5pt,linewidth=0.4pt,linecolor=darkgray]{c-c}(0,-4.354125270739658)(0,2.934144183383457)}
\psaxes[labelFontSize=\scriptstyle,xAxis=true,yAxis=true,Dx=1.,Dy=1.,ticksize=-2pt 0,subticks=2]{->}(0,0)(-2.740390687361422,-4.354125270739658)(4.547878766761693,2.934144183383457)[x,140] [y,-40]
\pscustom[fillcolor=black,fillstyle=solid,opacity=0.1]{\psplot{0.}{2.}{-x^(2.0)+2.0*x}\lineto(2.,0)\lineto(0.,0)\closepath}
\psplot[linewidth=1.2pt,plotpoints=200]{-2.740390687361422}{4.547878766761693}{-x^(2.0)+2.0*x}
\end{pspicture*}}},				%4. Antwort rechts
								%% LOESUNG: %%
								A1={F},				% 1. richtige Zuordnung
								A2={E},				% 2. richtige Zuordnung
								A3={C},				% 3. richtige Zuordnung
								A4={A},				% 4. richtige Zuordnung
								}
\end{beispiel}