\section{AN 4.3 - 15 Fl�cheninhaltsberechnung - MC - Matura NT 1 16/17}

\begin{beispiel}[AN 4.3]{1} %PUNKTE DES BEISPIELS
In der nachstehenden Abbildung sind die Graphen der Polynomfunktionen $f$ und $g$ dargestellt. Diese schneiden einander an den Stellen $-3,0$ und $3$ und begrenzen die beiden grau markierten Fl�chenst�cke.

\begin{center}
	\resizebox{0.6\linewidth}{!}{\psset{xunit=1.2cm,yunit=1.2cm,algebraic=true,dimen=middle,dotstyle=o,dotsize=5pt 0,linewidth=1.6pt,arrowsize=3pt 2,arrowinset=0.25}
\begin{pspicture*}(-4.5,-2.9)(4.9,3.84)
\multips(0,-2)(0,1.0){7}{\psline[linestyle=dashed,linecap=1,dash=1.5pt 1.5pt,linewidth=0.4pt,linecolor=lightgray]{c-c}(-4.5,0)(4.9,0)}
\multips(-4,0)(1.0,0){10}{\psline[linestyle=dashed,linecap=1,dash=1.5pt 1.5pt,linewidth=0.4pt,linecolor=lightgray]{c-c}(0,-2.9)(0,3.84)}
\psaxes[labelFontSize=\scriptstyle,xAxis=true,yAxis=true,Dx=1.,Dy=1.,ticksize=-2pt 0,subticks=2]{->}(0,0)(-4.5,-2.9)(4.9,3.84)[x,140] [\text{f(x),g(x)},-40]
\pscustom[linewidth=0.8pt,fillcolor=black,fillstyle=solid,opacity=0.1]{\psplot{-3.}{3.}{0.2*(x+3.0)*x*(x-3.0)}\lineto(3.,0.)\psplot{3.}{-3.}{-0.06*(x+3.0)*(x-3.0)*x*x}\lineto(-3.,0.)\closepath}
\psplot[linewidth=1.2pt,plotpoints=200]{-4.500000000000003}{4.900000000000003}{0.2*(x+3.0)*x*(x-3.0)}
\psplot[linewidth=2.8pt,plotpoints=200]{-4.500000000000003}{4.900000000000003}{-0.06*(x+3.0)*(x-3.0)*x*x}
\rput[tl](2.18,1.68){$g$}
\rput[tl](-1.54,2.46){$f$}
\end{pspicture*}}
\end{center}

Welche der nachstehenden Gleichungen geben den Inhalt $A$ der (gesamten) grau markierten Fl�che an? Kreuze die beiden zutreffenden Gleichungen an!

\multiplechoice[5]{  %Anzahl der Antwortmoeglichkeiten, Standard: 5
				L1={$$A=\left|\int^3_{-3}{(f(x)-g(x))}\,\text{d}x\right|$$},   %1. Antwortmoeglichkeit 
				L2={$$A=2\cdot\int^3_{0}{(g(x)-f(x))}\,\text{d}x$$},   %2. Antwortmoeglichkeit
				L3={$$A=\int^0_{-3}{(f(x)-g(x))}\,\text{d}x+\int^3_0{(g(x)-f(x))\text{d}x}$$},   %3. Antwortmoeglichkeit
				L4={$$A=\left|\int^0_{-3}{(f(x)-g(x))}\,\text{d}x\right|+\int^3_0{(f(x)-g(x))\text{d}x}$$},   %4. Antwortmoeglichkeit
				L5={$$A=\int^0_{-3}{(f(x)-g(x))}\,\text{d}x+\left|\int^3_0{(f(x)-g(x))\text{d}x}\right|$$},	 %5. Antwortmoeglichkeit
				L6={},	 %6. Antwortmoeglichkeit
				L7={},	 %7. Antwortmoeglichkeit
				L8={},	 %8. Antwortmoeglichkeit
				L9={},	 %9. Antwortmoeglichkeit
				%% LOESUNG: %%
				A1=3,  % 1. Antwort
				A2=5,	 % 2. Antwort
				A3=0,  % 3. Antwort
				A4=0,  % 4. Antwort
				A5=0,  % 5. Antwort
				}
\end{beispiel}