\section{AN 3.1 - 4 Ableitungs- und Stammfunktion - MC - Matura NT 2 15/16}

\begin{beispiel}[AN 3.1]{1} %PUNKTE DES BEISPIELS
Es sei $f$ eine Polynomfunktion und $F$ eine ihrer Stammfunktionen.

Kreuze die beiden zutreffenden Aussagen an.\leer

\multiplechoice[5]{  %Anzahl der Antwortmoeglichkeiten, Standard: 5
				L1={Eine Funktion $F$ hei�t Stammfunktion der Funktion $f$, wenn gilt: \mbox{$f(x)=F(X)+c\,(c\in\mathbb{R})$}.},   %1. Antwortmoeglichkeit 
				L2={Eine Funktion $f'$ hei�t Ableitungsfunktion von $f$, wenn gilt: $\int{f(x)dx=f'(x)}$.},   %2. Antwortmoeglichkeit
				L3={Wenn die Funktion $f$ an der Stelle $x_0$ definiert ist, gibt $f'(x_0)$ die Steigung der Tangente an den Graphen von $f$ an dieser Stelle an.},   %3. Antwortmoeglichkeit
				L4={Die Funktion $f$ hat unendlich viele Stammfunktionen, die sich nur durch eine additive Konstante unterscheiden.},   %4. Antwortmoeglichkeit
				L5={Wenn man die Stammfunktion $F$ einmal integriert, dann erh�lt man die Funktion $f$.},	 %5. Antwortmoeglichkeit
				L6={},	 %6. Antwortmoeglichkeit
				L7={},	 %7. Antwortmoeglichkeit
				L8={},	 %8. Antwortmoeglichkeit
				L9={},	 %9. Antwortmoeglichkeit
				%% LOESUNG: %%
				A1=3,  % 1. Antwort
				A2=4,	 % 2. Antwort
				A3=0,  % 3. Antwort
				A4=0,  % 4. Antwort
				A5=0,  % 5. Antwort
				}
\end{beispiel}