\section{AN 1.4 - 6 Nikotin - OA - Matura 2013/14 Haupttermin}

\begin{beispiel}[AN 1.4]{1} %PUNKTE DES BEISPIELS
				Die Nikotinmenge $x$ (in mg) im Blut eine bestimmten Rauchers kann modellhaft durch die Differenzengleichung $x_{n+1}=0,98\cdot x_n+0,03$ ($n$ in Tagen) beschrieben werden.
				
				Gib an, wie viel Milligramm Nikotin t�glich zugef�hrt werden und wie viel Prozent der im K�rper vorhandenen Nikotinmenge t�glich abgebaut werden!
				
				\antwort[\rule{3cm}{0.3pt}]{0,03}\,mg
				
				\antwort[\rule{3cm}{0.3pt}]{2}\,\%
\end{beispiel}