\section{AN 1.4 - 7 Differenzengleichung - OA - Matura 2016/17 - Haupttermin}

\begin{beispiel}[AN 1.4]{1} %PUNKTE DES BEISPIELS
Die nachstehende Tabelle enth�lt Werte einer Gr��e zum Zeitpunkt $n$ $(n \in \mathbb{N})$

\begin{center}
\begin{tabular}{|c|c|}\hline
\cellcolor{black!20} $n$ & \cellcolor{black!20} $x_n$ \\ \hline
0 & 10 \\ \hline
1 & 21 \\ \hline
2 & 43 \\ \hline
3 & 87 \\ \hline
\end{tabular}
\end{center}

Die zeitliche Entwicklung dieser Gr��e kann durch eine Differenzengleichung der Form
$x_{n+1} = a \cdot x_n + b$ beschrieben werden. \leer

Gib die Werte der (reellen) Parameter $a$ und $b$ so an, dass damit das in der Tabelle angegebene zeitliche Verhalten beschrieben wird! \leer

$a=$ \antwort[\rule{5cm}{0.3pt}]{2}

$b=$ \antwort[\rule{5cm}{0.3pt}]{1} 


\end{beispiel}