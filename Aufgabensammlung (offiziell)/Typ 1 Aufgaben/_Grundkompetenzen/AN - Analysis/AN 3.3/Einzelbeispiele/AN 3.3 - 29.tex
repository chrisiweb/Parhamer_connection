\section{AN 3.3 - 29 Wassermenge in einem Beh�lter - MC - Matura 2016/17 - Haupttermin}

\begin{beispiel}[AN 3.3]{1} %PUNKTE DES BEISPIELS
In der nachstehenden Abbildung ist die momentane �nderungsrate $R$ der Wassermenge in einem
Beh�lter (in m$^3$/h) in Abh�ngigkeit von der Zeit $t$ dargestellt.\leer

\begin{center}
\psset{xunit=1.0cm,yunit=1.0cm,algebraic=true,dimen=middle,dotstyle=o,dotsize=5pt 0,linewidth=0.8pt,arrowsize=3pt 2,arrowinset=0.25}
\begin{pspicture*}(-1.12,-1.7)(8.76,3.94)
\multips(0,-1)(0,1.0){6}{\psline[linestyle=dashed,linecap=1,dash=1.5pt 1.5pt,linewidth=0.4pt,linecolor=black!70]{c-c}(0.12,0)(8.76,0)}
\multips(0,0)(1.0,0){10}{\psline[linestyle=dashed,linecap=1,dash=1.5pt 1.5pt,linewidth=0.4pt,linecolor=black!70]{c-c}(0,-1.7)(0,3.94)}
\psaxes[labelFontSize=\scriptstyle,xAxis=true,yAxis=true,Dx=1.,Dy=1.,ticksize=-2pt 0,subticks=0]{->}(0,0)(0,-1.7)(8.76,3.94)[$t$ in h,128] [$R(t)$ in m$^3$/h,-40]
\psplot[plotpoints=200]{0}{8}{0.25*x^(2.0)-2.0*x+3.0}
\end{pspicture*}
\end{center}

\leer

Kreuze die beiden zutreffenden Aussagen �ber die Wassermenge im Beh�lter an. \leer

\multiplechoice[5]{  %Anzahl der Antwortmoeglichkeiten, Standard: 5
				L1={Zum Zeitpunkt $t=6$ befindet sich weniger Wasser
im Beh�lter als zum Zeitpunkt $t=2$.},   %1. Antwortmoeglichkeit 
				L2={Im Zeitintervall $(6; 8)$ nimmt die Wassermenge im
Beh�lter zu.},   %2. Antwortmoeglichkeit
				L3={Zum Zeitpunkt $t=2$ befindet sich kein Wasser im
Beh�lter.},   %3. Antwortmoeglichkeit
				L4={Im Zeitintervall $(0; 2)$ nimmt die Wassermenge im
Beh�lter ab.},   %4. Antwortmoeglichkeit
				L5={Zum Zeitpunkt $t=4$ befindet sich am wenigsten
Wasser im Beh�lter.},	 %5. Antwortmoeglichkeit
				L6={},	 %6. Antwortmoeglichkeit
				L7={},	 %7. Antwortmoeglichkeit
				L8={},	 %8. Antwortmoeglichkeit
				L9={},	 %9. Antwortmoeglichkeit
				%% LOESUNG: %%
				A1=1,  % 1. Antwort
				A2=2,	 % 2. Antwort
				A3=0,  % 3. Antwort
				A4=0,  % 4. Antwort
				A5=0,  % 5. Antwort
				}
\end{beispiel}