\section{AN 3.3 - 5 Steigung einer Funktion - OA - BIFIE}

\begin{beispiel}[AN 3.3]{1} %PUNKTE DES BEISPIELS
				Gegeben ist die Funktion $f$ mit der Gleichung $f(x)=\frac{1}{4}x�+\frac{3}{2}x�+4x+5$.

Berechne den Wert der Steigung der Funktion $f$ an der Stelle $x=2$.
\leer

\antwort{$f'(x)=\frac{3}{4}x�+3x+4$

$f'(2)=\frac{3}{4}\cdot 2�+3\cdot 2+4=13$

Der Wert der Steigung der Funktion $f$ an der Stelle $x=2$ ist 13.}
\end{beispiel}