\section{AN 3.3 - 1 Eigenschaften einer Polynomfunktion - LT - BIFIE}

\begin{beispiel}[AN 3.3]{1} %PUNKTE DES BEISPIELS
Eine Polynomfunktion dritten Grades $f$ hat die Gleichung \mbox{$f(x)=a\cdot x^3 + b \cdot x^2+c\cdot x+d$} mit $a,b,c,d \in \mathbb{R}$ und $a\neq 0$. \leer

\lueckentext{
				text={Die Funktion $f$ besitzt genau eine \gap, weil es genau ein $x \in \mathbb{R}$ gibt, f�r das \gap gilt}, 	%Lueckentext Luecke=\gap
				L1={Nullstelle}, 		%1.Moeglichkeit links  
				L2={lokale Extremstelle}, 		%2.Moeglichkeit links
				L3={Wendestelle}, 		%3.Moeglichkeit links
				R1={$f(x)=0$ und $f'(x)\neq 0$}, 		%1.Moeglichkeit rechts 
				R2={$f'(x)=0$ und $f''(x)=0$}, 		%2.Moeglichkeit rechts
				R3={$f''(x)=0$ und $f'''(x)\neq 0$}, 		%3.Moeglichkeit rechts
				%% LOESUNG: %%
				A1=3,   % Antwort links
				A2=3		% Antwort rechts 
				}
\end{beispiel}