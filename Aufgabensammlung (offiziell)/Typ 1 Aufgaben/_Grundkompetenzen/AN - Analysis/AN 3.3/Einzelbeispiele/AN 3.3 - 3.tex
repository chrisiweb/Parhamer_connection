\section{AN 3.3 - 3 Lokale Extrema - MC - BIFIE}

\begin{beispiel}[AN 3.3]{1} %PUNKTE DES BEISPIELS
				Von einer Polynomfunktion $f$ dritten Grades sind die beiden lokalen Extrempunkte $E_1=(0/-4)$ und $E_2=(4/0)$ bekannt.

Welche Bedingungen m�ssen in diesem Zusammenhang erf�llt sein? Kreuze die zutreffende(n) Aussage(n) an!

\multiplechoice[5]{  %Anzahl der Antwortmoeglichkeiten, Standard: 5
				L1={$f(0)=-4$},   %1. Antwortmoeglichkeit 
				L2={$f'(0)=0$},   %2. Antwortmoeglichkeit
				L3={$f(-4)=0$},   %3. Antwortmoeglichkeit
				L4={$f'(4)=0$},   %4. Antwortmoeglichkeit
				L5={$f''(0)=0$},	 %5. Antwortmoeglichkeit
				L6={},	 %6. Antwortmoeglichkeit
				L7={},	 %7. Antwortmoeglichkeit
				L8={},	 %8. Antwortmoeglichkeit
				L9={},	 %9. Antwortmoeglichkeit
				%% LOESUNG: %%
				A1=1,  % 1. Antwort
				A2=2,	 % 2. Antwort
				A3=4,  % 3. Antwort
				A4=0,  % 4. Antwort
				A5=0,  % 5. Antwort
				}
\end{beispiel}