\section{AN 3.3 - 4 Ermittlung einer Funktionsgleichung - OA - BIFIE}

\begin{beispiel}[AN 3.3]{1} %PUNKTE DES BEISPIELS
				Gegeben ist die Funktion $f$ mit der Gleichung $f(x)=x�+bx+c$ mit $b,c\in\mathbb{R}$.
Der Graph der Funktion $f$ verl�uft durch den Ursprung. Die Steigung der Funktion im Ursprung hat den Wert null.

Ermittle die Werte der Parameter $b$ und $c$ und gib die Gleichung der Funktion $f$ an!
\leer

\antwort{Die Funktion $f$ verl�uft durch den Koordinatenursprung, daher gilt: $f(0)=0\Rightarrow c=0$. Die Steigung der Funktion im Koordinatenursprung hat den Wert null, daher gilt: $f'(0)=0\Rightarrow b=0$.

Die gesuchte Funktionsgleichung lautet daher: $f(x)=x�$.}
\end{beispiel}