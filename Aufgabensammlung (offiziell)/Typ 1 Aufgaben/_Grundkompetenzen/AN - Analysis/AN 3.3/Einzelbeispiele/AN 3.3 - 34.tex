\section{AN 3.3 - 34 - Zweite Ableitung - MC - Matura - 1. NT 2017/18}

\begin{beispiel}[AN 3.3]{1}
Gegeben ist der Graph einer Polynomfunktion $f$ dritten Grades.

\begin{center}
	\resizebox{0.5\linewidth}{!}{\psset{xunit=1.0cm,yunit=0.7cm,algebraic=true,dimen=middle,dotstyle=o,dotsize=5pt 0,linewidth=1.6pt,arrowsize=3pt 2,arrowinset=0.25}
\begin{pspicture*}(-2.8,-3.21875)(6.9,7.1875)
\multips(0,-4)(0,2.0){6}{\psline[linestyle=dashed,linecap=1,dash=1.5pt 1.5pt,linewidth=0.4pt,linecolor=darkgray]{c-c}(-2.8,0)(6.9,0)}
\multips(-2,0)(1.0,0){10}{\psline[linestyle=dashed,linecap=1,dash=1.5pt 1.5pt,linewidth=0.4pt,linecolor=darkgray]{c-c}(0,-3.21875)(0,7.1875)}
\psaxes[labelFontSize=\scriptstyle,xAxis=true,yAxis=true,Dx=1.,Dy=2.,ticksize=-2pt 0,subticks=2]{->}(0,0)(-2.8,-3.21875)(6.9,7.1875)[x,140] [f(x),-40]
\psplot[linewidth=2.pt,plotpoints=200]{-2.799999999999999}{6.900000000000012}{0.2*(x+1.8)*(x-2.7)*(x-5.1)}
\rput[bl](-2.28,-1.78125){$f$}
\psdots[dotsize=5pt 0,dotstyle=*](0.,4.9572)
\rput[bl](0.08,5.21875){\darkgray{$H$}}
\psdots[dotsize=5pt 0,dotstyle=*](2.,1.6492)
\rput[bl](2.08,1.90625){\darkgray{$W$}}
\psdots[dotsize=5pt 0,dotstyle=*](4.,-1.6588)
\rput[bl](4.08,-1.40625){\darkgray{$T$}}
\end{pspicture*}}
\end{center}

Die eingezeichneten Punkte sind der Hochpunkt $H=(0\mid f(0))$, der Wendepunkt $W=(2\mid f(2))$ und der Tiefpunkt $T=(4\mid f(4))$ des Graphen.

Nachstehend sind f�nf Aussagen �ber die zweite Ableitung von $f$ gegeben.\\
Kreuze die beiden zutreffenden Aussagen an!

\multiplechoice[5]{  %Anzahl der Antwortmoeglichkeiten, Standard: 5
				L1={F�r alle $x$ aus dem Intervall $[-1;1]$ gilt: $f''(x)<0$.},   %1. Antwortmoeglichkeit 
				L2={F�r alle $x$ aus dem Intervall $[1;3]$ gilt: $f''(x)<0$.},   %2. Antwortmoeglichkeit
				L3={F�r alle $x$ aus dem Intervall $[3;5]$ gilt: $f''(x)<0$.},   %3. Antwortmoeglichkeit
				L4={$f''(0)=f''(4)$},   %4. Antwortmoeglichkeit
				L5={$f''(2)=0$},	 %5. Antwortmoeglichkeit
				L6={},	 %6. Antwortmoeglichkeit
				L7={},	 %7. Antwortmoeglichkeit
				L8={},	 %8. Antwortmoeglichkeit
				L9={},	 %9. Antwortmoeglichkeit
				%% LOESUNG: %%
				A1=1,  % 1. Antwort
				A2=5,	 % 2. Antwort
				A3=0,  % 3. Antwort
				A4=0,  % 4. Antwort
				A5=0,  % 5. Antwort
				}
\end{beispiel}