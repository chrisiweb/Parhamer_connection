\section{AN 3.3 - 26 Differenzierbare Funktion - MC - Matura 2015/16 - Nebentermin 1}

\begin{beispiel}[AN 3.3]{1} %PUNKTE DES BEISPIELS
Die nachstehende Abbildung zeigt den Ausschnitt eines Graphen einer Polynomfunktion $f$. Die Tangentensteigung an der Stelle $x=6$ ist maximal.\leer



\begin{center}
\resizebox{0.8\linewidth}{!}{\psset{xunit=1.0cm,yunit=1.0cm,algebraic=true,dimen=middle,dotstyle=o,dotsize=5pt 0,linewidth=0.8pt,arrowsize=3pt 2,arrowinset=0.25}
\begin{pspicture*}(-0.7,-0.66)(12.5,12.44)
\multips(0,0)(0,1.0){14}{\psline[linestyle=dashed,linecap=1,dash=1.5pt 1.5pt,linewidth=0.4pt,linecolor=black!60]{c-c}(0,0)(12.5,0)}
\multips(0,0)(1.0,0){14}{\psline[linestyle=dashed,linecap=1,dash=1.5pt 1.5pt,linewidth=0.4pt,linecolor=black!60]{c-c}(0,0)(0,12.44)}
\psaxes[labelFontSize=\scriptstyle,xAxis=true,yAxis=true,Dx=1.,Dy=1.,ticksize=-2pt 0,subticks=2]{->}(0,0)(0.,0.)(12.5,12.44)[x,140] [f(x),-40]
\psplot[linewidth=1.2pt,plotpoints=200]{0}{12}{2.9890266176687913E-5*x^(4.0)-0.012428419189220467*x^(3.0)+0.21617919832944305*x^(2.0)-0.022775063325552704*x}
\rput[tl](7.54,7.5){$f$}
\end{pspicture*}}
\end{center}

Kreuze die beiden f�r die gegebene Funktion $f$ zutreffenden Aussagen an.

\multiplechoice[5]{  %Anzahl der Antwortmoeglichkeiten, Standard: 5
				L1={$f''(6)=0$},   %1. Antwortmoeglichkeit 
				L2={$f''(11)<0$},   %2. Antwortmoeglichkeit
				L3={$f''(2)<f''(10)$},   %3. Antwortmoeglichkeit
				L4={$f'(6)=0$},   %4. Antwortmoeglichkeit
				L5={$f'(7)<f'(10)$},	 %5. Antwortmoeglichkeit
				L6={},	 %6. Antwortmoeglichkeit
				L7={},	 %7. Antwortmoeglichkeit
				L8={},	 %8. Antwortmoeglichkeit
				L9={},	 %9. Antwortmoeglichkeit
				%% LOESUNG: %%
				A1=1,  % 1. Antwort
				A2=2,	 % 2. Antwort
				A3=0,  % 3. Antwort
				A4=0,  % 4. Antwort
				A5=0,  % 5. Antwort
				}
\end{beispiel}