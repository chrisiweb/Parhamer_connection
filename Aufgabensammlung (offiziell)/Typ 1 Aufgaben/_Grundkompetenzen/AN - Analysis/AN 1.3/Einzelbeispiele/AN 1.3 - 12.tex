\section{AN 1.3 - 12 Finanzschulden - OA - Matura 2016/17 - Haupttermin}

\begin{beispiel}[AN 1.3]{1} %PUNKTE DES BEISPIELS
Die Finanzschulden �sterreichs haben im Zeitraum 2000 bis 2010 zugenommen. Im Jahr 2000
betrugen die Finanzschulden �sterreichs $F_0$, zehn Jahre sp�ter betrugen sie $F_1$ (jeweils in Milliarden Euro). \leer

Interpretieren Sie den Ausdruck $\frac{F_1-F_0}{10}$ im Hinblick auf die Entwicklung der Finanzschulden �sterreichs!

\antwort{Der Ausdruck beschreibt die durchschnittliche j�hrliche Zunahme (durchschnittliche j�hrliche �nderung) der Finanzschulden �sterreichs (in Milliarden Euro im angegebenen Zeitraum).}
\end{beispiel}
