\section{AN 1.1 - 3 �nderung der Spannung - OA - BIFIE}

\begin{beispiel}[AN 1.1]{1} %PUNKTE DES BEISPIELS
Die nachstehende Abbildung zeigt den zeitlichen Verlauf $t$ (in s) der Spannung $U$ (in V) w�hrend eines physikalischen Experiments. 

\begin{center}
\psset{xunit=1.0cm,yunit=0.25cm,algebraic=true,dimen=middle,dotstyle=o,dotsize=5pt 0,linewidth=0.8pt,arrowsize=3pt 2,arrowinset=0.25}
\begin{pspicture*}(-0.7,-3.1674818585673132)(10.859013885766789,37.50966662410777)
\multips(0,0)(0,4.0){11}{\psline[linestyle=dashed,linecap=1,dash=1.5pt 1.5pt,linewidth=0.4pt,linecolor=gray]{c-c}(0,0)(10.859013885766789,0)}
\multips(0,0)(1.0,0){13}{\psline[linestyle=dashed,linecap=1,dash=1.5pt 1.5pt,linewidth=0.4pt,linecolor=gray]{c-c}(0,0)(0,37.50966662410777)}
\psaxes[labelFontSize=\scriptstyle,xAxis=true,yAxis=true,Dx=1.,Dy=4.,ticksize=-2pt 0,subticks=2]{->}(0,0)(-1.2785208100687484,-3.1674818585673132)(10.859013885766789,37.50966662410777)[x,140] [y,-40]
\psplot[linewidth=1.2pt,plotpoints=200]{0}{10.859013885766789}{2.2806827703444222E-6*x^(7.0)-1.7282933713131396E-4*x^(6.0)+0.005965768992273854*x^(5.0)-0.09584712783113487*x^(4.0)+0.692048431766916*x^(3.0)-1.9191657168338834*x^(2.0)+2.3785028428305366*x+20.0}
\end{pspicture*}
\end{center}

Ermittle die absolute und die relative �nderung der Spannung w�hrend der ersten 10 Sekunden des Experiments.
\leer

absolute �nderung: \rule{3cm}{0.3pt} V
\leer

relative �nderung: \rule{3cm}{0.3pt} \% 

\antwort{absolute �nderung: 12\,V \\
relative �nderung: 60\,\%}  
\end{beispiel}