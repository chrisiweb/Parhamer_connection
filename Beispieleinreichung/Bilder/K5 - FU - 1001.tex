\section{K5 - FU - 1001 - Quadratische Funktion - MC - Thema Mathematik Schularbeiten 5. Klasse}

\begin{beispiel}[K5 - FU]{1} %PUNKTE DES BEISPIELS
Eine quadratische Funktion $f$ hat allgemein die Form $f(x)=a\cdot x^2 + b \cdot x + c$ mit $a\neq 0$ und $a, b, c \in \mathbb R$. Der Scheitel ist der h�chste bzw. tiefste Punkte des Funktionsgraphen.

Kreuze die richtige(n) Aussage(n) an!

\multiplechoice[5]{  %Anzahl der Antwortmoeglichkeiten, Standard: 5
				L1={F�r $a>0$ und $c<0$ liegt der Scheitel von $f$ unterhalb der $x$-Achse.},   %1. Antwortmoeglichkeit 
				L2={F�r $a<0$ ist der Graph von $f$ eine nach oben offenen Parabel.},   %2. Antwortmoeglichkeit
				L3={Der Graph von $f$ schneidet die $x$-Achse an zwei Stellen.},   %3. Antwortmoeglichkeit
				L4={Die Symmetrieachse des Graphen einer quadratischen Funktion verl�uft parallel zur $y$-Achse durch den Scheitel.},   %4. Antwortmoeglichkeit
				L5={F�r $a<0$ hat die Funktion $f$ zwei Nullstellen.},	 %5. Antwortmoeglichkeit
				L6={},	 %6. Antwortmoeglichkeit
				L7={},	 %7. Antwortmoeglichkeit
				L8={},	 %8. Antwortmoeglichkeit
				L9={},	 %9. Antwortmoeglichkeit
				%% LOESUNG: %%
				A1=1,  % 1. Antwort
				A2=4,	 % 2. Antwort
				A3=0,  % 3. Antwort
				A4=0,  % 4. Antwort
				A5=0,  % 5. Antwort
				}
\end{beispiel}