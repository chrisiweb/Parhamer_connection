\section{AG 1.1 - 1[2] - Rationale Zahlen - MC - BIFIE}

\begin{beispiel}[AG 1.1]{1} %PUNKTE DES BEISPIELS
Das ist die ZWEITE Variation der ersten Aufgabe!!!

				Gegeben sind fünf Zahlen.
				
				Kreuze diejenigen beiden Zahlen an, die aus der Zahlenmenge $\mathbb{Q}$ sind!
				\multiplechoice[5]{  %Anzahl der Antwortmoeglichkeiten, Standard: 5
								L1={$0,4$},   %1. Antwortmoeglichkeit 
								L2={$\sqrt{-8}$},   %2. Antwortmoeglichkeit
								L3={$\frac{\pi}{5}$},   %3. Antwortmoeglichkeit
								L4={$0$},   %4. Antwortmoeglichkeit
								L5={$e^2$},	 %5. Antwortmoeglichkeit
								L6={},	 %6. Antwortmoeglichkeit
								L7={},	 %7. Antwortmoeglichkeit
								L8={},	 %8. Antwortmoeglichkeit
								L9={},	 %9. Antwortmoeglichkeit
								%% LOESUNG: %%
								A1=1,  % 1. Antwort
								A2=4,	 % 2. Antwort
								A3=0,  % 3. Antwort
								A4=0,  % 4. Antwort
								A5=0,  % 5. Antwort
								}				
\end{beispiel}
