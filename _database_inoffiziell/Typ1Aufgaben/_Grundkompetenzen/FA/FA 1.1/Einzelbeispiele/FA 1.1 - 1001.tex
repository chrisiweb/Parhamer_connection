\section{FA 1.1 - 1001 Was ist eine Funktion? - MC - eSquirrel}

\begin{beispiel}[FA 1.1]{1} %PUNKTE DES BEISPIELS
				   Welche der hier dargestellten Zusammenhänge sind Graphen reeller Funktionen? Kreuze die beiden zutreffenden Zusammenhänge an!
					
					\langmultiplechoice[5]{  %Anzahl der Antwortmoeglichkeiten, Standard: 5
									L1={\resizebox{0.6\linewidth}{!}{\psset{xunit=1.0cm,yunit=1.0cm,algebraic=true,dimen=middle,dotstyle=o,dotsize=5pt 0,linewidth=0.8pt,arrowsize=3pt 2,arrowinset=0.25}
\begin{pspicture*}(-3.34,-3.9)(3.9,3.64)
\psaxes[labelFontSize=\scriptstyle,xAxis=true,yAxis=true,Dx=1.,Dy=1.,ticksize=-2pt 0,subticks=2]{->}(0,0)(-3.34,-3.9)(3.9,3.64)[x,140] [y,-40]
\psline[linewidth=2.pt,linecolor=red](2.,-3.9)(2.,3.64)
\rput[bl](2.26,2.46){\red{$f_1$}}
\end{pspicture*}}},   %1. Antwortmoeglichkeit 
									L2={\resizebox{0.6\linewidth}{!}{\newrgbcolor{ffxfqq}{1. 0.4980392156862745 0.}									\psset{xunit=1.0cm,yunit=1.0cm,algebraic=true,dimen=middle,dotstyle=o,dotsize=5pt 0,linewidth=0.8pt,arrowsize=3pt 2,arrowinset=0.25}
\begin{pspicture*}(-3.34,-3.9)(3.9,3.64)
\psaxes[labelFontSize=\scriptstyle,xAxis=true,yAxis=true,Dx=1.,Dy=1.,ticksize=-2pt 0,subticks=2]{->}(0,0)(-3.34,-3.9)(3.9,3.64)[x,140] [y,-40]
\psline[linewidth=2.pt,linecolor=ffxfqq](-1.8,-3.9)(3.2,3.64)
\rput[bl](2.26,1.96){\ffxfqq{$f_2$}}
\end{pspicture*}}},   %2. Antwortmoeglichkeit
									L3={\resizebox{0.6\linewidth}{!}{\psset{xunit=1.0cm,yunit=1.0cm,algebraic=true,dimen=middle,dotstyle=o,dotsize=5pt 0,linewidth=0.8pt,arrowsize=3pt 2,arrowinset=0.25}
\begin{pspicture*}(-3.34,-3.9)(3.9,3.64)
\psaxes[labelFontSize=\scriptstyle,xAxis=true,yAxis=true,Dx=1.,Dy=1.,ticksize=-2pt 0,subticks=2]{->}(0,0)(-3.34,-3.9)(3.9,3.64)[x,140] [y,-40]
\psline[linewidth=2.pt,linecolor=blue](-3.5,0.9)(1,-1.5)
\psline[linewidth=2.pt,linecolor=blue](-1,0.5)(3.5,2.8)
\rput[bl](2.26,1.76){\blue{$f_3$}}
\rput[bl](-1.76,-0.76){\blue{$f_3$}}
\end{pspicture*}}},   %3. Antwortmoeglichkeit
									L4={\resizebox{0.6\linewidth}{!}{\psset{xunit=1.0cm,yunit=1.0cm,algebraic=true,dimen=middle,dotstyle=o,dotsize=5pt 0,linewidth=0.8pt,arrowsize=3pt 2,arrowinset=0.25}
\begin{pspicture*}(-3.34,-3.9)(3.9,3.64)
\psaxes[labelFontSize=\scriptstyle,xAxis=true,yAxis=true,Dx=1.,Dy=1.,ticksize=-2pt 0,subticks=2]{->}(0,0)(-3.34,-3.9)(3.9,3.64)[x,140] [y,-40]
\pscircle[linewidth=2.pt,linecolor=green](0.,0.){2.}
\rput[bl](-1.28,1.14){\green{$f_4$}}
\end{pspicture*}}},   %4. Antwortmoeglichkeit
									L5={\resizebox{0.6\linewidth}{!}{\psset{xunit=1.0cm,yunit=1.0cm,algebraic=true,dimen=middle,dotstyle=o,dotsize=5pt 0,linewidth=0.8pt,arrowsize=3pt 2,arrowinset=0.25}
\begin{pspicture*}(-3.34,-3.9)(3.9,3.64)
\psaxes[labelFontSize=\scriptstyle,xAxis=true,yAxis=true,Dx=1.,Dy=1.,ticksize=-2pt 0,subticks=2]{->}(0,0)(-3.34,-3.9)(3.9,3.64)[x,140] [y,-40]
\psplot[linewidth=2.pt,linecolor=magenta,plotpoints=200]{-3.340000000000001}{3.9000000000000012}{(x+1.0)^(2.0)-2.0}
\rput[bl](-2.92,2.26){\magenta{$f_5$}}
\end{pspicture*}}},	 %5. Antwortmoeglichkeit
									L6={},	 %6. Antwortmoeglichkeit
									L7={},	 %7. Antwortmoeglichkeit
									L8={},	 %8. Antwortmoeglichkeit
									L9={},	 %9. Antwortmoeglichkeit
									%% LOESUNG: %%
									A1=2,  % 1. Antwort
									A2=5,	 % 2. Antwort
									A3=0,  % 3. Antwort
									A4=0,  % 4. Antwort
									A5=0,  % 5. Antwort
									}
\end{beispiel}