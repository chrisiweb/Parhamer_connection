\section{AG 4.1 - K5 - TR - 1001 Cheopspyramide - OA - Thema Mathematik Schularbeiten 5. Klasse}

\begin{beispiel}[K5 - TR]{1} %PUNKTE DES BEISPIELS
Die Cheopspyramide ist eine quadratische Pyramide mit einer H�he von 147\,m und einer Seitenl�nge von 230\,m. \leer

Berechne den Winkel, den eine Seitenfl�che mit der Grundfl�che einschlie�t!

\antwort{
\meinlr{\resizebox{0.8\linewidth}{!}{\newrgbcolor{zzttqq}{0.6 0.2 0.}
\newrgbcolor{qqwuqq}{0. 0.39215686274509803 0.}
\psset{xunit=1.0cm,yunit=1.0cm,algebraic=true,dimen=middle,dotstyle=o,dotsize=5pt 0,linewidth=0.8pt,arrowsize=3pt 2,arrowinset=0.25}
\begin{pspicture*}(-0.32,-2.76)(3.26,1.54)
\pspolygon[linecolor=zzttqq,fillcolor=zzttqq,fillstyle=solid,opacity=0.1](1.5,1.)(1.5,-1.5)(2.5,-1.5)
\psline(1.5,1.)(3.,-1.)
\psline(1.5,1.)(2.,-2.)
\psline(1.5,1.)(0.,-2.)
\psline[linestyle=dashed,dash=2pt 2pt](1.5,1.)(1.,-1.)
\psline(1.5,1.)(1.5,-1.5)
\psline(1.5,-1.5)(2.5,-1.5)
\psline(2.5,-1.5)(1.5,1.)
\psline[linecolor=zzttqq](1.5,1.)(1.5,-1.5)
\psline[linecolor=zzttqq](1.5,-1.5)(2.5,-1.5)
\psline[linecolor=zzttqq](2.5,-1.5)(1.5,1.)
\psline(0.,-2.)(2.,-2.)
\psline(2.,-2.)(3.,-1.)
\psline[linestyle=dashed,dash=2pt 2pt](3.,-1.)(1.,-1.)
\psline[linestyle=dashed,dash=2pt 2pt](1.,-1.)(0.,-2.)
\pscustom[linecolor=qqwuqq,fillcolor=qqwuqq,fillstyle=solid,opacity=0.1]{
\parametricplot{1.9513027039072617}{3.141592653589793}{0.6*cos(t)+2.5|0.6*sin(t)+-1.5}
\lineto(2.5,-1.5)\closepath}
\pscustom[linecolor=qqwuqq,fillcolor=qqwuqq,fillstyle=solid,opacity=0.1]{
\parametricplot{0.0}{1.5707963267948966}{0.4*cos(t)+1.5|0.4*sin(t)+-1.5}
\lineto(1.5,-1.5)\closepath}
\psellipse*[linecolor=qqwuqq,fillcolor=qqwuqq,fillstyle=solid,opacity=1](1.6663780661615406,-1.3336219338384594)(0.04,0.04)
\begin{scriptsize}
\rput[bl](2.2,-1.44){\qqwuqq{$\varphi$}}
\rput[tl](1.52,-1.6){115 m}
\rput[tl](0.78,-0.44){147 m}
\rput[tl](0.76,-2.12){230 m}
\end{scriptsize}
\end{pspicture*}}}{$\tan\varphi=\frac{147}{115} \Rightarrow \varphi\approx 52^\circ$}}
\end{beispiel}