\section{K5 - TR - 1002 Raumdiagonalen des W�rfels - OA - Mathematik verstehen 5}

\begin{beispiel}[K5 - TR]{0} %PUNKTE DES BEISPIELS
				\meinlr[0.2]{Zeige, welchen Winkel $\varphi$ zwei Raumdiagonalen eines W�rfels mit der Kantenl�nge $a$ miteinander einschlie�en.}
				{\newrgbcolor{srsrsr}{0.12941176470588237 0.12941176470588237 0.12941176470588237}
\newrgbcolor{srsrsr}{0.12941176470588237 0.12941176470588237 0.12941176470588237}
\newrgbcolor{srsrsr}{0.12941176470588237 0.12941176470588237 0.12941176470588237}
\psset{xunit=1.0cm,yunit=1.0cm,algebraic=true,dimen=middle,dotstyle=o,dotsize=5pt 0,linewidth=1.6pt,arrowsize=3pt 2,arrowinset=0.25}
\begin{pspicture*}(0.5326070623591328,0.47260706235913136)(6.7534936138241966,6.3178362133734005)
\psline[linewidth=2.pt](1.,1.)(5.,1.)
\psline[linewidth=2.pt](5.,1.)(5.,5.)
\psline[linewidth=2.pt](5.,5.)(1.,5.)
\psline[linewidth=2.pt](1.,5.)(1.,1.)
\psline[linewidth=2.pt](5.,1.)(6.,2.)
\psline[linewidth=2.pt](6.,2.)(6.,6.)
\psline[linewidth=2.pt](6.,6.)(2.,6.)
\psline[linewidth=2.pt](2.,6.)(1.,5.)
\psline[linewidth=2.pt](5.,5.)(6.,6.)
\psline[linewidth=2.pt,linestyle=dashed,dash=4pt 4pt](1.,1.)(2.,2.)
\psline[linewidth=2.pt,linestyle=dashed,dash=4pt 4pt](2.,2.)(2.,6.)
\psline[linewidth=2.pt,linestyle=dashed,dash=4pt 4pt](2.,2.)(6.,2.)
\psline[linewidth=2.8pt,linestyle=dashed,dash=1pt 2pt 4pt 2pt ](5.,1.)(2.,6.)
\psline[linewidth=2.8pt,linestyle=dashed,dash=1pt 2pt 4pt 2pt ](1.,1.)(6.,6.)
\pscustom[linewidth=2.pt,linecolor=srsrsr,fillcolor=srsrsr,fillstyle=solid,opacity=0.1]{
\parametricplot{0.7853981633974483}{2.1112158270654806}{0.4507888805409467*cos(t)+3.5|0.4507888805409467*sin(t)+3.5}
\lineto(3.5,3.5)\closepath}
\begin{scriptsize}
\rput[bl](2.9969196093163077,0.7731329827197622){a}
\rput[bl](5.656574004507893,1.3441322314049606){a}
\rput[bl](6.227573253193092,4.003786626596543){a}
\rput[bl](3.4777610818933176,3.688234410217881){\srsrsr{$\varphi$}}
\end{scriptsize}
\end{pspicture*}}

				\antwort{$\varphi$ = 70,5�}
\end{beispiel}
