\section{AG 3.3 - K5 - VAG2 - 1001 Vektor - MC - Thema Mathematik Schularbeiten 5. Klasse}

\begin{beispiel}[K5 - VAG2]{1} %PUNKTE DES BEISPIELS
Kreuze die beiden zutreffenden Aussagen an.\leer

Für einen Vektor $\vec{a}=\Vek{5}{-2}{}$ gilt:

\multiplechoice[5]{  %Anzahl der Antwortmoeglichkeiten, Standard: 5
				L1={$(-3)\cdot \vec{a}=\Vek{-15}{6}{}$},   %1. Antwortmoeglichkeit 
				L2={$\vec{a}\parallel\Vek{2}{-5}{}$},   %2. Antwortmoeglichkeit
				L3={$\vec{a}\perp \Vek{-5}{2}{}$},   %3. Antwortmoeglichkeit
				L4={$|\vec{a}|=\sqrt{25-4}$},   %4. Antwortmoeglichkeit
				L5={$\vec{a}\cdot \vec{a}\geq 0$},	 %5. Antwortmoeglichkeit
				L6={},	 %6. Antwortmoeglichkeit
				L7={},	 %7. Antwortmoeglichkeit
				L8={},	 %8. Antwortmoeglichkeit
				L9={},	 %9. Antwortmoeglichkeit
				%% LOESUNG: %%
				A1=1,  % 1. Antwort
				A2=5,	 % 2. Antwort
				A3=0,  % 3. Antwort
				A4=0,  % 4. Antwort
				A5=0,  % 5. Antwort
				}
\end{beispiel}