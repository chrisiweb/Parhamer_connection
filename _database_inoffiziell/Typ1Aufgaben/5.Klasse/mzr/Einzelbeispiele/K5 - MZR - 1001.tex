\section{K5 - MZR - 1001 Lichtjahre - OA - Thema Mathematik Schularbeiten 5. Klasse}

\begin{beispiel}[K5 - MZR]{1} %PUNKTE DES BEISPIELS
1 Lichtjahr ist jene Strecke, die das Licht in 1 Jahr bei einer Geschwindigkeit von $3\cdot 10^8\,$m/s zurücklegt. Stelle diese Strecke mittels Gleitkommadarstellung in Metern und in Kilometern dar. \leer

1 Lichtjahr = \antwort[\rule{3cm}{0.3pt}]{$9,4608 \cdot 10^{15}$}\,m

1 Lichtjahr = \antwort[\rule{3cm}{0.3pt}]{$9,4608 \cdot 10^{12}$}\,km \leer

\antwort{Weil 1 Jahr = $60\cdot 60 \cdot \cdot 24 \cdot 365=31\,536\,000\,$s $\Rightarrow$ 1 Lichtjahr = $3 \cdot 10^8 \cdot 31\,536\,000= 9,4608 \cdot 10^{15}$\,m }
\end{beispiel}