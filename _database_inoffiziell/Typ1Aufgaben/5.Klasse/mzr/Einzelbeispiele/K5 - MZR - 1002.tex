\section{K5 - MZR - 1002 Rechnen mit Bruchtermen - OA - Thema Mathematik Schularbeiten 5. Klasse}

\begin{beispiel}[K5 - MZR]{1} %PUNKTE DES BEISPIELS
Beantworte folgende Frage in ganzen Sätzen:

Für welche Rechenoperationen bei Bruchtermen benötigt man einen gemeinsamen Nenner?

\antwort{Einen gemeinsamen Nenner benötigt man beim Addieren und Subtrahieren von Bruchtermen.}
\end{beispiel}