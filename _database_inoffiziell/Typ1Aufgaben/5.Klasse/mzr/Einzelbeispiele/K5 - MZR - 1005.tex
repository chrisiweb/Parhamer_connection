\section{AG 1.1 - 1005 - K5 - Nat�rliche Zahlen - MC - Thema Mathematik Schularbeiten 5. Klasse}

\begin{beispiel}[K5 - MZR]{1} %PUNKTE DES BEISPIELS
Kreuze die beiden richtigen Aussagen an!

\multiplechoice[5]{  %Anzahl der Antwortmoeglichkeiten, Standard: 5
				L1={Jede nat�rliche Zahl hat einen Nachfolger.},   %1. Antwortmoeglichkeit 
				L2={Es gibt keine kleinste nat�rliche Zahl.},   %2. Antwortmoeglichkeit
				L3={Die Summe nat�rlicher Zahlen ist eine nat�rliche Zahl.},   %3. Antwortmoeglichkeit
				L4={Jede nat�rliche Zahl kann als Produkt von Primzahlen geschrieben werden.},   %4. Antwortmoeglichkeit
				L5={Alle nat�rlichen Zahlen sind positiv.},	 %5. Antwortmoeglichkeit
				L6={},	 %6. Antwortmoeglichkeit
				L7={},	 %7. Antwortmoeglichkeit
				L8={},	 %8. Antwortmoeglichkeit
				L9={},	 %9. Antwortmoeglichkeit
				%% LOESUNG: %%
				A1=1,  % 1. Antwort
				A2=3,	 % 2. Antwort
				A3=0,  % 3. Antwort
				A4=0,  % 4. Antwort
				A5=0,  % 5. Antwort
				}
\end{beispiel}