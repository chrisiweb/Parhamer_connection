\section{K7 - KKK -  - 1050 - Parameterdarstellung - OA - Dimensionen Mathematik, Schularbeiten-Trainer 7. Klasse}

\begin{beispiel}[K7 - KKK]{1} %PUNKTE DES BEISPIELS
Die unten abgebildete Kurve $k$ kann in der Form:

$$k=\{X\in\mathbb{R}^2\,|\,X=(f(t)\,|\,g(t)),t\in\mathbb{R}^+\}$$

beschrieben werden. Zu vier Punkten sind die zugeh�rigen Parameterwerte angef�hrt.

\begin{center}
	\resizebox{0.5\linewidth}{!}{\psset{xunit=1.0cm,yunit=1.0cm,algebraic=true,dimen=middle,dotstyle=o,dotsize=5pt 0,linewidth=1.6pt,arrowsize=3pt 2,arrowinset=0.25}
\begin{pspicture*}(-1.3775050694692372,-0.9101529822836161)(10.73995762029876,11.224087615927923)
\multips(0,0)(0,1.0){13}{\psline[linestyle=dashed,linecap=1,dash=1.5pt 1.5pt,linewidth=0.4pt,linecolor=darkgray]{c-c}(-1.3775050694692372,0)(10.73995762029876,0)}
\multips(-1,0)(1.0,0){13}{\psline[linestyle=dashed,linecap=1,dash=1.5pt 1.5pt,linewidth=0.4pt,linecolor=darkgray]{c-c}(0,-0.9101529822836161)(0,11.224087615927923)}
\psaxes[labelFontSize=\scriptstyle,xAxis=true,yAxis=true,Dx=2.,Dy=2.,ticksize=-2pt 0,subticks=2]{->}(0,0)(-1.3775050694692372,-0.9101529822836161)(10.73995762029876,11.224087615927923)[x,140] [y,-40]
\psplot[linewidth=1.2pt,plotpoints=200]{0}{10.73995762029876}{0.25*x^(2.0)}
\rput[tl](0.11847797865026855,0.5){$t=0$}
\rput[tl](1.4274631457548361,1.7){$t=1$}
\rput[tl](3.222642803498243,4.7){$t=2$}
\rput[tl](5.1,9.6){$t=3$}
\begin{scriptsize}
\psdots[dotsize=6pt 0,dotstyle=*](0.,0.)
\psdots[dotsize=6pt 0,dotstyle=*](2.,1.)
\psdots[dotsize=6pt 0,dotstyle=*](4.,4.)
\psdots[dotsize=6pt 0,dotstyle=*](6.,9.)
\end{scriptsize}
\end{pspicture*}}
\end{center}

Gib die Gleichung f�r $f(t)$ und $g(t)$ an.\leer

\antwort{$f(t)=2t$ und $g(t)=t^2$}
				\end{beispiel}