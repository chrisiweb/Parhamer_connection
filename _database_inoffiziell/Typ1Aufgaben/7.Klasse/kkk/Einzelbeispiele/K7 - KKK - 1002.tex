\section{K6 - KKK - 1002 - Eigenschaften einer Ellipse - MC - Dimensionen 7 Schulbuch}

\begin{beispiel}[K6 - KKK]{1} %PUNKTE DES BEISPIELS
				Kreuze die beiden Aussagen an, die f�r Ellipsen in 1. Hauptlage zutreffen.
				
				\multiplechoice[5]{  %Anzahl der Antwortmoeglichkeiten, Standard: 5
								L1={Die Gleichung einer Ellipse in 1. Hauptlage kannst du aufstellen, wenn du die Koordinaten eines Hauptscheitels und eines Brennpunkts kennst.},   %1. Antwortmoeglichkeit 
								L2={Die Gleichung einer Ellipse in 1. Hauptlage kannst du aufstellen, wenn du die Koordinaten der beiden Hauptscheitel kennst.},   %2. Antwortmoeglichkeit
								L3={Aus der Gleichung einer Ellipse kannst du alle erforderlichen Gr��en berechnen, um diese Ellipse zu konstruieren.},   %3. Antwortmoeglichkeit
								L4={Eine Ellipse 1. Hauptlage ist eindeutig bestimmt, wenn die Koordinaten der beiden Brennpunkte gegeben sind.},   %4. Antwortmoeglichkeit
								L5={Aus der Gleichung der Ellipse in 1. Hauptlage ell: $25x^2+36y^2=1$ kannst du unmittelbar ablesen, dass $a=6$ und $b=5$ ist.},	 %5. Antwortmoeglichkeit
								L6={},	 %6. Antwortmoeglichkeit
								L7={},	 %7. Antwortmoeglichkeit
								L8={},	 %8. Antwortmoeglichkeit
								L9={},	 %9. Antwortmoeglichkeit
								%% LOESUNG: %%
								A1=1,  % 1. Antwort
								A2=3,	 % 2. Antwort
								A3=0,  % 3. Antwort
								A4=0,  % 4. Antwort
								A5=0,  % 5. Antwort
								}
\end{beispiel}
