\section{K7 - KKK -  - 1048 - Parameterdarstellung - OA - Dimensionen Mathematik, Schularbeiten-Trainer 7. Klasse}

\begin{beispiel}[K7 - KKK]{1} %PUNKTE DES BEISPIELS
Eine Kurve $k$ wird durch nachfolgende Parameterdarstellung beschrieben:

$$k=\{X\in\mathbb{R}^2|X=(3\cdot\cos(t)\,|\,3\cdot\sin(t)),t\in[0;2\pi]\}$$

Stelle die Kurve im abgebildeten Koordinatensystem dar und begr�nde, dass die Kurve nicht als Graph einer Funktion aufgefasst werden kann.

\begin{center}
	\resizebox{0.5\linewidth}{!}{\psset{xunit=1.0cm,yunit=1.0cm,algebraic=true,dimen=middle,dotstyle=o,dotsize=5pt 0,linewidth=1.6pt,arrowsize=3pt 2,arrowinset=0.25}
\begin{pspicture*}(-3.52,-3.39)(3.84,3.37)
\multips(0,-3)(0,0.5){14}{\psline[linestyle=dashed,linecap=1,dash=1.5pt 1.5pt,linewidth=0.4pt,linecolor=gray]{c-c}(-3.52,0)(3.84,0)}
\multips(-3,0)(0.5,0){15}{\psline[linestyle=dashed,linecap=1,dash=1.5pt 1.5pt,linewidth=0.4pt,linecolor=gray]{c-c}(0,-3.39)(0,3.37)}
\psaxes[labelFontSize=\scriptstyle,xAxis=true,yAxis=true,Dx=1.,Dy=1.,ticksize=-2pt 0,subticks=2]{->}(0,0)(-3.52,-3.39)(3.84,3.37)[x,140] [y,-40]
\antwort{\pscircle[linewidth=0.8pt](0.,0.){3.}}
\end{pspicture*}}
\end{center}

\antwort{$k:x^2+y^2=9$ ist die Gleichung eines Kreises mit Mittelpunkt $M=(0|0)$ und Radius $r=3$.

Dieser Graph kann nicht der Graph einer Funktion sein, weil jedem $x\in(-3;3)$ zwei $y$-Werte zugeordnet werden, was der Definition einer Funktion widerspricht.}
				\end{beispiel}