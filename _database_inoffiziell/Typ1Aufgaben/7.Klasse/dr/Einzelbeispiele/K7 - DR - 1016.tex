\section{AN 3.3 - K7 - DR - 1016 - Extremstellen - OA - Dimensionen Mathematik, Schularbeiten-Trainer 7. Klasse}

\begin{beispiel}[K7 - DR]{1} %PUNKTE DES BEISPIELS
Gegeben ist die Funktion $f$ mit der Funktionsgleichung $f(x)=\frac{1}{4}x^4+x^3-2x^2+5$.

Berechne die Extremstellen der Funktion $f$ ohne Technologieeinsatz.\leer

\antwort{0 ist lokale Maximumstelle

1 ist lokale Minimumstelle

-4 ist lokale Minimumstelle}
				
				\end{beispiel}