\section{K7 - DR -  - 1032 - Differenzierbar - OA - Dimensionen Mathematik, Schularbeiten-Trainer 7. Klasse}

\begin{beispiel}[K7 - DR]{1} %PUNKTE DES BEISPIELS
Eine reelle Funktion $f$ ist wie folgt abschnittsweise definiert:

$f(x)=\begin{cases}2x&x\leq 0\\
ax^2+bx,&0<x<1\\
c,&x\geq 1\\
\end{cases}$

F�r die Parameter gilt dabei: $a,b,c\in\mathbb{R}$.

Ermittle die Werte der Parameter $a,b$ und $c$ so, dass die Funktion $f$ im gesamten Definitionsbereich $D=\mathbb{R}$ differenzierbar ist.\leer

\antwort{Damit die Funktion an der Stelle 0 knickfrei und somit differenzierbar ist, muss die "`mittlere"' Funktion $f$ an der Stelle 0 die Steigung 2 aufweisen, da die "`linke"' Funktion eine lineare Funktion mit Steigung 2 ist.

�hnliches glt f�r die Stelle 1. Hier muss die Tangensteigung 0 betragen, da dir rechte Funktion eine konstante Funktion ist, die �berall die Tangentensteigung 0 hat.

Daher:

$f(x)=\begin{cases}2x&x\leq 0\\
-x^2+2x,&0<x<1\\
1,&x\geq 1\\
\end{cases}$}
				\end{beispiel}