\section{K7 - DR - 1005 Graph einer Funktion - MC - Thema Mathematik Schularbeiten 7. Klasse}

\begin{beispiel}[K7 - DR]{1} %PUNKTE DES BEISPIELS
			Die Abbildung zeigt den Graphen der Funktion $f$ mit der Gleichung $f(x)=4-x�$.
			
			\begin{center}
				\resizebox{0.4\linewidth}{!}{\psset{xunit=1.0cm,yunit=1.0cm,algebraic=true,dimen=middle,dotstyle=o,dotsize=4pt 0,linewidth=0.8pt,arrowsize=3pt 2,arrowinset=0.25}
\begin{pspicture*}(-3.54,-1.6)(3.78,4.72)
\multips(0,-1)(0,1.0){7}{\psline[linestyle=dashed,linecap=1,dash=1.5pt 1.5pt,linewidth=0.4pt,linecolor=lightgray]{c-c}(-3.54,0)(3.78,0)}
\multips(-3,0)(1.0,0){8}{\psline[linestyle=dashed,linecap=1,dash=1.5pt 1.5pt,linewidth=0.4pt,linecolor=lightgray]{c-c}(0,-1.6)(0,4.72)}
\psaxes[labelFontSize=\scriptstyle,xAxis=true,yAxis=true,Dx=1.,Dy=1.,ticksize=-2pt 0,subticks=2]{->}(0,0)(-3.54,-1.6)(3.78,4.72)[x,140] [y,-40]
\psplot[linewidth=1.2pt,plotpoints=200]{-3.540000000000002}{3.7800000000000007}{-x^(2.0)+4.0}
\rput[tl](1.08,3.6){f}
\end{pspicture*}}
			\end{center}
			
			Kreuze die Aussage(n) an, die f�r die gegebene Funktion zutreffend sind!\leer
			
			\multiplechoice[5]{  %Anzahl der Antwortmoeglichkeiten, Standard: 5
							L1={Die absolute �nderung in den Intervallen $[-2;-1]$ und $[1;2]$ ist gleich gro�.},   %1. Antwortmoeglichkeit 
							L2={Die mittlere �nderungsrate im Intervall $[-1;1]$ ist gleich der momentanen �nderungsrate an der Stelle $x=0$.},   %2. Antwortmoeglichkeit
							L3={Die momentane �nderungsrate an der Stelle $x=2$ ist negativ.},   %3. Antwortmoeglichkeit
							L4={Die momentane �nderungsrate an der Stelle $x=-2$ ist kleiner als die momentane �nderungsrate an der Stelle $x=1$.},   %4. Antwortmoeglichkeit
							L5={Die Steigung der Sekante durch die Punkte $A(-2|0)$ und $B(1|3)$ ist gr��er als die momentane �nderungsrate an der Stelle $x=1$.},	 %5. Antwortmoeglichkeit
							L6={},	 %6. Antwortmoeglichkeit
							L7={},	 %7. Antwortmoeglichkeit
							L8={},	 %8. Antwortmoeglichkeit
							L9={},	 %9. Antwortmoeglichkeit
							%% LOESUNG: %%
							A1=2,  % 1. Antwort
							A2=3,	 % 2. Antwort
							A3=5,  % 3. Antwort
							A4=0,  % 4. Antwort
							A5=0,  % 5. Antwort
							}
			\end{beispiel}