\section{K7 - DR - 1002 Graph einer Funktion - MC - Thema Mathematik Schularbeiten 7. Klasse}

\begin{beispiel}[K7 - DR]{1} %PUNKTE DES BEISPIELS
			  Gegeben ist der Graph einer Polynomfunktion $f$ vierten Grades im Intervall $[-3;2]$.
				
				Kreuze die zutreffende(n) Aussage(n) an!
				
				\meinlr{\multiplechoice[5]{  %Anzahl der Antwortmoeglichkeiten, Standard: 5
								L1={$f$ ist im Intervall $[-3;-1]$ positiv gekr�mmt.},   %1. Antwortmoeglichkeit 
								L2={$f$ ist im Intervall $[-2;0]$ positiv gekr�mmt.},   %2. Antwortmoeglichkeit
								L3={$f$ ist im Intervall $[0;2]$ negativ gekr�mmt.},   %3. Antwortmoeglichkeit
								L4={Im Intervall $[-3;2]$ �ndert sich das Kr�mmungsverhalten der Funktion $f$ mehrmals.},   %4. Antwortmoeglichkeit
								L5={Im Intervall $[-3;-1]$ �ndert sich das Kr�mmungsverhalten der Funktion $f$.},	 %5. Antwortmoeglichkeit
								L6={},	 %6. Antwortmoeglichkeit
								L7={},	 %7. Antwortmoeglichkeit
								L8={},	 %8. Antwortmoeglichkeit
								L9={},	 %9. Antwortmoeglichkeit
								%% LOESUNG: %%
								A1=2,  % 1. Antwort
								A2=3,	 % 2. Antwort
								A3=4,  % 3. Antwort
								A4=5,  % 4. Antwort
								A5=0,  % 5. Antwort
								}}{\vspace{1cm}
								\begin{center}
					\resizebox{0.8\linewidth}{!}{\psset{xunit=1.0cm,yunit=1.0cm,algebraic=true,dimen=middle,dotstyle=o,dotsize=4pt 0,linewidth=0.8pt,arrowsize=3pt 2,arrowinset=0.25}
\begin{pspicture*}(-3.8556976473324927,-5.602289696150681)(3.751037706475264,3.608709748862973)
\multips(0,-5)(0,1.0){10}{\psline[linestyle=dashed,linecap=1,dash=1.5pt 1.5pt,linewidth=0.4pt,linecolor=lightgray]{c-c}(-3.8556976473324927,0)(3.751037706475264,0)}
\multips(-3,0)(1.0,0){8}{\psline[linestyle=dashed,linecap=1,dash=1.5pt 1.5pt,linewidth=0.4pt,linecolor=lightgray]{c-c}(0,-5.602289696150681)(0,3.608709748862973)}
\psaxes[labelFontSize=\scriptstyle,xAxis=true,yAxis=true,Dx=1.,Dy=1.,ticksize=-2pt 0,subticks=2]{->}(0,0)(-3.8556976473324927,-5.602289696150681)(3.751037706475264,3.608709748862973)[x,140] [y,-40]
\psplot[linewidth=1.2pt,plotpoints=200]{-3.8556976473324927}{3.751037706475264}{-0.24536523768700924*x^(4.0)-0.9953912860045361*x^(3.0)-0.018539049251963063*x^(2.0)+3.9815651440181443*x}
\rput[tl](1.6600867561394823,1.7881403869326815){f}
\end{pspicture*}}
				\end{center}}
\end{beispiel}