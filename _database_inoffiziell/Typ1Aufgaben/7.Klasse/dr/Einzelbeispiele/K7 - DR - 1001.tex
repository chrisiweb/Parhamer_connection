\section{K7 - DR - 1001 �nderungsraten - MC - Thema Mathematik Schularbeiten 7. Klasse}

\begin{beispiel}[K7 - DR]{1} %PUNKTE DES BEISPIELS
			  Die Abbildung zeigt den Graphen der Funktion $f$ mit der Gleichung\\ 
				$f(x)=0,1x^2$.
				
				Kreuze die beiden Aussagen an, die f�r die gegebene Funktion zutreffend sind!
				
				\meinlr{\multiplechoice[5]{  %Anzahl der Antwortmoeglichkeiten, Standard: 5
								L1={Die absolute �nderung in den Intervallen $[0;3]$ und $[4;5]$ ist gleich gro�.},   %1. Antwortmoeglichkeit 
								L2={Die mittlere �nderungsrate in den Intervallen $[0;2]$ und $[2;4]$ ist gleich.},   %2. Antwortmoeglichkeit
								L3={Die momentane �nderungsrate an der Stelle $x=5$ hat den Wert 2,5.},   %3. Antwortmoeglichkeit
								L4={Die momentane �nderungsrate an der Stelle $x=2$ ist gr��er als die momentane �nderungsrate an der Stelle $x=6$.},   %4. Antwortmoeglichkeit
								L5={Die Steigung der Sekante durch die Punkte $A(3|f(3))$ und $B(6|f(6))$ ist gr��er als die momentane �nderungsrate an der Stelle $x=3$.},	 %5. Antwortmoeglichkeit
								L6={},	 %6. Antwortmoeglichkeit
								L7={},	 %7. Antwortmoeglichkeit
								L8={},	 %8. Antwortmoeglichkeit
								L9={},	 %9. Antwortmoeglichkeit
								%% LOESUNG: %%
								A1=1,  % 1. Antwort
								A2=5,	 % 2. Antwort
								A3=0,  % 3. Antwort
								A4=0,  % 4. Antwort
								A5=0,  % 5. Antwort
								}}{
								\vspace{4cm}
								
								\begin{center}
								\resizebox{0.8\linewidth}{!}{\psset{xunit=1.0cm,yunit=1.0cm,algebraic=true,dimen=middle,dotstyle=o,dotsize=4pt 0,linewidth=0.8pt,arrowsize=3pt 2,arrowinset=0.25}
								\begin{pspicture*}(-0.5,-0.62)(8.56,6.56)
								\multips(0,0)(0,1.0){8}{\psline[linestyle=dashed,linecap=1,dash=1.5pt 1.5pt,linewidth=0.4pt,linecolor=lightgray]{c-c}(0,0)(8.56,0)}
\multips(0,0)(1.0,0){10}{\psline[linestyle=dashed,linecap=1,dash=1.5pt 1.5pt,linewidth=0.4pt,linecolor=lightgray]{c-c}(0,0)(0,6.56)}
\psaxes[labelFontSize=\scriptstyle,xAxis=true,yAxis=true,Dx=1.,Dy=1.,ticksize=-2pt 0,subticks=2]{->}(0,0)(0.,0.)(8.56,6.56)[$x$,140] [$f(x)$,-40]
\psplot[linewidth=1.2pt,plotpoints=200]{0}{8.560000000000006}{0.1*x^(2.0)}
\begin{scriptsize}
\rput[bl](4,2){$f$}
\end{scriptsize}
\end{pspicture*}}
\end{center}
}
\end{beispiel}