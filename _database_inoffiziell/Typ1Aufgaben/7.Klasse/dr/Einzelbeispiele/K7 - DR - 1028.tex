\section{K7 - DR -  - 1028 - Stetig - OA - Dimensionen Mathematik, Schularbeiten-Trainer 7. Klasse}

\begin{beispiel}[K7 - DR]{1} %PUNKTE DES BEISPIELS
Gegeben ist der Graph einer Funktion $f$. Der Definitionsbereich der Funktion ist $D=[-2;4]\backslash \{2\}$.

\begin{center}
	\resizebox{0.5\linewidth}{!}{\psset{xunit=1.0cm,yunit=1.0cm,algebraic=true,dimen=middle,dotstyle=o,dotsize=5pt 0,linewidth=1.6pt,arrowsize=3pt 2,arrowinset=0.25}
\begin{pspicture*}(-2.1428025477706947,-2.420573248407636)(4.95452229299362,4.835477707006365)
\multips(0,-2)(0,0.5){15}{\psline[linestyle=dashed,linecap=1,dash=1.5pt 1.5pt,linewidth=0.4pt,linecolor=gray]{c-c}(-2.1428025477706947,0)(4.95452229299362,0)}
\multips(-2,0)(0.5,0){15}{\psline[linestyle=dashed,linecap=1,dash=1.5pt 1.5pt,linewidth=0.4pt,linecolor=gray]{c-c}(0,-2.420573248407636)(0,4.835477707006365)}
\psaxes[labelFontSize=\scriptstyle,xAxis=true,yAxis=true,Dx=1.,Dy=1.,ticksize=-2pt 0,subticks=2]{->}(0,0)(-2.1428025477706947,-2.420573248407636)(4.95452229299362,4.835477707006365)[x,140] [f(x),-40]
\psplot[linewidth=1pt]{-2.1428025477706947}{0}{(--1.-0.*x)/1.}
\psplot[linewidth=1pt,plotpoints=200]{0}{4.95452229299362}{x^(2.0)+1.0}
\psplot[linewidth=1pt,plotpoints=200]{2}{3}{-(x-3.0)^(2.0)}
\psplot[linewidth=1pt]{3}{4.95452229299362}{(--3.-1.*x)/0.5}
\begin{scriptsize}
\psdots[dotstyle=*](-1.,0.)
\psdots(-1.,1.)
\psdots(2.,-1.)
\end{scriptsize}
\end{pspicture*}}
\end{center}

Gib alle Unstetigkeitsstellen der Funktion $f$ an.\leer

\antwort{$x_1=-1$

Da $x_2=2$ nicht zur Definitionsmenge geh�rt, kann man diese Stelle auch nicht als Unstetigkeitsstelle bezeichnen.}
				
				\end{beispiel}