\section{K7 - DR - AG-L 2.9 - 1014 - Substitution - OA - Dimensionen Mathematik, Schularbeiten-Trainer 7. Klasse}

\begin{beispiel}[K7 - DR]{1} %PUNKTE DES BEISPIELS
Gegeben ist die Funktion $f$ mit der Funktionsgleichung $f(x)=x^4-8x^2+7$.

Berechne die Nullstellen der Funktion $f$ ohne Technologieeinsatz.\leer

\antwort{$x_{1;2}=\pm\sqrt{7}, x_{3;4}=\pm 1$}
				
				\end{beispiel}