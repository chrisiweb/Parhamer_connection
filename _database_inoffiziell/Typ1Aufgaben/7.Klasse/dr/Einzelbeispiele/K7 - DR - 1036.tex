\section{K7 - DR -  - 1036 - Zusammenhang stetig und differenzierbar - MC - Dimensionen Mathematik, Schularbeiten-Trainer 7. Klasse}

\begin{beispiel}[K7 - DR]{1} %PUNKTE DES BEISPIELS
Gegeben ist der Graph einer Funktion $f$.

\begin{center}
	\resizebox{0.5\linewidth}{!}{\psset{xunit=1.0cm,yunit=1.0cm,algebraic=true,dimen=middle,dotstyle=o,dotsize=5pt 0,linewidth=0.6pt,arrowsize=3pt 2,arrowinset=0.25}
\begin{pspicture*}(-2.48,-1.35)(4.8,3.89)
\multips(0,-1)(0,0.5){11}{\psline[linestyle=dashed,linecap=1,dash=1.5pt 1.5pt,linewidth=0.4pt,linecolor=gray]{c-c}(-2.48,0)(4.8,0)}
\multips(-2,0)(0.5,0){15}{\psline[linestyle=dashed,linecap=1,dash=1.5pt 1.5pt,linewidth=0.4pt,linecolor=gray]{c-c}(0,-1.35)(0,3.89)}
\psaxes[labelFontSize=\scriptstyle,xAxis=true,yAxis=true,Dx=1.,Dy=1.,ticksize=-2pt 0,subticks=2]{->}(0,0)(-2.48,-1.35)(4.8,3.89)[x,140] [f(x),-40]
\psplot[linewidth=1.2pt,plotpoints=200]{-2.480000000000002}{0}{-x^(2.0)+3.0}
\psplot[linewidth=1.2pt]{0}{1}{(--1.5-1.*x)/0.5}
\psplot[linewidth=1.2pt]{1}{4.8}{(-1.--1.*x)/1.}
\rput[tl](2.6,1.53){f}
\begin{scriptsize}
\psdots[dotsize=4pt 0](1.,1.)
\end{scriptsize}
\end{pspicture*}}
\end{center}

Kreuze die zutreffende(n) Aussage(n) an.\leer

\multiplechoice[5]{  %Anzahl der Antwortmoeglichkeiten, Standard: 5
				L1={Die Funktion $f$ weist im Intervall $[-1;2]$ genau eine Stelle auf, an der sie nicht differenzierbar ist.},   %1. Antwortmoeglichkeit 
				L2={Die Funktion $f$ weist im Intervall $[-1;3]$ genau eine Unstetigkeitsstelle auf.},   %2. Antwortmoeglichkeit
				L3={Die Funktion $f$ ist im gesamten dargestellten Bereich differenzierbar.},   %3. Antwortmoeglichkeit
				L4={Die Funktion $f$ weist im Intervall $[-1;3]$ genau zwei Stellen auf, an denen sie nicht differenzierbar ist.},   %4. Antwortmoeglichkeit
				L5={Die Funktion $f$ weist im Intervall $[-1;3]$ genau zwei Stellen auf, an denen sie nicht stetig ist.},	 %5. Antwortmoeglichkeit
				L6={},	 %6. Antwortmoeglichkeit
				L7={},	 %7. Antwortmoeglichkeit
				L8={},	 %8. Antwortmoeglichkeit
				L9={},	 %9. Antwortmoeglichkeit
				%% LOESUNG: %%
				A1=2,  % 1. Antwort
				A2=4,	 % 2. Antwort
				A3=0,  % 3. Antwort
				A4=0,  % 4. Antwort
				A5=0,  % 5. Antwort
				}
				\end{beispiel}