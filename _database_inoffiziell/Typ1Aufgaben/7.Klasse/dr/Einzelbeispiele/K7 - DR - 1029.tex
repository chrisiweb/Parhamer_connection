\section{K7 - DR -  - 1029 - Stetig - LT - Dimensionen Mathematik, Schularbeiten-Trainer 7. Klasse}

\begin{beispiel}[K7 - DR]{1} %PUNKTE DES BEISPIELS
Gegeben ist die Funktion $f$ mit $f(x)=\begin{cases}-1,&x\leq 0\\
g(x),&0<x<2\\
3,&x\geq 2\\
\end{cases}$

\lueckentext{
				text={F�r \gap weist die Funktion $f$ \gap auf.}, 	%Lueckentext Luecke=\gap
				L1={$g(x)=x-1$}, 		%1.Moeglichkeit links  
				L2={$g(x)=x^2-1$}, 		%2.Moeglichkeit links
				L3={$g(x)=x^3-1$}, 		%3.Moeglichkeit links
				R1={genau die eine Unstetigkeitsstelle $x=0$}, 		%1.Moeglichkeit rechts 
				R2={die zwei Unstetigkeitsstellen $x_1=0$ und $x_2=2$}, 		%2.Moeglichkeit rechts
				R3={keine Unstetigkeitsstelle}, 		%3.Moeglichkeit rechts
				%% LOESUNG: %%
				A1=2,   % Antwort links
				A2=3		% Antwort rechts 
				}
				
				\end{beispiel}