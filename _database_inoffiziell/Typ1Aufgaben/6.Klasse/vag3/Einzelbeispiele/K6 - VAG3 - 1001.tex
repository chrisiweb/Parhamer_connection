\section{AG 3.3 - K6 - VAG3 - 1001 Normale Vektoren - MC - Thema Mathematik Schularbeiten 6. Klasse}

\begin{beispiel}[K6 - VAG3]{1} %PUNKTE DES BEISPIELS
			Die dreidimensionalen Vektoren $\vec{a}$ und $\vec{n}$ stehen aufeinander normal.
			
			Kreuze die beiden richtigen Aussagen an!
			
			\multiplechoice[5]{  %Anzahl der Antwortmoeglichkeiten, Standard: 5
							L1={Wenn $\vec{a}=\Vek{x}{y}{z}$ ist, dann ist $\vec{n}$ parallel zu $\Vek{y}{-x}{z}$.},   %1. Antwortmoeglichkeit 
							L2={Das skalare Produkt der Vektoren ist null.},   %2. Antwortmoeglichkeit
							L3={Wenn $\vec{a}=\Vek{x}{y}{0}$ ist, dann ist $\vec{n}$ parallel zu $\Vek{y}{-x}{0}$.},   %3. Antwortmoeglichkeit
							L4={$\vec{a}\cdot\vec{n}=\Vek{0}{0}{0}$},   %4. Antwortmoeglichkeit
							L5={$\left|\vec{a}\right|=\left|\vec{n}\right|$},	 %5. Antwortmoeglichkeit
							L6={},	 %6. Antwortmoeglichkeit
							L7={},	 %7. Antwortmoeglichkeit
							L8={},	 %8. Antwortmoeglichkeit
							L9={},	 %9. Antwortmoeglichkeit
							%% LOESUNG: %%
							A1=2,  % 1. Antwort
							A2=4,	 % 2. Antwort
							A3=0,  % 3. Antwort
							A4=0,  % 4. Antwort
							A5=0,  % 5. Antwort
							}
							\end{beispiel}