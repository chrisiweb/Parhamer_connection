\section{K6 - RF - 1006 Umkehrfunktion eines Graphen - OA - Thema Mathematik Schularbeiten 6. Klasse}

\begin{beispiel}[K6 - RF]{1} %PUNKTE DES BEISPIELS
			Eine Funktion $f$ ist durch ihren Graphen gegeben:
			\begin{center}
				\resizebox{0.7\linewidth}{!}{\psset{xunit=1.0cm,yunit=1.0cm,algebraic=true,dimen=middle,dotstyle=o,dotsize=4pt 0,linewidth=1.6pt,arrowsize=3pt 2,arrowinset=0.25}
\begin{pspicture*}(-8.12,-1.5)(7.6,5.52)
\psaxes[labelFontSize=\scriptstyle,xAxis=true,yAxis=true,labels=none,Dx=1.,Dy=1.,ticksize=-2pt 0,subticks=2]{->}(0,0)(-8.12,-1.5)(7.6,5.52)[x,140] [y,-40]
\psplot[linewidth=1.2pt,plotpoints=200]{-8.120000000000001}{7.600000000000005}{-0.10078105316200554*x^(2.0)+4.0}
\antwort{\psline[linewidth=0.8pt,linestyle=dashed,dash=2pt 2pt](-4.,2.3875031494079115)(4.,2.3875031494079115)
\psline[linewidth=0.8pt](4.,2.3875031494079115)(4.,0.)
\psline[linewidth=0.8pt](-4.,2.3875031494079115)(-4.,0.)}
\antwort{\rput[tl](-4.1,-0.15){a}
\rput[tl](4,-0.15){b}}
\rput[bl](-7.28,-0.94){$f$}
\end{pspicture*}}
			\end{center}
			
			Existiert f�r die Funktion $f$ eine Umkehrfunktion? Begr�nde deine Entscheidung!\leer
			
			\antwort{Die Grafik zeigt zwei Stellen $a$ und $b$, die denselben Funktionswert haben: $f(a)=f(b)$. Daher existiert keine Umkehrfunktion.}
							\end{beispiel}