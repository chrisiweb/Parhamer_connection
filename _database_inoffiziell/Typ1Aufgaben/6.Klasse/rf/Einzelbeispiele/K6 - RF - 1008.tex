\section{K6 - RF - 1008 Diskretes logistisches Wachstum - OA - Mathematik verstehen 6}

\begin{beispiel}[K6 - RF]{1} %PUNKTE DES BEISPIELS
				Es sei $N(n)$ die Einwohnerzahl einer Stadt nach $n$ Jahren. Man nimmt an, dass in dieser Stadt nicht mehr als 15.000 Einwohner Platz haben. Berechne $N(n)$ f�r $n=0,1,2$ unter der Annahme, dass das logistische Wachstum \mbox{($N(n+1)=N(n)+C\cdot N(n)\cdot [K-N(n)]$)} mit folgenden Werten vorliegt!
						$$N(0)=2500, C=0,000\,03$$\leer
						
						\antwort{$N(0)=2500$
						
						$N(1)=2500+0,000\,03\cdot 2500\cdot[15\,000-2500]=3437,5$
						
						$N(2)=3437,5+0,000\,03\cdot 3437,5\cdot[15\,000-3437,5]\approx 4629,883$}
\end{beispiel}