\section{K8 - DDG - 1010 - Koordinatensystem - OA - Dimensionen Mathematik 8 - Schularbeiten-Trainer}

\begin{beispiel}[K8 - DDG]{1}
Im abgebildeten Koordinatensystem sind sechs Punkte samt durch die jeweiligen Punkte verlaufenden Geraden-Ausschnitte eingezeichnet. Der Graph einer Polynomfunktion $f$ zweiten Grades kann so eingezeichnet werden, dass er durch vier der gekennzeichneten Punkte verl�uft und die Geraden-Ausschnitte Tangenten an den Funktionsgraphen in den jeweiligen Punkten darstellen.
\begin{center}
	\resizebox{0.5\linewidth}{!}{\newrgbcolor{ududff}{0.30196078431372547 0.30196078431372547 1.}
\psset{xunit=1.0cm,yunit=1.0cm,algebraic=true,dimen=middle,dotstyle=o,dotsize=5pt 0,linewidth=1.6pt,arrowsize=3pt 2,arrowinset=0.25}
\begin{pspicture*}(-2.8,-1.24)(5.78,5.3)
\multips(0,-1)(0,0.5){14}{\psline[linestyle=dashed,linecap=1,dash=1.5pt 1.5pt,linewidth=0.4pt,linecolor=darkgray]{c-c}(-2.8,0)(5.78,0)}
\multips(-2,0)(0.5,0){18}{\psline[linestyle=dashed,linecap=1,dash=1.5pt 1.5pt,linewidth=0.4pt,linecolor=darkgray]{c-c}(0,-1.24)(0,5.3)}
\psaxes[labelFontSize=\scriptstyle,xAxis=true,yAxis=true,Dx=1.,Dy=1.,ticksize=-2pt 0,subticks=2]{->}(0,0)(-2.8,-1.24)(5.78,5.3)[x,140] [f(x),-40]
\psline[linewidth=2.8pt,linecolor=red](-1.5,0.5)(-2.5,1.5)
\psline[linewidth=2.8pt,linecolor=red](-1.5,1.5)(-2.5,2.5)
\psline[linewidth=2.8pt,linecolor=red](-0.5,0.)(0.5,0.)
\psline[linewidth=2.8pt,linecolor=red](1.5,-0.5)(2.5,0.5)
\psline[linewidth=2.8pt,linecolor=red](1.5,0.5)(2.5,1.5)
\psline[linewidth=2.8pt,linecolor=red](3.5,3.)(4.5,5.)
\begin{scriptsize}
\psdots[dotstyle=*,linecolor=ududff](0.,0.)
\psdots[dotstyle=*,linecolor=ududff](-2.,1.)
\psdots[dotstyle=*,linecolor=ududff](-2.,2.)
\psdots[dotstyle=*,linecolor=ududff](4.,4.)
\psdots[dotstyle=*,linecolor=ududff](2.,0.)
\psdots[dotstyle=*,linecolor=ududff](2.,1.)
\end{scriptsize}
\end{pspicture*}}
\end{center}

Vervollst�ndige gem�� der Abbildung die nachfolgende Tabelle und beschreibe die Abh�ngigkeit der $f'(x)$-Werte von den zugeh�rigen $x$- und/oder $f(x)$-Werten verbal.

\begin{center}
	\begin{tabular}{|l|c|c|c|c|c|c|}\hline
	\cellcolor[gray]{0.9}$x$&$-2$&$-2$&$0$&\antwort{$2$}&\antwort{$2$}&\antwort{$4$}\\ \hline
	\cellcolor[gray]{0.9}$f(x)$&$2$&\antwort{$1$}&\antwort{$0$}&\antwort{$0$}&\antwort{$1$}&\antwort{$4$}\\ \hline
	\cellcolor[gray]{0.9}$f'(x)$&\antwort{$-1$}&\antwort{$-1$}&\antwort{$0$}&\antwort{$1$}&\antwort{$1$}&\antwort{2}\\ \hline
	\end{tabular}
\end{center}

Gib anschlie�end eine Differentialgleichung f�r die Funktion $f$ an und ermittle eine Funktionsgleichung von $f$.

\antwort{Die Werte der Funktion $f'$ ergeben sich durch Halbierung des zugeh�rigen $x$-Wertes, d.h. $f'(x)=\frac{1}{2}x \Rightarrow f(x)=\frac{1}{4}x^2$}
\end{beispiel}