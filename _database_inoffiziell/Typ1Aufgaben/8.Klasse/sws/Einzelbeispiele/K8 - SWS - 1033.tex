\section{K8 - SWS - 1033 - Partei-A-W�hler - MC - Dimensionen Mathematik 8 - Schularbeiten-Trainer}

\begin{beispiel}[K8 - SWS]{1}
Gegeben ist eine Hypothese �ber den Anteil $p_A$ der "`Partei-A-W�hler"' in der Grundmenge aller Personen, die bei einer bestimmten Wahl wahlberechtigt sind. Die Entscheidung wird auf Grundlage der Anzahl $x_A$ jener Personen getroffen, die in einer Stichprobe der Gr��e $n=300$ angegeben, dass sie die Partei A pr�ferieren. Die Grafik unten veranschaulicht den Sachverhalt.
\begin{center}
\resizebox{0.5\linewidth}{!}{\psset{xunit=0.14cm,yunit=50.0cm,algebraic=true,dimen=middle,dotstyle=o,dotsize=5pt 0,linewidth=1.6pt,arrowsize=3pt 2,arrowinset=0.25}
\begin{pspicture*}(54.960367305811914,-0.01)(127.28938615068517,0.06838810485219687)
\psaxes[labelFontSize=\scriptstyle,xAxis=true,yAxis=true,Dx=10.,Dy=0.02,ticksize=-2pt 0,subticks=2]{->}(0,0)(54.960367305811914,-0.007913861475156032)(127.28938615068517,0.06838810485219687)[x,140] [,-40]
\pscustom[linewidth=0.8pt,fillcolor=black,fillstyle=solid,opacity=0.3]{\psplot{74.}{106.}{EXP((-(x-90.0)^(2.0))/(7.937253933193771^(2.0)*2.0))/(abs(7.937253933193771)*sqrt(3.141592653589793*2.0))}\lineto(106.,0)\lineto(74.,0)\closepath}
\psplot[linewidth=1.2pt,plotpoints=200]{54.960367305811914}{127.28938615068517}{EXP((-(x-90.0)^(2.0))/(7.937253933193771^(2.0)*2.0))/(abs(7.937253933193771)*sqrt(3.141592653589793*2.0))}
\psline[linewidth=2.pt,linestyle=dotted](90.,0.)(90.07445529274197,0.050259791608164066)
\psline[linewidth=2.pt](90.97481652317607,0.03691039597127567)(94.27614103476779,0.05151605289202309)
\begin{scriptsize}
\rput[tl](84.22210729492039,0.06410713644439159){Annahmebereich f�r}
\rput[tl](85.57264914057154,0.05806341633925473){die Nullhypothese}
\rput[tl](72.66747150434935,-0.0015){74}
\rput[tl](104.03005436447074,-0.0015){106}
\end{scriptsize}
\end{pspicture*}}
\end{center}

Kreuze die beiden zutreffenden Aussagen an.

\multiplechoice[5]{  %Anzahl der Antwortmoeglichkeiten, Standard: 5
				L1={Es handelt sich um einen zweiseitigen Anteilstest.},   %1. Antwortmoeglichkeit 
				L2={Die Arbeitshypothese lautet: $p_A=0,3$.},   %2. Antwortmoeglichkeit
				L3={Ein Ergebnis von $x_A=110$ stellt einen Beweis f�r die Richtigkeit der Arbeitshypothese dar.},   %3. Antwortmoeglichkeit
				L4={F�r den Fall, dass $x_A=80$ gilt, f�llt die Entscheidung zugunsten der Nullhypothese aus.},   %4. Antwortmoeglichkeit
				L5={Die Alternativhypothese lautet sinngem��: Der Anteil der "`Partei-A-W�hler"' ist gr��er als 30\,\%.},	 %5. Antwortmoeglichkeit
				L6={},	 %6. Antwortmoeglichkeit
				L7={},	 %7. Antwortmoeglichkeit
				L8={},	 %8. Antwortmoeglichkeit
				L9={},	 %9. Antwortmoeglichkeit
				%% LOESUNG: %%
				A1=1,  % 1. Antwort
				A2=4,	 % 2. Antwort
				A3=0,  % 3. Antwort
				A4=0,  % 4. Antwort
				A5=0,  % 5. Antwort
				}
\end{beispiel}