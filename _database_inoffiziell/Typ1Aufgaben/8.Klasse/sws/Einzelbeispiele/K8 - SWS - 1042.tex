\section{K8 - SWS - 1042 - Gezinkter W�rfel - OA - Dimensionen Mathematik 8 - Schularbeiten-Trainer}

\begin{beispiel}[K8 - SWS]{1}
Es besteht der Verdacht, dass ein W�rfel insofern gezinkt ist, als dass die Seitenfl�che mit der Aufschrift "`6"' �fter nach oben zu liegen kommt, als dies bei einem fairen W�rfel zu erwarten ist. Zur Untermauerung des Verdachts bezieht man sich auf eine Serie von 350 W�rfen, in der man 73-mal einen "`6er"' gew�rfelt hat.

Teste die Vermutung bei einem Signifikanzniveau von $\alpha=0,05$.

\antwort{Nullhypothese $H_0$: $p=\frac{1}{6}$

Arbeitshypothese (Alternativhypothese): $p>\frac{1}{6}$

$n=300; p=\frac{1}{6}$\\
$\mu=300\cdot\frac{1}{6}=50$\\
$\sigma=\sqrt{300\cdot\frac{1}{6}\cdot\frac{5}{6}}\approx 6,45$\\
$\Phi(z)=0,95$ bzw. $z=1,645$\\
$x=50+1,645\cdot 6,45\approx 60,6$

Wird in einer Serie von 300 W�rfen mindestens 61-mal ein "`6er"' gew�rfelt, so darf die Nullhypothese verworfen werden, was in diesem Fall eindeutig der Fall ist. Die Irrtumswahrscheinlichkeit bei 73 "`6ern"' betr�gt nur 0,02\,\%.}
\end{beispiel}