\section{K8 - SWS - 1031 - Sigma-Regel - LT - Dimensionen Mathematik 8 - Schularbeiten-Trainer}

\begin{beispiel}[K8 - SWS]{1}
Die Zeit, die eine Person f�r eine bestimmte Wegstrecke ben�tigt, kann in einem konkreten Fall als normalverteilte Zufallsvariable mit den Parametern $\mu=40$ Minuten und $\sigma=2$ Minuten angesehen werden. Die nachfolgende Aufgabenstellung zielt auf eine intuitive, nicht mathematische exakte Interpretation eines oftmals als "`$\sigma$-Regel"' bekannten Sachverhalt ab.

\lueckentext{
				text={Das Ereignis, dass die Zufallsvariable $X$ \gap annimmt, kann als \gap Ereignis angesehen werden.}, 	%Lueckentext Luecke=\gap
				L1={einen Wert unter 40}, 		%1.Moeglichkeit links  
				L2={einen Wert �ber 44}, 		%2.Moeglichkeit links
				L3={einen Wert zwischen 38 und 42}, 		%3.Moeglichkeit links
				R1={"`sicheres"'}, 		%1.Moeglichkeit rechts 
				R2={"`unm�gliches"'}, 		%2.Moeglichkeit rechts
				R3={"`seltenes"'}, 		%3.Moeglichkeit rechts
				%% LOESUNG: %%
				A1=2,   % Antwort links
				A2=3		% Antwort rechts 
				}
\end{beispiel}