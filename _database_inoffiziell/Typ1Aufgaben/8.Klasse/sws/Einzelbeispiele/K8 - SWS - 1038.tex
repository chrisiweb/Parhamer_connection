\section{K8 - SWS - 1038 - Nullhypothese zu einer zweiseitigen Alternativhypothese - LT - Dimensionen Mathematik 8 - Schularbeiten-Trainer}

\begin{beispiel}[K8 - SWS]{1}
Die Nullhypothese zu einer zweiseitigen Alternativhypothese lautet: $p=0,8$. Die Entscheidung wird auf der Basis einer Zufallsstichprobe der Gr��e $n$ ermittelt. Als Signifikanzniveau wird $\alpha=0,05$ festgelegt. Vergleiche die Abbildung unten.

\begin{center}
	\resizebox{0.6\linewidth}{!}{\psset{xunit=1.3cm,yunit=10cm,algebraic=true,dimen=middle,dotstyle=o,dotsize=5pt 0,linewidth=1.6pt,arrowsize=3pt 2,arrowinset=0.25}
\begin{pspicture*}(16.460981169975998,-0.12157624936708347)(24.53026879927448,0.42700214262823616)
\psaxes[labelFontSize=\scriptstyle,xAxis=true,yAxis=true,labels=y,Dx=1.,Dy=0.1,ticksize=-2.5pt 0,subticks=2]{->}(0,0)(16.460981169975998,-0.12157624936708347)(24.53026879927448,0.42700214262823616)[x,140] [,-40]
\pscustom[linewidth=0.8pt,fillcolor=black,fillstyle=solid,opacity=0.1]{\psplot{15.}{18.}{EXP((-(x-20.0)^(2.0))/(1.0^(2.0)*2.0))/(abs(1.0)*sqrt(3.141592653589793*2.0))}\lineto(18.,0)\lineto(15.,0)\closepath}
\pscustom[linewidth=0.8pt,fillcolor=black,fillstyle=solid,opacity=0.1]{\psplot{22.}{25.}{EXP((-(x-20.0)^(2.0))/(1.0^(2.0)*2.0))/(abs(1.0)*sqrt(3.141592653589793*2.0))}\lineto(25.,0)\lineto(22.,0)\closepath}
\psplot[linewidth=1.6pt,plotpoints=200]{16.460981169975998}{24.53026879927448}{EXP((-(x-20.0)^(2.0))/(1.0^(2.0)*2.0))/(abs(1.0)*sqrt(3.141592653589793*2.0))}
\rput[tl](19.837585878117373,-0.013598149585199292){200}
\psline[linewidth=2.pt,linestyle=dashed,dash=2pt 2pt](20.,0.)(20.,0.3989422804014327)
\rput[tl](21.83164377662606,-0.013){$x_{k2}$}
\rput[tl](17.87011541825547,-0.013){$x_{k1}$}
\rput[tl](19,-0.07689565635389002){kritische Werte}
\psline[linewidth=2.pt]{->}(20.75485251143137,-0.060760997765792385)(21.63223798677519,-0.03221506334069656)
\psline[linewidth=2.pt]{->}(19.25,-0.060760997765792385)(18.37,-0.03221506334069656)
\end{pspicture*}}
\end{center}

\lueckentext{
				text={Anhand der Abbildung kann festgestellt werden, dass \gap gilt und die Gr��e der rot gef�rbten Fl�che den Wert \gap annimmt.}, 	%Lueckentext Luecke=\gap
				L1={$n=200$}, 		%1.Moeglichkeit links  
				L2={$n=250$}, 		%2.Moeglichkeit links
				L3={$n=400$}, 		%3.Moeglichkeit links
				R1={$0,1$}, 		%1.Moeglichkeit rechts 
				R2={$0,05$}, 		%2.Moeglichkeit rechts
				R3={$0,025$}, 		%3.Moeglichkeit rechts
				%% LOESUNG: %%
				A1=2,   % Antwort links
				A2=3		% Antwort rechts 
				}
\end{beispiel}