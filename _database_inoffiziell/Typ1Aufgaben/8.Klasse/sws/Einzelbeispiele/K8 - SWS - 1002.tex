\section{K8 - SWS - 1002 - Normalverteilte Zufallsvariable - MC - thema mathematik 8 - Schularbeiten}

\begin{beispiel}[K8 - SWS]{1}
Die Zufallsvariable $X$ ist normalverteilt mit Erwartungswert $\mu$ und Standardabweichung $\sigma$.\\
Kreuze die beiden zutreffenden Aussagen an!

\multiplechoice[5]{  %Anzahl der Antwortmoeglichkeiten, Standard: 5
				L1={Mit einer Wahrscheinlichkeit von 50\,\% gilt: $X<\mu$.},   %1. Antwortmoeglichkeit 
				L2={Die Zufallsvariable $X$ kann nur positive Werte annehmen.},   %2. Antwortmoeglichkeit
				L3={Mit einer Wahrscheinlichkeit von etwa 68\,\% hat $X$ einen Wert, der mehr als die Standardabweichung vom Erwartungswert abweicht.},   %3. Antwortmoeglichkeit
				L4={Die Wahrscheinlichkeit, dass $X$ genau den Erwartungswert annimmt, ist 0\,\%.},   %4. Antwortmoeglichkeit
				L5={Die Zufallsvariable kann nur ganzzahlige Werte annehmen.},	 %5. Antwortmoeglichkeit
				L6={},	 %6. Antwortmoeglichkeit
				L7={},	 %7. Antwortmoeglichkeit
				L8={},	 %8. Antwortmoeglichkeit
				L9={},	 %9. Antwortmoeglichkeit
				%% LOESUNG: %%
				A1=1,  % 1. Antwort
				A2=4,	 % 2. Antwort
				A3=0,  % 3. Antwort
				A4=0,  % 4. Antwort
				A5=0,  % 5. Antwort
				}
\end{beispiel}