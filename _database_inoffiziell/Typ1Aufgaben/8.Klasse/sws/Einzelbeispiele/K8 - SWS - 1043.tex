\section{K8 - SWS - 1043 - Automatisierter Produktionsprozess - OA - Dimensionen Mathematik 8 - Schularbeiten-Trainer}

\begin{beispiel}[K8 - SWS]{1}
Die Arbeitshypothese, dass der Anteil defekter St�cke bei einem bestimmten automatisierten Produktionsprozess gr��er als $2\,\%$ ist, ist bei einem Signifikanzniveau von $\alpha=0,05$ zu testen. Zur Verf�gung steht eine Produktionsserie von $n=500$ St�ck.

Ermittle, wie viele defekte St�cke in der Stichprobe mindestens vorhanden sein m�ssen, damit die Entscheidung zugunsten der Arbeitshypothese ausfallen kann.

\antwort{Nullhypothese: $H_0$: $p=0,02$\\
Arbeitshypothese $H_A$: $p>0,02$\\
$n=500; p=0,02; \mu=500\cdot 0,02=10; \sigma=\sqrt{500\cdot 0,02\cdot 0,98}\approx 3,13$\\
$\Phi(z)=0,95$ bzw. $z=1,645$\\
$x=10+1,645\cdot 3,13\approx 15,15$

Damit die Entscheidung zugunsten der Arbeitshypothese ausfallen kann, m�ssen mindestens 16 defekte St�cke in einer Stichprobe sein.}
\end{beispiel}