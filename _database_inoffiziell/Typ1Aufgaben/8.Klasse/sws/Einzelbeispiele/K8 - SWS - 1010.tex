\section{K8 - SWS - 1010 - Wert einer Verteilungsfunktion - LT - Dimensionen Mathematik 8 - Schularbeiten-Trainer}

\begin{beispiel}[K8 - SWS]{1}
Die Variable $Z$ ist standardnormalverteilt. Die Verteilungsfunktion von $Z$ wird mit $\Phi$ bezeichnet, die Dichtefunktion mit $\varphi$.

\lueckentext{
				text={Der Wert \gap entspricht dem Wert \gap.}, 	%Lueckentext Luecke=\gap
				L1={$\displaystyle\int^2_{0,5}{\Phi(z)}\,\text{d}z$}, 		%1.Moeglichkeit links  
				L2={$\Phi(2)-\Phi(0,5)$}, 		%2.Moeglichkeit links
				L3={$\Phi(0,5)+\Phi(2)$}, 		%3.Moeglichkeit links
				R1={$\varphi(2)-\varphi(0,5)$}, 		%1.Moeglichkeit rechts 
				R2={$\displaystyle\int^{0,5}_0{\varphi(z)}\,\text{d}z+\displaystyle\int^2_0{\varphi(z)}\,\text{d}z$}, 		%2.Moeglichkeit rechts
				R3={$\displaystyle\int^2_{0,5}{\varphi(z)}\,\text{d}z$}, 		%3.Moeglichkeit rechts
				%% LOESUNG: %%
				A1=2,   % Antwort links
				A2=3		% Antwort rechts 
				}
\end{beispiel}