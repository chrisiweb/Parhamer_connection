\section{K8 - IR - 1003 - Integral einer Funktion g - MC - Dimensionen Mathematik 8 - Schularbeiten-Trainer}

\begin{beispiel}[K8 - IR]{1}
Das bestimmte Integral einer Funktion $g$ im Intervall $[a;b]$ nimmt einen bestimmten Wert an. Eine Stammfunktion von $g$ wird mit $G$ bezeichnet.

Kreuze an, welche Formulierung(en) bzw. welcher Term den Wert zutreffend beschreiben.

\multiplechoice[5]{  %Anzahl der Antwortmoeglichkeiten, Standard: 5
				L1={$G(a)-G(b)$},   %1. Antwortmoeglichkeit 
				L2={Differenz der Funktionswerte der Stammfunktion $G$ an den Stellen $a$ und $b$.},   %2. Antwortmoeglichkeit
				L3={$G(b)-G(a)$},   %3. Antwortmoeglichkeit
				L4={Summe der Funktionswerte der Stammfunktion $G$ an den Stellen $a$ und $b$.},   %4. Antwortmoeglichkeit
				L5={Differenz der Funktionswerte der Integralfunktion von $g$ an den Stellen $b$ und $a$.},	 %5. Antwortmoeglichkeit
				L6={},	 %6. Antwortmoeglichkeit
				L7={},	 %7. Antwortmoeglichkeit
				L8={},	 %8. Antwortmoeglichkeit
				L9={},	 %9. Antwortmoeglichkeit
				%% LOESUNG: %%
				A1=3,  % 1. Antwort
				A2=5,	 % 2. Antwort
				A3=0,  % 3. Antwort
				A4=0,  % 4. Antwort
				A5=0,  % 5. Antwort
				}
\end{beispiel}