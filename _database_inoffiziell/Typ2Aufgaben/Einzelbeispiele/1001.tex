\section{1001 - K7 - DWV - WS 3.2, WS 3.3, WS 3.1 - Gut Flug - Thema Mathematik Schularbeiten 7. Klasse}

\begin{langesbeispiel} \item[0] %PUNKTE DES BEISPIELS
	
Die Fluggesellschaft "`Gut Flug"' hat sich auf Kurzstreckenfl�ge $(<2000$\,km) spezialisiert. Erfahrungsgem�� wird bei Firma "`Gut Flug"' ein gebuchter Platz nur mit der Wahrscheinlichkeit $92\,\%$ auch tats�chlich belegt.

\subsection{Aufgabenstellung:}
\begin{enumerate}
	\item F�r einen Kurzstreckenflug von Wien nach Z�rich werden 68 Flugtickets verkauft.\leer
	
	\fbox{A} Berechne die Wahrscheinlichkeit, dass genau 2 Personen, die ein Flugticket gekauft haben, nicht zu diesem Flug kommen.\leer
	
	Begr�nde warum f�r diese Berechnung die Binomialverteilung zugrunde gelegt werden kann.\leer
	
	\item Berechne Erwartungswert und Standardabweichung der zum Flug kommenden Flugg�ste bei diesem ausgebuchten Flug und interpretiere diese Werte im Kontext.\leer
	
	\item Die Fluggesellschaft hat diesen Kurzstreckenflug wie �blich �berbucht. Dies bedeutet, dass 68 Tickets verkauft wurden, obwohl aber nur 65 Pl�tze vorhanden sind.\leer
	
	Berechne die Wahrscheinlichkeit daf�r, dass diese �berbuchung gut geht.
	
	\item Wie viele Tickets k�nnte die Fluglinie maximal verkaufen, damit der zu erwartende Wert der auftauchenden Passagiere die Platzzahl von 65 nicht �berschreitet? Was spricht gegen eine Erh�hung der verkauften Tickets auf die H�he des Erwartungswert?
\end{enumerate}

\antwort{
\begin{enumerate}
	\item \subsection{L�sungserwartung:} 

$P(X=66)\approx 5,94\,\%$\leer

Jede Person, die ein Ticket gekauft hat, kommt mit einer Wahrscheinlichkeit von $92\,\%$ zum Flug und mit einer Wahrscheinlichkeit von $8\,\%$ nicht zum Flug.

	\item \subsection{L�sungserwartung:}
	
$\mu=E(X)=62,56; \sigma\approx 2,24$

In etwa $\frac{2}{3}$ der F�lle werden zwischen $\mu-\sigma\approx 60,32$ und $\mu+\sigma\approx 64,80$ Personen zum Flug kommen.

\item \subsection{L�sungserwartung:}

$P(X\leq 65)\approx 91,68\,\%$

	\item \subsection{L�sungserwartung:}
	
$n\cdot 0,92=65 \rightarrow n=70,65$

Die Fluglinie k�nnte bis zu 70 Tickets verkaufen um den Erwartungswert von 65 nicht zu �berschreiten.

Dagegen spricht, dass bei einem Erwartungswert von 65 die Wahrscheinlichkeit, dass 66 Leute erscheinen noch immer sehr gro� ist und dadurch die Gefahr, dass es zu �berbuchungen kommt deutlich h�her ist als wenn man einen niedrigeren Erwartungswert anpeilt.
\end{enumerate}}
		\end{langesbeispiel}