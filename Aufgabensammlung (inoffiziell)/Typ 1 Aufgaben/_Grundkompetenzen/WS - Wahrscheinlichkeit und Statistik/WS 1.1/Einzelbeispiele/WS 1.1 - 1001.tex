\section{WS 1.1 - 1001 Studiendauer - MC - neo-lernhilfen.at}

\begin{beispiel}[WS 1.1]{1} %PUNKTE DES BEISPIELS
				F�r eine technische Studienrichtung wurde die Studiendauer in Semester statistisch erfasst und im nachstehenden Boxplot dargestellt.
								
				\begin{center}
				\resizebox{0.8\linewidth}{!}{\psset{xunit=1.0cm,yunit=1.0cm,algebraic=true,dimen=middle,dotstyle=o,dotsize=5pt 0,linewidth=0.8pt,arrowsize=3pt 2,arrowinset=0.25}
\begin{pspicture*}(7.506361904761948,-0.8415428571428537)(24.534933333333477,2.2441714285714256)
\psaxes[labelFontSize=\scriptstyle,xAxis=true,yAxis=true,Dx=1.,Dy=1.,ticksize=-2pt 0,subticks=2]{}(0,0)(7.506361904761948,-0.8415428571428537)(24.534933333333477,2.2441714285714256)
\psframe[fillcolor=black,fillstyle=solid,opacity=0.1](12.,0.30000000000000004)(22.,1.7)
\psline[fillcolor=black,fillstyle=solid,opacity=0.1](10.,0.3)(10.,1.7)
\psline[fillcolor=black,fillstyle=solid,opacity=0.1](23.,0.3)(23.,1.7)
\psline[fillcolor=black,fillstyle=solid,opacity=0.1](13.,0.3)(13.,1.7)
\psline[fillcolor=black,fillstyle=solid,opacity=0.1](10.,1.)(12.,1.)
\psline[fillcolor=black,fillstyle=solid,opacity=0.1](22.,1.)(23.,1.)
\end{pspicture*}}
				\end{center}
				
				Kreuze die zutreffende(n) Aussage(n) an.
				\multiplechoice[5]{  %Anzahl der Antwortmoeglichkeiten, Standard: 5
								L1={$25\,\%$ der Studierenden studieren h�chstens 12 Semester lang.},   %1. Antwortmoeglichkeit 
								L2={Mindestens $50\,\%$ der Studierenden ben�tigen f�r den Abschluss des Studiums zwischen $10$ und $22$ Semester.},   %2. Antwortmoeglichkeit
								L3={Der Median betr�gt $12$ Semester.},   %3. Antwortmoeglichkeit
								L4={Der Quartilabstand betr�gt $10$ Semester.},   %4. Antwortmoeglichkeit
								L5={$\frac{3}{4}$ der Studierenden studieren mindestens $12$ Semester lang.},	 %5. Antwortmoeglichkeit
								L6={},	 %6. Antwortmoeglichkeit
								L7={},	 %7. Antwortmoeglichkeit
								L8={},	 %8. Antwortmoeglichkeit
								L9={},	 %9. Antwortmoeglichkeit
								%% LOESUNG: %%
								A1=1,  % 1. Antwort
								A2=4,	 % 2. Antwort
								A3=5,  % 3. Antwort
								A4=0,  % 4. Antwort
								A5=0,  % 5. Antwort
								}
\end{beispiel}