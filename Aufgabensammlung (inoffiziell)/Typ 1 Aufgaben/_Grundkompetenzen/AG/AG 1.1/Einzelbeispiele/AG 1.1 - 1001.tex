\section{AG 1.1 - 1001 Positive rationale Zahlen - MC - WS 1.1 - neo-lernhilfen.at}

\begin{beispiel}[AG 1.1]{1} %PUNKTE DES BEISPIELS 
				Gegeben ist die Zahlenmenge $\mathbb{Q}^{+}$.
				
				Kreuze die zutreffende(n) Zahl(en) an, die Element(e) dieser Zahlenmenge ist/sind.
				\multiplechoice[5]{  %Anzahl der Antwortmoeglichkeiten, Standard: 5
								L1={$0,8\cdot 10^{-7}$},   %1. Antwortmoeglichkeit 
								L2={$\pi^{0}$},   %2. Antwortmoeglichkeit
								L3={$\frac{27}{\pi}$},   %3. Antwortmoeglichkeit
								L4={$\sqrt{0,16}$},   %4. Antwortmoeglichkeit
								L5={$-\sqrt{0,36}$},	 %5. Antwortmoeglichkeit
								L6={},	 %6. Antwortmoeglichkeit
								L7={},	 %7. Antwortmoeglichkeit
								L8={},	 %8. Antwortmoeglichkeit
								L9={},	 %9. Antwortmoeglichkeit
								%% LOESUNG: %%
								A1=1,  % 1. Antwort
								A2=2,	 % 2. Antwort
								A3=4,  % 3. Antwort
								A4=0,  % 4. Antwort
								A5=0,  % 5. Antwort
								}
\end{beispiel}