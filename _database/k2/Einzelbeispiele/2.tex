\section{K2 - TBK.tut - 1002 - Teilbarkeitsregeln - RF - QueUnb}

\begin{beispiel}{1} %PUNKTE DES BEISPIELS
				Gegeben sind fünf Aussagen über Teilbarkeitsregeln. Kreuze jeweils an, ob es sich dabei um eine richtige oder falsche Aussage handelt!\leer
				
				\rfmultiplechoice[5]{Aussage}{  %Anzahl der Antwortmoeglichkeiten, Standard: 5
								L1={Teilt eine Zahl $t$ zwei Zahlen $a$ und $b$, dann teilt $t$ auch die Summe der beiden Zahlen $(a+b)$.},   %1. Antwortmoeglichkeit 
								L2={Eine Zahl ist durch 9 teilbar, wenn ihre Ziffernsumme durch 3 teilbar ist.},   %2. Antwortmoeglichkeit
								L3={Eine Zahl ist durch 6 teilbar, wenn die Zahl durch 2 und durch 3 teilbar ist.},   %3. Antwortmoeglichkeit
								L4={Eine Zahl ist durch 10 teilbar, wenn ihre Einerziffer 0 ist - sonst nicht.},   %4. Antwortmoeglichkeit
								L5={Eine Zahl ist durch 4 teilbar, wenn sie durch 2 teilbar ist.},	 %5. Antwortmoeglichkeit
								L6={},	 %6. Antwortmoeglichkeit
								L7={},	 %7. Antwortmoeglichkeit
								L8={},	 %8. Antwortmoeglichkeit
								L9={},	 %9. Antwortmoeglichkeit
								%% LOESUNG: %%
								A1=3,  % 1. Antwort
								A2=1,	 % 2. Antwort
								A3=4,  % 3. Antwort
								A4=0,  % 4. Antwort
								A5=0,  % 5. Antwort
								}
\end{beispiel}