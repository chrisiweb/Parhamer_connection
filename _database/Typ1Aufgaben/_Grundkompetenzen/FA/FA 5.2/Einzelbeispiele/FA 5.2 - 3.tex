\section{FA 5.2 - 3 Pulver - OA - BIFIE}

\begin{beispiel}[FA 5.2]{1} %PUNKTE DES BEISPIELS
Ein Pulver l�st sich in einer Fl�ssigkeit ann�hernd exponentiell auf. Die Menge an Pulver, die in
Abh�ngigkeit von der Zeit $t $noch vorhanden ist, wird f�r einen gewissen Zeitraum durch die
Gleichung $N(t) = N_0 \cdot 0,6^t$ beschrieben. $N_0$ gibt die urspr�ngliche Menge an Pulver in Milligramm
an, die Zeit $t$ wird in Sekunden gemessen. 

\leer

Gib an, wie viel Prozent der urspr�nglichen Pulvermenge $N_0$ nach drei Sekunden noch vorhanden sind.  

\antwort{$0,6^3\cdot 100 = 21,6$ \leer

Nach drei Sekunden sind noch 21,6\,\% der urspr�nglichen Menge an Pulver vorhanden.}
\end{beispiel}