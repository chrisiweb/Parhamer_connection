\section{FA 5.2 - 5 - MAT - Wachstum - OA - Matura 2013/14 Haupttermin}

\begin{beispiel}{1} %PUNKTE DES BEISPIELS
			Die Funktion $f$ beschreibt einen exponentiellen Wachstumsprozess der Form $f(t)=c\cdot a^t$ in Abhängigkeit von der Zeit $t$.
			
			Ermittle für $t=2$ und $t=3$ die Werte der Funktion $f$!
			
			\begin{center}
			\begin{tabular}{|c|c|}\hline
			$t$&$f(t)$\\ \hline
			0&400\\ \hline
			1&600\\ \hline
			2&$f(2)$\\ \hline
			3&$f(3)$\\ \hline			
			\end{tabular}
			\end{center}
			
			$f(2)=$ \antwort[\rule{5cm}{0.3pt}]{900}\leer
			
			$f(3)=$ \antwort[\rule{5cm}{0.3pt}]{1350}
\end{beispiel}