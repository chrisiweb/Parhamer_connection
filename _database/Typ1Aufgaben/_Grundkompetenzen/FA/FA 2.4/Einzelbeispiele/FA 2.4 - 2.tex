\section{FA 2.4 - 2 Charakteristische Eigenschaften einer linearen Funktion - MC - BIFIE}

\begin{beispiel}[FA 2.4]{2} %PUNKTE DES BEISPIELS
				Temperaturen werden bei uns in $^\circ$C (Celsius) gemessen; in einigen anderen L�ndern ist die Messung in $^\circ$F (Fahrenheit) �blich. Die Gerade f stellt den Zusammenhang
zwischen $^\circ$C und $^\circ$F dar. 

\psset{xunit=0.04cm,yunit=0.02cm,algebraic=true,dimen=middle,dotstyle=o,dotsize=3pt 0,linewidth=0.8pt,arrowsize=3pt 2,arrowinset=0.25}
\begin{pspicture*}(-40,-26.87273165082589)(280,380)
\psaxes[xAxis=true,yAxis=true,Dx=100.,Dy=100.,ticksize=-2pt 0,subticks=2]{->}(0,0)(0,0)(202.28231315540535,336.7720477830942)
\psplot[linecolor=black!50,plotpoints=200]{0}{160}{320.0}
\psplot[linecolor=black!50,plotpoints=200]{0}{160}{140.0}
\psline[linecolor=black!50](60.,0)(60.,140)
\psline[linecolor=black!50](160.,0)(160.,320)
\psplot[plotpoints=200]{0}{170}{9.0*x/5.0+32.0}
%\begin{small}
\rput[bl](30,60){$f$}
\rput[bl](-21,315){$320$}
\rput[bl](-21,135){$140$}
\rput[bl](55,-24){$60$}
\rput[bl](155,-24){$160$}
\rput[bl](190,-24){$x$ in $^\circ$C}
\rput[bl](-5,340){$f(x)$ in $^\circ$F}
%\end{small}
\end{pspicture*}
\leer 

\textbf{Aufgabenstellung:}\\
Welche der folgenden Aussagen kannst du der Abbildung entnehmen?
Kreuze die beiden zutreffenden Aussagen an!
\multiplechoice{
				L1={160 $^\circ$C entsprechen doppelt so vielen $^\circ$F.  }, %1. Antwort 
				L2={140 $^\circ$F entsprechen 160 $^\circ$C.  }, %2. Antwort
				L3={Eine Zunahme um 1 $^\circ$C bedeutet eine Zunahme um 1,8 $^\circ$F.}, %3. Antwort
				L4={Eine Abnahme um 1 $^\circ$F bedeutet eine Abnahme um 18 $^\circ$C.}, %4. Antwort
				L5={Der Anstieg der Geraden ist $k = \dfrac{x_2-x_1}{f(x_2)-f(x_1)}=\dfrac{100}{180}$ },	 %5. Antwort
				A1=1,
				A2=3}
\end{beispiel}