\section{FA 2.4 - 5 Eigenschaften einer linearen Funktion - MC - Matura 2013/14 1. Nebentermin}

\begin{beispiel}[FA 2.4]{1} %PUNKTE DES BEISPIELS
				Eine Funktion $f$ wird durch die Funktionsgleichung $f(x)=k\cdot x+d$ mit $k,d\in\mathbb{R}$ und $k\neq 0$ beschrieben.
				
				Kreuze die f�r $f$ zutreffende(n) Aussage(n) an!\leer
				
				\multiplechoice[5]{  %Anzahl der Antwortmoeglichkeiten, Standard: 5
								L1={$f$ kann lokale Extremstellen besitzen.},   %1. Antwortmoeglichkeit 
								L2={$f(x+1)=f(x)+k$},   %2. Antwortmoeglichkeit
								L3={$f$ besitzt immer genau eine Nullstelle.},   %3. Antwortmoeglichkeit
								L4={$\frac{f(x_2)-f(x_1)}{x_2-x_1}=k$ f�r $x_1\neq x_2$},   %4. Antwortmoeglichkeit
								L5={Die Kr�mmung des Graphen der Funktion $f$ ist null.},	 %5. Antwortmoeglichkeit
								L6={},	 %6. Antwortmoeglichkeit
								L7={},	 %7. Antwortmoeglichkeit
								L8={},	 %8. Antwortmoeglichkeit
								L9={},	 %9. Antwortmoeglichkeit
								%% LOESUNG: %%
								A1=2,  % 1. Antwort
								A2=3,	 % 2. Antwort
								A3=4,  % 3. Antwort
								A4=5,  % 4. Antwort
								A5=0,  % 5. Antwort
								}
\end{beispiel}