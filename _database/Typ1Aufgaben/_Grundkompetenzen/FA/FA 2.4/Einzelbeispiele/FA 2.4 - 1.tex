\section{FA 2.4 - 1 Temperaturskala - MC - BIFIE}

\begin{beispiel}[FA 2.4]{1} %PUNKTE DES BEISPIELS
Temperaturen werden bei uns in $^\circ C$ (Celsius) gemessen; in einigen anderen Ländern ist die Messung in $^\circ F$ (Fahrenheit) üblich.

Die Gerade $f$ stellt den Zusammenhang zwischen $^\circ C$ und $^\circ F$ dar.

\begin{center}
\psset{xunit=0.4cm,yunit=0.4cm,algebraic=true,dimen=middle,dotstyle=o,dotsize=5pt 0,linewidth=0.8pt,arrowsize=5pt 2,arrowinset=0.25}
\begin{pspicture*}(-1.5237726489884804,-4.322080487573234)(21.57878113707398,19.507775917680082)
\psline[linewidth=1.6pt](0.62,-0.02)(0.62,17.92482315826469)
\psline[linewidth=1.6pt](0.62,-0.02)(18.198963083187067,-0.02)
\psline(16.62,-0.02)(16.62,15.98)
\psline(16.62,15.98)(0.62,15.98)
\psline(16.62,6.98)(0.62,6.98)
\psline(6.62,6.98)(6.62,-0.02)
\psline[linewidth=1.6pt](0.62,1.58)(17.665199828755846,16.920679845880258)
\rput[tl](1.0003952646738996,18){$f(x) \text{ in } ^\circ\,F$}
\rput[tl](17.514442971007437,1){$x \text{ in } ^\circ\,C$}
%\begin{scriptsize}
\psdots[dotsize=3pt 0,dotstyle=+,linecolor=darkgray](16.62,-0.02)
\rput[bl](15.974272718603272,-1){\darkgray{160}}
\psdots[dotsize=3pt 0,dotstyle=+,linecolor=darkgray](1.62,-0.02)
\rput[bl](1.2143077997300336,-1){\darkgray{10}}
\psdots[dotsize=3pt 0,dotstyle=+,linecolor=darkgray](6.62,-0.02)
\rput[bl](6.262643627054794,-1){\darkgray{60}}
\psdots[dotsize=3pt 0,dotstyle=+,linecolor=darkgray](0.62,15.98)
\rput[bl](-1.4,15.6){\darkgray{320}}
\psdots[dotsize=3pt 0,dotstyle=+,linecolor=darkgray](0.62,6.98)
\rput[bl](-1.4,6.6){\darkgray{140}}
\psdots[dotsize=3pt 0,dotstyle=+,linecolor=darkgray](0.62,0.98)
\rput[bl](-1,0.6){\darkgray{20}}
\psdots[dotsize=3pt 0,dotstyle=triangle*](0.62,17.92482315826469)
\psdots[dotsize=3pt 0,dotstyle=triangle*,dotangle=270](18.198963083187067,-0.02)
\rput[bl](10.02750424404275,10.609014459344912){f}
%\end{scriptsize}
\end{pspicture*}
\end{center}

Welche der folgenden Aussagen kannst du der Abbildung entnehmen?
Kreuze die beiden zutreffenden Aussagen an!
\multiplechoice[5]{  %Anzahl der Antwortmoeglichkeiten, Standard: 5
				L1={$160^\circ C$ entsprechen doppelt so vielen $^\circ F$.},   %1. Antwortmoeglichkeit 
				L2={$140^\circ F$ entsprechen $160^\circ C$.},   %2. Antwortmoeglichkeit
				L3={Eine Zunahme um $1^\circ C$ bedeutet eine Zunahme um $1,8^\circ F$.},   %3. Antwortmoeglichkeit
				L4={Eine Abnahme um $1^\circ F$ bedeutet eine Abnahme um $18^\circ C$.},   %4. Antwortmoeglichkeit
				L5={Der Anstieg der Geraden ist $k=\dfrac{x_2-x_1}{f(x_2)-f(x_1)}=\dfrac{100}{180}$.},	 %5. Antwortmoeglichkeit
				L6={},	 %6. Antwortmoeglichkeit
				L7={},	 %7. Antwortmoeglichkeit
				L8={},	 %8. Antwortmoeglichkeit
				L9={},	 %9. Antwortmoeglichkeit
				%% LOESUNG: %%
				A1=1,  % 1. Antwort
				A2=3,	 % 2. Antwort
				A3=0,  % 3. Antwort
				A4=0,  % 4. Antwort
				A5=0,  % 5. Antwort
				}
\end{beispiel}