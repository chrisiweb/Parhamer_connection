\section{FA 1.2 - 5 - MAT - Funktionen zuordnen - MC - Matura-HT-18/19}

\begin{beispiel}[FA 1.2]{1}
Gegeben ist die Formel $F=\dfrac{a^2\cdot b}{c^n}+d$ mit $a,b,c,d \in \mathbb{R}, n\in \mathbb{N}$ und $c\neq 0, n\neq 0$.

Nimmt man an, dass eine der Größen $a,b,c,d$ oder $n$ variabel ist und die anderen Größen konstant sind, so kann $F$ als Funktion in Abhängigkeit von der variablen Größe interpretiert werden. \leer

Welche der unten angegebenen Zuordnungen beschreiben (mit geeignetem Definitions- und Wertebereich) eine lineare Funktion?

Kreuze die beiden zutreffenden Zuordnungen an!

\multiplechoice[5]{  %Anzahl der Antwortmoeglichkeiten, Standard: 5
				L1={$a \mapsto \dfrac{a^2\cdot b}{c^n}+d$},   %1. Antwortmoeglichkeit 
				L2={$b \mapsto \dfrac{a^2\cdot b}{c^n}+d$},   %2. Antwortmoeglichkeit
				L3={$c \mapsto \dfrac{a^2\cdot b}{c^n}+d$},   %3. Antwortmoeglichkeit
				L4={$d \mapsto \dfrac{a^2\cdot b}{c^n}+d$},   %4. Antwortmoeglichkeit
				L5={$n \mapsto \dfrac{a^2\cdot b}{c^n}+d$},	 %5. Antwortmoeglichkeit
				L6={},	 %6. Antwortmoeglichkeit
				L7={},	 %7. Antwortmoeglichkeit
				L8={},	 %8. Antwortmoeglichkeit
				L9={},	 %9. Antwortmoeglichkeit
				%% LOESUNG: %%
				A1=2,  % 1. Antwort
				A2=4,	 % 2. Antwort
				A3=0,  % 3. Antwort
				A4=0,  % 4. Antwort
				A5=0,  % 5. Antwort
				}
\end{beispiel}