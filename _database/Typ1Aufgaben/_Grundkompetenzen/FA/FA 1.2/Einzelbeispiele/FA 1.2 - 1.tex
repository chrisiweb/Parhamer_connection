\section{FA 1.2 - 1 - Funktionsdarstellung einer Formel - MC - BIFIE}

\begin{beispiel}[FA 1.2]{1} %PUNKTE DES BEISPIELS
Gegeben ist die Formel $r = \dfrac{2s^2t}{u}$ für $s,t,u>0$.

Wenn $u$ und $s$ konstant sind, dann kann $r$ als eine Funktion in Abhängigkeit von $t$ betrachtet
werden. Kreuze denjenigen/diejenigen der unten dargestellten Funktionsgraphen an, der/die dann für die Funktion $r$ möglich ist/sind! 

\langmultiplechoice[5]{  %Anzahl der Antwortmoeglichkeiten, Standard: 5
				L1={\psset{xunit=0.6cm,yunit=0.8cm,algebraic=true,dimen=middle,dotstyle=o,dotsize=5pt 0,linewidth=0.8pt,arrowsize=3pt 2,arrowinset=0.25}
\begin{pspicture*}(-1.3,-0.44114599741279703)(7.749030052594037,4.538849027380639)
\psaxes[labelFontSize=\scriptstyle,xAxis=true,yAxis=true,labels=none,Dx=1.,Dy=1.,ticksize=0pt 0,subticks=2]{->}(0,0)(-0.8318844516654383,-0.44114599741279703)(7.749030052594037,4.538849027380639)
\psplot[linewidth=1.2pt,plotpoints=200]{0.001}{7.749030052594037}{1.0/x}
\begin{scriptsize}
\rput[tl](5.9,-0.11936170350306731){1. Achse}
\rput[tl](0.35,4.4162645344626466){2. Achse}
\rput[tl](1,2.3){$f_1$}
\end{scriptsize}
\end{pspicture*}},   %1. Antwortmoeglichkeit 
				L2={\psset{xunit=0.6cm,yunit=0.8cm,algebraic=true,dimen=middle,dotstyle=o,dotsize=5pt 0,linewidth=0.8pt,arrowsize=3pt 2,arrowinset=0.25}
\begin{pspicture*}(-1.3,-0.44114599741279703)(7.749030052594037,4.538849027380639)
\psaxes[labelFontSize=\scriptstyle,xAxis=true,yAxis=true,labels=none,Dx=1.,Dy=1.,ticksize=0pt 0,subticks=2]{->}(0,0)(-0.8318844516654383,-0.44114599741279703)(7.749030052594037,4.538849027380639)
\psplot[linewidth=1.2pt,plotpoints=200]{0}{5.4}{(-x)/1.8+3.0}
\begin{scriptsize}
\rput[tl](5.9,-0.11936170350306731){1. Achse}
\rput[tl](0.35,4.4162645344626466){2. Achse}
\rput[tl](1.4,2.8){$f_2$}
\end{scriptsize}
\end{pspicture*}},   %2. Antwortmoeglichkeit
				L3={\psset{xunit=0.6cm,yunit=0.8cm,algebraic=true,dimen=middle,dotstyle=o,dotsize=5pt 0,linewidth=0.8pt,arrowsize=3pt 2,arrowinset=0.25}
\begin{pspicture*}(-1.3,-0.44114599741279703)(7.749030052594037,4.538849027380639)
\psaxes[labelFontSize=\scriptstyle,xAxis=true,yAxis=true,labels=none,Dx=1.,Dy=1.,ticksize=0pt 0,subticks=2]{->}(0,0)(-0.8318844516654383,-0.44114599741279703)(7.749030052594037,4.538849027380639)
\psplot[linewidth=1.2pt,plotpoints=200]{0}{7.749030052594037}{x/2.0}
\begin{scriptsize}
\rput[tl](5.9,-0.11936170350306731){1. Achse}
\rput[tl](0.35,4.4162645344626466){2. Achse}
\rput[tl](2.3,2){$f_3$}
\end{scriptsize}
\end{pspicture*}},   %3. Antwortmoeglichkeit
				L4={\psset{xunit=0.6cm,yunit=0.8cm,algebraic=true,dimen=middle,dotstyle=o,dotsize=5pt 0,linewidth=0.8pt,arrowsize=3pt 2,arrowinset=0.25}
\begin{pspicture*}(-1.3,-0.44114599741279703)(7.749030052594037,4.538849027380639)
\psaxes[labelFontSize=\scriptstyle,xAxis=true,yAxis=true,labels=none,Dx=1.,Dy=1.,ticksize=0pt 0,subticks=2]{->}(0,0)(-0.8318844516654383,-0.44114599741279703)(7.749030052594037,4.538849027380639)
\psplot[linewidth=1.2pt,plotpoints=200]{0}{7.749030052594037}{(x/2.0)^(2.0)}
\begin{scriptsize}
\rput[tl](5.9,-0.11936170350306731){1. Achse}
\rput[tl](0.35,4.4162645344626466){2. Achse}
\rput[tl](2,2){$f_4$}
\end{scriptsize}
\end{pspicture*}},   %4. Antwortmoeglichkeit
				L5={\psset{xunit=0.6cm,yunit=0.8cm,algebraic=true,dimen=middle,dotstyle=o,dotsize=5pt 0,linewidth=0.8pt,arrowsize=3pt 2,arrowinset=0.25}
\begin{pspicture*}(-1.3,-0.44114599741279703)(7.749030052594037,4.538849027380639)
\psaxes[labelFontSize=\scriptstyle,xAxis=true,yAxis=true,labels=none,Dx=1.,Dy=1.,ticksize=0pt 0,subticks=2]{->}(0,0)(-0.8318844516654383,-0.44114599741279703)(7.749030052594037,4.538849027380639)
\psplot[linewidth=1.2pt,plotpoints=200]{0}{7.749030052594037}{x/2.0+1.0}
\begin{scriptsize}
\rput[tl](5.9,-0.11936170350306731){1. Achse}
\rput[tl](0.35,4.4162645344626466){2. Achse}
\rput[tl](1.3,2.3){$f_5$}
\end{scriptsize}
\end{pspicture*}},	 %5. Antwortmoeglichkeit
				L6={},	 %6. Antwortmoeglichkeit
				L7={},	 %7. Antwortmoeglichkeit
				L8={},	 %8. Antwortmoeglichkeit
				L9={},	 %9. Antwortmoeglichkeit
				%% LOESUNG: %%
				A1=3,  % 1. Antwort
				A2=0,	 % 2. Antwort
				A3=0,  % 3. Antwort
				A4=0,  % 4. Antwort
				A5=0,  % 5. Antwort
				}
\end{beispiel}