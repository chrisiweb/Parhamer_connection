\section{FA 1.2 - 6 - MAT - Formel als Funktion betrachten - ZO - FilJan}

\begin{beispiel}[FA 1.2]{1}
Gegeben ist folgende Formel:
\begin{center}
$G = a \cdot \ln(t+1)\cdot s^3 + \sqrt{u} \cdot s^4$
\end{center}

Ordne zu den vier gegebenen Funktionen den jeweils passenden Funktionstyp (aus A bis F) zu! 

\zuordnen[0.07]{
				R1={Wenn $G$: $ t \mapsto G(t)$ gilt:},				% Response 1
				R2={Wenn $G$: $ s \mapsto G(s)$ gilt:},				% Response 2
				R3={Wenn $G$: $ u \mapsto G(u)$ gilt: },				% Response 3
				R4={Wenn $G$: $ a \mapsto G(a)$ gilt: },				% Response 4
				%% Moegliche Zuordnungen: %%
				A={Polynomfunktion vierten Grades.}, 			%Moeglichkeit A  
				B={Lineare Funktion.}, 				%Moeglichkeit B  
				C={Exponentialfunktion.}, 				%Moeglichkeit C  
				D={Logarithmusfunktion.}, 				%Moeglichkeit D  
				E={Quadratische Funktion.}, 				%Moeglichkeit E  
				F={Wurzelfunktion.}, 				%Moeglichkeit F  
				%% LOESUNG: %%
				A1={D},				% 1. richtige Zuordnung
				A2={A},				% 2. richtige Zuordnung
				A3={F},				% 3. richtige Zuordnung
				A4={B},				% 4. richtige Zuordnung
				}
				
\end{beispiel}