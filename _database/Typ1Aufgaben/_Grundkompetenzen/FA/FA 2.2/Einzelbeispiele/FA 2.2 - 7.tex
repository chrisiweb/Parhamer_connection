\section{FA 2.2 - 7 - MAT - Steigung einer linearen Funktion - OA - Matura 2016/17 2. NT}

\begin{beispiel}[FA 2.2]{1} %PUNKTE DES BEISPIELS
Der Graph einer linearen Funktion $f$ verläuft durch die Punkte $A=(a|b)$ und $B=(5\cdot a|-3\cdot b)$ mit $a,b\in\mathbb{R}\backslash\{0\}$.

Bestimme die Steigung $k$ der linearen Funktion $f$!\leer

$k=$\,\antwort[\rule{3cm}{0.3pt}]{$-\frac{b}{a}$}
\end{beispiel}