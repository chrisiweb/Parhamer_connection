\section{FA 2.2 - 10 - MAT - Kerzenhöhe - OA - Matura 1.NT 2018/19}

\begin{beispiel}[FA 2.2]{1}
Eine brennende Kerze, die vor $t$ Stunden angezündet wurde, hat die Höhe $h(t)$. Für die Höhe der Kerze gilt dabei näherungsweise $h(t)=a\cdot t+b$ mit $a,b\in\mathbb{R}$.

Gib für jeden Koeffizienten $a$ und $b$ an, ob er positiv, negativ oder genau null sein muss.

\antwort{$a<0$ und $b>0$}
\end{beispiel}