\section{FA 2.2 - 11 - Dieselskandal - OA - ThoLin}

\begin{beispiel}[FA 2.2]{1}

Im Zuge des großen Dieselskandals wurde bei vielen Automodellen auch der Treibstoffverbrauch erneut überprüft. In einer kürzlich veröffentlichten Studie wurde die Tankfüllung (in Litern) eines neuen Kleinwagens in Abhängigkeit von der gefahrenen Strecke $x$ (in Kilometern) während einer Fahrt mithilfe der Funktion $T$ mit $T(x)=-0,07\cdot x+45$ beschrieben. 

Erläutere die Bedeutung von $-0,07$ und $45$ im Kontext des Treibstoffverbrauchs!\leer

$-0,07$: \rule{10cm}{0.3pt}\leer

$45$: \rule{10cm}{0.3pt}

\antwort{\textbf{Lösungserwartung:}\\
$-0,07$: beschreibt die Abnahme der Tankfüllung pro gefahrenen Kilometer
\newline
$45$: gibt die Tankfüllung (in Litern) zu Beginn der Messung an\leer

\textbf{Lösungsschlüssel:}
\newline
Ein Punkt für die korrekte Beantwortung beider Fragestellungen. Äquivalente Aussagen sind als richtig zu werten.}
\end{beispiel}