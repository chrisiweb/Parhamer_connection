\section{FA 2.2 - 9 - MAT - Wasserbehälter - OA - Matura-HT-18/19}

\begin{beispiel}[FA 2.2]{1}
In einem quaderförmigen Wasserbehälter steht eine Flüssigkeit 40\,cm hoch. Diese Flüssigkeit fließt ab dem Öffnen des Ablaufs in 8 Minuten vollständig ab.

Eine lineare Funktion $h$ mit $h(t)=k\cdot t+d$ beschreibt für $t\in[0;8]$ die Höhe (in cm) des Flüssigkeitspegels im Wasserbehälter $t$ Minuten ab dem Öffnen des Ablaufs.

Bestimme die Werte $k$ und $d$!\leer

$k=$\,\antwort[\rule{3cm}{0.3pt}]{$-5$}\leer

$d=$\,\antwort[\rule{3cm}{0.3pt}]{$40$}
\end{beispiel}