\section{FA 2.2 - 5 Steigung einer linearen Funktion - MC - Matura 2013/14 Haupttermin}

\begin{beispiel}{1} %PUNKTE DES BEISPIELS
			F�nf lineare Funktionen sind in verschiedener Weise dargestellt.
				
				Kreuze jene beiden Darstellungen an, bei denen die Steigung der dargestellten linearen Funktion den Wert $k=-2$ annimmt!\leer
				
				\multiplechoice[5]{  %Anzahl der Antwortmoeglichkeiten, Standard: 5
								L1={\begin{tabular}{|c|c|}\hline
								$x$&$m(x)$\\ \hline
								5&3\\ \hline
								6&1\\ \hline
								8&-3\\ \hline
								\end{tabular}},   %1. Antwortmoeglichkeit 
								L2={$g(x)=-2+3x$},   %2. Antwortmoeglichkeit
								L3={\begin{tabular}{|c|c|}\hline
								$x$&$h(x)$\\ \hline
								0&-2\\ \hline
								1&0\\ \hline
								2&2\\ \hline
								\end{tabular}},   %3. Antwortmoeglichkeit
								L4={\resizebox{0.5\linewidth}{!}{\psset{xunit=1.0cm,yunit=1.0cm,algebraic=true,dimen=middle,dotstyle=o,dotsize=5pt 0,linewidth=0.8pt,arrowsize=3pt 2,arrowinset=0.25}
\begin{pspicture*}(-4.3,-2.8)(4.7,4.66)
\multips(0,-2)(0,1.0){8}{\psline[linestyle=dashed,linecap=1,dash=1.5pt 1.5pt,linewidth=0.4pt,linecolor=lightgray]{c-c}(-4.3,0)(4.7,0)}
\multips(-4,0)(1.0,0){10}{\psline[linestyle=dashed,linecap=1,dash=1.5pt 1.5pt,linewidth=0.4pt,linecolor=lightgray]{c-c}(0,-2.8)(0,4.66)}
\psaxes[labelFontSize=\scriptstyle,xAxis=true,yAxis=true,Dx=1.,Dy=1.,ticksize=-2pt 0,subticks=2]{->}(0,0)(-4.3,-2.8)(4.7,4.66)[x,140] [f(x),-40]
\psplot{-4.3}{4.7}{(--2.--1.*x)/1.}
\begin{scriptsize}
\rput[tl](-1.16,1.48){f}
\end{scriptsize}
\end{pspicture*}}},   %4. Antwortmoeglichkeit
								L5={$l(x)=\frac{3-4x}{2}$},	 %5. Antwortmoeglichkeit
								L6={},	 %6. Antwortmoeglichkeit
								L7={},	 %7. Antwortmoeglichkeit
								L8={},	 %8. Antwortmoeglichkeit
								L9={},	 %9. Antwortmoeglichkeit
								%% LOESUNG: %%
								A1=1,  % 1. Antwort
								A2=5,	 % 2. Antwort
								A3=0,  % 3. Antwort
								A4=0,  % 4. Antwort
								A5=0,  % 5. Antwort
								}
\end{beispiel}