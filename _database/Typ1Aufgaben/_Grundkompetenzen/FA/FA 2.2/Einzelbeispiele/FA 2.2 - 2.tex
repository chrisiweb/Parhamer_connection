\section{FA 2.2 - 2 Gesprächsgebühr - OA - BIFIE}

\begin{beispiel}[FA 2.2]{1} %PUNKTE DES BEISPIELS
In der nachstehenden Abbildung ist der Graph zur Berechnung eines Handytarifs dargestellt.

Der Tarif sieht eine monatliche Grundgebühr vor, die eine gewisse Anzahl an Freiminuten (für diese Anzahl an Minuten ist keine zusätzliche Gesprächsgebühr vorgesehen) beinhaltet.

\begin{center}
\psset{xunit=0.01cm,yunit=.1cm,algebraic=true,dimen=middle,dotstyle=o,dotsize=5pt 0,linewidth=0.8pt,arrowsize=3pt 2,arrowinset=0.25}
\begin{pspicture*}(-75.95300203062286,-7.260369620270227)(1196.6538277596112,55.7255157481786)
\multips(0,0)(0,10.0){7}{\psline[linestyle=dashed,linecap=1,dash=1.5pt 1.5pt,linewidth=0.4pt,linecolor=lightgray]{c-c}(0,0)(1196.6538277596112,0)}
\multips(0,0)(100.0,0){13}{\psline[linestyle=dashed,linecap=1,dash=1.5pt 1.5pt,linewidth=0.4pt,linecolor=lightgray]{c-c}(0,0)(0,55.7255157481786)}
\psaxes[labelFontSize=\scriptstyle,xAxis=true,yAxis=true,Dx=100.,Dy=10.,ticksize=-2pt 0,subticks=2]{->}(0,0)(0.,0.)(1196.6538277596112,55.7255157481786)
\rput[tl](27.181560152087858,53.883823193545595){Kosten in Euro}
\rput[tl](850.9757336924288,6.368155284014023){Gesprächsminuten}
\psline[linewidth=1.6pt](0.,10.)(500.,10.)
\psline[linewidth=1.6pt](500.,10.)(831.9994828993126,59.77723936837122)
\end{pspicture*}
\end{center}

Bestimme die Gesprächskosten pro Minute, wenn die Anzahl der Freiminuten überschritten wird!
\leer

\antwort{15 Cent bzw. \euro\,0,15}
\end{beispiel}