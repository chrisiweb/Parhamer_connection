\section{FA 3.3 - 1 - Verschiebung Quadratische Funktion - ZO - MatKon}

\begin{beispiel}[FA 3.3]{1} %PUNKTE DES BEISPIELS
				Ordne den folgenden Graphen jeweils die entsprechende Funktionsgleichung zu! \leer
				
				\zuordnen{
				R1={
\psset{xunit=0.5cm,yunit=0.5cm,algebraic=true,dimen=middle,dotstyle=o,dotsize=3pt 0,linewidth=0.8pt,arrowsize=3pt 2,arrowinset=0.25}				
				\begin{pspicture*}(-4.5,-3.5)(4.5,3.5)
\multips(0,-3)(0,1.0){7}{\psline[linestyle=dashed,linecap=1,dash=1.5pt 1.5pt,linewidth=0.4pt,linecolor=gray]{c-c}(-4.5,0)(4.5,0)}
\multips(-4,0)(1.0,0){9}{\psline[linestyle=dashed,linecap=1,dash=1.5pt 1.5pt,linewidth=0.4pt,linecolor=gray]{c-c}(0,-3.5)(0,3.5)}
\psaxes[labelFontSize=\scriptscriptstyle,showorigin=false,xAxis=true,yAxis=true,Dx=1.,Dy=1.,ticksize=-2pt 0,subticks=0]{->}(0,0)(-4.5,-3.5)(4.5,3.5)
\psplot[linewidth=1.pt,plotpoints=200]{-45.0}{4.5}{(x+3.0)^(2.0)-2.0}
\end{pspicture*}},				% Response 1
				R2={
\psset{xunit=0.5cm,yunit=0.5cm,algebraic=true,dimen=middle,dotstyle=o,dotsize=3pt 0,linewidth=0.8pt,arrowsize=3pt 2,arrowinset=0.25}				
				\begin{pspicture*}(-4.5,-3.5)(4.5,3.5)
\multips(0,-3)(0,1.0){7}{\psline[linestyle=dashed,linecap=1,dash=1.5pt 1.5pt,linewidth=0.4pt,linecolor=gray]{c-c}(-4.5,0)(4.5,0)}
\multips(-4,0)(1.0,0){9}{\psline[linestyle=dashed,linecap=1,dash=1.5pt 1.5pt,linewidth=0.4pt,linecolor=gray]{c-c}(0,-3.5)(0,3.5)}
\psaxes[labelFontSize=\scriptscriptstyle,showorigin=false,xAxis=true,yAxis=true,Dx=1.,Dy=1.,ticksize=-2pt 0,subticks=0]{->}(0,0)(-4.5,-3.5)(4.5,3.5)
\psplot[linewidth=1.pt,plotpoints=200]{-4.5}{4.5}{-(x+2.0)^(2.0)-1.0}
\end{pspicture*}},				% Response 2
				R3={\psset{xunit=0.5cm,yunit=0.5cm,algebraic=true,dimen=middle,dotstyle=o,dotsize=3pt 0,linewidth=0.8pt,arrowsize=3pt 2,arrowinset=0.25}
\begin{pspicture*}(-4.5,-3.5)(4.5,3.5)
\multips(0,-3)(0,1.0){7}{\psline[linestyle=dashed,linecap=1,dash=1.5pt 1.5pt,linewidth=0.4pt,linecolor=gray]{c-c}(-4.5,0)(4.5,0)}
\multips(-4,0)(1.0,0){9}{\psline[linestyle=dashed,linecap=1,dash=1.5pt 1.5pt,linewidth=0.4pt,linecolor=gray]{c-c}(0,-3.5)(0,3.5)}
\psaxes[labelFontSize=\scriptscriptstyle,showorigin=false,xAxis=true,yAxis=true,Dx=1.,Dy=1.,ticksize=-2pt 0,subticks=0]{->}(0,0)(-4.5,-3.5)(4.5,3.5)
\psplot[linewidth=1.pt,plotpoints=200]{-4.5}{4.5}{-(x-1.0)^(2.0)+1.0}
\end{pspicture*}},				% Response 3
				R4={
\psset{xunit=0.5cm,yunit=0.5cm,algebraic=true,dimen=middle,dotstyle=o,dotsize=3pt 0,linewidth=0.8pt,arrowsize=3pt 2,arrowinset=0.25}				
\begin{pspicture*}(-4.5,-3.5)(4.5,3.5)
\multips(0,-3)(0,1.0){7}{\psline[linestyle=dashed,linecap=1,dash=1.5pt 1.5pt,linewidth=0.4pt,linecolor=gray]{c-c}(-4.5,0)(4.5,0)}
\multips(-4,0)(1.0,0){9}{\psline[linestyle=dashed,linecap=1,dash=1.5pt 1.5pt,linewidth=0.4pt,linecolor=gray]{c-c}(0,-3.5)(0,3.5)}
\psaxes[labelFontSize=\scriptscriptstyle,showorigin=false,xAxis=true,yAxis=true,Dx=1.,Dy=1.,ticksize=-2pt 0,subticks=0]{->}(0,0)(-4.5,-3.5)(4.5,3.5)
\psplot[linewidth=1.pt,plotpoints=200]{-4.5}{4.5}{(x-1.0)^(2.0)+2.0}
\end{pspicture*}},				% Response 4
				%% Moegliche Zuordnungen: %%
				A={$(x-1)^2+2$}, 				%Moeglichkeit A  
				B={$(x+3)^2-2$}, 				%Moeglichkeit B  
				C={$-(x-2)^2+1$}, 				%Moeglichkeit C  
				D={$-(x+2)^2-1$}, 				%Moeglichkeit D  
				E={$-(x-1)^2+1$}, 				%Moeglichkeit E  
				F={$(x+2)^2-1$}, 				%Moeglichkeit F  
				%% LOESUNG: %%
				A1={B},				% 1. richtige Zuordnung
				A2={D},				% 2. richtige Zuordnung
				A3={E},				% 3. richtige Zuordnung
				A4={A},				% 4. richtige Zuordnung
				}
\end{beispiel}