\section{FA 3.3 - 1 Verschiebung Quadratische Funktion - ZO - MK}

\begin{beispiel}[FA 3.3]{1} %PUNKTE DES BEISPIELS
				Ordne den folgenden Graphen jeweils die entsprechende Funktionsgleichung zu! \leer
				
				
				\zuordnen[0.06]{
								title1={Gleichung}, 		%Titel Antwortmoeglichkeiten
								A={$(x-1)^2+2$}, 				%Moeglichkeit A  
								B={$(x+3)^2-2$}, 				%Moeglichkeit B  
								C={$-(x-2)^2+1$}, 				%Moeglichkeit C  
								D={$-(x+2)^2-1$}, 				%Moeglichkeit D  
								E={$-(x-1)^2+1$}, 				%Moeglichkeit E  
								F={$(x+2)^2-1$}, 				%Moeglichkeit F  
								title2={Graphen},		%Titel Zuordnung
								R1={\resizebox{0.7\linewidth}{!}{\newrgbcolor{cqcqcq}{0.7529411764705882 0.7529411764705882 0.7529411764705882}
\begin{pspicture*}(-4.,-3.)(4.,3.)
\psset{xunit=1.0cm,yunit=1.0cm,algebraic=true,dimen=middle,dotstyle=o,dotsize=3pt 0,linewidth=0.8pt,arrowsize=3pt 2,arrowinset=0.25}
\multips(0,-3)(0,1.0){7}{\psline[linestyle=dashed,linecap=1,dash=1.5pt 1.5pt,linewidth=0.4pt,linecolor=lightgray]{c-c}(-4.,0)(4.,0)}
\multips(-4,0)(1.0,0){9}{\psline[linestyle=dashed,linecap=1,dash=1.5pt 1.5pt,linewidth=0.4pt,linecolor=lightgray]{c-c}(0,-3.)(0,3.)}
\psaxes[labelFontSize=\scriptstyle,xAxis=true,yAxis=true,Dx=1.,Dy=1.,ticksize=-2pt 0,subticks=2]{->}(0,0)(-4.,-3.)(4.,3.)
\psplot[linewidth=2.pt,plotpoints=200]{-4.0}{4.0}{(x+3.0)^(2.0)-2.0}
\end{pspicture*}}},				%1. Antwort rechts
								R2={\resizebox{0.7\linewidth}{!}{\newrgbcolor{cqcqcq}{0.7529411764705882 0.7529411764705882 0.7529411764705882}
\begin{pspicture*}(-4.,-3.)(4.,3.)
\psset{xunit=1.0cm,yunit=1.0cm,algebraic=true,dimen=middle,dotstyle=o,dotsize=3pt 0,linewidth=0.8pt,arrowsize=3pt 2,arrowinset=0.25}
\multips(0,-3)(0,1.0){7}{\psline[linestyle=dashed,linecap=1,dash=1.5pt 1.5pt,linewidth=0.4pt,linecolor=lightgray]{c-c}(-4.,0)(4.,0)}
\multips(-4,0)(1.0,0){9}{\psline[linestyle=dashed,linecap=1,dash=1.5pt 1.5pt,linewidth=0.4pt,linecolor=lightgray]{c-c}(0,-3.)(0,3.)}
\psaxes[labelFontSize=\scriptstyle,xAxis=true,yAxis=true,Dx=1.,Dy=1.,ticksize=-2pt 0,subticks=2]{->}(0,0)(-4.,-3.)(4.,3.)
\psplot[linewidth=2.pt,plotpoints=200]{-4.0}{4.0}{-(x+2.0)^(2.0)-1.0}
\end{pspicture*}}},				%2. Antwort rechts
								R3={\resizebox{0.7\linewidth}{!}{\newrgbcolor{cqcqcq}{0.7529411764705882 0.7529411764705882 0.7529411764705882}
\begin{pspicture*}(-4.,-3.)(4.,3.)
\psset{xunit=1.0cm,yunit=1.0cm,algebraic=true,dimen=middle,dotstyle=o,dotsize=3pt 0,linewidth=0.8pt,arrowsize=3pt 2,arrowinset=0.25}
\multips(0,-3)(0,1.0){7}{\psline[linestyle=dashed,linecap=1,dash=1.5pt 1.5pt,linewidth=0.4pt,linecolor=lightgray]{c-c}(-4.,0)(4.,0)}
\multips(-4,0)(1.0,0){9}{\psline[linestyle=dashed,linecap=1,dash=1.5pt 1.5pt,linewidth=0.4pt,linecolor=lightgray]{c-c}(0,-3.)(0,3.)}
\psaxes[labelFontSize=\scriptstyle,xAxis=true,yAxis=true,Dx=1.,Dy=1.,ticksize=-2pt 0,subticks=2]{->}(0,0)(-4.,-3.)(4.,3.)
\psplot[linewidth=2.pt,plotpoints=200]{-4.0}{4.0}{-(x-1.0)^(2.0)+1.0}
\end{pspicture*}}},				%3. Antwort rechts
								R4={\resizebox{0.7\linewidth}{!}{\newrgbcolor{cqcqcq}{0.7529411764705882 0.7529411764705882 0.7529411764705882}
\begin{pspicture*}(-4.,-3.)(4.,3.)
\psset{xunit=1.0cm,yunit=1.0cm,algebraic=true,dimen=middle,dotstyle=o,dotsize=3pt 0,linewidth=0.8pt,arrowsize=3pt 2,arrowinset=0.25}
\multips(0,-3)(0,1.0){7}{\psline[linestyle=dashed,linecap=1,dash=1.5pt 1.5pt,linewidth=0.4pt,linecolor=lightgray]{c-c}(-4.,0)(4.,0)}
\multips(-4,0)(1.0,0){9}{\psline[linestyle=dashed,linecap=1,dash=1.5pt 1.5pt,linewidth=0.4pt,linecolor=lightgray]{c-c}(0,-3.)(0,3.)}
\psaxes[labelFontSize=\scriptstyle,xAxis=true,yAxis=true,Dx=1.,Dy=1.,ticksize=-2pt 0,subticks=2]{->}(0,0)(-4.,-3.)(4.,3.)
\psplot[linewidth=2.pt,plotpoints=200]{-4.0}{4.0}{(x-1.0)^(2.0)+2.0}
\end{pspicture*}}},				%4. Antwort rechts
								%% LOESUNG: %%
								A1={B},				% 1. richtige Zuordnung
								A2={D},				% 2. richtige Zuordnung
								A3={E},				% 3. richtige Zuordnung
								A4={A},				% 4. richtige Zuordnung
								}
\end{beispiel}