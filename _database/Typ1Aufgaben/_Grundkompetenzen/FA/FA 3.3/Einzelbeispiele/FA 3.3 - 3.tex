\section{FA 3.3 - 3 - MAT - Wurzelfunktion - OA - Matura NT 2 15/16}

\begin{beispiel}[FA 3.3]{1} %PUNKTE DES BEISPIELS
In der nachstehenden Abbildung ist der Graph einer Funktion $f$ mit \mbox{$f(x)=a\cdot x^{\frac{1}{2}}+b\,(a,b\in\mathbb{R},a\neq 0)$} dargestellt.

Die Koordinaten der hervorgehobenen Punkte des Graphen der Funktion sind ganzzahlig.

\begin{center}
\psset{xunit=1.0cm,yunit=1.0cm,algebraic=true,dimen=middle,dotstyle=o,dotsize=5pt 0,linewidth=0.8pt,arrowsize=3pt 2,arrowinset=0.25}
\begin{pspicture*}(-2.12,-1.5)(7.92,4.78)
\multips(0,-1)(0,1.0){7}{\psline[linestyle=dashed,linecap=1,dash=1.5pt 1.5pt,linewidth=0.4pt,linecolor=gray]{c-c}(-2.12,0)(7.92,0)}
\multips(-2,0)(1.0,0){11}{\psline[linestyle=dashed,linecap=1,dash=1.5pt 1.5pt,linewidth=0.4pt,linecolor=gray]{c-c}(0,-1.5)(0,4.78)}
\psaxes[labelFontSize=\scriptstyle,xAxis=true,yAxis=true,Dx=1.,Dy=1.,showorigin=false,ticksize=-2pt 0,subticks=0]{->}(0,0)(-2.12,-1.5)(7.92,4.78)[$x$,140] [$f(x)$,-40]
\psplot[linewidth=2.pt,plotpoints=200]{1.5200000842441344E-8}{7.920000000000004}{sqrt(x)+2.0}
\psdots[dotsize=6pt 0,dotstyle=*](0.,2.)
\psdots[dotsize=6pt 0,dotstyle=*](1.,3.)
\psdots[dotsize=6pt 0,dotstyle=*](4.,4.)
\rput[bl](0.74,3.32){$f$}
\end{pspicture*}
\end{center}

Gib die Werte von $a$ und $b$ an!\leer

$a=$ \antwort[\rule{3cm}{0.3pt}]{1}\leer

$b=$ \antwort[\rule{3cm}{0.3pt}]{2}
\end{beispiel}