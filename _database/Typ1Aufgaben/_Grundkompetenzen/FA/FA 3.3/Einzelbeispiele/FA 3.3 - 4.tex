\section{FA 3.3 - 4 - MAT - Quadratische Funktion - MC - Matura 2013/14 1. Nebentermin}

\begin{beispiel}[FA 3.3]{1} %PUNKTE DES BEISPIELS
				Eine quadratische Funktion $f$ der Form $f(x)=a\cdot x^2+b$ mit $a,b\in\mathbb{R}$ und $a\neq 0$ ist gegeben.
				
				Kreuze die zutreffende(n) Aussage(n) an!\leer
				
				\multiplechoice[5]{  %Anzahl der Antwortmoeglichkeiten, Standard: 5
								L1={Der Graph der Funktion $f$ hat zwei verschiedene reelle Nullstellen, wenn gilt: $a>0$ und $b<0$.},   %1. Antwortmoeglichkeit 
								L2={Der Graph der Funktion $f$ mit $b=0$ berührt die x-Achse in der lokalen Extremstelle.},   %2. Antwortmoeglichkeit
								L3={Der Graph der Funktion $f$ mit $b>0$ berührt die x-Achse im Ursprung.},   %3. Antwortmoeglichkeit
								L4={Für $a<0$ hat der Graph der Funktion $f$ einen Hochpunkt.},   %4. Antwortmoeglichkeit
								L5={Für die lokale Extremstelle $x_s$ der Funktion $f$ gilt immer: $x_s=b$.},	 %5. Antwortmoeglichkeit
								L6={},	 %6. Antwortmoeglichkeit
								L7={},	 %7. Antwortmoeglichkeit
								L8={},	 %8. Antwortmoeglichkeit
								L9={},	 %9. Antwortmoeglichkeit
								%% LOESUNG: %%
								A1=1,  % 1. Antwort
								A2=2,	 % 2. Antwort
								A3=4,  % 3. Antwort
								A4=0,  % 4. Antwort
								A5=0,  % 5. Antwort
								}
\end{beispiel}