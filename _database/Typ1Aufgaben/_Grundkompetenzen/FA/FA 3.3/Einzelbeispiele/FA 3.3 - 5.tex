\section{FA 3.3 - 5 - MAT - Parabeln - ZO - Matura 1.NT 2018/19}

\begin{beispiel}[FA 3.3]{1}
Die Graphen von Funktionen $f$: $\mathbb{R}\rightarrow\mathbb{R}$ mit $f(x)=a\cdot x^2$ mit $a\in\mathbb{R}\backslash\{0\}$ sind Parabeln. Für $a=1$ erhält man den oft als \textit{Normalparabel} bezeichneten Graphen. Je nach Wert des Parameters $a$ erhält man Parabeln, die im Vergleich zur Normalparabel "`steiler"' oder "`flacher"' bzw. "`nach unten offen"' oder "`nach oben offen"' sind.

\begin{center}
\newrgbcolor{wwwwww}{0.4 0.4 0.4}
\psset{xunit=0.8cm,yunit=0.8cm,algebraic=true,dimen=middle,dotstyle=o,dotsize=5pt 0,linewidth=1.6pt,arrowsize=3pt 2,arrowinset=0.25}
\begin{pspicture*}(-5.5,-5.5)(5.5,5.5)
\multips(0,-9)(0,1){16}{\psline[linestyle=dashed,linecap=1,dash=1.5pt 1.5pt,linewidth=0.4pt,linecolor=gray]{c-c}(-9.48,0)(14.28,0)}
\multips(-9,0)(1,0){24}{\psline[linestyle=dashed,linecap=1,dash=1.5pt 1.5pt,linewidth=0.4pt,linecolor=gray]{c-c}(0,-9.66)(0,5.7)}
\psaxes[labelFontSize=\scriptstyle,showorigin=false,xAxis=true,yAxis=true,Dx=1,Dy=1,ticksize=-2pt 0,subticks=0]{->}(0,0)(-5.5,-5.5)(5.5,5.5)[x,140] [f(x),-40]
\psplot[linewidth=3.2pt,linecolor=wwwwww,plotpoints=200]{-9.48000000000001}{14.280000000000015}{x^(2)}
\psplot[linewidth=3.2pt,plotpoints=200]{-9.48000000000001}{14.280000000000015}{3*x^(2)}
\psplot[linewidth=2pt,plotpoints=200]{-9.48000000000001}{14.280000000000015}{0.13*x^(2)}
\psplot[linewidth=2pt,plotpoints=200]{-9.48000000000001}{14.280000000000015}{-x^(2)}
\begin{scriptsize}
\rput[bl](-4.9,4.56){\wwwwww{Normalparabel}}
\rput[bl](-4.9,4.26){\wwwwww{nach oben offen}}
\rput[bl](-1.02,4.52){steiler}
\rput[bl](-3.36,1.7){flacher}
\rput[bl](-5,-3.98){Normalparabel}
\rput[bl](-5,-4.28){nach unten offen}
\end{scriptsize}
\end{pspicture*}
\end{center}

Nachstehend sind vier Parabeln beschrieben. Ordne den vier Beschreibungen jeweils diejenige Bedingung (aus A bis F) zu, die der Parameter $a$ erfüllen muss.

\zuordnen[0.22]{
				R1={Die Parabel ist im Vergleich zur Normalparabel "`flacher"' und "`nach oben offen"'.},				% Response 1
				R2={Die Parabel ist im Vergleich zur Normalparabel weder "`flacher"' noch "`steiler"', aber "`nach unten offen"'.},				% Response 2
				R3={Die Parabel ist im Vergleich zur Normalparabel "`steiler"' und "`nach unten offen"'.},				% Response 3
				R4={Die Parabel ist im Vergleich zur Normalparabel "`steiler"' und "`nach oben offen"'.},				% Response 4
				%% Moegliche Zuordnungen: %%
				A={$a<-1$}, 				%Moeglichkeit A  
				B={$a=1$}, 				%Moeglichkeit B  
				C={$-1<a<0$}, 				%Moeglichkeit C  
				D={$0<a<1$}, 				%Moeglichkeit D  
				E={$a=1$}, 				%Moeglichkeit E  
				F={$a>1$}, 				%Moeglichkeit F  
				%% LOESUNG: %%
				A1={D},				% 1. richtige Zuordnung
				A2={B},				% 2. richtige Zuordnung
				A3={A},				% 3. richtige Zuordnung
				A4={F},				% 4. richtige Zuordnung
				}
				
				\antwort{Ein Punkt ist genau dann zu geben, wenn jeder der vier Aussagen ausschließlich der laut Lösungserwartung richtige Buchstabe zugeordnet ist. Bei zwei oder drei richtigen Zuordnungen ist ein
halber Punkt zu geben.}
\end{beispiel}