\section{FA 3.3 - 2 Wirkung der Parameter - MC - BIFIE}

\begin{beispiel}[FA 3.3]{1} %PUNKTE DES BEISPIELS
Gegeben ist eine Potenzfunktion $g$ mit der Gleichung $g(x) = c \cdot x^2 + d$ mit $c < 0$ und $d > 0$.
\leer

Kreuzen die beiden f�r $g$ zutreffenden Aussagen an.

\multiplechoice[5]{  %Anzahl der Antwortmoeglichkeiten, Standard: 5
				L1={$g$ schneidet die y-Achse im Punkt $P = (d|0)$.},   %1. Antwortmoeglichkeit 
				L2={$g$ besitzt zwei Nullstellen.},   %2. Antwortmoeglichkeit
				L3={Je gr��er $d$ ist, umso steiler verl�uft der Graph von $g$.},   %3. Antwortmoeglichkeit
				L4={Je kleiner $c$ ist, umso flacher verl�uft der Graph von $g$.},   %4. Antwortmoeglichkeit
				L5={$g$ besitzt einen Hochpunkt},	 %5. Antwortmoeglichkeit
				L6={},	 %6. Antwortmoeglichkeit
				L7={},	 %7. Antwortmoeglichkeit
				L8={},	 %8. Antwortmoeglichkeit
				L9={},	 %9. Antwortmoeglichkeit
				%% LOESUNG: %%
				A1=2,  % 1. Antwort
				A2=5,	 % 2. Antwort
				A3=0,  % 3. Antwort
				A4=0,  % 4. Antwort
				A5=0,  % 5. Antwort
				}
\end{beispiel}