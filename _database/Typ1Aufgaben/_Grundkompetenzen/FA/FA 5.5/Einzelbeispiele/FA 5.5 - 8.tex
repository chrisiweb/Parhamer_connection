\section{FA 5.5 - 8 - MAT - Dicke einer Bleischicht - OA - Matura NT 1 16/17}

\begin{beispiel}[FA 5.5]{1} %PUNKTE DES BEISPIELS
Die Intensität elektromagnetischer Strahlung nimmt bei Durchdringung eines Körpers exponentiell ab.

Die Halbwertsdicke eines Materials ist diejenige Dicke, nach deren Durchdringung die Intensität der Strahlung auf die Hälfte gesunken ist. Die Halbwertsdicke von Blei liegt für die beobachtete Strahlung bei 0,4\,cm.

Bestimme diejenige Dicke $d$, die eine Bleischicht haben muss, damit die Intensität auf 12,5\,\% der ursprünglichen Intensität gesunken ist!\leer

$d=$ \antwort[\rule{3cm}{0.3pt}]{1,2}\,cm
\end{beispiel}