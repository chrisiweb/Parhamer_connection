\section{FA 5.5 - 8 Dicke einer Bleischicht - OA - Matura NT 1 16/17}

\begin{beispiel}[FA 5.5]{1} %PUNKTE DES BEISPIELS
Die Intensit�t elektromagnetischer Strahlung nimmt bei Durchdringung eines K�rpers exponentiell ab.

Die Halbwertsdicke eines Materials ist diejenige Dicke, nach deren Durchdringung die Intensit�t der Strahlung auf die H�lfte gesunken ist. Die Halbwertsdicke von Blei liegt f�r die beobachtete Strahlung bei 0,4\,cm.

Bestimme diejenige Dicke $d$, die eine Bleischicht haben muss, damit die Intensit�t auf 12,5\,\% der urspr�nglichen Intensit�t gesunken ist!\leer

$d=$ \antwort[\rule{3cm}{0.3pt}]{1,2}\,cm
\end{beispiel}