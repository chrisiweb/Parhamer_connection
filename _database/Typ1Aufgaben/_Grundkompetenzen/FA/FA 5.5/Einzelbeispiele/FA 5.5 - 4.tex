\section{FA 5.5 - 4 - Biologische Halbwertszeit - OA - BIFIE}


\begin{beispiel}[FA 5.5]{1} %PUNKTE DES BEISPIELS
Die biologische Halbwertszeit bezeichnet diejenige Zeitspanne, in der in einem biologischen
Organismus (Mensch, Tier, \ldots) der Gehalt von zum Beispiel einem Arzneimittel ausschließlich durch biologische Prozesse (Stoffwechsel, Ausscheidung usw.) auf die Hälfte abgesunken ist. Für das Arzneimittel \textit{Penicillin G} wird bei Erwachsenen eine biologische Halbwertszeit von 30 Minuten angegeben. 
\leer

Einer Person wird um 10:00 Uhr eine Dosis \textit{Penicillin G} verabreicht.
Ermittle, wie viel Prozent der ursprünglichen Dosis vom Körper der Person bis 11:00 Uhr noch nicht verarbeitet wurden.


\antwort{Zwischen 10:00 Uhr und 11:00 Uhr hat sich die noch nicht verarbeitete \textit{Penicillin-G}-Dosis
zweimal halbiert. Bis 11:00 Uhr wurden also 25\,\% der ursprünglichen Dosis noch nicht verarbeitet.}
\end{beispiel}