\section{FA 5.5 - 6 - MAT - Bienenbestand - OA - Matura 2015/16 - Nebentermin 1}

\begin{beispiel}[FA 5.5]{1} %PUNKTE DES BEISPIELS
Wegen eines Umweltgifts nimmt der Bienenbestand eines Imkers täglich um einen fixen Prozentsatz
ab. Der Imker stellt fest, dass er innerhalb von 14 Tagen einen Bestandsverlust von 50\,\%
erlitten hat.\leer

Berechne den täglichen relativen Bestandsverlust in Prozent. \leer

täglicher relativer Bestandsverlust:\rule{4cm}{0.3pt}\,\%

\antwort{$N_0\cdot 0,5=N_0\cdot a^{14}$ \\
$0,5=a^{14} \Rightarrow a\approx 0,9517$ \\
täglich relativer Bestandsverlust: 4,83\,\% \leer

Lösungsschlüssel:

Ein Punkt für die richtige Lösung.
Toleranzintervall: $[4,8\,\%;~ 4,9\,\%]$

Die Aufgabe ist auch dann als richtig gelöst zu werten, wenn bei korrektem Ansatz das Ergebnis
aufgrund eines Rechenfehlers nicht richtig ist.
}
\end{beispiel}