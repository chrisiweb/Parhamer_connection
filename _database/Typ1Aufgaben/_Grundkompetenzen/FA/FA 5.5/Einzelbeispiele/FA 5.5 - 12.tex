\section{FA 5.5 - 12 - Zerfall und Halbwertszeit - LT - FraKol UNIVIE}

\begin{beispiel}[FA 5.5]{1}
Der Zerfall des Bleiisotops $^{189}$Pb kann mit der Funktion $N$ mit
$N(t) = N_0 \cdot e^{-0,0135911\cdot t}$ ($t$ in Sekunden) beschrieben werden.
$N(t)$ gibt dabei die Anzahl der zum Zeitpunkt t noch nicht zerfallenen $^{189}$Pb-Atomkerne an.

\lueckentext{
				text={Der Zerfall dieses Bleiisotops hat eine Halbwertszeit von ca. \gap Sekunden, also sind nach 105 Sekunden ca. \gap\% der Atomkerne noch nicht zerfallen.}, 	%Lueckentext Luecke=\gap
				L1={51}, 		%1.Moeglichkeit links  
				L2={93}, 		%2.Moeglichkeit links
				L3={27}, 		%3.Moeglichkeit links
				R1={61,91}, 		%1.Moeglichkeit rechts 
				R2={24,00}, 		%2.Moeglichkeit rechts
				R3={14,12}, 		%3.Moeglichkeit rechts
				%% LOESUNG: %%
				A1=1,   % Antwort links
				A2=2		% Antwort rechts 
				}
\end{beispiel}