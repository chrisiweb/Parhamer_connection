\section{FA 5.5 - 7 - MAT - Halbwertszeit von Cobalt-60 - OA - Matura 2016/17 - Haupttermin}

\begin{beispiel}[FA 5.5]{1} %PUNKTE DES BEISPIELS
Das radioaktive Isotop Cobalt-60 wird unter anderem zur Konservierung von Lebensmitteln und in
der Medizin verwendet. Das Zerfallsgesetz für Cobalt-60 lautet $N(t) = N_0\cdot e^{-0,13149 \cdot t}$ mit $t$ in Jahren; dabei bezeichnet $N_0$ die vorhandene Menge des Isotops zum Zeitpunkt $t = 0$ und $N(t)$ die vorhandene Menge zum Zeitpunkt $t \geq 0$. \leer

Berechne die Halbwertszeit von Cobalt-60!

\antwort{Mögliche Berechnung: 

$\frac{N_0}{2}=N_0\cdot e^{-0,13149\cdot t} \Rightarrow t \approx 5,27$ \leer

Die Halbwertszeit von Cobalt-60 beträgt ca. 5,27 Jahre.

Lösungsschlüssel:

Ein Punkt für die richtige Lösung, wobei die Einheit "`Jahre"' nicht angegeben sein muss.
Toleranzintervall: [5 Jahre; 5,5 Jahre]

Die Aufgabe ist auch dann als richtig gelöst zu werten, wenn bei korrektem Ansatz das Ergebnis
aufgrund eines Rechenfehlers nicht richtig ist.}
\end{beispiel}