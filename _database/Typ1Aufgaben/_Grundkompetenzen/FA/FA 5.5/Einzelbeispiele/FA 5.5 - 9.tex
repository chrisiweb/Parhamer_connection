\section{FA 5.5 - 9 - MAT - Halbwertszeiten - ZO - Matura 2016/17 2. NT}

\begin{beispiel}[FA 5.5]{1} %PUNKTE DES BEISPIELS
Die nachstehenden Abbildungen zeigen die Graphen von Exponentialfunktionen, die jeweils die Abhängigkeit der Menge einer radioaktiven Substanz von der Zeit beschreiben.\\
Dabei gibt $M(t)$ die Menge (in mg) zum Zeitpunkt $t$ (in Tagen) an.

Ordne den vier Graphen jeweils die entsprechende Halbwertszeit (aus A bis F) zu!\leer

\zuordnen{
				R1={\psset{xunit=0.35cm,yunit=0.4cm,algebraic=true,dimen=middle,dotstyle=o,dotsize=5pt 0,linewidth=0.6pt,arrowsize=3pt 2,arrowinset=0.25}
\begin{pspicture*}(-1.2,-1.72)(10.5,9.44)
\multips(0,0)(0,2.0){6}{\psline[linestyle=dashed,linecap=1,dash=1.5pt 1.5pt,linewidth=0.4pt,linecolor=gray]{c-c}(0,0)(10.5,0)}
\multips(0,0)(1.0,0){12}{\psline[linestyle=dashed,linecap=1,dash=1.5pt 1.5pt,linewidth=0.4pt,linecolor=gray]{c-c}(0,0)(0,9.44)}
\begin{scriptsize}
\psaxes[xAxis=true,yAxis=true,Dx=1.,Dy=2.,ticksize=-2pt 0,subticks=0]{->}(0,0)(0.,0.)(10.5,9.44)
\psplot[linewidth=0.6pt,plotpoints=200]{0}{10.780}{8.05*2.71828^(-0.139954*x)}
\rput[tl](0.22,9.02){$M(t)$}
\rput[tl](6.18,0.68){$t$ in Tagen}
\end{scriptsize}
\end{pspicture*}},				% Response 1
				R2={\psset{xunit=0.34cm,yunit=3cm,algebraic=true,dimen=middle,dotstyle=o,dotsize=5pt 0,linewidth=0.6pt,arrowsize=3pt 2,arrowinset=0.25}
\begin{pspicture*}(-2,-0.2032685389273154)(10.5,1.07284537663376)
\multips(0,0)(0,0.1){12}{\psline[linestyle=dashed,linecap=1,dash=1.5pt 1.5pt,linewidth=0.4pt,linecolor=gray]{c-c}(0,0)(10.5,0)}
\multips(0,0)(1.0,0){12}{\psline[linestyle=dashed,linecap=1,dash=1.5pt 1.5pt,linewidth=0.4pt,linecolor=gray]{c-c}(0,0)(0,1.07284537663376)}
\begin{scriptsize}
\psaxes[comma,xAxis=true,yAxis=true,Dx=1.,Dy=0.2,ticksize=-2pt 0,subticks=0]{->}(0,0)(0.,0.)(10.5,1.07284537663376)
\rput[tl](6.54,0.09466282735003631){$t$ in Tagen}
\rput[tl](0.22,1.052765382953449){$M(t)$}
\psplot[linewidth=0.6pt,plotpoints=200]{0}{11.019999999999994}{1.0051346484550836*EXP(-0.07235929100754346*x)}
\end{scriptsize}
\end{pspicture*}},				% Response 2
				R3={\psset{xunit=0.34cm,yunit=0.035cm,algebraic=true,dimen=middle,dotstyle=o,dotsize=5pt 0,linewidth=0.6pt,arrowsize=3pt 2,arrowinset=0.25}
\begin{pspicture*}(-1.78,-13)(11.04,69.95983539094611)
\multips(0,0)(0,15.0){6}{\psline[linestyle=dashed,linecap=1,dash=1.5pt 1.5pt,linewidth=0.4pt,linecolor=gray]{c-c}(0,0)(11.04,0)}
\multips(0,0)(1.0,0){12}{\psline[linestyle=dashed,linecap=1,dash=1.5pt 1.5pt,linewidth=0.4pt,linecolor=gray]{c-c}(0,0)(0,69.95983539094611)}
\begin{scriptsize}
\psaxes[xAxis=true,yAxis=true,Dx=1.,Dy=15.,ticksize=-2pt 0,subticks=0]{->}(0,0)(0.,0.)(11.04,69.95983539094611)
\rput[tl](0.26,66.0184362139914){$M(t)$}
\rput[tl](6.52,8){$t$ in Tagen}
\psplot[linewidth=0.6pt,plotpoints=200]{0}{11.040000000000001}{60.000000000000036*EXP(-0.23104906018664828*x)}
\end{scriptsize}
\end{pspicture*}},				% Response 3
				R4={\psset{xunit=0.33cm,yunit=0.02cm,algebraic=true,dimen=middle,dotstyle=o,dotsize=5pt 0,linewidth=0.6pt,arrowsize=3pt 2,arrowinset=0.25}
\begin{pspicture*}(-2,-35.464763403394198)(11.16,225.20400134876877)
\multips(0,0)(0,20.0){13}{\psline[linestyle=dashed,linecap=1,dash=1.5pt 1.5pt,linewidth=0.4pt,linecolor=gray]{c-c}(0,0)(11.16,0)}
\multips(0,0)(1.0,0){12}{\psline[linestyle=dashed,linecap=1,dash=1.5pt 1.5pt,linewidth=0.4pt,linecolor=gray]{c-c}(0,0)(0,225.20400134876877)}
\begin{scriptsize}
\psaxes[xAxis=true,yAxis=true,Dx=1.,Dy=40.,ticksize=-2pt 0,subticks=0]{->}(0,0)(0.,0.)(11.16,225.20400134876877)
\rput[tl](0.28,213.26739350342768){$M(t)$}
\rput[tl](7.36,18){$t$ in Tagen}
\psplot[linewidth=0.6pt,plotpoints=200]{0}{11.16}{199.99999999999974*EXP(-0.05944582398978859*x)}
\end{scriptsize}
\end{pspicture*}},				% Response 4
				%% Moegliche Zuordnungen: %%
				A={1 Tag}, 				%Moeglichkeit A  
				B={2 Tage}, 				%Moeglichkeit B  
				C={3 Tage}, 				%Moeglichkeit C  
				D={5 Tage}, 				%Moeglichkeit D  
				E={10 Tage}, 				%Moeglichkeit E  
				F={mehr als 10 Tage}, 				%Moeglichkeit F  
				%% LOESUNG: %%
				A1={D},				% 1. richtige Zuordnung
				A2={E},				% 2. richtige Zuordnung
				A3={C},				% 3. richtige Zuordnung
				A4={F},				% 4. richtige Zuordnung
				}
\end{beispiel}