\section{FA 5.5 - 11 - MAT - Verzinsung - OA - Matura 2018/19 2. NT}

\begin{beispiel}[FA 5.5]{1}
Ein Kapital $K_0$ wird auf einem Sparbuch mit $1\,\%$ p.a. (pro Jahr) verzinst.

Für die nachstehende Aufgabenstellung gilt die Annahme, dass allfällige Steuern oder Gebühren nicht gesondert berücksichtigt werden müssen und dass keine weiteren Einzahlungen oder Auszahlungen erfolgen.

Berechne, in wie vielen Jahren sich das Kapital $K_0$ bei gleichbleibendem Zinssatz verdoppelt.

\antwort{mögliche Vorgehensweise:

$2\cdot K_0=K_0\cdot 1,01^n$

$2=1,01^n$

$\ln(2)=\ln(1,01)\cdot n$

$n=\dfrac{\ln(2)}{\ln(1,01)}=69,66\ldots\approx 69,7$

Das Kapital $K_0$ verdoppelt sich nach ca. 69,7 Jahren.\leer

Lösungsschlüssel:

Ein Punkt für die richtige Lösung, wobei die Einheit "`Jahr"' nicht angeführt sein muss.
Toleranzintervall: [69 Jahre; 70 Jahre]}
\end{beispiel}