\section{FA 2.1 - 2 Graph einer linearen Funktion zeichnen - OA - BIFIE}

\begin{beispiel}[FA 2.1]{1} %PUNKTE DES BEISPIELS
Zeichne in das nachstehende Koordinatensystem den Graphen einer linearen Funktion mit der Gleichung $f(x)=k\cdot x+d$ ein, f�r deren Parameter $k$ und $d$ die Bedingungen $k=-\frac{2}{3}$ und $d>0$ gelten!

\begin{center}
\psset{xunit=1.0cm,yunit=1.0cm,algebraic=true,dimen=middle,dotstyle=o,dotsize=5pt 0,linewidth=0.8pt,arrowsize=3pt 2,arrowinset=0.25}
\begin{pspicture*}(-5.650112240961741,-4.823292506939291)(4.943774160227248,4.707408162230707)
\multips(0,-4)(0,1.0){10}{\psline[linestyle=dashed,linecap=1,dash=1.5pt 1.5pt,linewidth=0.4pt,linecolor=lightgray]{c-c}(-5.650112240961741,0)(4.943774160227248,0)}
\multips(-5,0)(1.0,0){11}{\psline[linestyle=dashed,linecap=1,dash=1.5pt 1.5pt,linewidth=0.4pt,linecolor=lightgray]{c-c}(0,-4.823292506939291)(0,4.707408162230707)}
\psaxes[labelFontSize=\scriptstyle,xAxis=true,yAxis=true,Dx=1.,Dy=1.,ticksize=-2pt 0,subticks=2]{->}(0,0)(-5.650112240961741,-4.823292506939291)(4.943774160227248,4.707408162230707)[x,140] [y,-40]
\antwort{\psplot{-5.650112240961741}{4.943774160227248}{(-9.--2.*x)/-3.}}
\end{pspicture*}
\end{center}
\leer

\antwort{Alle Geraden, die zu der in der L�sung gezeigten Geraden parallel sind und die positive y-Achse schneiden, sind als richtig zu werten.}
\end{beispiel}