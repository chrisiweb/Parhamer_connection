\section{FA 2.1 - 9 - K5 - Marathonläufer - OA - MatKon}

\begin{beispiel}[FA 2.1]{1}
Ein Marathonläufer ist zu Beginn seines Laufs 42,195\,km vom Ziel entfernt. Er läuft mit einer konstanten Geschwindigkeit von 12\,km/h. Sei $D(t)$ die Distanz des Läufers zum Ziel nach $t$ Stunden. Gib die Funktionsgleichung von $D(t)$ an!

\antwort{$D(t)=-12\cdot t+42,195$}
\end{beispiel}