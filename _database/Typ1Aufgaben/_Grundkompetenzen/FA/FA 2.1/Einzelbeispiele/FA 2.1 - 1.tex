\section{FA 2.1 - 1 - Umrechnungsformel für Fahrenheit - OA - BIFIE}

\begin{beispiel}[FA 2.1]{1} %PUNKTE DES BEISPIELS
Temperaturen werden bei uns in $^\circ C$ (Celsius) gemessen; in einigen anderen Ländern ist die Messung in $^\circ F$ (Fahrenheit) üblich.

Eine Zunahme um $1\,^\circ C$ bedeutet eine Zunahme um $\frac{9}{5}\,^\circ F$.
Eine Temperatur von $50\,^\circ C$ entspricht einer Temperatur von $122\,^\circ F$.

Die Funktion $f$ soll der Temperatur in $^\circ C$ die Temperatur in $^\circ F$ zuordnen.

Bestimme den entsprechenden Funktionsterm, wenn $x$ die Temperatur in $^\circ C$ und $f(x)$ die Temperatur in $^\circ F$ sein soll!
\leer

$f(x)=\rule{5cm}{0.3pt}$
\leer

\antwort{$f(x)=\frac{9}{5}\cdot x+32$}
\end{beispiel}