\section{FA 2.1 - 3 - Graph einer linearen Funktion - MC - BIFIE}

\begin{beispiel}[FA 2.1]{1} %PUNKTE DES BEISPIELS
Gegeben sind fünf Abbildungen:
\leer

\psset{xunit=0.7cm,yunit=0.7cm,algebraic=true,dimen=middle,dotstyle=o,dotsize=5pt 0,linewidth=0.8pt,arrowsize=3pt 2,arrowinset=0.25}
\begin{pspicture*}(-2.669395099408567,-2.658950123900686)(2.855373615189992,2.7139349146951495)
\multips(0,-2)(0,1.0){6}{\psline[linestyle=dashed,linecap=1,dash=1.5pt 1.5pt,linewidth=0.4pt,linecolor=gray]{c-c}(-2.669395099408567,0)(2.855373615189992,0)}
\multips(-2,0)(1.0,0){6}{\psline[linestyle=dashed,linecap=1,dash=1.5pt 1.5pt,linewidth=0.4pt,linecolor=gray]{c-c}(0,-2.658950123900686)(0,2.7139349146951495)}
\psaxes[labelFontSize=\scriptstyle,xAxis=true,yAxis=true,Dx=1.,Dy=1.,showorigin=false,ticksize=-2pt 0,subticks=0]{->}(0,0)(-2.669395099408567,-2.658950123900686)(2.855373615189992,2.7139349146951495)
\psplot[linewidth=1.2pt,plotpoints=200]{-2.669395099408567}{2.855373615189992}{-x}
\end{pspicture*}\hspace{1cm}
\psset{xunit=0.7cm,yunit=0.7cm,algebraic=true,dimen=middle,dotstyle=o,dotsize=5pt 0,linewidth=0.8pt,arrowsize=3pt 2,arrowinset=0.25}
\begin{pspicture*}(-2.669395099408567,-2.658950123900686)(2.855373615189992,2.7139349146951495)
\multips(0,-2)(0,1.0){6}{\psline[linestyle=dashed,linecap=1,dash=1.5pt 1.5pt,linewidth=0.4pt,linecolor=gray]{c-c}(-2.669395099408567,0)(2.855373615189992,0)}
\multips(-2,0)(1.0,0){6}{\psline[linestyle=dashed,linecap=1,dash=1.5pt 1.5pt,linewidth=0.4pt,linecolor=gray]{c-c}(0,-2.658950123900686)(0,2.7139349146951495)}
\psaxes[labelFontSize=\scriptstyle,xAxis=true,yAxis=true,Dx=1.,Dy=1.,showorigin=false,ticksize=-2pt 0,subticks=0]{->}(0,0)(-2.669395099408567,-2.658950123900686)(2.855373615189992,2.7139349146951495)
\psline(-1.435340231886552,-2.658950123900686)(-1.435340231886552,2.7139349146951495)
\end{pspicture*}\hspace{1cm}
\psset{xunit=0.7cm,yunit=0.7cm,algebraic=true,dimen=middle,dotstyle=o,dotsize=5pt 0,linewidth=0.8pt,arrowsize=3pt 2,arrowinset=0.25}
\begin{pspicture*}(-2.669395099408567,-2.658950123900686)(2.855373615189992,2.7139349146951495)
\multips(0,-2)(0,1.0){6}{\psline[linestyle=dashed,linecap=1,dash=1.5pt 1.5pt,linewidth=0.4pt,linecolor=gray]{c-c}(-2.669395099408567,0)(2.855373615189992,0)}
\multips(-2,0)(1.0,0){6}{\psline[linestyle=dashed,linecap=1,dash=1.5pt 1.5pt,linewidth=0.4pt,linecolor=gray]{c-c}(0,-2.658950123900686)(0,2.7139349146951495)}
\psaxes[labelFontSize=\scriptstyle,xAxis=true,yAxis=true,Dx=1.,Dy=1.,showorigin=false,ticksize=-2pt 0,subticks=0]{->}(0,0)(-2.669395099408567,-2.658950123900686)(2.855373615189992,2.7139349146951495)
\psplot{-2.669395099408567}{2.855373615189992}{(--0.5-0.*x)/1.}
\end{pspicture*}

\hspace{1,3cm}Abb. 1\hspace{3,8cm}Abb. 2\hspace{3,8cm}Abb. 3\leer

\hspace{3cm}\psset{xunit=0.7cm,yunit=0.7cm,algebraic=true,dimen=middle,dotstyle=o,dotsize=5pt 0,linewidth=0.8pt,arrowsize=3pt 2,arrowinset=0.25}
\begin{pspicture*}(-2.669395099408567,-2.658950123900686)(2.855373615189992,2.7139349146951495)
\multips(0,-2)(0,1.0){6}{\psline[linestyle=dashed,linecap=1,dash=1.5pt 1.5pt,linewidth=0.4pt,linecolor=gray]{c-c}(-2.669395099408567,0)(2.855373615189992,0)}
\multips(-2,0)(1.0,0){6}{\psline[linestyle=dashed,linecap=1,dash=1.5pt 1.5pt,linewidth=0.4pt,linecolor=gray]{c-c}(0,-2.658950123900686)(0,2.7139349146951495)}
\psaxes[labelFontSize=\scriptstyle,xAxis=true,yAxis=true,Dx=1.,Dy=1.,showorigin=false,ticksize=-2pt 0,subticks=0]{->}(0,0)(-2.669395099408567,-2.658950123900686)(2.855373615189992,2.7139349146951495)
\psplot[linewidth=1.2pt,plotpoints=200]{-2.669395099408567}{-1.571}{TAN(x)}
\psplot[linewidth=1.2pt,plotpoints=200]{-1.57}{1.57}{TAN(x)}
\psplot[linewidth=1.2pt,plotpoints=200]{1.571}{3}{TAN(x)}
\end{pspicture*}\hspace{1cm}\psset{xunit=0.7cm,yunit=0.7cm,algebraic=true,dimen=middle,dotstyle=o,dotsize=5pt 0,linewidth=0.8pt,arrowsize=3pt 2,arrowinset=0.25}
\begin{pspicture*}(-2.669395099408567,-2.658950123900686)(2.855373615189992,2.7139349146951495)
\multips(0,-2)(0,1.0){6}{\psline[linestyle=dashed,linecap=1,dash=1.5pt 1.5pt,linewidth=0.4pt,linecolor=gray]{c-c}(-2.669395099408567,0)(2.855373615189992,0)}
\multips(-2,0)(1.0,0){6}{\psline[linestyle=dashed,linecap=1,dash=1.5pt 1.5pt,linewidth=0.4pt,linecolor=gray]{c-c}(0,-2.658950123900686)(0,2.7139349146951495)}
\psaxes[labelFontSize=\scriptstyle,xAxis=true,yAxis=true,Dx=1.,Dy=1.,showorigin=false,ticksize=-2pt 0,subticks=0]{->}(0,0)(-2.669395099408567,-2.658950123900686)(2.855373615189992,2.7139349146951495)
\psplot[linewidth=1.2pt,plotpoints=200]{-2.669395099408567}{2.855373615189992}{3.0/2.0*x+1.0}
\end{pspicture*}

\hspace{4,3cm}Abb. 4\hspace{3,8cm}Abb. 5

Welche Abbildungen stellen einen Graphen von einer linearen Funktion dar? Kreuze die zutreffende(n) Abbildung(en) an!
\multiplechoice[5]{  %Anzahl der Antwortmoeglichkeiten, Standard: 5
				L1={Abb. 1},   %1. Antwortmoeglichkeit 
				L2={Abb. 2},   %2. Antwortmoeglichkeit
				L3={Abb. 3},   %3. Antwortmoeglichkeit
				L4={Abb. 4},   %4. Antwortmoeglichkeit
				L5={Abb. 5},	 %5. Antwortmoeglichkeit
				L6={},	 %6. Antwortmoeglichkeit
				L7={},	 %7. Antwortmoeglichkeit
				L8={},	 %8. Antwortmoeglichkeit
				L9={},	 %9. Antwortmoeglichkeit
				%% LOESUNG: %%
				A1=1,  % 1. Antwort
				A2=3,	 % 2. Antwort
				A3=5,  % 3. Antwort
				A4=0,  % 4. Antwort
				A5=0,  % 5. Antwort
				}
\end{beispiel}