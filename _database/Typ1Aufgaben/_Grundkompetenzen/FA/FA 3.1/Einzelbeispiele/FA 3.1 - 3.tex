\section{FA 3.1 - 3 Funktionsgleichungen zuordnen - ZO - BIFIE}

\begin{beispiel}[FA 3.1]{1} %PUNKTE DES BEISPIELS
Gegeben sind vier Graphen von Potenzfunktionen und sechs Funktionsgleichungen.

Ordne den vier Graphen jeweils die entsprechende Funktionsgleichung (aus A bis F) zu!

\zuordnen{
				title1={Funktionsgleichungen}, 		%Titel Antwortmoeglichkeiten
				A={$f(x)=x^2+1$}, 				%Moeglichkeit A  
				B={$f(x)=x^2-1$}, 				%Moeglichkeit B  
				C={$f(x)=-x^2+1$}, 				%Moeglichkeit C  
				D={$f(x)=x^{-2}+1$}, 				%Moeglichkeit D  
				E={$f(x)=x^{-2}-1$}, 				%Moeglichkeit E  
				F={$f(x)=-x^{-2}$}, 				%Moeglichkeit F  
				title2={Graphen},		%Titel Zuordnung
				R1={\resizebox{0.8\linewidth}{!}{\newrgbcolor{ccqqqq}{0.8 0. 0.}
\psset{xunit=1.0cm,yunit=1.0cm,algebraic=true,dimen=middle,dotstyle=o,dotsize=5pt 0,linewidth=0.8pt,arrowsize=3pt 2,arrowinset=0.25}
\begin{pspicture*}(-3.7992750623912968,-3.776781160825846)(3.7340282377430136,3.7323769364234036)
\multips(0,-3)(0,1.0){8}{\psline[linestyle=dashed,linecap=1,dash=1.5pt 1.5pt,linewidth=0.4pt,linecolor=darkgray]{c-c}(-3.7992750623912968,0)(3.7340282377430136,0)}
\multips(-3,0)(1.0,0){8}{\psline[linestyle=dashed,linecap=1,dash=1.5pt 1.5pt,linewidth=0.4pt,linecolor=darkgray]{c-c}(0,-3.776781160825846)(0,3.7323769364234036)}
\psaxes[labelFontSize=\scriptstyle,xAxis=true,yAxis=true,Dx=1.,Dy=1.,ticksize=-2pt 0,subticks=2]{->}(0,0)(-3.7992750623912968,-3.776781160825846)(3.7340282377430136,3.7323769364234036)[x,140] [f(x),-40]
\psplot[linewidth=1.2pt,linecolor=ccqqqq,plotpoints=200]{-3.7992750623912968}{3.7340282377430136}{x^(2.0)-1.0}
\end{pspicture*}}},				%1. Antwort rechts
				R2={\resizebox{0.8\linewidth}{!}{\newrgbcolor{ccqqqq}{0.8 0. 0.}
\psset{xunit=1.0cm,yunit=1.0cm,algebraic=true,dimen=middle,dotstyle=o,dotsize=5pt 0,linewidth=0.8pt,arrowsize=3pt 2,arrowinset=0.25}
\begin{pspicture*}(-3.7992750623912968,-3.776781160825846)(3.7340282377430136,3.7323769364234036)
\multips(0,-3)(0,1.0){8}{\psline[linestyle=dashed,linecap=1,dash=1.5pt 1.5pt,linewidth=0.4pt,linecolor=darkgray]{c-c}(-3.7992750623912968,0)(3.7340282377430136,0)}
\multips(-3,0)(1.0,0){8}{\psline[linestyle=dashed,linecap=1,dash=1.5pt 1.5pt,linewidth=0.4pt,linecolor=darkgray]{c-c}(0,-3.776781160825846)(0,3.7323769364234036)}
\psaxes[labelFontSize=\scriptstyle,xAxis=true,yAxis=true,Dx=1.,Dy=1.,ticksize=-2pt 0,subticks=2]{->}(0,0)(-3.7992750623912968,-3.776781160825846)(3.7340282377430136,3.7323769364234036)[x,140] [f(x),-40]
\psplot[linewidth=1.2pt,linecolor=ccqqqq,plotpoints=200]{-3.7992750623912968}{3.7340282377430136}{x^(-2.0)+1.0}
\end{pspicture*}}},				%2. Antwort rechts
				R3={\resizebox{0.8\linewidth}{!}{\newrgbcolor{ccqqqq}{0.8 0. 0.}
\psset{xunit=1.0cm,yunit=1.0cm,algebraic=true,dimen=middle,dotstyle=o,dotsize=5pt 0,linewidth=0.8pt,arrowsize=3pt 2,arrowinset=0.25}
\begin{pspicture*}(-3.7992750623912968,-3.776781160825846)(3.7340282377430136,3.7323769364234036)
\multips(0,-3)(0,1.0){8}{\psline[linestyle=dashed,linecap=1,dash=1.5pt 1.5pt,linewidth=0.4pt,linecolor=darkgray]{c-c}(-3.7992750623912968,0)(3.7340282377430136,0)}
\multips(-3,0)(1.0,0){8}{\psline[linestyle=dashed,linecap=1,dash=1.5pt 1.5pt,linewidth=0.4pt,linecolor=darkgray]{c-c}(0,-3.776781160825846)(0,3.7323769364234036)}
\psaxes[labelFontSize=\scriptstyle,xAxis=true,yAxis=true,Dx=1.,Dy=1.,ticksize=-2pt 0,subticks=2]{->}(0,0)(-3.7992750623912968,-3.776781160825846)(3.7340282377430136,3.7323769364234036)[x,140] [f(x),-40]
\psplot[linewidth=1.2pt,linecolor=ccqqqq,plotpoints=200]{-3.7992750623912968}{3.7340282377430136}{-x^(2.0)+1.0}
\end{pspicture*}}},				%3. Antwort rechts
				R4={\resizebox{0.8\linewidth}{!}{\newrgbcolor{ccqqqq}{0.8 0. 0.}
\psset{xunit=1.0cm,yunit=1.0cm,algebraic=true,dimen=middle,dotstyle=o,dotsize=5pt 0,linewidth=0.8pt,arrowsize=3pt 2,arrowinset=0.25}
\begin{pspicture*}(-3.7992750623912968,-3.776781160825846)(3.7340282377430136,3.7323769364234036)
\multips(0,-3)(0,1.0){8}{\psline[linestyle=dashed,linecap=1,dash=1.5pt 1.5pt,linewidth=0.4pt,linecolor=darkgray]{c-c}(-3.7992750623912968,0)(3.7340282377430136,0)}
\multips(-3,0)(1.0,0){8}{\psline[linestyle=dashed,linecap=1,dash=1.5pt 1.5pt,linewidth=0.4pt,linecolor=darkgray]{c-c}(0,-3.776781160825846)(0,3.7323769364234036)}
\psaxes[labelFontSize=\scriptstyle,xAxis=true,yAxis=true,Dx=1.,Dy=1.,ticksize=-2pt 0,subticks=2]{->}(0,0)(-3.7992750623912968,-3.776781160825846)(3.7340282377430136,3.7323769364234036)[x,140] [f(x),-40]
\psplot[linewidth=1.2pt,linecolor=ccqqqq,plotpoints=200]{-3.7992750623912968}{3.7340282377430136}{-x^(-2.0)}
\end{pspicture*}}},				%4. Antwort rechts
				%% LOESUNG: %%
				A1={B},				% 1. richtige Zuordnung
				A2={D},				% 2. richtige Zuordnung
				A3={C},				% 3. richtige Zuordnung
				A4={F},				% 4. richtige Zuordnung
				}
\end{beispiel}