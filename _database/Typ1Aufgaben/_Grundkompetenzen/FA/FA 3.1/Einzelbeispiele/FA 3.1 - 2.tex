\section{FA 3.1 - 2 - Funktionsgraph - OA - BIFIE}

\begin{beispiel}[FA 3.1]{1} %PUNKTE DES BEISPIELS
Gegeben ist die Funktion $g$ mit der Gleichung $g(x)=2-\dfrac{x^2}{8}$.

Zeichne den Graphen der Funktion $g$!
\leer

\begin{center}
\resizebox{0.8\linewidth}{!}{\newrgbcolor{zzttqq}{0.6 0.2 0.}
\psset{xunit=1.0cm,yunit=1.0cm,algebraic=true,dimen=middle,dotstyle=o,dotsize=5pt 0,linewidth=0.8pt,arrowsize=3pt 2,arrowinset=0.25}
\begin{pspicture*}(-5.803326901850101,-5.7325425945145545)(5.834660888741994,5.833009587422386)
\multips(0,-5)(0,1.0){12}{\psline[linestyle=dashed,linecap=1,dash=1.5pt 1.5pt,linewidth=0.4pt,linecolor=darkgray]{c-c}(-5.803326901850101,0)(5.834660888741994,0)}
\multips(-5,0)(1.0,0){12}{\psline[linestyle=dashed,linecap=1,dash=1.5pt 1.5pt,linewidth=0.4pt,linecolor=darkgray]{c-c}(0,-5.7325425945145545)(0,5.833009587422386)}
\psaxes[labelFontSize=\scriptstyle,xAxis=true,yAxis=true,Dx=1.,Dy=1.,ticksize=-2pt 0,subticks=2]{->}(0,0)(-5.803326901850101,-5.7325425945145545)(5.834660888741994,5.833009587422386)[x,140] [f(x),-40]
\antwort{\psplot[linewidth=1.2pt,linecolor=zzttqq,plotpoints=200]{-5.803326901850101}{5.834660888741994}{-0.125*x^(2.0)+2.0}
\rput[tl](2.1645900502150277,2.211229154665515){g}}
\end{pspicture*}}
\end{center}
\leer

\antwort{Lösungsschlüssel:

Die Aufgabe gilt nur dann als richtig gelöst, wenn die Zeichnung als Parabel mit dem korrekten Scheitel und den richtigen Nullstellen erkennbar ist.}
\end{beispiel}