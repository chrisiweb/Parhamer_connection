\section{FA 3.1 - 1 - Funktionsgraphen zuordnen - ZO - BIFIE}

\begin{beispiel}[FA 3.1]{1} %PUNKTE DES BEISPIELS
				Ordne den jeweiligen Funktionsgleichungen die zugehörigen Funktionsgraphen.

\zuordnen{
				title1={Funktionsgraphen}, 		%Titel Antwortmoeglichkeiten
				A={\psset{xunit=0.5cm,yunit=0.5cm,algebraic=true,dimen=middle,dotstyle=o,dotsize=5pt 0,linewidth=0.8pt,arrowsize=3pt 2,arrowinset=0.25}
\begin{pspicture*}(-4.5,-3.2)(4.5,3.2)
\multips(0,-4)(0,1.0){9}{\psline[linestyle=dashed,linecap=1,dash=1.5pt 1.5pt,linewidth=0.4pt,linecolor=gray]{c-c}(-4.664551521417402,0)(4.6761251111553825,0)}
\multips(-4,0)(1.0,0){10}{\psline[linestyle=dashed,linecap=1,dash=1.5pt 1.5pt,linewidth=0.4pt,linecolor=gray]{c-c}(0,-4.313643913332305)(0,4.386529178721191)}
\psaxes[labelFontSize=\scriptstyle,xAxis=true,yAxis=true,Dx=1.,Dy=1.,showorigin=false,ticksize=-2pt 0,subticks=0]{->}(0,0)(-4.5,-3.2)(4.5,3.2)[x,140] [y,-40]
\psplot[linewidth=1.2pt,plotpoints=200]{-4.664551521417402}{4.6761251111553825}{x^(-1.0)}
\end{pspicture*}}, 				%Moeglichkeit A  
				B={\psset{xunit=0.5cm,yunit=0.5cm,algebraic=true,dimen=middle,dotstyle=o,dotsize=5pt 0,linewidth=0.8pt,arrowsize=3pt 2,arrowinset=0.25}
\begin{pspicture*}(-4.5,-3.2)(4.5,3.2)
\multips(0,-4)(0,1.0){9}{\psline[linestyle=dashed,linecap=1,dash=1.5pt 1.5pt,linewidth=0.4pt,linecolor=gray]{c-c}(-4.664551521417402,0)(4.6761251111553825,0)}
\multips(-4,0)(1.0,0){10}{\psline[linestyle=dashed,linecap=1,dash=1.5pt 1.5pt,linewidth=0.4pt,linecolor=gray]{c-c}(0,-4.313643913332305)(0,4.386529178721191)}
\psaxes[labelFontSize=\scriptstyle,xAxis=true,yAxis=true,Dx=1.,Dy=1.,showorigin=false,ticksize=-2pt 0,subticks=0]{->}(0,0)(-4.5,-3.2)(4.5,3.2)[x,140] [y,-40]
\psplot[linewidth=1.2pt,plotpoints=200]{-4.664551521417402}{4.6761251111553825}{2.0*x^(-2.0)}
\end{pspicture*}}, 				%Moeglichkeit B  
				C={\psset{xunit=0.5cm,yunit=0.5cm,algebraic=true,dimen=middle,dotstyle=o,dotsize=5pt 0,linewidth=0.8pt,arrowsize=3pt 2,arrowinset=0.25}
\begin{pspicture*}(-4.5,-3.2)(4.5,3.2)
\multips(0,-4)(0,1.0){9}{\psline[linestyle=dashed,linecap=1,dash=1.5pt 1.5pt,linewidth=0.4pt,linecolor=gray]{c-c}(-4.664551521417402,0)(4.6761251111553825,0)}
\multips(-4,0)(1.0,0){10}{\psline[linestyle=dashed,linecap=1,dash=1.5pt 1.5pt,linewidth=0.4pt,linecolor=gray]{c-c}(0,-4.313643913332305)(0,4.386529178721191)}
\psaxes[labelFontSize=\scriptstyle,xAxis=true,yAxis=true,Dx=1.,Dy=1.,showorigin=false,ticksize=-2pt 0,subticks=0]{->}(0,0)(-4.5,-3.2)(4.5,3.2)[x,140] [y,-40]
\psplot[linewidth=1.2pt,plotpoints=200]{-4.664551521417402}{-2.1}{(x+2.0)^(-1.0)}
\psplot[linewidth=1.2pt,plotpoints=200]{-1.9}{4.6761251111553825}{(x+2.0)^(-1.0)}
\end{pspicture*}}, 				%Moeglichkeit C  
				D={\psset{xunit=0.5cm,yunit=0.5cm,algebraic=true,dimen=middle,dotstyle=o,dotsize=5pt 0,linewidth=0.8pt,arrowsize=3pt 2,arrowinset=0.25}
\begin{pspicture*}(-4.5,-3.2)(4.5,3.2)
\multips(0,-4)(0,1.0){9}{\psline[linestyle=dashed,linecap=1,dash=1.5pt 1.5pt,linewidth=0.4pt,linecolor=gray]{c-c}(-4.664551521417402,0)(4.6761251111553825,0)}
\multips(-4,0)(1.0,0){10}{\psline[linestyle=dashed,linecap=1,dash=1.5pt 1.5pt,linewidth=0.4pt,linecolor=gray]{c-c}(0,-4.313643913332305)(0,4.386529178721191)}
\psaxes[labelFontSize=\scriptstyle,xAxis=true,yAxis=true,Dx=1.,Dy=1.,showorigin=false,ticksize=-2pt 0,subticks=0]{->}(0,0)(-4.5,-3.2)(4.5,3.2)[x,140] [y,-40]
\psplot[linewidth=1.2pt,plotpoints=200]{-4.664551521417402}{4.6761251111553825}{-x^(2.0)+2.0}
\end{pspicture*}}, 				%Moeglichkeit D  
				E={\psset{xunit=0.5cm,yunit=0.5cm,algebraic=true,dimen=middle,dotstyle=o,dotsize=5pt 0,linewidth=0.8pt,arrowsize=3pt 2,arrowinset=0.25}
\begin{pspicture*}(-4.5,-3.2)(4.5,3.2)
\multips(0,-4)(0,1.0){9}{\psline[linestyle=dashed,linecap=1,dash=1.5pt 1.5pt,linewidth=0.4pt,linecolor=gray]{c-c}(-4.664551521417402,0)(4.6761251111553825,0)}
\multips(-4,0)(1.0,0){10}{\psline[linestyle=dashed,linecap=1,dash=1.5pt 1.5pt,linewidth=0.4pt,linecolor=gray]{c-c}(0,-4.313643913332305)(0,4.386529178721191)}
\psaxes[labelFontSize=\scriptstyle,xAxis=true,yAxis=true,Dx=1.,Dy=1.,showorigin=false,ticksize=-2pt 0,subticks=0]{->}(0,0)(-4.5,-3.2)(4.5,3.2)[x,140] [y,-40]
\psplot[linewidth=1.2pt,plotpoints=200]{-4.664551521417402}{4.6761251111553825}{(x-2.0)^(2.0)}
\end{pspicture*}}, 				%Moeglichkeit E  
				F={\psset{xunit=0.5cm,yunit=0.5cm,algebraic=true,dimen=middle,dotstyle=o,dotsize=5pt 0,linewidth=0.8pt,arrowsize=3pt 2,arrowinset=0.25}
\begin{pspicture*}(-4.5,-3.2)(4.5,3.2)
\multips(0,-4)(0,1.0){9}{\psline[linestyle=dashed,linecap=1,dash=1.5pt 1.5pt,linewidth=0.4pt,linecolor=gray]{c-c}(-4.664551521417402,0)(4.6761251111553825,0)}
\multips(-4,0)(1.0,0){10}{\psline[linestyle=dashed,linecap=1,dash=1.5pt 1.5pt,linewidth=0.4pt,linecolor=gray]{c-c}(0,-4.313643913332305)(0,4.386529178721191)}
\psaxes[labelFontSize=\scriptstyle,xAxis=true,yAxis=true,Dx=1.,Dy=1.,showorigin=false,ticksize=-2pt 0,subticks=0]{->}(0,0)(-4.5,-3.2)(4.5,3.2)[x,140] [y,-40]
\psplot[linewidth=1.2pt,plotpoints=200]{-4.664551521417402}{4.6761251111553825}{0.8034864676622124*x^(3.0)-1.4080886953721717E-51*x^(2.0)+1.1965135323377876*x+2.0}
\end{pspicture*}}, 				%Moeglichkeit F  
				title2={Funktionsgleichungen},		%Titel Zuordnung
				R1={$-x^2+2$},				%1. Antwort rechts
				R2={$(x-2)^2$},				%2. Antwort rechts
				R3={$(x+2)^{-1}$},				%3. Antwort rechts
				R4={$2x^{-2}$},				%4. Antwort rechts
				%% LOESUNG: %%
				A1={D},				% 1. richtige Zuordnung
				A2={E},				% 2. richtige Zuordnung
				A3={C},				% 3. richtige Zuordnung
				A4={B},				% 4. richtige Zuordnung
				}
\end{beispiel}