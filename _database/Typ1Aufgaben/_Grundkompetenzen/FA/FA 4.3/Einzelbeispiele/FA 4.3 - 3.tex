\section{FA 4.3 - 3 Negative Funktionswerte - OA - Matura 2016/17 - Haupttermin}

\begin{beispiel}[FA 4.3]{1} %PUNKTE DES BEISPIELS
Gegeben ist die Gleichung einer reellen Funktion $f$ mit $f(x) = x^2-x-6$. Einen Funktionswert $f(x)$ nennt man negativ, wenn $f(x) < 0$ gilt. \leer

Bestimme alle $x \in \mathbb{R}$, deren Funktionswert $f(x)$ negativ ist.

\antwort{Für alle $x\in (-2; 3)$ gilt: $f(x)<0$ \leer

Lösungsschlüssel:

Ein Punkt für die richtige Lösungsmenge. Andere korrekte Schreibweisen der Lösungsmenge
oder eine korrekte verbale oder grafische Beschreibung der Lösungsmenge, aus der klar hervorgeht, dass die Endpunkte $-2$ und $3$ nicht inkludiert sind, sind ebenfalls als richtig zu werten.}			
\end{beispiel}