\section{FA 4.3 - 4 - MAT - Nullstellen von Funktionen - MC - JanRos UNIVIE}

\begin{beispiel}[FA 4.3]{1} %PUNKTE DES BEISPIELS
Gegeben sind Funktionsgleichungen von Polynomfunktionen unterschiedlichen Grades.

Kreuze die beiden Funktionsgleichungen jener  Funktionen an, die genau eine Nullstelle besitzen.\vspace{0,3cm}

\multiplechoice[5]{  %Anzahl der Antwortmoeglichkeiten, Standard: 5
				L1={$f_1(x)=x^2+10^{-8}$},   %1. Antwortmoeglichkeit 
				L2={$f_2(x)=-3x^4$},   %2. Antwortmoeglichkeit
				L3={$f_3(x)=x^3-x$},   %3. Antwortmoeglichkeit
				L4={$f_4(x)=x^2-4x$},   %4. Antwortmoeglichkeit
				L5={$f_5(x)=-2x+7$},	 %5. Antwortmoeglichkeit
				L6={},	 %6. Antwortmoeglichkeit
				L7={},	 %7. Antwortmoeglichkeit
				L8={},	 %8. Antwortmoeglichkeit
				L9={},	 %9. Antwortmoeglichkeit
				%% LOESUNG: %%
				A1=2,  % 1. Antwort
				A2=5,	 % 2. Antwort
				A3=0,  % 3. Antwort
				A4=0,  % 4. Antwort
				A5=0,  % 5. Antwort
				}

\end{beispiel}