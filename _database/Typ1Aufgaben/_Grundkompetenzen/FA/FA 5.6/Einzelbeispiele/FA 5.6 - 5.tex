\section{FA 5.6 - 5 - Wachstumsprozesse - MC - BIFIE}

\begin{beispiel}[FA 5.6]{1} %PUNKTE DES BEISPIELS
Zur Beschreibung von Wachstumsvorgängen aus der Natur bzw. dem Alltag können oft Exponentialfunktionen herangezogen werden. 

Welche der nachstehend angeführten Fallbeispiele werden am besten durch eine Exponentialfunktion modelliert? Kreuze die die beiden zutreffenden Beispiele an.

 \multiplechoice[5]{  %Anzahl der Antwortmoeglichkeiten, Standard: 5
				 L1={Ein Sparbuch hat eine Laufzeit von 6 Monaten. Eine Spareinlage wird mit 1,5\,\% effektiven Zinsen pro Jahr, also 0,125\,\% pro Monat, verzinst. Diese
werden ihm allerdings erst nach dem Ende des Veranlagungszeitraums gutgeschrieben. [Modell für das Kapitalwachstum in diesem halben Jahr]},   %1. Antwortmoeglichkeit 
				 L2={Festverzinsliche Anleihen garantieren einen fixen Ertrag von effektiv 6\,\% pro
Jahr. Allerdings muss der angelegte Betrag 5 Jahre gebunden bleiben. [Modell für das Kapitalwachstum über diese 5 Jahre]},   %2. Antwortmoeglichkeit
				 L3={Haare wachsen pro Tag ca. $\frac{1}{3}$\,mm. [Modell für das Haarwachstum]},   %3. Antwortmoeglichkeit
				 L4={Milchsäurebakterien vermehren sich an heißen Tagen abhängig von der Außentemperatur um 5\,\% pro Stunde. [Modell für die Vermehrung der
Milchsäurebakterien]},   %4. Antwortmoeglichkeit
				 L5={Die Sonneneinstrahlung auf einen Körper wird stärker, je höher die Sonne über den Horizont steigt. [Modell für die Steigerung der Sonneneinstrahlung abhängig vom Winkel des Sonneneinfalls (zur Horizontalen gemessen)]},	 %5. Antwortmoeglichkeit
				 L6={},	 %6. Antwortmoeglichkeit
				 L7={},	 %7. Antwortmoeglichkeit
				 L8={},	 %8. Antwortmoeglichkeit
				 L9={},	 %9. Antwortmoeglichkeit
				 %% LOESUNG: %%
				 A1=2,  % 1. Antwort
				 A2=4,	 % 2. Antwort
				 A3=0,  % 3. Antwort
				 A4=0,  % 4. Antwort
				 A5=0,  % 5. Antwort
				 }
\end{beispiel}