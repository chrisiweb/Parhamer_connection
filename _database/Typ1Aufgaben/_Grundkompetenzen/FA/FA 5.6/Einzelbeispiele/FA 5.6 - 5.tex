\section{FA 5.6 - 5 Wachstumsprozesse - MC - BIFIE}

\begin{beispiel}[FA 5.6]{1} %PUNKTE DES BEISPIELS
Zur Beschreibung von Wachstumsvorg�ngen aus der Natur bzw. dem Alltag k�nnen oft Exponentialfunktionen herangezogen werden. 

\leer

Welche der nachstehend angef�hrten Fallbeispiele werden am besten durch eine Exponentialfunktion modelliert? Kreuze die die beiden zutreffenden Beispiele an.

 \multiplechoice[5]{  %Anzahl der Antwortmoeglichkeiten, Standard: 5
				 L1={Ein Sparbuch hat eine Laufzeit von 6 Monaten. Eine Spareinlage wird mit 1,5\,\% effektiven Zinsen pro Jahr, also 0,125\,\% pro Monat, verzinst. Diese
werden ihm allerdings erst nach dem Ende des Veranlagungszeitraums gutgeschrieben. [Modell f�r das Kapitalwachstum in diesem halben Jahr]},   %1. Antwortmoeglichkeit 
				 L2={Festverzinsliche Anleihen garantieren einen fixen Ertrag von effektiv 6\,\% pro
Jahr. Allerdings muss der angelegte Betrag 5 Jahre gebunden bleiben. [Modell f�r das Kapitalwachstum �ber diese 5 Jahre]},   %2. Antwortmoeglichkeit
				 L3={Haare wachsen pro Tag ca. $\frac{1}{3}$\,mm. [Modell f�r das Haarwachstum]},   %3. Antwortmoeglichkeit
				 L4={Milchs�urebakterien vermehren sich an hei�en Tagen abh�ngig von der Au�entemperatur um 5\,\% pro Stunde. [Modell f�r die Vermehrung der
Milchs�urebakterien]},   %4. Antwortmoeglichkeit
				 L5={Die Sonneneinstrahlung auf einen K�rper wird st�rker, je h�her die Sonne �ber den Horizont steigt. [Modell f�r die Steigerung der Sonneneinstrahlung abh�ngig vom Winkel des Sonneneinfalls (zur Horizontalen gemessen)]},	 %5. Antwortmoeglichkeit
				 L6={},	 %6. Antwortmoeglichkeit
				 L7={},	 %7. Antwortmoeglichkeit
				 L8={},	 %8. Antwortmoeglichkeit
				 L9={},	 %9. Antwortmoeglichkeit
				 %% LOESUNG: %%
				 A1=2,  % 1. Antwort
				 A2=4,	 % 2. Antwort
				 A3=0,  % 3. Antwort
				 A4=0,  % 4. Antwort
				 A5=0,  % 5. Antwort
				 }
\end{beispiel}