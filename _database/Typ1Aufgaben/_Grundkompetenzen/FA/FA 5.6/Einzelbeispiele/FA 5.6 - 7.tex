\section{FA 5.6 - 7 - MAT - Dicke einer Bleiplatte - OA - Matura 2. NT 2017/18}

\begin{beispiel}[FA 5.6]{1}
In der Medizintechnik werden Röntgenstrahlen eingesetzt. Durch den Einbau von Bleiplatten in Schutzwänden sollen Personen vor diesen Strahlen geschützt werden. Man geht davon aus,
dass pro 1\,mm Dicke der Bleiplatte die Strahlungsintensität um 5\,\% abnimmt.\leer

Berechne die notwendige Dicke $x$ (in mm) einer Bleiplatte, wenn die Strahlungsintensität auf 10\,\% der ursprünglichen Strahlungsintensität, mit der die Strahlen auf die Bleiplatte auftreffen, gesenkt werden soll!

\antwort{Mögliche Vorgehensweise:

$0,1=0,95^x \Rightarrow x \approx 44,9\, $ mm

Toleranzintervall: $[40\,\text{mm}; 46\,\text{mm}]$}
\end{beispiel}