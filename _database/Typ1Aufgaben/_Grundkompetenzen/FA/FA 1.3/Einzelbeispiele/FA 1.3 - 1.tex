\section{FA 1.3 - 1 Luftfeuchte - OA - BIFIE}

\begin{beispiel}[FA 1.3]{1} %PUNKTE DES BEISPIELS
Wasserdampf ist dann gesättigt, wenn die maximal aufnehmbare Wassermenge (Sättigungsmenge, absolute Luftfeuchte) erreicht wird. Die nachstehende Tabelle enthält einige beispielhafte Werte zum Wassergehalt in der Luft (in g/$m³$) in Abhängigkeit von der Temperatur (in $^\circ C$) für $[0\,^\circ C;100\,^\circ C]$ (Werte gerundet).
\leer

\begin{longtable}{|l|c|c|c|c|c|c|}
\hline
Temperatur (in $^\circ C$)&0&20&40&60&80&100\\
\hline
Wassergehalt (in g/$m³$)&5&18&50&130&290&590\\
\hline
\end{longtable}
\leer
Stelle den Zusammenhang zwischen der Temperatur und dem Wassergehalt für den angegebenen Temperaturbereich grafisch dar! Skaliere und beschrifte dazu im vorgegebenen Koordinatensystem in geeigneter Weise die senkrechte Achse so, dass alle in der Tabelle angeführten Werte dargestellt werden können!
\leer

\psset{xunit=.1cm,yunit=.01cm,algebraic=true,dimen=middle,dotstyle=o,dotsize=5pt 0,linewidth=0.8pt,arrowsize=3pt 2,arrowinset=0.25}
\begin{pspicture*}(-12.645435936896371,-107.15396213407702)(108.50292598183132,660.442058982985)
\psaxes[labelFontSize=\scriptstyle,xAxis=true,yAxis=true,Dx=10.,Dy=1000.,ticksize=-2pt 0,subticks=2]{->}(0,0)(0.,0.)(108.50292598183132,660.442058982985)
\antwort{\psaxes[labelFontSize=\scriptstyle,xAxis=false,yAxis=true,Dx=10.,Dy=100.,ticksize=-2pt 0,subticks=2]{->}(0,0)(0.,0.)(108.50292598183132,660.442058982985)}
\antwort{\psplot[linewidth=1.2pt,linestyle=dotted,linecolor=red,plotpoints=200]{0}{100}{6.510416666666667E-8*x^(5.0)-1.2239583333333334E-5*x^(4.0)+0.001421875*x^(3.0)-0.035104166666666665*x^(2.0)+0.8708333333333333*x+5.0}}
\rput[tl](-11.288574283406621,450){\rotatebox{90.0}{\scriptsize{Wassergehalt (in g/m$^3$)}}}
\rput[tl](45,-55){\scriptsize{Temperatur (in °C)}}
\antwort{\begin{scriptsize}
\psdots[dotsize=3pt 0,dotstyle=*,linecolor=red](0.,5.)
\psdots[dotsize=3pt 0,dotstyle=*,linecolor=red](20.,18.)
\psdots[dotsize=3pt 0,dotstyle=*,linecolor=red](40.,50.)
\psdots[dotsize=3pt 0,dotstyle=*,linecolor=red](60.,130.)
\psdots[dotsize=3pt 0,dotstyle=*,linecolor=red](80.,290.)
\psdots[dotsize=3pt 0,dotstyle=*,linecolor=red](100.,590.)
\end{scriptsize}}
\end{pspicture*}
\leer

\antwort{Lösungsschlüssel:

Ein Punkt ist genau dann zu geben, wenn eine korrekte Skalierung angegeben ist und alle in der Tabelle angeführten Werte als Punkte richtig eingetragen sind. Die Darstellung des Verlaufes durch die Verbindung der Punkte ist dabei nicht erforderlich.}
\end{beispiel}