\section{FA 1.6 - 6 - MAT - Grafisches Lösen einer quadratischen Gleichung - OA - Matura 1. NT 2017/18}

\begin{beispiel}[FA 1.6]{1}
Gegeben ist die quadratische Gleichung $x^2+x-2=0$.

Man kann die gegebene Gleichung geometrisch mithilfe der Graphen zweier Funktionen $f$ und $g$ lösen, indem man die Gleichung $f(x)=g(x)$ betrachtet.

Die nachstehende Abbildung zeigt den Graphen der quadratischen Funktion $f$, wobei gilt: $f(x)\in\mathbb{Z}$ für jedes $x\in\mathbb{Z}$. Zeichne in dieser Abbildung den Graphen der Funktion $g$ ein!

\begin{center}
\psset{xunit=0.7cm,yunit=0.7cm,algebraic=true,dimen=middle,dotstyle=o,dotsize=5pt 0,linewidth=1.6pt,arrowsize=3pt 2,arrowinset=0.25}
\begin{pspicture*}(-6.62,-6.52)(6.78,6.7)
\multips(0,-6)(0,1.0){14}{\psline[linestyle=dashed,linecap=1,dash=1.5pt 1.5pt,linewidth=0.4pt,linecolor=gray]{c-c}(-6.62,0)(6.78,0)}
\multips(-6,0)(1.0,0){14}{\psline[linestyle=dashed,linecap=1,dash=1.5pt 1.5pt,linewidth=0.4pt,linecolor=gray]{c-c}(0,-6.52)(0,6.7)}
\psaxes[labelFontSize=\scriptstyle,xAxis=true,yAxis=true,Dx=1.,Dy=1.,showorigin=false,ticksize=-2pt 0,subticks=0]{->}(0,0)(-6.62,-6.52)(6.78,6.7)[\scriptsize{$x$},140] [\scriptsize{\text{$f(x), g(x)$} },-40]
\psplot[linewidth=2.pt,plotpoints=200]{-6.62}{6.780000000000005}{x^(2.0)}
\antwort{\psplot[linewidth=2.pt,plotpoints=200]{-6.62}{6.780000000000005}{-x+2.0}}
\begin{scriptsize}
\rput[bl](-2.7,5.14){$f$}
\antwort{\rput[bl](-4.32,5.6){$g$}}
\end{scriptsize}
\end{pspicture*}
\end{center}
\end{beispiel}