\section{FA 1.6 - 5 Schnittpunkt - OA - Matura NT 2 15/16}

\begin{beispiel}[FA 1.6]{1} %PUNKTE DES BEISPIELS
Die Funktion $E$ gibt den Erlös $E(x)$ und die Funktion $K$ die Kosten $K(x)$ jeweils in Euro bezogen auf die Produktionsmenge $x$ an. Die Produktionsmenge $x$ wird in Mengeneinheiten (ME) angegeben. Im folgenden Koordinatensystem sind die Graphen beider Funktion dargestellt:


\begin{center}
\resizebox{0.5\linewidth}{!}{\psset{xunit=1.0cm,yunit=1.0cm,algebraic=true,dimen=middle,dotstyle=o,dotsize=5pt 0,linewidth=0.8pt,arrowsize=3pt 2,arrowinset=0.25}
\begin{pspicture*}(-0.6,-0.44)(7.42,6.)
\psaxes[labelFontSize=\scriptstyle,xAxis=true,yAxis=true,labels=none,Dx=1.,Dy=1.,ticksize=-2pt 0,subticks=2]{->}(0,0)(0.,0.)(7.42,6.)[,140] [,-40]
\psplot[linewidth=1.6pt,plotpoints=200]{-0.5999999999999998}{7.420000000000003}{x}
\psplot[linewidth=1.6pt,plotpoints=200]{-0.5999999999999998}{7.420000000000003}{0.5*x+2.0}
\rput[tl](6.34,4.94){K}
\rput[tl](4.9,5.74){E}
\begin{scriptsize}
\psdots[dotsize=3pt 0,dotstyle=*,linecolor=darkgray](4.,4.)
\rput[bl](3.78,4.14){\darkgray{S}}
\rput[bl](6.5,0.2){$x$ in ME}
\rput[bl](0.2,5.7){$E(x)$, $K(x)$ in \euro}
\end{scriptsize}
\end{pspicture*}}
\end{center}

Interpretiere die beiden Koordinaten des Schnittpunkts $S$ der beiden Funktionsgraphen im gegebenen Zusammenhang!

\antwort{Die erste Koordinate des Schnittpunkts gibt diejenige Produktionsmenge an, bei der kosten-deckend produziert wird (d. 
h., bei der Erlös und Kosten gleich hoch sind), die zweite Koordinate gibt dabei den zugehörigen Erlös bzw. die zugehörigen Kosten an.

oder:

Die erste Koordinate des Schnittpunkts gibt diejenige Produktionsmenge an, bei der weder Gewinn noch Verlust gemacht wird, die zweite Koordinate gibt dabei den zugehörigen Erlös bzw. die zugehörigen Kosten an.}
\end{beispiel}