\section{FA 1.6 - 1 Schnittpunkte - MC - BIFIE}

\begin{beispiel}[FA 1.6]{1} %PUNKTE DES BEISPIELS
In der nachstehenden Abbildung sind die Graphen zweier Funktionen mit den Gleichungen $f_1(x)=\frac{a}{x} ~,a>1$ und $f_2=\frac{a}{x^2} ~,a>1$ dargestellt.

\begin{center}
\newrgbcolor{wwwwww}{0.4 0.4 0.4}
\psset{xunit=1.0cm,yunit=1.0cm,algebraic=true,dimen=middle,dotstyle=o,dotsize=5pt 0,linewidth=0.8pt,arrowsize=3pt 2,arrowinset=0.25}
\begin{pspicture*}(-3.600230956701424,-3.9393641070667043)(4.726618633579011,3.9890716272194457)
\psaxes[labelFontSize=\scriptstyle,xAxis=true,yAxis=true,labels=none,Dx=1.,Dy=1.,ticksize=0pt 0,subticks=0]{->}(0,0)(-3.600230956701424,-3.9393641070667043)(4.726618633579011,3.9890716272194457)
\psplot[linewidth=1.2pt,plotpoints=200]{-3.600230956701424}{-0.0000001}{1.0/x}
\psplot[linewidth=1.2pt,plotpoints=200]{0.0000001}{4.726618633579011}{1.0/x}
\psplot[linewidth=1.2pt,linestyle=dashed,dash=5pt 5pt,linecolor=wwwwww,plotpoints=200]{-3.600230956701424}{4.726618633579011}{1.0/x^(2.0)}
\rput[tl](-0.014506252752911581,2.8735128304354647){$f_1$}
\rput[tl](-1.209747820735749,-1.2500705791053217){$f_1$}
\rput[tl](0.7624007664359328,2.3555748176429026){$f_2$}
\rput[tl](-1.3691133631334607,1.8376368048503398){$f_2$}
\begin{scriptsize}
\psdots[dotsize=3pt 0,dotstyle=*,linecolor=darkgray](1.,1.)
\rput[bl](1.2006560080296398,1.060729785661496){\darkgray{$S$}}
\end{scriptsize}
\end{pspicture*}
\end{center}

Welcher der unten angegebenen Punkte gibt die Koordinaten des Schnittpunktes korrekt an? \\
Kreuze den zutreffenden Punkt an!

\multiplechoice[6]{  %Anzahl der Antwortmoeglichkeiten, Standard: 5
				L1={$S=(1|1)$},   %1. Antwortmoeglichkeit 
				L2={$S=(a|1)$},   %2. Antwortmoeglichkeit
				L3={$S=(1|a)$},   %3. Antwortmoeglichkeit
				L4={$S=(a|a)$},   %4. Antwortmoeglichkeit
				L5={$S=(0|a)$},	 %5. Antwortmoeglichkeit
				L6={$S=\left(1|\frac{1}{a}\right)$},	 %6. Antwortmoeglichkeit
				L7={},	 %7. Antwortmoeglichkeit
				L8={},	 %8. Antwortmoeglichkeit
				L9={},	 %9. Antwortmoeglichkeit
				%% LOESUNG: %%
				A1=3,  % 1. Antwort
				A2=0,	 % 2. Antwort
				A3=0,  % 3. Antwort
				A4=0,  % 4. Antwort
				A5=0,  % 5. Antwort
				}

\end{beispiel}