\section{FA 1.6 - 2 - Kosten und Erlösfunktion - OA - BIFIE}

\begin{beispiel}[FA 1.6]{1} %PUNKTE DES BEISPIELS
Die Herstellungskosten eines Produkts können annähernd durch eine lineare Funktion $K$ mit $K(x) = 392 + 30x$ beschrieben werden.

Beim Verkauf dieses Produkts wird ein Erlös erzielt, der annähernd durch die quadratische Funktion $E$ mit $E(x) = -2x^2 + 100x$ angegeben werden kann.

$x$ gibt die Anzahl der produzierten und verkauften Einheiten des Produkts an.


Ermittle die x-Koordinaten der Schnittpunkte dieser Funktionsgraphen und interpretiere diese im gegebenen Zusammenhang.


\antwort{$x_1=7$, $x_2=28$

Bei der Herstellung und dem Verkauf von 7 (bzw. 28) Stück des Produkts sind die Herstellungskosten genauso hoch wie der Erlös. Das heißt, in diesen Fällen wird kein Gewinn/Verlust erzielt.}
\end{beispiel}