\section{FA 3.2 - 8 - Funktionen zuordnen - ZO - ChrGru}


\begin{beispiel}[FA 3.2]{1} %PUNKTE DES BEISPIELS
		Ordne die 4 Funktionsgraphen den jeweiligen Funktionstermen zu!
\zuordnen{
				R1={
\psset{xunit=0.6cm,yunit=0.55cm,algebraic=true,dimen=middle,dotstyle=o,dotsize=5pt 0,linewidth=0.6pt,arrowsize=3pt 2,arrowinset=0.25}
\begin{pspicture*}(-3.6,-0.84)(3.76,8.66)
\multips(0,0)(0,1.0){10}{\psline[linestyle=dashed,linecap=1,dash=1.5pt 1.5pt,linewidth=0.4pt,linecolor=gray]{c-c}(-3.6,0)(3.76,0)}
\multips(-3,0)(1.0,0){8}{\psline[linestyle=dashed,linecap=1,dash=1.5pt 1.5pt,linewidth=0.4pt,linecolor=gray]{c-c}(0,-0.84)(0,8.66)}
\psaxes[labelFontSize=\scriptstyle,xAxis=true,yAxis=true,Dx=1.,Dy=1.,ticksize=-2pt 0,subticks=0]{->}(0,0)(-3.6,-0.84)(3.76,8.66)
\psplot[linewidth=1.pt,plotpoints=200]{-3.5999999999999996}{3.760000000000003}{x^(2.0)+2.0}
\begin{scriptsize}
\rput[bl](-2.2,7.5){$f$}
\end{scriptsize}
\end{pspicture*}},				% Response 1
				R2={\psset{xunit=0.6cm,yunit=0.55cm,algebraic=true,dimen=middle,dotstyle=o,dotsize=5pt 0,linewidth=0.6pt,arrowsize=3pt 2,arrowinset=0.25}
\begin{pspicture*}(-3.6,-0.84)(3.76,8.66)
\multips(0,0)(0,1.0){10}{\psline[linestyle=dashed,linecap=1,dash=1.5pt 1.5pt,linewidth=0.4pt,linecolor=gray]{c-c}(-3.6,0)(3.76,0)}
\multips(-3,0)(1.0,0){8}{\psline[linestyle=dashed,linecap=1,dash=1.5pt 1.5pt,linewidth=0.4pt,linecolor=gray]{c-c}(0,-0.84)(0,8.66)}
\psaxes[labelFontSize=\scriptstyle,xAxis=true,yAxis=true,Dx=1.,Dy=1.,ticksize=-2pt 0,subticks=0]{->}(0,0)(-3.6,-0.84)(3.76,8.66)
\psplot[linewidth=1.pt,plotpoints=200]{-3.5999999999999996}{3.760000000000003}{3*x^(2.0)+2.0}
\begin{scriptsize}
\rput[bl](-1.9,7.5){$f$}
\end{scriptsize}
\end{pspicture*}},				% Response 2
				R3={\psset{xunit=0.6cm,yunit=0.55cm,algebraic=true,dimen=middle,dotstyle=o,dotsize=5pt 0,linewidth=0.6pt,arrowsize=3pt 2,arrowinset=0.25}
\begin{pspicture*}(-3.6,-4.84)(3.76,4.66)
\multips(0,-4)(0,1.0){10}{\psline[linestyle=dashed,linecap=1,dash=1.5pt 1.5pt,linewidth=0.4pt,linecolor=gray]{c-c}(-3.6,0)(3.76,0)}
\multips(-3,0)(1.0,0){8}{\psline[linestyle=dashed,linecap=1,dash=1.5pt 1.5pt,linewidth=0.4pt,linecolor=gray]{c-c}(0,-6.84)(0,8.66)}
\psaxes[labelFontSize=\scriptstyle,xAxis=true,yAxis=true,Dx=1.,Dy=1.,showorigin=false,ticksize=-2pt 0,subticks=0]{->}(0,0)(-3.6,-6.84)(3.76,8.66)
\psplot[linewidth=1.pt,plotpoints=200]{-3.5999999999999996}{3.760000000000003}{3/x}
\begin{scriptsize}
\rput[bl](-2.38,-2){$f$}
\end{scriptsize}
\end{pspicture*}},				% Response 3
				R4={\psset{xunit=0.6cm,yunit=0.55cm,algebraic=true,dimen=middle,dotstyle=o,dotsize=5pt 0,linewidth=0.6pt,arrowsize=3pt 2,arrowinset=0.25}
\begin{pspicture*}(-3.6,-0.84)(3.76,8.66)
\multips(0,0)(0,1.0){10}{\psline[linestyle=dashed,linecap=1,dash=1.5pt 1.5pt,linewidth=0.4pt,linecolor=gray]{c-c}(-3.6,0)(3.76,0)}
\multips(-3,0)(1.0,0){8}{\psline[linestyle=dashed,linecap=1,dash=1.5pt 1.5pt,linewidth=0.4pt,linecolor=gray]{c-c}(0,-0.84)(0,8.66)}
\psaxes[labelFontSize=\scriptstyle,xAxis=true,yAxis=true,Dx=1.,Dy=1.,showorigin=false,ticksize=-2pt 0,subticks=0]{->}(0,0)(-3.6,-0.84)(3.76,8.66)
\psplot[linewidth=1.pt,plotpoints=200]{-3.5999999999999996}{3.760000000000003}{3/x^2}
\begin{scriptsize}
\rput[bl](-1.3,5.34){$f$}
\end{scriptsize}
\end{pspicture*}},				% Response 4
				%% Moegliche Zuordnungen: %%
				A={$f(x)=\frac{3}{x}$}, 				%Moeglichkeit A  
				B={$f(x)=x^2+2$}, 				%Moeglichkeit B  
				C={$f(x)=3x+2$}, 				%Moeglichkeit C  
				D={$f(x)=3x^2+2$}, 				%Moeglichkeit D  
				E={$f(x)=\frac{3}{x^2}$}, 				%Moeglichkeit E  
				F={$f(x)=-2x^2+2$}, 				%Moeglichkeit F  
				%% LOESUNG: %%
				A1={B},				% 1. richtige Zuordnung
				A2={D},				% 2. richtige Zuordnung
				A3={A},				% 3. richtige Zuordnung
				A4={E},				% 4. richtige Zuordnung
				}
\end{beispiel}