\section{FA 3.2 - 7 Parameter reeller Funktionen - OA - Matura NT 1 16/17}

\begin{beispiel}[FA 3.2]{1} %PUNKTE DES BEISPIELS
Die nachstehende Abbildung zeigt die Graphen zweier reeller Funktionen $f$ und $g$ mit den Funktionsgleichungen $f(x)=a\cdot x^3+b$ und $g(x)=c\cdot x^3+d$ mit $a,b,c,d\in\mathbb{R}$.

\begin{center}
	\resizebox{0.6\linewidth}{!}{\psset{xunit=3cm,yunit=3cm,algebraic=true,dimen=middle,dotstyle=o,dotsize=5pt 0,linewidth=1.6pt,arrowsize=3pt 2,arrowinset=0.25}
\begin{pspicture*}(-1.783129300558507,-2.4069058517783084)(2.4308339141495816,1.2)
\multips(0,-2)(0,0.5){8}{\psline[linestyle=dashed,linecap=1,dash=1.5pt 1.5pt,linewidth=0.4pt,linecolor=lightgray]{c-c}(-1.783129300558507,0)(2.4308339141495816,0)}
\multips(-1.5,0)(0.5,0){9}{\psline[linestyle=dashed,linecap=1,dash=1.5pt 1.5pt,linewidth=0.4pt,linecolor=lightgray]{c-c}(0,-2.4069058517783084)(0,1.180657425608306)}
\psaxes[labelFontSize=\scriptstyle,xAxis=true,yAxis=true,Dx=0.5,Dy=0.5,ticksize=-2pt 0,subticks=2]{->}(0,0)(-1.783129300558507,-2.4069058517783084)(2.4308339141495816,1.180657425608306)[$x$,140] [\text{$f(x)$,$g(x)$},-40]
\psplot[linewidth=2.pt,plotpoints=200]{-1.783129300558507}{2.4308339141495816}{2.0*x^(3.0)-1.0}
\psplot[linewidth=2.pt,plotpoints=200]{-1.783129300558507}{2.4308339141495816}{5.0*x^(3.0)-2.0}
\rput[tl](-0.8378348496915573,-1.6495313821078008){$f$}
\rput[tl](-0.16587855329215936,-2.0766222484633503){$g$}
\end{pspicture*}}
\end{center}

Welche der nachstehenden Aussagen treffen f�r die Parameter $a,b,c$ und $d$ zu?
Kreuze die beiden zutreffenden Aussagen an!

\multiplechoice[5]{  %Anzahl der Antwortmoeglichkeiten, Standard: 5
				L1={$a>0$},   %1. Antwortmoeglichkeit 
				L2={$b>d$},   %2. Antwortmoeglichkeit
				L3={$a>0$},   %3. Antwortmoeglichkeit
				L4={$b>0$},   %4. Antwortmoeglichkeit
				L5={$c<1$},	 %5. Antwortmoeglichkeit
				L6={},	 %6. Antwortmoeglichkeit
				L7={},	 %7. Antwortmoeglichkeit
				L8={},	 %8. Antwortmoeglichkeit
				L9={},	 %9. Antwortmoeglichkeit
				%% LOESUNG: %%
				A1=2,  % 1. Antwort
				A2=3,	 % 2. Antwort
				A3=0,  % 3. Antwort
				A4=0,  % 4. Antwort
				A5=0,  % 5. Antwort
				}
\end{beispiel}