\section{FA 3.2 - 4 Potenzfunktion - MC - Matura 2014/15 - Nebentermin 1}

\begin{beispiel}[FA 3.2]{1} %PUNKTE DES BEISPIELS
In der nachstehenden Abbildung ist der Graph einer Potenzfunktion $f$ vom Typ $f(x)=a \cdot x^z$ mit $a \in \mathbb{R};~ a\neq 0;~ z\in \mathbb{Z}$ dargestellt. 

\begin{center}
\resizebox{0.7\linewidth}{!}{
\psset{xunit=1.0cm,yunit=1.0cm,algebraic=true,dimen=middle,dotstyle=o,dotsize=5pt 0,linewidth=0.8pt,arrowsize=3pt 2,arrowinset=0.25}
\begin{pspicture*}(-3.351892332615817,-3.707209456261729)(3.5841631897196793,2.660316924898724)
\multips(0,-4)(0,0.25){30}{\psline[linestyle=dashed,linecap=1,dash=1.5pt 1.5pt,linewidth=0.4pt,linecolor=lightgray]{c-c}(-3.351892332615817,0)(3.5841631897196793,0)}
\multips(-4,0)(0.25,0){30}{\psline[linestyle=dashed,linecap=1,dash=1.5pt 1.5pt,linewidth=0.4pt,linecolor=lightgray]{c-c}(0,-3.707209456261729)(0,2.660316924898724)}
\psaxes[labelFontSize=\scriptstyle,xAxis=true,yAxis=true,Dx=1.,Dy=1.,ticksize=-2pt 0,subticks=2]{->}(0,0)(-3.351892332615817,-3.707209456261729)(3.5841631897196793,2.660316924898724)[x,140] [f(x),-40]
\psplot[linewidth=1.2pt,plotpoints=200]{-3.351892332615817}{3.5841631897196793}{-x^(-2.0)}
\begin{scriptsize}
\rput[bl](-1.5,-1){$f$}
\end{scriptsize}
\end{pspicture*}}
\end{center}
\leer

Eine der nachstehenden Gleichungen ist eine Gleichung dieser Funktion $f$.\\
Kreuze die zutreffende Gleichung an.

\multiplechoice[6]{  %Anzahl der Antwortmoeglichkeiten, Standard: 5
				L1={$f(x)=2x^{-4}$},   %1. Antwortmoeglichkeit 
				L2={$f(x)=-x^{-2}$},   %2. Antwortmoeglichkeit
				L3={$f(x)=-x^2$},   %3. Antwortmoeglichkeit
				L4={$f(x)=-x^{-1}$},   %4. Antwortmoeglichkeit
				L5={$f(x)=x^{-2}$},	 %5. Antwortmoeglichkeit
				L6={$f(x)=x^{-1}$},	 %6. Antwortmoeglichkeit
				L7={},	 %7. Antwortmoeglichkeit
				L8={},	 %8. Antwortmoeglichkeit
				L9={},	 %9. Antwortmoeglichkeit
				%% LOESUNG: %%
				A1=2,  % 1. Antwort
				A2=0,	 % 2. Antwort
				A3=0,  % 3. Antwort
				A4=0,  % 4. Antwort
				A5=0,  % 5. Antwort
				}

\end{beispiel}