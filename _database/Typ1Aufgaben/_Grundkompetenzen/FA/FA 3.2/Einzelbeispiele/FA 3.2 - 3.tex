\section{FA 3.2 - 3 Potenzfunktionen - ZO - Matura 2015/16 - Haupttermin}

\begin{beispiel}[FA 3.2]{1} %PUNKTE DES BEISPIELS
Gegeben sind die Graphen von vier verschiedenen Potenzfunktionen $f$ mit $f(x) = a \cdot x^z$ sowie sechs Bedingungen f�r den Parameter $a$ und den Exponenten $z$. Dabei ist $a$ eine reelle, $z$ eine nat�rliche Zahl. \leer

Ordne den vier Graphen jeweils die entsprechende Bedingung f�r den Parameter $a$ und den
Exponenten $z$ der Funktionsgleichung (aus A bis F) zu.

\zuordnen[0.15]{
				R1={\resizebox{0.9\linewidth}{!}{\psset{xunit=1.0cm,yunit=1.0cm,algebraic=true,dimen=middle,dotstyle=o,dotsize=5pt 0,linewidth=0.3pt,arrowsize=3pt 2,arrowinset=0.25}
\begin{pspicture*}(-1.4804291276946526,-0.8981054417866586)(1.531400534608247,1.044678879630568)
\psaxes[labelFontSize=\scriptstyle,xAxis=true,yAxis=true,labels=none,Dx=0.5,Dy=0.5,ticksize=0pt 0,subticks=0]{->}(0,0)(-1.4804291276946526,-0.8981054417866586)(1.531400534608247,1.044678879630568)[\tiny $x$,140] [\tiny $f(x)$,-40]
\psplot[linewidth=.5pt,plotpoints=200]{-1.4804291276946526}{1.531400534608247}{-0.5*x^(2.0)}
\rput[tl](0.6,-0.35844313028187363){\tiny $f$}
\end{pspicture*}}},				% Response 1
				R2={\resizebox{0.9\linewidth}{!}{\psset{xunit=1.0cm,yunit=1.0cm,algebraic=true,dimen=middle,dotstyle=o,dotsize=5pt 0,linewidth=0.3pt,arrowsize=3pt 2,arrowinset=0.25}
\begin{pspicture*}(-1.4804291276946526,-0.8981054417866586)(1.531400534608247,1.044678879630568)
\psaxes[labelFontSize=\scriptstyle,xAxis=true,yAxis=true,labels=none,Dx=0.5,Dy=0.5,ticksize=0pt 0,subticks=0]{->}(0,0)(-1.4804291276946526,-0.8981054417866586)(1.531400534608247,1.044678879630568)[\tiny $x$,140] [\tiny $f(x)$,-40]
\psplot[linewidth=.5pt,plotpoints=200]{-1.4804291276946526}{1.531400534608247}{-0.5*x^(3.0)}
\rput[tl](-0.7,0.5){\tiny $f$}
\end{pspicture*}}},				% Response 2
				R3={\resizebox{.9\linewidth}{!}{\psset{xunit=1.0cm,yunit=1.0cm,algebraic=true,dimen=middle,dotstyle=o,dotsize=5pt 0,linewidth=0.3pt,arrowsize=3pt 2,arrowinset=0.25}
\begin{pspicture*}(-1.4804291276946526,-0.8981054417866586)(1.531400534608247,1.044678879630568)
\psaxes[labelFontSize=\scriptstyle,xAxis=true,yAxis=true,labels=none,Dx=0.5,Dy=0.5,ticksize=0pt 0,subticks=0]{->}(0,0)(-1.4804291276946526,-0.8981054417866586)(1.531400534608247,1.044678879630568)[\tiny $x$,140] [\tiny $f(x)$,-40]
\psplot[linewidth=.5pt,plotpoints=200]{-1.4804291276946526}{1.531400534608247}{0.5*x^(2.0)}
\rput[tl](-0.7,0.5){\tiny $f$}
\end{pspicture*}}},				% Response 3
				R4={\resizebox{0.9\linewidth}{!}{\psset{xunit=1.0cm,yunit=1.0cm,algebraic=true,dimen=middle,dotstyle=o,dotsize=5pt 0,linewidth=0.3pt,arrowsize=3pt 2,arrowinset=0.25}
\begin{pspicture*}(-1.4804291276946526,-0.8981054417866586)(1.531400534608247,1.044678879630568)
\psaxes[labelFontSize=\scriptstyle,xAxis=true,yAxis=true,labels=none,Dx=0.5,Dy=0.5,ticksize=0pt 0,subticks=0]{->}(0,0)(-1.4804291276946526,-0.8981054417866586)(1.531400534608247,1.044678879630568)[\tiny $x$,140] [\tiny $f(x)$,-40]
\psplot[linewidth=.5pt,plotpoints=200]{-1.4804291276946526}{1.531400534608247}{0.5*x^(3.0)}
\rput[tl](0.6,-0.35844313028187363){\tiny $f$}
\end{pspicture*}}},				% Response 4
				%% Moegliche Zuordnungen: %%
				A={$a>0,~ z=1$}, 				%Moeglichkeit A  
				B={$a>0,~ z=2$}, 				%Moeglichkeit B  
				C={$a>0,~ z=3$}, 				%Moeglichkeit C  
				D={$a<0,~ z=1$}, 				%Moeglichkeit D  
				E={$a<0,~ z=2$}, 				%Moeglichkeit E  
				F={$a<0,~ z=3$}, 				%Moeglichkeit F  
				%% LOESUNG: %%
				A1={E},				% 1. richtige Zuordnung
				A2={F},				% 2. richtige Zuordnung
				A3={B},				% 3. richtige Zuordnung
				A4={C},				% 4. richtige Zuordnung
				}


\end{beispiel}
