\section{FA 3.2 - 1 Potenzfunktion - OA - BIFIE}

\begin{beispiel}[FA 3.2]{1} %PUNKTE DES BEISPIELS
Von einer Funktion $f$ mit der Gleichung $f(x)=a\cdot x^2+b$ ist der Graph gegeben:
\leer

\newrgbcolor{zzttff}{0.6 0.2 1.}
\psset{xunit=1.0cm,yunit=1.0cm,algebraic=true,dimen=middle,dotstyle=o,dotsize=5pt 0,linewidth=0.8pt,arrowsize=3pt 2,arrowinset=0.25}
\begin{pspicture*}(-6.452211463544192,-2.5700012939853227)(7.705016581270089,5.633252339458454)
\multips(0,-2)(0,1.0){9}{\psline[linestyle=dashed,linecap=1,dash=1.5pt 1.5pt,linewidth=0.4pt,linecolor=darkgray]{c-c}(-6.452211463544192,0)(7.705016581270089,0)}
\multips(-6,0)(1.0,0){15}{\psline[linestyle=dashed,linecap=1,dash=1.5pt 1.5pt,linewidth=0.4pt,linecolor=darkgray]{c-c}(0,-2.5700012939853227)(0,5.633252339458454)}
\psaxes[labelFontSize=\scriptstyle,xAxis=true,yAxis=true,Dx=1.,Dy=1.,ticksize=-2pt 0,subticks=2]{->}(0,0)(-6.452211463544192,-2.5700012939853227)(7.705016581270089,5.633252339458454)[x,140] [f(x),-40]
\psplot[linewidth=1.2pt,linecolor=zzttff,plotpoints=200]{-6.452211463544192}{7.705016581270089}{7.0710678118654755}
\psplot[linewidth=1.2pt,plotpoints=200]{-6.452211463544192}{7.705016581270089}{-0.2*x^(2.0)+5.0}
\rput[tl](-3.2271419907185086,3.466667206431973){f}
\end{pspicture*}
\leer

Ermittle die Werte der Parameter $a$ und $b$!

$a=\rule{5cm}{0.3pt}$

$b=\rule{5cm}{0.3pt}$
\leer

\antwort{$a=-0,2$ und $b=5$}
\end{beispiel}