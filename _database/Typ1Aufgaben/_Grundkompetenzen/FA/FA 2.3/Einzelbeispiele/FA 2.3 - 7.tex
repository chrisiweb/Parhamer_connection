\section{FA 2.3 - 7 - MAT - Funktionsgleichung einer linearen Funktion - OA - Matura 2015/16 - Nebentermin 1}

\begin{beispiel}[FA 2.3]{1} %PUNKTE DES BEISPIELS
Gegeben ist eine lineare Funktion $f$ mit folgenden Eigenschaften:

\begin{itemize}
	\item Wenn das Argument $x$ um 2 zunimmt, dann nimmt der Funktionswert $f(x)$ um 4 ab.
	\item $f(0)=1$
\end{itemize} 


Gib eine Funktionsgleichung dieser linearen Funktion $f$ an. \leer

$f(x)=\,\antwort[\rule{5cm}{0.3pt}]{-2\cdot x +1}$
\end{beispiel}