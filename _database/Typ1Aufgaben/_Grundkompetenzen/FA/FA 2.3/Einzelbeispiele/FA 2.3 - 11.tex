\section{FA 2.3 - 11 - MAT - Graph zeichnen - OA - Matura 2018/19 2. NT}

\begin{beispiel}[FA 2.3]{1}
Von einer linearen Funktion $f$ sind nachstehende Eigenschaften bekannt:\vspace{-0,2cm}
\begin{itemize}
\item Die Steigung von $f$ ist $-0,4$.
\item Der Funktionswert von $f$ an der Stelle 2 ist 1.
\end{itemize}

Zeichne im nachstehenden Koordinatensystem den Graphen von $f$ auf dem Intervall $[-7;7]$ ein.

\begin{center}
\psset{xunit=0.8cm,yunit=0.8cm,algebraic=true,dimen=middle,dotstyle=o,dotsize=5pt 0,linewidth=1.6pt,arrowsize=3pt 2,arrowinset=0.25}
\begin{pspicture*}(-8.6,-8.42)(8.94,8.64)
\multips(0,-8)(0,1.0){18}{\psline[linestyle=dashed,linecap=1,dash=1.5pt 1.5pt,linewidth=0.4pt,linecolor=gray]{c-c}(-8.6,0)(8.94,0)}
\multips(-8,0)(1.0,0){18}{\psline[linestyle=dashed,linecap=1,dash=1.5pt 1.5pt,linewidth=0.4pt,linecolor=gray]{c-c}(0,-8.42)(0,8.64)}
\psaxes[labelFontSize=\scriptstyle,showorigin=false,xAxis=true,yAxis=true,Dx=1.,Dy=1.,ticksize=-2pt 0,subticks=0]{->}(0,0)(-8.6,-8.42)(8.94,8.64)[$x$,140] [$f(x)$,-40]
\antwort{\psplot[linewidth=2.pt]{-7}{7}{(--9.-2.*x)/5.}
\rput[bl](-6.88,4.8){$f$}}
\end{pspicture*}
\end{center}
\end{beispiel}