\section{FA 2.3 - 6 - MAT - Modellierung - MC - Matura 2014/15 - Nebentermin 1}

\begin{beispiel}[FA 2.3]{1} %PUNKTE DES BEISPIELS
Eine lineare Funktion f wird allgemein durch eine Funktionsgleichung $f(x) = k \cdot x + d$ mit den Parametern
$k\in \mathbb{R}$ und $d\in \mathbb{R}$ dargestellt. \leer

Welche der nachstehend angegebenen Aufgabenstellungen kann/können mithilfe einer linearen Funktion modelliert werden? Kreuze die zutreffende(n) Aufgabenstellung(en) an!.

\multiplechoice[5]{  %Anzahl der Antwortmoeglichkeiten, Standard: 5
				L1={Die Gesamtkosten bei der Herstellung einer Keramikglasur setzen sich aus einmaligen Kosten von \euro\,1.000 für die Maschine und \euro\,8 pro erzeugtem Kilogramm Glasur zusammen. \\
Stelle die Gesamtkosten für die Herstellung einer Keramikglasur in Abhängigkeit von den erzeugten Kilogramm Glasur dar.},   %1. Antwortmoeglichkeit 
				L2={Eine Bakterienkultur besteht zu Beginn einer Messung aus 20\,000 Bakterien. Die Anzahl der Bakterien verdreifacht sich alle vier Stunden.\\
Stelle die Anzahl der Bakterien in dieser Kultur in Abhängigkeit von der verstrichenen Zeit (in Stunden) dar.},   %2. Antwortmoeglichkeit
				L3={Die Anziehungskraft zweier Planeten verhält sich indirekt proportional zum Quadrat des Abstandes der beiden Planeten. \\
Stelle die Abhängigkeit der Anziehungskraft zweier Planeten von ihrem Abstand dar.},   %3. Antwortmoeglichkeit
				L4={Ein zinsenloses Wohnbaudarlehen von \euro\,240.000 wird 40 Jahre lang mit gleichbleibenden Jahresraten von \euro\,6.000 zurückgezahlt.\\
Stelle die Restschuld in Abhängigkeit von der Anzahl der vergangenen Jahre dar.},   %4. Antwortmoeglichkeit
				L5={Bleibt in einem Stromkreis die Spannung konstant, so ist die Leistung direkt proportional zur Stromstärke. \\
Stelle die Leistung im Stromkreis in Abhängigkeit von der Stromstärke dar.},	 %5. Antwortmoeglichkeit
				L6={},	 %6. Antwortmoeglichkeit
				L7={},	 %7. Antwortmoeglichkeit
				L8={},	 %8. Antwortmoeglichkeit
				L9={},	 %9. Antwortmoeglichkeit
				%% LOESUNG: %%
				A1=1,  % 1. Antwort
				A2=4,	 % 2. Antwort
				A3=5,  % 3. Antwort
				A4=0,  % 4. Antwort
				A5=0,  % 5. Antwort
				}

\end{beispiel}