\section{FA 2.3 - 10 - MAT - Wert eines Gegenstandes - OA - Matura NT 1 16/17}

\begin{beispiel}[FA 2.3]{1} %PUNKTE DES BEISPIELS
Der Wert eines bestimmten Gegenstandes $t$ Jahre nach der Anschaffung wird mit $W(t)$ angegeben und kann mithilfe der Gleichung $W(t)=-k\cdot t+d$ ($k,d\in\mathbb{R}^+$) berechnet werden ($W(t)$ in Euro).

Gib die Bedeutung der Parameter $k$ und $d$ im Hinblick auf den Wert des Gegenstandes an!

\antwort{$k$ ... jährliche Wertminderung (des Gegenstandes), jährlicher Werteverlust, jährliche Abnahme des Wertes

$d$ ... Wert des Gegenstandes zum Zeitpunkt der Anschaffung}
\end{beispiel}