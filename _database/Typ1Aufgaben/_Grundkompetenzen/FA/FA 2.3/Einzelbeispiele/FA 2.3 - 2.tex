\section{FA 2.3 - 2 - Parameter eine linearen Funktion - OA - BIFIE}

\begin{beispiel}[FA 2.3]{1} %PUNKTE DES BEISPIELS
Der Verlauf einer linearen Funktion $f$ mit der Gleichung $f(x)=k\cdot x+d$ wird durch ihre Parameter $k$ und $d$ mit $k,d\in\mathbb{R}$ bestimmt.

Zeichne den Graphen einer linearen Funktion $f(x)=k\cdot x+d$\,, für deren Parameter $k$ und $d$ die nachfolgenden Bedingungen gelten, in das Koordinatensystem ein!

\begin{center}
	$k=\frac{2}{3}$, $d<0$
\leer

\psset{xunit=1.0cm,yunit=1.0cm,algebraic=true,dimen=middle,dotstyle=o,dotsize=5pt 0,linewidth=0.8pt,arrowsize=3pt 2,arrowinset=0.25}
\begin{pspicture*}(-4.72,-4.3)(4.98,4.6)
\multips(0,-4)(0,1.0){9}{\psline[linestyle=dashed,linecap=1,dash=1.5pt 1.5pt,linewidth=0.4pt,linecolor=gray]{c-c}(-4.72,0)(4.98,0)}
\multips(-4,0)(1.0,0){10}{\psline[linestyle=dashed,linecap=1,dash=1.5pt 1.5pt,linewidth=0.4pt,linecolor=gray]{c-c}(0,-4.3)(0,4.6)}
\psaxes[labelFontSize=\scriptstyle,xAxis=true,yAxis=true,Dx=1.,Dy=1.,showorigin=false,ticksize=-2pt 0,subticks=0]{->}(0,0)(-4.72,-4.3)(4.98,4.6)[$x$,140] [$f(x)$,-40]
\antwort{\psplot[linewidth=1.2pt,plotpoints=200]{-4.720000000000004}{4.9799999999999995}{2.0/3.0*x-1.0}}
\end{pspicture*}
\end{center}

\antwort{Die Lösung gilt nur dann als richtig, wenn ein Graph gezeichnet worden ist, der die Bedingungen für die Parameter $k$ und $d$ erfüllt. D.h. richtig sind alle Graphen, deren Steigung $k=\frac{2}{3}$ und deren $d<0$ ist.}
\end{beispiel}