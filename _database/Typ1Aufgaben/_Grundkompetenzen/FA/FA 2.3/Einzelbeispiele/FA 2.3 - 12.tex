\section{FA 2.3 - 12 - MAT - Zug - LT - Matura 2019/20 1. HT}

\begin{beispiel}[FA 2.3]{1}
Ein Zug bewegt sich bis zum Zeitpunkt $t=0$ mit konstanter Geschwindigkeit vorwärts. Ab dem Zeitpunkt $t=0$ erhöht der Zug seine Geschwindigkeit.

Die Funktion $v$ ordnet dem Zeitpunkt $t$ mit $0\leq t\leq 60$ die Geschwindigkeit $v(t)=a\cdot t+b$ zu ($t$ in s, $v(t)$ in m/s, $a,b\in\mathbb{R}$).

\lueckentext{
				text={Für den Parameter $a$ gilt \gap und für den Parameter $b$ gilt \gap.}, 	%Lueckentext Luecke=\gap
				L1={$a<0$}, 		%1.Moeglichkeit links  
				L2={$a=0$}, 		%2.Moeglichkeit links
				L3={$a>0$}, 		%3.Moeglichkeit links
				R1={$b<0$}, 		%1.Moeglichkeit rechts 
				R2={$b=0$}, 		%2.Moeglichkeit rechts
				R3={$b>0$}, 		%3.Moeglichkeit rechts
				%% LOESUNG: %%
				A1=3,   % Antwort links
				A2=3		% Antwort rechts 
				}
				
				\antwort{\textbf{Lösungsschlüssel:}\\
				Ist nur für eine der beiden Lücken der richtige Satzteil angekreuzt, ist ein halber Punkt zu geben.}
\end{beispiel}