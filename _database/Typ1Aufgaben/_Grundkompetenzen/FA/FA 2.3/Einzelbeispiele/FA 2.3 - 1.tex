\section{FA 2.3 - 1 Aussagen �ber lineare Funktionen - MC - BIFIE}

\begin{beispiel}[FA 2.3]{1} %PUNKTE DES BEISPIELS
Betrachte die lineare Funktion $f(x)=k\cdot x+d$.

Kreuze die beiden zutreffenden Aussagen betreffend lineare Funktionen dieser Form an!
\multiplechoice[5]{  %Anzahl der Antwortmoeglichkeiten, Standard: 5
				L1={Jede lineare Funktion mit $k=0$ schneidet jede Koordinatenachse mindestens einmal.},   %1. Antwortmoeglichkeit 
				L2={Jede lineare Funktion mit $d\neq0$ hat genau eine Nullstelle.},   %2. Antwortmoeglichkeit
				L3={Jede lineare Funktion mit $d=0$ und $k\neq0$ l�sst sich als direktes Verh�ltnis interpretieren.},   %3. Antwortmoeglichkeit
				L4={Der Graph einer linearen Funktion mit $k=0$ ist stets eine Gerade.},   %4. Antwortmoeglichkeit
				L5={Zu jeder Geraden im Koordinatensystem l�sst sich eine lineare Funktion aufstellen.},	 %5. Antwortmoeglichkeit
				L6={},	 %6. Antwortmoeglichkeit
				L7={},	 %7. Antwortmoeglichkeit
				L8={},	 %8. Antwortmoeglichkeit
				L9={},	 %9. Antwortmoeglichkeit
				%% LOESUNG: %%
				A1=3,  % 1. Antwort
				A2=4,	 % 2. Antwort
				A3=0,  % 3. Antwort
				A4=0,  % 4. Antwort
				A5=0,  % 5. Antwort
				}
\end{beispiel}