\section{FA 2.3 - 5 - MAT - Produktionskosten - OA - Matura 2014/15 Haupttermin}

\begin{beispiel}[FA 2.3]{1} %PUNKTE DES BEISPIELS
Ein Betrieb gibt für die Abschätzung der Gesamtkosten $K(x)$ für $x$ produzierte Stück einer Ware folgende Gleichung an: $K(x) = 25x + 12\,000$. \leer

Interpretiere die beiden Zahlenwerte 25 und 12\,000 in diesem Kontext.

\antwort{
25 \ldots

\ldots der Kostenzuwachs für die Produktion eines weiteren Stücks \\
\ldots zusätzliche (variable) Kosten, die pro Stück für die Produktion anfallen \\


12\,000 \ldots 

\ldots Fixkosten \\
\ldots jene Kosten, die unabhängig von der produzierten Stückzahl anfallen}

\end{beispiel}