\section{FA 4.4 - 5 Eigenschaften einer Polynomfunktion - MC - Matura 2014/15 - Nebentermin 1}

\begin{beispiel}[FA 4.4]{1} %PUNKTE DES BEISPIELS
Eine reelle Funktion $f$ mit $f(x)=ax^3+bx^2+cx+d$ (mit $a,b,c,d \in \mathbb{R}$ und $a\neq 0$) heißt Polynomfunktion dritten Grades. \leer

Kreuze die beiden zutreffenden Aussagen an.

\multiplechoice[5]{  %Anzahl der Antwortmoeglichkeiten, Standard: 5
				L1={Jede Polynomfunktion dritten Grades hat immer zwei
Nullstellen.},   %1. Antwortmoeglichkeit 
				L2={Jede Polynomfunktion dritten Grades hat genau eine
Wendestelle.},   %2. Antwortmoeglichkeit
				L3={Jede Polynomfunktion dritten Grades hat mehr Nullstellen als lokale Extremstellen.},   %3. Antwortmoeglichkeit
				L4={Jede Polynomfunktion dritten Grades hat mindestens
eine lokale Maximumstelle.},   %4. Antwortmoeglichkeit
				L5={Jede Polynomfunktion dritten Grades hat höchstens
zwei lokale Extremstellen.},	 %5. Antwortmoeglichkeit
				L6={},	 %6. Antwortmoeglichkeit
				L7={},	 %7. Antwortmoeglichkeit
				L8={},	 %8. Antwortmoeglichkeit
				L9={},	 %9. Antwortmoeglichkeit
				%% LOESUNG: %%
				A1=2,  % 1. Antwort
				A2=5,	 % 2. Antwort
				A3=0,  % 3. Antwort
				A4=0,  % 4. Antwort
				A5=0,  % 5. Antwort
				}

\end{beispiel}