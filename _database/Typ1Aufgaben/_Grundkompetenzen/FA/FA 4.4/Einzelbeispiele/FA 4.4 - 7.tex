\section{FA 4.4 - 7 - MAT - Eigenschaften einer Polynomfunktion - MC - Matura 1. NT 2017/18}

\begin{beispiel}[FA 4.4]{1}
Gegeben ist eine Polynomfunktion $f\!:\mathbb{R}\rightarrow\mathbb{R}$ mit $f(x)=a\cdot x^3+b\cdot x^2+c\cdot x+d$ $(a, b, c, d\in\mathbb{R}; a\neq 0)$.

Nachstehend sind Aussagen über die Funktion $f$ gegeben.\\
Welche dieser Aussagen trifft/treffen für beliebige Werte von $a\neq 0, b, c$ und $d$ auf jeden Fall zu?\\
Kreuze die zutreffende(n) Aussage(n) an!

\multiplechoice[5]{  %Anzahl der Antwortmoeglichkeiten, Standard: 5
				L1={Die Funktion $f$ hat mindestens einen Schnittpunkt mit der $x$-Achse.},   %1. Antwortmoeglichkeit 
				L2={Die Funktion $f$ hat höchstens zwei lokale Extremstellen.},   %2. Antwortmoeglichkeit
				L3={Die Funktion $f$ hat höchstens zwei Punkte mit der $x$-Achse gemeinsam.},   %3. Antwortmoeglichkeit
				L4={Die Funktion $f$ hat genau eine Wendestelle.},   %4. Antwortmoeglichkeit
				L5={Die Funktion $f$ hat mindestens eine lokale Extremstelle.},	 %5. Antwortmoeglichkeit
				L6={},	 %6. Antwortmoeglichkeit
				L7={},	 %7. Antwortmoeglichkeit
				L8={},	 %8. Antwortmoeglichkeit
				L9={},	 %9. Antwortmoeglichkeit
				%% LOESUNG: %%
				A1=1,  % 1. Antwort
				A2=2,	 % 2. Antwort
				A3=4,  % 3. Antwort
				A4=0,  % 4. Antwort
				A5=0,  % 5. Antwort
				}
\end{beispiel}