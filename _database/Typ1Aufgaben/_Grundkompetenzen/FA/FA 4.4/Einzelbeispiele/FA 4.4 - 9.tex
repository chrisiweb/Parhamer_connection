\section{FA 4.4 - 9 - Extremstellen von Polynomfunktionen - LT - AngMac UNIVIE}

\begin{beispiel}[FA 4.4]{1}
Gegeben sind drei Funktionen mit den Parametern $a,b,c,d,e\in\mathbb{R}$, wobei $a\neq0$.

\lueckentext[+0.15]{
				text={Die Funktion \gap hat jedenfalls \gap Extremstelle(n).}, 	%Lueckentext Luecke=\gap
				L1={$f$ mit $f(x)= ax^4+bx^3+cx^2+dx+e$}, 		%1.Moeglichkeit links  
				L2={$g$ mit $g(x)= ax^3+bx^2+cx+d$}, 		%2.Moeglichkeit links
				L3={$h$ mit $h(x)= ax^2+bx+c$}, 		%3.Moeglichkeit links
				R1={genau vier}, 		%1.Moeglichkeit rechts 
				R2={höchstens eine}, 		%2.Moeglichkeit rechts
				R3={mindestens zwei}, 		%3.Moeglichkeit rechts
				%% LOESUNG: %%
				A1=3,   % Antwort links
				A2=2		% Antwort rechts 
				}
\end{beispiel}