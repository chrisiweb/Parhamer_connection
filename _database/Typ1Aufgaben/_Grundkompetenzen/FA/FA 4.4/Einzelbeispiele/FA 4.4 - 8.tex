\section{FA 4.4 - 8 - MAT - Verlauf einer Polynomfunktion vierten Grades - OA - Matura-HT-18/19}

\begin{beispiel}[FA 4.4]{1}
Es gibt Polynomfunktionen vierten Grades, die Genau drei Nullstellen $x_1, x_2$ und $x_3$ mit $x_1,x_2,x_3\in\mathbb{R}$ und $x_1<x_2<x_3$ haben.

Skizziere im nachstehenden Koordinatensystem im Intervall $[-4;4]$ den Verlauf des Graphen einer solchen Funktion $f$ mit allen drei Nullstellen im Intervall $[-3;3]$.

\begin{center}
\psset{xunit=0.8cm,yunit=0.8cm,algebraic=true,dimen=middle,dotstyle=o,dotsize=5pt 0,linewidth=0.6pt,arrowsize=3pt 2,arrowinset=0.25}
\begin{pspicture*}(-4.78,-3.14)(5.04,4.32)
\psaxes[labelFontSize=\scriptstyle,xAxis=true,yAxis=true,labels=none,Dx=1.,Dy=1.,ticksize=0pt 0,subticks=2]{->}(0,0)(-4.78,-3.14)(5.04,4.32)[$x$,140] [$f(x)$,-40]
\rput[tl](-3.12,-0.18){-3}
\rput[tl](-4.12,-0.16){-4}
\rput[tl](2.92,-0.18){3}
\rput[tl](3.94,-0.2){4}
\antwort{\psplot[linewidth=1.pt,plotpoints=200]{-4.780000000000002}{5.040000000000002}{-0.15697731858873898*x^(4.0)-0.06048385034975184*x^(3.0)+1.0134831347706148*x^(2.0)+0.01308312223032506*x}}
\begin{scriptsize}
\psdots[dotstyle=+](-3.,0.)
\psdots[dotstyle=+](-4.,0.)
\psdots[dotstyle=+](3.,0.)
\psdots[dotstyle=+](4.,0.)
\rput[bl](-2.86,-1.96){$f$}
\end{scriptsize}
\end{pspicture*}
\end{center}
\end{beispiel}