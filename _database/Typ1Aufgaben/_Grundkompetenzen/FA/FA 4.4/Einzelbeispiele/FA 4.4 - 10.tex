\section{FA 4.4 - 10 - K7 - Polynomfunktion - MC - CleTur UNIVIE}

\begin{beispiel}[FA 4.4]{1}%PUNKTE DES BEISPIELS
Eine reelle Funktion $f$ mit $f(x)=ax^4+bx^3+cx^2+dx+e$ (mit $a,b,c,d,e \in \mathbb{R}$ und $a\neq 0$) heißt Polynomfunktion vierten Grades. \leer

Kreuze die beiden zutreffenden Aussagen an!

\multiplechoice[5]{  %Anzahl der Antwortmoeglichkeiten, Standard: 5
				L1={Jede Polynomfunktion vierten Grades hat mindestens
eine lokale Extremstelle.},   %1. Antwortmoeglichkeit 
				L2={Jede Polynomfunktion vierten Grades hat vier verschiedene
Nullstellen.},   %2. Antwortmoeglichkeit
				L3={Jede Polynomfunktion vierten Grades hat mehr Nullstellen als Wendestellen.},   %3. Antwortmoeglichkeit
				L4={Jede Polynomfunktion vierten Grades hat mindestens eine
Wendestelle.},   %4. Antwortmoeglichkeit
				L5={Jede Polynomfunktion vierten Grades hat höchstens
drei lokale Extremstellen.},	 %5. Antwortmoeglichkeit
				L6={},	 %6. Antwortmoeglichkeit
				L7={},	 %7. Antwortmoeglichkeit
				L8={},	 %8. Antwortmoeglichkeit
				L9={},	 %9. Antwortmoeglichkeit
				%% LOESUNG: %%
				A1=1,  % 1. Antwort
				A2=5,	 % 2. Antwort
				A3=0,  % 3. Antwort
				A4=0,  % 4. Antwort
				A5=0,  % 5. Antwort
				}
\end{beispiel}