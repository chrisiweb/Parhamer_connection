\section{FA 4.4 - 4 Polynomfunktion 3. Grades - MC - BIFIE - Kompetenzcheck 2016}

\begin{beispiel}[FA 4.4]{1} %PUNKTE DES BEISPIELS
				Eine Polynomfunktion 3. Grades hat allgemein die Form $f(x)=ax�+bx�+cx+d$ mit $a,b,c,d\in\mathbb{R}$ und $a\neq 0$.

Welche der folgenden Eigenschaften treffen f�r die Polynomfunktion 3. Grades zu? Kreuze die beiden zutreffenden Antworten an.

\multiplechoice[5]{  %Anzahl der Antwortmoeglichkeiten, Standard: 5
				L1={Es gibt Polynomfunktionen 3. Grades, die keine lokale Extremstelle haben.},   %1. Antwortmoeglichkeit 
				L2={Es gibt Polynomfunktionen 3. Grades, die keine Nullstelle haben.},   %2. Antwortmoeglichkeit
				L3={Es gibt Polynomfuntkionen 3. Grades, die mehr als eine Wendestelle haben.},   %3. Antwortmoeglichkeit
				L4={Es gibt Polynomfunktionen 3. Grades, die keine Wendestelle haben.},   %4. Antwortmoeglichkeit
				L5={Es gibt Polynomfunktionen 3. Grades, die genau zwei verschiedene reelle Nullstellen haben.},	 %5. Antwortmoeglichkeit
				L6={},	 %6. Antwortmoeglichkeit
				L7={},	 %7. Antwortmoeglichkeit
				L8={},	 %8. Antwortmoeglichkeit
				L9={},	 %9. Antwortmoeglichkeit
				%% LOESUNG: %%
				A1=1,  % 1. Antwort
				A2=5,	 % 2. Antwort
				A3=0,  % 3. Antwort
				A4=0,  % 4. Antwort
				A5=0,  % 5. Antwort
				}
\end{beispiel}	