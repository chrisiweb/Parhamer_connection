\section{FA 4.4 - 1 Nullstellen einer Polynomfunktion - OA - BIFIE}

\begin{beispiel}[FA 4.4]{1} %PUNKTE DES BEISPIELS
				Wie viele verschiedene reelle Nullstellen kann eine Polynomfuntkion 3. Grades haben?

Veranschauliche deine Lösungsfälle durch jeweils einen möglichen Graphen!

\antwort{Eine Nullstelle:

\resizebox{0.3\linewidth}{!}{\newrgbcolor{qqwuqq}{0. 0 0.}
\psset{xunit=1.0cm,yunit=1.0cm,algebraic=true,dimen=middle,dotstyle=o,dotsize=5pt 0,linewidth=0.8pt,arrowsize=3pt 2,arrowinset=0.25}
\begin{pspicture*}(0.2539645667233967,-2.842896858315764)(5.687822009072037,2.806671370969217)
\psaxes[labelFontSize=\scriptstyle,xAxis=true,yAxis=true,Dx=1.,Dy=1.,ticksize=-2pt 0,subticks=2]{->}(0,0)(0.2539645667233967,-2.842896858315764)(5.687822009072037,2.806671370969217)[x,140] [,-40]
\psplot[linewidth=1.2pt,linecolor=qqwuqq,plotpoints=200]{0.2539645667233967}{5.687822009072037}{2.1464724255473993*x^(3.0)-18.038919590612885*x^(2.0)+48.41162186766384*x-39.839344777255334}
\end{pspicture*}}
\resizebox{0.3\linewidth}{!}{\psset{xunit=1.0cm,yunit=1.0cm,algebraic=true,dimen=middle,dotstyle=o,dotsize=5pt 0,linewidth=0.8pt,arrowsize=3pt 2,arrowinset=0.25}
\begin{pspicture*}(0.17178902884288796,-2.925072396196273)(5.605646471191529,2.724495833088708)
\psaxes[labelFontSize=\scriptstyle,xAxis=true,yAxis=true,Dx=1.,Dy=1.,ticksize=-2pt 0,subticks=2]{->}(0,0)(0.17178902884288796,-2.925072396196273)(5.605646471191529,2.724495833088708)[x,140] [,-40]
\psplot[linewidth=1.2pt,plotpoints=200]{0.17178902884288796}{5.605646471191529}{0.6151237119828777*x^(3.0)-5.594901570088693*x^(2.0)+16.94201224132092*x-17.080262816702223}
\end{pspicture*}}
\resizebox{0.3\linewidth}{!}{\psset{xunit=1.0cm,yunit=1.0cm,algebraic=true,dimen=middle,dotstyle=o,dotsize=5pt 0,linewidth=0.8pt,arrowsize=3pt 2,arrowinset=0.25}
\begin{pspicture*}(0.171789028842888,-2.925072396196273)(5.759725604717483,2.724495833088708)
\psaxes[labelFontSize=\scriptstyle,xAxis=true,yAxis=true,Dx=1.,Dy=1.,ticksize=-2pt 0,subticks=2]{->}(0,0)(0.171789028842888,-2.925072396196273)(5.759725604717483,2.724495833088708)[x,140] [,-40]
\psplot[linewidth=1.2pt,plotpoints=200]{0.171789028842888}{5.759725604717483}{(x-3.0)^(3.0)+1.0}
\end{pspicture*}}

Zwei Nullstellen:

\resizebox{0.3\linewidth}{!}{\psset{xunit=1.0cm,yunit=1.0cm,algebraic=true,dimen=middle,dotstyle=o,dotsize=5pt 0,linewidth=0.8pt,arrowsize=3pt 2,arrowinset=0.25}
\begin{pspicture*}(0.3111651988974048,-3.40438518128004)(6.366946538919905,2.718187864698604)
\psaxes[labelFontSize=\scriptstyle,xAxis=true,yAxis=true,Dx=1.,Dy=1.,ticksize=-2pt 0,subticks=2]{->}(0,0)(0.3111651988974048,-3.40438518128004)(6.366946538919905,2.718187864698604)[x,140] [,-40]
\psplot[linewidth=1.2pt,plotpoints=200]{0.3111651988974048}{6.366946538919905}{x^(3.0)-10.45*x^(2.0)+34.77*x-35.77}

\end{pspicture*}}
\resizebox{0.3\linewidth}{!}{\psset{xunit=1.0cm,yunit=1.0cm,algebraic=true,dimen=middle,dotstyle=o,dotsize=5pt 0,linewidth=0.8pt,arrowsize=3pt 2,arrowinset=0.25}
\begin{pspicture*}(0.3111651988974048,-3.40438518128004)(6.366946538919905,2.718187864698604)
\psaxes[labelFontSize=\scriptstyle,xAxis=true,yAxis=true,Dx=1.,Dy=1.,ticksize=-2pt 0,subticks=2]{->}(0,0)(0.3111651988974048,-3.40438518128004)(6.366946538919905,2.718187864698604)[x,140] [,-40]
\psplot[linewidth=1.2pt,plotpoints=200]{0.3111651988974048}{6.366946538919905}{x^(3.0)-10.45*x^(2.0)+34.77*x-37.38}

\end{pspicture*}}

Drei Nullstellen:

\resizebox{0.3\linewidth}{!}{\psset{xunit=1.0cm,yunit=1.0cm,algebraic=true,dimen=middle,dotstyle=o,dotsize=5pt 0,linewidth=0.8pt,arrowsize=3pt 2,arrowinset=0.25}
\begin{pspicture*}(0.3111651988974048,-3.40438518128004)(6.366946538919905,2.718187864698604)
\psaxes[labelFontSize=\scriptstyle,xAxis=true,yAxis=true,Dx=1.,Dy=1.,ticksize=-2pt 0,subticks=2]{->}(0,0)(0.3111651988974048,-3.40438518128004)(6.366946538919905,2.718187864698604)[x,140] [,-40]
\psplot[linewidth=1.2pt,plotpoints=200]{0.3111651988974048}{6.366946538919905}{x^(3.0)-10.45*x^(2.0)+34.77*x-36.38}
\begin{scriptsize}
\rput[bl](1.702659072983457,-3.3487254263165975){$g$}
\end{scriptsize}
\end{pspicture*}}
\resizebox{0.3\linewidth}{!}{\psset{xunit=1.0cm,yunit=1.0cm,algebraic=true,dimen=middle,dotstyle=o,dotsize=5pt 0,linewidth=0.8pt,arrowsize=3pt 2,arrowinset=0.25}
\begin{pspicture*}(0.3111651988974048,-3.40438518128004)(6.366946538919905,2.718187864698604)
\psaxes[labelFontSize=\scriptstyle,xAxis=true,yAxis=true,Dx=1.,Dy=1.,ticksize=-2pt 0,subticks=2]{->}(0,0)(0.3111651988974048,-3.40438518128004)(6.366946538919905,2.718187864698604)[x,140] [,-40]
\psplot[linewidth=1.2pt,plotpoints=200]{0.3111651988974048}{6.366946538919905}{-(x^(3.0)-10.45*x^(2.0)+34.77*x-36.38)}
\begin{scriptsize}
\rput[bl](1.7917146809249642,2.5400766488155893){$g$}
\end{scriptsize}
\end{pspicture*}}}
\end{beispiel}