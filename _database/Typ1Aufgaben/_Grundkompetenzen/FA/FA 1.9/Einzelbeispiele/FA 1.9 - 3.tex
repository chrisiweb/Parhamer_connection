\section{FA 1.9 - 3 - Funktionstypen - LT - BIFIE}

\begin{beispiel}[FA 1.9]{1} %PUNKTE DES BEISPIELS
				Gegeben ist die Funktion $g$ mit der Funktionsgleichung $g(x)=a^x$ mit $a\in\mathbb{R^+}$.

\lueckentext{
				text={$g$ ist eine \gap und es gilt: \gap.}, 	%Lueckentext Luecke=\gap
				L1={lineare Funktion}, 		%1.Moeglichkeit links  
				L2={quadratische Funktion}, 		%2.Moeglichkeit links
				L3={Exponentialfunktion}, 		%3.Moeglichkeit links
				R1={$g(x+2)=g(x)\cdot 2a$}, 		%1.Moeglichkeit rechts 
				R2={$g(x+2)=g(x)\cdot a^2$}, 		%2.Moeglichkeit rechts
				R3={$g(x+2)=g(x)+2a$}, 		%3.Moeglichkeit rechts
				%% LOESUNG: %%
				A1=3,   % Antwort links
				A2=2		% Antwort rechts 
				}
\end{beispiel}