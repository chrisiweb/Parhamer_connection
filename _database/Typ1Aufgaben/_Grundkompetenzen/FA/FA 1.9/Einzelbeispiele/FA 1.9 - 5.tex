\section{FA 1.9 - 5 - MAT - Funktionstypen - ZO - Matura NT 1 16/17}

\begin{beispiel}[FA 1.9]{1} %PUNKTE DES BEISPIELS
Im Folgenden sind vier Funktionsgleichungen (mit $a,b\in\mathbb{R}^+$ angeführt und die Graphen von sechs reellen Funktionen dargestellt.

Ordne den vier Funktionsgleichungen jeweils den passenden Graphen (aus A bis F) zu!

\zuordnen{
				R1={$f(x)=a\cdot\sin(b\cdot x)$},				% Response 1
				R2={$f(x)=a\cdot b^x$},				% Response 2
				R3={$f(x)=a\cdot\sqrt{x}+b$},				% Response 3
				R4={$f(x)=a\cdot x+b$},				% Response 4
				%% Moegliche Zuordnungen: %%
				A={\resizebox{0.9\linewidth}{!}{\psset{xunit=1.0cm,yunit=0.7cm,algebraic=true,dimen=middle,dotstyle=o,dotsize=5pt 0,linewidth=1.6pt,arrowsize=3pt 2,arrowinset=0.25}
\begin{pspicture*}(-1.82,-1.68)(4.68,3.76)
\multips(0,-1)(0,1.0){6}{\psline[linestyle=dashed,linecap=1,dash=1.5pt 1.5pt,linewidth=0.4pt,linecolor=darkgray]{c-c}(-1.82,0)(4.68,0)}
\multips(-1,0)(1.0,0){7}{\psline[linestyle=dashed,linecap=1,dash=1.5pt 1.5pt,linewidth=0.4pt,linecolor=darkgray]{c-c}(0,-1.68)(0,3.76)}
\psaxes[labelFontSize=\scriptstyle,xAxis=true,yAxis=true,Dx=1.,Dy=1.,ticksize=-2pt 0,subticks=2]{->}(0,0)(-1.82,-1.68)(4.68,3.76)[x,140] [f(x),-40]
\psplot[linewidth=2.pt,plotpoints=200]{-1.8199999999999994}{4.679999999999998}{0.49999999999999994*EXP(0.39422868018213514*x)}
\rput[tl](2.36,1.74){f}
\end{pspicture*}}}, 				%Moeglichkeit A  
				B={\resizebox{0.9\linewidth}{!}{\psset{xunit=1.0cm,yunit=0.7cm,algebraic=true,dimen=middle,dotstyle=o,dotsize=5pt 0,linewidth=1.6pt,arrowsize=3pt 2,arrowinset=0.25}
\begin{pspicture*}(-1.82,-1.68)(4.68,3.76)
\multips(0,-1)(0,1.0){6}{\psline[linestyle=dashed,linecap=1,dash=1.5pt 1.5pt,linewidth=0.4pt,linecolor=darkgray]{c-c}(-1.82,0)(4.68,0)}
\multips(-1,0)(1.0,0){7}{\psline[linestyle=dashed,linecap=1,dash=1.5pt 1.5pt,linewidth=0.4pt,linecolor=darkgray]{c-c}(0,-1.68)(0,3.76)}
\psaxes[labelFontSize=\scriptstyle,xAxis=true,yAxis=true,Dx=1.,Dy=1.,ticksize=-2pt 0,subticks=2]{->}(0,0)(-1.82,-1.68)(4.68,3.76)[x,140] [f(x),-40]
\rput[tl](1.66,1.56){f}
\psplot[linewidth=2.pt]{-1.82}{4.68}{(--2.--1.*x)/2.}
\end{pspicture*}}}, 				%Moeglichkeit B  
				C={\resizebox{0.9\linewidth}{!}{\psset{xunit=1.0cm,yunit=0.7cm,algebraic=true,dimen=middle,dotstyle=o,dotsize=5pt 0,linewidth=1.6pt,arrowsize=3pt 2,arrowinset=0.25}
\begin{pspicture*}(-1.82,-1.68)(4.68,3.76)
\multips(0,-1)(0,1.0){6}{\psline[linestyle=dashed,linecap=1,dash=1.5pt 1.5pt,linewidth=0.4pt,linecolor=darkgray]{c-c}(-1.82,0)(4.68,0)}
\multips(-1,0)(1.0,0){7}{\psline[linestyle=dashed,linecap=1,dash=1.5pt 1.5pt,linewidth=0.4pt,linecolor=darkgray]{c-c}(0,-1.68)(0,3.76)}
\psaxes[labelFontSize=\scriptstyle,xAxis=true,yAxis=true,Dx=1.,Dy=1.,ticksize=-2pt 0,subticks=2]{->}(0,0)(-1.82,-1.68)(4.68,3.76)[x,140] [f(x),-40]
\rput[tl](2.32,1.2){f}
\psplot[linewidth=2.pt,plotpoints=200]{-1.8199999999999994}{4.679999999999998}{COS(2.2*x)}
\end{pspicture*}}}, 				%Moeglichkeit C  
				D={\resizebox{0.9\linewidth}{!}{\psset{xunit=1.0cm,yunit=0.7cm,algebraic=true,dimen=middle,dotstyle=o,dotsize=5pt 0,linewidth=1.6pt,arrowsize=3pt 2,arrowinset=0.25}
\begin{pspicture*}(-1.82,-1.68)(4.68,3.76)
\multips(0,-1)(0,1.0){6}{\psline[linestyle=dashed,linecap=1,dash=1.5pt 1.5pt,linewidth=0.4pt,linecolor=darkgray]{c-c}(-1.82,0)(4.68,0)}
\multips(-1,0)(1.0,0){7}{\psline[linestyle=dashed,linecap=1,dash=1.5pt 1.5pt,linewidth=0.4pt,linecolor=darkgray]{c-c}(0,-1.68)(0,3.76)}
\psaxes[labelFontSize=\scriptstyle,xAxis=true,yAxis=true,Dx=1.,Dy=1.,ticksize=-2pt 0,subticks=2]{->}(0,0)(-1.82,-1.68)(4.68,3.76)[x,140] [f(x),-40]
\rput[tl](1.46,2.34){f}
\psplot[linewidth=2.pt,plotpoints=200]{-1.8199999999999994}{-0.2}{1.0/(x+0.1)+2.0}
\psplot[linewidth=2.pt,plotpoints=200]{0.2}{4.679999999999998}{1.0/(x+0.1)+2.0}
\end{pspicture*}}}, 				%Moeglichkeit D  
				E={\resizebox{0.9\linewidth}{!}{\psset{xunit=1.0cm,yunit=0.7cm,algebraic=true,dimen=middle,dotstyle=o,dotsize=5pt 0,linewidth=1.6pt,arrowsize=3pt 2,arrowinset=0.25}
\begin{pspicture*}(-1.82,-1.68)(4.68,3.76)
\multips(0,-1)(0,1.0){6}{\psline[linestyle=dashed,linecap=1,dash=1.5pt 1.5pt,linewidth=0.4pt,linecolor=darkgray]{c-c}(-1.82,0)(4.68,0)}
\multips(-1,0)(1.0,0){7}{\psline[linestyle=dashed,linecap=1,dash=1.5pt 1.5pt,linewidth=0.4pt,linecolor=darkgray]{c-c}(0,-1.68)(0,3.76)}
\psaxes[labelFontSize=\scriptstyle,xAxis=true,yAxis=true,Dx=1.,Dy=1.,ticksize=-2pt 0,subticks=2]{->}(0,0)(-1.82,-1.68)(4.68,3.76)[x,140] [f(x),-40]
\rput[tl](1.46,1.14){f}
\psplot[linewidth=2.pt,plotpoints=200]{-1.8199999999999996}{4.679999999999997}{1.3*SIN(2.0*x)}
\end{pspicture*}}}, 				%Moeglichkeit E  
				F={\resizebox{0.9\linewidth}{!}{\psset{xunit=1.0cm,yunit=0.7cm,algebraic=true,dimen=middle,dotstyle=o,dotsize=5pt 0,linewidth=1.6pt,arrowsize=3pt 2,arrowinset=0.25}
\begin{pspicture*}(-1.82,-1.68)(4.68,3.76)
\multips(0,-1)(0,1.0){6}{\psline[linestyle=dashed,linecap=1,dash=1.5pt 1.5pt,linewidth=0.4pt,linecolor=darkgray]{c-c}(-1.82,0)(4.68,0)}
\multips(-1,0)(1.0,0){7}{\psline[linestyle=dashed,linecap=1,dash=1.5pt 1.5pt,linewidth=0.4pt,linecolor=darkgray]{c-c}(0,-1.68)(0,3.76)}
\psaxes[labelFontSize=\scriptstyle,xAxis=true,yAxis=true,Dx=1.,Dy=1.,ticksize=-2pt 0,subticks=2]{->}(0,0)(-1.82,-1.68)(4.68,3.76)[x,140] [f(x),-40]
\rput[tl](1.3,2.64){f}
\psplot[linewidth=2.pt,plotpoints=200]{6.499999998917529E-6}{4.679999999999997}{sqrt(x)+1.0}
\end{pspicture*}}}, 				%Moeglichkeit F  
				%% LOESUNG: %%
				A1={E},				% 1. richtige Zuordnung
				A2={A},				% 2. richtige Zuordnung
				A3={F},				% 3. richtige Zuordnung
				A4={B},				% 4. richtige Zuordnung
				}
\end{beispiel}