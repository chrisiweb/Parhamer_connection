\section{FA 1.9 - 6 - MAT - Funktionale Zusammenhänge - MC - Matura 2018/19 2. NT}

\begin{beispiel}[FA 1.9]{1}
Gegeben ist die Gleichung $w=\dfrac{y\cdot z^2}{2\cdot x}$ mit $w,x,y,z\in\mathbb{R}^+$.

Die gegebene Gleichung beschreibt funktionale Zusammenhänge zwischen zwei Variablen, wenn die beiden anderen Variablen als konstant angenommen werden.

Kreuze die beiden zutreffenden Aussagen an.

\multiplechoice[5]{  %Anzahl der Antwortmoeglichkeiten, Standard: 5
				L1={Betrachtet man $z$ in Abhängigkeit von $x$, so ist $z$: $\mathbb{R}^+\rightarrow\mathbb{R}^+$, $x\rightarrow z(x)$ eine Exponentialfunktion.},   %1. Antwortmoeglichkeit 
				L2={Betrachtet man $w$ in Abhängigkeit von $z$, so ist $w$: $\mathbb{R}^+\rightarrow\mathbb{R}^+$, $z\rightarrow w(z)$ eine quadratische Funktion.},   %2. Antwortmoeglichkeit
				L3={Betrachtet man $w$ in Abhängigkeit von $x$, so ist $w$: $\mathbb{R}^+\rightarrow\mathbb{R}^+$, $x\rightarrow w(x)$ eine lineare Funktion.},   %3. Antwortmoeglichkeit
				L4={Betrachtet man $y$ in Abhängigkeit von $z$, so ist $y$: $\mathbb{R}^+\rightarrow\mathbb{R}^+$, $z\rightarrow y(z)$ eine Polynomfunktion vom Grad 2.},   %4. Antwortmoeglichkeit
				L5={Betrachtet man $x$ in Abhängigkeit von $y$, so ist $x$: $\mathbb{R}^+\rightarrow\mathbb{R}^+$, $y\rightarrow x(y)$ eine lineare Funktion.},	 %5. Antwortmoeglichkeit
				L6={},	 %6. Antwortmoeglichkeit
				L7={},	 %7. Antwortmoeglichkeit
				L8={},	 %8. Antwortmoeglichkeit
				L9={},	 %9. Antwortmoeglichkeit
				%% LOESUNG: %%
				A1=2,  % 1. Antwort
				A2=5,	 % 2. Antwort
				A3=0,  % 3. Antwort
				A4=0,  % 4. Antwort
				A5=0,  % 5. Antwort
				}
\end{beispiel}