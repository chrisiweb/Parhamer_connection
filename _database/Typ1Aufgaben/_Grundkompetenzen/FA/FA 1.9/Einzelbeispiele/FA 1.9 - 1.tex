\section{FA 1.9 - 1 Eigenschaften von Funktionen - ZO - BIFIE}

\begin{beispiel}[FA 1.9]{1} %PUNKTE DES BEISPIELS
Es sind vier Funktionen $f_1,f_2,f_3,f_4$ durch ihre Gleichungen gegeben.

Ordne den vier Funktionsgleichungen jeweils die entsprechende Aussage (aus A bis F) zu!

\zuordnen[-0.12]{
				title1={Aussagen}, 		%Titel Antwortmoeglichkeiten
				A={Der Graph der Funktion hat genau ein lokales Maximum (einen Hochpunkt).}, 				%Moeglichkeit A  
				B={Die Funktion besitzt keine Nullstelle und ist stets streng monoton wachsend.}, 				%Moeglichkeit B  
				C={Der Graph der Funktion ist symmetrisch zur 2. Achse.}, 				%Moeglichkeit C  
				D={Die Funktion hat genau eine Wendestelle.}, 				%Moeglichkeit D  
				E={Der Graph der Funktion $f$ geht durch (0/0).}, 				%Moeglichkeit E  
				F={Mit wachsenden x-Werten nähert sich der Graph der Funktion der x-Achse.}, 				%Moeglichkeit F  
				title2={Funktionsgleichungen},		%Titel Zuordnung
				R1={$f_1(x)=2\cdot x^3+1$},				%1. Antwort rechts
				R2={$f_2(x)=\sin(x)$},				%2. Antwort rechts
				R3={$f_3(x)=e^x$},				%3. Antwort rechts
				R4={$f_4(x)=e^{-x}$},				%4. Antwort rechts
				%% LOESUNG: %%
				A1={D},				% 1. richtige Zuordnung
				A2={E},				% 2. richtige Zuordnung
				A3={B},				% 3. richtige Zuordnung
				A4={F},				% 4. richtige Zuordnung
				}
\end{beispiel}