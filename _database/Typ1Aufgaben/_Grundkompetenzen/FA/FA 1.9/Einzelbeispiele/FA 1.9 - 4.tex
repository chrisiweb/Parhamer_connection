\section{FA 1.9 - 4 - MAT - Eigenschaften von Funktionen zuordnen - ZO - Matura 2013/14 1. Nebentermin}

\begin{beispiel}[FA 1.9]{1} %PUNKTE DES BEISPIELS
				Gegeben sind vier Funktionstypen. Für alle unten angeführten Funktionen gilt: $a\neq 0; b\neq 0; a,b\in\mathbb{R}$.
				
				Ordne den vier Funktionstypen jeweils die passende Eigenschaft (aus A bis F) zu!\leer
				
			
				\zuordnen[0.05]{
								R1={lineare Funktion $f$ mit\\
								$f(x)=a\cdot x+b$},				% Response 1
								R2={Exponentialfunktion $f$ mit\\
								$f(x)=a\cdot b^x(b>0,b\neq 1)$},				% Response 2
								R3={Wurzelfunktion $f$ mit\\
								$f(x)=a\cdot x^{\frac{1}{2}}+b$},				% Response 3
								R4={Sinusfunktion $f$ mit\\
								$f(x)=a\cdot\sin(b\cdot x)$},				% Response 4
								%% Moegliche Zuordnungen: %%
								A={Die Funktion $f$ ist für $a>0$ und \mbox{$0<b<1$} streng monoton fallend.}, 				%Moeglichkeit A  
								B={Die Funktion $f$ besitzt genau drei Nullstellen.}, 				%Moeglichkeit B  
								C={Die Funktion $f$ besitzt in jedem Punkt die gleiche Steigung.}, 				%Moeglichkeit C  
								D={Der Graph der Funktion $f$ besitzt einen Wendepunkt im Ursprung.}, 				%Moeglichkeit D  
								E={Die Funktion $f$ ist für $b=2$ konstant.}, 				%Moeglichkeit E  
								F={Die Funktion $f$ ist nur für $x\geq 0$ definiert.}, 				%Moeglichkeit F  
								%% LOESUNG: %%
								A1={C},				% 1. richtige Zuordnung
								A2={A},				% 2. richtige Zuordnung
								A3={F},				% 3. richtige Zuordnung
								A4={D},				% 4. richtige Zuordnung
								}
\end{beispiel}