\section{FA 6.6 - 1 - Ableitung der Sinusfunktion - MC - BIFIE}

\begin{beispiel}[FA 6.6]{1} %PUNKTE DES BEISPIELS
Gegeben ist die Funktion $f$ mit $f(x)=\sin (x)$.

Kreuze von den gegebenen Graphen von Ableitungsfunktionen $f'$ denjenigen an, der zur Funktion $f$ gehört!

\multiplechoice[6]{  %Anzahl der Antwortmoeglichkeiten, Standard: 5
				L1={\psset{xunit=1cm,yunit=1cm,algebraic=true,dimen=middle,dotstyle=o,dotsize=5pt 0,linewidth=0.8pt,arrowsize=3pt 2,arrowinset=0.25}
\begin{pspicture*}(-0.9267087200010348,-1.506049799210546)(9.755428270191501,1.4373654977675405)
\multips(0,-1)(0,1.0){3}{\psline[linestyle=dashed,linecap=1,dash=1.5pt 1.5pt,showorigin=false,linewidth=0.4pt,linecolor=gray]{c-c}(-0.9267087200010348,0)(9.755428270191501,0)}
\multips(0,0)(1.5707963267948966,0){7}{\psline[linestyle=dashed,linecap=1,dash=1.5pt 1.5pt,linewidth=0.4pt,linecolor=gray]{c-c}(0,-1.506049799210546)(0,1.4373654977675405)}
\psaxes[labelFontSize=\scriptstyle,xAxis=true,yAxis=true,labels=y,showorigin=false,Dx=3.141592653589793,Dy=1.,ticksize=-2pt 0,subticks=0]{->}(0,0)(-0.9267087200010348,-1.506049799210546)(9.755428270191501,1.4373654977675405)[\scriptsize{$x$},140] [\scriptsize{$f'(x)$},-40]
\psplot[linewidth=1.2pt,plotpoints=200]{-0.9267087200010348}{9.755428270191501}{SIN(x)}
\begin{scriptsize}
\rput[tl](2.945445536137288,-0.12){$\pi$}
\rput[tl](6.316724117352246,-0.12){$2\pi$}
\end{scriptsize}
\end{pspicture*}},   %1. Antwortmoeglichkeit 
				L2={\psset{xunit=1cm,yunit=1cm,algebraic=true,dimen=middle,dotstyle=o,dotsize=5pt 0,linewidth=0.8pt,arrowsize=3pt 2,arrowinset=0.25}
\begin{pspicture*}(-0.9267087200010348,-1.506049799210546)(9.755428270191501,1.4373654977675405)
\multips(0,-1)(0,1.0){3}{\psline[linestyle=dashed,linecap=1,dash=1.5pt 1.5pt,linewidth=0.4pt,linecolor=gray]{c-c}(-0.9267087200010348,0)(9.755428270191501,0)}
\multips(0,0)(1.5707963267948966,0){7}{\psline[linestyle=dashed,linecap=1,dash=1.5pt 1.5pt,linewidth=0.4pt,linecolor=gray]{c-c}(0,-1.506049799210546)(0,1.4373654977675405)}
\psaxes[labelFontSize=\scriptstyle,xAxis=true,yAxis=true,labels=y,showorigin=false,Dx=3.141592653589793,Dy=1.,ticksize=-2pt 0,subticks=0]{->}(0,0)(-0.9267087200010348,-1.506049799210546)(9.755428270191501,1.4373654977675405)[\scriptsize{$x$},140] [\scriptsize{$f'(x)$},-40]
\psplot[linewidth=1.2pt,plotpoints=200]{-0.9267087200010348}{9.755428270191501}{COS(x)}
\begin{scriptsize}
\rput[tl](2.945445536137288,-0.12){$\pi$}
\rput[tl](6.316724117352246,-0.12){$2\pi$}
\end{scriptsize}
\end{pspicture*}},   %2. Antwortmoeglichkeit
				L3={\psset{xunit=1cm,yunit=1cm,algebraic=true,dimen=middle,dotstyle=o,dotsize=5pt 0,linewidth=0.8pt,arrowsize=3pt 2,arrowinset=0.25}
\begin{pspicture*}(-0.9267087200010348,-1.506049799210546)(9.755428270191501,1.4373654977675405)
\multips(0,-1)(0,1.0){3}{\psline[linestyle=dashed,linecap=1,dash=1.5pt 1.5pt,linewidth=0.4pt,linecolor=gray]{c-c}(-0.9267087200010348,0)(9.755428270191501,0)}
\multips(0,0)(1.5707963267948966,0){7}{\psline[linestyle=dashed,linecap=1,dash=1.5pt 1.5pt,linewidth=0.4pt,linecolor=gray]{c-c}(0,-1.506049799210546)(0,1.4373654977675405)}
\psaxes[labelFontSize=\scriptstyle,xAxis=true,yAxis=true,labels=y,showorigin=false,Dx=3.141592653589793,Dy=1.,ticksize=-2pt 0,subticks=0]{->}(0,0)(-0.9267087200010348,-1.506049799210546)(9.755428270191501,1.4373654977675405)[\scriptsize{$x$},140] [\scriptsize{$f'(x)$},-40]
\psplot[linewidth=1.2pt,plotpoints=200]{-0.9267087200010348}{9.755428270191501}{SIN(2*x)}
\begin{scriptsize}
\rput[tl](2.945445536137288,-0.12){$\pi$}
\rput[tl](6.316724117352246,-0.12){$2\pi$}
\end{scriptsize}
\end{pspicture*}},   %3. Antwortmoeglichkeit
				L4={\psset{xunit=1cm,yunit=1cm,algebraic=true,dimen=middle,dotstyle=o,dotsize=5pt 0,linewidth=0.8pt,arrowsize=3pt 2,arrowinset=0.25}
\begin{pspicture*}(-0.9267087200010348,-1.506049799210546)(9.755428270191501,1.4373654977675405)
\multips(0,-1)(0,1.0){3}{\psline[linestyle=dashed,linecap=1,dash=1.5pt 1.5pt,linewidth=0.4pt,linecolor=gray]{c-c}(-0.9267087200010348,0)(9.755428270191501,0)}
\multips(0,0)(1.5707963267948966,0){7}{\psline[linestyle=dashed,linecap=1,dash=1.5pt 1.5pt,linewidth=0.4pt,linecolor=gray]{c-c}(0,-1.506049799210546)(0,1.4373654977675405)}
\psaxes[labelFontSize=\scriptstyle,xAxis=true,yAxis=true,labels=y,showorigin=false,Dx=3.141592653589793,Dy=1.,ticksize=-2pt 0,subticks=0]{->}(0,0)(-0.9267087200010348,-1.506049799210546)(9.755428270191501,1.4373654977675405)[\scriptsize{$x$},140] [\scriptsize{$f'(x)$},-40]
\psplot[linewidth=1.2pt,plotpoints=200]{-0.9267087200010348}{9.755428270191501}{-SIN(x)}
\begin{scriptsize}
\rput[tl](2.945445536137288,-0.12){$\pi$}
\rput[tl](6.316724117352246,-0.12){$2\pi$}
\end{scriptsize}
\end{pspicture*}},   %4. Antwortmoeglichkeit
				L5={\psset{xunit=1cm,yunit=1cm,algebraic=true,dimen=middle,dotstyle=o,dotsize=5pt 0,linewidth=0.8pt,arrowsize=3pt 2,arrowinset=0.25}
\begin{pspicture*}(-0.9267087200010348,-1.506049799210546)(9.755428270191501,1.4373654977675405)
\multips(0,-1)(0,1.0){3}{\psline[linestyle=dashed,linecap=1,dash=1.5pt 1.5pt,linewidth=0.4pt,linecolor=gray]{c-c}(-0.9267087200010348,0)(9.755428270191501,0)}
\multips(0,0)(1.5707963267948966,0){7}{\psline[linestyle=dashed,linecap=1,dash=1.5pt 1.5pt,linewidth=0.4pt,linecolor=gray]{c-c}(0,-1.506049799210546)(0,1.4373654977675405)}
\psaxes[labelFontSize=\scriptstyle,xAxis=true,yAxis=true,labels=y,showorigin=false,Dx=3.141592653589793,Dy=1.,ticksize=-2pt 0,subticks=0]{->}(0,0)(-0.9267087200010348,-1.506049799210546)(9.755428270191501,1.4373654977675405)[\scriptsize{$x$},140] [\scriptsize{$f'(x)$},-40]
\psplot[linewidth=1.2pt,plotpoints=200]{-0.9267087200010348}{9.755428270191501}{-COS(x)}
\begin{scriptsize}
\rput[tl](2.945445536137288,-0.12){$\pi$}
\rput[tl](6.316724117352246,-0.12){$2\pi$}
\end{scriptsize}
\end{pspicture*}},	 %5. Antwortmoeglichkeit
				L6={\psset{xunit=1cm,yunit=1cm,algebraic=true,dimen=middle,dotstyle=o,dotsize=5pt 0,linewidth=0.8pt,arrowsize=3pt 2,arrowinset=0.25}
\begin{pspicture*}(-0.9267087200010348,-1.506049799210546)(9.755428270191501,1.4373654977675405)
\multips(0,-1)(0,1.0){3}{\psline[linestyle=dashed,linecap=1,dash=1.5pt 1.5pt,linewidth=0.4pt,linecolor=gray]{c-c}(-0.9267087200010348,0)(9.755428270191501,0)}
\multips(0,0)(1.5707963267948966,0){7}{\psline[linestyle=dashed,linecap=1,dash=1.5pt 1.5pt,linewidth=0.4pt,linecolor=gray]{c-c}(0,-1.506049799210546)(0,1.4373654977675405)}
\psaxes[labelFontSize=\scriptstyle,xAxis=true,yAxis=true,labels=y,showorigin=false,Dx=3.141592653589793,Dy=1.,ticksize=-2pt 0,subticks=0]{->}(0,0)(-0.9267087200010348,-1.506049799210546)(9.755428270191501,1.4373654977675405)[\scriptsize{$x$},140] [\scriptsize{$f'(x)$},-40]
\psplot[linewidth=1.2pt,plotpoints=200]{-0.9267087200010348}{9.755428270191501}{COS(2*x)}
\begin{scriptsize}
\rput[tl](2.945445536137288,-0.12){$\pi$}
\rput[tl](6.316724117352246,-0.12){$2\pi$}
\end{scriptsize}
\end{pspicture*}},	 %6. Antwortmoeglichkeit
				L7={},	 %7. Antwortmoeglichkeit
				L8={},	 %8. Antwortmoeglichkeit
				L9={},	 %9. Antwortmoeglichkeit
				%% LOESUNG: %%
				A1=2,  % 1. Antwort
				A2=0,	 % 2. Antwort
				A3=0,  % 3. Antwort
				A4=0,  % 4. Antwort
				A5=0,  % 5. Antwort
				}
\end{beispiel}