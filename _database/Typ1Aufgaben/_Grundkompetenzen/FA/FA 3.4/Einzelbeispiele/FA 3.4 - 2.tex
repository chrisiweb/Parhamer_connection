\section{FA 3.4 - 2 Ideales Gas - OA - BIFIE}

\begin{beispiel}[FA 3.4]{1} %PUNKTE DES BEISPIELS
Die Abhängigkeit des Volumens $V$ vom Druck $p$ kann durch eine Funktion beschrieben werden. Bei gleichbleibender Temperatur ist das Volumen $V$ eines idealen Glases zum Druck $p$ indirekt proportional.

$200\,cm^3$ eines idealen Glases stehen bei konstanter Temperatur unter einem Druck von 1 bar.

Gib den Term der Funktionsgleichung an und zeichne deren Graphen!

$V(p)=\rule{5cm}{0.3pt}$
\leer

\resizebox{0.8\linewidth}{!}{\newrgbcolor{zzttff}{0.6 0.2 1.}
\psset{xunit=.25cm,yunit=0.016cm,algebraic=true,dimen=middle,dotstyle=o,dotsize=5pt 0,linewidth=0.8pt,arrowsize=3pt 2,arrowinset=0.25}
\begin{pspicture*}(-4.6085080925933988,-50.07821127942692)(23.490720118471412,261.26520642761443)
\multips(0,0)(0,20.0){16}{\psline[linestyle=dashed,linecap=1,dash=1.5pt 1.5pt,linewidth=0.4pt,linecolor=lightgray]{c-c}(0,0)(23.490720118471412,0)}
\multips(0,0)(2.0,0){14}{\psline[linestyle=dashed,linecap=1,dash=1.5pt 1.5pt,linewidth=0.4pt,linecolor=lightgray]{c-c}(0,0)(0,261.26520642761443)}
\psaxes[labelFontSize=\scriptstyle,xAxis=true,yAxis=true,Dx=2.,Dy=20.,ticksize=-2pt 0,subticks=2]{->}(0,0)(0.,0.)(23.490720118471412,261.26520642761443)
\rput[tl](-4.337515810482751,242.47640820127612){\rotatebox{90.0}{ \scriptsize{$V(p) \text{ in } cm^3$ }}}
\rput[tl](18.667057496901876,-30.836491340520676){\scriptsize{$p \text{ in bar }$}}
\antwort{\psplot[linewidth=1.2pt,linecolor=zzttff,plotpoints=200]{0.1}{23.490720118471412}{200.0/x}
\rput[tl](2.461719026685119,114.56805104504994){v}}
\end{pspicture*}}
\leer

\antwort{$V(p)=\frac{c}{p}$

$200=\frac{c}{1}$

$V(p)=\frac{200}{p}$}
\end{beispiel}