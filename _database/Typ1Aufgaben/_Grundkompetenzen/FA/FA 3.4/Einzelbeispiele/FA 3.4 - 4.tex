\section{FA 3.4 - 4 - MAT - Weinlese - OA - Matura 2019/20 1. HT}

\begin{beispiel}[FA 3.4]{1}
Sie sogenannte \textit{Weinlese} (Ernte der Weintrauben) in einem Weingarten erfolgt umso schneller, je mehr Personen daran beteiligt sind. Die Funktion $f$ modelliert den indirekt proportionalen Zusammenhang zwischen der für die Weinlese benötigte Zeit und der Anzahl der beteiligten Personen. Dabei ist $f(n)$ die benötigte Zeit für die Weinlese, wenn $n$ Personen beteiligt sind ($n\in\mathbb{N}\backslash\{0\}$, $f(n)$ in Stunden).

Gib $f(n)$ an, wenn bekannt ist, dass die benötigte Zeit für die Weinlese bei einer Anzahl von 8 beteiligten Personen 6 Stunden beträgt.\leer

$f(n)=\,\antwort[\rule{5cm}{0.3pt}]{\dfrac{48}{n}}$ mit $n\in\mathbb{N}\backslash\{0\}$
\end{beispiel}