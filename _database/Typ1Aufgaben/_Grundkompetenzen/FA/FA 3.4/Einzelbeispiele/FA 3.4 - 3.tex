\section{FA 3.4 - 3 Gleichung einer indirekten Proportionalität - OA - BIFIE}

\begin{beispiel}[FA 3.4]{1} %PUNKTE DES BEISPIELS
Gegeben ist eine Funktion f mit der Gleichung $f(x) = a \cdot x^z + b$, wobei $z\in \mathbb{Z}$ und $a, b \in \mathbb{R}$ gilt.
\leer

Welche Werte müssen die Parameter $b$ und $z$ annehmen, damit durch $f$ ein indirekt proportionaler Zusammenhang beschrieben wird? \leer

Ermittle die Werte der Parameter $b$ und $z$.

\leer

$b=$\rule{5cm}{0.3pt} \leer

$z=$\rule{5cm}{0.3pt}


\antwort{$b=0$

$z=-1$}
\end{beispiel}