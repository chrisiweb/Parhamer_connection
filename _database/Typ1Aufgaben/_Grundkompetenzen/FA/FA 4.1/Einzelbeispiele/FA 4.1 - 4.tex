\section{FA 4.1 - 4 - MAT - Polynomfunktion vom Grad $n$ - LT - Matura 2015/16 Nebentermin 1}

\begin{beispiel}[FA 4.1]{1} %PUNKTE DES BEISPIELS
Die nachstehende Abbildung zeigt den Graphen einer Polynomfunktion $f$. Alle charakteristischen
Punkte des Graphen (Schnittpunkte mit den Achsen, Extrempunkte, Wendepunkte) sind in dieser
Abbildung enthalten. 


\begin{center}

\psset{xunit=1.0cm,yunit=1.0cm,algebraic=true,dimen=middle,dotstyle=o,dotsize=5pt 0,linewidth=0.8pt,arrowsize=3pt 2,arrowinset=0.25}
\begin{pspicture*}(-3.3,-3.34)(1.72,3.54)
\multips(0,-3)(0,1.0){7}{\psline[linestyle=dashed,linecap=1,dash=1.5pt 1.5pt,linewidth=0.4pt,linecolor=black!60]{c-c}(-3.72,0)(1.72,0)}
\multips(-3,0)(1.0,0){6}{\psline[linestyle=dashed,linecap=1,dash=1.5pt 1.5pt,linewidth=0.4pt,linecolor=black!60]{c-c}(0,-3.34)(0,3.54)}
\psaxes[labelFontSize=\scriptstyle,xAxis=true,yAxis=true,Dx=1.,Dy=1.,showorigin=false,ticksize=-2pt 0,subticks=0]{->}(0,0)(-3.72,-3.34)(1.72,3.54)[\scriptsize{$x$},140] [\scriptsize{$f(x)$},-40]
\psplot[linewidth=1.2pt,plotpoints=200]{-3.72}{1.7200000000000017}{0.5898906689021344*x^(4.0)+1.513114671137602*x^(3.0)-0.08989066890213436*x^(2.0)-0.013114671137602042*x}
\begin{scriptsize}
\rput[tl](1.22,3.){$f$}
\end{scriptsize}
\end{pspicture*}
\end{center}


\lueckentext{
				text={Die Polynomfunktion $f$ ist vom Grad \gap, weil $f$ genau \gap hat.}, 	%Lueckentext Luecke=\gap
				L1={$n<3$}, 		%1.Moeglichkeit links  
				L2={$n=3$}, 		%2.Moeglichkeit links
				L3={$n>3$}, 		%3.Moeglichkeit links
				R1={eine Extremstelle}, 		%1.Moeglichkeit rechts 
				R2={zwei Wendestellen}, 		%2.Moeglichkeit rechts
				R3={zwei Nullstellen}, 		%3.Moeglichkeit rechts
				%% LOESUNG: %%
				A1=3,   % Antwort links
				A2=2		% Antwort rechts 
				}
\end{beispiel}