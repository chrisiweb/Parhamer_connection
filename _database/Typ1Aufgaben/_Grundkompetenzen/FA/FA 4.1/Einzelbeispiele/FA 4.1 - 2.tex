\section{FA 4.1 - 2 - Graphen von Polynomfunktionen - MC - BIFIE}

\begin{beispiel}[FA 4.1]{1} %PUNKTE DES BEISPIELS
				Gegeben ist eine Polynomfunktion $f$ dritten Grades.

Kreuze diejenige(n) Abbildung(en) an, die einen möglichen Funktionsgraphen von $f$ zeigt/zeigen.

\langmultiplechoice[5]{  %Anzahl der Antwortmoeglichkeiten, Standard: 5
				L1={\resizebox{0.9\linewidth}{!}{\psset{xunit=1.0cm,yunit=1.0cm,algebraic=true,dimen=middle,dotstyle=o,dotsize=5pt 0,linewidth=0.8pt,arrowsize=3pt 2,arrowinset=0.25}
\begin{pspicture*}(-3.5293578935800984,-3.395479620485891)(3.57282683975511,3.33935073009062)
\multips(0,-3)(0,1.0){7}{\psline[linestyle=dashed,linecap=1,dash=1.5pt 1.5pt,linewidth=0.4pt,linecolor=darkgray]{c-c}(-3.5293578935800984,0)(3.57282683975511,0)}
\multips(-3,0)(1.0,0){8}{\psline[linestyle=dashed,linecap=1,dash=1.5pt 1.5pt,linewidth=0.4pt,linecolor=darkgray]{c-c}(0,-3.395479620485891)(0,3.33935073009062)}
\psaxes[labelFontSize=\scriptstyle,xAxis=true,yAxis=true,labels=none,Dx=1.,Dy=1.,ticksize=-2pt 0,subticks=0]{->}(0,0)(-3.5293578935800984,-3.395479620485891)(3.57282683975511,3.33935073009062)[x,140] [f(x),-40]
\psplot[linewidth=1.2pt,plotpoints=200]{-3.5293578935800984}{3.57282683975511}{0.6515379335652558*x^(3.0)-0.2932894957441645*x^(2.0)-1.247903296439932*x+0.8896548586188406}
\end{pspicture*}}},   %1. Antwortmoeglichkeit 
				L2={\resizebox{0.9\linewidth}{!}{\psset{xunit=1.0cm,yunit=1.0cm,algebraic=true,dimen=middle,dotstyle=o,dotsize=5pt 0,linewidth=0.8pt,arrowsize=3pt 2,arrowinset=0.25}
\begin{pspicture*}(-3.5293578935800984,-3.395479620485891)(3.57282683975511,3.33935073009062)
\multips(0,-3)(0,1.0){7}{\psline[linestyle=dashed,linecap=1,dash=1.5pt 1.5pt,linewidth=0.4pt,linecolor=darkgray]{c-c}(-3.5293578935800984,0)(3.57282683975511,0)}
\multips(-3,0)(1.0,0){8}{\psline[linestyle=dashed,linecap=1,dash=1.5pt 1.5pt,linewidth=0.4pt,linecolor=darkgray]{c-c}(0,-3.395479620485891)(0,3.33935073009062)}
\psaxes[labelFontSize=\scriptstyle,xAxis=true,yAxis=true,labels=none,Dx=1.,Dy=1.,ticksize=-2pt 0,subticks=0]{->}(0,0)(-3.5293578935800984,-3.395479620485891)(3.57282683975511,3.33935073009062)[x,140] [f(x),-40]
\psplot[linewidth=1.2pt,plotpoints=200]{-3.5293578935800984}{3.57282683975511}{0.4166666666666667*x^(4.0)-1.4166666666666667*x^(2.0)-1.0}
\end{pspicture*}}},   %2. Antwortmoeglichkeit
				L3={\resizebox{0.9\linewidth}{!}{\psset{xunit=1.0cm,yunit=1.0cm,algebraic=true,dimen=middle,dotstyle=o,dotsize=5pt 0,linewidth=0.8pt,arrowsize=3pt 2,arrowinset=0.25}
\begin{pspicture*}(-3.5293578935800984,-3.395479620485891)(3.57282683975511,3.33935073009062)
\multips(0,-3)(0,1.0){7}{\psline[linestyle=dashed,linecap=1,dash=1.5pt 1.5pt,linewidth=0.4pt,linecolor=darkgray]{c-c}(-3.5293578935800984,0)(3.57282683975511,0)}
\multips(-3,0)(1.0,0){8}{\psline[linestyle=dashed,linecap=1,dash=1.5pt 1.5pt,linewidth=0.4pt,linecolor=darkgray]{c-c}(0,-3.395479620485891)(0,3.33935073009062)}
\psaxes[labelFontSize=\scriptstyle,xAxis=true,yAxis=true,labels=none,Dx=1.,Dy=1.,ticksize=-2pt 0,subticks=0]{->}(0,0)(-3.5293578935800984,-3.395479620485891)(3.57282683975511,3.33935073009062)[x,140] [f(x),-40]
\psplot[linewidth=1.2pt,plotpoints=200]{-3.5293578935800984}{3.57282683975511}{-0.5158841765564153*x^(3.0)+0.3019605653462957*x^(2.0)+0.7053075398560196*x-1.4913839286459}
\end{pspicture*}}},   %3. Antwortmoeglichkeit
				L4={\resizebox{0.9\linewidth}{!}{\psset{xunit=1.0cm,yunit=1.0cm,algebraic=true,dimen=middle,dotstyle=o,dotsize=5pt 0,linewidth=0.8pt,arrowsize=3pt 2,arrowinset=0.25}
\begin{pspicture*}(-3.5293578935800984,-3.395479620485891)(3.57282683975511,3.33935073009062)
\multips(0,-3)(0,1.0){7}{\psline[linestyle=dashed,linecap=1,dash=1.5pt 1.5pt,linewidth=0.4pt,linecolor=darkgray]{c-c}(-3.5293578935800984,0)(3.57282683975511,0)}
\multips(-3,0)(1.0,0){8}{\psline[linestyle=dashed,linecap=1,dash=1.5pt 1.5pt,linewidth=0.4pt,linecolor=darkgray]{c-c}(0,-3.395479620485891)(0,3.33935073009062)}
\psaxes[labelFontSize=\scriptstyle,xAxis=true,yAxis=true,labels=none,Dx=1.,Dy=1.,ticksize=-2pt 0,subticks=0]{->}(0,0)(-3.5293578935800984,-3.395479620485891)(3.57282683975511,3.33935073009062)[x,140] [f(x),-40]
\psplot[linewidth=1.2pt,plotpoints=200]{-3.5293578935800984}{3.57282683975511}{x^(3.0)+1.0}
\end{pspicture*}}},   %4. Antwortmoeglichkeit
				L5={\resizebox{0.9\linewidth}{!}{\psset{xunit=1.0cm,yunit=1.0cm,algebraic=true,dimen=middle,dotstyle=o,dotsize=5pt 0,linewidth=0.8pt,arrowsize=3pt 2,arrowinset=0.25}
\begin{pspicture*}(-3.3824161404766113,-3.138331552554788)(3.719768592858597,3.596498798021723)
\multips(0,-3)(0,1.0){7}{\psline[linestyle=dashed,linecap=1,dash=1.5pt 1.5pt,linewidth=0.4pt,linecolor=darkgray]{c-c}(-3.3824161404766113,0)(3.719768592858597,0)}
\multips(-3,0)(1.0,0){8}{\psline[linestyle=dashed,linecap=1,dash=1.5pt 1.5pt,linewidth=0.4pt,linecolor=darkgray]{c-c}(0,-3.138331552554788)(0,3.596498798021723)}
\psaxes[labelFontSize=\scriptstyle,xAxis=true,yAxis=true,labels=none,Dx=1.,Dy=1.,ticksize=-2pt 0,subticks=0]{->}(0,0)(-3.3824161404766113,-3.138331552554788)(3.719768592858597,3.596498798021723)[x,140] [f(x),-40]
\psplot[linewidth=1.2pt,plotpoints=200]{-3.3824161404766113}{3.719768592858597}{-1.2864373177962225*x^(4.0)+2.442645498117902*x^(3.0)+0.9316905690702632*x^(2.0)-3.0873058683058248*x+1.7100863785380367}
\end{pspicture*}}},	 %5. Antwortmoeglichkeit
				L6={},	 %6. Antwortmoeglichkeit
				L7={},	 %7. Antwortmoeglichkeit
				L8={},	 %8. Antwortmoeglichkeit
				L9={},	 %9. Antwortmoeglichkeit
				%% LOESUNG: %%
				A1=1,  % 1. Antwort
				A2=3,	 % 2. Antwort
				A3=4,  % 3. Antwort
				A4=0,  % 4. Antwort
				A5=0,  % 5. Antwort
				}
\end{beispiel}