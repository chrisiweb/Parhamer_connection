\section{FA 4.1 - 6 - Polynomfunktion - MC - MarPar UNIVIE}

\begin{beispiel}[FA 4.1]{1}
Gegeben ist eine Polynomfunktion $f$ mit $f(x)=ax^3 +bx^2+cx +d$\\
 mit $a>0,\ b=0,\ c\geq0,\ d\neq0$.

Kreuze die beiden zutreffenden Aussagen an.

\multiplechoice[5]{  %Anzahl der Antwortmoeglichkeiten, Standard: 5
				L1={Für alle ${\displaystyle x_1,x_2\in \mathbb{R}} $ mit $ {\displaystyle x_1<x_2} $ gilt:\ $ {\displaystyle f(x_1)<f(x_2)}.$},   %1. Antwortmoeglichkeit 
				L2={Die Funktion hat keine Nullstellen.},   %2. Antwortmoeglichkeit
				L3={Für alle $x \in \mathbb{R}$ gilt: \ $f(-x)=-f(x)$.},   %3. Antwortmoeglichkeit
				L4={Es gibt $x_1,x_2 \in \mathbb{R}$ mit $x_1 \neq x_2, $ für die gilt: \ $ f^\prime (x_1)= f^\prime (x_2) $.},   %4. Antwortmoeglichkeit
				L5={Der Graph der Funktion $f$ ist symmetrisch zur y-Achse.},	 %5. Antwortmoeglichkeit
				L6={},	 %6. Antwortmoeglichkeit
				L7={},	 %7. Antwortmoeglichkeit
				L8={},	 %8. Antwortmoeglichkeit
				L9={},	 %9. Antwortmoeglichkeit
				%% LOESUNG: %%
				A1=1,  % 1. Antwort
				A2=4,	 % 2. Antwort
				A3=0,  % 3. Antwort
				A4=0,  % 4. Antwort
				A5=0,  % 5. Antwort
				}
\end{beispiel}