\section{FA 4.1 - 5 - K6 - Grad der Polynomfunktionen - ZO - MarStr}

\begin{beispiel}[FA 4.1]{1}
Gegeben sind Ausschnitte von vier Polynomfunktionsgraphen, wobei jeweils alle Null–, Extrem– und Wendepunkte zu sehen sind. 

Ordne jedem Ausschnitt eines Funktionsgraphen die richtige Aussage über den kleinstmöglichen Grad $n$ der zugehörigen Polynomfunktion (aus A bis F) zu!

\zuordnen{
				R1={\psset{xunit=0.5cm,yunit=0.5cm,algebraic=true,dimen=middle,dotstyle=o,dotsize=5pt 0,linewidth=0.2pt,arrowsize=4pt 2,arrowinset=0.25}
\begin{pspicture*}(-3.9832409357854655,-2.6053160737909917)(3.986567090562436,4.368265949263422)
\begin{tiny}
\psaxes[labelFontSize=\scriptstyle, showorigin=false, xAxis=true,yAxis=true,Dx=1.,Dy=1.,labels=none,ticks=none,ticksize=-2pt 0,subticks=0]{->}(0,0)(-3.9832409357854655,-2.6053160737909917)(3.986567090562436,4.368265949263422)
\end{tiny}
\psplot[linewidth=1.pt,plotpoints=200]{-3.9832409357854655}{3.986567090562436}{0.23757196889798787*x^(3.0)+0.7743604507288552*x^(2.0)-0.5328052539262742*x-0.4431015801589173}
\end{pspicture*}},				% Response 1
				R2={\psset{xunit=0.5cm,yunit=0.5cm,algebraic=true,dimen=middle,dotstyle=o,dotsize=5pt 0,linewidth=0.2pt,arrowsize=4pt 2,arrowinset=0.25}
\begin{pspicture*}(-3.9832409357854655,-2.6053160737909917)(3.986567090562436,4.368265949263422)
\begin{tiny}
\psaxes[labelFontSize=\scriptstyle, showorigin=false, xAxis=true,yAxis=true,Dx=1.,Dy=1.,labels=none,ticks=none,ticksize=-2pt 0,subticks=0]{->}(0,0)(-3.9832409357854655,-2.6053160737909917)(3.986567090562436,4.368265949263422)
\end{tiny}
\psplot[linewidth=1pt,plotpoints=200]{-3.9832409357854655}{3.986567090562436}{2.0}
\end{pspicture*}},				% Response 2
				R3={\psset{xunit=0.5cm,yunit=0.5cm,algebraic=true,dimen=middle,dotstyle=o,dotsize=5pt 0,linewidth=0.2pt,arrowsize=4pt 2,arrowinset=0.25}
\begin{pspicture*}(-3.9832409357854655,-2.6053160737909917)(3.986567090562436,4.368265949263422)
\begin{tiny}
\psaxes[labelFontSize=\scriptstyle, showorigin=false, xAxis=true,yAxis=true,Dx=1.,Dy=1.,labels=none,ticks=none,ticksize=-2pt 0,subticks=0]{->}(0,0)(-3.9832409357854655,-2.6053160737909917)(3.986567090562436,4.368265949263422)
\end{tiny}
\psplot[linewidth=1pt,plotpoints=200]{-3.9832409357854655}{3.986567090562436}{0.09792291817322572*x^(4.0)+0.09707999845884406*x^(3.0)-0.7298174235170047*x^(2.0)-0.11308763088028295*x+1.093337505773384}
\end{pspicture*}},				% Response 3
				R4={\psset{xunit=0.5cm,yunit=0.5cm,algebraic=true,dimen=middle,dotstyle=o,dotsize=5pt 0,linewidth=0.2pt,arrowsize=4pt 2,arrowinset=0.25}
\begin{pspicture*}(-3.9832409357854655,-2.6053160737909917)(3.986567090562436,4.368265949263422)
\begin{tiny}
\psaxes[labelFontSize=\scriptstyle, showorigin=false, xAxis=true,yAxis=true,Dx=1.,Dy=1.,labels=none,ticks=none,ticksize=-2pt 0,subticks=0]{->}(0,0)(-3.9832409357854655,-2.6053160737909917)(3.986567090562436,4.368265949263422)
\end{tiny}
\psplot[linewidth=1pt,plotpoints=200]{-3.9832409357854655}{3.986567090562436}{-0.5*x^(2.0)+3.0}
\end{pspicture*}},				% Response 4
				%% Moegliche Zuordnungen: %%
				A={$n \geq 0$}, %Moeglichkeit A  
				B={$n \geq 3$},%Moeglichkeit B  
				C={$n=4$}, 				%Moeglichkeit C  
				D={$n=1$}, 				%Moeglichkeit D  
				E={$n > 4$}, 				%Moeglichkeit E  
				F={$n \geq 2$}, 				%Moeglichkeit F  
				%% LOESUNG: %%
				A1={B},				% 1. richtige Zuordnung
				A2={A},				% 2. richtige Zuordnung
				A3={C},				% 3. richtige Zuordnung
				A4={F},				% 4. richtige Zuordnung
				}
\end{beispiel}