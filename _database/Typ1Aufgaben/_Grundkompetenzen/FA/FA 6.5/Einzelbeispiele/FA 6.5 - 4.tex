\section{FA 6.5 - 4 - MAT - Winkelfunktionen - OA - Matura 2. NT 2017/18}

\begin{beispiel}[FA 6.5]{1}
In der unten stehenden Abbildung sind die Graphen der Funktionen $f$ und $g$ mit den Funktionsgleichungen $f(x)=\sin(x)$ und $g(x) = cos(x)$ dargestellt.
Für die in der Abbildung eingezeichneten Stellen $a$ und $b$ gilt: $\cos(a) = \sin(b)$.\leer

\begin{center}
\winkelfunktion\psset{xunit=2.0cm,yunit=1.5cm,trigLabels,algebraic=true,dimen=middle,dotstyle=o,dotsize=5pt 0,linewidth=0.8pt,arrowsize=3pt 2,arrowinset=0.25}
\begin{pspicture*}(-.5,-1.3)(4.5,1.5)
\begin{scriptsize}
\psaxes[trigLabelBase=2,xAxis=true,showorigin=false,yAxis=true,Dx=4,Dy=1.,ticksize=-2pt 0,subticks=0]{->}(0,0)(-1,-8)(4.5,1.5)[$x$,140] [$f(x)\text{, }g(x)$,-40]
\psplot[xunit=0.63661977cm,linewidth=0.8pt,plotpoints=200]{-10}{20}{SIN(0.5*x)}
\psplot[xunit=0.63661977cm,linewidth=0.8pt,plotpoints=200]{-10}{20}{COS(0.5*x)}

\psline[linewidth=0.8pt,linestyle=dashed,dash=1pt 1pt](1.69,0.)(1.69,-0.85)
\psline[linewidth=0.8pt,linestyle=dashed,dash=1pt 1pt](2.69,0.)(2.69,-0.85)

\rput[bl](1.63,0.1){$a$}
\rput[bl](2.63,0.1){$b$}
\rput[bl](3.7,-0.8){$f$}
\rput[bl](3.5,0.8){$g$}
\end{scriptsize}

\end{pspicture*}
\end{center}

Bestimme $k\in \mathbb{R}$ so, dass $b-a=k\cdot \pi$ gilt!

\antwort{$k=\frac{1}{2}$}
\end{beispiel}