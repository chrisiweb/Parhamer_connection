\section{FA 6.5 - 1 - Cosinusfunktion - OA - BIFIE}

\begin{beispiel}[FA 6.5]{1} %PUNKTE DES BEISPIELS
				Die Cosinusfunktion ist eine periodische Funktion.

Zeichne in der nachstehenden Abbildung die Koordinatenachsen und deren Skalierung so ein, dass der angegebene Graph dem Graphen der Cosinusfunktion entspricht! Die Skalierung beider Achsen muss jeweils zwei Werte umfassen!
\leer

\resizebox{1\linewidth}{!}{\psset{xunit=1.0cm,yunit=1.0cm,algebraic=true,trigLabels,dimen=middle,dotstyle=o,dotsize=5pt 0,linewidth=0.8pt,arrowsize=3pt 2,arrowinset=0.25}
\begin{pspicture*}(-3.9355818181818245,-2.483719400187443)(9.84441818181818,2.43705)
\multips(0,-2)(0,1.0){6}{\psline[linestyle=dashed,linecap=1,dash=1.5pt 1.5pt,linewidth=0.4pt,linecolor=lightgray]{c-c}(-3.9355818181818245,0)(9.84441818181818,0)}
\multips(-3,0)(1,0){13}{\psline[linestyle=dashed,linecap=1,dash=1.5pt 1.5pt,linewidth=0.4pt,linecolor=lightgray]{c-c}(0,-2.483719400187443)(0,2.43705)}
\antwort{\psaxes[labelFontSize=\scriptstyle,xAxis=true,yAxis=true,Dx=1.,Dy=1.,showorigin=false,trigLabelBase=2,ticksize=-2pt 0,subticks=0]{->}(0,0)(-3.9355818181818245,-2.483719400187443)(9.84441818181818,2.43705)[$x$,140] [$f(x)$,-40]}
\psplot[xunit=0.63661977cm,linewidth=1.2pt,plotpoints=200]{-13.9355818181818245}{19.84441818181818}{COS(x)}
\end{pspicture*}}
\end{beispiel}