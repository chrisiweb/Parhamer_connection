\section{FA 1.1 - 2 Reelle Funktion - MC - BIFIE}

\begin{beispiel}[FA 1.1]{1} %PUNKTE DES BEISPIELS
Eine reelle Funktion $f: [-3; 3]\rightarrow \mathbb{R}$ kann in einem Koordinatensystem als Graph dargestellt werden. 

\leer

Kreuze die beiden Diagramme an, die einen m�glichen Graphen der Funktion $f$ zeigen.

\multiplechoice[5]{  %Anzahl der Antwortmoeglichkeiten, Standard: 5
				L1={\resizebox{0.3\linewidth}{!}{\psset{xunit=1.0cm,yunit=1.0cm,algebraic=true,dimen=middle,dotstyle=o,dotsize=5pt 0,linewidth=0.8pt,arrowsize=3pt 2,arrowinset=0.25}
\begin{pspicture*}(-3.4209447215039956,-2.5648363038864415)(3.55129775839589,2.6942265952380398)
\psaxes[labelFontSize=\scriptstyle,xAxis=true,yAxis=true,Dx=1.,Dy=1.,ticksize=-2pt 0,subticks=2]{->}(0,0)(-3.4209447215039956,-2.5648363038864415)(3.55129775839589,2.6942265952380398)[x,140] [f(x),-40]
\rput{0.}(-0.004545906353054498,0.){\psellipse(0,0)(2.9834810591487266,1.6583524947006554)}
\rput[tl](-2.325306617519728,1.9372402688489094){$f$}
\end{pspicture*}}},   %1. Antwortmoeglichkeit 
				L2={\resizebox{0.3\linewidth}{!}{\psset{xunit=1.0cm,yunit=1.0cm,algebraic=true,dimen=middle,dotstyle=o,dotsize=5pt 0,linewidth=0.8pt,arrowsize=3pt 2,arrowinset=0.25}
\begin{pspicture*}(-3.5,-2.94)(3.44,2.94)
\psaxes[labelFontSize=\scriptstyle,xAxis=true,yAxis=true,Dx=1.,Dy=1.,ticksize=-2pt 0,subticks=2]{->}(0,0)(-3.5,-2.94)(3.44,2.94)[x,140] [f(x),-40]
\psplot[linewidth=1.2pt,plotpoints=200]{-3}{3}{0.1130972583234016*x^(3.0)+0.3307862988886959*x^(2.0)-1.0255164282445268*x-2.0}
\rput[tl](-2.88,1.5){$f$}
\end{pspicture*}}},   %2. Antwortmoeglichkeit
				L3={\resizebox{0.3\linewidth}{!}{\newrgbcolor{qqwuqq}{0. 0.39215686274509803 0.}
\psset{xunit=1.0cm,yunit=1.0cm,algebraic=true,dimen=middle,dotstyle=o,dotsize=5pt 0,linewidth=0.8pt,arrowsize=3pt 2,arrowinset=0.25}
\begin{pspicture*}(-3.4209447215039996,-2.8636466958821516)(3.8301874575918813,2.7938300592366088)
\psaxes[labelFontSize=\scriptstyle,xAxis=true,yAxis=true,Dx=1.,Dy=1.,ticksize=-2pt 0,subticks=2]{->}(0,0)(-3.4209447215039996,-2.8636466958821516)(3.8301874575918813,2.7938300592366088)[x,140] [f(x),-40]
\rput[tl](1.140893929630497,2.753988673637181){$f$}
\psline(1.,-2.8636466958821516)(1.,2.7938300592366088)
\pscustom[linecolor=qqwuqq,fillcolor=qqwuqq,fillstyle=solid,opacity=0.1]{
\parametricplot{0.0}{1.5707963267948966}{0.5976207839914188*cos(t)+1.|0.5976207839914188*sin(t)+0.}
\lineto(1.,0.)\closepath}
\psellipse*[linecolor=qqwuqq,fillcolor=qqwuqq,fillstyle=solid,opacity=1](1.2485774758460901,0.24857747584609008)(0.03984138559942792,0.03984138559942792)
\end{pspicture*}}},   %3. Antwortmoeglichkeit
				L4={\resizebox{0.3\linewidth}{!}{\psset{xunit=1.0cm,yunit=1.0cm,algebraic=true,dimen=middle,dotstyle=o,dotsize=5pt 0,linewidth=0.8pt,arrowsize=3pt 2,arrowinset=0.25}
\begin{pspicture*}(-3.4209447215039996,-2.8636466958821516)(3.8301874575918813,2.7938300592366088)
\psaxes[labelFontSize=\scriptstyle,xAxis=true,yAxis=true,Dx=1.,Dy=1.,ticksize=-2pt 0,subticks=2]{->}(0,0)(-3.4209447215039996,-2.8636466958821516)(3.8301874575918813,2.7938300592366088)[x,140] [f(x),-40]
\rput[tl](-2.1260996895225923,1.937240268848909){$f$}
\psline(1.4197836288264922,0.32366415207207944)(-3.0026101727100065,1.3993815632566324)
\psline(0.2843041392427967,2.275892046444046)(2.993518360003895,-0.7321325663127597)
\end{pspicture*}}},   %4. Antwortmoeglichkeit
				L5={\resizebox{0.3\linewidth}{!}{\psset{xunit=1.0cm,yunit=1.0cm,algebraic=true,dimen=middle,dotstyle=o,dotsize=5pt 0,linewidth=0.8pt,arrowsize=3pt 2,arrowinset=0.25}
\begin{pspicture*}(-3.4209447215039996,-2.8636466958821516)(3.7106633007935974,2.7938300592366088)
\psaxes[labelFontSize=\scriptstyle,xAxis=true,yAxis=true,Dx=1.,Dy=1.,ticksize=-2pt 0,subticks=2]{->}(0,0)(-3.4209447215039996,-2.8636466958821516)(3.7106633007935974,2.7938300592366088)[x,140] [f(x),-40]
\rput[tl](-2.5842756239160134,2.554781745640041){$f$}
\psline(-1.,-1.)(-3.0026101727100065,2.574702438439755)
\psline(-1.,-1.)(3.,0.32366)
\begin{scriptsize}
\rput[bl](1.081131851231355,-0.6325291023141899){$j$}
\end{scriptsize}
\end{pspicture*}}},	 %5. Antwortmoeglichkeit
				L6={},	 %6. Antwortmoeglichkeit
				L7={},	 %7. Antwortmoeglichkeit
				L8={},	 %8. Antwortmoeglichkeit
				L9={},	 %9. Antwortmoeglichkeit
				%% LOESUNG: %%
				A1=2,  % 1. Antwort
				A2=5,	 % 2. Antwort
				A3=0,  % 3. Antwort
				A4=0,  % 4. Antwort
				A5=0,  % 5. Antwort
				}
\end{beispiel}