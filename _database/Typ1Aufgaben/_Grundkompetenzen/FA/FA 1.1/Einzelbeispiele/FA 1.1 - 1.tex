\section{FA 1.1 - 1 - Funktionsgraph - MC - BIFIE}

\begin{beispiel}[FA 1.1]{1} %PUNKTE DES BEISPIELS
Im Folgenden sind Darstellungen von Kurven und Geraden gegeben.
\leer

Kreuze diejenige(n) Abbildung(en) an, die Graph(en) einer Funktion \mbox{$f: x\rightarrow f(x)$} ist/sind!

\langmultiplechoice[5]{  %Anzahl der Antwortmoeglichkeiten, Standard: 5
				L1={\psset{xunit=0.6cm,yunit=0.6cm,algebraic=true,dimen=middle,dotstyle=o,dotsize=5pt 0,linewidth=0.8pt,arrowsize=3pt 2,arrowinset=0.25}
\begin{pspicture*}(-1.54,-1.5)(7.62,5.62)
\psaxes[labelFontSize=\scriptstyle,xAxis=true,yAxis=true,Dx=1.,Dy=1.,showorigin=false,ticksize=-2pt 0,subticks=0]{->}(0,0)(-1.54,-2.32)(7.62,5.62)
\psplot[plotpoints=200]{-1.5400000000000018}{7.6199999999999966}{0.004464285714285714*x^(4.0)-0.019345238095238096*x^(3.0)-0.3273809523809524*x^(2.0)+1.6964285714285714*x+2.0}
\begin{scriptsize}
\rput[tl](3.5,4.52){$f$}
\end{scriptsize}
\end{pspicture*}},   %1. Antwortmoeglichkeit 
				L2={\psset{xunit=0.75cm,yunit=0.7cm,algebraic=true,dimen=middle,dotstyle=o,dotsize=5pt 0,linewidth=0.8pt,arrowsize=3pt 2,arrowinset=0.25}
\begin{pspicture*}(-2.56,-3.38)(4.54,3.52)
\psaxes[labelFontSize=\scriptstyle,xAxis=true,yAxis=true,Dx=1.,Dy=1.,showorigin=false,ticksize=-2pt 0,subticks=0]{->}(0,0)(-2.56,-3.38)(4.54,3.52)
\begin{scriptsize}
\rput[tl](1.76,3.3){$f$}
\end{scriptsize}
\rput{0.}(0.01,0.){\psellipse(0,0)(4.002818528382616,3.0000093618459216)}
\end{pspicture*}},   %2. Antwortmoeglichkeit
				L3={\psset{xunit=0.75cm,yunit=0.7cm,algebraic=true,dimen=middle,dotstyle=o,dotsize=5pt 0,linewidth=0.8pt,arrowsize=3pt 2,arrowinset=0.25}
\begin{pspicture*}(-2.26,-0.8)(4.42,5.4)
\psaxes[labelFontSize=\scriptstyle,xAxis=true,yAxis=true,Dx=1.,Dy=1.,showorigin=false,ticksize=-2pt 0,subticks=0]{->}(0,0)(-2.26,-0.8)(4.42,5.4)
\begin{scriptsize}
\rput[tl](2.06,3.18){$f$}
\end{scriptsize}
\psline(2.,-0.8)(2.,5.4)
\end{pspicture*}},   %3. Antwortmoeglichkeit
				L4={\psset{xunit=0.7cm,yunit=0.7cm,algebraic=true,dimen=middle,dotstyle=o,dotsize=5pt 0,linewidth=0.8pt,arrowsize=3pt 2,arrowinset=0.25}
\begin{pspicture*}(-2.26,-0.8)(4.42,5.4)
\psaxes[labelFontSize=\scriptstyle,xAxis=true,yAxis=true,Dx=1.,Dy=1.,showorigin=false,ticksize=-2pt 0,subticks=0]{->}(0,0)(-2.26,-0.8)(4.42,5.4)
\begin{scriptsize}
\rput[tl](2.08,2.86){$f$}
\end{scriptsize}
\psplot{-2.26}{4.42}{(--2.-0.*x)/1.}
\end{pspicture*}},   %4. Antwortmoeglichkeit
				L5={\psset{xunit=0.7cm,yunit=0.7cm,algebraic=true,dimen=middle,dotstyle=o,dotsize=5pt 0,linewidth=0.8pt,arrowsize=3pt 2,arrowinset=0.25}
\begin{pspicture*}(-1.367814021398395,-1.3041748207011512)(5.445940009284722,4.578439189711757)
\psaxes[labelFontSize=\scriptstyle,xAxis=true,yAxis=true,Dx=1.,Dy=1.,showorigin=false,ticksize=-2pt 0,subticks=0]{->}(0,0)(-1.367814021398395,-1.3041748207011512)(5.445940009284722,4.578439189711757)
\psplot[plotpoints=200]{-1.367814021398395}{2}{x}
\psplot[plotpoints=200]{2}{5.445940009284722}{0.5*x+1.0}
\begin{scriptsize}
\rput[tl](2.9373392487920627,3.1){$f$}
\end{scriptsize}
\end{pspicture*}},	 %5. Antwortmoeglichkeit
				L6={},	 %6. Antwortmoeglichkeit
				L7={},	 %7. Antwortmoeglichkeit
				L8={},	 %8. Antwortmoeglichkeit
				L9={},	 %9. Antwortmoeglichkeit
				%% LOESUNG: %%
				A1=1,  % 1. Antwort
				A2=4,	 % 2. Antwort
				A3=5,  % 3. Antwort
				A4=0,  % 4. Antwort
				A5=0,  % 5. Antwort
				}
\end{beispiel}