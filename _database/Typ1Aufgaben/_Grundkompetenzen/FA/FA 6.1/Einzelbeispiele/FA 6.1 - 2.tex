\section{FA 6.1 - 2 Graphen von Winkelfunktionen - ZO - BIFIE}

\begin{beispiel}[FA 6.1]{1} %PUNKTE DES BEISPIELS
				Die nachstehende Abbildung zeigt die Graphen der Funktionen $f_1,f_2,f_3$ und $f_4$.

\resizebox{1\linewidth}{!}{\psset{xunit=1.0cm,yunit=1.0cm,algebraic=true,trigLabels,dimen=middle,dotstyle=o,dotsize=5pt 0,linewidth=0.8pt,arrowsize=3pt 2,arrowinset=0.25}
\begin{pspicture*}(-1.5,-3.483719400187443)(6.85,3.43705)
\multips(0,-3)(0,1.0){8}{\psline[linestyle=dashed,linecap=1,dash=1.5pt 1.5pt,linewidth=0.4pt,linecolor=lightgray]{c-c}(-3.9355818181818245,0)(9.84441818181818,0)}
\multips(-3,0)(1,0){13}{\psline[linestyle=dashed,linecap=1,dash=1.5pt 1.5pt,linewidth=0.4pt,linecolor=lightgray]{c-c}(0,-3.483719400187443)(0,3.43705)}
\psaxes[labelFontSize=\scriptstyle,xAxis=true,yAxis=true,Dx=1.,Dy=1.,trigLabelBase=2,ticksize=-2pt 0,subticks=2]{->}(0,0)(-1.5,-3.483719400187443)(6.85,3.43705)[x,140] [f(x),-40]
\psplot[xunit=0.63661977cm,linestyle=dotted,linewidth=1.2pt,plotpoints=200]{-13.9355818181818245}{19.84441818181818}{COS(x)}
\rput[tl](-1.5,-0.7){\tiny{$f_3$}}
\psplot[xunit=0.63661977cm,linewidth=1.2pt,plotpoints=200]{-13.9355818181818245}{19.84441818181818}{0.5*SIN(x)}
\rput[tl](-0.5,-0.4){\tiny{$f_4$}}
\psplot[xunit=0.63661977cm,linestyle=dashed,linewidth=1.2pt,plotpoints=200]{-13.9355818181818245}{19.84441818181818}{3*COS(x)}
\rput[tl](-1.4,-2.1){\tiny{$f_1$}}
\psplot[xunit=0.63661977cm,linestyle=dashed,dash=2pt 2pt,linewidth=1.2pt,plotpoints=200]{-13.9355818181818245}{19.84441818181818}{-2*SIN(x)}
\rput[tl](-1.4,1.6){\tiny{$f_2$}}
\end{pspicture*}}
\leer

Ordne den vier dargestellten Funktionsgraphen jeweils die passende Funktionsgleichung zu!

\zuordnen{
				title1={Funktionsgleichung}, 		%Titel Antwortmoeglichkeiten
				A={$\sin(2x)$}, 				%Moeglichkeit A  
				B={$-2\cdot\sin(x)$}, 				%Moeglichkeit B  
				C={$\frac{1}{2}\cdot\sin(x)$}, 				%Moeglichkeit C  
				D={$\cos(x)$}, 				%Moeglichkeit D  
				E={$\cos(\frac{x}{2})$}, 				%Moeglichkeit E  
				F={$3\cdot\cos(x)$}, 				%Moeglichkeit F  
				title2={Funktionsgraphen},		%Titel Zuordnung
				R1={$f_1$},				%1. Antwort rechts
				R2={$f_2$},				%2. Antwort rechts
				R3={$f_3$},				%3. Antwort rechts
				R4={$f_4$},				%4. Antwort rechts
				%% LOESUNG: %%
				A1={F},				% 1. richtige Zuordnung
				A2={B},				% 2. richtige Zuordnung
				A3={D},				% 3. richtige Zuordnung
				A4={C},				% 4. richtige Zuordnung
				}
\end{beispiel}