\section{FA 1.7 - 4 Räuber-Beute-Modell - OA - Matura 2016/17 - Haupttermin}

\begin{beispiel}[FA 1.7]{1} %PUNKTE DES BEISPIELS
Das Räuber-Beute-Modell zeigt vereinfacht Populationsschwankungen einer Räuberpopulation
(z.B. der Anzahl von Kanadischen Luchsen) und einer Beutepopulation (z.B. der Anzahl von
Schneeschuhhasen). Die in der unten stehenden Grafik abgebildeten Funktionen $R$ und $B$ beschreiben modellhaft die Anzahl der Räuber $R(t)$ bzw. die Anzahl der Beutetiere $B(t)$ für einen beobachteten Zeitraum von 24 Jahren ($B(t)$, $R(t)$ in 10000 Individuen, $t$ in Jahren). 

\begin{center}
\resizebox{0.7\linewidth}{!}{
\psset{xunit=0.5cm,yunit=1cm,algebraic=true,dimen=middle,dotstyle=o,dotsize=5pt 0,linewidth=0.8pt,arrowsize=3pt 2,arrowinset=0.25}
\begin{pspicture*}(-1.0665933309009596,-0.6602356117642768)(25.87549958994915,14.492128671727498)
\multips(0,0)(0,1.0){16}{\psline[linestyle=dashed,linecap=1,dash=1.5pt 1.5pt,linewidth=0.4pt,linecolor=black!70]{c-c}(0,0)(25.87549958994915,0)}
\multips(0,0)(1.0,0){27}{\psline[linestyle=dashed,linecap=1,dash=1.5pt 1.5pt,linewidth=0.4pt,linecolor=black!70]{c-c}(0,0)(0,14.492128671727498)}
\psaxes[labelFontSize=\scriptstyle,xAxis=true,yAxis=true,Dx=1.,Dy=1.,ticksize=-2pt 0,subticks=2]{->}(0,0)(0.,0.)(25.87549958994915,14.492128671727498)[$t$,140] [$B(t)\text{, }R(t)$,-40]
\psplot[plotpoints=300]{0}{24}{2.785321060047514E-10*x^(10.0)-2.2504378492410093E-8*x^(9.0)+4.867803181775444E-7*x^(8.0)+7.030065912963373E-6*x^(7.0)-4.7093866521687503E-4*x^(6.0)+0.007770498841685383*x^(5.0)-0.05262022074081365*x^(4.0)+0.10837921624621992*x^(3.0)+0.06300942000338298*x^(2.0)+0.5847305705155729*x+1.451663605014681}
\psplot[plotpoints=300]{0}{24}{3.2132513282586096E-9*x^(9.0)-1.0673312154762772E-7*x^(8.0)-8.837538434234892E-6*x^(7.0)+6.102079037291394E-4*x^(6.0)-0.01471067310678362*x^(5.0)+0.16625810315615447*x^(4.0)-0.8408544386931532*x^(3.0)+1.3330173311751785*x^(2.0)+0.31662191392895234*x+7.500306992876758}
\rput[tl](15.295499630163263,11.66259379876151){B}
\rput[tl](12.137902742940343,4.9373225064945245){R}
\end{pspicture*}}
\end{center}
Gib alle Zeitintervalle im dargestellten Beobachtungszeitraum an, in denen sowohl die
Räuberpopulation als auch die Beutepopulation abnimmt! 

\antwort{

In den beiden Zeitintervallen [4,2 Jahre; 6,8 Jahre] und [15,3 Jahre; 19,6 Jahre] nimmt sowohl die Räuberpopulation als auch die Beutepopulation ab. 

\begin{scriptsize}
Lösungsschlüssel:\\
Andere Schreibweisen der Intervalle (offen oder halboffen) sowie korrekte formale oder verbale Beschreibungen sind ebenfalls als richtig zu werten.

\textbf{1. Zeitintervall:}
Toleranzintervall: [3,9 Jahre; 4,5 Jahre] und [6,5 Jahre; 7,1 Jahre]

\textbf{2. Zeitintervall:}
Toleranzintervall: [15 Jahre; 15,6 Jahre] und [19,3 Jahre; 19,9 Jahre]
\end{scriptsize}}
\end{beispiel}