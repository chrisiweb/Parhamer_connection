\section{FA 1.7 - 3 - MAT - Lorenz-Kurve - MC - Matura 2014/15 - Haupttermin}

\begin{beispiel}[FA 1.7]{1} %PUNKTE DES BEISPIELS
Die in der unten stehenden Abbildung dargestellte Lorenz-Kurve kann als Graph einer Funktion $f$
verstanden werden, die gewissen Bevölkerungsanteilen deren jeweiligen Anteil am Gesamteinkommen
zuordnet. \leer

Dieser Lorenz-Kurve kann man z.B. entnehmen, dass die einkommensschwächsten 80\,\% der
Bevölkerung über ca. 43\,\% des Gesamteinkommens verfügen. Das bedeutet zugleich, dass die
einkommensstärksten 20\,\% der Bevölkerung über ca. 57\,\% des Gesamteinkommens verfügen. 

\begin{center}
\resizebox{0.45\linewidth}{!}{
\psset{xunit=.1cm,yunit=0.1cm,algebraic=true,dimen=middle,dotstyle=o,dotsize=5pt 0,linewidth=0.8pt,arrowsize=3pt 2,arrowinset=0.25}
\begin{pspicture*}(-11,-11.811255228840242)(105.0526625921452,105.42050241751232)
\multips(0,0)(0,10.0){12}{\psline[linestyle=dashed,linecap=1,dash=1.5pt 1.5pt,linewidth=0.4pt,linecolor=gray]{c-c}(0,0)(105.0526625921452,0)}
\multips(0,0)(10.0,0){12}{\psline[linestyle=dashed,linecap=1,dash=1.5pt 1.5pt,linewidth=0.4pt,linecolor=gray]{c-c}(0,0)(0,105.42050241751232)}
\psaxes[labelFontSize=\scriptstyle,xAxis=true,yAxis=true,Dx=10.,Dy=10.,ticksize=-2pt 0,subticks=2]{->}(0,0)(0.,0.)(105.0526625921452,105.42050241751232)
\psline(0.,100.)(100.,100.)
\psline(100.,100.)(100.,0.)
\parametricplot[linewidth=1.2pt]{-1.5707963267948966}{0.0}{1.*100.*cos(t)+0.*100.*sin(t)+0.|0.*100.*cos(t)+1.*100.*sin(t)+100.}
\rput[tl](-11,73.13131721288313){$\rotatebox{90.0}{\normalsize \text{Anteil Gesamteinkommen in \%}}$}
\rput[tl](32,-7){$\rotatebox{0.0}{ \small \text{Bevölkerungsanteil in \% }  }$}
\rput[tl](70,25){Lorenz-Kurve $f$}
\end{pspicture*}}
\end{center}
\begin{flushleft} \vspace{-1.5cm}
\scalebox{0.9}{\tiny Quelle: http://www.lai.fu-berlin.de/e-learning/projekte/vwl\_basiswissen/Umverteilung/Gini\_Koeffizient/index.html [21.01.2015] (adaptiert)}
\end{flushleft}

Kreuze die beiden für die oben dargestellte Lorenz-Kurve zutreffenden Aussagen an.
\multiplechoice[5]{  %Anzahl der Antwortmoeglichkeiten, Standard: 5
				L1={Die einkommensstärksten 10\,\% der Bevölkerung verfügen über ca. 60\,\%
des Gesamteinkommens.},   %1. Antwortmoeglichkeit 
				L2={Die einkommensstärksten 40\,\% der Bevölkerung verfügen über ca. 90\,\%
des Gesamteinkommens.},   %2. Antwortmoeglichkeit
				L3={Die einkommensschwächsten 40\,\% der Bevölkerung verfügen über ca. 10\,\%
des Gesamteinkommens.},   %3. Antwortmoeglichkeit
				L4={Die einkommensschwächsten 60\,\% der Bevölkerung verfügen über ca. 90\,\%
des Gesamteinkommens.},   %4. Antwortmoeglichkeit
				L5={Die einkommensschwächsten 90\,\% der Bevölkerung verfügen über ca. 60\,\%
des Gesamteinkommens.},	 %5. Antwortmoeglichkeit
				L6={},	 %6. Antwortmoeglichkeit
				L7={},	 %7. Antwortmoeglichkeit
				L8={},	 %8. Antwortmoeglichkeit
				L9={},	 %9. Antwortmoeglichkeit
				%% LOESUNG: %%
				A1=3,  % 1. Antwort
				A2=5,	 % 2. Antwort
				A3=0,  % 3. Antwort
				A4=0,  % 4. Antwort
				A5=0,  % 5. Antwort
				} 

\end{beispiel}