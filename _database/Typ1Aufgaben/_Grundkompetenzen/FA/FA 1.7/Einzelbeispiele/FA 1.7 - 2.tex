\section{FA 1.7 - 2 - Schulweg - ZO - BIFIE}

\begin{beispiel}[FA 1.7]{1} %PUNKTE DES BEISPIELS
Die grafische Darstellung veranschaulicht die Erzählung von einem Schulweg. Die zurückgelegte Strecke s (in m) wird dabei in Abhängigkeit von der Zeit t (in min) dargestellt.

Gib an, welche Abschnitte des Schulwegs den Teilen
des Funktionsgraphen entsprechen! Ordnen Sie dazu den
Textstellen die passenden Abschnitte (Intervalle) des Funktionsgraphen
zu. 
\begin{center}
\psset{xunit=0.2cm,yunit=0.002cm,algebraic=true,dimen=middle,dotstyle=o,dotsize=5pt 0,linewidth=0.8pt,arrowsize=3pt 2,arrowinset=0.25}
\begin{pspicture*}(-4.122239574211525,-414.43064773889193)(51,5404.240795870296)
\multips(0,0)(0,100.0){59}{\psline[linestyle=dashed,linecap=1,dash=1.5pt 1.5pt,linewidth=0.4pt,linecolor=lightgray]{c-c}(0,0)(55.595704189146154,0)}
\multips(0,0)(1.0,0){60}{\psline[linestyle=dashed,linecap=1,dash=1.5pt 1.5pt,linewidth=0.4pt,linecolor=lightgray]{c-c}(0,0)(0,5404.240795870296)}
\psaxes[labelFontSize=\scriptstyle,xAxis=true,yAxis=true,Dx=5.,Dy=500.,ticksize=-2pt 0,subticks=0]{->}(0,0)(0.,0.)(51,5404.240795870296)[t in min,140] [S in m,-40]
\psline[linewidth=1.6pt](0.,0.)(10.,400.)
\psline[linewidth=1.6pt](10.,400.)(25.,1400.)
\psline[linewidth=1.6pt](25.,1400.)(30.,1400.)
\psline[linewidth=1.6pt](30.,1400.)(43.,4750.)
\psline[linewidth=1.6pt](43.,4750.)(49.,4900.)
\end{pspicture*}
\end{center}


\zuordnen[0.30]{
				title1={Intervalle}, 		%Titel Antwortmoeglichkeiten
				A={$[0;10]$}, 				%Moeglichkeit A  
				B={$[0;25]$}, 				%Moeglichkeit B  
				C={$[10;25]$}, 				%Moeglichkeit C  
				D={$[25;30]$}, 				%Moeglichkeit D  
				E={$[30;43]$}, 				%Moeglichkeit E  
				F={$[43;49]$}, 				%Moeglichkeit F  
				title2={Textstellen},		%Titel Zuordnung
				R1={Mit dem Bus bin ich etwas mehr als
10 Minuten gefahren.},				%1. Antwort rechts
				R2={Ich bemerkte, dass ich zu spät zur Busstation
kommen werde, daher bin ich etwas
schneller gegangen.},				%2. Antwort rechts
				R3={Auf den letzten Metern zur Schule habe ich
mit meinen Freundinnen geredet.},				%3. Antwort rechts
				R4={Ich musste noch auf den Bus warten.},				%4. Antwort rechts
				%% LOESUNG: %%
				A1={E},				% 1. richtige Zuordnung
				A2={C},				% 2. richtige Zuordnung
				A3={F},				% 3. richtige Zuordnung
				A4={D},				% 4. richtige Zuordnung
				}
\end{beispiel}