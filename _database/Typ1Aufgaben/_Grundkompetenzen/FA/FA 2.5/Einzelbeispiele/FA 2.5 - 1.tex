\section{FA 2.5 - 1 Modellierung mittels linearer Funktionen - MC - BIFIE}

\begin{beispiel}[FA 2.5]{1} %PUNKTE DES BEISPIELS
Reale Sachverhalte können durch eine lineare Funktion $f(x)=k\cdot x+d$ mathematisch modelliert werden.

In welchem Sachverhalt ist eine Modellierung mittels einer linearen Funktion sinnvoll möglich? Kreuze die beiden zutreffenden Sachverhalte an!

\multiplechoice[5]{  %Anzahl der Antwortmoeglichkeiten, Standard: 5
				L1={der zurückgelegte Weg in Abhängigkeit von der Zeit bei einer gleichbleibenden Geschwindigkeit von $30\,km/h$},   %1. Antwortmoeglichkeit 
				L2={die Einwohnerzahl einer Stadt in Abhängigkeit von der Zeit, wenn die Anzahl der Einwohner/innen in einem bestimmten Zeitraum jährlich um $3\,\%$ wächst},   %2. Antwortmoeglichkeit
				L3={Der Flächeninhalt eines Quadrates in Abhängigkeit von der Seitenlänge},   %3. Antwortmoeglichkeit
				L4={Die Stromkosten in Abhängigkeit von der verbrauchten Energie (in kWh) bei einer monatlichen Grundgebühr von \euro $12$ und Kosten von \euro $0,4$ pro kWh},   %4. Antwortmoeglichkeit
				L5={die Fahrzeit in Abhängigkeit von der Geschwindigkeit für eine bestimmte Entfernung},	 %5. Antwortmoeglichkeit
				L6={},	 %6. Antwortmoeglichkeit
				L7={},	 %7. Antwortmoeglichkeit
				L8={},	 %8. Antwortmoeglichkeit
				L9={},	 %9. Antwortmoeglichkeit
				%% LOESUNG: %%
				A1=1,  % 1. Antwort
				A2=4,	 % 2. Antwort
				A3=0,  % 3. Antwort
				A4=0,  % 4. Antwort
				A5=0,  % 5. Antwort
				}
\end{beispiel}