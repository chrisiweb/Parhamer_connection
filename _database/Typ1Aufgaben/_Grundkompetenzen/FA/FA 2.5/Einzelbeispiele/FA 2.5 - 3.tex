\section{FA 2.5 - 3 - MAT - Bruttogehalt und Nettogehalt - OA - Matura 2018/19 2. NT}

\begin{beispiel}[FA 2.5]{1}
Auf der Website des Finanzministeriums findet man einen Brutto-Netto Rechner, der für jedes monatliche Bruttogehalt das entsprechende Nettogehalt berechnet.

Folgende Tabelle gibt Auskunft über einige Gehälter:

\begin{center}
\begin{tabular}{|l|c|c|c|}\hline
Bruttogehalt in \euro &1\,500&2\,000&2\,500\\ \hline
Nettogehalt in \euro &1\,199&1\,483&1\,749\\ \hline
\end{tabular}
\end{center}

Zeige unter Verwendung der in der obigen Tabelle angeführten Werte, dass zwischen dem Bruttogehalt und dem Nettogehalt kein linearer Zusammenhang besteht.

\antwort{Mögliche Vorgehensweise:

Es besteht kein linearer Zusammenhang, da die gleiche Zunahme des Bruttogehalts (jeweils \euro 500) nicht die gleiche Erhöhung des Nettogehalts (\euro 284 bzw. \euro 266) bewirkt.}
\end{beispiel}