\section{FA 6.4 - 1 - Atemzyklus - OA - BIFIE}

\begin{beispiel}[FA 6.4]{1} %PUNKTE DES BEISPIELS
				Der Luftstrom beim Ein- und Ausatmen einer Person im Ruhezustand ändert sich in Abhängigkeit von der Zeit nach einer Funktion $f$. Zum Zeitpunkt $t=0$ beginnt ein Atemzyklus. $f(t)$ ist die bewegte Luftmenge in Litern pro Sekunde zum Zeitpunkt $t$ in Sekunden und wird durch die Gleichung \begin{center}$f(t)=0,5\cdot\sin(0,4\cdot\pi\cdot t)$\end{center} festgelegt.

\begin{tiny} (Quelle: Timischl, W. (1995). Biomathematik: Eine Einführung für Biologen und Mediziner. 2. Auflage. Wien u.a.: Springer.)\end{tiny}

Berechne die Dauer eines gesamten Atemzyklus!

\antwort{Periodenlänge: $2\cdot\pi=0,4\cdot\pi\cdot t,\, t=5$

Ein Atmenzyklus dauert fünf Sekunden. Im Zeitintervall $[0;2,5]$ wird eingeatmetet, von 2,5 bis 5 Sekunden wird ausgeatmet.}
\end{beispiel}