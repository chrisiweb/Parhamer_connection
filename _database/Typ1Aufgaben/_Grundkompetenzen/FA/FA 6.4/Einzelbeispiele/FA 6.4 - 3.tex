\section{FA 6.4 - 3 Periodische Funktion - OA - Matura 2015/16 - Nebentermin 1}

\begin{beispiel}[FA 6.4]{1} %PUNKTE DES BEISPIELS
Gegeben ist die periodische Funktion $f$ mit der Funktionsgleichung $f(x) = \sin(x)$. \leer

Gib die kleinste Zahl $a>0$ (Maßzahl für den Winkel in Radiant) so an, dass für alle $x\in \mathbb{R}$ die Gleichung $f(x+a)=f(x)$ gilt. \leer

$a=\rule{2cm}{0.3pt}$\,rad

\antwort{$a=2\cdot \pi$\,rad \leer

Toleranzintervall: $[6,2\,$rad$;~ 6,3\,$rad$]$}

\end{beispiel}