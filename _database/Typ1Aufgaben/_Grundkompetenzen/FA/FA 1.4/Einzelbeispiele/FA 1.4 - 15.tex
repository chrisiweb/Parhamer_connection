\section{FA 1.4 - 15 - MAT - Gewinnfunktion - OA - Matura 2018/19 2. NT}

\begin{beispiel}[FA 1.4]{1}
Die unten stehende Abbildung zeigt eine lineare Kostenfunktion $K$: $x\rightarrow K(x)$ und eine lineare Erlösfunktion $E$: $x\rightarrow E(x)$ mit $x\in [0;6]$.

Für die Gewinnfunktion $G$: $x\rightarrow G(x)$ gilt für alle $x\in [0;6]$: $G(x)=E(x)-K(x)$.

Zeichne in der nachstehenden Abbildung den Graphen von $G$ ein.

\begin{center}
\psset{xunit=1.0cm,yunit=1.0cm,algebraic=true,dimen=middle,dotstyle=o,dotsize=5pt 0,linewidth=1.6pt,arrowsize=3pt 2,arrowinset=0.25}
\begin{pspicture*}(-0.6,-8.46)(6.7,8.6)
\multips(0,-8)(0,1.0){18}{\psline[linestyle=dashed,linecap=1,dash=1.5pt 1.5pt,linewidth=0.4pt,linecolor=gray]{c-c}(0,0)(6.7,0)}
\multips(0,0)(1.0,0){8}{\psline[linestyle=dashed,linecap=1,dash=1.5pt 1.5pt,linewidth=0.4pt,linecolor=gray]{c-c}(0,-8.46)(0,8.6)}
\psaxes[labelFontSize=\scriptstyle,showorigin=false,xAxis=true,yAxis=true,Dx=1.,Dy=1.,ticksize=-2pt 0,subticks=0]{->}(0,0)(0.,-10)(6.7,8.6)[$x$,140] [\text{$K(x), E(x), G(x)$},-40]
\psplot[linewidth=2.pt]{0}{6.7}{(--15.--3.*x)/5.}
\psplot[linewidth=2.pt]{0}{6.7}{(-0.--6.*x)/5.}
\rput[tl](2.2,4.94){$K$}
\rput[tl](3.26,3.86){$E$}
\antwort{\rput[tl](3.36,-1.16){$G$}
\psplot[linewidth=2.pt,plotpoints=200]{0}{6.700000000000003}{1.2*x-(0.6*x+3.0)}}
\end{pspicture*}
\end{center}

\antwort{Lösungsschlüssel:

Ein Punkt für die Darstellung des Graphen der Funktion $G$, wobei $G$ eine lineare Funktion sein
muss, deren Graph durch die beiden Punkte $(0\mid -3)$ und $(5\mid 0)$ verläuft.}
\end{beispiel}