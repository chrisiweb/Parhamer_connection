\section{FA 1.4 - 10 Chemisches Experiment - OA - BIFIE}

\begin{beispiel}[FA 1.4]{1} %PUNKTE DES BEISPIELS
In der nachstehenden Grafik wird der Temperaturverlauf (T in �C) eines chemischen Experiments innerhalb der ersten 8 Minuten ann�hernd wiedergegeben.

\begin{center}
\resizebox{0.9\linewidth}{!}{
\psset{xunit=2.0cm,yunit=0.5cm,algebraic=true,dimen=middle,dotstyle=o,dotsize=5pt 0,linewidth=0.8pt,arrowsize=3pt 2,arrowinset=0.25}
\begin{pspicture*}(-0.4133086038643893,-1.4092020596083639)(8.356454160613545,35.42693921229549)
\multips(0,0)(0,2.0){19}{\psline[linestyle=dashed,linecap=1,dash=1.5pt 1.5pt,linewidth=0.4pt,linecolor=gray]{c-c}(0,0)(8.356454160613545,0)}
\multips(0,0)(0.5,0){18}{\psline[linestyle=dashed,linecap=1,dash=1.5pt 1.5pt,linewidth=0.4pt,linecolor=gray]{c-c}(0,0)(0,35.42693921229549)}
\psaxes[labelFontSize=\scriptstyle,xAxis=true,yAxis=true,Dx=0.5,Dy=2.,ticksize=-2pt 0,subticks=0]{->}(0,0)(0.,0.)(8.356454160613545,35.42693921229549)[t,140] [T,-40]
\psplot[linewidth=1.2pt,plotpoints=200]{0}{8.356454160613545}{0.25*x^(3.0)-2.75*x^(2.0)+6.5*x+26.0}
\end{pspicture*}}
\end{center}

Bestimme die Werte $T(1)$ und $T(3,5)$ m�glichst genau und erkl�re in Worten, was durch diese Werte bestimmt wird! 

\antwort{$T(1) = 30^\circ$, $T(3,5) \approx 25,8^\circ$ \\
L�sungsintervall f�r $T(3,5): [25,5^\circ; 26^\circ]$ \\
$T(1)$ gibt die Temperatur nach einer Minute an, $T(3,5)$ gibt die Temperatur nach 3,5 Minuten an}
\end{beispiel}