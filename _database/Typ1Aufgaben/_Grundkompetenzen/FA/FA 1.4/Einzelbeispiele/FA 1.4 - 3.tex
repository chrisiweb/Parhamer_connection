\section{FA 1.4 - 3 - Argument bestimmen - OA - BIFIE}

\begin{beispiel}[FA 1.4]{1} %PUNKTE DES BEISPIELS
Gegeben ist eine Polynomfunktion dritten Grades durch ihren Funktionsgraphen:

\begin{center}
\psset{xunit=1.0cm,yunit=1.0cm,algebraic=true,dimen=middle,dotstyle=o,dotsize=5pt 0,linewidth=0.8pt,arrowsize=3pt 2,arrowinset=0.25}
\begin{pspicture*}(-3.520548185502567,-3.4612674798735723)(6.539401678352975,3.570737078425455)
\multips(0,-3)(0,1.0){8}{\psline[linestyle=dashed,linecap=1,dash=1.5pt 1.5pt,linewidth=0.4pt,linecolor=gray]{c-c}(-3.520548185502567,0)(6.539401678352975,0)}
\multips(-3,0)(1.0,0){11}{\psline[linestyle=dashed,linecap=1,dash=1.5pt 1.5pt,linewidth=0.4pt,linecolor=gray]{c-c}(0,-3.4612674798735723)(0,3.570737078425455)}
\psaxes[labelFontSize=\scriptstyle,xAxis=true,yAxis=true,Dx=1.,Dy=1.,showorigin=false,ticksize=-2pt 0,subticks=0]{->}(0,0)(-3.520548185502567,-3.4612674798735723)(6.539401678352975,3.570737078425455)[x,140] [y,-40]
\psplot[linewidth=0.4pt,plotpoints=200]{-3.520548185502567}{6.539401678352975}{1.0057181080940905*x^(3.0)-2.017154324282272*x^(2.0)+1.0114362161881811*x}
\end{pspicture*}
\end{center}

Ermittle denjenigen Wert $x$, für den gilt: $f(x-3)=2$.

\leer

$x=$ \rule{5cm}{0.3pt}

\antwort{Durch Ablesen erhält man $x-3=2$ und daraus folgt: $x=5$.}
\end{beispiel}