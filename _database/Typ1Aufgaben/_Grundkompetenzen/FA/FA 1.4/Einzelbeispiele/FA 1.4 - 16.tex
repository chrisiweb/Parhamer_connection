\section{FA 1.4 - 16 - Frequenz - MC - MagWie}

\begin{beispiel}[FA 1.4]{1}
Die Tonhöhe ist abhängig von der Frequenz der erzeugten Schallwelle. Je größer die Frequenz, desto höher ist der Ton.
In der untenstehenden Graphik ist ein möglicher zeitlicher Verlauf der Frequenz beim Spielen einer Geige dargestellt.
\begin{center}
\psset{xunit=1.5cm,yunit=0.02cm,algebraic=true,dimen=middle,dotstyle=o,dotsize=5pt 0,linewidth=1.6pt,arrowsize=3pt 2,arrowinset=0.25}
\begin{pspicture*}(-0.6,320)(7.9,681.5680058008134)
\multips(0,350)(0,50.0){39}{\psline[linestyle=dashed,linecap=1,dash=1.5pt 1.5pt,linewidth=0.4pt,linecolor=gray]{c-c}(0,0)(7.396422558564643,0)}
\multips(0,0)(1.0,0){8}{\psline[linestyle=dashed,linecap=1,dash=1.5pt 1.5pt,linewidth=0.4pt,linecolor=gray]{c-c}(0,350)(0,681.5680058008134)}
\psaxes[labelFontSize=\scriptstyle,showorigin=false,xAxis=true,yAxis=true,Dx=1.,Oy=350,Dy=50.,ticksize=-2pt 0,subticks=0]{->}(0,350)(0.,350)(7.9,681.5680058008134)[Zeit t in s,140] [Frequenz f in Hz,-40]
\pszigzag[coilarm=0,coilheight=0.5, coilwidth=0.5](0,370)(0,390)
\psplot[linewidth=2.pt,plotpoints=300]{0}{6.5}{0.033456760766359186*x^(9.0)-1.1189207504139134*x^(8.0)+15.433226752536209*x^(7.0)-113.41040304173075*x^(6.0)+478.50059481576346*x^(5.0)-1161.144415553922*x^(4.0)+1519.9145220363616*x^(3.0)-883.5708848165564*x^(2.0)+111.34990117524194*x+490.6820721034495}
\end{pspicture*}
\end{center}
Kreuze die beiden richtigen Aussagen an!

\multiplechoice[5]{  %Anzahl der Antwortmoeglichkeiten, Standard: 5
				L1={Zum Zeitpunkt $t=3$ ist der Ton höher als zum Zeitpunkt $t=1$.},   %1. Antwortmoeglichkeit 
				L2={Zum Zeitpunkt $t=2$ beträgt die Frequenz 520 Hz.},   %2. Antwortmoeglichkeit
				L3={Im Intervall $[3,4]$ die gespielte Tonhöhe stets tiefer.},   %3. Antwortmoeglichkeit
				L4={Zum Zeitpunkt $t=3$ beträgt die Frequenz 430 Hz.},   %4. Antwortmoeglichkeit
				L5={Die Frequenz zum Zeitpunkt $t=3$ ist gleich der Frequenz zum Zeitpunkt $t=4$.},	 %5. Antwortmoeglichkeit
				L6={},	 %6. Antwortmoeglichkeit
				L7={},	 %7. Antwortmoeglichkeit
				L8={},	 %8. Antwortmoeglichkeit
				L9={},	 %9. Antwortmoeglichkeit
				%% LOESUNG: %%
				A1=1,  % 1. Antwort
				A2=2,	% 2. Antwort
				A3=0,  % 3. Antwort
				A4=0,  % 4. Antwort
				A5=0,  % 5. Antwort
				}



\end{beispiel}