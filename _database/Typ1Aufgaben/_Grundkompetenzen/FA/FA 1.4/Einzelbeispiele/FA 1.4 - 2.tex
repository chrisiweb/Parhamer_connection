\section{FA 1.4 - 2 Funktionale Abh�ngigkeit - MC - BIFIE}

\begin{beispiel}[FA 1.4]{1} %PUNKTE DES BEISPIELS
Die in der nachstehenden Abbildung dargestellte Polynomfunktion 2. Grades beschreibt die H�he (in m) eines senkrecht nach oben geworfenen K�rpers in Abh�ngigkeit von der Zeit (in s).
\leer

\begin{center}
\resizebox{0.8\linewidth}{!}{\psset{xunit=1.0cm,yunit=0.1cm,algebraic=true,dimen=middle,dotstyle=o,dotsize=5pt 0,linewidth=0.8pt,arrowsize=3pt 2,arrowinset=0.25}
\begin{pspicture*}(-0.62,-5.2)(6.46,35.8)
\multips(0,0)(0,10.0){5}{\psline[linestyle=dashed,linecap=1,dash=1.5pt 1.5pt,linewidth=0.4pt,linecolor=gray]{c-c}(0,0)(6.46,0)}
\multips(0,0)(1.0,0){8}{\psline[linestyle=dashed,linecap=1,dash=1.5pt 1.5pt,linewidth=0.4pt,linecolor=gray]{c-c}(0,0)(0,35.8)}
\psaxes[labelFontSize=\scriptstyle,xAxis=true,yAxis=true,Dx=1.,Dy=10.,ticksize=-2pt 0,subticks=2]{->}(0,0)(0.,0.)(6.46,35.8)[\tiny Zeit (in s),140] [\tiny H�he (in m),-40]
\psplot[linewidth=0.8pt,plotpoints=200]{0}{5}{1.0E-50*x^(4.0)-1.0E-49*x^(3.0)-5.0*x^(2.0)+25.0*x}
\end{pspicture*}}
\end{center}

Kreuze die zutreffende(n) Aussage(n) an!

\multiplechoice[5]{  %Anzahl der Antwortmoeglichkeiten, Standard: 5
				L1={Der K�rper befindet sich nach einer Sekunde
und nach vier Sekunden in 20\,m H�he.},   %1. Antwortmoeglichkeit 
				L2={Nach f�nf Sekunden ist der K�rper in derselben
H�he wie zu Beginn der Bewegung.},   %2. Antwortmoeglichkeit
				L3={Der K�rper erreicht maximal 30\,m H�he.},   %3. Antwortmoeglichkeit
				L4={Der K�rper befindet sich nach 4,8 Sekunden
in einer H�he von 10\,m.},   %4. Antwortmoeglichkeit
				L5={Der K�rper befindet sich nach ca. 2,5 Sekunden
in der maximalen H�he.},	 %5. Antwortmoeglichkeit
				L6={},	 %6. Antwortmoeglichkeit
				L7={},	 %7. Antwortmoeglichkeit
				L8={},	 %8. Antwortmoeglichkeit
				L9={},	 %9. Antwortmoeglichkeit
				%% LOESUNG: %%
				A1=1,  % 1. Antwort
				A2=2,	 % 2. Antwort
				A3=5,  % 3. Antwort
				A4=0,  % 4. Antwort
				A5=0,  % 5. Antwort
				}

\end{beispiel}