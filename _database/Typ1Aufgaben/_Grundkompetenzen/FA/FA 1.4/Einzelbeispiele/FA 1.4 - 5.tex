\section{FA 1.4 - 5 Funktionswerte - LT - BIFIE}

\begin{beispiel}[FA 1.4]{1} %PUNKTE DES BEISPIELS
Die nachstehende Abbildung zeigt den Graphen einer Polynomfunktion $f$ vierten Grades. 


\begin{center}
\psset{xunit=1.0cm,yunit=1.0cm,algebraic=true,dimen=middle,dotstyle=o,dotsize=5pt 0,linewidth=0.8pt,arrowsize=3pt 2,arrowinset=0.25}
\begin{pspicture*}(-1.3890340559331749,-2.106660369493021)(9.905998761504648,8.491148199954798)
\multips(0,-2)(0,1.0){11}{\psline[linestyle=dashed,linecap=1,dash=1.5pt 1.5pt,linewidth=0.4pt,linecolor=lightgray]{c-c}(-1.3890340559331749,0)(9.905998761504648,0)}
\multips(-1,0)(1.0,0){12}{\psline[linestyle=dashed,linecap=1,dash=1.5pt 1.5pt,linewidth=0.4pt,linecolor=lightgray]{c-c}(0,-2.106660369493021)(0,8.491148199954798)}
\psaxes[labelFontSize=\scriptstyle,xAxis=true,yAxis=true,Dx=1.,Dy=1.,ticksize=-2pt 0,subticks=2]{->}(0,0)(-1.3890340559331749,-2.106660369493021)(9.905998761504648,8.491148199954798)[x,140] [f(x),-40]
\psplot[plotpoints=200]{-1.3890340559331749}{9.905998761504648}{-0.019450467641740948*x^(4.0)+0.33665411810640195*x^(3.0)-1.6121260286149794*x^(2.0)+2.0317693856368777*x+2.0}
\rput[tl](5.662891195165572,4.646454489610006){$f$}
\end{pspicture*}
\end{center}

\lueckentext{
				text={Für alle reellen Werte \gap gilt für die Funktion $f$ \gap.}, 	%Lueckentext Luecke=\gap
				L1={$x<6$}, 		%1.Moeglichkeit links  
				L2={$x\in [-1;1]$}, 		%2.Moeglichkeit links
				L3={$x \in [1;5]$}, 		%3.Moeglichkeit links
				R1={$f(x)>3$}, 		%1.Moeglichkeit rechts 
				R2={$f(x) \in [-1;1]$}, 		%2.Moeglichkeit rechts
				R3={$f(x) \in [0;3]$}, 		%3.Moeglichkeit rechts
				%% LOESUNG: %%
				A1=3,   % Antwort links
				A2=3		% Antwort rechts 
				}

\end{beispiel}