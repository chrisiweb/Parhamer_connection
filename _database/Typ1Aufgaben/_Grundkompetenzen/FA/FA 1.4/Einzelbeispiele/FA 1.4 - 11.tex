\section{FA 1.4 - 11 - MAT - Volumen eines Drehkegels - MC - Matura 2014/15 Haupttermin}

\begin{beispiel}[FA 1.4]{1} %PUNKTE DES BEISPIELS
Das Volumen $V$ eines Drehkegels hängt vom Radius $r$ und der Höhe $h$ ab. Es wird durch die
Formel $V=\frac{1}{3}\cdot r^2 \cdot \pi \cdot h$ beschrieben. \leer

Eine der nachstehenden Abbildungen stellt die Abhängigkeit des Volumens eines Drehkegels vom
Radius bei konstanter Höhe dar. \leer

Kreuze die entsprechende Abbildung an.

\langmultiplechoice[6]{  %Anzahl der Antwortmoeglichkeiten, Standard: 5
				L1={\resizebox{0.7\linewidth}{!}{\psset{xunit=1.0cm,yunit=1.0cm,algebraic=true,dimen=middle,dotstyle=o,dotsize=5pt 0,linewidth=0.8pt,arrowsize=3pt 2,arrowinset=0.25}
\begin{pspicture*}(-0.3591171748218726,-0.37436244959140136)(6.8,6.8)
\multips(0,0)(0,1.0){7}{\psline[linestyle=dashed,linecap=1,dash=1.5pt 1.5pt,linewidth=0.4pt,linecolor=darkgray]{c-c}(0,0)(6.8,0)}
\multips(0,0)(1.0,0){7}{\psline[linestyle=dashed,linecap=1,dash=1.5pt 1.5pt,linewidth=0.4pt,linecolor=darkgray]{c-c}(0,0)(0,6.8)}
\psaxes[labelFontSize=\scriptstyle,xAxis=true,yAxis=true,labels=none,Dx=1.,Dy=1.,ticksize=-2pt 0,subticks=2]{->}(0,0)(0.,0.)(6.8,6.8)[$r$,140] [$V(r)$,-40]
\psplot[linewidth=1.2pt,plotpoints=200]{0}{2.45}{-1.0/3.0*x^(2.0)*PI+PI*2}
\rput[tl](1.6,4.7){$V$}
\end{pspicture*}}},   %1. Antwortmoeglichkeit 
				L2={\resizebox{0.7\linewidth}{!}{\psset{xunit=1.0cm,yunit=1.0cm,algebraic=true,dimen=middle,dotstyle=o,dotsize=5pt 0,linewidth=0.8pt,arrowsize=3pt 2,arrowinset=0.25}
\begin{pspicture*}(-0.3591171748218726,-0.37436244959140136)(6.4493110859082226,6.404845088989656)
\multips(0,0)(0,1.0){7}{\psline[linestyle=dashed,linecap=1,dash=1.5pt 1.5pt,linewidth=0.4pt,linecolor=darkgray]{c-c}(0,0)(6.4493110859082226,0)}
\multips(0,0)(1.0,0){7}{\psline[linestyle=dashed,linecap=1,dash=1.5pt 1.5pt,linewidth=0.4pt,linecolor=darkgray]{c-c}(0,0)(0,6.404845088989656)}
\psaxes[labelFontSize=\scriptstyle,xAxis=true,yAxis=true,labels=none,Dx=1.,Dy=1.,ticksize=-2pt 0,subticks=2]{->}(0,0)(0.,0.)(6.4493110859082226,6.404845088989656)[$r$,140] [$V(r)$,-40]
\psplot[linewidth=1.2pt,plotpoints=200]{0}{6.4493110859082226}{x}
\rput[tl](2.5,3.3){$V$}
\end{pspicture*}}},   %2. Antwortmoeglichkeit
				L3={\resizebox{0.7\linewidth}{!}{\psset{xunit=1.0cm,yunit=1.0cm,algebraic=true,dimen=middle,dotstyle=o,dotsize=5pt 0,linewidth=0.8pt,arrowsize=3pt 2,arrowinset=0.25}
\begin{pspicture*}(-0.3591171748218726,-0.37436244959140136)(6.4493110859082226,6.404845088989656)
\multips(0,0)(0,1.0){7}{\psline[linestyle=dashed,linecap=1,dash=1.5pt 1.5pt,linewidth=0.4pt,linecolor=darkgray]{c-c}(0,0)(6.4493110859082226,0)}
\multips(0,0)(1.0,0){7}{\psline[linestyle=dashed,linecap=1,dash=1.5pt 1.5pt,linewidth=0.4pt,linecolor=darkgray]{c-c}(0,0)(0,6.404845088989656)}
\psaxes[labelFontSize=\scriptstyle,xAxis=true,yAxis=true,labels=none,Dx=1.,Dy=1.,ticksize=-2pt 0,subticks=2]{->}(0,0)(0.,0.)(6.4493110859082226,6.404845088989656)[$r$,140] [$V(r)$,-40]
\psplot[linewidth=1.2pt,plotpoints=200]{0.01}{6.4493110859082226}{PI/x}
\rput[tl](1.5,2.8){$V$}
\end{pspicture*}}},   %3. Antwortmoeglichkeit
				L4={\resizebox{0.7\linewidth}{!}{\psset{xunit=1.0cm,yunit=1.0cm,algebraic=true,dimen=middle,dotstyle=o,dotsize=5pt 0,linewidth=0.8pt,arrowsize=3pt 2,arrowinset=0.25}
\begin{pspicture*}(-0.3591171748218726,-0.37436244959140136)(6.4493110859082226,6.404845088989656)
\multips(0,0)(0,1.0){7}{\psline[linestyle=dashed,linecap=1,dash=1.5pt 1.5pt,linewidth=0.4pt,linecolor=darkgray]{c-c}(0,0)(6.4493110859082226,0)}
\multips(0,0)(1.0,0){7}{\psline[linestyle=dashed,linecap=1,dash=1.5pt 1.5pt,linewidth=0.4pt,linecolor=darkgray]{c-c}(0,0)(0,6.404845088989656)}
\psaxes[labelFontSize=\scriptstyle,xAxis=true,yAxis=true,labels=none,Dx=1.,Dy=1.,ticksize=-2pt 0,subticks=2]{->}(0,0)(0.,0.)(6.4493110859082226,6.404845088989656)[$r$,140] [$V(r)$,-40]
\psplot[linewidth=1.2pt,plotpoints=200]{0}{6.4493110859082226}{1.0/3.0*x^(2.0)*PI}
\rput[tl](2.1538649299969608,4.3){$V$}
\end{pspicture*}}},   %4. Antwortmoeglichkeit
				L5={\resizebox{0.7\linewidth}{!}{\psset{xunit=1.0cm,yunit=1.0cm,algebraic=true,dimen=middle,dotstyle=o,dotsize=5pt 0,linewidth=0.8pt,arrowsize=3pt 2,arrowinset=0.25}
\begin{pspicture*}(-0.3591171748218726,-0.37436244959140136)(6.4493110859082226,6.404845088989656)
\multips(0,0)(0,1.0){7}{\psline[linestyle=dashed,linecap=1,dash=1.5pt 1.5pt,linewidth=0.4pt,linecolor=darkgray]{c-c}(0,0)(6.4493110859082226,0)}
\multips(0,0)(1.0,0){7}{\psline[linestyle=dashed,linecap=1,dash=1.5pt 1.5pt,linewidth=0.4pt,linecolor=darkgray]{c-c}(0,0)(0,6.404845088989656)}
\psaxes[labelFontSize=\scriptstyle,xAxis=true,yAxis=true,labels=none,Dx=1.,Dy=1.,ticksize=-2pt 0,subticks=2]{->}(0,0)(0.,0.)(6.4493110859082226,6.404845088989656)[$r$,140] [$V(r)$,-40]
\psplot[linewidth=1.2pt,plotpoints=200]{0}{6.4493110859082226}{sqrt(PI*x)}
\rput[tl](2.6,3.5){$V$}
\end{pspicture*}}},	 %5. Antwortmoeglichkeit
				L6={\resizebox{0.7\linewidth}{!}{\psset{xunit=1.0cm,yunit=1.0cm,algebraic=true,dimen=middle,dotstyle=o,dotsize=5pt 0,linewidth=0.8pt,arrowsize=3pt 2,arrowinset=0.25}
\begin{pspicture*}(-0.3591171748218726,-0.37436244959140136)(6.4493110859082226,6.404845088989656)
\multips(0,0)(0,1.0){7}{\psline[linestyle=dashed,linecap=1,dash=1.5pt 1.5pt,linewidth=0.4pt,linecolor=darkgray]{c-c}(0,0)(6.4493110859082226,0)}
\multips(0,0)(1.0,0){7}{\psline[linestyle=dashed,linecap=1,dash=1.5pt 1.5pt,linewidth=0.4pt,linecolor=darkgray]{c-c}(0,0)(0,6.404845088989656)}
\psaxes[labelFontSize=\scriptstyle,xAxis=true,yAxis=true,labels=none,Dx=1.,Dy=1.,ticksize=-2pt 0,subticks=2]{->}(0,0)(0.,0.)(6.4493110859082226,6.404845088989656)[$r$,140] [$V(r)$,-40]
\psplot[linewidth=1.2pt,plotpoints=200]{0}{6.4493110859082226}{2.0*x^(2.0)+1}
\rput[tl](1.3,3.6){$V$}
\end{pspicture*}}},	 %6. Antwortmoeglichkeit
				L7={},	 %7. Antwortmoeglichkeit
				L8={},	 %8. Antwortmoeglichkeit
				L9={},	 %9. Antwortmoeglichkeit
				%% LOESUNG: %%
				A1=4,  % 1. Antwort
				A2=0,	 % 2. Antwort
				A3=0,  % 3. Antwort
				A4=0,  % 4. Antwort
				A5=0,  % 5. Antwort
				}
\end{beispiel}