\section{FA 5.3 - 10 - MAT - Wirkstoff - OA - Matura-HT-18/19}

\begin{beispiel}[FA 5.3]{1}
Die Abnahme der Menge des Wirkstoffs eines Medikaments im Blut lässt sich durch eine Exponentialfunktion modellieren.\\
Nach einer Stunde sind 10\,\% der Anfangsmenge des Wirkstoffs abgebaut worden.

Berechne, welcher Prozentsatz der Anfangsmenge des Wirkstoffs nach insgesamt vier Stunden noch im Blut vorhanden ist!\leer

\antwort[\rule{3cm}{0.3pt}]{$0,9^4=0,6561$}\,\% der Anfangsmenge
\end{beispiel}