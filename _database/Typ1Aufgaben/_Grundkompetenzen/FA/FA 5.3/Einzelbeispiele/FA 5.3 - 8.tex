\section{FA 5.3 - 8 - MAT - �nderungsprozess - MC - Matura 2016/17 2. NT}

\begin{beispiel}{1} %PUNKTE DES BEISPIELS
Durch die Gleichung $N(t)=1,2 \cdot 0,98^t$ wird ein �nderungsprozess einer Gr��e $N$ in Abh�ngigkeit von der Zeit $t$ beschrieben.

Welche der angef�hrten �nderungsprozesse kann durch die angegebene Gleichung beschrieben
werden? Kreuze den zutreffenden �nderungsprozess an! \leer

\multiplechoice[6]{  %Anzahl der Antwortmoeglichkeiten, Standard: 5
				L1={Von einer radioaktiven Substanz zerfallen pro Zeiteinheit 0,02\,\%
der am jeweiligen Tag vorhandenen Menge},   %1. Antwortmoeglichkeit 
				L2={In ein Speicherbecken flie�en pro Zeiteinheit 0,02 \,m$^3$ Wasser zu.},   %2. Antwortmoeglichkeit
				L3={Vom Wirkstoff eines Medikaments werden pro Zeiteinheit 1,2\,mg
abgebaut.},   %3. Antwortmoeglichkeit
				L4={Die Einwohnerzahl eines Landes nimmt pro Zeiteinheit um 1,2\,\% zu.},   %4. Antwortmoeglichkeit
				L5={Der Wert einer Immobilie steigt pro Zeiteinheit um 2\,\%.},	 %5. Antwortmoeglichkeit
				L6={Pro Zeiteinheit nimmt die Temperatur eines K�rpers um 2\,\% ab.},	 %6. Antwortmoeglichkeit
				L7={},	 %7. Antwortmoeglichkeit
				L8={},	 %8. Antwortmoeglichkeit
				L9={},	 %9. Antwortmoeglichkeit
				%% LOESUNG: %%
				A1=6,  % 1. Antwort
				A2=0,	 % 2. Antwort
				A3=0,  % 3. Antwort
				A4=0,  % 4. Antwort
				A5=0,  % 5. Antwort
				}
\end{beispiel}