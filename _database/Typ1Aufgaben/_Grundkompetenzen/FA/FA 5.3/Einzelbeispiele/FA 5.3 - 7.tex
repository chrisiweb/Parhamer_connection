\section{FA 5.3 - 7 Wachstum einer Population - OA - Matura NT 2 15/16}

\begin{beispiel}[FA 5.3]{1} %PUNKTE DES BEISPIELS
Die Größe einer Population wird in Abhängigkeit von der Zeit mithilfe der Funktion $N$ mit $N(t)=N_0\cdot e^{0,1188\cdot t}$ beschrieben, wobei die Zeit $t$ in Stunden angegeben wird. Dabei bezeichnet $N_0$ die Größe der Population zum Zeitpunkt $t=0$ und $N(t)$ die Größe der Population zum Zeitpunkt $t\geq 0$.

Bestimme denjenigen Prozentsatz $p$, um den die Population pro Stunde wächst!\leer

$p\approx$ \antwort[\rule{3cm}{0.3pt}\%]{12,6\% Toleranzintervall: $\left[12\%;13\%\right]$}
\end{beispiel}