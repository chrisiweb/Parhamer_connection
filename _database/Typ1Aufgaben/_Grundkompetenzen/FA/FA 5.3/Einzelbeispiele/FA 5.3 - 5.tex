\section{FA 5.3 - 5 Bakterienkolonie - OA - BIFIE}

\begin{beispiel}[FA 5.3]{1} %PUNKTE DES BEISPIELS
Das Wachstum einer Bakterienkolonie in Abhängigkeit von der Zeit $t$ (in Stunden) kann näherungsweise
durch die Funktionsgleichung $A = 2 \cdot 1,35^t$
 beschrieben werden, wobei $A(t)$ die
zum Zeitpunkt $t$ besiedelte Fläche (in $\text{mm}^2$) angibt. 

\leer

Interpretiere die in der Funktionsgleichung vorkommenden Werte $2$ und $1,35$ im Hinblick auf den Wachstumsprozess.


\antwort{Zum Zeitpunkt $t = 0$ beträgt der Inhalt der besiedelten Fläche $2\,\text{mm}^2$. Die Bakterienkolonie
wächst pro Stunde um 35\%.\\

Lösungsschlüssel:

Die Aufgabe ist als richtig gelöst zu werten, wenn die Interpretation beider Werte sinngemäß richtig ist. Die Einheit muss nicht angegeben sein. }
\end{beispiel}