\section{FA 5.3 - 1 Exponentielle Abnahme - MC - BIFIE}

\begin{beispiel}[FA 5.3]{1} %PUNKTE DES BEISPIELS
Die angegebenen Funktionsgleichungen beschreiben exponentielle Zusammenhänge.

\leer

Kreuze die beiden Funktionsgleichungen an, die eine exponentielle Abnahme beschreiben.


\multiplechoice[5]{  %Anzahl der Antwortmoeglichkeiten, Standard: 5
				L1={$f(x)=100\cdot 1,2^x$},   %1. Antwortmoeglichkeit 
				L2={$f(x)=100\cdot e^{0,2x}$},   %2. Antwortmoeglichkeit
				L3={$f(x)=100\cdot 0,2^x$},   %3. Antwortmoeglichkeit
				L4={$f(x)=100\cdot 0,2^{-x}$},   %4. Antwortmoeglichkeit
				L5={$f(x)=100\cdot e^{-0,2x}$},	 %5. Antwortmoeglichkeit
				L6={},	 %6. Antwortmoeglichkeit
				L7={},	 %7. Antwortmoeglichkeit
				L8={},	 %8. Antwortmoeglichkeit
				L9={},	 %9. Antwortmoeglichkeit
				%% LOESUNG: %%
				A1=3,  % 1. Antwort
				A2=5,	 % 2. Antwort
				A3=0,  % 3. Antwort
				A4=0,  % 4. Antwort
				A5=0,  % 5. Antwort
				}
\end{beispiel}