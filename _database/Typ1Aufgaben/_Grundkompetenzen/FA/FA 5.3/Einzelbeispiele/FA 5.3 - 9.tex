\section{FA 5.3 - 9 - MAT - Zellkulturen - ZO - Matura HT 2017/18}

\begin{beispiel}[FA 5.3]{1} %PUNKTE DES BEISPIELS
Im Rahmen eines biologischen Experiments werden sechs Zellkulturen günstigen und ungünstigen äußeren Bedingungen ausgesetzt, wodurch die Anzahl der Zellen entweder exponentiell zunimmt oder exponentiell abnimmt.

Dabei gibt $N_i(t)$ die Anzahl der Zellen in der jeweiligen Zellkultur $t$ Tage nach Beginn des Experiments an $(i=1,2,3,4,5,6)$.

Ordne den vier beschriebenen Veränderungen jeweils die zugehörige Funktionsgleichung (aus A bis F) zu!

\zuordnen[0.08]{
				R1={Die Anzahl der Zellen verdoppelt sich pro Tag.},				% Response 1
				R2={Die Anzahl der Zellen nimmt pro Tag um 85\,\% zu.},				% Response 2
				R3={Die Anzahl der Zellen nimmt pro Tag um 85\,\% ab.},				% Response 3
				R4={Die Anzahl der Zellen nimmt pro Tag um die Hälfte ab.},				% Response 4
				%% Moegliche Zuordnungen: %%
				A={$N_1(t)=N_1(0)\cdot 0,15^t$}, 				%Moeglichkeit A  
				B={$N_2(t)=N_2(0)\cdot 0,5^t$}, 				%Moeglichkeit B  
				C={$N_3(t)=N_3(0)\cdot 0,85^t$}, 				%Moeglichkeit C  
				D={$N_4(t)=N_4(0)\cdot 1,5^t$}, 				%Moeglichkeit D  
				E={$N_5(t)=N_5(0)\cdot 1,85^t$}, 				%Moeglichkeit E  
				F={$N_6(t)=N_6(0)\cdot 2^t$}, 				%Moeglichkeit F  
				%% LOESUNG: %%
				A1={F},				% 1. richtige Zuordnung
				A2={E},				% 2. richtige Zuordnung
				A3={A},				% 3. richtige Zuordnung
				A4={B},				% 4. richtige Zuordnung
				}
\end{beispiel}