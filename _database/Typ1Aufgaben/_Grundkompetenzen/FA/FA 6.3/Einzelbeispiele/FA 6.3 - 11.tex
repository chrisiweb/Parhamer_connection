\section{FA 6.3 - 11 Sinusfunktion - OA - Matura 17/18}

\begin{beispiel}[FA 6.3]{1} %PUNKTE DES BEISPIELS
Für $a,b\in\mathbb{R}^+$ sei die Funktion $f: \mathbb{R}\rightarrow\mathbb{R}$ mit $f(x)=a\cdot\sin(b\cdot x)$ für $x\in\mathbb{R}$ gegeben.

Die beiden nachstehenden Eigenschaften der Funktion $f$ sind bekannt:
\begin{itemize}
	\item Die (kleinste) Periode der Funktion $f$ ist $\pi$.
	\item Die Differenz zwischen dem größten und dem kleinsten Funktionswert von $f$ beträgt 6.
\end{itemize}

Gib $a$ und $b$ an!\leer

$a=$\,\antwort[\rule{3cm}{0.3pt}]{3}\leer

$b=$\,\antwort[\rule{3cm}{0.3pt}]{2}
\end{beispiel}