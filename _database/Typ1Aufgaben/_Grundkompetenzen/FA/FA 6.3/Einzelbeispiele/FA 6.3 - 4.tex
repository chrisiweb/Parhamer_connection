\section{FA 6.3 - 4 - Sinusfunktion - OA - BIFIE}

\begin{beispiel}[FA 6.3]{1} %PUNKTE DES BEISPIELS
				Gegeben ist der Graph der Funktion $f(x)=\sin(x)$.

\winkelfunktion\psset{xunit=1.0cm,yunit=1.0cm,trigLabels,algebraic=true,dimen=middle,dotstyle=o,dotsize=5pt 0,linewidth=0.8pt,arrowsize=3pt 2,arrowinset=0.25}
\begin{pspicture*}(-4.5,-2.5)(4.5,2.5)
\multips(0,-4)(0,1.0){10}{\psline[linestyle=dashed,linecap=1,dash=1.5pt 1.5pt,linewidth=0.4pt,linecolor=gray]{c-c}(-10,0)(10,0)}
\multips(-8,0)(1,0){20}{\psline[linestyle=dashed,linecap=1,dash=1.5pt 1.5pt,linewidth=0.4pt,linecolor=gray]{c-c}(0,-5)(0,5)}
\psaxes[labelFontSize=\scriptstyle,trigLabelBase=2,xAxis=true,yAxis=true,Dx=1,Dy=1.,showorigin=false,ticksize=-2pt 0,subticks=0]{->}(0,0)(-10,-8)(4.5,4.5)[x,140] [y,-40]
\psplot[xunit=0.63661977cm,linewidth=1.2pt,plotpoints=200]{-10}{20}{SIN(x)}
\begin{scriptsize}
\rput[bl](-1.5882409839303135,-1.2899119647047494){$f$}
\end{scriptsize}
\antwort{\psplot[linewidth=1.2pt,linestyle=dashed,dash=2pt 2pt,linecolor=red,plotpoints=200]{-10}{20}{SIN(3.1416*(x+0.1))}
\begin{scriptsize}
\rput[bl](-6.724136647424078,-0.9230621316019116){$g$}
\end{scriptsize}}
\end{pspicture*}


Zeichne in die gegebene Abbildung den Graphen der Funktion $g(x)=\sin(2x)$ ein!
\end{beispiel}