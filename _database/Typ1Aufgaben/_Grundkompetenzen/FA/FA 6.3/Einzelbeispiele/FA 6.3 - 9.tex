\section{FA 6.3 - 9 Periodizität - MC - Matura NT 1 16/17}

\begin{beispiel}[FA 6.3]{1} %PUNKTE DES BEISPIELS
Gegeben ist eine reelle Funktion $f$ mit der Funktionsgleichung $f(x)=3\cdot\sin(b\cdot x)$ mit $b\in\mathbb{R}$.

Einer der nachstehend angegebenen Werte gibt die (kleinste) Periodenlänge der Funktion $f$ an. Kreuze den zutreffenden Wert an!

\multiplechoice[6]{  %Anzahl der Antwortmoeglichkeiten, Standard: 5
				L1={$\frac{b}{2}$},   %1. Antwortmoeglichkeit 
				L2={$b$},   %2. Antwortmoeglichkeit
				L3={$\frac{b}{3}$},   %3. Antwortmoeglichkeit
				L4={$\frac{\pi}{b}$},   %4. Antwortmoeglichkeit
				L5={$\frac{2\pi}{b}$},	 %5. Antwortmoeglichkeit
				L6={$\frac{\pi}{3}$},	 %6. Antwortmoeglichkeit
				L7={},	 %7. Antwortmoeglichkeit
				L8={},	 %8. Antwortmoeglichkeit
				L9={},	 %9. Antwortmoeglichkeit
				%% LOESUNG: %%
				A1=5,  % 1. Antwort
				A2=0,	 % 2. Antwort
				A3=0,  % 3. Antwort
				A4=0,  % 4. Antwort
				A5=0,  % 5. Antwort
				}
\end{beispiel}