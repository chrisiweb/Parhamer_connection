\section{FA 6.3 - 13 - MAT - Sinusfunktion - OA - Matura 2018/19 2. NT}

\begin{beispiel}[FA 6.3]{1}
Gegeben ist eine Funktion $f$: $\mathbb{R}\rightarrow\mathbb{R}$ mit $f(x)=a\cdot\sin\left(\frac{\pi\cdot x}{b}\right)$ mit $a,b,\in\mathbb{R}^+$.

Ergänze in der nachstehenden Abbildung $a$ und $b$ auf der jeweils entsprechenden Achse so, dass der abgebildete Graph dem Graphen der Funktion $f$ entspricht.

\begin{center}
\psset{xunit=0.8cm,yunit=1.0cm,algebraic=true,dimen=middle,dotstyle=o,dotsize=5pt 0,linewidth=1.6pt,arrowsize=3pt 2,arrowinset=0.25}
\begin{pspicture*}(-6.88,-3.64)(10.66,4.82)
\multips(0,-3)(0,1.0){9}{\psline[linestyle=dashed,linecap=1,dash=1.5pt 1.5pt,linewidth=0.4pt,linecolor=gray]{c-c}(-6.88,0)(10.66,0)}
\multips(-6,0)(1.0,0){18}{\psline[linestyle=dashed,linecap=1,dash=1.5pt 1.5pt,linewidth=0.4pt,linecolor=gray]{c-c}(0,-3.64)(0,4.82)}
\psaxes[labelFontSize=\scriptstyle,xAxis=true,yAxis=true,labels=none,Dx=1.,Dy=1.,ticksize=-2pt 0,subticks=0]{->}(0,0)(-6.88,-3.64)(10.66,4.82)[$x$,140] [$f(x)$,-40]
\psplot[linewidth=2.pt,plotpoints=200]{-6.88}{10.66}{2.0*SIN(PI*x/4.0)}
\rput[tl](3.32,1.76){$f$}
\rput[tl](0.2,-0.15){0}
\rput[tl](-0.4,0.42){0}
\antwort{\rput[tl](-0.4,2.1){$a$}
\rput[tl](3.8,-0.15){$b$}}
\end{pspicture*}
\end{center}
\end{beispiel}