\section{FA 6.3 - 1 - Wirkung der Parameter einer Sinusfunktion - ZO - BIFIE}

\begin{beispiel}[FA 6.3]{1} %PUNKTE DES BEISPIELS
				Gegeben ist eine Sinusfunktion der Art $f(x)=a\cdot \sin(b\cdot x)$.

Dabei beeinflussen die Parameter $a$ und $b$ das Aussehen des Graphen von $f$ im Vergleich zum Graphen von $g(x)=\sin(x)$.

Ordne den Parameterwerten die entsprechenden Auswirkungen auf das Aussehen von $f$ im Vergleich zu $g$ zu!

\zuordnen[-0.29]{
				title1={Auswirkungen}, 		%Titel Antwortmoeglichkeiten
				A={Dehnung des Graphen der Funktion entlang der x-Achse auf das Doppelte}, 				%Moeglichkeit A  
				B={Phasenverschiebung um 2}, 				%Moeglichkeit B  
				C={doppelte Frequenz}, 				%Moeglichkeit C  
				D={Streckung entlang der y-Achse auf das Doppelte}, 				%Moeglichkeit D  
				E={halbe Amplitude}, 				%Moeglichkeit E  
				F={Verschiebung entlang der y-Achse um -2}, 				%Moeglichkeit F  
				title2={Parameter},		%Titel Zuordnung
				R1={$a=2$},				%1. Antwort rechts
				R2={$a=\frac{1}{2}$},				%2. Antwort rechts
				R3={$b=2$},				%3. Antwort rechts
				R4={$b=\frac{1}{2}$},				%4. Antwort rechts
				%% LOESUNG: %%
				A1={D},				% 1. richtige Zuordnung
				A2={E},				% 2. richtige Zuordnung
				A3={C},				% 3. richtige Zuordnung
				A4={A},				% 4. richtige Zuordnung
				}
\end{beispiel}