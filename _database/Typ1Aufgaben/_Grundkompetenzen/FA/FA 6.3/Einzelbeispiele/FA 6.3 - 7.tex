\section{FA 6.3 - 7 - MAT - Sinusfunktion - ZO - Matura 1. NT 2014/15}

\begin{beispiel}[FA 6.3]{1} %PUNKTE DES BEISPIELS
Gegeben sind die Graphen von vier Funktionen der Form$ f(x)=a \cdot \sin(b \cdot x)$ mit $a, b \in \mathbb{R}$. \leer

Ordne jedem Graphen den dazugehörigen Funktionsterm (aus A bis F) zu.

\zuordnen[0.24]{
				R1={\resizebox{0.8\linewidth}{!}{\winkelfunktion\psset{xunit=1.0cm,yunit=1.0cm,trigLabels,algebraic=true,dimen=middle,dotstyle=o,dotsize=5pt 0,linewidth=0.8pt,arrowsize=3pt 2,arrowinset=0.25}
				\begin{pspicture*}(-2.5,-3.0827743166790045)(7,4.028913012818873)
				\multips(0,-4)(0,0.5){20}{\psline[linestyle=dashed,linecap=1,dash=1.5pt 1.5pt,linewidth=0.4pt,linecolor=lightgray]{c-c}(-10,0)(10,0)}
				\multips(-2,0)(0.5,0){24}{\psline[linestyle=dashed,linecap=1,dash=1.5pt 1.5pt,linewidth=0.4pt,linecolor=lightgray]{c-c}(0,-5)(0,5)}
				\psaxes[labelFontSize=\scriptstyle,trigLabelBase=3,xAxis=true,yAxis=true,Dx=1,Dy=1.,ticksize=-2pt 0,subticks=2]{->}(0,0)(-2.5,-3.0827743166790045)(7,4.028913012818873)[$x$,140] [$f_1(x)$,-40]
				\psplot[xunit=0.63661977cm,linewidth=1.2pt,plotpoints=600]{-10}{20}{2*SIN(2*x)}
				\begin{scriptsize}
				\rput[bl](3,1.5){$f_1$}
				\end{scriptsize}
				\end{pspicture*}}},				% Response 1
				R2={\resizebox{0.8\linewidth}{!}{\winkelfunktion\psset{xunit=1.0cm,yunit=1.0cm,trigLabels,algebraic=true,dimen=middle,dotstyle=o,dotsize=5pt 0,linewidth=0.8pt,arrowsize=3pt 2,arrowinset=0.25}
				\begin{pspicture*}(-2.5,-3.0827743166790045)(7,4.028913012818873)
				\multips(0,-4)(0,0.5){20}{\psline[linestyle=dashed,linecap=1,dash=1.5pt 1.5pt,linewidth=0.4pt,linecolor=lightgray]{c-c}(-10,0)(10,0)}
				\multips(-2,0)(0.5,0){24}{\psline[linestyle=dashed,linecap=1,dash=1.5pt 1.5pt,linewidth=0.4pt,linecolor=lightgray]{c-c}(0,-5)(0,5)}
				\psaxes[labelFontSize=\scriptstyle,trigLabelBase=3,xAxis=true,yAxis=true,Dx=1,Dy=1.,ticksize=-2pt 0,subticks=2]{->}(0,0)(-2.5,-3.0827743166790045)(7,4.028913012818873)[$x$,140] [$f_2(x)$,-40]
				\psplot[xunit=0.63661977cm,linewidth=1.2pt,plotpoints=600]{-10}{20}{SIN(1/3*x)}
				\begin{scriptsize}
				\rput[bl](3,1.1){$f_2$}
				\end{scriptsize}
				\end{pspicture*}}},				% Response 2
				R3={\resizebox{0.8\linewidth}{!}{\winkelfunktion\psset{xunit=1.0cm,yunit=1.0cm,trigLabels,algebraic=true,dimen=middle,dotstyle=o,dotsize=5pt 0,linewidth=0.8pt,arrowsize=3pt 2,arrowinset=0.25}
				\begin{pspicture*}(-2.5,-3.0827743166790045)(7,4.028913012818873)
				\multips(0,-4)(0,0.5){20}{\psline[linestyle=dashed,linecap=1,dash=1.5pt 1.5pt,linewidth=0.4pt,linecolor=lightgray]{c-c}(-10,0)(10,0)}
				\multips(-2,0)(0.5,0){24}{\psline[linestyle=dashed,linecap=1,dash=1.5pt 1.5pt,linewidth=0.4pt,linecolor=lightgray]{c-c}(0,-5)(0,5)}
				\psaxes[labelFontSize=\scriptstyle,trigLabelBase=3,xAxis=true,yAxis=true,Dx=1,Dy=1.,ticksize=-2pt 0,subticks=2]{->}(0,0)(-2.5,-3.0827743166790045)(7,4.028913012818873)[$x$,140] [$f_3(x)$,-40]
				\psplot[xunit=0.63661977cm,linewidth=1.2pt,plotpoints=600]{-10}{20}{2*SIN(2/3*x)}
				\begin{scriptsize}
				\rput[bl](2.6,1.5){$f_3$}
				\end{scriptsize}
				\end{pspicture*}}},				% Response 3
				R4={\resizebox{0.8\linewidth}{!}{\winkelfunktion\psset{xunit=1.0cm,yunit=1.0cm,trigLabels,algebraic=true,dimen=middle,dotstyle=o,dotsize=5pt 0,linewidth=0.8pt,arrowsize=3pt 2,arrowinset=0.25}
				\begin{pspicture*}(-2.5,-3.0827743166790045)(7,4.028913012818873)
				\multips(0,-4)(0,0.5){20}{\psline[linestyle=dashed,linecap=1,dash=1.5pt 1.5pt,linewidth=0.4pt,linecolor=lightgray]{c-c}(-10,0)(10,0)}
				\multips(-2,0)(0.5,0){24}{\psline[linestyle=dashed,linecap=1,dash=1.5pt 1.5pt,linewidth=0.4pt,linecolor=lightgray]{c-c}(0,-5)(0,5)}
				\psaxes[labelFontSize=\scriptstyle,trigLabelBase=3,xAxis=true,yAxis=true,Dx=1,Dy=1.,ticksize=-2pt 0,subticks=2]{->}(0,0)(-2.5,-3.0827743166790045)(7,4.028913012818873)[$x$,140] [$f_4(x)$,-40]
				\psplot[xunit=0.63661977cm,linewidth=1.2pt,plotpoints=600]{-10}{20}{2*SIN(4/3*x)}
				\begin{scriptsize}
				\rput[bl](3,1.5){$f_4$}
				\end{scriptsize}
				\end{pspicture*}}},				% Response 4
				%% Moegliche Zuordnungen: %%
				A={$\sin(x)$}, 				%Moeglichkeit A  
				B={$1,5\cdot \sin(x)$}, 				%Moeglichkeit B  
				C={$\sin(0,5x)$}, 				%Moeglichkeit C  
				D={$1,5\cdot \sin(2x)$}, 				%Moeglichkeit D  
				E={$2\cdot \sin(0,5x)$}, 				%Moeglichkeit E  
				F={$2\cdot \sin(3x)$}, 				%Moeglichkeit F  
				%% LOESUNG: %%
				A1={F},				% 1. richtige Zuordnung
				A2={C},				% 2. richtige Zuordnung
				A3={B},				% 3. richtige Zuordnung
				A4={D},				% 4. richtige Zuordnung
				}
\end{beispiel}