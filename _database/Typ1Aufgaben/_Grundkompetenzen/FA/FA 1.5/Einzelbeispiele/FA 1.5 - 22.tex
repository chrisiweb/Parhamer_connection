\section{FA 1.5 - 22 - MAT - Polynomfunktionen dritten Grades - OA - Matura 2. NT 2017/18}

\begin{beispiel}[FA 1.5]{1}
Eine Polynomfunktion dritten Grades ändert an höchstens zwei Stellen ihr Monotonieverhalten.

Skizziere im nachstehenden Koordinatensystem den Graphen einer Polynomfunktion dritten Grades $f$, die an den Stellen $x = -3$ und $x = 1$ ihr Monotonieverhalten ändert!

\antwort{Mögliche Graphen:}
\begin{center}
\psset{xunit=1.0cm,yunit=1.0cm,algebraic=true,dimen=middle,dotstyle=o,dotsize=5pt 0,linewidth=1.6pt,arrowsize=3pt 2,arrowinset=0.25}
\begin{pspicture*}(-5.5,-5.5)(5.5,5.5)
\multips(0,-5)(0,1.0){12}{\psline[linestyle=dashed,linecap=1,dash=1.5pt 1.5pt,linewidth=0.4pt,linecolor=lightgray]{c-c}(-5.5,0)(5.5,0)}
\multips(-6,0)(1.0,0){12}{\psline[linestyle=dashed,linecap=1,dash=1.5pt 1.5pt,linewidth=0.4pt,linecolor=lightgray]{c-c}(0,-5.5)(0,5.5)}
\psaxes[labelFontSize=\scriptstyle,xAxis=true,yAxis=true,Dx=1.,Dy=1.,ticksize=-2pt 0,subticks=2]{->}(0,0)(-5.5,-5.5)(5.5,5.5)
\antwort{\psplot[linewidth=2.pt,linecolor=red,plotpoints=200]{-6}{5.0}{x^(3.0)/6.0+x^(2.0)/2.0-3.0/2.0*x}
\psplot[linewidth=2.pt,linecolor=red,plotpoints=200]{-6}{5.0}{(-x^(3.0))/6.0-x^(2.0)/2.0+3.0/2.0*x+2.0}}
\end{pspicture*}
\end{center}
\end{beispiel}