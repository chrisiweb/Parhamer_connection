\section{FA 1.5 - 12 - MAT - Funktionseigenschaften erkennen - MC - Matura 2015/16 - Haupttermin}

\begin{beispiel}[FA 1.5]{1} %PUNKTE DES BEISPIELS
Gegeben ist der Graph einer Polynomfunktion $f$ dritten Grades. \leer

\begin{center}
\psset{xunit=1.0cm,yunit=1.0cm,algebraic=true,dimen=middle,dotstyle=o,dotsize=5pt 0,linewidth=0.8pt,arrowsize=3pt 2,arrowinset=0.25}
\begin{pspicture*}(-4.56,-5.32)(4.5,5.5)
\multips(0,-5)(0,1.0){11}{\psline[linestyle=dashed,linecap=1,dash=1.5pt 1.5pt,linewidth=0.4pt,linecolor=lightgray]{c-c}(-4.56,0)(4.5,0)}
\multips(-4,0)(1.0,0){10}{\psline[linestyle=dashed,linecap=1,dash=1.5pt 1.5pt,linewidth=0.4pt,linecolor=lightgray]{c-c}(0,-5.32)(0,5.5)}
\psaxes[labelFontSize=\scriptstyle,xAxis=true,yAxis=true,Dx=1.,Dy=1.,ticksize=-2pt 0,subticks=2]{->}(0,0)(-4.56,-5.32)(4.5,5.5)[$x$,140] [$f(x)$,-40]
\psplot[linewidth=1.6pt,plotpoints=200]{-4.560000000000001}{4.500000000000002}{0.5*x^(3.0)-4.0*x}
\rput[tl](3.36,3.8){$f$}
\end{pspicture*}
\end{center}


Kreuze die für den dargestellten Funktionsgraphen von $f$ zutreffende(n) Aussage(n) an.

\multiplechoice[5]{  %Anzahl der Antwortmoeglichkeiten, Standard: 5
				L1={Die Funktion $f$ ist im Intervall $(2;~3)$ monoton steigend.},   %1. Antwortmoeglichkeit 
				L2={Die Funktion $f$ hat im Intervall $(1;~2)$ eine lokale Maximumstelle.},   %2. Antwortmoeglichkeit
				L3={Die Funktion $f$ ändert im Intervall $(-1;~1)$ das Krümmungsverhalten.},   %3. Antwortmoeglichkeit
				L4={Der Funktionsgraph von $f$ ist symmetrisch bezüglich der senkrechten Achse.},   %4. Antwortmoeglichkeit
				L5={Die Funktion $f$ ändert im Intervall $(-3;~0)$ das Monotonieverhalten.},	 %5. Antwortmoeglichkeit
				L6={},	 %6. Antwortmoeglichkeit
				L7={},	 %7. Antwortmoeglichkeit
				L8={},	 %8. Antwortmoeglichkeit
				L9={},	 %9. Antwortmoeglichkeit
				%% LOESUNG: %%
				A1=1,  % 1. Antwort
				A2=3,	 % 2. Antwort
				A3=5,  % 3. Antwort
				A4=0,  % 4. Antwort
				A5=0,  % 5. Antwort
				}


\end{beispiel}