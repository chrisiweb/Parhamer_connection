\section{FA 1.5 - 21 - MAT - Eigenschaften von Funktionsgraphen - ZO - Matura 2. NT 2017/18}

\begin{beispiel}[FA 1.5]{1}
Nachstehend sind Eigenschaften von Funktionen angeführt sowie charakteristische Ausschnitte
von Funktionsgraphen abgebildet.

Ordne den vier Eigenschaften jeweils den passenden Graphen (aus A bis F) zu!

\zuordnen{
				R1={Die Funktion ist auf ihrem
gesamten Definitionsbereich
monoton steigend.},				% Response 1
				R2={Die Funktion ist auf ihrem gesamten Definitionsbereich negativ
gekrümmt (rechtsgekrümmt).},				% Response 2
				R3={Die Funktion ist auf dem Intervall
$(-\infty; 0)$ positiv gekrümmt (linksgekrümmt).},				% Response 3
				R4={Die Funktion ist auf dem Intervall
$(-\infty; 0)$ monoton fallend.},				% Response 4
				%% Moegliche Zuordnungen: %%
				A={\resizebox{1\linewidth}{!}{\psset{xunit=1.0cm,yunit=1.0cm,algebraic=true,dimen=middle,dotstyle=o,dotsize=5pt 0,linewidth=1.6pt,arrowsize=3pt 2,arrowinset=0.25}
\begin{pspicture*}(-3.7,-1)(3.7,3.7)
\psaxes[labelFontSize=\scriptstyle,xAxis=true,yAxis=true,labels=none,Dx=1.,Dy=1.,ticksize=0pt 0,subticks=2]{->}(0,0)(-3.7,-3.7)(3.7,3.7)[x,140] [$f_1(x)$,-40]
\psplot[linewidth=0.8pt,plotpoints=200]{-3.7}{3.7}{1.0/(0.8*x)^(2.0)}
\rput[bl](2,1.5){$f_1$}
\end{pspicture*}}}, 				%Moeglichkeit A  
				B={\resizebox{1\linewidth}{!}{\psset{xunit=1.0cm,yunit=1.0cm,algebraic=true,dimen=middle,dotstyle=o,dotsize=5pt 0,linewidth=1.6pt,arrowsize=3pt 2,arrowinset=0.25}
\begin{pspicture*}(-3.7,-2.5)(3.7,3)
\psaxes[labelFontSize=\scriptstyle,xAxis=true,yAxis=true,labels=none,Dx=1.,Dy=1.,ticksize=0pt 0,subticks=2]{->}(0,0)(-3.7,-3)(3.7,3)[x,140] [$f_2(x)$,-40]
\psplot[linewidth=0.8pt,plotpoints=200]{-3.7}{3.7}{1.0/x}
\rput[bl](2,0.8){$f_2$}
\end{pspicture*}}}, 				%Moeglichkeit B  
				C={\resizebox{1\linewidth}{!}{\psset{xunit=1.0cm,yunit=1.0cm,algebraic=true,dimen=middle,dotstyle=o,dotsize=5pt 0,linewidth=1.6pt,arrowsize=3pt 2,arrowinset=0.25}
\begin{pspicture*}(-3.7,-1.5)(3.7,3)
\psaxes[labelFontSize=\scriptstyle,xAxis=true,yAxis=true,labels=none,Dx=1.,Dy=1.,ticksize=0pt 0,subticks=2]{->}(0,0)(-3.7,-3)(3.7,3)[x,140] [$f_3(x)$,-40]
\psplot[linewidth=0.8pt,plotpoints=200]{-3.7}{3.7}{-0.016196514837819188*x^(4.0)+0.16760697032436161*x^(3.0)+0.14576863354037267*x^(2.0)-1.5084627329192546*x}
\rput[bl](2.5,0.3){$f_3$}
\end{pspicture*}}}, 				%Moeglichkeit C  
				D={\resizebox{1\linewidth}{!}{\psset{xunit=1.0cm,yunit=1.0cm,algebraic=true,dimen=middle,dotstyle=o,dotsize=5pt 0,linewidth=1.6pt,arrowsize=3pt 2,arrowinset=0.25}
\begin{pspicture*}(-3.7,-1.5)(3.7,1.5)
\psaxes[labelFontSize=\scriptstyle,xAxis=true,yAxis=true,labels=none,Dx=1.,Dy=1.,ticksize=0pt 0,subticks=2]{->}(0,0)(-3.7,-1.5)(3.7,1.5)[x,140] [$f_4(x)$,-40]
\psplot[linewidth=0.8pt,plotpoints=200]{-3.7}{3.7}{SIN(2*x)}
\rput[bl](2.7,0.3){$f_4$}
\end{pspicture*}}}, 				%Moeglichkeit D  
				E={\resizebox{1\linewidth}{!}{\psset{xunit=1.0cm,yunit=1.0cm,algebraic=true,dimen=middle,dotstyle=o,dotsize=5pt 0,linewidth=1.6pt,arrowsize=3pt 2,arrowinset=0.25}
\begin{pspicture*}(-3.7,-2)(3.7,3)
\psaxes[labelFontSize=\scriptstyle,xAxis=true,yAxis=true,labels=none,Dx=1.,Dy=1.,ticksize=0pt 0,subticks=2]{->}(0,0)(-3.7,-3)(3.7,3)[x,140] [$f_5(x)$,-40]
\psplot[linewidth=0.8pt,plotpoints=200]{-3.7}{3.7}{-x^2+2.3}
\rput[bl](1.3,1.5){$f_5$}
\end{pspicture*}}}, 				%Moeglichkeit E  
				F={\resizebox{1\linewidth}{!}{\psset{xunit=1.0cm,yunit=1.0cm,algebraic=true,dimen=middle,dotstyle=o,dotsize=5pt 0,linewidth=1.6pt,arrowsize=3pt 2,arrowinset=0.25}
\begin{pspicture*}(-3.7,-2)(3.7,3)
\psaxes[labelFontSize=\scriptstyle,xAxis=true,yAxis=true,labels=none,Dx=1.,Dy=1.,ticksize=0pt 0,subticks=2]{->}(0,0)(-3.7,-3)(3.7,3)[x,140] [$f_6(x)$,-40]
\psplot[linewidth=0.8pt,plotpoints=200]{-3.7}{3.7}{x^3}
\rput[bl](1.5,1.5){$f_6$}
\end{pspicture*}}}, 				%Moeglichkeit F  
				%% LOESUNG: %%
				A1={F},				% 1. richtige Zuordnung
				A2={E},				% 2. richtige Zuordnung
				A3={A},				% 3. richtige Zuordnung
				A4={B},				% 4. richtige Zuordnung
				}
				
\end{beispiel}