\section{FA 1.5 - 25 - MAT - Kostenfunktion - OA - Matura 2019/20 1. HT}

\begin{beispiel}[FA 1.5]{1}
Die Gesamtkosten, die bei der Herstellung eines Produkts anfallen, können mithilfe einer differenzierbaren Kostenfunktion $K$ modelliert werden. Dabei ordnet $K$ der Produktionsmenge $x$ die Kosten $K(x)$ zu ($x$ in Mengeneinheiten (ME), $K(x)$ in Geldeinheiten (GE)).

Für eine Kostenfunktion $K$: $[0;x_2]\rightarrow\mathbb{R}$ und $x_1$ mit $0<x_1<x_2$ gelten nachstehende Bedingungen:
\begin{itemize}
\item $K$ ist im Intervall $[0;x_2]$ streng monoton steigend.
\item Die Fixkosten betragen 10\,GE
\item Die Kostenfunktion hat im Intervall $[0;x_1)$ einen degressiven Verlauf, d.h. die Kosten steigen bei zunehmender Produktionsmenge immer schwächer.
\item Bei der Produktionsmenge $x_1$ liegt die Kostenkehre. Die Kostenkehre von $K$ ist diejenige Stelle, ab der die Kosten immer stärker steigen.
\end{itemize}

Skizziere im nachstehenden Koordinatensystem den Verlauf des Graphen einer solchen Kostenfunktion $K$.

\begin{center}
\psset{xunit=1.0cm,yunit=0.2cm,algebraic=true,dimen=middle,dotstyle=o,dotsize=5pt 0,linewidth=1.6pt,arrowsize=3pt 2,arrowinset=0.25}
\begin{pspicture*}(-0.58,-2.8210526315789552)(10.08,32.53614035087735)
\multips(0,0)(0,5.0){8}{\psline[linestyle=dashed,linecap=1,dash=1.5pt 1.5pt,linewidth=0.4pt,linecolor=gray]{c-c}(0,0)(10.08,0)}
\multips(0,0)(1.0,0){12}{\psline[linestyle=dashed,linecap=1,dash=1.5pt 1.5pt,linewidth=0.4pt,linecolor=gray]{c-c}(0,0)(0,32.53614035087735)}
\psaxes[labelFontSize=\scriptstyle,xAxis=true,yAxis=true,labels=y,Dx=1.,Dy=5.,ticksize=-2pt 0,subticks=2]{->}(0,0)(0.,0.)(10.08,32.53614035087735)[$x$ in ME,140] [$K(x)$ in GE,-40]
\psplot[linewidth=2.pt,plotpoints=200]{0}{7}{0.155*(x-4.0)^(3.0)+x+20.0}
\rput[tl](3.92,-0.5){$x_1$}
\rput[tl](6.9,-0.5){$x_2$}
\begin{scriptsize}
\rput[bl](1,14){$K$}
\end{scriptsize}
\end{pspicture*}
\end{center}

\antwort{\textbf{Lösungsschlüssel:}\\
Ein Punkt für die Darstellung des Graphen einer solchen Funktion $K$, der folgende Bedingungen erfüllt:
\begin{itemize}
\item Er muss im Punkt $(0\mid 10)$ beginnen.
\item Er muss im Intervall $[0;x_2]$ streng monoton steigend sein.
\item Er muss an der Stelle $x_1$ eine Wendestelle aufweisen.
\item Er muss im Intervall $(0;x_1)$ rechtsgekrümmt und im Intervall $(x_1;x_2)$ linksgekrümmt sein.
\end{itemize}}
\end{beispiel}