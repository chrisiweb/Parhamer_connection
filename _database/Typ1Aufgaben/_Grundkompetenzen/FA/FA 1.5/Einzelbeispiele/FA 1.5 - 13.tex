\section{FA 1.5 - 13 Den Graphen einer Polynomfunktion skizzieren - OA - Matura 2014/15 - Haupttermin}

\begin{beispiel}[FA 1.5]{1} %PUNKTE DES BEISPIELS
Eine Polynomfunktion $f$ hat folgende Eigenschaften: 

\begin{itemize}
	\item Die Funktion ist f�r $x \leq 0$ streng monoton steigend. 
	\item Die Funktion ist im Intervall $[0; 3]$ streng monoton fallend.
	\item Die Funktion ist f�r $x \geq 3$ streng monoton steigend.
	\item Der Punkt $P = (0|1)$ ist ein lokales Maximum (Hochpunkt).
	\item Die Stelle 3 ist eine Nullstelle.
\end{itemize}

Erstelle anhand der gegebenen Eigenschaften eine Skizze eines m�glichen Funktionsgraphen
von $f$ im Intervall $[-2; 4]$.

\leer

\begin{center}
\psset{xunit=1.0cm,yunit=1.0cm,algebraic=true,dimen=middle,dotstyle=o,dotsize=5pt 0,linewidth=0.8pt,arrowsize=3pt 2,arrowinset=0.25}
\begin{pspicture*}(-4.5,-4.5)(4.5,4.5)
\multips(0,-4)(0,1.0){10}{\psline[linestyle=dashed,linecap=1,dash=1.5pt 1.5pt,linewidth=0.4pt,linecolor=lightgray]{c-c}(-4.5,0)(4.5,0)}
\multips(-4,0)(1.0,0){10}{\psline[linestyle=dashed,linecap=1,dash=1.5pt 1.5pt,linewidth=0.4pt,linecolor=lightgray]{c-c}(0,-4.5)(0,4.5)}
\psaxes[labelFontSize=\scriptstyle,xAxis=true,yAxis=true,Dx=1.,Dy=1.,ticksize=-2pt 0,subticks=2]{->}(0,0)(-4.5,-4.5)(4.5,4.5)[$x$,140] [$f(x)$,-40]
\antwort{\psplot[linewidth=1.2pt,linecolor=red,plotpoints=200]{-2}{4}{0.0761904761904762*x^(3.0)-0.34285714285714286*x^(2.0)+0.009523809523809525*x+1.0}
\rput[tl](-1.52,1.16){$\red{f}$}
\begin{scriptsize}
\psdots[dotstyle=*,linecolor=red](0.,1.)
\rput[bl](0.08,1.2){\red{$P$}}
\psdots[dotstyle=*,linecolor=red](3.,0.)
\rput[bl](3.08,0.2){\red{$N$}}
\end{scriptsize}}
\end{pspicture*}
\end{center}


\end{beispiel}