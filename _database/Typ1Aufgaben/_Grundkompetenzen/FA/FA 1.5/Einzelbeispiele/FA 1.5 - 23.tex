\section{FA 1.5 - 23 - MAT - Arbeitslosenrate - MC - Matura-HT-18/19}

\begin{beispiel}[FA 1.5]{1}
Ein Politiker, der die erfolgreiche Arbeitsmarktpolitik einer Regierungspartei hervorheben möchte, sagt: "`Die Zunahme der Arbeitslosenrate verringerte sich während des ganzen Jahres."'\\
Ein Politiker der Opposition sagt darauf: "`Die Arbeitslosenrate ist während des ganzen Jahres gestiegen."'

Die Entwicklung der Arbeitslosenrate während dieses Jahres kann durch eine Funktion $f$ in Abhängigkeit von der Zeit modelliert werden.\\
Welche er nachstehenden Graphen stellt die Entwicklung der Arbeitslosenrate während dieses Jahres dar, wenn die Aussagen beider Politiker zutreffen?\\
Kreuze den zutreffenden Graphen an.

\langmultiplechoice[6]{  %Anzahl der Antwortmoeglichkeiten, Standard: 5
				L1={\psset{xunit=0.7cm,yunit=0.7cm,algebraic=true,dimen=middle,dotstyle=o,dotsize=5pt 0,linewidth=1.pt,arrowsize=3pt 2,arrowinset=0.25}
\begin{pspicture*}(-0.48,-0.38)(5.5,5.32)
\begin{scriptsize}
\psaxes[labelFontSize=\scriptstyle,xAxis=true,yAxis=true,labels=none,Dx=1.,Dy=1.,ticksize=0pt 0,subticks=2]{->}(0,0)(-0.48,-0.38)(5.5,5.32)[Zeit,140] [Arbeitslosenrate,-40]
\end{scriptsize}
\psplot[linewidth=2.pt,plotpoints=200]{0}{5.500000000000002}{0.05*x^(2.0)+3.0}
\begin{scriptsize}
\rput[bl](1.12,3.28){$f$}
\end{scriptsize}
\end{pspicture*}},   %1. Antwortmoeglichkeit 
				L2={\psset{xunit=0.7cm,yunit=0.7cm,algebraic=true,dimen=middle,dotstyle=o,dotsize=5pt 0,linewidth=1.pt,arrowsize=3pt 2,arrowinset=0.25}
\begin{pspicture*}(-0.48,-0.38)(5.5,5.32)
\begin{scriptsize}
\psaxes[labelFontSize=\scriptstyle,xAxis=true,yAxis=true,labels=none,Dx=1.,Dy=1.,ticksize=0pt 0,subticks=2]{->}(0,0)(-0.48,-0.38)(5.5,5.32)[Zeit,140] [Arbeitslosenrate,-40]
\end{scriptsize}
\psplot[linewidth=2.pt,plotpoints=200]{0}{5.500000000000002}{0.05*(x-6.0)^(2.0)+2.5}
\begin{scriptsize}
\rput[bl](1.12,3.){$f$}
\end{scriptsize}
\end{pspicture*}},   %2. Antwortmoeglichkeit
				L3={\psset{xunit=0.7cm,yunit=0.7cm,algebraic=true,dimen=middle,dotstyle=o,dotsize=5pt 0,linewidth=1.pt,arrowsize=3pt 2,arrowinset=0.25}
\begin{pspicture*}(-0.48,-0.38)(5.5,5.32)
\begin{scriptsize}
\psaxes[labelFontSize=\scriptstyle,xAxis=true,yAxis=true,labels=none,Dx=1.,Dy=1.,ticksize=0pt 0,subticks=2]{->}(0,0)(-0.48,-0.38)(5.5,5.32)[Zeit,140] [Arbeitslosenrate,-40]
\end{scriptsize}
\psplot[linewidth=2.pt,plotpoints=200]{0}{5.500000000000002}{-0.05*(x - 6)^2 + 4.5}
\begin{scriptsize}
\rput[bl](1.12,2.8){$f$}
\end{scriptsize}
\end{pspicture*}},   %3. Antwortmoeglichkeit
				L4={\psset{xunit=0.7cm,yunit=0.7cm,algebraic=true,dimen=middle,dotstyle=o,dotsize=5pt 0,linewidth=1.pt,arrowsize=3pt 2,arrowinset=0.25}
\begin{pspicture*}(-0.48,-0.38)(5.5,5.32)
\begin{scriptsize}
\psaxes[labelFontSize=\scriptstyle,xAxis=true,yAxis=true,labels=none,Dx=1.,Dy=1.,ticksize=0pt 0,subticks=2]{->}(0,0)(-0.48,-0.38)(5.5,5.32)[Zeit,140] [Arbeitslosenrate,-40]
\end{scriptsize}
\psplot[linewidth=2.pt,plotpoints=200]{0}{5.500000000000002}{0.03881692279402462*x^(3.0)-0.27024698553058885*x^(2.0)+0.6967327062927434*x+2.42256617699547}
\begin{scriptsize}
\rput[bl](1.12,3){$f$}
\end{scriptsize}
\end{pspicture*}},   %4. Antwortmoeglichkeit
				L5={\psset{xunit=0.7cm,yunit=0.7cm,algebraic=true,dimen=middle,dotstyle=o,dotsize=5pt 0,linewidth=1.pt,arrowsize=3pt 2,arrowinset=0.25}
\begin{pspicture*}(-0.48,-0.38)(5.5,5.32)
\begin{scriptsize}
\psaxes[labelFontSize=\scriptstyle,xAxis=true,yAxis=true,labels=none,Dx=1.,Dy=1.,ticksize=0pt 0,subticks=2]{->}(0,0)(-0.48,-0.38)(5.5,5.32)[Zeit,140] [Arbeitslosenrate,-40]
\end{scriptsize}
\psplot[linewidth=2.pt,plotpoints=200]{0}{5.500000000000002}{-0.05*x^(2.0)+4.0}
\begin{scriptsize}
\rput[bl](1.12,3){$f$}
\end{scriptsize}
\end{pspicture*}},	 %5. Antwortmoeglichkeit
				L6={\psset{xunit=0.7cm,yunit=0.7cm,algebraic=true,dimen=middle,dotstyle=o,dotsize=5pt 0,linewidth=1.pt,arrowsize=3pt 2,arrowinset=0.25}
\begin{pspicture*}(-0.48,-0.38)(5.5,5.32)
\begin{scriptsize}
\psaxes[labelFontSize=\scriptstyle,xAxis=true,yAxis=true,labels=none,Dx=1.,Dy=1.,ticksize=0pt 0,subticks=2]{->}(0,0)(-0.48,-0.38)(5.5,5.32)[Zeit,140] [Arbeitslosenrate,-40]
\end{scriptsize}
\psplot[linewidth=2.pt,plotpoints=200]{0}{5.500000000000002}{-0.018867541450495427*x^(3.0)+0.17045810918982574*x^(2.0)-0.010132725440091786*x+2.998355376180443-0.5}
\begin{scriptsize}
\rput[bl](1.12,3){$f$}
\end{scriptsize}
\end{pspicture*}},	 %6. Antwortmoeglichkeit
				L7={},	 %7. Antwortmoeglichkeit
				L8={},	 %8. Antwortmoeglichkeit
				L9={},	 %9. Antwortmoeglichkeit
				%% LOESUNG: %%
				A1=3,  % 1. Antwort
				A2=0,	 % 2. Antwort
				A3=0,  % 3. Antwort
				A4=0,  % 4. Antwort
				A5=0,  % 5. Antwort
				}
\end{beispiel}