\section{FA 1.5 - 4 - Monotonie einer linearen Funktion - LT - BIFIE}

\begin{beispiel}[FA 1.5]{1} %PUNKTE DES BEISPIELS
Gegeben ist die Gerade mit der Gleichung $y = -2x + 4$. Auf dieser Geraden liegen die Punkte $A = (x_A|y_A)$ und $B = (x_B|y_B)$. \leer

\lueckentext{
				text={Wenn $x_A < x_B$ ist, gilt \gap, weil die Gerade \gap ist.}, 	%Lueckentext Luecke=\gap
				L1={$y_A<y_B$}, 		%1.Moeglichkeit links  
				L2={$y_A=y_B$}, 		%2.Moeglichkeit links
				L3={$y_A>y_B$}, 		%3.Moeglichkeit links
				R1={monoton steigend}, 		%1.Moeglichkeit rechts 
				R2={monoton fallend}, 		%2.Moeglichkeit rechts
				R3={konstant}, 		%3.Moeglichkeit rechts
				%% LOESUNG: %%
				A1=3,   % Antwort links
				A2=2		% Antwort rechts 
				}
\end{beispiel}