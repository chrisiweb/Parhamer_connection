\section{FA 1.5 - 1 Funktion skizzieren - OA - MK}

\begin{beispiel}[FA 1.5]{1} %PUNKTE DES BEISPIELS
				Skizziere den Graph einer Funktion mit folgenden Eigenschaften:
								\begin{center}
				Definitionsmenge: $[-3; 4]$, Wertemenge: $[1; 3]$, Maximum: $(0|3)$
				\end{center}
				\leer
				
				
				\begin{center}
				\resizebox{0.9\linewidth}{!}{\begin{pspicture*}(-7.8,-2.96)(7.66,5.68)
				\multips(0,-2)(0,1.0){9}{\psline[linestyle=dashed,linecap=1,dash=1.5pt 1.5pt,linewidth=0.4pt,linecolor=lightgray]{c-c}(-7.8,0)(7.66,0)}
				\multips(-7,0)(1.0,0){16}{\psline[linestyle=dashed,linecap=1,dash=1.5pt 1.5pt,linewidth=0.4pt,linecolor=lightgray]{c-c}(0,-2.96)(0,5.68)}
\psaxes[labelFontSize=\scriptstyle,xAxis=true,yAxis=true,Dx=1.,Dy=1.,ticksize=-2pt 0,subticks=2]{->}(0,0)(-7.8,-2.96)(7.66,5.68)[x,140] [y,-40]
\end{pspicture*}}	
\end{center}
			
\end{beispiel}
