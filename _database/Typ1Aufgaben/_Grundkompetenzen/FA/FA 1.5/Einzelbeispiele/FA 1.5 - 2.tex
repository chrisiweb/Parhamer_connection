\section{FA 1.5 - 2 - Funktionseigenschaften erkennen - MC - BIFIE}

\begin{beispiel}[FA 1.5]{1} %PUNKTE DES BEISPIELS
Gegeben ist die Funktion $f$ mit $f(x)=x^3-2x+3$.

Kreuze die beiden für die Funktion $f$ zutreffenden Aussagen an!

\multiplechoice[5]{  %Anzahl der Antwortmoeglichkeiten, Standard: 5
				L1={Die Funktion $f$ ist an jeder Stelle monoton fallend.},   %1. Antwortmoeglichkeit 
				L2={Die Funktion $f$ besitzt kein lokales Maximum.},   %2. Antwortmoeglichkeit
				L3={Der Graph der Funktion $f$ geht durch $P=(0\mid 3)$.},   %3. Antwortmoeglichkeit
				L4={Eine Skizze des Graphen der Funktion $f$ könnte wie folgt aussehen: 
				\centering{
				\psset{xunit=0.6cm,yunit=0.6cm,algebraic=true,dimen=middle,dotstyle=o,dotsize=5pt 0,linewidth=0.8pt,arrowsize=3pt 2,arrowinset=0.25}
				\begin{pspicture*}(-4.326002991348994,-1.5288150209773705)(3.604793321575732,7.375628160721592)
				\begin{scriptsize}
				\psaxes[xAxis=true,yAxis=true,Dx=1.,Dy=1.,showorigin=false,ticksize=-2pt 0,subticks=0]{->}(0,0)(-4.326002991348994,-1.5288150209773705)(3.604793321575732,7.375628160721592)[x,140] [f(x),-40]
\psplot[linewidth=0.4pt,plotpoints=200]{-4.326002991348994}{3.604793321575732}{x^(3.0)-2.0*x+3.0}
\rput[tl](1.1264194737867552,3.817391422110277){$f$}
				\end{scriptsize}
\end{pspicture*}}},   %4. Antwortmoeglichkeit
				L5={Die Skizze des Graphen der Funktion $f$ könnte wie folgt aussehen:
				\centering{\psset{xunit=0.6cm,yunit=0.6cm,algebraic=true,dimen=middle,dotstyle=o,dotsize=5pt 0,linewidth=0.8pt,arrowsize=3pt 2,arrowinset=0.25}
\begin{pspicture*}(-4.610506737485461,-1.614177339159061)(3.370532627063107,7.614472776629269)
\begin{scriptsize}
\psaxes[xAxis=true,yAxis=true,Dx=1.,Dy=1.,showorigin=false,ticksize=-2pt 0,subticks=0]{->}(0,0)(-4.610506737485461,-1.614177339159061)(3.370532627063107,7.614472776629269)[x,140] [f(x),-40]
\rput[tl](-2.096938018075912,4.6789180678298){$f$}
\end{scriptsize}
\psplot[linewidth=0.4pt,plotpoints=200]{-4.610506737485461}{3.370532627063107}{-1.0000000000000007*x^(3.0)+1.2320704279452426E-15*x^(2.0)+2.0000000000000004*x+2.999999999999999}
\end{pspicture*}}},	 %5. Antwortmoeglichkeit
				L6={},	 %6. Antwortmoeglichkeit
				L7={},	 %7. Antwortmoeglichkeit
				L8={},	 %8. Antwortmoeglichkeit
				L9={},	 %9. Antwortmoeglichkeit
				%% LOESUNG: %%
				A1=3,  % 1. Antwort
				A2=4,	 % 2. Antwort
				A3=0,  % 3. Antwort
				A4=0,  % 4. Antwort
				A5=0,  % 5. Antwort
				}

\end{beispiel}