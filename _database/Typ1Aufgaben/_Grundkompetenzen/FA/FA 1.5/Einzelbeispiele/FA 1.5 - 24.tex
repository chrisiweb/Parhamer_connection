\section{FA 1.5 - 24 - MAT - Quadratische Funktion - LT - Matura 1.NT 2018/19}

\begin{beispiel}[FA 1.5]{1}
Gegeben ist eine quadratische Funktion $f$: $\mathbb{R}\rightarrow\mathbb{R}$ mit $f(x)=a\cdot x^2+b\cdot x+c$ $(a,b,c\in\mathbb{R}$ und $a\neq 0)$.

\lueckentext{
				text={Wenn \gap gilt, so hat die Funktion $f$ auf jeden Fall \gap.}, 	%Lueckentext Luecke=\gap
				L1={$a<0$}, 		%1.Moeglichkeit links  
				L2={$b=0$}, 		%2.Moeglichkeit links
				L3={$c>0$}, 		%3.Moeglichkeit links
				R1={einen zur senkrechten Achse symmetrischen Graphen}, 		%1.Moeglichkeit rechts 
				R2={zwei reelle Nullstellen}, 		%2.Moeglichkeit rechts
				R3={ein lokales Minimum}, 		%3.Moeglichkeit rechts
				%% LOESUNG: %%
				A1=2,   % Antwort links
				A2=1		% Antwort rechts 
				}
\end{beispiel}