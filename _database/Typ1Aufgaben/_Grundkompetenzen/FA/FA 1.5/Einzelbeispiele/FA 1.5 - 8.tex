\section{FA 1.5 - 8 - Polynomfunktion skizzieren - OA - BIFIE}

\begin{beispiel}[FA 1.5]{1} %PUNKTE DES BEISPIELS
Eine Polynomfunktion vierten Grades soll die nachstehenden Eigenschaft erfüllen:
\begin{itemize}
	\item Ihr Graph ist zur y-Achse symmetrisch.
	\item Im Intervall $(-\infty;-2)$ ist die Funktion streng monoton fallend.
	\item Ihre Wertemenge ist $[-4;\infty)$.
	\item Die Stelle $x=2$ ist eine lokale Extremstelle.
	\item An der Stelle $x=0$ berührt der Graph die x-Achse.
\end{itemize}

Skizziere den Graphen einer Polynomfunktion vierten Grades mit den oben angegebenen Eigenschaften im nachstehenden Koordinatensystem!
\leer

\begin{center}
\newrgbcolor{ccqqqq}{0.8 0. 0.}
\psset{xunit=1.0cm,yunit=1.0cm,algebraic=true,dimen=middle,dotstyle=o,dotsize=5pt 0,linewidth=0.8pt,arrowsize=3pt 2,arrowinset=0.25}
\begin{pspicture*}(-5.7180568649419055,-4.96536293979641)(5.841803065773493,3.8913660580144263)
\multips(0,-4)(0,1.0){9}{\psline[linestyle=dashed,linecap=1,dash=1.5pt 1.5pt,linewidth=0.4pt,linecolor=gray]{c-c}(-5.7180568649419055,0)(5.841803065773493,0)}
\multips(-5,0)(1.0,0){12}{\psline[linestyle=dashed,linecap=1,dash=1.5pt 1.5pt,linewidth=0.4pt,linecolor=gray]{c-c}(0,-4.96536293979641)(0,3.8913660580144263)}
\psaxes[labelFontSize=\scriptstyle,xAxis=true,yAxis=true,Dx=1.,Dy=1.,showorigin=false,ticksize=-2pt 0,subticks=0]{->}(0,0)(-5.7180568649419055,-4.96536293979641)(5.841803065773493,3.8913660580144263)[x,140] [f(x),-40]
\antwort{\psplot[linewidth=1.2pt,linecolor=ccqqqq,plotpoints=200]{-5.7180568649419055}{5.841803065773493}{0.24444444444444444*x^(4.0)-1.9777777777777779*x^(2.0)}
\begin{scriptsize}
\rput[bl](-2.8082159195211114,2.8101136848526007){\ccqqqq{$f$}}
\end{scriptsize}}
\end{pspicture*}
\end{center}
\end{beispiel}