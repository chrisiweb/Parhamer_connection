\section{FA 1.5 - 19 - Monotonie-Definition - LT - MatKon}

\begin{beispiel}[FA 1.5]{1}
Es sei $f\!:A\rightarrow\mathbb{R}$ eine reelle Funktion und $M$ eine Teilmenge von $A$. 
				
				\lueckentext{
								text={Die Funktion $f$ heißt monoton steigend in $M$, wenn für alle $x_1, x_2\in M$ gilt: $x_1<x_2 \Rightarrow$ \gap.\\
				Die Funktion $f$ heißt streng monoton fallend in $M$, wenn für alle $x_1, x_2\in M$ gilt: $x_1<x_2 \Rightarrow$ \gap.}, 	%Lueckentext Luecke=\gap
								L1={$f(x_1)<f(x_2)$}, 		%1.Moeglichkeit links  
								L2={$f(x_1)\geq f(x_2)$}, 		%2.Moeglichkeit links
								L3={$f(x_1)\leq f(x_2)$}, 		%3.Moeglichkeit links
								R1={$f(x_1)>f(x_2)$}, 		%1.Moeglichkeit rechts 
								R2={$f(x_1)\geq f(x_2)$}, 		%2.Moeglichkeit rechts
								R3={$f(x_1)\leq f(x_2)$}, 		%3.Moeglichkeit rechts
								%% LOESUNG: %%
								A1=3,   % Antwort links
								A2=1		% Antwort rechts 
								}
\end{beispiel}