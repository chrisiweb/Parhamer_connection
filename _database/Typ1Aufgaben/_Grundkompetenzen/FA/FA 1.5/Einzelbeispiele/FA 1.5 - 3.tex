\section{FA 1.5 - 3 Polynomfunktion 4. Grades - MC - BIFIE}

\begin{beispiel}[FA 1.5]{1} %PUNKTE DES BEISPIELS
Die nachstehende Abbildung zeigt den Graphen einer Polynomfunktion $f$, die vom Grad 4 ist.

\begin{center}
\psset{xunit=1.0cm,yunit=1.0cm,algebraic=true,dimen=middle,dotstyle=o,dotsize=5pt 0,linewidth=0.8pt,arrowsize=3pt 2,arrowinset=0.25}
\begin{pspicture*}(-3.74,-1.5)(3.66,4.64)
\multips(0,-1)(0,1.0){7}{\psline[linestyle=dashed,linecap=1,dash=1.5pt 1.5pt,linewidth=0.4pt,linecolor=lightgray]{c-c}(-3.74,0)(3.66,0)}
\multips(-3,0)(1.0,0){8}{\psline[linestyle=dashed,linecap=1,dash=1.5pt 1.5pt,linewidth=0.4pt,linecolor=lightgray]{c-c}(0,-1.5)(0,4.64)}
\psaxes[labelFontSize=\scriptstyle,xAxis=true,yAxis=true,Dx=1.,Dy=1.,ticksize=-2pt 0,subticks=2]{->}(0,0)(-3.74,-1.5)(3.66,4.64)[x,140] [f(x),-40]
\psplot[linewidth=0.4pt,plotpoints=200]{-3.7400000000000007}{3.660000000000002}{-0.26041666666666674*x^(4.0)-1.5945921863897833E-17*x^(3.0)+2.041666666666667*x^(2.0)+6.378368745559133E-17*x}
\rput[tl](-1.36,3.86){$f$}
\end{pspicture*}
\end{center}

Kreuze die beiden für die Funktion $f$ zutreffenden Aussagen an!

\multiplechoice[5]{  %Anzahl der Antwortmoeglichkeiten, Standard: 5
				L1={Die Funktion besitzt drei Wendepunkte.},   %1. Antwortmoeglichkeit 
				L2={Die Funktion ist symmetrisch bezüglich
der $y$-Achse.},   %2. Antwortmoeglichkeit
				L3={Die Funktion ist streng monoton steigend
für $x \in [0;4]$.},   %3. Antwortmoeglichkeit
				L4={Die Funktion besitzt einen Wendepunkt,
der gleichzeitig auch Tiefpunkt ist.},   %4. Antwortmoeglichkeit
				L5={Die Funktion hat drei Nullstellen.},	 %5. Antwortmoeglichkeit
				L6={},	 %6. Antwortmoeglichkeit
				L7={},	 %7. Antwortmoeglichkeit
				L8={},	 %8. Antwortmoeglichkeit
				L9={},	 %9. Antwortmoeglichkeit
				%% LOESUNG: %%
				A1=2,  % 1. Antwort
				A2=5,	 % 2. Antwort
				A3=0,  % 3. Antwort
				A4=0,  % 4. Antwort
				A5=0,  % 5. Antwort
				}

\end{beispiel}