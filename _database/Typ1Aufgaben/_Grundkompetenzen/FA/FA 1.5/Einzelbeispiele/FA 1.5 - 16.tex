\section{FA 1.5 - 16 - MAT - Graphen und Funktionstypen - ZO - Matura 2015/16 - Nebentermin 1}

\begin{beispiel}[FA 1.5]{1} %PUNKTE DES BEISPIELS
Im Folgenden sind die Graphen von vier Funktionen dargestellt. Weiters sind sechs Funktionstypen
angeführt, wobei die Parameter $a,b \in \mathbb{R}$ sind. \leer

Ordne den vier Graphen jeweils den entsprechenden Funktionstyp (aus A bis F) zu. 


\zuordnen[0.19]{
				R1={\resizebox{0.7\linewidth}{!}{\psset{xunit=1.0cm,yunit=1.0cm,algebraic=true,dimen=middle,dotstyle=o,dotsize=5pt 0,linewidth=0.3pt,arrowsize=3pt 2,arrowinset=0.25}
\begin{pspicture*}(-2.310813833414203,-1.0487257318417198)(2.5280167339841286,3.063574881559499)
\psaxes[labelFontSize=\scriptstyle,xAxis=true,yAxis=true,labels=none,Dx=0.5,Dy=0.5,ticksize=0pt 0,subticks=2]{->}(0,0)(-2.310813833414203,-1.0487257318417198)(2.5280167339841286,3.063574881559499)[\footnotesize$x$,140] [\footnotesize$f(x)$,-40]
\psplot[linewidth=0.4pt,plotpoints=200]{-2.310813833414203}{2.5280167339841286}{x+1.0}
\rput[tl](1.3712117586877344,2.076058439233306){\footnotesize $f$}
\end{pspicture*}}},				% Response 1
				R2={\resizebox{0.7\linewidth}{!}{\psset{xunit=1.0cm,yunit=1.0cm,algebraic=true,dimen=middle,dotstyle=o,dotsize=5pt 0,linewidth=0.3pt,arrowsize=3pt 2,arrowinset=0.25}
\begin{pspicture*}(-2.310813833414203,-1.0487257318417198)(2.5280167339841286,3.063574881559499)
\psaxes[labelFontSize=\scriptstyle,xAxis=true,yAxis=true,labels=none,Dx=0.5,Dy=0.5,ticksize=0pt 0,subticks=2]{->}(0,0)(-2.310813833414203,-1.0487257318417198)(2.5280167339841286,3.063574881559499)[\footnotesize$x$,140] [\footnotesize$f(x)$,-40]
\psplot[linewidth=0.4pt,plotpoints=200]{-2.310813833414203}{2.5280167339841286}{1.5*0.25^x}
\rput[tl](-0.6,2.076058439233306){\footnotesize $f$}
\end{pspicture*}}},				% Response 2
				R3={\resizebox{0.7\linewidth}{!}{\psset{xunit=1.0cm,yunit=1.0cm,algebraic=true,dimen=middle,dotstyle=o,dotsize=5pt 0,linewidth=0.3pt,arrowsize=3pt 2,arrowinset=0.25}
\begin{pspicture*}(-2.310813833414203,-1.0487257318417198)(2.5280167339841286,3.063574881559499)
\psaxes[labelFontSize=\scriptstyle,xAxis=true,yAxis=true,labels=none,Dx=0.5,Dy=0.5,ticksize=0pt 0,subticks=2]{->}(0,0)(-2.310813833414203,-1.0487257318417198)(2.5280167339841286,3.063574881559499)[\footnotesize$x$,140] [\footnotesize$f(x)$,-40]
\psplot[linewidth=0.4pt,plotpoints=200]{0}{2.5280167339841286}{x^(1.0/2.0)}
\rput[tl](1.3712117586877344,1.7){\footnotesize $f$}
\end{pspicture*}}},				% Response 3
				R4={\resizebox{0.7\linewidth}{!}{\psset{xunit=1.0cm,yunit=1.0cm,algebraic=true,dimen=middle,dotstyle=o,dotsize=5pt 0,linewidth=0.3pt,arrowsize=3pt 2,arrowinset=0.25}
\begin{pspicture*}(-2.310813833414203,-1.0487257318417198)(2.5280167339841286,3.063574881559499)
\psaxes[labelFontSize=\scriptstyle,xAxis=true,yAxis=true,labels=none,Dx=0.5,Dy=0.5,ticksize=0pt 0,subticks=2]{->}(0,0)(-2.310813833414203,-1.0487257318417198)(2.5280167339841286,3.063574881559499)[\footnotesize$x$,140] [\footnotesize$f(x)$,-40]
\psplot[linewidth=0.4pt,plotpoints=200]{-2.310813833414203}{2.5280167339841286}{0.5/x^(2)}
\rput[tl](0.75,1.5){\footnotesize $f$}
\end{pspicture*}}},				% Response 4
				%% Moegliche Zuordnungen: %%
				A={$f(x)=a\cdot b^x$}, 				%Moeglichkeit A  
				B={$f(x)=a\cdot x^{\frac{1}{2}}$}, 				%Moeglichkeit B  
				C={$f(x)=a\cdot \frac{1}{x^2}$}, 				%Moeglichkeit C  
				D={$f(x)=a\cdot x^2+b$}, 				%Moeglichkeit D  
				E={$f(x)=a\cdot x^3$}, 				%Moeglichkeit E  
				F={$f(x)=a\cdot x+b$}, 				%Moeglichkeit F  
				%% LOESUNG: %%
				A1={F},				% 1. richtige Zuordnung
				A2={A},				% 2. richtige Zuordnung
				A3={B},				% 3. richtige Zuordnung
				A4={C},				% 4. richtige Zuordnung
				}

\end{beispiel}