\section{FA 1.5 - 15 - Funktionen vergleichen - MC - Matura 2014/15 Kompensationsprüfung}

\begin{beispiel}[FA 1.5]{1} %PUNKTE DES BEISPIELS
				Gegeben sind fünf reelle Funktionen $f,g,h,i$ und $j$. Kreuze jene Funktionsgleichung(en) an die im gesamten Definitionsbereich monoton steigend ist/sind.
				
				\multiplechoice[5]{  %Anzahl der Antwortmoeglichkeiten, Standard: 5
								L1={$f(x)=3x$ mit $x\in\mathbb{R}$},   %1. Antwortmoeglichkeit 
								L2={$g(x)=x^3$ mit $x\in\mathbb{R}$},   %2. Antwortmoeglichkeit
								L3={$h(x)=3^{x}$ mit $x\in\mathbb{R}$},   %3. Antwortmoeglichkeit
								L4={$i(x)=\sin(3x)$ mit $x\in\mathbb{R}$},   %4. Antwortmoeglichkeit
								L5={$j(x)=\frac{1}{3}x$ mit $x\in\mathbb{R}$},	 %5. Antwortmoeglichkeit
								L6={},	 %6. Antwortmoeglichkeit
								L7={},	 %7. Antwortmoeglichkeit
								L8={},	 %8. Antwortmoeglichkeit
								L9={},	 %9. Antwortmoeglichkeit
								%% LOESUNG: %%
								A1=1,  % 1. Antwort
								A2=2,	 % 2. Antwort
								A3=3,  % 3. Antwort
								A4=5,  % 4. Antwort
								A5=0,  % 5. Antwort
								}
\end{beispiel}