\section{FA 1.5 - 9 - Funktionseigenschaften - MC - BIFIE}


\begin{beispiel}[FA 1.5]{1} %PUNKTE DES BEISPIELS
Gegeben ist der Graph einer reellen Funktion $f$, der die x-Achse an den Stellen $x_1 = 2$, $x_2 = 4$ und $x_3 = 9$ schneidet. \leer

\begin{center}
\psset{xunit=1.0cm,yunit=1.0cm,algebraic=true,dimen=middle,dotstyle=o,dotsize=5pt 0,linewidth=0.8pt,arrowsize=3pt 2,arrowinset=0.25}
\begin{pspicture*}(-3.500627492702854,-3.740157179069565)(9.60718836950893,4.546851025611436)
\multips(0,-3)(0,1.0){9}{\psline[linestyle=dashed,linecap=1,dash=1.5pt 1.5pt,linewidth=0.4pt,linecolor=darkgray]{c-c}(-3.500627492702854,0)(9.60718836950893,0)}
\multips(-3,0)(1.0,0){14}{\psline[linestyle=dashed,linecap=1,dash=1.5pt 1.5pt,linewidth=0.4pt,linecolor=darkgray]{c-c}(0,-3.740157179069565)(0,4.546851025611436)}
\psaxes[labelFontSize=\scriptstyle,xAxis=true,yAxis=true,Dx=1.,Dy=1.,showorigin=false,ticksize=-2pt 0,subticks=0]{->}(0,0)(-3.500627492702854,-3.740157179069565)(9.60718836950893,4.546851025611436)[x,140] [y,-40]
\psplot[plotpoints=200]{-3.500627492702854}{9.60718836950893}{-0.041666666666666664*x^(3.0)+0.4583333333333333*x^(2.0)-0.4166666666666667*x-3.0}
\rput[tl](2.475580347211334,-1.8){$f$}
\end{pspicture*}
\end{center}

Kreuze die zutreffende(n) Aussage(n) an.

\multiplechoice[5]{  %Anzahl der Antwortmoeglichkeiten, Standard: 5
				L1={$f$ ist im Intervall $[2; 4]$ monoton fallend.},   %1. Antwortmoeglichkeit 
				L2={$f(2) = f(9)$},   %2. Antwortmoeglichkeit
				L3={$f(-1) > f(1) $},   %3. Antwortmoeglichkeit
				L4={Zu jedem $x \in[3; 9]$ gibt es genau ein $f(x)$.},   %4. Antwortmoeglichkeit
				L5={Zu jedem $f(x) \in [3; 0]$ gibt es genau ein $x$.},	 %5. Antwortmoeglichkeit
				L6={},	 %6. Antwortmoeglichkeit
				L7={},	 %7. Antwortmoeglichkeit
				L8={},	 %8. Antwortmoeglichkeit
				L9={},	 %9. Antwortmoeglichkeit
				%% LOESUNG: %%
				A1=2,  % 1. Antwort
				A2=3,	 % 2. Antwort
				A3=4,  % 3. Antwort
				A4=0,  % 4. Antwort
				A5=0,  % 5. Antwort
				}

\end{beispiel}