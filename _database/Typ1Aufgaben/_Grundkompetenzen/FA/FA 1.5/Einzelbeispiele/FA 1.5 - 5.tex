\section{FA 1.5 - 5 Achsenschnittpunkte eines Funktionsgraphen - MC - BIFIE}

\begin{beispiel}[FA 1.5]{1} %PUNKTE DES BEISPIELS
Der Graph einer reellen Funktion f hat für $x_0 = 3$ einen Punkt mit der $x$-Achse gemeinsam. 

\leer

Kreuze diejenige Gleichung an, die diesen geometrischen Sachverhalt korrekt beschreibt.

\multiplechoice[6]{  %Anzahl der Antwortmoeglichkeiten, Standard: 5
				L1={$f(0)=3$},   %1. Antwortmoeglichkeit 
				L2={$f(3)=3$},   %2. Antwortmoeglichkeit
				L3={$f(3)=0$},   %3. Antwortmoeglichkeit
				L4={$f(3)=x_0$},   %4. Antwortmoeglichkeit
				L5={$f(0)=-3$},	 %5. Antwortmoeglichkeit
				L6={$f(x_0)=3$},	 %6. Antwortmoeglichkeit
				L7={},	 %7. Antwortmoeglichkeit
				L8={},	 %8. Antwortmoeglichkeit
				L9={},	 %9. Antwortmoeglichkeit
				%% LOESUNG: %%
				A1=3,  % 1. Antwort
				A2=0,	 % 2. Antwort
				A3=0,  % 3. Antwort
				A4=0,  % 4. Antwort
				A5=0,  % 5. Antwort
				}

\end{beispiel}