\section{FA 4.2 - 1 Skalierung der Achsen - OA - BIFIE}

\begin{beispiel}[FA 4.2]{1} %PUNKTE DES BEISPIELS
				Die unten stehende Grafik zeigt einen Ausschnitt des Graphen einer Polynomfunktion $f$ vom Grad 3. In der nebenstehenden Wertetabelle sind die Koordinaten einzelner Punkte angef�hrt.

Trage die Skalierung der Achsen so ein, dass eine �bereinstimmung mit den Werten der Tabelle und der Grafik gegeben ist! Zeichne dazu auf jeder Achse zumindest zwei ganzzahlige Werte ein!

\meinlr{\resizebox{1\linewidth}{!}{\psset{xunit=1.0cm,yunit=1.0cm,algebraic=true,dimen=middle,dotstyle=o,dotsize=5pt 0,linewidth=0.8pt,arrowsize=3pt 2,arrowinset=0.25}
\begin{pspicture*}(-4.570792985132227,-2.465844625046254)(5.760579952472768,5.4026959759175615)
\psaxes[labelFontSize=\scriptstyle,xAxis=true,yAxis=true,labels=none,Dx=1.,Dy=1.,ticksize=0pt 0,subticks=0]{->}(0,0)(-4.570792985132227,-2.465844625046254)(5.760579952472768,5.4026959759175615)[x,140] [f(x),-40]
\psplot[linewidth=1.2pt,plotpoints=200]{-4.570792985132227}{5.760579952472768}{-0.062*(x+1.0)^(2.0)*(x-5.0)}
\end{pspicture*}}}{\footnotesize \begin{longtable}{l|l}
x&y\\ \hline
-4&5.06\\ \hline
-3&2\\ \hline
-2&0.44\\ \hline
-1&0\\ \hline
0&0.31\\ \hline
1&1\\ \hline
2&1.69\\ \hline
3&2\\ \hline
4&1.56\\ \hline
5&0\end{longtable}}

\antwort{
\begin{center}
\resizebox{0.7\linewidth}{!}{
\psset{xunit=1.0cm,yunit=1.0cm,algebraic=true,dimen=middle,dotstyle=o,dotsize=5pt 0,linewidth=0.8pt,linecolor=red, arrowsize=3pt 2,arrowinset=0.25}
\begin{pspicture*}(-4.570792985132227,-2.465844625046254)(5.760579952472768,5.4026959759175615)
\psaxes[labelFontSize=\scriptstyle,xAxis=true,yAxis=true,Dx=1.,Dy=1.,ticksize=-2pt 0,subticks=2]{->}(0,0)(-4.570792985132227,-2.465844625046254)(5.760579952472768,5.4026959759175615)[x,140] [f(x),-40]
\psplot[linewidth=1.2pt,plotpoints=200]{-4.570792985132227}{5.760579952472768}{-0.062*(x+1.0)^(2.0)*(x-5.0)}
\end{pspicture*}}
\end{center}}

\antwort{Aus einer der Nullstellen ergibt sich die Skalierung der x-Achse, aus dem Punkt (1/1) die Skalierung der y-Achse. Die Aufgabe ist dann als richtig gel�st zu werten, wenn die Punkte mit ganzzahligen Koordinaten gut ablesbar sind und mindestens zwei ganzzahlige Werte auf jeder Achse eingetragen sind.}
\end{beispiel}