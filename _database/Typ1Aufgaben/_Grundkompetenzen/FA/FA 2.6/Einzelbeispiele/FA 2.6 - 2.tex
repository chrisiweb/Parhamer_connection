\section{FA 2.6 - 2 Celsius - Fahrenheit - LT - BIFIE}

\begin{beispiel}[FA 2.6]{1} %PUNKTE DES BEISPIELS
Temperaturen werden bei uns in $^\circ C$ (Celsius) gemessen; in einigen anderen L�ndern ist die Messung in $^\circ F$ (Fahrenheit) �blich.

Zwischen der Temperatur x in $^\circ C$ und der Temperatur $f(x)$ in $^\circ F$ besteht folgender Zusammenhang:
\begin{center}
	$f(x)=\frac{9}{5}\cdot x+32$
\end{center}

\lueckentext[-0.09]{
				text={Die Temperatur $^\circ C$ und jene in $^\circ F$ sind zueinander \gap, da \gap.}, 	%Lueckentext Luecke=\gap
				L1={direkt proportional}, 		%1.Moeglichkeit links  
				L2={indirekt proportional}, 		%2.Moeglichkeit links
				L3={nicht proportional}, 		%3.Moeglichkeit links
				R1={es beispielsweise bei $320\,^\circ F$ genau halb so viele $^\circ C$ hat}, 		%1.Moeglichkeit rechts 
				R2={eine Erw�rmung auf z.B. dreimal so viele $^\circ C$ weder bedeutet, dass die Temperatur auf dreimal so viele $^\circ F$ ansteigt, noch dass sie auf ein Drittel absinkt}, 		%2.Moeglichkeit rechts
				R3={eine Zunahme um $1\,^\circ C$ immer eine Erw�rmung um gleich viele $^\circ F$ bedeutet}, 		%3.Moeglichkeit rechts
				%% LOESUNG: %%
				A1=3,   % Antwort links
				A2=2		% Antwort rechts 
				}
\end{beispiel}