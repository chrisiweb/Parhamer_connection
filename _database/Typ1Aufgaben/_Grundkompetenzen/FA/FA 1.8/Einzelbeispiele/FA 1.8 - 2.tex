\section{FA 1.8 - 2 - Drehkegel - LT - BIFIE}

\begin{beispiel}[FA 1.8]{1} %PUNKTE DES BEISPIELS
Das Volumen eines Drehkegels kann durch eine Funktion $V$ in Abhängigkeit vom Radius $r$ und von der Höhe $h$ folgendermaßen angegeben werden: \mbox{$V(r,h)=\frac{1}{3}\cdot r^2\cdot \pi h$}.
\leer

\lueckentext{
				text={Das Volumen $V(r,h)$ bleibt unverändert, wenn der Radius $r$ \gap wird und die Höhe $h$ \gap wird.}, 	%Lueckentext Luecke=\gap
				L1={verdoppelt}, 		%1.Moeglichkeit links  
				L2={halbiert}, 		%2.Moeglichkeit links
				L3={vervierfacht}, 		%3.Moeglichkeit links
				R1={verdoppelt}, 		%1.Moeglichkeit rechts 
				R2={halbiert}, 		%2.Moeglichkeit rechts
				R3={vervierfacht}, 		%3.Moeglichkeit rechts
				%% LOESUNG: %%
				A1=2,   % Antwort links
				A2=3		% Antwort rechts 
				}
\end{beispiel}