\section{FA 1.8 - 5 - Volumen eines Drehzylinders - OA - Matura - 1. NT 2017/18}

\begin{beispiel}[FA 1.8]{1}
Das Volumen eines Drehzylinders kann als Funktion $V$ der beiden Gr��en $h$ und $r$ aufgefasst werden. Dabei ist $h$ die H�he des Zylinders und $r$ der Radius der Grundfl�che.

Verdoppelt man den Radius $r$ und die H�he $h$ eines Zylinders, so erh�lt man einen Zylinder, dessen Volumen $x$-mal so gro� wie jenes des urspr�nglichen Zylinders ist.

Gib $x$ an!\leer

$x=$\,\antwort[\rule{3cm}{0.3pt}]{$8$}
\end{beispiel}