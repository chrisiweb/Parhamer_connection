\section{FA 1.8 - 6 - MAT - Schwingung einer Saite - OA - Matura 1.NT 2018/19}

\begin{beispiel}[FA 1.8]{1}
Die Frequenz $f$ der Grundschwingung einer Saite eines Musikinstruments kann mithilfe der nachstehenden Formel berechnet werden.

$f=\dfrac{1}{2\cdot l}\cdot\sqrt{\dfrac{F}{\varrho\cdot A}}$\leer

$l\ldots$ Länge der Saite\\
$A\ldots$ Querschnittsfläche der Saite\\
$\varrho\ldots$ Dichte des Materials der Saite\\
$F\ldots$ Kraft, mit der die Saite gespannt ist\leer

Gib an, wie die Länge $l$ einer Saite zu ändern ist, wenn die Saite mit einer doppelt so hohen Frequenz schwingen soll und die anderen Größen $(F,\varrho, A)$ dabei konstant gehalten werden.

\antwort{Wenn die anderen Größen $(F\varrho, A)$ konstant gehalten werden, ist die Länge $l$ einer Saite zu halbieren, damit die Saite mit einer doppelt so hohen Frequenz schwingt.}
\end{beispiel}