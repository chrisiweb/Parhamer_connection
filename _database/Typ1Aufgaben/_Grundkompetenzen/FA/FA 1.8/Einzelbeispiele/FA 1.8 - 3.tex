\section{FA 1.8 - 3 Formel als Funktion interpretieren - LT - Matura 2014/15 - Kompensationspr�fung}

\begin{beispiel}[FA 1.8]{1} %PUNKTE DES BEISPIELS
				Gegeben ist folgende Formel:
				\begin{center}
					$F=\dfrac{5\cdot a�\cdot b}{3}$ mit $F,a,b\in\mathbb{R}$
				\end{center}
				
				\lueckentext{
								text={$F$ in Abh�ngigkeit von \gap beschreibt eine \gap.}, 	%Lueckentext Luecke=\gap
								L1={$a$ bei konstantem $b$}, 		%1.Moeglichkeit links  
								L2={$b$ bei konstantem $a$}, 		%2.Moeglichkeit links
								L3={$b$ mit $a=3$}, 		%3.Moeglichkeit links
								R1={quadratische Funktion}, 		%1.Moeglichkeit rechts 
								R2={konstante Funktion}, 		%2.Moeglichkeit rechts
								R3={Funktion dritten Grades}, 		%3.Moeglichkeit rechts
								%% LOESUNG: %%
								A1=1,   % Antwort links
								A2=1		% Antwort rechts 
								}
\end{beispiel}