\section{FA 5.4 - 1 Exponentialfunktion - MC - BIFIE}

\begin{beispiel}[FA 5.4]{1} %PUNKTE DES BEISPIELS
Gegeben ist die Exponentialfunktion $f$ mit $f(x)=e^x$.

\leer

Kreuze die zutreffende(n) Aussage(n) an.

\multiplechoice[5]{  %Anzahl der Antwortmoeglichkeiten, Standard: 5
				L1={Die Steigung der Tangente an der Stelle $x = 0$ des Graphen hat den Wert $0$. },   %1. Antwortmoeglichkeit 
				L2={Wird das Argument $x$ um $1$ erhöht, dann steigen die Funktionswerte auf das
$e$-Fache.},   %2. Antwortmoeglichkeit
				L3={Die Steigung der Tangente an der Stelle $x = 1$ des Graphen hat den Wert $e$.},   %3. Antwortmoeglichkeit
				L4={Wird das Argument $x$ um $1$ vermindert, dann sinken die Funktionswerte auf
das $\frac{1}{e}$-Fache.},   %4. Antwortmoeglichkeit
				L5={Der Graph von $f$ hat an jeder Stelle eine positive Krümmung.},	 %5. Antwortmoeglichkeit
				L6={},	 %6. Antwortmoeglichkeit
				L7={},	 %7. Antwortmoeglichkeit
				L8={},	 %8. Antwortmoeglichkeit
				L9={},	 %9. Antwortmoeglichkeit
				%% LOESUNG: %%
				A1=2,  % 1. Antwort
				A2=3,	 % 2. Antwort
				A3=4,  % 3. Antwort
				A4=5,  % 4. Antwort
				A5=0,  % 5. Antwort
				}
\end{beispiel}