\section{FA 5.4 - 4 - Eigenschaften einer Exponentialfunktion - MC - BIFIE Kompetenzcheck 2016}

\begin{beispiel}[FA 5.4]{1} %PUNKTE DES BEISPIELS
				Gegeben ist die Funktion $f$ mit $f(x)=50\cdot 1,97^{x}$.

Welche der folgenden Aussagen trifft/treffen auf diese Funktion zu? Kreuze die zutreffende(n) Aussage(n) an.

\multiplechoice[5]{  %Anzahl der Antwortmoeglichkeiten, Standard: 5
				L1={Der Graph der Funktion $f$ verläuft durch den Punkt $P=(50/0)$.},   %1. Antwortmoeglichkeit 
				L2={Die Funktion $f$ ist im Intervall $\left[0;5\right]$ streng monoton steigend.},   %2. Antwortmoeglichkeit
				L3={Wenn man den Wert des Arguments $x$ um 5 vergrößert, wird der Funktionswert 50-mal so groß.},   %3. Antwortmoeglichkeit
				L4={Der Funktionswert $f(x)$ ist positiv für alle $x\in\mathbb{R}$.},   %4. Antwortmoeglichkeit
				L5={Wenn man den Wert des Argument $x$ um 1 vergrößert, wird der zugehörige Funktionswert um 97\% größer.},	 %5. Antwortmoeglichkeit
				L6={},	 %6. Antwortmoeglichkeit
				L7={},	 %7. Antwortmoeglichkeit
				L8={},	 %8. Antwortmoeglichkeit
				L9={},	 %9. Antwortmoeglichkeit
				%% LOESUNG: %%
				A1=2,  % 1. Antwort
				A2=4,	 % 2. Antwort
				A3=5,  % 3. Antwort
				A4=0,  % 4. Antwort
				A5=0,  % 5. Antwort
				}
\end{beispiel}	