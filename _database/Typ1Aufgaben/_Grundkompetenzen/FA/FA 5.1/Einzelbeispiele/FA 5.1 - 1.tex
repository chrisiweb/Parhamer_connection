\section{FA 5.1 - 1 Radioaktives Element - OA - BIFIE}

\begin{beispiel}[FA 5.1]{1} %PUNKTE DES BEISPIELS
Ein radioaktives Element $X$ zerf�llt mit einer Halbwertszeit von 8 Tagen. Zum Zeitpunkt $t = 0$ sind 40\,g des radioaktiven Elements vorhanden.

Die Funktion $m$ beschreibt die zum Zeitpunkt $t$ noch vorhandene Menge von $X$.

Zeichne im gegebenen Koordinatensystem den Graphen von $m$.
\leer
 
\begin{center}
\psset{xunit=0.25cm,yunit=0.1cm,algebraic=true,dimen=middle,dotstyle=o,dotsize=5pt 0,linewidth=0.8pt,arrowsize=3pt 2,arrowinset=0.25}
\begin{pspicture*}(-2.5,-4.5)(41.88740129675805,45.29716272403959)
\multips(0,0)(0,10.0){5}{\psline[linestyle=dashed,linecap=1,dash=1.5pt 1.5pt,linewidth=0.4pt,linecolor=lightgray]{c-c}(0,0)(41.88740129675805,0)}
\multips(0,0)(4.0,0){11}{\psline[linestyle=dashed,linecap=1,dash=1.5pt 1.5pt,linewidth=0.4pt,linecolor=lightgray]{c-c}(0,0)(0,45.29716272403959)}
\psaxes[labelFontSize=\scriptstyle,xAxis=true,yAxis=true,Dx=4.,Dy=10.,ticksize=-2pt 0,subticks=2]{->}(0,0)(0.,0.)(41.88740129675805,45.29716272403959)[t in Tagen,140] [m(t) in kg,-40]
\antwort{\psplot[linewidth=1.2pt,linecolor=red,plotpoints=200]{0}{24}{2.2124242508071686E-5*x^(4.0)-0.0026895678070541075*x^(3.0)+0.13276296672568247*x^(2.0)-3.4012990063181294*x+40.0}
\begin{scriptsize}
\psdots[dotstyle=*,linecolor=red](0.,40.)
\psdots[dotstyle=*,linecolor=red](8.,20.)
\psdots[dotstyle=*,linecolor=red](16.,10.)
\psdots[dotstyle=*,linecolor=red](24.,5.)
\end{scriptsize}}
\end{pspicture*}
\end{center}

\antwort{L�sungsschl�ssel:

Ein Punkt wird f�r einen qualitativ richtigen Graphen, der durch die Punkte $A = (0|40)$,
$B = (8|20)$ und $C = (16|10)$ verl�uft, vergeben.}

\end{beispiel}