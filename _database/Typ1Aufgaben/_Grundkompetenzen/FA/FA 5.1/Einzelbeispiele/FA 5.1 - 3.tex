\section{FA 5.1 - 3 - MAT - Ausbreitung eines Ölteppichs - OA - Matura 2015/16 - Haupttermin}

\begin{beispiel}[FA 5.1]{1} %PUNKTE DES BEISPIELS
Der Flächeninhalt eines Ölteppichs beträgt momentan 1,5\,km$^2$ und wächst täglich um 5\,\%. \leer

Gib an, nach wie vielen Tagen der Ölteppich erstmals größer als 2\,km$^2$ ist.

\antwort{\leer

$1,5 \cdot 1,05^d = 2 \Rightarrow d=5,896\ldots ~ \Rightarrow$ Nach 6 Tagen ist der Ölteppich erstmals größer als 2\,km$^2$. \leer

Lösungsschlüssel: Ein Punkt für die richtige Lösung, wobei die Einheit "`Tage"' nicht angeführt sein muss. Die Aufgabe ist auch dann als richtig gelöst zu werten, wenn bei korrektem Ansatz das Ergebnis aufgrund eines Rechenfehlers nicht richtig ist.
Toleranzintervall: $[5,89;~6]$}
\end{beispiel}