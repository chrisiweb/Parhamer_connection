\section{WS 1.1 - 1 - Studiendauer - MC - BIFIE}

\begin{beispiel}[WS 1.1]{1} %PUNKTE DES BEISPIELS
Das nachstehende Kastenschaubild (Boxplot) zeigt die Studiendauer in Semestern für eine technische Studienrichtung.\\

\resizebox{1\linewidth}{!}{\newrgbcolor{uuuuuu}{0.26666666666666666 0.26666666666666666 0.26666666666666666}
\psset{xunit=1.0cm,yunit=1.0cm,algebraic=true,dimen=middle,dotstyle=o,dotsize=5pt 0,linewidth=0.8pt,arrowsize=3pt 2,arrowinset=0.25}
\begin{pspicture*}(11.174341159747245,-0.9390367939586384)(20.985619227058667,1.5233020741635206)
\psaxes[labelFontSize=\scriptstyle,xAxis=true,yAxis=false,Dx=2.,Dy=1.,ticksize=-2pt 0,subticks=2]{}(0,0)(11.174341159747245,-0.9390367939586384)(20.985619227058667,1.5233020741635206)
\psframe[linecolor=darkgray,fillcolor=darkgray,fillstyle=solid,opacity=0.1](14.,0.29999999999999993)(17.,1.1)
\psline[linecolor=darkgray,fillcolor=darkgray,fillstyle=solid,opacity=0.1](12.,0.3)(12.,1.1)
\psline[linecolor=darkgray,fillcolor=darkgray,fillstyle=solid,opacity=0.1](20.,0.3)(20.,1.1)
\psline[linecolor=darkgray,fillcolor=darkgray,fillstyle=solid,opacity=0.1](15.,0.3)(15.,1.1)
\psline[linecolor=darkgray,fillcolor=darkgray,fillstyle=solid,opacity=0.1](12.,0.7)(14.,0.7)
\psline[linecolor=darkgray,fillcolor=darkgray,fillstyle=solid,opacity=0.1](17.,0.7)(20.,0.7)
\end{pspicture*}}\\

Welche Aussagen kannst du diesem Kastenschaubild entnehmen? Kreuze die zutreffende(n) Aussage(n) an.\\

\multiplechoice[5]{  %Anzahl der Antwortmoeglichkeiten, Standard: 5
				L1={Die Spannweite beträgt 12 Semester.},   %1. Antwortmoeglichkeit 
				L2={25\% der Studierenden studieren höchstens 14 Semester lang.},   %2. Antwortmoeglichkeit
				L3={$\frac{1}{4}$ der Studierenden benötigt für den Abschluss des Studiums mindestens 17 Semester.},   %3. Antwortmoeglichkeit
				L4={Mindestens 50\% der Studierenden benötigen für den Abschluss des Studiums zwischen 15 und 17 Semestern.},   %4. Antwortmoeglichkeit
				L5={Es gibt Studierende, die ihr Studium erst nach 10 Jahren beenden.},	 %5. Antwortmoeglichkeit
				L6={},	 %6. Antwortmoeglichkeit
				L7={},	 %7. Antwortmoeglichkeit
				L8={},	 %8. Antwortmoeglichkeit
				L9={},	 %9. Antwortmoeglichkeit
				%% LOESUNG: %%
				A1=2,  % 1. Antwort
				A2=3,	 % 2. Antwort
				A3=5,  % 3. Antwort
				A4=0,  % 4. Antwort
				A5=0,  % 5. Antwort
				}
\end{beispiel}