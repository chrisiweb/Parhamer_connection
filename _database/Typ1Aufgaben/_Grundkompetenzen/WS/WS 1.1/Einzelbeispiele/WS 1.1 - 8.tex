\section{WS 1.1 - 8 - MAT - Anzahl der Heizungstage - MC - Matura 2. NT 2014/15}

\begin{beispiel}[WS 1.1]{1} %PUNKTE DES BEISPIELS
				Die Körpergrößen der $450$ SchülerInnen einer Schulstufe einer Gemeinde wurden in Zentimetern gemessen und deren Verteilung wurde in einem Kastenschaubild (Boxplot) grafisch dargestellt.
				\begin{center}

\psset{xunit=0.53cm,yunit=0.7cm,algebraic=true,dimen=middle,dotstyle=o,dotsize=5pt 0,linewidth=0.8pt,arrowsize=3pt 2,arrowinset=0.25}
\begin{pspicture*}(162.71202623579816,-2.7882401108569232)(187.56415568828166,3.3920920819306586)
\multips(0,0)(0,10.0){1}{\psline[linestyle=dashed,linecap=1,dash=1.5pt 1.5pt,linewidth=0.4pt,linecolor=gray]{c-c}(162.71202623579816,0)(187.56415568828166,0)}
\multips(162,0)(1.0,0){26}{\psline[linestyle=dashed,linecap=1,dash=1.5pt 1.5pt,linewidth=0.4pt,linecolor=gray]{c-c}(0,0)(0,3.3920920819306586)}
\psaxes[labelFontSize=\scriptstyle,xAxis=true,yAxis=true,Dx=2.,Dy=2.,ticksize=-2pt 0,subticks=2]{}(0,0)(162.71202623579816,-2.7882401108569232)(187.56415568828166,3.3920920819306586)
\psframe[linecolor=darkgray,fillcolor=darkgray,fillstyle=solid,opacity=0.1](164.,1)(178.,3)
\psline[linecolor=darkgray,fillcolor=darkgray,fillstyle=solid,opacity=0.1](164.,1)(164.,3)
\psline[linecolor=darkgray,fillcolor=darkgray,fillstyle=solid,opacity=0.1](185.,1)(185.,3)
\psline[linecolor=darkgray,fillcolor=darkgray,fillstyle=solid,opacity=0.1](170.,1)(170.,3)
\psline[linecolor=darkgray,fillcolor=darkgray,fillstyle=solid,opacity=0.1](178.,2)(185.,2)
\begin{scriptsize}
\rput[tl](182,-1.382132786571918){Körpergrößen in cm}
\end{scriptsize}
\end{pspicture*}
				\end{center}\vspace{-0,5cm}
				
				Zur Interpretation dieses Kastenschaubilds werden verschiedene Aussagen getätigt. Kreuze die beiden zutreffenden Aussagen an.\\
				\multiplechoice[5]{  %Anzahl der Antwortmoeglichkeiten, Standard: 5
								L1={$60\,\%$ der SchülerInnen sind genau $172\,cm$ groß.},   %1. Antwortmoeglichkeit 
								L2={Mindestens eine Schülerin bzw. ein Schüler ist genau $185\,cm$ groß.},   %2. Antwortmoeglichkeit
								L3={Höchstens $50\,\%$ der SchülerInnen sind kleiner als $170\,cm$.},   %3. Antwortmoeglichkeit
								L4={Mindestens $75\,\%$ der SchülerInnen sind größer als $178\,cm$.},   %4. Antwortmoeglichkeit
								L5={Höchstens $50\,\%$ der SchülerInnen sind mindestens $164\,cm$ und höchstens $178\,cm$ groß.},	 %5. Antwortmoeglichkeit
								L6={},	 %6. Antwortmoeglichkeit
								L7={},	 %7. Antwortmoeglichkeit
								L8={},	 %8. Antwortmoeglichkeit
								L9={},	 %9. Antwortmoeglichkeit
								%% LOESUNG: %%
								A1=2,  % 1. Antwort
								A2=3,	 % 2. Antwort
								A3=0,  % 3. Antwort
								A4=0,  % 4. Antwort
								A5=0,  % 5. Antwort
								}
\end{beispiel}