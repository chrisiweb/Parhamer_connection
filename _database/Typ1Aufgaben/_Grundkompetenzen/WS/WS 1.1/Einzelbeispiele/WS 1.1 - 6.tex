\section{WS 1.1 - 6 Internetplattform - MC - Matura 2014/15 - Haupttermin}

\begin{beispiel}[WS 1.1]{1} %PUNKTE DES BEISPIELS
Die Nutzung einer bestimmten Internetplattform durch Jugendliche wird für Mädchen und
Burschen getrennt untersucht. Dabei wird erfasst, wie oft die befragten Jugendlichen diese
Plattform pro Woche besuchen. Die nachstehenden Kastenschaubilder (Boxplots) zeigen das
Ergebnis der Untersuchung. \leer

\begin{center}
\resizebox{0.9\linewidth}{!}{
\psset{xunit=0.2cm,yunit=0.3cm,algebraic=true,dimen=middle,dotstyle=o,dotsize=5pt 0,linewidth=0.8pt,arrowsize=3pt 2,arrowinset=0.25}
\begin{pspicture*}(-7,-9)(42,3)
\multips(0,0)(1.0,0){41}{\psline[linecap=1,dash=1.5pt 1.5pt,linewidth=0.4pt,linecolor=lightgray]{c-c}(0,-9)(0,0)}
\psaxes[labelFontSize=\scriptstyle,xAxis=true,yAxis=false,xlabelPos=top, Dx=5.,Dy=2.,ticksize=-2pt 0,subticks=5]{->}(0,0)(0,-17.48305948624292)(42,0)
\psframe[fillcolor=white,fillstyle=solid,opacity=1](8.,-4.0)(25.,-2.)
\psframe[fillcolor=white,fillstyle=solid,opacity=1](4.,-8.0)(20.,-6.)
\psline[fillcolor=black,fillstyle=solid,opacity=0.3](0.,-4.)(0.,-2.)
\psline[fillcolor=black,fillstyle=solid,opacity=0.3](32.,-4.)(32.,-2.)
\psline[fillcolor=black,fillstyle=solid,opacity=0.3](14.,-4.)(14.,-2.)
\psline[fillcolor=black,fillstyle=solid,opacity=0.3](0.,-3.)(8.,-3.)
\psline[fillcolor=black,fillstyle=solid,opacity=0.3](25.,-3.)(32.,-3.)
\psline[fillcolor=black,fillstyle=solid,opacity=0.3](0.,-8.)(0.,-6.)
\psline[fillcolor=black,fillstyle=solid,opacity=0.3](32.,-8.)(32.,-6.)
\psline[fillcolor=black,fillstyle=solid,opacity=0.3](12.,-8.)(12.,-6.)
\psline[fillcolor=black,fillstyle=solid,opacity=0.3](0.,-7.)(4.,-7.)
\psline[fillcolor=black,fillstyle=solid,opacity=0.3](20.,-7.)(32.,-7.)
\rput[tl](28,3){\scriptsize Besuche pro Woche}
\rput[tl](-7,-2.6360466036086576){\scriptsize Burschen}
\rput[tl](-7,-6.717031820039541){\scriptsize Mädchen}
\end{pspicture*}}

\end{center}
Kreuze die beiden zutreffenden Aussagen an.

\multiplechoice[5]{  %Anzahl der Antwortmoeglichkeiten, Standard: 5
				L1={Der Median der Anzahl von Besuchen pro Woche ist bei den Burschen
etwas höher als bei den Mädchen.},   %1. Antwortmoeglichkeit 
				L2={Die Spannweite der wöchentlichen Nutzung der Plattform ist bei den
Burschen größer als bei den Mädchen.},   %2. Antwortmoeglichkeit
				L3={Aus der Grafik kann man ablesen, dass genauso viele Mädchen wie
Burschen die Plattform wöchentlich besuchen.},   %3. Antwortmoeglichkeit
				L4={Der Anteil der Burschen, die mehr als 20-mal pro Woche die Plattform
nützen, ist zumindest gleich groß oder größer als jener der Mädchen.},   %4. Antwortmoeglichkeit
				L5={Ca. 80\,\% der Mädchen und ca. 75\,\% der Burschen nützen die Plattform
genau 25-mal pro Woche.},	 %5. Antwortmoeglichkeit
				L6={},	 %6. Antwortmoeglichkeit
				L7={},	 %7. Antwortmoeglichkeit
				L8={},	 %8. Antwortmoeglichkeit
				L9={},	 %9. Antwortmoeglichkeit
				%% LOESUNG: %%
				A1=1,  % 1. Antwort
				A2=4,	 % 2. Antwort
				A3=0,  % 3. Antwort
				A4=0,  % 4. Antwort
				A5=0,  % 5. Antwort
				}
\end{beispiel}