\section{WS 1.1 - 11 - MAT - Computer- und Videospiele - MC - Matura 1. NT 2013/14}

\begin{beispiel}[WS 1.1]{1} %PUNKTE DES BEISPIELS
				Computer- und Videospiele müssen vor ihrer Markteinführung ein Einstufungsverfahren durchlaufen, bei dem festgelegt wird, welches Mindestalter für den Erwerb des Spiels erreicht sein muss. Im Jahr 2009 wurden 3 100 Spiele dieser Einstufung unterzogen. Im Jahr 2008 waren es um 114 Spiele weniger. Die nachstehende Graphik stellt die Ergebnisse der Auswertungen dar.
				
				\begin{center}
					\textbf{Verteilung der Freigaben für die Jahre 2008 und 2009}
					
					\resizebox{0.7\linewidth}{!}{\newrgbcolor{aqaqaq}{0.6274509803921569 0.6274509803921569 0.6274509803921569}
\psset{xunit=1.0cm,yunit=0.1cm,algebraic=true,dimen=middle,dotstyle=o,dotsize=5pt 0,linewidth=0.8pt,arrowsize=3pt 2,arrowinset=0.25}
\begin{pspicture*}(-1.4360108987544093,-30.11026597063495)(12.745094125926263,71.23077345736026)
\multips(0,0)(0,10.0){12}{\psline[linestyle=dashed,linecap=1,dash=1.5pt 1.5pt,linewidth=0.4pt,linecolor=lightgray]{c-c}(0,0)(12.745094125926263,0)}
\multips(0,0)(100.0,0){1}{\psline[linestyle=dashed,linecap=1,dash=1.5pt 1.5pt,linewidth=0.4pt,linecolor=lightgray]{c-c}(0,0)(0,71.23077345736026)}
\pspolygon[fillcolor=black,fillstyle=solid,opacity=1.0](0.5,0.)(0.99,0.)(0.99,48.6)(0.5,48.6)
\pspolygon[linecolor=aqaqaq,fillcolor=aqaqaq,fillstyle=solid,opacity=1.0](1.02,0.)(1.02,52.5)(1.5,52.5)(1.5,0.)
\pspolygon[fillcolor=black,fillstyle=solid,opacity=1.0](2.5,0.)(2.5,14.5)(2.99,14.5)(2.99,0.)
\pspolygon[linecolor=aqaqaq,fillcolor=aqaqaq,fillstyle=solid,opacity=1.0](3.02,0.)(3.5,0.)(3.5,12.7)(3.02,12.7)
\pspolygon[fillcolor=black,fillstyle=solid,opacity=1.0](4.5,0.)(4.99,0.)(4.99,19.5)(4.5,19.5)
\pspolygon[linecolor=aqaqaq,fillcolor=aqaqaq,fillstyle=solid,opacity=1.0](5.02,0.)(5.5,0.)(5.5,17.5)(5.02,17.5)
\pspolygon[fillcolor=black,fillstyle=solid,opacity=1.0](6.5,0.)(6.99,0.)(6.99,10.8)(6.5,10.8)
\pspolygon[linecolor=aqaqaq,fillcolor=aqaqaq,fillstyle=solid,opacity=1.0](7.02,0.)(7.5,0.)(7.5,10.4)(7.02,10.4)
\pspolygon[fillcolor=black,fillstyle=solid,opacity=1.0](8.5,0.)(8.99,0.)(8.99,5.2)(8.5,5.2)
\pspolygon[linecolor=aqaqaq,fillcolor=aqaqaq,fillstyle=solid,opacity=1.0](9.02,0.)(9.5,0.)(9.5,5.8)(9.02,5.8)
\pspolygon[fillcolor=black,fillstyle=solid,opacity=1.0](10.5,0.)(10.99,0.)(10.99,1.4)(10.5,1.4)
\pspolygon[linecolor=aqaqaq,fillcolor=aqaqaq,fillstyle=solid,opacity=1.0](11.02,0.)(11.5,0.)(11.5,1.1)(11.02,1.1)
\pspolygon[fillcolor=black,fillstyle=solid,opacity=1.0](4.2,60.)(4.2,57.5)(4.5,57.5)(4.5,60.)
\pspolygon[linecolor=aqaqaq,fillcolor=aqaqaq,fillstyle=solid,opacity=1.0](4.2,55.)(4.2,52.5)(4.5,52.5)(4.5,55.)
\psaxes[labelFontSize=\scriptstyle,xAxis=true,yAxis=true,labels=y,ylabelFactor={\,\%},Dx=2.,Dy=10.,ticksize=-2pt 0]{->}(0,0)(0.,0.)(12.745094125926263,71.23077345736026)
\begin{scriptsize}
\rput[tl](4.754502440192882,60.018700096128484){2008}
\rput[tl](4.732550265232075,55.035556380025476){2009}
\rput[tl](-0.8,-0.4){$\rotatebox{60}{\text{ohne Altersbeschränkung}}$}
\rput[tl](2.0,-0.4){$\rotatebox{60}{\text{ab 6 Jahren}}$}
\rput[tl](4.0,-0.4){$\rotatebox{60}{\text{ab 12 Jahren}}$}
\rput[tl](6.0,-0.4){$\rotatebox{60}{\text{ab 16 Jahren}}$}
\rput[tl](7.5,-0.4){$\rotatebox{60}{\text{keine Jugendfreigabe}}$}
\rput[tl](9.5,-0.4){$\rotatebox{60}{\text{keine Kennzeichnung}}$}
\end{scriptsize}
\begin{tiny}
\rput[tl](0.32,51){$48,6\%$}
\rput[tl](1.0,54.9){$52,5\%$}
\rput[tl](2.32,16.9){$14,5\%$}
\rput[tl](3,15.1){$12,7\%$}
\rput[tl](4.32,21.9){$19,5\%$}
\rput[tl](5,19.9){$17,5\%$}
\rput[tl](6.32,13.2){$10,8\%$}
\rput[tl](7,12.8){$10,4\%$}
\rput[tl](8.45,7.6){$5,2\%$}
\rput[tl](9,8.2){$5,8\%$}
\rput[tl](10.5,3.6){$1,4\%$}
\rput[tl](11,3.3){$1,1\%$}
\end{tiny}
\end{pspicture*}}

\begin{scriptsize}Datenquelle: http://www.usk.de/pruefverfahren/statistik/jahresbilanz-2009/ [21.05.2014]\end{scriptsize}
				\end{center}

Kreuze die beiden zutreffenden Aussagen an!\leer

\multiplechoice[5]{  %Anzahl der Antwortmoeglichkeiten, Standard: 5
				L1={Die Anzahl der im Jahr 2009 ohne Altersbeschränkung freigegebenen Spiele hat sich im Vergleich zum Jahr 2008 um etwa 10\,\% verringert. },   %1. Antwortmoeglichkeit 
				L2={Die Anzahl der in der Kategorie "`freigegeben ab 16 Jahren"' eingestuften Spiele ist in den beiden Jahren 2008 und 2009 nahezu gleich.},   %2. Antwortmoeglichkeit
				L3={Im Jahr 2008 wurde annähernd jedes dritte Spiel für Kinder ab 6 Jahren freigegeben.},   %3. Antwortmoeglichkeit
				L4={Im Jahr 2009 wurden weniger als 500 Spiele der Kategorie "`freigegeben ab 12 Jahren"' zugeordnet.},   %4. Antwortmoeglichkeit
				L5={Im Jahr 2008 erhielt etwa jedes zwanzigste Spiel keine Jugendfreigabe.},	 %5. Antwortmoeglichkeit
				L6={},	 %6. Antwortmoeglichkeit
				L7={},	 %7. Antwortmoeglichkeit
				L8={},	 %8. Antwortmoeglichkeit
				L9={},	 %9. Antwortmoeglichkeit
				%% LOESUNG: %%
				A1=2,  % 1. Antwort
				A2=5,	 % 2. Antwort
				A3=0,  % 3. Antwort
				A4=0,  % 4. Antwort
				A5=0,  % 5. Antwort
				}
\end{beispiel}