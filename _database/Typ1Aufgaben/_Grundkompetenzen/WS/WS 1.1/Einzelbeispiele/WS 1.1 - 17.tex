\section{WS 1.1 - 17 - MAT - PKW-Dichte - OA - Matura 1.NT 2018/19}

\begin{beispiel}[WS 1.1]{1}
In 32 europäischen Ländern wurde die Anzahl der Personenkraftwagen (PKWs) pro 1\,000 Einwohner/innen erhoben. Aus diesen Daten ist das nachstehende Histogramm erstellt worden.\\
Dabei sind die absoluten Häufigkeiten der Länder als Flächeninhalte von Rechtecken dargestellt.

\begin{center}
\psset{xunit=0.013cm,yunit=70.0cm,algebraic=true,dimen=middle,dotstyle=o,dotsize=5pt 0,linewidth=1.6pt,arrowsize=3pt 2,arrowinset=0.25}
\begin{pspicture*}(-76.08129729730605,-0.016)(887.6151351352346,0.10618019058787712)
\multips(0,0)(0,0.01){12}{\psline[linestyle=dashed,linecap=1,dash=1.5pt 1.5pt,linewidth=0.4pt,linecolor=gray]{c-c}(0,0)(887.6151351352346,0)}
\multips(0,0)(100.0,0){10}{\psline[linestyle=dashed,linecap=1,dash=1.5pt 1.5pt,linewidth=0.4pt,linecolor=gray]{c-c}(0,0)(0,0.10618019058787712)}
\psaxes[comma,labelFontSize=\scriptstyle,xAxis=true,yAxis=true,Dx=200.,Dy=0.01,ticksize=-2pt 0,subticks=0]{}(0,0)(0.,0.)(887.6151351352346,0.10618019058787712)
\pspolygon[linewidth=2.pt,fillcolor=black,fillstyle=solid,opacity=0.1](0.,0.)(0.,0.024)(200.,0.024)(200.,0.)
\pspolygon[linewidth=2.pt,fillcolor=black,fillstyle=solid,opacity=0.1](200.,0.)(200.,0.06)(300.,0.06)(300.,0.)
\pspolygon[linewidth=2.pt,fillcolor=black,fillstyle=solid,opacity=0.1](300.,0.)(300.,0.06)(400.,0.06)(400.,0.)
\pspolygon[linewidth=2.pt,fillcolor=black,fillstyle=solid,opacity=0.1](400.,0.)(400.,0.09)(500.,0.09)(500.,0.)
\pspolygon[linewidth=2.pt,fillcolor=black,fillstyle=solid,opacity=0.1](500.,0.)(500.,0.03)(700.,0.03)(700.,0.)
\psline[linewidth=2.pt](0.,0.)(0.,0.024)
\psline[linewidth=2.pt](0.,0.024)(200.,0.024)
\psline[linewidth=2.pt](200.,0.024)(200.,0.)
\psline[linewidth=2.pt](200.,0.)(0.,0.)
\psline[linewidth=2.pt](200.,0.)(200.,0.06)
\psline[linewidth=2.pt](200.,0.06)(300.,0.06)
\psline[linewidth=2.pt](300.,0.06)(300.,0.)
\psline[linewidth=2.pt](300.,0.)(200.,0.)
\psline[linewidth=2.pt](300.,0.)(300.,0.06)
\psline[linewidth=2.pt](300.,0.06)(400.,0.06)
\psline[linewidth=2.pt](400.,0.06)(400.,0.)
\psline[linewidth=2.pt](400.,0.)(300.,0.)
\psline[linewidth=2.pt](400.,0.)(400.,0.09)
\psline[linewidth=2.pt](400.,0.09)(500.,0.09)
\psline[linewidth=2.pt](500.,0.09)(500.,0.)
\psline[linewidth=2.pt](500.,0.)(400.,0.)
\psline[linewidth=2.pt](500.,0.)(500.,0.03)
\psline[linewidth=2.pt](500.,0.03)(700.,0.03)
\psline[linewidth=2.pt](700.,0.03)(700.,0.)
\psline[linewidth=2.pt](700.,0.)(500.,0.)
\rput[tl](124.99070270271653,-0.009){Anzahl der PKWs pro 1\,000 Einwohner/innen}
\end{pspicture*}
\end{center}

Gib an, in wie vielen Ländern die Anzahl der PKWs pro 1\,000 Einwohner/innen zwischen 500 und 700 PKWs liegt.\leer

Anzahl der Länder\,$=\antwort[\rule{3cm}{0.3pt}]{6}$
\end{beispiel}