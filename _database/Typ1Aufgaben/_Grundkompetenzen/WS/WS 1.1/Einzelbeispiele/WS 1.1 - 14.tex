\section{WS 1.1 - 14 - MAT - Hausübungen und Schularbeit - MC - Matura HT 2017/18}

\begin{beispiel}[WS 1.1]{1} %PUNKTE DES BEISPIELS
In einer Klasse, in der ausschließlich Mädchen sind, waren bis zu einer Schularbeit 15 Hausübungen abzugeben. Bei der Schularbeit waren maximal 48 Punkte zu erreichen. Im nachstehenden Punktwolkendiagramm werden für jede der insgesamt 20 Schülerinnen dieser Klasse die Anzahl der abgegebenen Hausübungen und die Anzahl der bei der Schularbeit erreichten Punkte dargestellt.

\begin{center}
	\resizebox{0.8\linewidth}{!}{\psset{xunit=0.8cm,yunit=0.16cm,algebraic=true,dimen=middle,dotstyle=o,dotsize=5pt 0,linewidth=1.6pt,arrowsize=3pt 2,arrowinset=0.25}
\begin{pspicture*}(-1.66,-7)(16.64,51)
\multips(0,0)(0,5.0){10}{\psline[linestyle=dashed,linecap=1,dash=1.5pt 1.5pt,linewidth=0.4pt,linecolor=darkgray]{c-c}(0,0)(16,0)}
\multips(0,0)(2.0,0){8}{\psline[linestyle=dashed,linecap=1,dash=1.5pt 1.5pt,linewidth=0.4pt,linecolor=darkgray]{c-c}(0,0)(0,50)}
\psaxes[labelFontSize=\scriptstyle,xAxis=true,yAxis=true,Dx=2.,Dy=5.,ticksize=0pt 0,subticks=0](0,0)(0.,0.)(16,50)
\psline[linewidth=2.pt](16.,0.)(16.,50.)
\psline[linewidth=2.pt](16.,50.)(0.,50.)
\psline[linewidth=2.pt](0.,50.)(0.,0.)
\psline[linewidth=2.pt](0.,0.)(16.,0.)
\rput[tl](3,-4.1){Anzahl der abgegebenen Hausübungen (von 15)}
\rput[tl](-1.45,51){$\rotatebox{90}{erreichte Punkte bei der Schularbeit (von 48)}$}
\begin{scriptsize}
\psdots[dotsize=4pt 0,dotstyle=square*,dotangle=45](2.,16.)
\psdots[dotsize=4pt 0,dotstyle=square*,dotangle=45](5.,29.)
\psdots[dotsize=4pt 0,dotstyle=square*,dotangle=45](5.,31.)
\psdots[dotsize=4pt 0,dotstyle=square*,dotangle=45](6.,38.)
\psdots[dotsize=4pt 0,dotstyle=square*,dotangle=45](7.,27.)
\psdots[dotsize=4pt 0,dotstyle=square*,dotangle=45](8.,27.)
\psdots[dotsize=4pt 0,dotstyle=square*,dotangle=45](8.,22.)
\psdots[dotsize=4pt 0,dotstyle=square*,dotangle=45](8.,33.)
\psdots[dotsize=4pt 0,dotstyle=square*,dotangle=45](9.,32.)
\psdots[dotsize=4pt 0,dotstyle=square*,dotangle=45](9.,35.)
\psdots[dotsize=4pt 0,dotstyle=square*,dotangle=45](10.,40.)
\psdots[dotsize=4pt 0,dotstyle=square*,dotangle=45](11.,35.)
\psdots[dotsize=4pt 0,dotstyle=square*,dotangle=45](12.,38.)
\psdots[dotsize=4pt 0,dotstyle=square*,dotangle=45](12.,41.)
\psdots[dotsize=4pt 0,dotstyle=square*,dotangle=45](13.,41.)
\psdots[dotsize=4pt 0,dotstyle=square*,dotangle=45](13.,38.)
\psdots[dotsize=4pt 0,dotstyle=square*,dotangle=45](13.,46.)
\psdots[dotsize=4pt 0,dotstyle=square*,dotangle=45](14.,35.)
\psdots[dotsize=4pt 0,dotstyle=square*,dotangle=45](15.,43.)
\psdots[dotsize=4pt 0,dotstyle=square*,dotangle=45](15.,44.)
\end{scriptsize}
\end{pspicture*}}
\end{center}

Zwei der nachstehenden fünf Aussagen interpretieren das dargestellte Punktwolkendiagramm korrekt. Kreuzen Sie die beiden zutreffenden Aussagen an!

\multiplechoice[5]{  %Anzahl der Antwortmoeglichkeiten, Standard: 5
				L1={Nur Schülerinnen, die mehr als 10 Hausübungen abgegeben haben,
konnten mehr als 35 Punkte bei der Schularbeit erzielen.},   %1. Antwortmoeglichkeit 
				L2={Die Schülerin mit der geringsten Punkteanzahl bei der Schularbeit
hat die wenigsten Hausübungen abgegeben.},   %2. Antwortmoeglichkeit
				L3={Die Schülerin mit den meisten Punkten bei der Schularbeit hat alle
Hausübungen abgegeben.},   %3. Antwortmoeglichkeit
				L4={Schülerinnen mit mindestens 10 abgegebenen Hausübungen
haben bei der Schularbeit im Durchschnitt mehr Punkte erzielt als
jene mit weniger als 10 abgegebenen Hausübungen.},   %4. Antwortmoeglichkeit
				L5={Aus der Anzahl der bei der Schularbeit erreichten Punkte kann man
eindeutig auf die Anzahl der abgegebenen Hausübungen schließen.},	 %5. Antwortmoeglichkeit
				L6={},	 %6. Antwortmoeglichkeit
				L7={},	 %7. Antwortmoeglichkeit
				L8={},	 %8. Antwortmoeglichkeit
				L9={},	 %9. Antwortmoeglichkeit
				%% LOESUNG: %%
				A1=2,  % 1. Antwort
				A2=4,	 % 2. Antwort
				A3=0,  % 3. Antwort
				A4=0,  % 4. Antwort
				A5=0,  % 5. Antwort
				}
\end{beispiel}