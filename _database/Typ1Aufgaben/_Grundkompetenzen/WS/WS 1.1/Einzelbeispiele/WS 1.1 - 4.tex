\section{WS 1.1 - 4 Nationalratswahl - MC - BIFIE}

\begin{beispiel}[WS 1.1]{1} %PUNKTE DES BEISPIELS
				In der folgenden Abbildung sind die Ergebnisse der Nationalratswahl 2006 (linksstehende Balken) und der Nationalratswahl 2008 (rechtsstehende Balken) dargestellt. Alle Prozentsätze beziehen sich auf die Anzahl der gültigen abgegebenen Stimmen, die 2006 und 2008 ungefähr gleich war.\\

\resizebox{1\linewidth}{!}{\newrgbcolor{ffqqtt}{1. 0. 0.2}
\newrgbcolor{zzttqq}{0.6 0.2 0.}
\newrgbcolor{ttttff}{0.2 0.2 1.}
\newrgbcolor{qqccqq}{0. 0.8 0.}
\newrgbcolor{ffzztt}{1. 0.6 0.2}
\psset{xunit=1.0cm,yunit=0.2cm,algebraic=true,dimen=middle,dotstyle=o,dotsize=5pt 0,linewidth=0.8pt,arrowsize=3pt 2,arrowinset=0.25}
\begin{pspicture*}(-1.2640767792503411,-5.816017797133461)(16.092704537167783,44.556975297834015)
\multips(0,0)(0,5.0){11}{\psline[linestyle=dashed,linecap=1,dash=1.5pt 1.5pt,linewidth=0.4pt,linecolor=lightgray]{c-c}(0,0)(16.092704537167783,0)}
\multips(0,0)(100.0,0){1}{\psline[linestyle=dashed,linecap=1,dash=1.5pt 1.5pt,linewidth=0.4pt,linecolor=lightgray]{c-c}(0,0)(0,44.556975297834015)}
\psaxes[labelFontSize=\scriptstyle,xAxis=true,yAxis=true,labels=y,Dx=1.,Dy=5.,ticksize=-2pt 0,subticks=2]{}(0,0)(0.,0.)(16.092704537167783,44.556975297834015)
\pspolygon[linecolor=gray,fillcolor=gray,fillstyle=solid,opacity=1.0](0.7,0.)(0.7,35.3)(1.7,35.3)(1.7,0.)
\pspolygon[linecolor=ffqqtt,fillcolor=ffqqtt,fillstyle=solid,opacity=1.0](1.7,0.)(1.7,29.3)(2.7,29.3)(2.7,0.)
\pspolygon[linecolor=gray,fillcolor=gray,fillstyle=solid,opacity=1.0](3.7,0.)(3.7,34.3)(4.7,34.3)(4.7,0.)
\pspolygon[fillcolor=black,fillstyle=solid,opacity=1.0](4.7,0.)(4.7,26.)(5.7,26.)(5.7,0.)
\pspolygon[linecolor=gray,fillcolor=gray,fillstyle=solid,opacity=1.0](6.7,0.)(6.7,11.)(7.7,11.)(7.7,0.)
\pspolygon[linecolor=ttttff,fillcolor=ttttff,fillstyle=solid,opacity=1.0](7.7,0.)(7.7,17.5)(8.7,17.5)(8.7,0.)
\pspolygon[linecolor=gray,fillcolor=gray,fillstyle=solid,opacity=1.0](9.7,0.)(9.7,11.)(10.7,11.)(10.7,0.)
\pspolygon[linecolor=qqccqq,fillcolor=qqccqq,fillstyle=solid,opacity=1.0](10.7,0.)(10.7,10.4)(11.7,10.4)(11.7,0.)
\pspolygon[linecolor=gray,fillcolor=gray,fillstyle=solid,opacity=1.0](12.7,0.)(12.7,4.1)(13.7,4.1)(13.7,0.)
\pspolygon[linecolor=ffzztt,fillcolor=ffzztt,fillstyle=solid,opacity=1.0](13.7,0.)(13.7,10.7)(14.7,10.7)(14.7,0.)
\psline[linecolor=gray](0.7,0.)(0.7,35.3)
\psline[linecolor=gray](0.7,35.3)(1.7,35.3)
\psline[linecolor=gray](1.7,35.3)(1.7,0.)
\psline[linecolor=gray](1.7,0.)(0.7,0.)
\psline[linecolor=ffqqtt](1.7,0.)(1.7,29.3)
\psline[linecolor=ffqqtt](1.7,29.3)(2.7,29.3)
\psline[linecolor=ffqqtt](2.7,29.3)(2.7,0.)
\psline[linecolor=ffqqtt](2.7,0.)(1.7,0.)
\psline[linecolor=zzttqq](3.7,0.)(3.7,34.3)
\psline[linecolor=zzttqq](3.7,34.3)(4.7,34.3)
\psline[linecolor=zzttqq](4.7,34.3)(4.7,0.)
\psline[linecolor=zzttqq](4.7,0.)(3.7,0.)
\psline(4.7,0.)(4.7,26.)
\psline(4.7,26.)(5.7,26.)
\psline(5.7,26.)(5.7,0.)
\psline(5.7,0.)(4.7,0.)
\psline[linecolor=zzttqq](6.7,0.)(6.7,11.)
\psline[linecolor=zzttqq](6.7,11.)(7.7,11.)
\psline[linecolor=zzttqq](7.7,11.)(7.7,0.)
\psline[linecolor=zzttqq](7.7,0.)(6.7,0.)
\psline[linecolor=ttttff](7.7,0.)(7.7,17.5)
\psline[linecolor=ttttff](7.7,17.5)(8.7,17.5)
\psline[linecolor=ttttff](8.7,17.5)(8.7,0.)
\psline[linecolor=ttttff](8.7,0.)(7.7,0.)
\psline[linecolor=zzttqq](9.7,0.)(9.7,11.)
\psline[linecolor=zzttqq](9.7,11.)(10.7,11.)
\psline[linecolor=zzttqq](10.7,11.)(10.7,0.)
\psline[linecolor=zzttqq](10.7,0.)(9.7,0.)
\psline[linecolor=qqccqq](10.7,0.)(10.7,10.4)
\psline[linecolor=qqccqq](10.7,10.4)(11.7,10.4)
\psline[linecolor=qqccqq](11.7,10.4)(11.7,0.)
\psline[linecolor=qqccqq](11.7,0.)(10.7,0.)
\psline[linecolor=zzttqq](12.7,0.)(12.7,4.1)
\psline[linecolor=zzttqq](12.7,4.1)(13.7,4.1)
\psline[linecolor=zzttqq](13.7,4.1)(13.7,0.)
\psline[linecolor=zzttqq](13.7,0.)(12.7,0.)
\psline[linecolor=ffzztt](13.7,0.)(13.7,10.7)
\psline[linecolor=ffzztt](13.7,10.7)(14.7,10.7)
\psline[linecolor=ffzztt](14.7,10.7)(14.7,0.)
\psline[linecolor=ffzztt](14.7,0.)(13.7,0.)
\rput[tl](1.3640793822262207,-1.3445021057323576){SPÖ}
\rput[tl](4.35725723279675,-1.4357575280058494){ÖVP}
\rput[tl](7.386937252276674,-1.5270129502793413){FPÖ}
\rput[tl](10.380115102847203,-1.5270129502793413){GRÜNE}
\rput[tl](13.464548375691223,-1.5270129502793413){BZÖ}
\rput[tl](0.8165468485852703,37.89532947186912){35,3\%}
\rput[tl](1.802105409138981,31.781216179545165){29,3\%}
\rput[tl](3.754971445791704,36.80026440458722){34,3\%}
\rput[tl](4.813534344164207,28.40476555542597){26,0\%}
\rput[tl](6.566400380816931,13.530131724846806){11,0\%}
\rput[tl](7.751958941370641,19.82675586171774){17,5\%}
\rput[tl](9.75957823138746,13.438876302573314){11,0\%}
\rput[tl](10.799890045305265,12.982599191205855){10,4\%}
\rput[tl](12.73450499750329,6.503464209787937){4,1\%}
\rput[tl](13.720063558057,13.073854613479346){10,7\%}
\begin{scriptsize}
\rput[bl](1.1998196221339354,1.4844159847458882){\gray{$1$}}
\end{scriptsize}
\end{pspicture*}}

Überprüfe anhand der Abbildung die folgenden Aussagen und kreuze die beiden zutreffenden Aussagen an.

\multiplechoice[5]{  %Anzahl der Antwortmoeglichkeiten, Standard: 5
				L1={Das BZÖ hat seinen Stimmenanteil von 2006 auf 2008 um mehr als 100\% gesteigert.},   %1. Antwortmoeglichkeit 
				L2={Die GRÜNEN erreichten 2006 weniger Stimmenanteile als 2008.},   %2. Antwortmoeglichkeit
				L3={Der Stimmenanteil der ÖVP hat von 2006 auf 2008 um fast ein Viertel abgenommen.},   %3. Antwortmoeglichkeit
				L4={Die Anzahl der erreichten Stimmen für die SPÖ hat von 2006 auf 2008 um 6\% abgenommen.},   %4. Antwortmoeglichkeit
				L5={Das BZÖ hat von 2006 auf 2008 deutlich mehr Stimmen dazugewonnen als die FPÖ.},	 %5. Antwortmoeglichkeit
				L6={},	 %6. Antwortmoeglichkeit
				L7={},	 %7. Antwortmoeglichkeit
				L8={},	 %8. Antwortmoeglichkeit
				L9={},	 %9. Antwortmoeglichkeit
				%% LOESUNG: %%
				A1=1,  % 1. Antwort
				A2=3,	 % 2. Antwort
				A3=0,  % 3. Antwort
				A4=0,  % 4. Antwort
				A5=0,  % 5. Antwort
				}
				\end{beispiel}