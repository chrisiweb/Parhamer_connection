\section{WS 1.1 - 12 Wanderungsbilanz f�r �sterreich - MC - Matura 2016/17 - Haupttermin}

\begin{langesbeispiel} \item[1] %PUNKTE DES BEISPIELS
Die Differenz aus der Anzahl der in einem bestimmten Zeitraum in ein Land zugewanderten
Personen und der Anzahl der in diesem Zeitraum aus diesem Land abgewanderten Personen
bezeichnet man als Wanderungsbilanz. 

In der nachstehenden Grafik ist die j�hrliche Wanderungsbilanz f�r �sterreich in den Jahren von
1961 bis 2012 dargestellt.
				
\begin{figure}[ht]
\centering
\footnotesize
\begin{tikzpicture}
\pgfplotstableread{
First
-2000
1500
5000
4000
10100
19000
21000
-6000
5000
10000
32000
35000
34000
-18000
-24000
8000
12000
-9000
-1000
10000
29500
-23000
-1500
3000
5000
5500
2000
13000
45000
59000
77000
72000
33000
3000
2000
4000
1000
8000
20000
17000
32000
32500
40000
51000
45000
24000
25000
24000
17000
21000
30500
44000
}\dataabsolutzahlen


%\pgfplotstableread{
%
%First Second Third Fourth
%69.8	25.3	0.0	4.9
%66.5	32.2	0.0	1.2
%27.8	64.5	0.8	6.9
%7.3	75.1	2.0	15.5
%15.9	34.7	1.6	47.8
%22.9	39.2	1.6	36.3
%}\dataprozent

%\pgfplotsset{minor grid style={dashed}}
\begin{axis}
[
ybar,
title=Wanderungsbilanz 1961-2012,
xmin=-0.5,
xmax=51.5,
%axis lines=left,
%legend entries={naiv, informiert, nicht codierbar, keine Aussage}, 
%reverse legend,
bar width=7pt,
height=7.9cm, 
width=0.95\linewidth,
scaled ticks = false,
minor x grid style={dashed},
%point meta=rawy,
%enlarge x limits=auto,
%xlabel={Something in \%},
%ylabel={Years},
xtick={-0.5,4.5,...,61.5},
minor xtick={0.5,1.5,2.5,...,60.5},
xtick pos=left,
xticklabels={1961,1966,...,2006},
extra x tick labels={2012},
extra x ticks={50.5},
xticklabel style={font=\scriptsize,  xshift=4pt,
%minor x tick num={0}, 
%text width=2cm, 
align=center},
%/pgfplots/every y tick scale label/.style={yshift=1cm},
%visualization depends on=value\coordindex \as\INDEX,
%absolute values/.style={
   %nodes near coords={\pgfplotstablegetelem{\INDEX}{#1}\of\dataabsolutzahlen\pgfmathprintnumber\pgfplotsretval},
%},
%legend style={draw=none, at={(1,1)},xshift=0.5cm,yshift=-2.5cm, anchor=north west,nodes=right}, 
%axis on top,
major y grid style={color=black},
minor y grid style={color=black},
ymajorgrids,
yminorgrids, minor tick num=1,
%xmajorgrids,
xminorgrids,
ytick={-40000,-20000,0,20000,40000,60000,80000},
yticklabels={-40\,000,-20\,000,0,20\,000,40\,000,60\,000,80\,000},
%yticklabel=\pgfmathprintnumber{\tick}\,$\%$,
ymin=-40000,
ymax=80000
]
%fill=blue1
\addplot +[
%ybar stacked,
%absolute values=First,
style={black!60, text=black},
] 
table
[y=First, x expr=\coordindex] {\dataabsolutzahlen};

%\foreach \x in {0.5, 1.5,59.5}
	%\draw[color=black!60, dashed] (\x ,0) -- (\x ,-40000);

%\draw[color=black!60, dashed] (0.5,0) -- (0.5,-40000);
%\addlegendentry{informiert}
%\addplot +[
%%ybar stacked,
%absolute values=Second,
%style={black!60, text=black},
%] table [y=Second, x expr=\coordindex] {\dataprozent};
%%\addlegendentry{naiv}
%\addplot +[
%%ybar stacked,
%absolute values=Third,
%style={black!40, text=black},
%] table [y=Third, x expr=\coordindex] {\dataprozent};
%%\addlegendentry{nicht codierbar}
%\addplot +[
%style={black, text=black},
%] coordinates{(1,1000)};
%%\addlegendentry{keine Aussage}
%\begin{scope}
%\draw ({rel axis cs:0,0}|-{axis cs:0,1}) -- ({rel axis cs:1,0}|-{axis cs:0,1});
%\end{scope}
\fill [black] (-0.5,-40000) rectangle (0.5,-38000);
\fill [black] (4.5,-40000) rectangle (5.5,-38000);
\fill [black] (9.5,-40000) rectangle (10.5,-38000);
\fill [black] (14.5,-40000) rectangle (15.5,-38000);
\fill [black] (19.5,-40000) rectangle (20.5,-38000);
\fill [black] (24.5,-40000) rectangle (25.5,-38000);
\fill [black] (29.5,-40000) rectangle (30.5,-38000);
\fill [black] (34.5,-40000) rectangle (35.5,-38000);
\fill [black] (39.5,-40000) rectangle (40.5,-38000);
\fill [black] (44.5,-40000) rectangle (45.5,-38000);
\fill [black] (50.5,-40000) rectangle (51.5,-38000);
\end{axis}
\end{tikzpicture}
\begin{flushright}
\tiny Quelle: STATISTIK AUSTRIA, Errechnete Wanderungsbilanz 1961-1995; Wanderungsstatistik 1996-2012; 2007-2011: revidierte Daten.
Wanderungsbilanz: Zuz�ge aus dem Ausland minus Wegz�ge in das Ausland (adaptiert). \normalsize
\end{flushright}
\end{figure} \vspace{-0.5cm}

Kreuze die beiden Aussagen an, die eine korrekte Interpretation der Grafik darstellen!

\multiplechoice[5]{  %Anzahl der Antwortmoeglichkeiten, Standard: 5
				L1={Aus dem angegebenen Wert f�r das Jahr 2003 kann man ablesen, dass in diesem
Jahr um ca. 40\,000 Personen mehr zugewandert als abgewandert sind. },   %1. Antwortmoeglichkeit 
				L2={Der Zuwachs der Wanderungsbilanz vom Jahr 2003 auf das Jahr 2004 betr�gt
ca. 50\,\%.},   %2. Antwortmoeglichkeit
				L3={Im Zeitraum 1961 bis 2012 gibt es acht Jahre, in denen die Anzahl der Zuwanderungen
geringer als die Anzahl der Abwanderungen war.},   %3. Antwortmoeglichkeit
				L4={Im Zeitraum 1961 bis 2012 gibt es drei Jahre, in denen die Anzahl der Zuwanderungen
gleich der Anzahl der Abwanderungen war.},   %4. Antwortmoeglichkeit
				L5={Die Wanderungsbilanz des Jahres 1981 ist ann�hernd doppelt so gro� wie die
des Jahres 1970.},	 %5. Antwortmoeglichkeit
				L6={},	 %6. Antwortmoeglichkeit
				L7={},	 %7. Antwortmoeglichkeit
				L8={},	 %8. Antwortmoeglichkeit
				L9={},	 %9. Antwortmoeglichkeit
				%% LOESUNG: %%
				A1=1,  % 1. Antwort
				A2=3,	 % 2. Antwort
				A3=0,  % 3. Antwort
				A4=0,  % 4. Antwort
				A5=0,  % 5. Antwort
				}


\end{langesbeispiel}
\newpage