\section{WS 1.1 - 10 - MAT - Schulstatistik - MC - Matura HT 2013/14}

\begin{beispiel}[WS 1.1]{1} %PUNKTE DES BEISPIELS
			Das nachstehende Diagramm stellt für das Schuljahr 2009/10 folgende Daten dar:
			\begin{itemize}
				\item die Anzahl der Schüler/innen \textbf{nur} aus der AHS-Unterstufe
				\item die Gesamtanzahl der Schüler/innen der 1.-4. Klasse (Hauptschule \textbf{und} AHS-Unterstufe)
			\end{itemize}\leer
			
			\resizebox{1\linewidth}{!}{\newrgbcolor{aqaqaq}{0.6274509803921569 0.6274509803921569 0.6274509803921569}
\psset{xunit=1.0cm,yunit=1.0cm,algebraic=true,dimen=middle,dotstyle=o,dotsize=5pt 0,linewidth=0.8pt,arrowsize=3pt 2,arrowinset=0.25}
\begin{pspicture*}(-2.,-2.5)(18.,7.5)
\multips(0,0)(0,0.5){21}{\psline[linestyle=dashed,linecap=1,dash=1.5pt 1.5pt,linewidth=0.4pt,linecolor=lightgray]{c-c}(0,0)(18.,0)}
\multips(0,0)(100.0,0){1}{\psline[linestyle=dashed,linecap=1,dash=1.5pt 1.5pt,linewidth=0.4pt,linecolor=lightgray]{c-c}(0,0)(0,7.5)}
\psaxes[labelFontSize=\scriptstyle,xAxis=true,yAxis=true,labels=y,showorigin=false,ylabelFactor={0\,000},Dx=1.,Dy=1.,ticksize=-2pt 0,subticks=2]{->}(0,0)(0.,0.)(18.,7.5)
\pspolygon[linewidth=0.2pt,fillcolor=black,fillstyle=solid,opacity=0.6](0.5,0.)(1.,0.)(1.,0.3)(0.5,0.3)
\pspolygon[linewidth=0.2pt,linecolor=aqaqaq,fillcolor=aqaqaq,fillstyle=solid,opacity=0.6](1.,0.)(1.5,0.)(1.5,0.9)(1.,0.9)
\pspolygon[linewidth=0.2pt,fillcolor=black,fillstyle=solid,opacity=0.6](2.5,0.)(2.5,0.75)(3.,0.75)(3.,0.)
\pspolygon[linewidth=0.2pt,linecolor=aqaqaq,fillcolor=aqaqaq,fillstyle=solid,opacity=0.6](3.,0.)(3.5,0.)(3.5,2.15)(3.,2.15)
\pspolygon[linewidth=0.2pt,fillcolor=black,fillstyle=solid,opacity=0.6](4.5,0.)(5.,0.)(5.,2.05)(4.5,2.05)
\pspolygon[linewidth=0.2pt,linecolor=aqaqaq,fillcolor=aqaqaq,fillstyle=solid,opacity=0.6](5.,0.)(5.5,0.)(5.5,6.25)(5.,6.25)
\pspolygon[linewidth=0.2pt,fillcolor=black,fillstyle=solid,opacity=0.6](6.5,0.)(7.,0.)(7.,1.6)(6.5,1.6)
\pspolygon[linewidth=0.2pt,linecolor=aqaqaq,fillcolor=aqaqaq,fillstyle=solid,opacity=0.6](7.5,0.)(7.,0.)(7.,6.2)(7.5,6.2)
\pspolygon[linewidth=0.2pt,fillcolor=black,fillstyle=solid,opacity=0.6](8.5,0.65)(8.5,0.)(9.,0.)(9.,0.65)
\pspolygon[linewidth=0.2pt,linecolor=aqaqaq,fillcolor=aqaqaq,fillstyle=solid,opacity=0.6](9.,0.)(9.,2.4)(9.5,2.4)(9.5,0.)
\pspolygon[linewidth=0.2pt,fillcolor=black,fillstyle=solid,opacity=0.6](10.5,1.4)(10.5,0.)(11.,0.)(11.,1.4)
\pspolygon[linewidth=0.2pt,linecolor=aqaqaq,fillcolor=aqaqaq,fillstyle=solid,opacity=0.6](11.,0.)(11.5,0.)(11.5,4.4)(11.,4.4)
\pspolygon[linewidth=0.2pt,fillcolor=black,fillstyle=solid,opacity=0.6](12.5,0.)(13.,0.)(13.,0.7)(12.5,0.7)
\pspolygon[linewidth=0.2pt,linecolor=aqaqaq,fillcolor=aqaqaq,fillstyle=solid,opacity=0.6](13.,0.)(13.5,0.)(13.5,3.05)(13.,3.05)
\pspolygon[linewidth=0.2pt,fillcolor=black,fillstyle=solid,opacity=0.6](14.5,0.)(15.,0.)(15.,0.45)(14.5,0.45)
\pspolygon[linewidth=0.2pt,linecolor=aqaqaq,fillcolor=aqaqaq,fillstyle=solid,opacity=0.6](15.,0.)(15.5,0.)(15.5,1.3)(15.,1.3)
\pspolygon[linewidth=0.2pt,fillcolor=black,fillstyle=solid,opacity=0.6](16.5,0.)(17.,0.)(17.,3.25)(16.5,3.25)
\pspolygon[linewidth=0.2pt,linecolor=aqaqaq,fillcolor=aqaqaq,fillstyle=solid,opacity=0.6](17.,0.)(17.5,0.)(17.5,6.1)(17.,6.1)
\pspolygon[linewidth=0.2pt,fillcolor=black,fillstyle=solid,opacity=0.6](9.25,6.25)(9.75,6.25)(9.75,6.)(9.25,6.)
\pspolygon[linewidth=0.2pt,fillcolor=aqaqaq,fillstyle=solid,opacity=0.6](9.25,5.75)(9.25,5.5)(9.75,5.5)(9.75,5.75)
\begin{scriptsize}
\rput[tl](10,6.2){AHS Unterstufe}
\rput[tl](10,5.715355805243391){Gesamt 1.-4. Klasse}
\rput[tl](-0.1,-0.15){$\rotatebox{50}{\text{Burgenland}}$}
\rput[tl](2.2,-0.15){$\rotatebox{50}{\text{Kärnten}}$}
\rput[tl](3.63,-0.15){$\rotatebox{50}{\text{Niederösterreich}}$}
\rput[tl](5.63,-0.15){$\rotatebox{50}{\text{Oberösterreich}}$}
\rput[tl](8.2,-0.15){$\rotatebox{50}{\text{Salzburg}}$}
\rput[tl](10,-0.15){$\rotatebox{50}{\text{Steiermark}}$}
\rput[tl](12.5,-0.15){$\rotatebox{50}{\text{Tirol}}$}
\rput[tl](14,-0.15){$\rotatebox{50}{\text{Vorarlberg}}$}
\rput[tl](16.5,-0.15){$\rotatebox{50}{\text{Wien}}$}
\rput[tl](-1.5,5){$\rotatebox{90}{\text{Anzahl der SchülerInnen}}$}
\end{scriptsize}
\end{pspicture*}}

\begin{scriptsize}\textit{Quelle: http://www.bmukk.gv.at/schulstatistik}\end{scriptsize}

Kreuze jene beiden Aussagen an, die aus dem Diagramm gefolgert werden können!

\multiplechoice[5]{  %Anzahl der Antwortmoeglichkeiten, Standard: 5
				L1={In Kärnten ist der Anteil an AHS-SchülerInnen größer als in Tirol.},   %1. Antwortmoeglichkeit 
				L2={In Wien gibt es die meisten SchülerInnen in den 1.-4. Klassen.},   %2. Antwortmoeglichkeit
				L3={Der Anteil an AHS-SchülerInnen ist in Wien höher als in allen anderen Bundesländern.},   %3. Antwortmoeglichkeit
				L4={Es gehen in Salzburg mehr SchülerInnen in die AHS als im Burgenland in die 1.-4. Klasse insgesamt.},   %4. Antwortmoeglichkeit
				L5={In Niederösterreich gehen ca. 3-mal so viele SchülerInnen in die Hauptschule wie in die AHS.},	 %5. Antwortmoeglichkeit
				L6={},	 %6. Antwortmoeglichkeit
				L7={},	 %7. Antwortmoeglichkeit
				L8={},	 %8. Antwortmoeglichkeit
				L9={},	 %9. Antwortmoeglichkeit
				%% LOESUNG: %%
				A1=1,  % 1. Antwort
				A2=3,	 % 2. Antwort
				A3=0,  % 3. Antwort
				A4=0,  % 4. Antwort
				A5=0,  % 5. Antwort
				}
\end{beispiel}