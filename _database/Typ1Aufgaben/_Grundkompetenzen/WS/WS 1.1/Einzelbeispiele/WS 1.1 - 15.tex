\section{WS 1.1 - 15 - Bruttoinlandsprodukt - OA - Matura - 1. NT 2017/18}

\begin{beispiel}[WS 1.1]{1}
Das \textit{nominale Bruttoinlandsprodukt} gibt den Gesamtwert aller G�ter, die w�hrend eines Jahres innerhalb der Landesgrenzen einer Volkswirtschaft hergestellt wurden, in aktuellen Marktpreisen an. Dividiert man das nominale Bruttoinlandsprodukt einer Volkswirtschaft durch die Einwohnerzahl, dann erh�lt man das sogenannte \textit{BIP pro Kopf}.

Die nachstehende Grafik zeigt die relative Ver�nderung der BIP pro Kopf in �sterreich von 2012 bezogen auf 2002.

\begin{center}
BIP pro Kopf

	\resizebox{0.5\linewidth}{!}{\psset{xunit=1.0cm,yunit=0.05cm,algebraic=true,dimen=middle,dotstyle=o,dotsize=5pt 0,linewidth=1.6pt,arrowsize=3pt 2,arrowinset=0.25}
\begin{pspicture*}(-1.28,-29.736130536130652)(8.38,168.5047397047398)
\multips(0,0)(0,20.0){10}{\psline[linestyle=dashed,linecap=1,dash=1.5pt 1.5pt,linewidth=0.4pt,linecolor=darkgray]{c-c}(0,0)(8.38,0)}
\psaxes[labelFontSize=\scriptstyle,xAxis=true,yAxis=true,labels=y,Dx=1.,Dy=20.,ticksize=0pt 0,subticks=2]{->}(0,0)(0.,0.)(8.38,168.5047397047398)
\pspolygon[linewidth=2.pt,fillcolor=black,fillstyle=solid,opacity=0.3](1.5,0.)(1.5,100.)(3.,100.)(3.,0.)
\pspolygon[linewidth=2.pt,fillcolor=black,fillstyle=solid,opacity=0.3](5.,0.)(5.,135.)(6.5,135.)(6.5,0.)
\rput[tl](-1.2,100.60724164724166){$\rotatebox{90}{Prozent}$}
\rput[tl](3.68,-16){Jahr}
\psline[linewidth=2.pt](1.5,0.)(1.5,100.)
\psline[linewidth=2.pt](1.5,100.)(3.,100.)
\psline[linewidth=2.pt](3.,100.)(3.,0.)
\psline[linewidth=2.pt](3.,0.)(1.5,0.)
\psline[linewidth=2.pt](5.,0.)(5.,135.)
\psline[linewidth=2.pt](5.,135.)(6.5,135.)
\psline[linewidth=2.pt](6.5,135.)(6.5,0.)
\psline[linewidth=2.pt](6.5,0.)(5.,0.)
\rput[tl](1.8,-5){2002}
\rput[tl](5.3,-5){2012}
\end{pspicture*}}
\end{center} 

Gib an, ob ausschlie�lich anhand der Daten in der gegebenen Grafik der Wert de relativen �nderung des nominalen Bruttoinlandsprodukts in �sterreich von 2012 bezogen auf 2002 ermittelt werden kann, und begr�nde deine Entscheidung!

\antwort{Die relative �nderung des (nominalen) Bruttoinlandprodukts in �sterreich kann ausschlie�lich anhand der gegebenen Daten nicht ermittelt werden, da die Einwohnerzahlen �sterreichs der Jahre 2002 und 2012 nicht angegeben sind.}
\end{beispiel}