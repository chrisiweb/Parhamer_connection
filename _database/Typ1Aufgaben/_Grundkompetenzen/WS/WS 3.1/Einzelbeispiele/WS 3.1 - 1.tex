\section{WS 3.1 - 1 Wahrscheinlichkeitsverteilung - OA - BIFIE}

\begin{beispiel}[WS 3.1]{1} %PUNKTE DES BEISPIELS
Gustav kommt in der Nacht nach Hause und muss im Dunkeln die Haustüre aufsperren. An
seinem ringförmigen Schlüsselbund hängen fünf gleiche Schlüsseltypen, von denen nur einer
sperrt. Er beginnt die Schlüssel zufällig und nacheinander zu probieren. Die Zufallsvariable $X$
gibt die Anzahl $k$ der Schlüssel an, die er probiert, bis die Tür geöffnet ist. \leer

Ergänze in der Tabelle die fehlenden Wahrscheinlichkeiten und ermittle den Erwartungswert
$E(X)$ dieser Zufallsvariablen $X$.\leer

\antwort{Gleichwahrscheinlichkeit liegt vor, weil:}

\renewcommand{\arraystretch}{1.5}
\begin{tabular}{|c|c|c|c|c|c|}\hline
k&1&2&3&4&5\\ \hline
$P(X=k)$&\antwort{$\frac{1}{5}$}&\antwort{$\frac{4}{5}\cdot \frac{1}{4}=\frac{1}{5}$}&\antwort{$\frac{4}{5}\cdot \frac{3}{4}\cdot \frac{1}{3}=\frac{1}{5}$}&\antwort{$\frac{4}{5}\cdot \frac{3}{4}\cdot \frac{2}{3}\cdot \frac{1}{2}=\frac{1}{5}$}&\antwort{$\frac{4}{5}\cdot \frac{3}{4}\cdot \frac{2}{3}\cdot \frac{1}{2}\cdot \frac{1}{1}=\frac{1}{5}$}\\ \hline
\end{tabular}
\leer


$E(X)=\rule{8cm}{0.3pt}$

\antwort{$E(X)=\left( 1 \cdot \frac{1}{5} + 2 \cdot \frac{1}{5} + 3 \cdot \frac{1}{5} + 4 \cdot \frac{1}{5}+ 5 \cdot \frac{1}{5}\right)= 3$\leer

Lösungsschlüssel: Die Aufgabe gilt nur dann als richtig gelöst, wenn die Tabelle korrekt ausgefüllt und der Erwartungswert richtig berechnet ist. }

\end{beispiel}