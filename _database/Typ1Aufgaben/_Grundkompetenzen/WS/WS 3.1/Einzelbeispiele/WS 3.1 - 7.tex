\section{WS 3.1 - 7 - MAT - Vergleich zweier Wahrscheinlichkeitsverteilungen - MC - Matura HT 2017/18}

\begin{beispiel}[WS 3.1]{1} %PUNKTE DES BEISPIELS
In den nachstehenden Diagrammen sind die Wahrscheinlichkeitsverteilungen zweier Zufallsvariablen $X$ und $Y$ dargestellt. Die Erwartungswerte der Zufallsvariablen werden mit $E(X)$ und $E(Y)$, die Standardabweichung mit $\sigma(X)$ und $\sigma(Y)$ bezeichnet.

\meinlr{\resizebox{0.9\linewidth}{!}{\psset{xunit=0.8cm,yunit=16.0cm,algebraic=true,dimen=middle,dotstyle=o,dotsize=5pt 0,linewidth=1.6pt,arrowsize=3pt 2,arrowinset=0.25}
\begin{pspicture*}(-1.9,-0.03235284139100884)(9.78,0.38130134496546797)
\multips(0,0)(0,0.05){9}{\psline[linestyle=dashed,linecap=1,dash=1.5pt 1.5pt,linewidth=0.4pt,linecolor=darkgray]{c-c}(0,0)(9.78,0)}
\multips(0,0)(19.0,0){1}{\psline[linestyle=dashed,linecap=1,dash=1.5pt 1.5pt,linewidth=0.4pt,linecolor=darkgray]{c-c}(0,0)(0,0.38130134496546797)}
\psaxes[labelFontSize=\scriptstyle,xAxis=true,yAxis=true,Dx=1.,Dy=0.05,ticksize=-2pt 0,subticks=2]{->}(0,0)(0.,0.)(9.78,0.38130134496546797)[k,140] [,-40]
\psframe[linewidth=0.8pt,linecolor=gray,fillcolor=gray,fillstyle=solid,opacity=1](1.9,0)(2.1,0.05)
\psframe[linewidth=0.8pt,linecolor=gray,fillcolor=gray,fillstyle=solid,opacity=1](2.9,0)(3.1,0.1)
\psframe[linewidth=0.8pt,linecolor=gray,fillcolor=gray,fillstyle=solid,opacity=1](3.9,0)(4.1,0.2)
\psframe[linewidth=0.8pt,linecolor=gray,fillcolor=gray,fillstyle=solid,opacity=1](4.9,0)(5.1,0.3)
\psframe[linewidth=0.8pt,linecolor=gray,fillcolor=gray,fillstyle=solid,opacity=1](5.9,0)(6.1,0.2)
\psframe[linewidth=0.8pt,linecolor=gray,fillcolor=gray,fillstyle=solid,opacity=1](6.9,0)(7.1,0.1)
\psframe[linewidth=0.8pt,linecolor=gray,fillcolor=gray,fillstyle=solid,opacity=1](7.9,0)(8.1,0.05)
\rput[tl](-1.8,0.24264631043257065){$\rotatebox{90}{P(X=k)}$}
\end{pspicture*}}}
{\resizebox{0.9\linewidth}{!}{\psset{xunit=0.8cm,yunit=16.0cm,algebraic=true,dimen=middle,dotstyle=o,dotsize=5pt 0,linewidth=1.6pt,arrowsize=3pt 2,arrowinset=0.25}
\begin{pspicture*}(-1.9,-0.03235284139100884)(9.78,0.38130134496546797)
\multips(0,0)(0,0.05){9}{\psline[linestyle=dashed,linecap=1,dash=1.5pt 1.5pt,linewidth=0.4pt,linecolor=darkgray]{c-c}(0,0)(9.78,0)}
\multips(0,0)(19.0,0){1}{\psline[linestyle=dashed,linecap=1,dash=1.5pt 1.5pt,linewidth=0.4pt,linecolor=darkgray]{c-c}(0,0)(0,0.38130134496546797)}
\psaxes[labelFontSize=\scriptstyle,xAxis=true,yAxis=true,Dx=1.,Dy=0.05,ticksize=-2pt 0,subticks=2]{->}(0,0)(0.,0.)(9.78,0.38130134496546797)[k,140] [,-40]
\psframe[linewidth=0.8pt,linecolor=gray,fillcolor=gray,fillstyle=solid,opacity=1](1.9,0)(2.1,0.1)
\psframe[linewidth=0.8pt,linecolor=gray,fillcolor=gray,fillstyle=solid,opacity=1](2.9,0)(3.1,0.2)
\psframe[linewidth=0.8pt,linecolor=gray,fillcolor=gray,fillstyle=solid,opacity=1](3.9,0)(4.1,0.15)
\psframe[linewidth=0.8pt,linecolor=gray,fillcolor=gray,fillstyle=solid,opacity=1](4.9,0)(5.1,0.1)
\psframe[linewidth=0.8pt,linecolor=gray,fillcolor=gray,fillstyle=solid,opacity=1](5.9,0)(6.1,0.15)
\psframe[linewidth=0.8pt,linecolor=gray,fillcolor=gray,fillstyle=solid,opacity=1](6.9,0)(7.1,0.2)
\psframe[linewidth=0.8pt,linecolor=gray,fillcolor=gray,fillstyle=solid,opacity=1](7.9,0)(8.1,0.1)
\rput[tl](-1.8,0.24264631043257065){$\rotatebox{90}{P(Y=k)}$}
\end{pspicture*}}}

Kreuze die beiden zutreffenden Aussagen an!\leer

\multiplechoice[5]{  %Anzahl der Antwortmoeglichkeiten, Standard: 5
				L1={$E(X)=E(Y)$},   %1. Antwortmoeglichkeit 
				L2={$\sigma(X)>\sigma(Y)$},   %2. Antwortmoeglichkeit
				L3={$P(X\leq 3)<P(Y\leq 3)$},   %3. Antwortmoeglichkeit
				L4={$P(3\leq X\leq 7)=P(3\leq Y\leq 7)$},   %4. Antwortmoeglichkeit
				L5={$P(X\leq 5)=0,3$},	 %5. Antwortmoeglichkeit
				L6={},	 %6. Antwortmoeglichkeit
				L7={},	 %7. Antwortmoeglichkeit
				L8={},	 %8. Antwortmoeglichkeit
				L9={},	 %9. Antwortmoeglichkeit
				%% LOESUNG: %%
				A1=1,  % 1. Antwort
				A2=3,	 % 2. Antwort
				A3=0,  % 3. Antwort
				A4=0,  % 4. Antwort
				A5=0,  % 5. Antwort
				}
\end{beispiel}