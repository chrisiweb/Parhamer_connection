\section{WS 3.1 - 3 Bernoulli-Experiment - MC - BIFIE}

\begin{beispiel}[WS 3.1]{1} %PUNKTE DES BEISPIELS
Beim Realisieren eines Bernoulli-Experiments tritt Erfolg mit der Wahrscheinlichkeit $p$ mit
$0 < p < 1$ ein. Die Werte der binomialverteilten Zufallsvariablen $X$ beschreiben die Anzahl der
Erfolge beim $n$-maligen unabhängigen Wiederholen des Experiments. $E$ bezeichnet den Erwartungswert,
$V$ die Varianz und $\sigma$ die Standardabweichung.\leer

Kreuze die beiden für $n>1$ zutreffenden Aussagen an.

\multiplechoice[5]{  %Anzahl der Antwortmoeglichkeiten, Standard: 5
				L1={$E=\sqrt{n\cdot p}$},   %1. Antwortmoeglichkeit 
				L2={$V(X)=n\cdot p\cdot (1-p)$},   %2. Antwortmoeglichkeit
				L3={$P(X=0)=0$},   %3. Antwortmoeglichkeit
				L4={$P(X=1)=p$},   %4. Antwortmoeglichkeit
				L5={$V(X)=\sigma^2$},	 %5. Antwortmoeglichkeit
				L6={},	 %6. Antwortmoeglichkeit
				L7={},	 %7. Antwortmoeglichkeit
				L8={},	 %8. Antwortmoeglichkeit
				L9={},	 %9. Antwortmoeglichkeit
				%% LOESUNG: %%
				A1=2,  % 1. Antwort
				A2=5,	 % 2. Antwort
				A3=0,  % 3. Antwort
				A4=0,  % 4. Antwort
				A5=0,  % 5. Antwort
				}
\end{beispiel} 