\section{WS 3.1 - 8 - MAT - Häufigkeit von Nebenwirkungen - OA - Matura-HT-18/19}

\begin{beispiel}[WS 3.1]{1}
Pharmaunternehmen sind verpflichtet, alle bekannt gewordenen Nebenwirkungen eines Medikaments im Beipackzettel anzugeben. Die Häufigkeitsangaben zu Nebenwirkungen basieren auf
folgenden Kategorien:

\begin{center}
\begin{tabular}{|l|p{10cm}|} \hline

\rowcolor{lightgray} Häufigkeitsangabe & Auftreten von Nebenwirkungen \\ \hline

sehr häufig & Nebenwirkungen treten bei mehr als 1 von 10 Behandelten auf. \\ \hline

häufig & Nebenwirkungen treten bei 1 bis 10 Behandelten von 100 auf. \\ \hline

gelegentlich & Nebenwirkungen treten bei 1 bis 10 Behandelten von 1\,000 auf. \\ \hline

selten & Nebenwirkungen treten bei 1 bis 10 Behandelten von 10\,000 auf.\\ \hline
 
sehr selten & Nebenwirkungen treten bei weniger als 1 von 10\,000 Behandelten auf. \\ \hline
  
nicht bekannt   & Die Häufigkeit von Nebenwirkungen ist auf Grundlage der verfügbaren Daten nicht abschätzbar.\\ \hline

\end{tabular}
\end{center}

Eine bestimmte Nebenwirkung ist im Beipackzettel eines Medikaments mit der Häufigkeitsangabe
"`selten"' kategorisiert.\\
Es werden 50\,000 Personen unabhängig voneinander mit diesem Medikament behandelt. Bei einer gewissen Anzahl dieser Personen tritt diese Nebenwirkung auf.\leer

Verwende die obigen Häufigkeitsangaben als Wahrscheinlichkeiten und bestimme unter
dieser Voraussetzung, wie groß die erwartete Anzahl an von dieser Nebenwirkung betroffenen
Personen mindestens ist!

\antwort{Die Zufallsvariable $X$ beschreibt die Anzahl an von dieser Nebenwirkung betroffenen Personen.

$n=50\,000$

$p=0,0001$

$E(X)=n\cdot p= 50\,000 \cdot 0,0001= 5$

Die erwartete Anzahl an von dieser Nebenwirkung betroffenen Personen ist mindestens 5.}
\end{beispiel}