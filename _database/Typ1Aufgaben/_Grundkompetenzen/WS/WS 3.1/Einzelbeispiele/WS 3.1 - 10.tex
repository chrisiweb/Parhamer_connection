\section{WS 3.1 - 10 - MAT - Spielkarten - OA - Matura 1.NT 2018/19}

\begin{beispiel}[WS 3.1]{1}
Fünf Spielkarten (drei Könige und zwei Damen) werden gemischt und verdeckt auf einen Tisch gelegt. Laura dreht während eines Spieldurchgangs nacheinander die Karten einzeln um und lässt sie aufgedeckt liegen, bis die erste Dame aufgedeckt ist.

Die Zufallsvariable $X$ gibt die Anzahl der am Ende eines Spieldurchgangs aufgedeckten Spielkarten an.

Berechne den Erwartungswert der Zufallsvariablen $X$.\leer

$E(X)=\antwort[\rule{5cm}{0.3pt}]{1\cdot\frac{2}{5}+2\cdot\frac{3}{5}\cdot\frac{2}{4}+3\cdot\frac{3}{5}\cdot\frac{2}{4}\cdot\frac{2}{3}+4\cdot\frac{3}{5}\cdot\frac{2}{4}\cdot\frac{1}{3}=2}$
\end{beispiel}