\section{WS 3.1 - 11 - MAT - Wahrscheinlichkeitsverteilung - MC - Matura 2019/20 1. HT}

\begin{beispiel}[WS 3.1]{1}
In einer Urne befinden sich ausschließlich weiße und schwarze Kugeln. Drei Kugeln werden ohne Zurücklegen gezogen. Die Zufallsvariable $X$ gibt die Anzahl der gezogenen weißen Kugeln an.

Durch die nachstehende Tabelle ist die Wahrscheinlichkeitsverteilung der Zufallsvariablen $X$ gegeben.

\begin{center}
\begin{tabular}{|c|c|c|c|}\hline
\cellcolor[gray]{0.9}$x$&1&2&3\\ \hline
\cellcolor[gray]{0.9}$P(X=x)$&0,3&0,6&0,1\\ \hline
\end{tabular}
\end{center}

Kreuze die beiden zutreffenden Aussagen an.

\multiplechoice[5]{  %Anzahl der Antwortmoeglichkeiten, Standard: 5
				L1={Die Wahrscheinlichkeit, höchstens zwei weiße Kugeln zu ziehen, ist 0,9.},   %1. Antwortmoeglichkeit 
				L2={Die Wahrscheinlichkeit, mindestens eine weiße Kugel zu ziehen, ist 0,3.},   %2. Antwortmoeglichkeit
				L3={Die Wahrscheinlichkeit, mehr als eine weiße Kugel zu ziehen, ist 0,6.},   %3. Antwortmoeglichkeit
				L4={Die Wahrscheinlichkeit, genau zwei schwarze Kugeln und eine weiße Kugel zu ziehen, ist 0,1.},   %4. Antwortmoeglichkeit
				L5={Die Wahrscheinlichkeit, mindestens eine schwarze Kugel zu ziehen, ist 0,9.},	 %5. Antwortmoeglichkeit
				L6={},	 %6. Antwortmoeglichkeit
				L7={},	 %7. Antwortmoeglichkeit
				L8={},	 %8. Antwortmoeglichkeit
				L9={},	 %9. Antwortmoeglichkeit
				%% LOESUNG: %%
				A1=1,  % 1. Antwort
				A2=5,	 % 2. Antwort
				A3=0,  % 3. Antwort
				A4=0,  % 4. Antwort
				A5=0,  % 5. Antwort
				}
\end{beispiel}