\section{WS 3.2 - 11 - MAT - Zufallsexperiment - MC - Matura 2. NT 2015/16}

\begin{beispiel}{0} %PUNKTE DES BEISPIELS
Bei einem Zufallsexperiment, das 25-mal wiederholt wird, gibt es die Ausgänge "`günstig"' und
"`ungünstig"'. Die Zufallsvariable $X$ beschreibt, wie oft dabei das Ergebnis "`günstig"' eingetreten ist. $X$ ist binomialverteilt mit dem Erwartungswert 10.\leer

Zwei der nachstehenden Aussagen lassen sich aus diesen Informationen ableiten.

Kreuze die beiden zutreffenden Aussagen an. \leer

\multiplechoice[5]{  %Anzahl der Antwortmoeglichkeiten, Standard: 5
				L1={$P(X=25)=10$},   %1. Antwortmoeglichkeit 
				L2={Wenn man das Zufallsexperiment 25-mal durchführt, werden mit
Sicherheit genau 10 Ergebnisse "`günstig"' sein.},   %2. Antwortmoeglichkeit
				L3={Die Wahrscheinlichkeit, dass ein einzelnes Zufallsexperiment
"`günstig"' ausgeht, ist 40\,\%.},   %3. Antwortmoeglichkeit
				L4={Wenn man das Zufallsexperiment 50-mal durchführt, dann ist der
Erwartungswert für die Anzahl der "`günstigen"' Ergebnisse 20.},   %4. Antwortmoeglichkeit
				L5={$P(X>10)>P(X>8)$},	 %5. Antwortmoeglichkeit
				L6={},	 %6. Antwortmoeglichkeit
				L7={},	 %7. Antwortmoeglichkeit
				L8={},	 %8. Antwortmoeglichkeit
				L9={},	 %9. Antwortmoeglichkeit
				%% LOESUNG: %%
				A1=3,  % 1. Antwort
				A2=4,	 % 2. Antwort
				A3=0,  % 3. Antwort
				A4=0,  % 4. Antwort
				A5=0,  % 5. Antwort
				}
\end{beispiel}