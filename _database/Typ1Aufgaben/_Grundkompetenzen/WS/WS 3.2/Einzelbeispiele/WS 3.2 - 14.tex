\section{WS 3.2 - 14 Binomialverteilung - OA - Matura 2013/14 1. Nebentermin}

\begin{beispiel}[WS 3.2]{1} %PUNKTE DES BEISPIELS
				In der untenstehenden Abbildung ist die Wahrscheinlichkeitsverteilung einer binomialverteilten Zufallsvariablen $X$ mit den Parametern $n=6$ und $p=0,5$ durch ein Säulendiagramm (Säulenbreite = 1) dargestellt. $\mu$ bezeichnet den Erwartungswert von $X$.

				Schraffieren Sie diejenigen Rechtecksflächen, die $P(X>\mu)$ veranschaulichen!
				\begin{center}\resizebox{0.8\linewidth}{!}{\psset{xunit=1.0cm,yunit=16.0cm,algebraic=true,dimen=middle,dotstyle=o,dotsize=5pt 0,linewidth=0.8pt,arrowsize=3pt 2,arrowinset=0.25}
\begin{pspicture*}(-1.5789473684210529,-0.02)(8.272445820433439,0.37165132216530544)
\psaxes[labelFontSize=\scriptstyle,xAxis=true,yAxis=true,labels=none,Dx=1.,Dy=0.05,ticksize=-2pt 0]{->}(0,0)(0.,0.)(8.272445820433439,0.37165132216530544)[k,140] [P(X=k),-40]
\psframe[linewidth=1.2pt](0,0)(1,0.015625)
\psframe[linewidth=1.2pt](1,0)(2,0.09375)
\psframe[linewidth=1.2pt](2,0)(3,0.234375)
\psframe[linewidth=1.2pt](3,0)(4,0.3125)
\psframe[linewidth=1.2pt](4,0)(5,0.234375)
\psframe[linewidth=1.2pt](5,0)(6,0.09375)
\psframe[linewidth=1.2pt](6,0)(7,0.015625)
\antwort{\psframe[linewidth=1.2pt,linecolor=blue,fillcolor=blue,fillstyle=solid,opacity=0.5](4,0)(5,0.234375)
\psframe[linewidth=1.2pt,linecolor=blue,fillcolor=blue,fillstyle=solid,opacity=0.5](5,0)(6,0.09375)
\psframe[linewidth=1.2pt,linecolor=blue,fillcolor=blue,fillstyle=solid,opacity=0.5](6,0)(7,0.015625)}
\begin{scriptsize}
\rput[tl](0.45,-0.005){0}
\rput[tl](1.45,-0.005){1}
\rput[tl](2.45,-0.005){2}
\rput[tl](3.45,-0.005){3}
\rput[tl](4.45,-0.005){4}
\rput[tl](5.45,-0.005){5}
\rput[tl](6.45,-0.005){6}
\rput[tl](-0.7,0.055){0,05}
\rput[tl](-0.55,0.105){0,1}
\rput[tl](-0.7,0.155){0,15}
\rput[tl](-0.55,0.205){0,2}
\rput[tl](-0.7,0.255){0,25}
\rput[tl](-0.55,0.305){0,3}
\rput[tl](-0.7,0.355){0,35}
\end{scriptsize}
\end{pspicture*}}\end{center}
\end{beispiel}