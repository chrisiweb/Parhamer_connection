\section{WS 3.2 - 22 - MAT - Pasch - OA - Matura 1.NT 2018/19}

\begin{beispiel}[WS 3.2]{1}
Bei einem Spiel werden in jeder Spielrunde zwei Würfel geworden. Zeigen nach einem Wurf beide Würfel die gleiche Augenzahl, spricht man von einem Pasch. Die Wahrscheinlichkeit, einen Pasch zu werfen, beträgt $\frac{1}{6}$.

Es werden acht Runden (unabhängig voneinander) gespielt. Die Zufallsvariable $X$ bezeichnet dabei die Anzahl der geworfenen Pasche.\\
Berechne die Wahrscheinlichkeit für den Fall, dass die Anzahl $X$ der geworfenen Pasche unter dem Erwartungswert $E(X)$ liegt.

\antwort{mögliche Vorgehensweise:\\
$\mu=n\cdot p=8\cdot\frac{1}{6}=\frac{4}{3}$

$P\left(X\leq \frac{4}{3}\right)=P(X\leq 1)=\left(\frac{5}{6}\right)^8+8\cdot\left(\frac{1}{6}\right)\cdot\left(\frac{5}{6}\right)^7\approx 0,6047$

Toleranzintervall: $[0,6;0,61]$

Die Aufgabe ist auch dann als richtig gelöst zu werten, wenn bei korrektem Ansatz das Ergebnis aufgrund eines Rechenfehlers nicht richtig ist.}
\end{beispiel}