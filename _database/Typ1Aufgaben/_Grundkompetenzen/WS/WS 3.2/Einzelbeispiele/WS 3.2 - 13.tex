\section{WS 3.2 - 13 - MAT - Multiple-Choice-Antwort - OA - Matura HT 2013/14}

\begin{beispiel}[WS 3.2]{1} %PUNKTE DES BEISPIELS
				Bei einer schriftlichen Prüfung werden der Kandidatin/dem Kandidaten fünf Fragen mit je vier Antwortmöglichkeiten vorgelegt. Genau eine der Antworten ist jeweils richtig.
				
				Berechne die Wahrscheinlichkeit, dass die Kandidatin/der Kandidat bei zufälligem Ankreuzen mindestens viermal die richtige Antwort kennzeichnet!
				
				\antwort{$X$ ... Anzahl der richtigen Antworten
				
				$W(X\geq 4)=5\cdot \left(\frac{1}{4}\right)^4\cdot\left(\frac{3}{4}\right)+\left(\frac{1}{4}\right)^5=\frac{1}{64}\approx 0,02=2\,\%$
				
				 Toleranzintervall: $[0,015; 0,02]$ bzw. $[1,5\,\%; 2\,\%]$.
}
\end{beispiel}