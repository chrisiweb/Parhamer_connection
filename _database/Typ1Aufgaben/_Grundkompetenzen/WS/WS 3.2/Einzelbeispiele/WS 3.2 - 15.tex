\section{WS 3.2 - 15 - MAT - Aussagen zu einer Zufallsvariablen - MC - Matura HT 2016/17}

\begin{beispiel}[WS 3.2]{1} %PUNKTE DES BEISPIELS
Die Zufallsvariable $X$ kann nur die Werte 10, 20 und 30 annehmen. Die nachstehende Tabelle gibt
die Wahrscheinlichkeitsverteilung von $X$ an, wobei a und b positive reelle Zahlen sind. \leer

\begin{center}
\begin{tabular}{|l|c|c|c|} \hline
\cellcolor{black!20} $k$ & 10 & 20 & 30 \\ \hline
\cellcolor{black!20} $P(X=k)$ & $a$ & $b$ & $a$ \\ \hline
\end{tabular}
\end{center}

Kreuze die beiden zutreffenden Aussagen an. \leer

\multiplechoice[5]{  %Anzahl der Antwortmoeglichkeiten, Standard: 5
				L1={Der Erwartungswert von $X$ ist 20.},   %1. Antwortmoeglichkeit 
				L2={Die Standardabweichung von $X$ ist 20.},   %2. Antwortmoeglichkeit
				L3={$a+b=1$},   %3. Antwortmoeglichkeit
				L4={$P(10\leq X \leq 30)=1$},   %4. Antwortmoeglichkeit
				L5={$P(X\leq 10)=P(X \geq 10)$},	 %5. Antwortmoeglichkeit
				L6={},	 %6. Antwortmoeglichkeit
				L7={},	 %7. Antwortmoeglichkeit
				L8={},	 %8. Antwortmoeglichkeit
				L9={},	 %9. Antwortmoeglichkeit
				%% LOESUNG: %%
				A1=1,  % 1. Antwort
				A2=4,	 % 2. Antwort
				A3=0,  % 3. Antwort
				A4=0,  % 4. Antwort
				A5=0,  % 5. Antwort
				}
\end{beispiel}