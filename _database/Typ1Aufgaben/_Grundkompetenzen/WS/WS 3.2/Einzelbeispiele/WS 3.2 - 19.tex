\section{WS 3.2 - 19 - Binomialverteilung - ZO - Matura - 1. NT 2017/18}

\begin{beispiel}[WS 3.2]{1}
Der relative Anteil der österreichischen Bevölkerung mit der Blutgruppe "`AB Rhesusfaktor negativ"' (AB-) ist bekannt und wird mit $p$ bezeichnet.\\
In einer Zufallsstichprobe von 100 Personen soll ermittelt werden, wie viele dieser zufällig ausgewählten Personen die genannte Blutgruppe haben.

Ordne den vier angeführten Ereignissen jeweils denjenigen Term (aus A bis F) zu, der die diesem Ereignis entsprechende Wahrscheinlichkeit angibt!

\zuordnen{
				R1={Genau einer Person hat die Blutgruppe AB-.},				% Response 1
				R2={Mindestens eine Person hat die Blutgruppe AB-.},				% Response 2
				R3={Höchsten eines Person hat die Blutgruppe AB-.},				% Response 3
				R4={Keine Person hat die Blutgruppe AB-.},				% Response 4
				%% Moegliche Zuordnungen: %%
				A={$1-p^{100}$}, 				%Moeglichkeit A  
				B={$p\cdot(1-p)^{90}$}, 				%Moeglichkeit B  
				C={$1-(1-p)^{100}$}, 				%Moeglichkeit C  
				D={$(1-p)^{100}$}, 				%Moeglichkeit D  
				E={$p\cdot(1-p)^{99}\cdot 100$}, 				%Moeglichkeit E  
				F={$(1-p)^{100}+p\cdot(1-p)^{90}\cdot 100$}, 				%Moeglichkeit F  
				%% LOESUNG: %%
				A1={E},				% 1. richtige Zuordnung
				A2={C},				% 2. richtige Zuordnung
				A3={F},				% 3. richtige Zuordnung
				A4={D},				% 4. richtige Zuordnung
				}
\end{beispiel}