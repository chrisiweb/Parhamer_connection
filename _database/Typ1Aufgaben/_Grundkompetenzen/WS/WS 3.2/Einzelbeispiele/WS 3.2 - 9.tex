\section{WS 3.2 - 9 Gewinn beim Glücksrad - OA - Matura 2014/15 - Nebentermin 1}

\begin{beispiel}[WS 3.2]{1}
Das unten abgebildete Glücksrad ist in acht gleich große Sektoren unterteilt, die mit gleicher Wahrscheinlichkeit auftreten. Für einmaliges Drehen des Glücksrades muss ein Einsatz von 5\,\euro gezahlt werden. Die Gewinne, die ausbezahlt werden, wenn das Glücksrad im entsprechenden
Sektor stehen bleibt, sind auf dem Glücksrad abgebildet.\leer

\begin{center}
\psset{xunit=1.0cm,yunit=1.0cm,algebraic=true,dimen=middle,dotstyle=o,dotsize=5pt 0,linewidth=0.8pt,arrowsize=3pt 2,arrowinset=0.25}
\begin{pspicture*}(0.7823214592356473,0.7419987008660722)(7.117101769544686,7.19630316797339)
\pscircle(4.,4.){3.}
\psline(4.,4.)(1.8786796564403576,6.121320343559642)
\psline(4.,4.)(4.,7.)
\psline(4.,4.)(6.121320343559642,6.121320343559642)
\psline(4.,4.)(7.,4.)
\psline(4.,4.)(6.121320343559642,1.878679656440358)
\psline(4.,4.)(4.,1.)
\psline(4.,4.)(1.8786796564403576,1.8786796564403576)
\psline(4.,4.)(1.,4.)
\rput[tl](4.527411705581872,6){$5$\,\euro}
\rput[tl](5.563287731166997,5){$0$\,\euro}
\rput[tl](5.503525652767855,3.5){$10$\,\euro}
\rput[tl](4.447728934383016,2.594623131239469){$0$\,\euro}
\rput[tl](3,2.614543824039183){$5$\,\euro}
\rput[tl](1.82,3.5){$0$\,\euro}
\rput[tl](1.82,4.9){$15$\,\euro}
\rput[tl](2.953676974404469,6){$0$\,\euro}
\end{pspicture*}
\end{center}
\leer

Das Glücksrad wird einmal gedreht. Berechne den entsprechenden Erwartungswert des Reingewinns $G$ (in Euro) aus der Sicht des Betreibers des Glücksrades. Der Reingewinn ist die
Differenz aus Einsatz und Auszahlungsbetrag.

\antwort{
$G=5-\left( \frac{1}{4}\cdot 5 + \frac{1}{8} \cdot 10 + \frac{1}{8} \cdot 15\right) = \frac{5}{8} \Rightarrow G \approx$ \euro\, 0,63 \leer

Lösungsschlüssel:\\
Ein Punkt für die richtige Lösung, wobei die Einheit nicht angeführt sein muss.\\
Toleranzintervall: $[0,62; 0,63]$}
\end{beispiel}