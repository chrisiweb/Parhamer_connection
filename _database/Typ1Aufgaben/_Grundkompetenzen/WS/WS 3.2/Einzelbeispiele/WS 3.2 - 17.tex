\section{WS 3.2 - 17 - MAT - Wahrscheinlichkeit - OA - Matura 2. NT 2016/17}

\begin{beispiel}[WS 3.2]{1} %PUNKTE DES BEISPIELS
Die Zufallsvariable $X$ hat den Wertebereich $\{0, 1, \ldots, 9, 10\}$.

Gegeben sind die beiden Wahrscheinlichkeiten $P(X=0)=0,35$ und\\ 
\mbox{$P(X=1)=0,38$}.\leer

Berechne die Wahrscheinlichkeit $P(X\geq 2)$!\leer

$P(X\geq 2)=\antwort[\rule{5cm}{0.3pt}]{1-(P(X=0)+P(X=1))=0,27}$
\end{beispiel}