\section{WS 3.2 - 12 Diskrete Zufallsvariable - MC- Matura 2013/14 Haupttermin}

\begin{beispiel}[WS 3.2]{1} %PUNKTE DES BEISPIELS
				Die unten stehende Abbildung zeigt die Wahrscheinlichkeitsverteilung einer diskreten Zufallsvariablen $X$.\leer
				
				\resizebox{1\linewidth}{!}{\psset{xunit=1.0cm,yunit=20cm,algebraic=true,dimen=middle,dotstyle=o,dotsize=5pt 0,linewidth=0.8pt,arrowsize=3pt 2,arrowinset=0.25}
\begin{pspicture*}(-0.9083775185577952,-0.030903858771872447)(12.291622481442204,0.3357062859807456)
\psaxes[labelFontSize=\scriptstyle,xAxis=true,yAxis=true,labels=x,Dx=1.,Dy=0.05,ticksize=-2pt 0,subticks=2]{->}(0,0)(-0.9083775185577952,-0.030903858771872447)(12.291622481442204,0.3357062859807456)[$k$,140] [$P(X=k)$,-40]
\psframe[linewidth=1.2pt](-0.5,0)(0.5,0.00604661760000001)
\psframe[linewidth=1.2pt](0.5,0)(1.5,0.040310784)
\psframe[linewidth=1.2pt](1.5,0)(2.5,0.120932352)
\psframe[linewidth=1.2pt](2.5,0)(3.5,0.214990848)
\psframe[linewidth=1.2pt](3.5,0)(4.5,0.250822656)
\psframe[linewidth=1.2pt](4.5,0)(5.5,0.2006581248)
\psframe[linewidth=1.2pt](5.5,0)(6.5,0.111476736)
\psframe[linewidth=1.2pt](6.5,0)(7.5,0.0424673280000001)
\psframe[linewidth=1.2pt](7.5,0)(8.5,0.010616832)
\psframe[linewidth=1.2pt](8.5,0)(9.5,0.001572864)
\psframe[linewidth=1.2pt](9.5,0)(10.5,00.00048576)
\psframe[linewidth=1.2pt,fillcolor=black,fillstyle=solid,opacity=0.5](2.5,0)(3.5,0.214990848)
\psframe[linewidth=1.2pt,fillcolor=black,fillstyle=solid,opacity=0.5](3.5,0)(4.5,0.250822656)
\psframe[linewidth=1.2pt,fillcolor=black,fillstyle=solid,opacity=0.5](4.5,0)(5.5,0.2006581248)
\psframe[linewidth=1.2pt,fillcolor=black,fillstyle=solid,opacity=0.5](5.5,0)(6.5,0.111476736)
\end{pspicture*}}\leer

Welcher der folgenden Ausdrücke beschreibt die Wahrscheinlichkeit, die dem Inhalt der schraffierten Fläche entspricht?\\
Kreuze den zutreffenden Ausdruck an!\leer

\multiplechoice[6]{  %Anzahl der Antwortmoeglichkeiten, Standard: 5
				L1={$1-P(X\leq 2)$},   %1. Antwortmoeglichkeit 
				L2={$P(X\leq 6)-P(X\leq 3)$},   %2. Antwortmoeglichkeit
				L3={$P(X\geq 3)+P(X\leq 6)$},   %3. Antwortmoeglichkeit
				L4={$P(3\leq X \leq 6)$},   %4. Antwortmoeglichkeit
				L5={$P(X\leq 6)-P(X<2)$},	 %5. Antwortmoeglichkeit
				L6={$P(3<X<6)$},	 %6. Antwortmoeglichkeit
				L7={},	 %7. Antwortmoeglichkeit
				L8={},	 %8. Antwortmoeglichkeit
				L9={},	 %9. Antwortmoeglichkeit
				%% LOESUNG: %%
				A1=4,  % 1. Antwort
				A2=0,	 % 2. Antwort
				A3=0,  % 3. Antwort
				A4=0,  % 4. Antwort
				A5=0,  % 5. Antwort
				}
\end{beispiel}