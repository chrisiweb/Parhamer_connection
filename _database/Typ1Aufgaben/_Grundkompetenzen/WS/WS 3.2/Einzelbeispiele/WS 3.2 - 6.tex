\section{WS 3.2 - 6 - MAT - Wahrscheinlichkeitsverteilung - OA - Matura HT 2015/16}

\begin{beispiel}[WS 3.2]{1} %PUNKTE DES BEISPIELS
Der Wertebereich einer Zufallsvariablen X besteht aus den Werten $x_1, x_2, x_3$.
Man kennt die Wahrscheinlichkeit $P(X = x_1) = 0,4$. Außerdem weiß man, dass $x_3$ doppelt so wahrscheinlich wie $x_2$ ist. \leer

Berechne $P(X=x_2)$ und $P(X=x_3)$. \leer

$P(X=x_2)=$\,\antwort[\rule{5cm}{0.3pt}]{$0,2$} \leer

$P(X=x_3)=$\,\antwort[\rule{5cm}{0.3pt}]{$0,5$} \leer
\end{beispiel}