\section{WS 3.2 - 20 - MAT - Computerchips - OA - Matura 2. NT 2017/18}

\begin{beispiel}[WS 3.2]{1}
Ein Unternehmen stellt Computerchips her. Jeder produzierte Computerchip ist unabhängig von
den anderen mit einer Wahrscheinlichkeit von 97\,\% funktionsfähig. Das Unternehmen produziert an einem bestimmten Tag 500 Computerchips.

Berechne den Erwartungswert und die Standardabweichung für die Anzahl der funktionsfähigen Computerchips, die an diesem bestimmten Tag produziert werden!\leer

Erwartungswert: \antwort[\rule{5cm}{0.3pt}]{$500 \cdot 0,97 =485$} \leer

Standardabweichung: \antwort[\rule{5cm}{0.3pt}]{$\sqrt{500 \cdot 0,97 \cdot 0,03}\approx 3,81$}

\antwort{Toleranzintervall für die Standardabweichung: $[3,8 ; 3,82]$}
\end{beispiel}