\section{WS 3.2 - 18 Massenproduktion - OA - Matura 17/18}

\begin{beispiel}[WS 3.2]{1} %PUNKTE DES BEISPIELS
Bei einer Massenproduktion eines bestimmten Produkts werden Packungen zu 100 Stück erzeugt. In einer solchen Packung ist jedes einzelne Stück (unabhängig von den anderen) mit einer Wahrscheinlichkeit von $6\,\%$ mangelhaft.

Ermittle, mit welcher Wahrscheinlichkeit in dieser Packung höchsten zwei mangelhafte Stücke zu finden sind!\leer

\antwort{Mögliche Vorgehensweise:

Die (binomialverteilte) Zufallsvariable $X$ (mit den Parametern $n=100$ und $p=0,06$) beschreibt die Anzahl der mangelhaften Stücke in dieser Packung.

$P(X\leq 2)=P(X=0)+P(X=1)+P(X=2)\approx 0,057$}
\end{beispiel}