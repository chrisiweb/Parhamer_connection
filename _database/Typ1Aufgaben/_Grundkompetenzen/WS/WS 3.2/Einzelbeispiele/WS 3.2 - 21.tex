\section{WS 3.2 - 21 - MAT - Trefferwahrscheinlichkeit - ZO - Matura-HT-18/19}

\begin{beispiel}[WS 3.2]{1}
Bei einem Training wirft eine Basketballspielerin einen Ball sechsmal hintereinander zum Korb.
Fällt der Ball in den Korb, spricht man von einem Treffer. Die Trefferwahrscheinlichkeit dieser
Spielerin beträgt bei jedem Wurf 0,85 (unabhängig von den anderen Würfen).\leer


Ordne den vier Ereignissen jeweils denjenigen Term (aus A bis F) zu, der die Wahrscheinlichkeit des Eintretens dieses Ereignisses beschreibt.

\zuordnen{
				R1={Die Spielerin trifft genau einmal.},				% Response 1
				R2={Die Spielerin trifft höchstens einmal.},				% Response 2
				R3={Die Spielerin trifft mindestens einmal.},				% Response 3
				R4={Die Spielerin trifft genau zweimal.},				% Response 4
				%% Moegliche Zuordnungen: %%
				A={$1-0,85^6$}, 				%Moeglichkeit A  
				B={$0,15^6+\binom{6}{1}\cdot 0,85^1\cdot 0,15^5$}, 				%Moeglichkeit B  
				C={$1-0,15^6$}, 				%Moeglichkeit C  
				D={$0,85^6+\binom{6}{1}\cdot 0,85^5\cdot 0,15^1$}, 				%Moeglichkeit D  
				E={$6\cdot 0,85 \cdot 0,15^5$}, 				%Moeglichkeit E  
				F={$\binom{6}{2}\cdot 0,85^2 \cdot 0,15^4$}, 				%Moeglichkeit F  
				%% LOESUNG: %%
				A1={E},				% 1. richtige Zuordnung
				A2={B},				% 2. richtige Zuordnung
				A3={C},				% 3. richtige Zuordnung
				A4={F},				% 4. richtige Zuordnung
				}
\end{beispiel}