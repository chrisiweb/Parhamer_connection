\section{WS 3.2 - 16 - MAT - Wahrscheinlichkeit bestimmen - OA - Matura 1. NT 2016/17}

\begin{beispiel}[WS 3.2]{1} %PUNKTE DES BEISPIELS
Die nachstehende Abbildung zeigt die Wahrscheinlichkeitsverteilung einer Zufallsvariablen X.

\begin{center}
	\resizebox{0.8\linewidth}{!}{\newrgbcolor{rctzbb}{0.10980392156862745 0.2235294117647059 0.7333333333333333}
	\psset{xunit=1cm,yunit=20cm,algebraic=true,dimen=middle,dotstyle=o,dotsize=5pt 0,linewidth=1.6pt,arrowsize=3pt 2,arrowinset=0.25}\begin{pspicture*}(-0.8143970021774691,-0.025)(11.5,0.27435976364031134)
	\multips(0,0)(0,0.05){6}{\psline[linestyle=dashed,linecap=1,dash=1.5pt 1.5pt,linewidth=0.4pt,linecolor=gray]{c-c}(0,0)(11.5,0)}
	\multips(0,0)(1,0){12}{\psline[linestyle=dashed,linecap=1,dash=1.5pt 1.5pt,linewidth=0.4pt,linecolor=gray]{c-c}(0,0)(0,0.27435976364031134)}
	\psaxes[comma,labelFontSize=\scriptstyle,xAxis=true,yAxis=true,Dx=1,Dy=0.05,ticksize=-2pt 0,subticks=0]{->}(0,0)(0,0)(11.5,0.27435976364031134)[k,120] [$P(X=k)$,-40]
	\psframe[linewidth=1.2pt,linecolor=rctzbb](0.9,0)(1.1,0.00604661760000001)
	\psframe[linewidth=1.2pt,linecolor=rctzbb](1.9,0)(2.1,0.040310784)
	\psframe[linewidth=1.2pt,linecolor=rctzbb](2.9,0)(3.1,0.120932352)
	\psframe[linewidth=1.2pt,linecolor=rctzbb](3.9,0)(4.1,0.214990848)
	\psframe[linewidth=1.2pt,linecolor=rctzbb](4.9,0)(5.1,0.250822656)
	\psframe[linewidth=1.2pt,linecolor=rctzbb](5.9,0)(6.1,0.2006581248)
	\psframe[linewidth=1.2pt,linecolor=rctzbb](6.9,0)(7.1,0.111476736)
	\psframe[linewidth=1.2pt,linecolor=rctzbb](7.9,0)(8.1,0.0424673280000001)
	\psframe[linewidth=1.2pt,linecolor=rctzbb](8.9,0)(9.1,0.010616832)
	\psframe[linewidth=1.2pt,linecolor=rctzbb](9.9,0)(10.1,0.001572864)
	\psframe[linewidth=1.2pt,linecolor=rctzbb](10.9,0)(11.1,0.0001048576)
	\psframe[linewidth=1.2pt,linecolor=blue,fillcolor=blue,fillstyle=solid,opacity=0.5](0.9,0)(1.1,0.00604661760000001)
	\psframe[linewidth=1.2pt,linecolor=blue,fillcolor=blue,fillstyle=solid,opacity=0.5](1.9,0)(2.1,0.040310784)
	\psframe[linewidth=1.2pt,linecolor=blue,fillcolor=blue,fillstyle=solid,opacity=0.5](2.9,0)(3.1,0.120932352)
	\psframe[linewidth=1.2pt,linecolor=blue,fillcolor=blue,fillstyle=solid,opacity=0.5](3.9,0)(4.1,0.214990848)
	\psframe[linewidth=1.2pt,linecolor=blue,fillcolor=blue,fillstyle=solid,opacity=0.5](4.9,0)(5.1,0.250822656)
	\psframe[linewidth=1.2pt,linecolor=blue,fillcolor=blue,fillstyle=solid,opacity=0.5](5.9,0)(6.1,0.2006581248)
	\psframe[linewidth=1.2pt,linecolor=blue,fillcolor=blue,fillstyle=solid,opacity=0.5](6.9,0)(7.1,0.111476736)
	\psframe[linewidth=1.2pt,linecolor=blue,fillcolor=blue,fillstyle=solid,opacity=0.5](7.9,0)(8.1,0.0424673280000001)
	\psframe[linewidth=1.2pt,linecolor=blue,fillcolor=blue,fillstyle=solid,opacity=0.5](8.9,0)(9.1,0.010616832)
	\psframe[linewidth=1.2pt,linecolor=blue,fillcolor=blue,fillstyle=solid,opacity=0.5](9.9,0)(10.1,0.001572864)
	\psframe[linewidth=1.2pt,linecolor=blue,fillcolor=blue,fillstyle=solid,opacity=0.5](10.9,0)(11.1,0.0001048576)
	\end{pspicture*}}
\end{center}

Gib mithilfe dieser Abbildung näherungsweise die Wahrscheinlichkeit \mbox{$P(4\leq X\leq 7)$} an!\leer

$P(4\leq X<7)\approx$ \antwort[\rule{3cm}{0.3pt}]{0,55}

\antwort{$[0,54; 0,56]$ bzw. $[54\,\%;56\,\%]$}
\end{beispiel}