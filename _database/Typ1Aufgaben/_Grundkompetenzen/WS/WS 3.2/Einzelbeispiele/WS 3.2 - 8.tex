\section{WS 3.2 - 8 Tennisspiel - OA - Matura 2014/15 - Haupttermin}

\begin{beispiel}[WS 3.2]{1} %PUNKTE DES BEISPIELS
Stefan und Helmut spielen im Training 5 Sätze Tennis. Stefan hat eine konstante Gewinnwahrscheinlichkeit von 60\,\% für jeden gespielten Satz. \leer

Es wird folgender Wert berechnet:\\

$\begin{pmatrix}5 \\ 3 \\\end{pmatrix} \cdot 0,4^3 \cdot 0,6^2 =0,2304$ \leer

Gib an, was dieser Wert im Zusammenhang mit der Angabe aussagt.

\antwort{Dieser Wert gibt die Wahrscheinlichkeit an, mit der Helmut 3 von 5 Sätzen im Training gewinnt.}

\end{beispiel}