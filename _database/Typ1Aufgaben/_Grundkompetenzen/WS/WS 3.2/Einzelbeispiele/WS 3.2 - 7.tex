\section{WS 3.2 - 7 - MAT - Erwartungswert des Gewinns - OA - Matura HT 2014/15}

\begin{beispiel}[WS 3.2]{1} %PUNKTE DES BEISPIELS
Bei einem Gewinnspiel gibt es 100 Lose. Der Lospreis beträgt \euro\,5. Für den Haupttreffer werden \euro\,100 ausgezahlt, für zwei weitere Treffer werden je \euro\,50 ausgezahlt und für fünf weitere Treffer werden je \euro\,20 ausgezahlt. Für alle weiteren Lose wird nichts ausgezahlt. Unter \textit{Gewinn} versteht man \textit{Auszahlung minus Lospreis}.

Berechne den Erwartungswert des Gewinns aus der Sicht einer Person, die ein Los kauft.

\antwort{$E=\frac{1}{100}\cdot 100 + \frac{2}{100} \cdot 50 + \frac{5}{100} \cdot 20 -5 =-2$ \leer

$E=\frac{92}{100}\cdot (-5) + \frac{5}{100} \cdot 15 +\frac{2}{100}\cdot 45 + \frac{1}{100} \cdot 95 = -2$ \leer

Der Erwartungswert des Gewinns beträgt \euro\,-2 \leer

Lösungsschlüssel:\\
Ein Punkt für die richtige Lösung, wobei die Einheit Euro nicht angeführt werden muss. Der Wert
$E = 2$ ist nur dann als richtig zu werten, wenn aus der Antwort klar hervorgeht, dass es sich dabei um einen Verlust von \euro\,2 aus Sicht der Person, die ein Los kauft, handelt. Die Aufgabe ist auch dann als richtig gelöst zu werten, wenn bei korrektem Ansatz das Ergebnis aufgrund eines Rechenfehlers nicht richtig ist.}
\end{beispiel}