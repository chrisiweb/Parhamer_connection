\section{WS 3.2 - 3 Kennzahlen der Binomialverteilung - OA - BIFIE}

\begin{beispiel}[WS 3.2]{1} %PUNKTE DES BEISPIELS
Auf einer Sortieranlage werden Flaschen von einem Scanner untersucht und es wird die Art
des Kunststoffes ermittelt. 95\,\% der Flaschen werden richtig erkannt und in die bereitgestellten Behälter einsortiert. Die Werte der Zufallsvariablen $X$ beschreiben die Anzahl der falschen Entscheidungen bei einem Stichprobenumfang von 500 Stück. Verwenden Sie die Binomialverteilung als Modell. \leer

Berechne den Erwartungswert und die Standardabweichung für die Zufallsvariable $X$.

\antwort{\leer

$\mu=n\cdot p = 500 \cdot 0,05 =25$\\

$\sigma=\sqrt{n\cdot p \cdot (1-p)}=\sqrt{500 \cdot 0,05 \cdot 0,95}=4,8734$\leer

Lösungsschlüssel: Die Aufgabe gilt nur dann als richtig gelöst, wenn beide Werte richtig berechnet sind und $\sigma$ im Lösungsintervall $[4,8~; 4,9]$ liegt. }
\end{beispiel} 