\section{WS 3.2 - 23 - MAT - Drei Würfe mit einem Kegel - OA - Matura 2018/19 2. NT}

\begin{beispiel}[WS 3.2]{1}
Wirft man einen Kegel, kann dieser entweder auf der Mantelfläche oder auf der Grundfläche zu liegen kommen.

Die Wahrscheinlichkeit dafür, dass dieser Kegel auf der Grundfläche zu liegen kommt, beträgt bei jedem Wurf unabhängig von den anderen Würfen 30\,\%.

Der Kegel wird im Zuge eines Zufallsexperiments dreimal geworfen. Die Zufallsvariable $X$ beschreibt, wie oft der Kegel dabei auf der Grundfläche zu liegen kommt.

Die unten stehende Tabelle soll die Wahrscheinlichkeitsverteilung der Zufallsvariablen $X$ angeben.

Ergänze die fehlenden Werte.

\begin{center}
\begin{tabular}{|c|c|}\hline
\cellcolor[gray]{0.9}$x$&\cellcolor[gray]{0.9}Wahrscheinlichkeit (gerundet)\\ \hline
$0$&$0,343$\\ \hline
$1$&$0,441$\\ \hline
$2$&\antwort{$0,189$}\\ \hline
$3$&\antwort{$0,027$}\\ \hline
\end{tabular}
\end{center}

\antwort{Toleranzintervall: $[0,18;0,19]$\\
Toleranzintervall: $[0,02; 0,03]$}
\end{beispiel}