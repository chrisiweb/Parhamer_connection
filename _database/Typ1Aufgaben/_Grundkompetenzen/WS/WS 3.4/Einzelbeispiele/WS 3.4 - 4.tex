\section{WS 3.4 - 4 - MAT - Rosenstöcke - OA - Matura 2016/17 2. NT}


\begin{langesbeispiel} \item[0] %PUNKTE DES BEISPIELS
Ein bestimmter Prozentsatz der Stöcke einer Rosensorte bringt gelbe Blüten hervor.

In einem Beet wird eine gewisse Anzahl an Rosenstöcken dieser Sorte gepflanzt. Die Zufallsvariable $X$ ist binomialverteilt und gibt die Anzahl der gelbblühenden Rosenstöcke an. Dabei beträgt der Erwartungswert für die Anzahl $X$ der gelbblühenden Rosenstöcke 32, und die Standardabweichung hat den Wert 4.

Es wird folgender Vergleich angestellt:
"`Die Wahrscheinlichkeit, dass sich in diesem Beet mindestens 28 und höchstens 36 gelbblühende Rosenstöcke befinden, ist größer als die Wahrscheinlichkeit, dass mehr als 32 gelbblühende Rosenstöcke vorhanden sind."'

Gib an, ob dieser Vergleich zutrifft, und begründe deine Entscheidung!\leer

\antwort{Der Vergleich trifft zu.\leer

Mögliche Begründung:

Erwartungswert: $\mu=32$, Standardabweichung: $\sigma=0,4$ unter Einbeziehung der Wahrscheinlichkeiten für $\sigma$-Umgebungen (bei Approximation durch die normalverteilte Zufallsvariable $Y$):

$P(28\leq X\leq 36)\approx P(\mu-\sigma\leq Y\leq \mu+\sigma)\approx 0,683$

$P(X>32)\approx P(Y>\mu)=0,5$ $\Rightarrow$ $P(28\leq X\leq 36)>P(X>32)$\leer

Weitere Begründungsvarianten:

$n$ ... Anzahl der Rosenstöcke
$p$ ... Wahrscheinlichkeit für einen gelbblühenden Rosenstock

$\mu=32=n\cdot p, \sigma^2=16=n\cdot p\cdot (1-p)\Rightarrow n=64,\, p=0,5$

\begin{itemize}
	\item mittels Binomialverteilung:
	
	$P(28\leq X\leq 36)\approx 0,7396$
	
	$P(X>32)\approx 0,4503$ $\Rightarrow$ $P(28\leq X\leq 36)>P(X>32)$
	
	\item mittels Approximation mit Stetigkeitskorrektur durch die normalverteilte Zufallsvariable $Y$:
	
	$P(28\leq X\leq 36)\approx P(27,5\leq Y\leq 36,5)\approx 0,7394$
	
	$P(X>32)\approx P(Y>32,5)\approx 0,4503$ $\Rightarrow$ $P(28\leq X\leq 36)>P(X>32)$
	
	\item mittels Approximation ohne Stetigkeitskorrektur durch die normalverteilte Zufallsvariable $Y$:
	
	$P(28\leq X\leq 36)\approx P(28\leq Y\leq 36)\approx 0,6827$
	
	$P(X>32)\approx P(Y>32)\approx 0,5$ $\Rightarrow$ $P(28\leq X\leq 36)>P(X>32)$
\end{itemize}
}
\end{langesbeispiel}