\section{WS 3.4 - 3 Grafische Deutung - OA - Matura 2016/17 - Haupttermin}

\begin{beispiel}[WS 3.4]{1} %PUNKTE DES BEISPIELS
In nachstehender Abbildung ist die Dichtefunktion $f$ der approximierenden Normalverteilung einer binomialverteilten Zufallsvariablen $X$ dargestellt. \leer


\resizebox{1\linewidth}{!}{\psset{xunit=0.3cm,yunit=40cm,algebraic=true,dimen=middle,dotstyle=o,dotsize=5pt 0,linewidth=0.8pt,arrowsize=3pt 2,arrowinset=0.25}
\begin{pspicture*}(23.5,-0.02)(71.5,0.12)
\pszigzag[coilarm=0.1,coilwidth=0.6,coilheight=0.5](24,0)(26,0)
\psaxes[comma, labelFontSize=\scriptstyle,xAxis=true,yAxis=false,Ox=50,Dx=2.,Dy=0.02,ticksize=-2pt 0,subticks=1]{->}(26,0)(26,0)(71.5,0.07)[$x$,140] [,-40]
\psaxes[comma, labels=none, xAxis=false,yAxis=true,Dy=0.02,ticksize=0pt 0,subticks=2]{->}(24,0)(24,0)(84,0.12)[,140] [$f(x)$,-40]
\psplot[linewidth=1.2pt,linecolor=black,plotpoints=200]{26}{85}{EXP((-(x-46.0)^(2.0))/(5^(2.0)*2.0))/(abs(5)*sqrt(3.141592653589793*2.0))}
\pscustom[linewidth=0.8pt,linecolor=black,fillcolor=black,fillstyle=solid,opacity=0.5]{\psplot{40.}{72.}{EXP((-(x-46.0)^(2.0))/(5.0^(2.0)*2.0))/(abs(5.0)*sqrt(3.141592653589793*2.0))}\lineto(72.,0)\lineto(40.,0)\closepath}
\begin{scriptsize}
\rput[bl](52,0.045){$f$}
\end{scriptsize}
\end{pspicture*}}


Deute den Fl�cheninhalt der grau markierten Fl�che im Hinblick auf die Berechnung einer
Wahrscheinlichkeit!

\antwort{$P(X\geq 64)$ 

oder: 

Der Fl�cheninhalt der dargestellten Fl�che beschreibt die Wahrscheinlichkeit, dass die Zufallsvariable X mindestens den Wert 64 annimmt. \leer

L�sungsschl�ssel:

Ein Punkt f�r eine (sinngem��) korrekte Deutung, wobei auch die Deutungen $P(X > 64)$ bzw.
$P(X \geq 65)$ oder $P(64 \leq X \leq b)$ mit $b \geq 85$ als richtig zu werten sind.
}
\end{beispiel}