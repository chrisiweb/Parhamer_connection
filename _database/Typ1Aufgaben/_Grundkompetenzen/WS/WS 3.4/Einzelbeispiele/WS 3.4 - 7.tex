\section{WS 3.4 - 7 - Approximation Normalverteilung - MC - VikRaf UNIVIE}

\begin{beispiel}[WS 3.4]{1}
In einer Fabrik werden Flaschen produziert. 20 $\%$ der produzierten Glasflaschen gehen beim Verarbeitungsprozess kaputt. Gesucht ist die Wahrscheinlichkeit, dass bei 300 Glasflaschen
mindestens 45 aber höchstens 70 Flaschen bei der Produktion zerstört wurden. 

Welche der folgenden grau markierten Flächen beschreibt die gesuchte Wahrscheinlichkeit?

Kreuze die richtige Antwort an. 

\langmultiplechoice[6]{  %Anzahl der Antwortmoeglichkeiten, Standard: 5
				L1={%{0.49019607843137253 0.49019607843137253 1.}
\psset{xunit=0.07692307692307693cm,yunit=62.652005629930194cm,algebraic=true,dimen=middle,dotstyle=o,dotsize=5pt 0,linewidth=1pt,arrowsize=1.5pt 2,arrowinset=0.25}
\begin{pspicture*}(22.,-0.008032664459910055)(100.,0.07)
%\multips(0,0)(0,0.01){8}{\psline[linestyle=dashed,linecap=1,dash=1.5pt 1.5pt,linewidth=0.4pt,linecolor=lightgray]{c-c}(22.,0)(100.,0)}
%\multips(22,0)(10.0,0){8}{\psline[linestyle=dashed,linecap=1,dash=1.5pt 1.5pt,linewidth=0.4pt,linecolor=lightgray]{c-c}(0,-0.008032664459910055)(0,0.07)}
\psaxes[labelFontSize=\scriptstyle,xAxis=true,yAxis=true,Dx=10.,Dy=0.01,ticksize=-4pt 0,subticks=10]{->}(0,0)(22.,-0.008032664459910055)(100.,0.07)
\pscustom[linewidth=0.8pt,linecolor=gray,fillcolor=gray,fillstyle=solid,opacity=0.25]{\psplot{24.35}{70.}{EXP((-(x-60.0)^(2.0))/(6.93^(2.0)*2.0))/(abs(6.93)*sqrt(3.141592653589793*2.0))}\lineto(70.,0)\lineto(24.35,0)\closepath}
\psplot[linewidth=1.2pt,plotpoints=200]{22.0}{100.0}{EXP((-(x-60.0)^(2.0))/(6.93^(2.0)*2.0))/(abs(6.93)*sqrt(3.141592653589793*2.0))}
\end{pspicture*}},   %1. Antwortmoeglichkeit 
				L2={%{0.49019607843137253 0.49019607843137253 1.}
\psset{xunit=0.07692307692307693cm,yunit=62.652005629930194cm,algebraic=true,dimen=middle,dotstyle=o,dotsize=5pt 0,linewidth=1pt,arrowsize=1.5pt 2,arrowinset=0.25}
\begin{pspicture*}(22.,-0.01232075731388861)(100.,0.07)
%\multips(0,0)(0,0.01){9}{\psline[linestyle=dashed,linecap=1,dash=1.5pt 1.5pt,linewidth=0.4pt,linecolor=lightgray]{c-c}(22.,0)(100.,0)}
%\multips(22,0)(10.0,0){8}{\psline[linestyle=dashed,linecap=1,dash=1.5pt 1.5pt,linewidth=0.4pt,linecolor=lightgray]{c-c}(0,-0.01232075731388861)(0,0.07)}
\psaxes[labelFontSize=\scriptstyle,xAxis=true,yAxis=true,Dx=10.,Dy=0.01,ticksize=-4pt 0,subticks=10]{->}(0,0)(22.,-0.01232075731388861)(100.,0.07)
\pscustom[linewidth=0.8pt,linecolor=gray,fillcolor=gray,fillstyle=solid,opacity=0.25]{\psplot{46.}{69.}{EXP((-(x-60.0)^(2.0))/(6.93^(2.0)*2.0))/(abs(6.93)*sqrt(3.141592653589793*2.0))}\lineto(69.,0)\lineto(46.,0)\closepath}
\psplot[linewidth=1.6pt,plotpoints=200]{22.0}{100.0}{EXP((-(x-60.0)^(2.0))/(6.93^(2.0)*2.0))/(abs(6.93)*sqrt(3.141592653589793*2.0))}
\end{pspicture*}},   %2. Antwortmoeglichkeit
				L3={%{0.49019607843137253 0.49019607843137253 1.}
\psset{xunit=0.07692307692307693cm,yunit=62.652005629930194cm,algebraic=true,dimen=middle,dotstyle=o,dotsize=5pt 0,linewidth=1pt,arrowsize=1.5pt 2,arrowinset=0.25}
\begin{pspicture*}(22.,-0.008032664459910055)(100.,0.07)
%\multips(0,0)(0,0.01){8}{\psline[linestyle=dashed,linecap=1,dash=1.5pt 1.5pt,linewidth=0.4pt,linecolor=lightgray]{c-c}(22.,0)(100.,0)}
%\multips(22,0)(10.0,0){8}{\psline[linestyle=dashed,linecap=1,dash=1.5pt 1.5pt,linewidth=0.4pt,linecolor=lightgray]{c-c}(0,-0.008032664459910055)(0,0.07)}
\psaxes[labelFontSize=\scriptstyle,xAxis=true,yAxis=true,Dx=10.,Dy=0.01,ticksize=-4pt 0,subticks=10]{->}(0,0)(22.,-0.008032664459910055)(100.,0.07)
\pscustom[linewidth=0.8pt,linecolor=gray,fillcolor=gray,fillstyle=solid,opacity=0.25]{\psplot{70.}{95.65}{EXP((-(x-60.0)^(2.0))/(6.93^(2.0)*2.0))/(abs(6.93)*sqrt(3.141592653589793*2.0))}\lineto(95.65,0)\lineto(70.,0)\closepath}
\psplot[linewidth=1.6pt,plotpoints=200]{22.0}{100.0}{EXP((-(x-60.0)^(2.0))/(6.93^(2.0)*2.0))/(abs(6.93)*sqrt(3.141592653589793*2.0))}
\end{pspicture*}},   %3. Antwortmoeglichkeit
				L4={%{0.49019607843137253 0.49019607843137253 1.}
\psset{xunit=0.07692307692307693cm,yunit=62.652005629930194cm,algebraic=true,dimen=middle,dotstyle=o,dotsize=5pt 0,linewidth=1pt,arrowsize=1.5pt 2,arrowinset=0.25}
\begin{pspicture*}(22.,-0.010316623519109312)(100.,0.07)
%\multips(0,0)(0,0.01){9}{\psline[linestyle=dashed,linecap=1,dash=1.5pt 1.5pt,linewidth=0.4pt,linecolor=lightgray]{c-c}(22.,0)(100.,0)}
%\multips(22,0)(10.0,0){8}{\psline[linestyle=dashed,linecap=1,dash=1.5pt 1.5pt,linewidth=0.4pt,linecolor=lightgray]{c-c}(0,-0.010316623519109312)(0,0.07)}
\psaxes[labelFontSize=\scriptstyle,xAxis=true,yAxis=true,Dx=10.,Dy=0.01,ticksize=-4pt 0,subticks=10]{->}(0,0)(22.,-0.010316623519109312)(100.,0.07)
\pscustom[linewidth=0.8pt,linecolor=gray,fillcolor=gray,fillstyle=solid,opacity=0.25]{\psplot{45.}{70.}{EXP((-(x-60.0)^(2.0))/(6.93^(2.0)*2.0))/(abs(6.93)*sqrt(3.141592653589793*2.0))}\lineto(70.,0)\lineto(45.,0)\closepath}
\psplot[linewidth=1.2pt,plotpoints=200]{22.0}{100.0}{EXP((-(x-60.0)^(2.0))/(6.93^(2.0)*2.0))/(abs(6.93)*sqrt(3.141592653589793*2.0))}
\end{pspicture*}},   %4. Antwortmoeglichkeit
				L5={%{0.49019607843137253 0.49019607843137253 1.}
\psset{xunit=0.07692307692307693cm,yunit=62.652005629930194cm,algebraic=true,dimen=middle,dotstyle=o,dotsize=5pt 0,linewidth=1pt,arrowsize=1.5pt 2,arrowinset=0.25}
\begin{pspicture*}(22.,-0.010434986343445854)(100.,0.07)
%\multips(0,0)(0,0.01){9}{\psline[linestyle=dashed,linecap=1,dash=1.5pt 1.5pt,linewidth=0.4pt,linecolor=lightgray]{c-c}(22.,0)(100.,0)}
%\multips(22,0)(10.0,0){8}{\psline[linestyle=dashed,linecap=1,dash=1.5pt 1.5pt,linewidth=0.4pt,linecolor=lightgray]{c-c}(0,-0.010434986343445854)(0,0.07)}
\psaxes[labelFontSize=\scriptstyle,xAxis=true,yAxis=true,Dx=10.,Dy=0.01,ticksize=-4pt 0,subticks=10]{->}(0,0)(22.,-0.010434986343445854)(100.,0.07)
\pscustom[linewidth=0.8pt,linecolor=gray,fillcolor=gray,fillstyle=solid,opacity=0.25]{\psplot{71.}{95.65}{EXP((-(x-60.0)^(2.0))/(6.93^(2.0)*2.0))/(abs(6.93)*sqrt(3.141592653589793*2.0))}\lineto(95.65,0)\lineto(71.,0)\closepath}
\psplot[linewidth=1.6pt,plotpoints=200]{22.0}{100.0}{EXP((-(x-60.0)^(2.0))/(6.93^(2.0)*2.0))/(abs(6.93)*sqrt(3.141592653589793*2.0))}
\end{pspicture*}},	 %5. Antwortmoeglichkeit
				L6={
				%{0.49019607843137253 0.49019607843137253 1.}
\psset{xunit=0.07692307692307693cm,yunit=62.652005629930194cm,algebraic=true,dimen=middle,dotstyle=o,dotsize=5pt 0,linewidth=1pt,arrowsize=1.5pt 2,arrowinset=0.25}
\begin{pspicture*}(22.,-0.009805904850576612)(100.,0.07)
%\multips(0,0)(0,0.01){8}{\psline[linestyle=dashed,linecap=1,dash=1.5pt 1.5pt,linewidth=0.4pt,linecolor=lightgray]{c-c}(22.,0)(100.,0)}
%\multips(22,0)(10.0,0){8}{\psline[linestyle=dashed,linecap=1,dash=1.5pt 1.5pt,linewidth=0.4pt,linecolor=lightgray]{c-c}(0,-0.009805904850576612)(0,0.07)}
\psaxes[labelFontSize=\scriptstyle,xAxis=true,yAxis=true,Dx=10.,Dy=0.01,ticksize=-4pt 0,subticks=10]{->}(0,0)(22.,-0.009805904850576612)(100.,0.07)
\pscustom[linewidth=0.8pt,linecolor=gray,fillcolor=gray,fillstyle=solid,opacity=0.25]{\psplot{24.35}{71.}{EXP((-(x-60.0)^(2.0))/(6.93^(2.0)*2.0))/(abs(6.93)*sqrt(3.141592653589793*2.0))}\lineto(71.,0)\lineto(24.35,0)\closepath}
\psplot[linewidth=1.6pt,plotpoints=200]{22.0}{100.0}{EXP((-(x-60.0)^(2.0))/(6.93^(2.0)*2.0))/(abs(6.93)*sqrt(3.141592653589793*2.0))}
\end{pspicture*}},	 %6. Antwortmoeglichkeit
				L7={},	 %7. Antwortmoeglichkeit
				L8={},	 %8. Antwortmoeglichkeit
				L9={},	 %9. Antwortmoeglichkeit
				%% LOESUNG: %%
				A1=4,  % 1. Antwort
				A2=0,	 % 2. Antwort
				A3=0,  % 3. Antwort
				A4=0,  % 4. Antwort
				A5=0,  % 5. Antwort
				}
\end{beispiel}