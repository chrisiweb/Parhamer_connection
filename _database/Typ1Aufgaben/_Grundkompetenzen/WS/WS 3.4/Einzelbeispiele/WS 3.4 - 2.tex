\section{WS 3.4 - 2 Benutzung des Autos - OA - BIFIE}

\begin{beispiel}[WS 3.4]{1} %PUNKTE DES BEISPIELS
Einer Veröffentlichung der Statistik Austria kann man entnehmen, dass von den
über 15-Jährigen Österreicherinnen und Österreichern ca. 38,6\,\% täglich das Auto benutzen
(als Lenker/in oder als Mitfahrer/in). 

\tiny
\singlespacing
\begin{flushright}
 Quelle: Statistik Austria (Hrsg.) (2013). Umweltbedingungen, Umweltverhalten 2011. Ergebnisse des Mikrozensus. Wien: Statistik Austria. S. 95.
\end{flushright}

\onehalfspacing 
\normalsize

\leer

Es werden 500 über 15-jährige Österreicher/innen zufällig ausgewählt. \leer

Gib für die Anzahl derjenigen Personen, die täglich das Auto (als Lenker/in oder als
Mitfahrer/in) benutzen, näherungsweise ein um den Erwartungswert symmetrisches Intervall mit
95\,\%iger Wahrscheinlichkeit an. 

\antwort{\leer

Die binomialverteilte Zufallsvariable $X$ gibt die Anzahl der über 15-Jährigen an, die täglich das Auto benutzen. \leer

$n=500$ \\
$p=0,386 \Rightarrow 1-p=0,614$\leer

Approximation der Binomialverteilung durch die Normalverteilung: \\
$\mu=193$\\
$\sigma=\sqrt{500 \cdot 0,386 \cdot 0,614} \approx 10,886$\\
$2 \cdot \Phi(z)-1=D(z)=0,95 \Rightarrow z \approx 1,96 $\\
$x_{1,2}=\mu\pm z \cdot \sigma \Rightarrow x_1 \approx 171;~ x_2\approx 215 \Rightarrow [171;~215]$ \leer

Lösungsschlüssel: Ein Punkt für die Angabe eines symmetrischen Lösungsintervalls laut Lösungserwartung.\\
Toleranzintervall für die untere Grenze: $[170;~173]$\\
Toleranzintervall für die obere Grenze: $[213;~216]$ \\
}

\end{beispiel} 