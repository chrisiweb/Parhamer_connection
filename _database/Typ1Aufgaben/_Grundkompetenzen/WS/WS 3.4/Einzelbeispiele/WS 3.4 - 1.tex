\section{WS 3.4 - 1 Sch�lerarbeit - LT - BIFIE}

\begin{beispiel}[WS 3.4]{1} %PUNKTE DES BEISPIELS
Die Spinde einer Schule werden mit Vorh�ngeschl�ssern gesichert, die im Eigentum der Sch�ler/innen stehen. Erfahrungsgem�� m�ssen 5\,\% aller Spindschl�sser innerhalb eines Jahres
aufgebrochen werden, weil die Schl�ssel verloren wurden. Ein Sch�ler berechnet die Wahrscheinlichkeit, dass innerhalb eines Jahres von 200 Schl�ssern mindestens zw�lf aufgebrochen
werden m�ssen. Die nachstehenden Aufzeichnungen zeigen seine Vorgehensweise. 

\begin{quote}
\footnotesize
\color[rgb]{0.3,0.3,0.3}
$P(X\geq12)$ \ldots Berechnung bzw. Berechnung der Gegen-WSK zu umst�ndlich!

$\mu= 200\cdot 0,05 =10$ \\
$\sigma=\sqrt{200 \cdot 0,05\cdot 0,95} \approx 3,08 > 3$ \checkmark \leer

$z=\dfrac{x - \mu}{\sigma}=\dfrac{11,5 - 10}{\sigma} \approx 0,49$ \leer

$\Phi(0,49) = 0,6879$

$\Rightarrow P(X\geq 12)\cong 1 - 0,6879 \cong 0,3121$

$\Rightarrow \underline{\underline{z_n \approx 31\,\%}}$

\end{quote}
\color[rgb]{0,0,0} \vspace{-0.7cm}\leer

\normalsize
\lueckentext[0.02]{
				text={Bei der Anzahl der Schl�sser, die aufgebrochen werden m�ssen, handelt es sich um
eine \gap, und \gap.}, 	%Lueckentext Luecke=\gap
				L1={gleichverteilte Zufallsvariable }, 		%1.Moeglichkeit links  
				L2={binomialverteilte Zufallsvariable}, 		%2.Moeglichkeit links
				L3={normalverteilte Zufallsvariable}, 		%3.Moeglichkeit links
				R1={der Sch�ler rechnet mit der Normalverteilung,
obwohl es nicht zul�ssig ist}, 		%1.Moeglichkeit rechts 
				R2={der Sch�ler verwechselt den Mittelwert
mit dem Erwartungswert, also ist die
Aufgabe deshalb nicht richtig gel�st }, 		%2.Moeglichkeit rechts
				R3={der Sch�ler rechnet zul�ssigerweise
mit der Normalverteilung}, 		%3.Moeglichkeit rechts
				%% LOESUNG: %%
				A1=2,   % Antwort links
				A2=3		% Antwort rechts 
				}



\end{beispiel}