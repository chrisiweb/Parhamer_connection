\section{WS 3.4 - 1 - Schülerarbeit - LT - BIFIE}

\begin{beispiel}[WS 3.4]{1} %PUNKTE DES BEISPIELS
Die Spinde einer Schule werden mit Vorhängeschlössern gesichert, die im Eigentum der Schüler/innen stehen. Erfahrungsgemäß müssen 5\,\% aller Spindschlösser innerhalb eines Jahres
aufgebrochen werden, weil die Schlüssel verloren wurden. Ein Schüler berechnet die Wahrscheinlichkeit, dass innerhalb eines Jahres von 200 Schlössern mindestens zwölf aufgebrochen
werden müssen. Die nachstehenden Aufzeichnungen zeigen seine Vorgehensweise. 

\begin{quote}
\footnotesize
\color[rgb]{0.3,0.3,0.3}
$P(X\geq12)$ \ldots Berechnung bzw. Berechnung der Gegen-WSK zu umständlich!

$\mu= 200\cdot 0,05 =10$ \\
$\sigma=\sqrt{200 \cdot 0,05\cdot 0,95} \approx 3,08 > 3$ \checkmark \leer

$z=\dfrac{x - \mu}{\sigma}=\dfrac{11,5 - 10}{\sigma} \approx 0,49$ \leer

$\Phi(0,49) = 0,6879$

$\Rightarrow P(X\geq 12)\cong 1 - 0,6879 \cong 0,3121$

$\Rightarrow \underline{\underline{z_n \approx 31\,\%}}$

\end{quote}
\color[rgb]{0,0,0} \vspace{-0.7cm}\leer

\normalsize
\lueckentext[0.02]{
				text={Bei der Anzahl der Schlösser, die aufgebrochen werden müssen, handelt es sich um
eine \gap, und \gap.}, 	%Lueckentext Luecke=\gap
				L1={gleichverteilte Zufallsvariable }, 		%1.Moeglichkeit links  
				L2={binomialverteilte Zufallsvariable}, 		%2.Moeglichkeit links
				L3={normalverteilte Zufallsvariable}, 		%3.Moeglichkeit links
				R1={der Schüler rechnet mit der Normalverteilung,
obwohl es nicht zulässig ist}, 		%1.Moeglichkeit rechts 
				R2={der Schüler verwechselt den Mittelwert
mit dem Erwartungswert, also ist die
Aufgabe deshalb nicht richtig gelöst }, 		%2.Moeglichkeit rechts
				R3={der Schüler rechnet zulässigerweise
mit der Normalverteilung}, 		%3.Moeglichkeit rechts
				%% LOESUNG: %%
				A1=2,   % Antwort links
				A2=3		% Antwort rechts 
				}



\end{beispiel}