\section{WS 1.3 - 20 - MAT - Freizeitverhalten von Jugendlichen - OA - Matura-HT-18/19}

\begin{beispiel}[WS 1.3]{1}
Es wurden 400 Jugendliche zu ihrem Freizeitverhalten befragt. Von allen Befragten gaben 330 an,
Mitglied in einem Sportverein zu sein, 146 gaben an, ein Instrument zu spielen, und 98 gaben an,
sowohl Mitglied in einem Sportverein zu sein als auch ein Instrument zu spielen.\leer

Das Ergebnis dieser Befragung ist in der nachstehenden Tabelle eingetragen.


\begin{center}
\begin{tabu}{|l|c|c|c|} \cline{2-4}
\multicolumn{1}{l|}{}& spielt Instrument &  spielt kein Instrument &gesamt \\ \hline
Mitglied in Sportverein & 98 & \antwort{232} & 330 \\ \hline
kein Mitglied in Sportverein &\antwort{48} & \antwort{22} & \antwort{70} \\ \hline
gesamt & 146 &\antwort{254} & 400 \\ \hline
\end{tabu}
\end{center}

Gib die relative Häufigkeit $h$ der befragten Jugendlichen an, die weder Mitglied in einem
Sportverein sind noch ein Instrument spielen!\leer

$h=\antwort[\rule{6cm}{0.3pt}]{\frac{22}{400}=0,055}$

\antwort{Toleranzintervall: $[0,05; 0,06]$ bzw. $[5\,\%; 6\,\%]$}
\end{beispiel}