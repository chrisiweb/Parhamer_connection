\section{WS 1.3 - 14 Mittlere Fehlstundenanzahl - OA - Matura NT 2 15/16}

\begin{beispiel}[WS 1.3]{1} %PUNKTE DES BEISPIELS
In einer Schule gibt es vier Sportklassen: $S1,S2,S3$ und $S4$. Die nachstehende Tabelle gibt eine Übersicht über die Anzahl der SchülerInnen pro Klasse sowie das jeweilige arithmetische Mittel der während des ersten Semesters eines Schuljahres versäumten Unterrichtsstunden.

\begin{center}
	\begin{tabular}{|c|c|c|}\hline
	Klasse&Anzahl der&arithmetisches Mittel der\\
	&SchülerInnen&versäumten Stunden\\ \hline
	$S1$&$18$&$45,5$\\ \hline
	$S2$&$20$&$63,2$\\ \hline
	$S3$&$16$&$70,5$\\ \hline
	$S4$&$15$&$54,6$\\ \hline
	\end{tabular}
\end{center}

Berechne das arithmetische Mittel $\overline{x}_{ges}$ der versäumten Unterrichtsstunden aller SchülerInnen der vier Sportklassen für den angegebenen Zeitraum!\leer

\antwort{$\overline{x}_{ges}=\dfrac{18\cdot 45,5+20\cdot 63,2+16\cdot 70,5+15\cdot 54,6}{18+20+16+5}=58,405...$

$\overline{x}_{ges}\approx58,4 h$

Einheit "`h"' muss nicht angegeben sein! Toleranzintervall: $[58\,h;60\,h]$.}
\end{beispiel}