\section{WS 1.3 - 7 - Sportwettbewerb - MC - BIFIE}

\begin{beispiel}[WS 1.3]{1} %PUNKTE DES BEISPIELS
				150 Grazer und 170 Wiener Schüler/innen nahmen an einem Sportwettbewerb teil. Der Vergleich der Listen der Hochsprungergebnisse ergibt für beide Schülergruppen das gleiche
arithmetische Mittel von 1,05 m sowie eine empirische Standardabweichung für die Grazer von 0,22 m und für die Wiener von 0,3 m.

Entscheide, welche Aussagen aus den gegebenen Daten geschlossen werden können, und kreuze die beiden zutreffenden Aussagen an.

\multiplechoice[5]{  %Anzahl der Antwortmoeglichkeiten, Standard: 5
				L1={Die Sprunghöhen der Grazer Schüler/innen weichen
vom arithmetischen Mittel nicht so stark ab wie die Höhen der Wiener Schüler/innen.},   %1. Antwortmoeglichkeit 
				L2={Das arithmetische Mittel repräsentiert die Leistungen
der Grazer Schüler/innen besser als die der Wiener.},   %2. Antwortmoeglichkeit
				L3={Die Standardabweichung der Grazer ist aufgrund der
geringeren Teilnehmerzahl kleiner als die der Wiener.},   %3. Antwortmoeglichkeit
				L4={Von den Sprunghöhen (gemessen in m) der Wiener
liegt kein Wert außerhalb des Intervalls $\left[0,45; 1,65\right]$.},   %4. Antwortmoeglichkeit
				L5={Beide Listen haben den gleichen Median.},	 %5. Antwortmoeglichkeit
				L6={},	 %6. Antwortmoeglichkeit
				L7={},	 %7. Antwortmoeglichkeit
				L8={},	 %8. Antwortmoeglichkeit
				L9={},	 %9. Antwortmoeglichkeit
				%% LOESUNG: %%
				A1=1,  % 1. Antwort
				A2=2,	 % 2. Antwort
				A3=0,  % 3. Antwort
				A4=0,  % 4. Antwort
				A5=0,  % 5. Antwort
				}
				\end{beispiel}