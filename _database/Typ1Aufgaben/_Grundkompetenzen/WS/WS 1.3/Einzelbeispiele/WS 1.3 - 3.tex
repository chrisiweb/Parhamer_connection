\section{WS 1.3 - 3 Mittelwert einfacher Datensätze - MC - BIFIE}

\begin{beispiel}[WS 1.3]{1} %PUNKTE DES BEISPIELS
Die unten stehende Tabelle bietet eine Übersicht über die Zahl der Einbürgerungen in Österreich und in den jeweiligen Bundesländern im Jahr 2010 nach Quartalen. Ein Quartal fasst dabei jeweils den Zeitraum von drei Monaten zusammen. Das 1. Quartal ist der Zeitraum von Jänner bis März, das 2. Quartal der Zeitraum von April bis Juni usw.\\

\begin{scriptsize}
\begin{longtable}{|C{1.1cm}|C{1.1cm}|C{1.1cm}|C{1.1cm}|C{1.1cm}|C{1.1cm}|C{1.1cm}|C{1.1cm}|C{1.1cm}|C{1.1cm}|C{1.1cm}|} \hline
\multirow{2}{*}{Quartal}&\multirow{2}{1.1cm}{Öster-\newline reich}&\multicolumn{9}{|c|}{Bundesland des Wohnortes} \\ \cline{3-11}
&&Burgen-\newline land&Kärnten&Nieder-\newline österreich&Ober-\newline österreich&Salzburg&Steier-\newline mark&Tirol&Vorarl-\newline berg&Wien\\ \hline
\multicolumn{8}{c}{\vspace{0.5cm}}\\ \hline
1.Quartal 2010&1142&1&119&87&216&112&101&131&97&278\\ \hline
2.Quartal\newline 2010&1605&80&120&277&254&148&106&138&125&357\\ \hline
3.Quartal\newline 2010&1532&4&119&187&231&98&121&122&61&589\\ \hline
4.Quartal\newline 2010&1856&53&113&248&294&158&102&183&184&52 \\ \hline
\end{longtable} \vspace{-0.5cm} \tiny{Quelle: STATISTIK AUSTRIA}
\end{scriptsize}

\normalsize

Kreuze die beiden korrekten Berechnungsmöglichkeiten für den Mittelwert der Einbürgerungen im Bundesland Kärnten pro Quartal im Jahr 2010 an.\\

\multiplechoice[5]{  %Anzahl der Antwortmoeglichkeiten, Standard: 5
				L1={$\overline{m}=(1142+1605+1532+1856):9$},   %1. Antwortmoeglichkeit 
				L2={$\overline{m}=\frac{2\cdot 119+113+120}{4}$},   %2. Antwortmoeglichkeit
				L3={$\overline{m}=119+120+119+113:4$},   %3. Antwortmoeglichkeit
				L4={$\overline{m}=\frac{1}{12}\cdot (113+2\cdot 119+120)\cdot 3$},   %4. Antwortmoeglichkeit
				L5={$\overline{m}=\frac{113+119+119+120}{12}\cdot 4$},	 %5. Antwortmoeglichkeit
				L6={},	 %6. Antwortmoeglichkeit
				L7={},	 %7. Antwortmoeglichkeit
				L8={},	 %8. Antwortmoeglichkeit
				L9={},	 %9. Antwortmoeglichkeit
				%% LOESUNG: %%
				A1=2,  % 1. Antwort
				A2=4,	 % 2. Antwort
				A3=0,  % 3. Antwort
				A4=0,  % 4. Antwort
				A5=0,  % 5. Antwort
				}
\end{beispiel}