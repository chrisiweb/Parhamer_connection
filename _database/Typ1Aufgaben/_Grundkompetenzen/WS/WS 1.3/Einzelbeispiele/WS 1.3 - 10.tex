\section{WS 1.3 - 10 - Median und Modus - OA - BIFIE Kompetenzcheck 2016}

\begin{beispiel}[WS 1.3]{1} %PUNKTE DES BEISPIELS
				Zwei unterscheidbare, faire Spielwürfel mit den Augenzahlen 1,2,3,4,5,6 werden geworfen und die Augensumme wird ermittelt. (Ein Würfel ist "`fair"', wenn die Wahrscheinlichkeit, nach einem Wurf nach oben zu zeigen, für alle sechs Seitenflächen gleich groß ist.)

Jemand behauptet, dass die Ereignisse "`Augensumme 5"' und "`Augensumme 9"' gleichwahrscheinlich sind. Gib an, ob es sich hierbei um eine wahre oder eine falsche Aussage handelt, und begründe deine Entscheidung.\\

\antwort{Die Aussage ist wahr. Mögliche Begründung:\\
Augensumme 5: $(1;4),(2;3),(3;2),(4;1)\Rightarrow$ 4 Möglichkeiten\\
Augensumme 9: $(3;6),(4;5),(5;4),(6;3)\Rightarrow$ 4 Möglichkeiten\\
P("`Augensumme 5"')=$\frac{4}{36}$\\
P("`Augensumme 9"')=$\frac{4}{36}$}
\end{beispiel}