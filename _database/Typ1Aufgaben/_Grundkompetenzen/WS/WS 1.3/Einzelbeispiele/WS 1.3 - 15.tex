\section{WS 1.3 - 15 Boxplot Analyse - MC - Matura 2013/14 Haupttermin}

\begin{beispiel}[WS 1.3]{1} %PUNKTE DES BEISPIELS
			Alle Mädchen und Burschen einer Schulklasse wurden über die Länge ihres Schulweges befragt. Die beiden Kastenschaubilder (Boxplots) geben Auskunft über ihre Antworten.
			
			\resizebox{1\linewidth}{!}{\psset{xunit=1.0cm,yunit=1.0cm,algebraic=true,dimen=middle,dotstyle=o,dotsize=5pt 0,linewidth=0.8pt,arrowsize=3pt 2,arrowinset=0.25}
\begin{pspicture*}(-0.78,-1.5)(15.28,6.94)
\psaxes[labelFontSize=\scriptstyle,xAxis=true,yAxis=false,labels=x,Dx=1.,Dy=1.,ticksize=-2pt 0,subticks=2]{->}(0,0)(-0.78,-1.5)(15.28,6.94)
\psframe(4.,4.5)(8.,5.5)
\psframe(4.,1.0)(6.,2.)
\psline(2.,4.5)(2.,5.5)
\psline(10.,4.5)(10.,5.5)
\psline(6.,4.5)(6.,5.5)
\psline(2.,5.)(4.,5.)
\psline(8.,5.)(10.,5.)
\psline(1.,1.)(1.,2.)
\psline(8.,1.)(8.,2.)
\psline(5.,1.)(5.,2.)
\psline(1.,1.5)(4.,1.5)
\psline(6.,1.5)(8.,1.5)
\begin{scriptsize}
\rput[tl](3.86,-0.8){Entfernung des Wohnortes gemessen in km}
\rput[tl](0.38,3.22){Burschen}
\rput[tl](0.46,6.5){Mädchen}
\rput[tl](1.92,6){2}
\rput[tl](0.92,2.5){1}
\rput[tl](3.92,6){4}
\rput[tl](3.92,2.5){4}
\rput[tl](5.92,6){6}
\rput[tl](4.92,2.5){5}
\rput[tl](7.92,6){8}
\rput[tl](5.92,2.5){6}
\rput[tl](9.92,6){10}
\rput[tl](7.92,2.5){8}
\end{scriptsize}
\end{pspicture*}}

Kreuze die beiden zutreffenden Aussagen an!\leer

\multiplechoice[5]{  %Anzahl der Antwortmoeglichkeiten, Standard: 5
				L1={Mehr als 60\,\% der befragten Mädchen haben einen Schulweg von mindestens 4\,km.},   %1. Antwortmoeglichkeit 
				L2={Der Median der erhobenen Daten ist bei Burschen und Mädchen gleich.},   %2. Antwortmoeglichkeit
				L3={Mindestens 50\,\% der Mädchen und mindestens 75\,\% der Burschen haben einen Schulweg, der kleiner oder gleich 6 km ist.},   %3. Antwortmoeglichkeit
				L4={Höchstens 40\,\% der befragten Burschen haben einen Schulweg zwischen 4\,km und 8\,km.},   %4. Antwortmoeglichkeit
				L5={Die Spannweite ist bei den Umfragedaten der Burschen genauso groß wie bei den Umfragedaten der Mädchen.},	 %5. Antwortmoeglichkeit
				L6={},	 %6. Antwortmoeglichkeit
				L7={},	 %7. Antwortmoeglichkeit
				L8={},	 %8. Antwortmoeglichkeit
				L9={},	 %9. Antwortmoeglichkeit
				%% LOESUNG: %%
				A1=1,  % 1. Antwort
				A2=0,	 % 2. Antwort
				A3=3,  % 3. Antwort
				A4=0,  % 4. Antwort
				A5=0,  % 5. Antwort
				}
\end{beispiel}