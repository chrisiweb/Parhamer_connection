\section{WS 1.3 - 5 - Arithmetisches Mittel einer Datenreihe - OA - BIFIE}

\begin{beispiel}[WS 1.3]{1} %PUNKTE DES BEISPIELS
				Für das arithmetische Mittel einer Datenreihe $x_{1},x_{2},...,x_{24}$ gilt: $\overline{x}=115$.

Die Standardabweichung der Datenreihe ist $s_{x}=12$. Die Werte einer zweiten Datenreihe $y_{1},y_{2},...,y_{24}$ entstehen, indem man zu den Werten der ersten Datenreihe jeweils 8 addiert, also $y_{1}=x_{1}+8,y_{2}=x_{2}+8$ usw.

Gib den Mittelwert $\overline{y}$ und die Standardabweichung $s_{y}$ der zweiten Datenreihe an.

$\overline{y}=$\antwort[\rule{5cm}{0.3pt}]{123}

$s_{y}=$\antwort[\rule{5cm}{0.3pt}]{12}


				\end{beispiel}