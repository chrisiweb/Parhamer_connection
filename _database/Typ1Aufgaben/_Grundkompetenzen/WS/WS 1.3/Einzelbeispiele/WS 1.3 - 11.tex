\section{WS 1.3 - 11 - MAT - Nettojahreseinkommen - OA - Matura HT 2014/15}

\begin{beispiel}[WS 1.3]{1} %PUNKTE DES BEISPIELS
Im Jahre 2012 gab es in Österreich unter den etwas mehr als 4 Millionen unselbstständig
Erwerbstätigen (ohne Lehrlinge) 40\,\% Arbeiterinnen und Arbeiter, 47\,\% Angestellte,
8\,\% Vertragsbedienstete und 5\,\% Beamtinnen und Beamte (Prozentzahlen gerundet). \leer

Die folgende Tabelle zeigt deren durchschnittliches Nettojahreseinkommen (arithmetisches Mittel).

\begin{center}
\begin{longtable}{|c|C{7cm}|} \cline{2-2}
\multicolumn{1}{c|}{} & arithmetisches Mittel der Nettojahreseinkommen 2012 (in Euro)  \\ \hline
Arbeiterinnen und Arbeiter & 14062 \\ \hline
Angestellte & 24141 \\ \hline
Vertragsbedienstete & 22853 \\ \hline
Beamtinnen und Beamte & 35708 \\ \hline
\end{longtable}
\end{center} \vspace{-1.5cm}
\tiny Datenquelle: Statistik Austria (Hrsg.) (2014). Statistisches Jahrbuch Österreichs 2015. Wien: Verlag Österreich. S. 246.
\normalsize

Ermittle das durchschnittliche Nettojahreseinkommen (arithmetisches Mittel) aller in
Österreich unselbstständig Erwerbstätigen (ohne Lehrlinge).

\antwort{
$14\,062 \cdot 0,4 + 24\,141 \cdot 0,47 + 22\,853 \cdot 0,08 + 35\,708 \cdot 0,05 = 20\, 584,71$ 

Das durchschnittliche Nettojahreseinkommen beträgt \euro\,20.584,71. \leer

Lösungsschlüssel:\\
Ein Punkt für die richtige Lösung, wobei die Einheit Euro nicht angeführt werden muss.
Toleranzintervall: $[20580; 20590]$
Die Aufgabe ist auch dann als richtig gelöst zu werten, wenn bei korrektem Ansatz das Ergebnis
aufgrund eines Rechenfehlers nicht richtig ist.}


\end{beispiel}