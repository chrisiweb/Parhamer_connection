\section{WS 1.3 - 8 - Mittlere Punktezahl - OA - BIFIE}

\begin{beispiel}[WS 1.3]{1} %PUNKTE DES BEISPIELS
				Ein Test enthält fünf Aufgaben, die jeweils nur mit einem Punkt (alles richtig) oder keinem Punkt (nicht alles richtig) bewertet werden.\\
Die nebenstehende Grafik zeigt das Ergebnis dieses Tests für
eine bestimmte Klasse.

				\begin{center}~

\psset{xunit=1.0cm,yunit=0.6cm,algebraic=true,dimen=middle,dotstyle=o,dotsize=5pt 0,linewidth=0.8pt,arrowsize=3pt 2,arrowinset=0.25}
\begin{pspicture*}(-1.6214400331880683,-1.2574818535446604)(5.900313706672241,10.401509941128166)
\multips(0,0)(0,1.0){12}{\psline[linestyle=dashed,linecap=1,dash=1.5pt 1.5pt,linewidth=0.4pt,linecolor=gray]{c-c}(0,0)(6.000313706672241,0)}
\multips(0,0)(1.0,0){8}{\psline[linestyle=dashed,linecap=1,dash=1.5pt 1.5pt,linewidth=0.4pt,linecolor=gray]{c-c}(0,0)(0,10.401509941128166)}
\psaxes[labelFontSize=\scriptstyle,xAxis=true,yAxis=true,labels=y,Dx=1,Dy=1.,ticksize=-2pt 0,subticks=2]{}(0,0)(0.,0.)(6.000313706672241,10.401509941128166)
\pspolygon[linecolor=darkgray,fillcolor=darkgray,fillstyle=solid,opacity=0.1](1.,0.)(1.,3.)(2.,3.)(2.,0.)
\pspolygon[linecolor=darkgray,fillcolor=darkgray,fillstyle=solid,opacity=0.1](2.,0.)(2.,4.)(3.,4.)(3.,0.)
\pspolygon[linecolor=darkgray,fillcolor=darkgray,fillstyle=solid,opacity=0.1](3.,0.)(3.,3.)(4.,3.)(4.,0.)
\pspolygon[linecolor=darkgray,fillcolor=darkgray,fillstyle=solid,opacity=0.1](4.,0.)(4.,6.)(5.,6.)(5.,0.)
\pspolygon[linecolor=darkgray,fillcolor=darkgray,fillstyle=solid,opacity=0.1](5.,0.)(5.,5.)(6.,5.)(6.,0.)
\psline(1.,0.)(1.,3.)
\psline(1.,3.)(2.,3.)
\psline(2.,3.)(2.,0.)
\psline(2.,0.)(1.,0.)
\psline[linecolor=darkgray](1.,0.)(1.,3.)
\psline[linecolor=darkgray](1.,3.)(2.,3.)
\psline[linecolor=darkgray](2.,3.)(2.,0.)
\psline[linecolor=darkgray](2.,0.)(1.,0.)
\psline[linecolor=darkgray](2.,0.)(2.,4.)
\psline[linecolor=darkgray](2.,4.)(3.,4.)
\psline[linecolor=darkgray](3.,4.)(3.,0.)
\psline[linecolor=darkgray](3.,0.)(2.,0.)
\psline[linecolor=darkgray](3.,0.)(3.,3.)
\psline[linecolor=darkgray](3.,3.)(4.,3.)
\psline[linecolor=darkgray](4.,3.)(4.,0.)
\psline[linecolor=darkgray](4.,0.)(3.,0.)
\psline[linecolor=darkgray](4.,0.)(4.,6.)
\psline[linecolor=darkgray](4.,6.)(5.,6.)
\psline[linecolor=darkgray](5.,6.)(5.,0.)
\psline[linecolor=darkgray](5.,0.)(4.,0.)
\psline[linecolor=darkgray](5.,0.)(5.,5.)
\psline[linecolor=darkgray](5.,5.)(6.,5.)
\psline[linecolor=darkgray](5.9,5.)(5.9,0.)
\psline[linecolor=darkgray](6.,0.)(5.,0.)
\rput[tl](1.9132863389210606,-0.8051807938360951){erzielte Punkte}
\rput[tl](-0.885038705665333,7.713644783879816){$\rotatebox{90}{\text{Anzahl der SchülerInnen}}$}
\rput[tl](0.5,-0.2){0}
\rput[tl](1.5,-0.2){1}
\rput[tl](2.5,-0.2){2}
\rput[tl](3.5,-0.2){3}
\rput[tl](4.5,-0.2){4}
\rput[tl](5.5,-0.2){5}
\end{pspicture*}
\end{center}
				
Wie viele Punkte hat die Hälfte aller SchülerInnen bei diesem Test mindestens erreicht?

Gib an, welchen Mittelwert du zur Beantwortung dieser Frage heranziehst, und berechne diesen.

\antwort{Der Median (Zentralwert) ist hier anzugeben. Er beträgt 4.}
\end{beispiel}