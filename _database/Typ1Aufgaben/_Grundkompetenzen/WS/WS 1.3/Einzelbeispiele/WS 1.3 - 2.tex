\section{WS 1.3 - 2 - Geldausgaben - OA - BIFIE}

\begin{beispiel}[WS 1.3]{1} %PUNKTE DES BEISPIELS
Karin hat das arithmetische Mittel ihrer monatlichen Ausgaben im Zeitraum Jänner bis (einschließlich) Oktober mit \EUR{25} errechnet. Im November gibt sie \EUR{35} und im Dezember \EUR{51} aus.

Berechne das arithmetische Mittel für die monatlichen Ausgaben in diesem Jahr.
\\

\antwort{$\overline{x}=\frac{25\cdot 10+35+51}{12}$\\

$\overline{x}=28$\\

Die monatlichen Ausgaben betragen durchschnittlich \EUR{28}.}
\end{beispiel}