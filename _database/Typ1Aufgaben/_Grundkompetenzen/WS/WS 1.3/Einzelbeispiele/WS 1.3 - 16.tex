\section{WS 1.3 - 16 - MAT - Arithmetisches Mittel - OA - Matura 2016/17 2. NT}

\begin{beispiel}[WS 1.3]{1} %PUNKTE DES BEISPIELS
In einer Klasse sind 25 Schüler/innen, von denen eine Schülerin als außerordentliche Schülerin geführt wird. Bei einem Test beträgt das arithmetische Mittel der von allen 25 Schülerinnen und Schülern erreichten Punkte 12,6. Das arithmetische Mittel der von den nicht als außerordentlich geführten Schülerinnen und Schülern erreichten Punkte beträgt 12,5.\leer

Berechne, wie viele Punkte die als außerordentlich geführte Schülerin bei diesem Test erreicht hat!

\antwort{$25\cdot 12,6 - 24\cdot 12,5 =15$

Die als außerordentlich geführte Schülerin hat 15 Punkte erreicht.}

\end{beispiel}