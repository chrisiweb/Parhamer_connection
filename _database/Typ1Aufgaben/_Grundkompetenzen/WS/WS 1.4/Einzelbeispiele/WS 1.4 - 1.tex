\section{WS 1.4 - 1 Eigenschaften des arithemtischen Mittels - MC - BIFIE}

\begin{beispiel}[WS 1.4]{1} %PUNKTE DES BEISPIELS
				Gegeben ist das arithmetische Mittel $\overline{x}$ von Messwerten.

Welche der folgenden Eigenschaften treffen für das arithmetische Mittel zu?
Kreuze die beiden zutreffenden Antworten an.

\multiplechoice[5]{  %Anzahl der Antwortmoeglichkeiten, Standard: 5
				L1={Das arithmetische Mittel teilt die geordnete Liste
der Messwerte immer in eine untere und eine obere
Teilliste mit jeweils gleich vielen Messwerten.},   %1. Antwortmoeglichkeit 
				L2={Das arithmetische Mittel kann durch Ausreißer
stark beeinflusst werden.},   %2. Antwortmoeglichkeit
				L3={Das arithmetische Mittel kann für alle Arten von
Daten sinnvoll berechnet werden.},   %3. Antwortmoeglichkeit
				L4={Das arithmetische Mittel ist immer gleich einem
der Messwerte.},   %4. Antwortmoeglichkeit
				L5={Multipliziert man das arithmetische Mittel mit der
Anzahl der Messwerte, so erhält man immer die
Summe aller Messwerte.},	 %5. Antwortmoeglichkeit
				L6={},	 %6. Antwortmoeglichkeit
				L7={},	 %7. Antwortmoeglichkeit
				L8={},	 %8. Antwortmoeglichkeit
				L9={},	 %9. Antwortmoeglichkeit
				%% LOESUNG: %%
				A1=2,  % 1. Antwort
				A2=5,	 % 2. Antwort
				A3=0,  % 3. Antwort
				A4=0,  % 4. Antwort
				A5=0,  % 5. Antwort
				}
				\end{beispiel}