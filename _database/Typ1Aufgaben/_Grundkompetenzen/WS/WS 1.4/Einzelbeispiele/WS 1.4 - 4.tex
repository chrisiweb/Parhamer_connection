\section{WS 1.4 - 4 Statistische Kennzahlen - MC - Matura 2013/14 1. Nebentermin}

\begin{beispiel}[WS 1.4]{1} %PUNKTE DES BEISPIELS
				Um Aussagen über die Daten einer statistischen Erhebung treffen zu können, gibt es bestimmte statistische Kennzahlen.

Welche der folgenden statistischen Kennzahlen geben Auskunft darüber, wie stark die erhobenen Daten streuen? Kreuze die beiden zutreffenden Kennzahlen an!\leer

\multiplechoice[5]{  %Anzahl der Antwortmoeglichkeiten, Standard: 5
				L1={Median},   %1. Antwortmoeglichkeit 
				L2={Spannweite},   %2. Antwortmoeglichkeit
				L3={Modus},   %3. Antwortmoeglichkeit
				L4={empirische Varianz},   %4. Antwortmoeglichkeit
				L5={arithmetisches Mittel},	 %5. Antwortmoeglichkeit
				L6={},	 %6. Antwortmoeglichkeit
				L7={},	 %7. Antwortmoeglichkeit
				L8={},	 %8. Antwortmoeglichkeit
				L9={},	 %9. Antwortmoeglichkeit
				%% LOESUNG: %%
				A1=2,  % 1. Antwort
				A2=4,	 % 2. Antwort
				A3=0,  % 3. Antwort
				A4=0,  % 4. Antwort
				A5=0,  % 5. Antwort
				}
\end{beispiel}