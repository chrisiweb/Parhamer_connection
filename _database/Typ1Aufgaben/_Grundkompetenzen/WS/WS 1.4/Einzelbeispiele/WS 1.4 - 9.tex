\section{WS 1.4 - 9 - Maximilians Meerschweinchen - ZO - MarNeu UNIVIE}

\begin{beispiel}[WS 1.4]{1}
Maximilian möchte ein weiteres Meerschweinchen haben und kauft sich in der Tierhandlung eines von 16 Meerschweinchen. \\
Zuhause wiegt er sein neues Haustier: Das Meerschweinchen ist 637 Gramm schwer.\\
Die Massen (in Gramm) der 16 Meerschweinchen bilden einen Datensatz, von dem die Tierhandlung folgende statistische Kennzahlen ermittelt hat: Minimum, Maximum, erstes Quartil, drittes Quartil, Median und arithmetisches Mittel.\\
Maximilian stellt sich nun vier Fragen, die sich alle auf die Meerschweinchen in dieser Tierhandlung beziehen und mithilfe einer dieser Kennzahlen beantwortet werden können. 

Ordne den vier Fragen die jeweils passende statistische Kennzahl (aus A bis F) zu!

\zuordnen[0.1]{
				R1={Ist mein Meerschweinchen das schwerste?},				% Response 1
				R2={Gehört mein Meerschweinchen zu den leichtesten 75$\%?$},				% Response 2
				R3={Ist mein Meerschweinchen eines der 8 schwersten?},				% Response 3
				R4={Ist mein Meerschweinchen eines der 4 leichtesten?},				% Response 4
				%% Moegliche Zuordnungen: %%
				A={Minimum}, 				%Moeglichkeit A  
				B={arithmetisches\\ Mittel}, 				%Moeglichkeit B  
				C={drittes Quartil}, 				%Moeglichkeit C  
				D={Median}, 				%Moeglichkeit D  
				E={Maximum}, 				%Moeglichkeit E  
				F={erstes Quartil}, 				%Moeglichkeit F  
				%% LOESUNG: %%
				A1={E},				% 1. richtige Zuordnung
				A2={C},				% 2. richtige Zuordnung
				A3={D},				% 3. richtige Zuordnung
				A4={F},				% 4. richtige Zuordnung
				}
\end{beispiel}