\section{WS 1.4 - 6 - Statistische Kennzahlen im Kontext des Würfelns - MC - FraKol UNIVIE}

\begin{beispiel}[WS 1.4]{1}
Herr Mustermann führt ein Experiment 10 Mal durch. Er würfelt jeweils so oft bis der Würfel die Augenzahl 6 zeigt.\\
Gegeben ist der Datensatz mit der Anzahl der Würfe, die er pro Durchgang dafür gebraucht hat.\\
Datensatz: 15, 9, 1, 21, 8, 6, 7, 2, 4, 12

Kreuze die richtigen Aussagen an!
\multiplechoice[5]{  %Anzahl der Antwortmoeglichkeiten, Standard: 5
				L1={Der Median und das arithmetische Mittel haben denselben Wert.},
				L2={Die Spannweite beträgt 21.},   %1.%2. 
				L3={Das arithmetische Mittel dieses Datensatzes beträgt 8,5.},
				L4={Das erste Quartil in diesem Datensatz ist 3, das zweite Quartil ist 7,5 und das dritte Quartil ist 12.},   %4.
				L5={Der Median dieses Datensatzes liegt bei 7,5.},	 %5.
				L6={},	 %6. 
				L7={},	 %7. 
				L8={},	 %8. 
				L9={},	 %9. 
				%% LOESUNG: %%
				A1=3,  % 1. 
				A2=5,	 % 2. 
				A3=0,  % 3. 
				A4=0,  % 4. 
				A5=0,  % 5. 
				}
\end{beispiel}