\section{WS 1.4 - 5 - MAT - Lawinengefahr - OA - Matura-HT-18/19}

\begin{beispiel}[WS 1.4]{1}
In den Wintermonaten wird täglich vom Lawinenwarndienst der sogenannte Lawinenlagebericht
veröffentlicht. Dieser enthält unter anderem eine Einschätzung der Lawinengefahr entsprechend
den fünf Gefahrenstufen.\leer

In einer bestimmten Region wurden im Winter 2013/14 Aufzeichnungen über die Gefahrenstufen
geführt. Die Aufzeichnungen listen in einer Datenliste alle Tage auf, an denen eine der Gefahrenstufen 1 bis 4 galt. (Für die Gefahrenstufe 5 gibt es in dieser Datenliste keinen Eintrag, da diese Gefahrenstufe im betrachteten Zeitraum nicht auftrat.)\\
Die nachstehende Abbildung zeigt den relativen Anteil der Tage mit einer entsprechenden Gefahrenstufe.\leer

\begin{center}
\begin{tikzpicture}
\pie[color={black!30 ,black!10 , black!50, black!20}, %Farbe
text=pin, %Format: inside,pin, legend
rotate=-10,
before number=\phantom,after number=,radius=2]
{27/Gefahrenstufe 1, 2/Gefahrenstufe 4, 15/Gefahrenstufe 3,56/Gefahrenstufe 2} %Werte
\end{tikzpicture}
\end{center}

Begründe warum die Gefahrenstufe 2 der Median der Datenliste (die der obigen Abbildung zugrunde liegt) sein muss!

\antwort{Der Zentriwinkel des Sektors für "`Gefahrenstufe 2"' ist größer als 180\degre. Das bedeutet, dass mehr als 50\,\% der Einträge in der zugrunde liegenden Datenliste Tage mit "`Gefahrenstufe 2"' sind. Daher beträgt der Median 2. (Andere richtige Begründungen (z.B. grafische Begründungen) sind ebenfalls als richtig zu werten.)}
\end{beispiel}