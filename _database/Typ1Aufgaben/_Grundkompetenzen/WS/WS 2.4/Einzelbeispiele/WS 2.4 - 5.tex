\section{WS 2.4 - 5 Binomialkoeffizient - MC - Matura 2013/14 1. Nebentermin}

\begin{beispiel}[WS 2.4]{1} %PUNKTE DES BEISPIELS
				Betrachtet wird der Binomialkoeffizient $\binom{6}{2}$.
				
				Kreuze die beiden Aufgabenstellungen an, die mit der Rechnung $\binom{6}{2}=15$ gelöst werden können!\leer
				
				\multiplechoice[5]{  %Anzahl der Antwortmoeglichkeiten, Standard: 5
								L1={Gegeben sind sechs verschiedene Punkte einer Ebene, von denen nie mehr als zwei auf einer Geraden liegen. Wie viele Möglichkeiten gibt es, zwei Punkte auszuwählen, um jeweils eine Gerade durchzulegen?},   %1. Antwortmoeglichkeit 
								L2={An einem Wettrennen nehmen sechs Personen teil. Wie viele Möglichkeiten gibt es für den Zieleinlauf, wenn nur die ersten beiden Plätze relevant sind?},   %2. Antwortmoeglichkeit
								L3={Von sechs Kugeln sind vier rot und zwei blau. Sie unterscheiden sich nur durch ihre Farbe. Wie viele Möglichkeiten gibt es, die Kugeln in einer Reihe anzuordnen?},   %3. Antwortmoeglichkeit
								L4={Sechs Mädchen einer Schulklasse kandidieren für das Amt der Klassensprecherin. Die Siegerin der Wahl soll Klassensprecherin werden, die Zweitplatzierte deren Stellvertreterin. Wie viele Möglichkeiten gibt es für die Vergabe der beiden Ämter?},   %4. Antwortmoeglichkeit
								L5={Wie viele sechsstellige Zahlen können aus den Ziffern 6 und 2 gebildet werden?},	 %5. Antwortmoeglichkeit
								L6={},	 %6. Antwortmoeglichkeit
								L7={},	 %7. Antwortmoeglichkeit
								L8={},	 %8. Antwortmoeglichkeit
								L9={},	 %9. Antwortmoeglichkeit
								%% LOESUNG: %%
								A1=1,  % 1. Antwort
								A2=3,	 % 2. Antwort
								A3=0,  % 3. Antwort
								A4=0,  % 4. Antwort
								A5=0,  % 5. Antwort
								}
\end{beispiel}