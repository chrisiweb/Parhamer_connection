\section{WS 2.4 - 6 - MAT - Jugendgruppe - LT - Matura HT 2016/17}

\begin{beispiel}[WS 2.4]{1} %PUNKTE DES BEISPIELS
	Eine Jugendgruppe besteht aus 21 Jugendlichen. Für ein Spiel sollen Teams gebildet werden.
	
\lueckentext[0.3]{
					text={Der Bionmialkoeffizient $\Vek{21}{3}{}$ gibt an, \gap; sein Wert beträgt \gap.}, 	%Lueckentext Luecke=\gap
					L1={wie viele der 21 Jugendlichen in einem Team sind, wenn
man drei gleich große Teams bildet}, 		%1.Moeglichkeit links  
					L2={wie viele verschiedene Möglichkeiten es gibt, aus den
21 Jugendlichen ein Dreierteam auszuwählen}, 		%2.Moeglichkeit links
					L3={auf wie viele Arten drei unterschiedliche Aufgaben auf drei
Mitglieder der Jugendgruppe aufgeteilt werden können}, 		%3.Moeglichkeit links
					R1={7}, 		%1.Moeglichkeit rechts 
					R2={1\,330}, 		%2.Moeglichkeit rechts
					R3={7\,980}, 		%3.Moeglichkeit rechts
					%% LOESUNG: %%
					A1=2,   % Antwort links
					A2=2		% Antwort rechts 
					}
\end{beispiel}