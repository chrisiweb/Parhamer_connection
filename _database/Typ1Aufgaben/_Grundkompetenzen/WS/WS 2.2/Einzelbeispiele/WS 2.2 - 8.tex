\section{WS 2.2 - 8 - MAT - Schätzwert für eine Wahrscheinlichkeit - OA - Matura 1. NT 2016/17}

\begin{beispiel}[WS 2.2]{1} %PUNKTE DES BEISPIELS
In einer Fabrik wird mithilfe einer Maschine ein Produkt erzeugt, von dem jeweils 100 Stück in eine Packung kommen.

Im Anschluss an eine Neueinstellung der Maschine werden drei Packungen erzeugt. Diese Packungen werden kontrolliert un es wird die jeweilige Anzahl darin enthaltener defekter Stücke ermittelt. Die Ergebnisse dieser Kontrollen sind in der nachstehenden Tabelle zusammengefasst.

\begin{center}
	\begin{tabular}{|l|l|}\hline
	\cellcolor[gray]{0.9}in der ersten Packung&6 defekte Stücke\\ \hline
	\cellcolor[gray]{0.9}in der zweiten Packung&3 defekte Stücke\\ \hline
	\cellcolor[gray]{0.9}in der dritten Packung&4 defekte Stücke\\ \hline	
	\end{tabular}
\end{center}

Die Fabriksleitung benötigt einen auf dem vorliegenden Datenmaterial besierenden Schätzwert für die Wahrscheinlichkeit $\rho$, dass ein von der neu eingestellten Maschine erzeugte Stück fehlerhaft ist.

Gib einen möglichst zuverlässigen Schätzwert für die Wahrscheinlichkeit $\rho$ an, dass ein von der neu eingestellten Maschine erzeugtes Stück fehlerhaft ist!

$\rho=$ \antwort[\rule{3cm}{0.3pt}]{$\rho=\frac{13}{300}=0,04\dot{3}$}

\antwort{Toleranzintervall: $[0,04;0,05]$ bzw. $[4\,\%;5\,\%]$}
\end{beispiel}