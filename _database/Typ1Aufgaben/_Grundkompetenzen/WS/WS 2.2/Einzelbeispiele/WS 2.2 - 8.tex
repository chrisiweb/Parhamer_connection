\section{WS 2.2 - 8 Sch�tzwert f�r eine Wahrscheinlichkeit - OA - Matura NT 1 16/17}

\begin{beispiel}[WS 2.2]{1} %PUNKTE DES BEISPIELS
In einer Fabrik wird mithilfe einer Maschine ein Produkt erzeugt, von dem jeweils 100 St�ck in eine Packung kommen.

Im Anschluss an eine Neueinstellung der Maschine werden drei Packungen erzeugt. Diese Packungen werden kontrolliert un es wird die jeweilige Anzahl darin enthaltener defekter St�cke ermittelt. Die Ergebnisse dieser Kontrollen sind in der nachstehenden Tabelle zusammengefasst.

\begin{center}
	\begin{tabular}{|l|l|}\hline
	\cellcolor[gray]{0.9}in der ersten Packung&6 defekte St�cke\\ \hline
	\cellcolor[gray]{0.9}in der zweiten Packung&3 defekte St�cke\\ \hline
	\cellcolor[gray]{0.9}in der dritten Packung&4 defekte St�cke\\ \hline	
	\end{tabular}
\end{center}

Die Fabriksleitung ben�tigt einen auf dem vorliegenden Datenmaterial besierenden Sch�tzwert f�r die Wahrscheinlichkeit $\rho$, dass ein von der neu eingestellten Maschine erzeugte St�ck fehlerhaft ist.

Gib einen m�glichst zuverl�ssigen Sch�tzwert f�r die Wahrscheinlichkeit $\rho$ an, dass ein von der neu eingestellten Maschine erzeugtes St�ck fehlerhaft ist!

$\rho=$ \antwort[\rule{3cm}{0.3pt}]{$\rho=\frac{13}{300}=0,04\dot{3}$}

\antwort{Toleranzintervall: $[0,04;0,05]$ bzw. $[4\,\%;5\,\%]$}
\end{beispiel}