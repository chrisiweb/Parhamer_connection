\section{WS 2.2 - 9 - MAT - Grippe in Österreich - OA - Matura 2018/19 2. NT}

\begin{beispiel}[WS 2.2]{1}
Die Medizinische Universität Wien hat die Daten einer Grippe-Virusinfektion für eine bestimmte Woche veröffentlicht. Dazu wurden Blutproben von Personen, die in dieser Woche an Grippe erkrankt waren untersucht. Von den 1\,954 untersuchten Blutproben waren 547 Blutproben mit dem Virus $A(H1N1)$, 117 Blutproben mit dem Virus $A(H3N2)$ und die restlichen Blutproben mit dem Virus \textit{Influenza} $B$ infiziert.

Verwende die obigen Häufigkeitsangaben als Wahrscheinlichkeiten und bestimme unter dieser Voraussetzung die Wahrscheinlichkeit dafür, dass eine zufällig ausgewählte an Grippe erkrankte Person mit dem Virus \textit{Influenza} $B$ infiziert ist.

\antwort{$\dfrac{1\,290}{1\,954}=0,66018\ldots\approx 0,6602$

Toleranzintervall: $[0,660;0,661]$}
\end{beispiel}