\section{WS 2.2 - 2 Reißnagel - OA - BIFIE}

\begin{beispiel}[WS 2.2]{1}
Wenn man einen Reißnagel fallen lässt, bleibt dieser auf eine der beiden dargestellten Arten liegen.

\begin{center}
\resizebox{0.2\linewidth}{!}{
\psset{xunit=1.0cm,yunit=1.0cm,algebraic=true,dimen=middle,dotstyle=o,dotsize=5pt 0,linewidth=0.8pt,arrowsize=3pt 2,arrowinset=0.25}
\begin{pspicture*}(1.6588319424230604,-3.4811881726732845)(6.997577612746404,-1.0309429583084695)
\psline[linewidth=2.pt](2.,-3.)(4.,-3.)
\psline[linewidth=2.pt](3.,-3.)(3.,-2.)
\psline[linewidth=2.pt](5.,-3.)(6.459718907154124,-1.528960278301318)
\psline[linewidth=2.pt](5.72551133558688,-2.2688619788643476)(6.519480985553267,-3.023012238279864)
\end{pspicture*}}
\end{center}

Beschreibe eine Methode, wie man die Wahrscheinlichkeit für die beiden Fälle herausfinden kann.

\antwort{Der Reißnagel wird eine bestimmte Anzahl (n-mal) fallen gelassen und man notiert, wie oft er auf welche Art zu liegen kommt.
Wenn er $k_1$ -mal bzw. $k_2$-mal auf eine bestimmte Art zu liegen kommt, dann sind die relativen Häufigkeiten  $\frac{k_1}{n}$ und $\frac{k_2}{n}$ Näherungswerte für die gesuchten Wahrscheinlichkeiten. Je öfter der Reißnagel fallen gelassen wird, desto zuverlässiger ist der ermittelte Näherungswert.\leer

Lösungsschlüssel: Die Aufgabe gilt bei einer sinngemäß richtigen Erklärung als korrekt gelöst.}

\end{beispiel}