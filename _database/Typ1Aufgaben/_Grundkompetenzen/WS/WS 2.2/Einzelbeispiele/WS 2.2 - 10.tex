\section{WS 2.2 - 10 - WS 2.2.-Gesetz der großen Zahlen-MC-MirDar - MC - MirDar UNIVIE}

\begin{beispiel}[WS 2.2]{1}
Ein fairer sechsseitiger Würfel wird 100.000 Mal geworfen und es werden die Anzahlen der dabei vorkommenden Augenzahlen 1, 2, 3, 4, 5, 6 notiert.

Kreuze das Diagramm an, welches die relative Häufigkeitsverteilung des Experiments am ehesten darstellt. 

\langmultiplechoice[6]{  %Anzahl der Antwortmoeglichkeiten, Standard: 5
				L1={
\psset{xunit=0.5cm,yunit=2.0cm,algebraic=true,dimen=middle,dotstyle=o,dotsize=5pt 0,linewidth=1.6pt,arrowsize=3pt 2,arrowinset=0.25}
\begin{pspicture*}(-1.4,-0.25)(7.,1.05)
\multips(0,0)(0,0.2){8}{\psline[linestyle=dashed,linecap=1,dash=1.5pt 1.5pt,linewidth=0.4pt,linecolor=gray]{c-c}(0,0)(7.,0)}
\multips(0,0)(1.0,0){9}{\psline[linestyle=dashed,linecap=1,dash=1.5pt 1.5pt,linewidth=0.4pt,linecolor=gray]{c-c}(0,0)(0,1.05)}
\begin{scriptsize}
\psaxes[comma,xAxis=true,yAxis=true,Dx=1.,Dy=0.2,ticksize=-2pt 0,subticks=0]{->}(0,0)(0.,0.)(7.,1.05)[,140] [rel. H.,-40]
\end{scriptsize}
\psframe[linewidth=0.8pt,linecolor=gray,fillcolor=gray,fillstyle=solid,opacity=0.6](0.5,0)(1.5,0.03)
\psframe[linewidth=0.8pt,linecolor=gray,fillcolor=gray,fillstyle=solid,opacity=0.6](1.5,0)(2.5,0.)
\psframe[linewidth=0.8pt,linecolor=gray,fillcolor=gray,fillstyle=solid,opacity=0.6](2.5,0)(3.5,0.15)
\psframe[linewidth=0.8pt,linecolor=gray,fillcolor=gray,fillstyle=solid,opacity=0.6](3.5,0)(4.5,0.24)
\psframe[linewidth=0.8pt,linecolor=gray,fillcolor=gray,fillstyle=solid,opacity=0.6](4.5,0)(5.5,0.5)
\psframe[linewidth=0.8pt,linecolor=gray,fillcolor=gray,fillstyle=solid,opacity=0.6](5.5,0)(6.5,0.08)
\end{pspicture*}},   %1. Antwortmoeglichkeit 
				L2={\psset{xunit=0.5cm,yunit=2.0cm,algebraic=true,dimen=middle,dotstyle=o,dotsize=5pt 0,linewidth=1.6pt,arrowsize=3pt 2,arrowinset=0.25}
\begin{pspicture*}(-1.4,-0.25)(7.,1.05)
\multips(0,0)(0,0.2){8}{\psline[linestyle=dashed,linecap=1,dash=1.5pt 1.5pt,linewidth=0.4pt,linecolor=gray]{c-c}(0,0)(7.,0)}
\multips(0,0)(1.0,0){9}{\psline[linestyle=dashed,linecap=1,dash=1.5pt 1.5pt,linewidth=0.4pt,linecolor=gray]{c-c}(0,0)(0,1.05)}
\begin{scriptsize}
\psaxes[comma,xAxis=true,yAxis=true,Dx=1.,Dy=0.2,ticksize=-2pt 0,subticks=0]{->}(0,0)(0.,0.)(7.,1.05)[,140] [rel. H.,-40]
\end{scriptsize}
\psframe[linewidth=0.8pt,linecolor=gray,fillcolor=gray,fillstyle=solid,opacity=0.6](0.5,0)(1.5,0.1)
\psframe[linewidth=0.8pt,linecolor=gray,fillcolor=gray,fillstyle=solid,opacity=0.6](1.5,0)(2.5,0.167)
\psframe[linewidth=0.8pt,linecolor=gray,fillcolor=gray,fillstyle=solid,opacity=0.6](2.5,0)(3.5,0.233)
\psframe[linewidth=0.8pt,linecolor=gray,fillcolor=gray,fillstyle=solid,opacity=0.6](3.5,0)(4.5,0.233)
\psframe[linewidth=0.8pt,linecolor=gray,fillcolor=gray,fillstyle=solid,opacity=0.6](4.5,0)(5.5,0.167)
\psframe[linewidth=0.8pt,linecolor=gray,fillcolor=gray,fillstyle=solid,opacity=0.6](5.5,0)(6.5,0.1)
\end{pspicture*}},   %2. Antwortmoeglichkeit
				L3={\psset{xunit=0.5cm,yunit=2.0cm,algebraic=true,dimen=middle,dotstyle=o,dotsize=5pt 0,linewidth=1.6pt,arrowsize=3pt 2,arrowinset=0.25}
\begin{pspicture*}(-1.4,-0.25)(7.,1.05)
\multips(0,0)(0,0.2){8}{\psline[linestyle=dashed,linecap=1,dash=1.5pt 1.5pt,linewidth=0.4pt,linecolor=gray]{c-c}(0,0)(7.,0)}
\multips(0,0)(1.0,0){9}{\psline[linestyle=dashed,linecap=1,dash=1.5pt 1.5pt,linewidth=0.4pt,linecolor=gray]{c-c}(0,0)(0,1.05)}
\begin{scriptsize}
\psaxes[comma,xAxis=true,yAxis=true,Dx=1.,Dy=0.2,ticksize=-2pt 0,subticks=0]{->}(0,0)(0.,0.)(7.,1.05)[,140] [rel. H.,-40]
\end{scriptsize}
\psframe[linewidth=0.8pt,linecolor=gray,fillcolor=gray,fillstyle=solid,opacity=0.6](0.5,0)(1.5,0.21)
\psframe[linewidth=0.8pt,linecolor=gray,fillcolor=gray,fillstyle=solid,opacity=0.6](1.5,0)(2.5,0.4)
\psframe[linewidth=0.8pt,linecolor=gray,fillcolor=gray,fillstyle=solid,opacity=0.6](2.5,0)(3.5,0.3)
\psframe[linewidth=0.8pt,linecolor=gray,fillcolor=gray,fillstyle=solid,opacity=0.6](3.5,0)(4.5,0.04)
\psframe[linewidth=0.8pt,linecolor=gray,fillcolor=gray,fillstyle=solid,opacity=0.6](4.5,0)(5.5,0.03)
\psframe[linewidth=0.8pt,linecolor=gray,fillcolor=gray,fillstyle=solid,opacity=0.6](5.5,0)(6.5,0.02)
\end{pspicture*}},   %3. Antwortmoeglichkeit
				L4={\psset{xunit=0.5cm,yunit=2.0cm,algebraic=true,dimen=middle,dotstyle=o,dotsize=5pt 0,linewidth=1.6pt,arrowsize=3pt 2,arrowinset=0.25}
\begin{pspicture*}(-1.4,-0.25)(7.,1.05)
\multips(0,0)(0,0.2){8}{\psline[linestyle=dashed,linecap=1,dash=1.5pt 1.5pt,linewidth=0.4pt,linecolor=gray]{c-c}(0,0)(7.,0)}
\multips(0,0)(1.0,0){9}{\psline[linestyle=dashed,linecap=1,dash=1.5pt 1.5pt,linewidth=0.4pt,linecolor=gray]{c-c}(0,0)(0,1.05)}
\begin{scriptsize}
\psaxes[comma,xAxis=true,yAxis=true,Dx=1.,Dy=0.2,ticksize=-2pt 0,subticks=0]{->}(0,0)(0.,0.)(7.,1.05)[,140] [rel. H.,-40]
\end{scriptsize}
\psframe[linewidth=0.8pt,linecolor=gray,fillcolor=gray,fillstyle=solid,opacity=0.6](0.5,0)(1.5,0.16666)
\psframe[linewidth=0.8pt,linecolor=gray,fillcolor=gray,fillstyle=solid,opacity=0.6](1.5,0)(2.5,0.16)
\psframe[linewidth=0.8pt,linecolor=gray,fillcolor=gray,fillstyle=solid,opacity=0.6](2.5,0)(3.5,0.166)
\psframe[linewidth=0.8pt,linecolor=gray,fillcolor=gray,fillstyle=solid,opacity=0.6](3.5,0)(4.5,0.1569)
\psframe[linewidth=0.8pt,linecolor=gray,fillcolor=gray,fillstyle=solid,opacity=0.6](4.5,0)(5.5,0.16666)
\psframe[linewidth=0.8pt,linecolor=gray,fillcolor=gray,fillstyle=solid,opacity=0.6](5.5,0)(6.5,0.18378)
\end{pspicture*}},   %4. Antwortmoeglichkeit
				L5={\psset{xunit=0.5cm,yunit=2.0cm,algebraic=true,dimen=middle,dotstyle=o,dotsize=5pt 0,linewidth=1.6pt,arrowsize=3pt 2,arrowinset=0.25}
\begin{pspicture*}(-1.4,-0.25)(7.,1.05)
\multips(0,0)(0,0.2){8}{\psline[linestyle=dashed,linecap=1,dash=1.5pt 1.5pt,linewidth=0.4pt,linecolor=gray]{c-c}(0,0)(7.,0)}
\multips(0,0)(1.0,0){9}{\psline[linestyle=dashed,linecap=1,dash=1.5pt 1.5pt,linewidth=0.4pt,linecolor=gray]{c-c}(0,0)(0,1.05)}
\begin{scriptsize}
\psaxes[comma,xAxis=true,yAxis=true,Dx=1.,Dy=0.2,ticksize=-2pt 0,subticks=0]{->}(0,0)(0.,0.)(7.,1.05)[,140] [rel. H.,-40]
\end{scriptsize}
\psframe[linewidth=0.8pt,linecolor=gray,fillcolor=gray,fillstyle=solid,opacity=0.6](0.5,0)(1.5,0.2)
\psframe[linewidth=0.8pt,linecolor=gray,fillcolor=gray,fillstyle=solid,opacity=0.6](1.5,0)(2.5,0.05)
\psframe[linewidth=0.8pt,linecolor=gray,fillcolor=gray,fillstyle=solid,opacity=0.6](2.5,0)(3.5,0.14)
\psframe[linewidth=0.8pt,linecolor=gray,fillcolor=gray,fillstyle=solid,opacity=0.6](3.5,0)(4.5,0.002)
\psframe[linewidth=0.8pt,linecolor=gray,fillcolor=gray,fillstyle=solid,opacity=0.6](4.5,0)(5.5,0.2405)
\psframe[linewidth=0.8pt,linecolor=gray,fillcolor=gray,fillstyle=solid,opacity=0.6](5.5,0)(6.5,0.3675)
\end{pspicture*}},	 %5. Antwortmoeglichkeit
				L6={\psset{xunit=0.5cm,yunit=2.0cm,algebraic=true,dimen=middle,dotstyle=o,dotsize=5pt 0,linewidth=1.6pt,arrowsize=3pt 2,arrowinset=0.25}
\begin{pspicture*}(-1.4,-0.25)(7.,1.05)
\multips(0,0)(0,0.2){8}{\psline[linestyle=dashed,linecap=1,dash=1.5pt 1.5pt,linewidth=0.4pt,linecolor=gray]{c-c}(0,0)(7.,0)}
\multips(0,0)(1.0,0){9}{\psline[linestyle=dashed,linecap=1,dash=1.5pt 1.5pt,linewidth=0.4pt,linecolor=gray]{c-c}(0,0)(0,1.05)}
\begin{scriptsize}
\psaxes[comma,xAxis=true,yAxis=true,Dx=1.,Dy=0.2,ticksize=-2pt 0,subticks=0]{->}(0,0)(0.,0.)(7.,1.05)[,140] [rel. H.,-40]
\end{scriptsize}
\psframe[linewidth=0.8pt,linecolor=gray,fillcolor=gray,fillstyle=solid,opacity=0.6](0.5,0)(1.5,0.12)
\psframe[linewidth=0.8pt,linecolor=gray,fillcolor=gray,fillstyle=solid,opacity=0.6](1.5,0)(2.5,0.01)
\psframe[linewidth=0.8pt,linecolor=gray,fillcolor=gray,fillstyle=solid,opacity=0.6](2.5,0)(3.5,0.001)
\psframe[linewidth=0.8pt,linecolor=gray,fillcolor=gray,fillstyle=solid,opacity=0.6](3.5,0)(4.5,0.0015)
\psframe[linewidth=0.8pt,linecolor=gray,fillcolor=gray,fillstyle=solid,opacity=0.6](4.5,0)(5.5,0.6)
\psframe[linewidth=0.8pt,linecolor=gray,fillcolor=gray,fillstyle=solid,opacity=0.6](5.5,0)(6.5,0.2675)
\end{pspicture*}},	 %6. Antwortmoeglichkeit
				L7={},	 %7. Antwortmoeglichkeit
				L8={},	 %8. Antwortmoeglichkeit
				L9={},	 %9. Antwortmoeglichkeit
				%% LOESUNG: %%
				A1=4,  % 1. Antwort
				A2=0,	 % 2. Antwort
				A3=0,  % 3. Antwort
				A4=0,  % 4. Antwort
				A5=0,  % 5. Antwort
				A6=0,
				}
\end{beispiel}