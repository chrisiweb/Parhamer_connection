\section{WS 2.2 - 7 - MAT - Online-Glücksspiel - MC - Matura 2. NT 2015/16}

\begin{beispiel}{1} %PUNKTE DES BEISPIELS
Ein Mann spielt über einen längeren Zeitraum regelmäßig dasselbe Online-Glücksspiel mit konstanter Gewinnwahrscheinlichkeit. Von 768 Spielen gewinnt er 162. \leer

Mit welcher ungefähren Wahrscheinlichkeit wird er das nächste Spiel gewinnen?

Kreuze den zutreffenden Schätzwert für diese Wahrscheinlichkeit an. \leer

\multiplechoice[6]{  %Anzahl der Antwortmoeglichkeiten, Standard: 5
				L1={$0,162\,\%$},   %1. Antwortmoeglichkeit 
				L2={$4,74\,\%$},   %2. Antwortmoeglichkeit
				L3={$16,2\,\%$},   %3. Antwortmoeglichkeit
				L4={$21,1\,\%$},   %4. Antwortmoeglichkeit
				L5={$7,68\,\%$},	 %5. Antwortmoeglichkeit
				L6={$76,6\,\%$},	 %6. Antwortmoeglichkeit
				L7={},	 %7. Antwortmoeglichkeit
				L8={},	 %8. Antwortmoeglichkeit
				L9={},	 %9. Antwortmoeglichkeit
				%% LOESUNG: %%
				A1=4,  % 1. Antwort
				A2=0,	 % 2. Antwort
				A3=0,  % 3. Antwort
				A4=0,  % 4. Antwort
				A5=0,  % 5. Antwort
				}
\end{beispiel}