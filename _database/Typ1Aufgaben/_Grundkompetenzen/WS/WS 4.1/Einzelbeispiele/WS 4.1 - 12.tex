\section{WS 4.1 - 12 - MAT - Sicherheit eines Konfidenzintervalls - OA - Matura 2. NT 2016/17}

\begin{beispiel}[WS 4.1]{1} %PUNKTE DES BEISPIELS
Die Abfüllanlagen eines Betriebes müssen in bestimmten Zeitabständen überprüft und eventuell neu eingestellt werden.

Nach der Einstellung einer Abfüllanlage sind von 1\,000 überprüften Packungen 30 nicht ordnungsgemäß befüllt. Für den unbekannten relativen Anteil $p$ der nicht ordnungsgemäß befüllten Packungen wird vom Betrieb das symmetrische Konfidenzintervall $[0,02; 0,04]$ angegeben.

Ermittle unter Verwendung einer die Binomialverteilung approximierenden Normalverteilung die Sicherheit dieses Konfidenzintervalls!\leer

\antwort{Mögliche Vorgehensweise:

$n=1\,000$, $h=\frac{30}{1\,000}=0,03$ Intervallbreite des Konfidenzintervalls\,=\,0,02

aus $z\cdot\sqrt{\frac{h\cdot(1-h)}{n}}=0,01$ folgt: $z\approx 1,85$ mit $\Phi(1,85)\approx 0,9678$

$\Rightarrow \gamma=2\cdot\Phi(1,85)-1\approx 0,9356$

Somit liegt die Sicherheit dieses Konfidenzintervalls bei ca. 93,56\,\%.

Toleranzintervall: $[93\,\%;\,94\,\%]$}
\end{beispiel}