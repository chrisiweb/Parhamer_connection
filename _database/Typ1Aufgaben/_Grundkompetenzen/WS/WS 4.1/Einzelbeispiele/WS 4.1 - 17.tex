\section{WS 4.1 - 17 - MAT - Sonntagsfrage - OA - Matura 1.NT 2018/19}

\begin{beispiel}[WS 4.1]{1}
\textit{Sonntagsumfrage} nennt man in der Meinungsforschung die Frage "`Welche Partei würden Sie wählen, wenn am kommenden Sonntag Wahlen wären?"'. Bei einer solchen Sonntagsumfrage, bei der die Parteien $A$ und $B$ zur Auswahl standen, gaben 234 von 1\,000 befragten Personen an, Partei $A$ zu wählen. Bei der darauffolgenden Wahl lag der tatsächliche Anteil der Personen, die die Partei $A$ gewählt haben bei $29,5\,\%$.

Ermittle auf Basis dieses Umfrageergebnisses ein symmetrisches $95-\%-$Konfidenzinervall für den (unbekannten) Stimmenanteil der Partei $A$ und gib an, ob der tatsächliche Anteil in diesem Intervall enthalten ist.

\antwort{mögliche Vorgehensweise:\\
$h\ldots$ relative Häufigkeit\\
$h=0,234$

$0,234\pm 1,96\cdot\sqrt{\dfrac{0,234\cdot (1-0,234)}{1\,000}}\approx 0,234\pm 0,026\Rightarrow [0,208; 0,260]$

$0,295\notin[0,208; 0,260]$

Toleranzintervall für den unteren Wert: $[0,2; 0,22]$
Toleranzintervall für den oberen Wert: $[0,25; 0,27]$
Die Aufgabe ist auch dann als richtig gelöst zu werten, wenn bei korrektem Ansatz das Ergebnis aufgrund eines Rechenfehlers nicht richtig ist.}
\end{beispiel}