\section{WS 4.1 - 10 Wahlprognose - MC - Matura 2016/17 - Haupttermin}

\begin{beispiel}[WS 4.1]{1} %PUNKTE DES BEISPIELS
Um den Stimmenanteil einer bestimmten Partei $A$ in der Grundgesamtheit zu schätzen, wird eine
zufällig aus allen Wahlberechtigten ausgewählte Personengruppe befragt.\leer

Die Umfrage ergibt für den Stimmenanteil ein 95-\%-Konfidenzintervall von $[9,8\,\%; 12,2\,\%]$. \leer

Welche der folgenden Aussagen sind in diesem Zusammenhang auf jeden Fall korrekt?
Kreuze die beiden zutreffenden Aussagen an!\leer

\multiplechoice[5]{  %Anzahl der Antwortmoeglichkeiten, Standard: 5
				L1={Die Wahrscheinlichkeit, dass eine zufällig ausgewählte wahlberechtigte
Person die Partei $A$ wählt, liegt sicher zwischen 9,8\,\% und 12,2\,\%.},   %1. Antwortmoeglichkeit 
				L2={Ein anhand der erhobenen Daten ermitteltes 90-\%-Konfidenzintervall
hätte eine geringere Intervallbreite.},   %2. Antwortmoeglichkeit
				L3={Unter der Voraussetzung, dass der Anteil der Partei-$A$-Wähler/innen in
der Stichprobe gleich bleibt, würde eine Vergrößerung der Stichprobe
zu einer Verkleinerung des 95-\%-Konfidenzintervalls führen.},   %3. Antwortmoeglichkeit
				L4={95 von 100 Personen geben an, die Partei $A$ mit einer Wahrscheinlichkeit
von 11\,\% zu wählen.},   %4. Antwortmoeglichkeit
				L5={Die Wahrscheinlichkeit, dass die Partei $A$ einen Stimmenanteil von
mehr als 12,2\,\% erhält, beträgt 5\,\%.},	 %5. Antwortmoeglichkeit
				L6={},	 %6. Antwortmoeglichkeit
				L7={},	 %7. Antwortmoeglichkeit
				L8={},	 %8. Antwortmoeglichkeit
				L9={},	 %9. Antwortmoeglichkeit
				%% LOESUNG: %%
				A1=2,  % 1. Antwort
				A2=3,	 % 2. Antwort
				A3=0,  % 3. Antwort
				A4=0,  % 4. Antwort
				A5=0,  % 5. Antwort
				}
\end{beispiel}