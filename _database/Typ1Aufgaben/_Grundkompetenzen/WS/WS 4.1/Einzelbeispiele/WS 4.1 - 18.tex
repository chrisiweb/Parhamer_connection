\section{WS 4.1 - 18 - MAT - Frühstück - OA - Matura 2018/19 2. NT}

\begin{beispiel}[WS 4.1]{1}
Im Rahmen einer Studie gaben 252 von 450 Jugendlichen eines Bundeslandes an, dass sie immer frühstücken, bevor sie in die Schule gehen. Der Anteil dieser Jugendlichen wird mit $h$ bezeichnet.

Der Anteil aller Jugendlichen dieses Bundeslandes, die immer frühstücken, bevor sie in die Schule gehen, wird mit $p$ bezeichnet.

Gib auf Basis dieser Studie für $p$ ein um $h$ symmetrisches $95\,\%$-Konfidenzintervall an.

\antwort{$h=\dfrac{252}{450}=0,56$

$0,56\pm 1,96\cdot\sqrt{\dfrac{0,56\cdot (1-0,56)}{450}}=0,56\pm 0,0458\ldots \Rightarrow [0,514; 0,606]$

Toleranzintervall für Untergrenze: $[0,51; 0,52]$\\
Toleranzintervall für Obergrenze: $[0,60; 0,61]$}
\end{beispiel}