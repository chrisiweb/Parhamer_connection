\section{WS 4.1 - 11 Konfidenzintervall - OA - Matura NT 1 16/17}

\begin{beispiel}[WS 4.1]{1} %PUNKTE DES BEISPIELS
Für eine Wahlprognose wird aus allen Wahlberechtigten eine Zufallsstichprobe ausgewählt. Von 400 befragten Personen geben 80 an, die Partei $Y$ zu wählen.

Gib ein symmetrisches $95-\%-$Konfidenzintervall für den Stimmenanteil der Partei $Y$ in der Grundgesamtheit an!

\antwort{$n=400$, $h=0,2$

$0,2\pm 1,96\cdot\sqrt{\frac{0,2\cdot (1-0,2)}{400}}=0,2\pm 0,0392\Rightarrow [0,1608;0,2392]$

Toleranzintervall für den unteren Wert: $[0,160;0,165]$

Toleranzintervall für den unteren Wert: $[0,239;0,243]$}
\end{beispiel}