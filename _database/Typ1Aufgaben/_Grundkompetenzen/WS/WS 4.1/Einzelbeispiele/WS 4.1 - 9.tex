\section{WS 4.1 - 9 - MAT - Blutgruppe - OA - Matura 2. NT 2015/16}

\begin{beispiel}[WS 4.1]{1} %PUNKTE DES BEISPIELS
In Europa beträgt die Wahrscheinlichkeit, mit Blutgruppe $B$ geboren zu werden, ca. 0,14. Für eine Untersuchung wurden $n$ in Europa geborene Personen zufällig ausgewählt. Die Zufallsvariable $X$ beschreibt die Anzahl der Personen mit Blutgruppe $B$. Die Verteilung von $X$ kann durch eine Normalverteilung approximiert werden, deren Dichtefunktion in der nachstehenden Abbildung dargestellt ist. \leer

\psset{xunit=0.2cm,yunit=80cm,algebraic=true,dimen=middle,dotstyle=o,dotsize=5pt 0,linewidth=0.8pt,arrowsize=3pt 2,arrowinset=0.25}
\begin{pspicture*}(19,-0.01)(84,0.07)
\multips(0,0.01)(0,0.01){14}{\psline[linestyle=dashed,linecap=1,dash=1.5pt 1.5pt,linewidth=0.4pt,linecolor=black!60]{c-c}(24,0)(100.89337142722376,0)}
\multips(0,0)(4.0,0){40}{\psline[linestyle=dashed,linecap=1,dash=1.5pt 1.5pt,linewidth=0.4pt,linecolor=black!60]{c-c}(24,0)(24,0.1339331903967604)}
\pszigzag[coilarm=0.22,coilwidth=0.5,coilheight=0.5](24,0)(28,0)
\psaxes[comma, labelFontSize=\scriptstyle,xAxis=true,yAxis=false,Ox=28,Dx=4.,Dy=0.02,ticksize=-2pt 0,subticks=1]{->}(28,0)(28,0)(84,0.07)[$x$,140] [,-40]
\psaxes[comma, labelFontSize=\scriptstyle,xAxis=false,yAxis=true,Dy=0.02,ticksize=-2pt 0,subticks=2]{->}(24,0)(24,0)(84,0.07)[,140] [$\varphi(x)$,-40]
\psplot[linewidth=1.2pt,linecolor=black,plotpoints=200]{28}{85}{EXP((-(x-56.0)^(2.0))/(7.0^(2.0)*2.0))/(abs(7.0)*sqrt(3.141592653589793*2.0))}
\begin{scriptsize}
\rput[bl](62,0.045){$\varphi$}
\end{scriptsize}
\end{pspicture*}

Schätze anhand der obigen Abbildung den Stichprobenumfang $n$ dieser Untersuchung.\leer

$n\approx \antwort[\rule{5cm}{0.3pt}]{400}$ 

\antwort{\leer

Toleranzintervall: $[385;415]$} 
			
\end{beispiel}