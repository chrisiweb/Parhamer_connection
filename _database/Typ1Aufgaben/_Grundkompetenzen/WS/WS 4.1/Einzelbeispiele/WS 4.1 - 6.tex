\section{WS 4.1 - 6 - MAT - Vergleich zweier Konfidenzintervalle - LT - Matura HT 2015/16}

\begin{beispiel}[WS 4.1]{1} %PUNKTE DES BEISPIELS
Auf der Grundlage einer Zufallsstichprobe der Größe $n_1$ gibt ein Meinungsforschungsinstitut für den aktuellen Stimmenanteil einer politischen Partei das Konfidenzintervall [0,23; 0,29] an. Das zugehörige Konfidenzniveau (die zugehörige Sicherheit) beträgt $\gamma_1$.
Ein anderes Institut befragt $n_2$ zufällig ausgewählte Wahlberechtigte und gibt als entsprechendes Konfidenzintervall mit dem Konfidenzniveau (der zugehörigen Sicherheit) $\gamma_2$ das Intervall $[0,24;~0,28]$ an. Dabei verwenden beide Institute dieselbe Berechnungsmethode.
\leer

\lueckentext{
				text={Unter der Annahme von $n_1 = n_2$ kann man aus den Angaben \gap folgern;\\
				unter der Annahme von $\gamma_1 = \gamma_2$ kann man aus den Angaben \gap folgern.}, 	%Lueckentext Luecke=\gap
				L1={$\gamma_1< \gamma_2$}, 		%1.Moeglichkeit links  
				L2={$\gamma_1= \gamma_2$}, 		%2.Moeglichkeit links
				L3={$\gamma_1> \gamma_2$}, 		%3.Moeglichkeit links
				R1={$n_1< n_2$}, 		%1.Moeglichkeit rechts 
				R2={$n_1= n_2$}, 		%2.Moeglichkeit rechts
				R3={$n_1> n_2$}, 		%3.Moeglichkeit rechts
				%% LOESUNG: %%
				A1=3,   % Antwort links
				A2=1		% Antwort rechts 
				}
\end{beispiel}