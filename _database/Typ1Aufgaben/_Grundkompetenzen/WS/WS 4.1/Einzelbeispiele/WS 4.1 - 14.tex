\section{WS 4.1 - 14 - Konfidenzintervall verkürzen - OA - Matura - 1. NT 2017/18}

\begin{beispiel}[WS 4.1]{1}
Ein Spielzeug produzierendes Unternehmen führt in einer Gemeinde in 500 zufällig ausgewählten Haushalten eine Befragen durch und erhält ein 95-\%-Konfidenzintervall für den unbekannten Anteil aller Haushalte dieser Gemeinde, die die Spielzeuge dieses Unternehmens kennen.

Bei einer anderen Befragung von $n$ zufällig ausgewählten Haushalten ergab sich derselbe Wert für die relative Häufigkeit. Das aus dieser Befragung mi9t derselben Berechnungsmethode ermittelte symmetrische 95-\%-Konfidenzintervall hatte aber eine geringere Breite als jenes aus der ersten Befragung.

Gib alle $n\in\mathbb{N}$ an, für die dieser Fall unter der angegebenen Bedingung eintritt!

\antwort{$n>500$}
\end{beispiel}