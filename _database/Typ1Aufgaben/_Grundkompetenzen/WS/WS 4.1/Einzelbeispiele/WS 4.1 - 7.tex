\section{WS 4.1 - 7 Meinungsbefragung - MC - BIFIE Kompetenzcheck 2016}

\begin{beispiel}[WS 4.1]{1} %PUNKTE DES BEISPIELS
				Bei einer Meinungsbefragung wurden 500 zufällig ausgewählte BewohnerInnen einer Stadt zu ihrer Meinung bezüglich der Einrichtung einer Fußgängerzone im Stadtzentrum befragt. Es sprachen sich 60\,\% der Befragten für die Einrichtung einer solchen Fußgängerzone aus, 40\,\% sprachen sich dagegen aus.

Als 95-\%-Konfidenzintervall für den Anteil der BewohnerInnen dieser Stadt, die die Einrichtung einer Fußgängerzone im Stadtzentrum befürworten, erhält man mit Normalapproximation das Intervall $\left[55,7\,\%;64,3\,\%\right]$.\\

Kreuze die beiden zutreffenden Aussagen an.

\multiplechoice[5]{  %Anzahl der Antwortmoeglichkeiten, Standard: 5
				L1={Das Konfidenzintervall wäre breiter, wenn man einen größeren Stichprobenumfang gewählt hätte und der relative Anteil der BefürworterInnen gleich groß geblieben wäre.},   %1. Antwortmoeglichkeit 
				L2={Das Konfidenzintervall wäre breiter, wenn man ein höheres Konfidenzniveau (eine höhere Sicherheit) gewählt hätte.},   %2. Antwortmoeglichkeit
				L3={Das Konfidenzintervall wäre breiter, wenn man die Befragung in einer größeren Stadt durchgeführt hätte.},   %3. Antwortmoeglichkeit
				L4={Das Konfidenzintervall wäre breiter, wenn der Anteil der BefürworterInnen in der Stichprobe größer gewesen wäre.},   %4. Antwortmoeglichkeit
				L5={Das Konfidenzintervall wäre breiter, wenn der Anteil der BefürworterInnen und der Anteil der GegnerInnen in der Stichprobe gleich groß gewesen wären.},	 %5. Antwortmoeglichkeit
				L6={},	 %6. Antwortmoeglichkeit
				L7={},	 %7. Antwortmoeglichkeit
				L8={},	 %8. Antwortmoeglichkeit
				L9={},	 %9. Antwortmoeglichkeit
				%% LOESUNG: %%
				A1=2,  % 1. Antwort
				A2=5,	 % 2. Antwort
				A3=0,  % 3. Antwort
				A4=0,  % 4. Antwort
				A5=0,  % 5. Antwort
				}
\end{beispiel}