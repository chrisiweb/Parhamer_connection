\section{WS 4.1 - 19 - MAT - Konditionierungsexperiment - OA - Matura 2019/20 1. HT}

\begin{beispiel}[WS 4.1]{1}
Bei einem Konditionierungsexperiment lernen Schäferhunde die Bedienung eines Mechanismus, um Futter zu erhalten. Nach einer Trainingsphase, an der 50 Schäferhunde teilnehmen, können 40 von ihnen den Mechanismus bedienen.

Der relative Anteil dieser Schäferhunde, die nach der Trainingsphase den Mechanismus bedienen können, wird mit $h$ bezeichnet.

Aus diesen Daten wird ein um $h$ symmetrischen Konfidenzintervall $[a;0,91]$ mit $a\in\mathbb{R}$ für den unbekannten Anteil $p$ aller Schäferhunder ermittelt, die nach einer solchen Trainingsphase den Mechanismus bedienen können.

Ermittle die untere Grenze $a$ des Konfidenzintervalls.

\antwort{mögliche Vorgehensweise:

$h=\frac{40}{50}=0,8$

$0,91-0,8=0,11$

$a=0,8-0,11=0,69$}
\end{beispiel}