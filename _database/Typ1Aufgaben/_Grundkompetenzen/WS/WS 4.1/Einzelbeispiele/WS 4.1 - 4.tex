\section{WS 4.1 - 4 Linkshänder - MC - BIFIE}

\begin{beispiel}[WS 4.1]{1} %PUNKTE DES BEISPIELS
				Bei einer Umfrage in einem Bezirk werden 500 Personen befragt, ob sie Linkshänder sind. Als Ergebnis der Befragung wird das 95-\%-Konfidenzintervall $\left[0,09;0,15\right]$ für den Anteil der Linkshänder in der Bezirkszeitung bekanntgegeben.

Welche der nachstehenden Aussagen kannst du aufgrund dieses Ergebnisses tätigen? Kreuze die zutreffende(n) Aussage(n) an.

\multiplechoice[5]{  %Anzahl der Antwortmoeglichkeiten, Standard: 5
				L1={Ungefähr 60 Personen haben angegeben, Linkshänder zu sein.},   %1. Antwortmoeglichkeit 
				L2={Hätte man 10.000 Personen befragt, wäre das 95-\%-Konfidenzintervall schmäler geworden.},   %2. Antwortmoeglichkeit
				L3={Das Konfidenzintervall wäre breiter, wenn der Anteil der Linkshänder in der Umfrage kleiner gewesen wäre.},   %3. Antwortmoeglichkeit
				L4={Der Anteil der Linkshänder im gesamten Bezirk liegt jedenfalls zwischen 9\% und 15\%.},   %4. Antwortmoeglichkeit
				L5={Das entsprechende 99-\%-Konfidenzintervall ist breiter als das 95-\%-Konfidenzintervall.},	 %5. Antwortmoeglichkeit
				L6={},	 %6. Antwortmoeglichkeit
				L7={},	 %7. Antwortmoeglichkeit
				L8={},	 %8. Antwortmoeglichkeit
				L9={},	 %9. Antwortmoeglichkeit
				%% LOESUNG: %%
				A1=1,  % 1. Antwort
				A2=2,	 % 2. Antwort
				A3=5,  % 3. Antwort
				A4=0,  % 4. Antwort
				A5=0,  % 5. Antwort
				}
\end{beispiel}	