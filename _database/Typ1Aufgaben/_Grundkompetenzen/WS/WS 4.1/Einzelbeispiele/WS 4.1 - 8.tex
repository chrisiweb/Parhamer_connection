\section{WS 4.1 - 8 500-Euro-Scheine in �sterreich - OA - Matura 2015/16 - Nebentermin 1}

\begin{beispiel}[WS 4.1]{1} %PUNKTE DES BEISPIELS
Bei einer repr�sentativen Umfrage in �sterreich geht es um die in Diskussion stehende Abschaffung
der 500-Euro-Scheine. Es sprechen sich 234 von 1000 Befragten f�r eine Abschaffung aus. \leer

Geben Sie ein symmetrisches 95-\%-Konfidenzintervall f�r den relativen Anteil der �sterreicherinnen
und �sterreicher, die eine Abschaffung der 500-Euro-Scheine in �sterreich bef�rworten, an.

\antwort{$n=1000$, $h=0,234$ 

$0,234 \pm 1,96 \cdot \sqrt{\dfrac{0,234\cdot(1-0,234)}{1000}}\approx 0,234 \pm 0,026 \Rightarrow [0,208;~ 0,206]$ \leer

L�sungsschl�ssel:

Ein Punkt f�r ein korrektes Intervall. Andere Schreibweisen des Ergebnisses (als Bruch oder in
Prozent) sind ebenfalls als richtig zu werten.

Toleranzintervall f�r den unteren Wert: $[0,20;~0,21]$

Toleranzintervall f�r den oberen Wert: $[0,26;~0,27]$

Die Aufgabe ist auch dann als richtig gel�st zu werten, wenn bei korrektem Ansatz das Ergebnis
aufgrund eines Rechenfehlers nicht richtig ist.}

\end{beispiel}