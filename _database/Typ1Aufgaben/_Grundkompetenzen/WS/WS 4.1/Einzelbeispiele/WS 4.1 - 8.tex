\section{WS 4.1 - 8 - MAT - 500-Euro-Scheine in Österreich - OA - Matura 1. NT 2015/16}

\begin{beispiel}[WS 4.1]{1} %PUNKTE DES BEISPIELS
Bei einer repräsentativen Umfrage in Österreich geht es um die in Diskussion stehende Abschaffung
der 500-Euro-Scheine. Es sprechen sich 234 von 1000 Befragten für eine Abschaffung aus. \leer

Geben Sie ein symmetrisches 95-\%-Konfidenzintervall für den relativen Anteil der Österreicherinnen
und Österreicher, die eine Abschaffung der 500-Euro-Scheine in Österreich befürworten, an.

\antwort{$n=1000$, $h=0,234$ 

$0,234 \pm 1,96 \cdot \sqrt{\dfrac{0,234\cdot(1-0,234)}{1000}}\approx 0,234 \pm 0,026 \Rightarrow [0,208;~ 0,206]$ \leer

Lösungsschlüssel:

Ein Punkt für ein korrektes Intervall. Andere Schreibweisen des Ergebnisses (als Bruch oder in
Prozent) sind ebenfalls als richtig zu werten.

Toleranzintervall für den unteren Wert: $[0,20;~0,21]$

Toleranzintervall für den oberen Wert: $[0,26;~0,27]$

Die Aufgabe ist auch dann als richtig gelöst zu werten, wenn bei korrektem Ansatz das Ergebnis
aufgrund eines Rechenfehlers nicht richtig ist.}

\end{beispiel}