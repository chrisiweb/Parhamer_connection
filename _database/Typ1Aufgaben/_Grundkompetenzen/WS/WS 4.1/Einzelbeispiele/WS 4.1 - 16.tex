\section{WS 4.1 - 16 - MAT - Konfidenzintervalle - OA - Matura-HT-18/19}

\begin{beispiel}[WS 4.1]{1}
Jemand möchte den unbekannten Anteil $p$ derjenigen Wählerinnen und Wähler ermitteln, die bei
einer Wahl für den Kandidaten A stimmen werden, und beauftragt ein Meinungsforschungsinstitut
damit, diesen Anteil $p$ zu schätzen. Im Zuge dieser Schätzung werden 200 Stichproben
mit jeweils gleichem Umfang ermittelt. Für jede dieser Stichproben wird das entsprechende
95-\%-Konfidenzintervall berechnet.\leer


Berechne die erwartete Anzahl derjenigen Intervall, die den unbekannten Anteil $p$ enthalten!

\antwort{$200\cdot 0,95 =190$ 

Die erwartete Anzahl ist 190.}
\end{beispiel}