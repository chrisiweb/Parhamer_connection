\section{WS 4.1 - 15 - MAT - Telefonumfrage - OA - Matura 2. NT 2017/18}

\begin{beispiel}[WS 4.1]{1}
Bei einer repräsentativen Telefonumfrage mit 400 zufällig ausgewählten Personen erhält man für den relativen Anteil der Befürworter/innen von kürzeren Sommerferien den Wert 20\,\%.


Zeigen Sie durch eine Rechnung, dass das Intervall [16,0\,\%; 24,0\,\%] ein symmetrisches
95-\%-Konfidenzintervall für den relativen Anteil $p$ der Befürworter/innen in der gesamten Bevölkerung sein kann (wobei die Intervallgrenzen des Konfidenzintervalls gerundete Werte sind)!

\antwort{Möglicher rechnerischer Nachweis:

$0,2 \pm 1,96 \cdot \sqrt{\dfrac{0,2 \cdot 0,8}{400}}\approx 0,2 \pm 0,004 \Rightarrow [16\,\%; 24\,\%]$ \leer

Das Intervall [16\,\%; 24\,\%] kann daher ein symmetrisches 95-\%-Konfidenzintervall für den relativen Anteil $p$ der Befürworter/innen in der gesamten Bevölkerung sein.}
\end{beispiel}