\section{WS 4.1 - 13 Intervallbreite von Konfidenzintervallen - OA - Matura 17/18}

\begin{beispiel}[WS 4.1]{1} %PUNKTE DES BEISPIELS
Vier Konfidenzintervalle ($A, B, C$ und $D$) für einen unbekannten Anteil werden auf dieselbe Art und Weise ausschließlich unter Verwendung des Stichprobenumfangs $n$, des Konfidenzniveaus $\gamma$ und des relativen Anteils berechnet, wobei der relative Anteil für alle vier Konfidenzintervalle derselbe ist. Die Konfidenzintervalle liegen symmetrisch um den relativen Anteil.

\begin{center}
	\begin{tabular}{|c|c|c|}\hline
	\cellcolor[gray]{0.9}{Konfidenzintervall}&\cellcolor[gray]{0.9}{Stichprobenumfang $n$}&\cellcolor[gray]{0.9}{Konfidenzniveau $\sigma$}\\ \hline
	$A$&500&90\,\%\\ \hline
	$B$&500&95\,\%\\ \hline
	$C$&2000&90\,\%\\ \hline
	$D$&2000&95\,\%\\ \hline
	\end{tabular}
\end{center}

Vergleiche diese vier Konfidenzintervalle bezüglich ihrer Intervallbreite und gib das Konfidenzintervall mit der kleinsten und jenes mit der größten Intervallbreite an!\leer

Konfidenzintervall mit der kleinsten Intervallbreite: \antwort[\rule{1.5cm}{0.3pt}]{C}\leer

Konfidenzintervall mit der größten Intervallbreite: \antwort[\rule{1.5cm}{0.3pt}]{B}
\end{beispiel}