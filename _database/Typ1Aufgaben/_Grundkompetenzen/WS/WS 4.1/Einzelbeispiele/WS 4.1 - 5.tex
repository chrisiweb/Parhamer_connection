\section{WS 4.1 - 5 - Essgewohnheiten - OA - BIFIE}

\begin{beispiel}[WS 4.1]{1} %PUNKTE DES BEISPIELS
				Um die Essgewohnheiten von Jugendlichen zu untersuchen, wurden 400 Jugendliche eines Bezirks zufällig ausgewählt und befragt.

Dabei gaben 240 der befragten Jugendlichen an, täglich zu frühstücken.

Berechne aufgrund des in der Umfrage erhobenen Stichprobenergebnisses ein 99-\%-Konfidenzintervall für den tatsächlichen (relativen) Anteil $p$ derjenigen Jugendlichen dieses Bezirks, die täglich frühstücken.\\

\antwort{Die Zufallsvariable $X$ gibt die Anzahl der Jugendlichen, die täglich frühstücken, an.\\
$h=\frac{240}{400}=0,6$\\
$2\cdot \Theta(z)-1=D(z)=0,99\Rightarrow z\approx2,58$\\
$p_{1,2}=0,6\pm 2,58\cdot \sqrt{\dfrac{0,6\cdot 0,4}{400}}\Rightarrow p_{1}\approx 0,536; p_{2}\approx0,664$\\
99-\%-Konfidenzintervall: $\left[0,536;0,664\right]$ bzw. $0,6\pm 0,064$\\

Ein Punkt ist genau dann zu geben, wenn das Konfidenzintervall richtig berechnet wurde.\\
Toleranzintervall für die untere Grenze: $\left[0,53;0,54\right]$\\
Toleranzintervall für die obere Grenze: $\left[0,66;0,67\right]$}	
\end{beispiel}	