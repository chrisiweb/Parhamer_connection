\section{WS 3.3 - 5 Binomialverteilte Zufallsvariable - MC - Matura 2013/14 1. Nebentermin}

\begin{beispiel}[WS 3.3]{1} %PUNKTE DES BEISPIELS
				In einer Urne befinden sich sieben weiße und drei rote Kugeln, die gleich groß und durch Tasten nicht unterscheidbar sind. Jemand nimmt, ohne hinzusehen, Kugeln aus der Urne.
				
				In welchen der folgenden Fälle ist die Zufallsvariable $X$ binomialverteilt? Kreuze die beiden zutreffenden Aussagen an!\leer
				
				\multiplechoice[5]{  %Anzahl der Antwortmoeglichkeiten, Standard: 5
								L1={$X$ beschreibt die Anzahl der roten Kugeln bei dreimaligem Ziehen,  wenn jede entnommene Kugel wieder zurückgelegt wird. },   %1. Antwortmoeglichkeit 
								L2={$X$ beschreibt die Anzahl der weißen Kugeln bei viermaligem Ziehen, wenn die entnommenen Kugeln nicht zurückgelegt werden.},   %2. Antwortmoeglichkeit
								L3={$X$ beschreibt die Anzahl der weißen Kugeln bei fünfmaligem Ziehen, wenn jede entnommene Kugel wieder zurückgelegt wird. },   %3. Antwortmoeglichkeit
								L4={$X$ beschreibt die Anzahl der Züge, bis die erste rote Kugel gezogen wird, wenn jede entnommene Kugel wieder zurückgelegt wird. },   %4. Antwortmoeglichkeit
								L5={$X$ beschreibt die Anzahl der Züge, bis alle weißen Kugeln gezogen  wurden, wenn die entnommenen Kugeln nicht zurückgelegt werden.},	 %5. Antwortmoeglichkeit
								L6={},	 %6. Antwortmoeglichkeit
								L7={},	 %7. Antwortmoeglichkeit
								L8={},	 %8. Antwortmoeglichkeit
								L9={},	 %9. Antwortmoeglichkeit
								%% LOESUNG: %%
								A1=1,  % 1. Antwort
								A2=3,	 % 2. Antwort
								A3=0,  % 3. Antwort
								A4=0,  % 4. Antwort
								A5=0,  % 5. Antwort
								}
\end{beispiel}