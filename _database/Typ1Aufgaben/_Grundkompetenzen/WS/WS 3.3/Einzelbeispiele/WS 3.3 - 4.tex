\section{WS 3.3 - 4 Sammelwahrscheinlichkeit bei �berraschungseiern - OA - Matura 2014/15 - Nebentermin 1}

\begin{beispiel}[WS 3.3]{1}
Ein italienischer S��warenhersteller stellt �berraschungseier her. Das Ei besteht aus Schokolade. Im Inneren des Eies befindet sich in einer gelben Kapsel ein Spielzeug oder eine Sammelfigur. Der Hersteller wirbt f�r die Star-Wars-Sammelfiguren mit dem Slogan "`Wir sind jetzt mit dabei, in
jedem 7.�Ei!"'. 

Peter kauft in einem Gesch�ft zehn �berraschungseier aus dieser Serie. Berechne die Wahrscheinlichkeit, dass Peter mindestens eine Star-Wars-Sammelfigur erh�lt. 

\antwort{
$1-\left(\frac{6}{7}\right)^{10}$ \leer

L�sungsschl�ssel:\\
Ein Punkt f�r die richtige L�sung. Andere Schreibweisen des Ergebnisses (als Dezimalzahl, in Prozent)
sind ebenfalls als richtig zu werten. \\
Toleranzintervalle: $[0,78; 0,79]$ bzw. $[78\,\%; 79\,\%]$}
\end{beispiel}