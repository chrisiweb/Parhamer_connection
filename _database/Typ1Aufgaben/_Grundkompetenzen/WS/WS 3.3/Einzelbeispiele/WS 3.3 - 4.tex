\section{WS 3.3 - 4 - MAT - Sammelwahrscheinlichkeit bei Überraschungseiern - OA - Matura 1. NT 2014/15}

\begin{beispiel}[WS 3.3]{1}
Ein italienischer Süßwarenhersteller stellt Überraschungseier her. Das Ei besteht aus Schokolade. Im Inneren des Eies befindet sich in einer gelben Kapsel ein Spielzeug oder eine Sammelfigur. Der Hersteller wirbt für die Star-Wars-Sammelfiguren mit dem Slogan "`Wir sind jetzt mit dabei, in
jedem 7. Ei!"'. 

Peter kauft in einem Geschäft zehn Überraschungseier aus dieser Serie. Berechne die Wahrscheinlichkeit, dass Peter mindestens eine Star-Wars-Sammelfigur erhält. 

\antwort{
$1-\left(\frac{6}{7}\right)^{10}$ \leer

Lösungsschlüssel:\\
Ein Punkt für die richtige Lösung. Andere Schreibweisen des Ergebnisses (als Dezimalzahl, in Prozent)
sind ebenfalls als richtig zu werten. \\
Toleranzintervalle: $[0,78; 0,79]$ bzw. $[78\,\%; 79\,\%]$}
\end{beispiel}