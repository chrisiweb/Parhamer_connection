\section{WS 3.3 - 6 - MAT - Reifen - OA - Matura 1. NT 2016/17}

\begin{beispiel}[WS 3.3]{1} %PUNKTE DES BEISPIELS
Die Wahrscheinlichkeit, dass ein neuer Autoreifen einer bestimmten Marke innerhalb der ersten 10\,000 Kilometer Fahrt durch einen Materialfehler defekt wird, liegt bei $p\,\%$.

Eine Zufallsstichprobe von 80 neuen Reifen dieser Marke wird getestet.

Gib einen Ausdruck an, mit dem man die Wahrscheinlichkeit, dass mindestens einer dieser Reifen innerhalb der ersten 10\,000 Kilometer Fahrt durch einen Materialfehler defekt wird, berechnen kann!

\antwort{$1-(1-\frac{p}{100})^{80}$}
\end{beispiel}