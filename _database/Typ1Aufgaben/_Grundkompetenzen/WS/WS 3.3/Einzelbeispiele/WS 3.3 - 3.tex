\section{WS 3.3 - 3 Modellierung mit Binomialverteilung - MC - BIFIE}

\begin{beispiel}[WS 3.3]{1} %PUNKTE DES BEISPIELS
Gegeben sind fünf Situationen, bei denen nach einer Wahrscheinlichkeit gefragt wird. \leer

Kreuze  diejenige(n) Situation(en) an, die mithilfe der Binomialverteilung modelliert werden
kann/können.

\multiplechoice[5]{  %Anzahl der Antwortmoeglichkeiten, Standard: 5
				L1={In der Kantine eines Betriebes essen 80 Personen. Am Montag werden ein
vegetarisches Gericht und drei weitere Menüs angeboten. Erfahrungsgemäß
wählt jede vierte Person das vegetarische Gericht. Es werden 20 vegetarische
Gerichte vorbereitet. \\
Wie groß ist die Wahrscheinlichkeit, dass diese nicht ausreichen? },   %1. Antwortmoeglichkeit 
				L2={Bei einer Lieferung von 20 Smartphones sind fünf defekt. Es werden nacheinander
drei Geräte entnommen, getestet und nicht zurückgelegt.
Mit welcher Wahrscheinlichkeit sind mindestens zwei davon defekt? },   %2. Antwortmoeglichkeit
				L3={In einer Klasse müssen die Schüler/innen bei der Überprüfung der Bildungsstandards
auf einem anonymen Fragebogen ihr Geschlecht (m, w) ankreuzen.
In der Klasse sind 16 Schülerinnen und 12 Schüler. Fünf Personen haben auf
dem Fragebogen das Geschlecht nicht angekreuzt. \\
Mit welcher Wahrscheinlichkeit befinden sich drei Schüler unter den fünf Personen? },   %3. Antwortmoeglichkeit
				L4={Ein Großhändler erhält eine Lieferung von 2 000 Smartphones, von denen
erfahrungsgemäß 5\,\% defekt sind.\\
Mit welcher Wahrscheinlichkeit befinden sich 80 bis 90 defekte Geräte in der
Lieferung? },   %4. Antwortmoeglichkeit
				L5={In einer Klinik werden 500 kranke Personen mit einem bestimmten Medikament
behandelt. Die Wahrscheinlichkeit, dass schwere Nebenwirkungen auftreten,
beträgt 0,001. \\
Wie groß ist die Wahrscheinlichkeit, dass bei mehr als zwei Personen schwere
Nebenwirkungen auftreten? },	 %5. Antwortmoeglichkeit
				L6={},	 %6. Antwortmoeglichkeit
				L7={},	 %7. Antwortmoeglichkeit
				L8={},	 %8. Antwortmoeglichkeit
				L9={},	 %9. Antwortmoeglichkeit
				%% LOESUNG: %%
				A1=1,  % 1. Antwort
				A2=4,	 % 2. Antwort
				A3=5,  % 3. Antwort
				A4=0,  % 4. Antwort
				A5=0,  % 5. Antwort
				} 

\end{beispiel}