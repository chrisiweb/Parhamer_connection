\section{WS 3.3 - 2 - Binomialverteilung - MC - BIFIE}

\begin{beispiel}[WS 3.3]{1} %PUNKTE DES BEISPIELS
Einige der unten angeführten Situationen können mit einer Binomialverteilung modelliert werden.

Kreuze diejenige(n) Situation(en) an, bei der/denen die Zufallsvariable $X$ binomialverteilt ist.

\multiplechoice[5]{  %Anzahl der Antwortmoeglichkeiten, Standard: 5
				L1={Aus einer Urne mit vier blauen, zwei grünen und drei weißen
Kugeln werden drei Kugeln mit Zurücklegen gezogen.
($X$ = Anzahl der grünen Kugeln)},   %1. Antwortmoeglichkeit 
				L2={In einer Gruppe mit 25 Kindern sind sieben Linkshänder. Es
werden drei Kinder zufällig ausgewählt.
($X$ = Anzahl der Linkshänder)},   %2. Antwortmoeglichkeit
				L3={In einem U-Bahn-Waggon sitzen 35 Personen. Vier haben keinen
Fahrschein. Drei werden kontrolliert.
($X$ = Anzahl der Personen ohne Fahrschein)
},   %3. Antwortmoeglichkeit
				L4={Bei einem Multiple-Choice-Test sind pro Aufgabe drei von fünf
Wahlmöglichkeiten richtig. Die Antworten werden nach dem
Zufallsprinzip angekreuzt. Sieben Aufgaben werden gestellt.
($X$ = Anzahl der richtig gelösten Aufgaben).},   %4. Antwortmoeglichkeit
				L5={Die Wahrscheinlichkeit für die Geburt eines Mädchens liegt bei
52\,\%. Eine Familie hat drei Kinder.
($X$ = Anzahl der Mädchen)
},	 %5. Antwortmoeglichkeit
				L6={},	 %6. Antwortmoeglichkeit
				L7={},	 %7. Antwortmoeglichkeit
				L8={},	 %8. Antwortmoeglichkeit
				L9={},	 %9. Antwortmoeglichkeit
				%% LOESUNG: %%
				A1=1,  % 1. Antwort
				A2=4,	 % 2. Antwort
				A3=5,  % 3. Antwort
				A4=0,  % 4. Antwort
				A5=0,  % 5. Antwort
				} 
\end{beispiel} 