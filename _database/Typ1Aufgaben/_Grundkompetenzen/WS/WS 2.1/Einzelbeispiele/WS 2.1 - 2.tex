\section{WS 2.1 - 2 Schülerinnenbefragung - MC - BIFIE}

\begin{beispiel}[WS 2.1]{1} %PUNKTE DES BEISPIELS
In einer Schule wird unter den Mädchen eine Umfrage durchgeführt. Dazu werden pro Klasse
zwei Schülerinnen zufällig für ein Interview ausgewählt. Eva und Sonja gehen in die 1A.
Für das Ereignis $E_1$ gilt: Eva und Sonja werden für das Interview ausgewählt. \leer

Welche der nachstehenden Aussagen beschreibt das Gegenereignis $E_2$? (Das Gegenereignis
$E_2$ enthält diejenigen Elemente des Grundraums, die nicht Elemente von $E_1$ sind.)
Kreuze die zutreffende Aussage an.

\multiplechoice[6]{  %Anzahl der Antwortmoeglichkeiten, Standard: 5
				L1={Nur Eva wird ausgewählt.},   %1. Antwortmoeglichkeit 
				L2={Keines der beiden Mädchen wird ausgewählt.},   %2. Antwortmoeglichkeit
				L3={Mindestens eines der beiden Mädchen wird ausgewählt.},   %3. Antwortmoeglichkeit
				L4={Nur Sonja wird ausgewählt.},   %4. Antwortmoeglichkeit
				L5={Höchstens eines der beiden Mädchen wird ausgewählt.},	 %5. Antwortmoeglichkeit
				L6={Genau eines der beiden Mädchen wird ausgewählt.},	 %6. Antwortmoeglichkeit
				L7={},	 %7. Antwortmoeglichkeit
				L8={},	 %8. Antwortmoeglichkeit
				L9={},	 %9. Antwortmoeglichkeit
				%% LOESUNG: %%
				A1=5,  % 1. Antwort
				A2=0,	 % 2. Antwort
				A3=0,  % 3. Antwort
				A4=0,  % 4. Antwort
				A5=0,  % 5. Antwort
				} 
\end{beispiel}