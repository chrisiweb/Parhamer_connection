\section{WS 2.1 - 6 - K6 - Würfelwurf mit speziellen Würfeln - MC - CleTur}

\begin{beispiel}[WS 2.1]{1} %PUNKTE DES BEISPIELS
In einem Spiel werden ein 6-seitiger und zwei 4-seitige Würfel geworfen. Dabei wird die Summe der gewürfelten Augenzahl(Punkte) notiert. $\Omega $ bezeichnet den Grundraum dieses Zufallversuchs.

Kreuze die zutreffende Aussage an!

\multiplechoice[5]{  %Anzahl der Antwortmoeglichkeiten, Standard: 5
				L1={$\Omega = \{3,4,5,6,7,8,9,10,11,12,13,14,15,16,17,18 \}$},   %1. Antwortmoeglichkeit 
				L2={$\Omega = \{2,3,4,5,6,7,8,9,10,11,12,13,14 \}$},   %2. Antwortmoeglichkeit
				L3={$\Omega = \{1,2,3,4,5,6,7,8,9,10\}$},   %3. Antwortmoeglichkeit
				L4={$\Omega = \{3,4,5,6,7,8,9,10,11,12,13,14\}$},   %4. Antwortmoeglichkeit
				L5={$\Omega = \{2,3,4,5,6,7,8,9,10,11,12\}$},	 %5. Antwortmoeglichkeit
				%% LOESUNG: %%
				A1=4,  % 1. Antwort
				A2=0,	 % 2. Antwort
				A3=0,  % 3. Antwort
				A4=0,  % 4. Antwort
				A5=0,  % 5. Antwort
				}
\end{beispiel}