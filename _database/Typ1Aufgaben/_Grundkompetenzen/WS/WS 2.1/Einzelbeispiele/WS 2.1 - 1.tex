\section{WS 2.1 - 1 Ereignisse - OA - BIFIE}


\begin{beispiel}[WS 2.1]{1} %PUNKTE DES BEISPIELS
In einer Schachtel befinden sich:

\hspace{4cm}\begin{tabular}{l}
3 rote Kugeln, \\
20 grüne Kugeln und \\
47 blaue Kugeln. \\
\end{tabular}

Die Kugeln sind -- abgesehen von ihrer Farbe -- nicht unterscheidbar.
Es werden nacheinander 3 Kugeln nach dem Zufallsprinzip entnommen, wobei diese nach
jedem Zug wieder zurückgelegt werden. 

\leer

Der Grundraum dieses Zufallsexperiments ist die Menge aller möglichen Farbtripel ($x$; $y$; $z$).
$x$, $y$ und $z$ nehmen dabei die Buchstaben $r$, $g$ oder $b$ -- entsprechend der Farbe der Kugeln -- an. Für das Ereignis $E$ gilt: Es werden keine blauen Kugeln gezogen.
Gib alle Elemente des Ereignisses $E$ an! 


\leer

$E=\{\hrulefill\}$


\antwort{$E=\{(r,r,r);(r,r,g);(r,g,r);(g,r,r);(g,g,r);(g,r,g);(r,g,g);(g,g,g)\}$}
\end{beispiel}