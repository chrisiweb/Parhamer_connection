\section{WS 2.1 - 4 - MAT - Rote und blaue Kugeln - LT - Matura 1. NT 2014/15}

\begin{beispiel}[WS 2.1]{1}
In einem Behälter befinden sich 15 rote Kugeln und 18 blaue Kugeln. Die Kugeln sind bis auf ihre Farbe nicht unterscheidbar. Es sollen nun in einem Zufallsexperiment zwei Kugeln nacheinander gezogen werden, wobei die erste Kugel nach dem Ziehen nicht zurückgelegt wird und es auf die Reihenfolge der Ziehung ankommt. \leer

Die Buchstaben $r$ und $b$ haben folgende Bedeutung:
\begin{itemize}
	\item[$r$ \ldots] das Ziehen einer roten Kugel
	\item[$b$ \ldots] das Ziehen einer blauen Kugel
\end{itemize}

\lueckentext[0.01]{
				text={Ein Grundraum $G$ für dieses Zufallsexperiment lautet \gap, und \gap ist ein Ereignis}, 	%Lueckentext Luecke=\gap
				L1={$G=\{r,b\}$}, 		%1.Moeglichkeit links  
				L2={$G=\{(r,r),(r,b),(b,b)\}$}, 		%2.Moeglichkeit links
				L3={$G=\{(r,r),(r,b),(b,r),(b,b)\}$}, 		%3.Moeglichkeit links
				R1={die Wahrscheinlichkeit, dass genau eine blaue Kugel gezogen wird,}, 		%1.Moeglichkeit rechts 
				R2={jede Teilmenge des Grund\-raumes}, 		%2.Moeglichkeit rechts
				R3={$b$}, 		%3.Moeglichkeit rechts
				%% LOESUNG: %%
				A1=3,   % Antwort links
				A2=2		% Antwort rechts 
				}
\end{beispiel}