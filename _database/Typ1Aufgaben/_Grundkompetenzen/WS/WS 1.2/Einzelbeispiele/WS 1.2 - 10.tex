\section{WS 1.2 - 10 - K6 - Sitze im Nationalrat - OA - MarStr UNIVIE}

\begin{beispiel}[WS 1.2]{1}
Im österreichischen Nationalrat befinden sich 183 Sitze, welche auf die einzelnen Fraktionen aufgeteilt sind.  Im folgenden Kreisdiagramm ist die Anzahl der Sitze der jeweiligen Fraktion dargestellt.


\begin{center}

\psset{xunit=0.45cm,yunit=0.45cm,algebraic=true,dimen=middle,dotstyle=o,dotsize=5pt 0,linewidth=1.6pt,arrowsize=3pt 2,arrowinset=0.25}
\begin{pspicture*}(-6.5,-6.5)(6.5,6,5)
\pscustom[linewidth=0.8pt]{\parametricplot{-0.8726646259971647}{1.5707963267948966}{1.*5.*cos(t)+0.*5.*sin(t)+0.|0.*5.*cos(t)+1.*5.*sin(t)+0.}\lineto(0.,0.)\closepath}
\pscustom[linewidth=0.8pt,fillcolor=black,fillstyle=solid,opacity=0.15]{\parametricplot{4.031710572106901}{5.410520681182422}{1.*5.*cos(t)+0.*5.*sin(t)+0.|0.*5.*cos(t)+1.*5.*sin(t)+0.}\lineto(0.,0.)\closepath}
\pscustom[linewidth=0.8pt,fillcolor=black,fillstyle=solid,opacity=0.3]{\parametricplot{3.001966313430247}{4.031710572106901}{1.*5.*cos(t)+0.*5.*sin(t)+0.|0.*5.*cos(t)+1.*5.*sin(t)+0.}\lineto(0.,0.)\closepath}
\pscustom[linewidth=0.8pt,fillcolor=black,fillstyle=solid,opacity=0.45]{\parametricplot{2.1118483949131392}{3.001966313430247}{1.*5.*cos(t)+0.*5.*sin(t)+0.|0.*5.*cos(t)+1.*5.*sin(t)+0.}\lineto(0.,0.)\closepath}
\pscustom[linewidth=0.8pt,fillcolor=black,fillstyle=solid,opacity=0.6]{\parametricplot{1.6057029118347834}{2.1118483949131392}{1.*5.*cos(t)+0.*5.*sin(t)+0.|0.*5.*cos(t)+1.*5.*sin(t)+0.}\lineto(0.,0.)\closepath}
\pscustom[linewidth=0.8pt,fillcolor=black,fillstyle=solid,opacity=0.9]{\parametricplot{1.570796326794897}{1.6057029118347834}{1.*5.*cos(t)+0.*5.*sin(t)+0.|0.*5.*cos(t)+1.*5.*sin(t)+0.}\lineto(0.,0.)\closepath}
\begin{scriptsize}
\rput[bl](4.9,1.8){ÖVP}
\rput[bl](5.25,1.2){$71$}
\rput[bl](-0.5,-5.7){SPÖ}
\rput[bl](-0.2,-6.3){$40$}
\rput[bl](-6.4,-1.7){FPÖ}
\rput[bl](-6.1,-2.3){$30$}
\rput[bl](-5.6,3.5){Grüne}
\rput[bl](-5.1,2.9){$26$}
\rput[bl](-3.2,5.35){NEOS}
\rput[bl](-2.6,4.75){$15$}
\rput[bl](-1.3,6){fraktionslos}
\rput[bl](-0.25,5.3){$1$}
\end{scriptsize}
\end{pspicture*}
\end{center}

Konstruiere ein zugehöriges Säulendiagramm mit allen Fraktionen, in welchem die absoluten Häufigkeiten abzulesen sind, und beschrifte die senkrechte Achse. 

\vspace{0.2cm}
\psset{xunit=1.0cm,yunit=1.0cm,algebraic=true,dimen=middle,dotstyle=o,dotsize=5pt 0,linewidth=1.6pt,arrowsize=3pt 2,arrowinset=0.25}
\begin{pspicture*}(-2,-2)(13.612058734878518,7.5)
\multips(-0.1,0)(0,1.0){9}{\psline[linestyle=dashed,linecap=1,dash=1.5pt 1.5pt,linewidth=0.4pt,linecolor=darkgray]{c-c}(0,0)(13.612058734878518,0)}
\multips(0,0)(1.0,0){15}{\psline[linestyle=dashed,linecap=1,dash=1.5pt 1.5pt,linewidth=0.4pt,linecolor=darkgray]{c-c}(0,0)(0,7.582834522735398)}
\psaxes[labelFontSize=\scriptstyle, showorigin=false, xAxis=true,yAxis=true,Dx=1.,Dy=1., labels=none, ticksize=-2pt 0,subticks=0]{-}(0,0)(-0.8296804330705843,-0.6203244597978499)(13.612058734878518,7.582834522735398)
\antwort{
\psframe[linewidth=0.8pt,linecolor=red,fillcolor=red,fillstyle=solid,opacity=0.5](1.,0)(2.,7.1)
\psframe[linewidth=0.8pt,linecolor=red,fillcolor=red,fillstyle=solid,opacity=0.5](3.,0)(4.,4.)
\psframe[linewidth=0.8pt,linecolor=red,fillcolor=red,fillstyle=solid,opacity=0.5](5.,0)(6.,3.)
\psframe[linewidth=0.8pt,linecolor=red,fillcolor=red,fillstyle=solid,opacity=0.5](7.,0)(8.,2.6)
\psframe[linewidth=0.8pt,linecolor=red,fillcolor=red,fillstyle=solid,opacity=0.5](9.,0)(10.,1.5)
\psframe[linewidth=0.8pt,linecolor=red,fillcolor=red,fillstyle=solid,opacity=0.5](11.,0)(12.,0.1)}
\begin{scriptsize}
	\rput[bl](1.2,-0.4){ÖVP}
	\rput[bl](3.2,-0.4){SPÖ}
	\rput[bl](5.2,-0.4){FPÖ}
	\rput[bl](7.1,-0.4){Grüne}
	\rput[bl](9.1,-0.4){NEOS}
	\rput[bl](10.7,-0.4){fraktionslos}
	\rput[bl](6,-1.1){\normalsize{Fraktionen}}
	\antwort{
	\rput[bl](-0.45,-0.35){0}
	\rput[bl](-0.6,0.9){10}
	\rput[bl](-0.6,1.9){20}
	\rput[bl](-0.6,2.9){30}
	\rput[bl](-0.6,3.9){40}
	\rput[bl](-0.6,4.9){50}
	\rput[bl](-0.6,5.9){60}
	\rput[bl](-0.6,6.9){70}
	\rput[bl](-1.3,2.3){\rotatebox{90}{\normalsize{Anzahl der Sitze}}}}
\end{scriptsize}\end{pspicture*}

\antwort{
\footnotesize
Lösungsschlüssel: \\
Ein Punkt ist genau dann zu geben, wenn die senkrechte Achse korrekt beschriftet wurde und alle Säulen dementsprechend richtig eingezeichnet wurden, wobei die Breite variieren kann. Die notwendige Zeichengenauigkeit ist nach eigenem Ermessen festzulegen.}
\end{beispiel}