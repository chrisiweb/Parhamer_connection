\section{WS 1.2 - 8 - MAT - Histogramm - OA - Matura 2018/19 2. NT}

\begin{beispiel}[WS 1.2]{1}
Ein Betrieb hat insgesamt 200 Beschäftigte. In der nachstehenden Tabelle sind die Stundenlöhne dieser Beschäftigten in Klassen zusammengefasst.

\begin{center}
\begin{tabular}{|c|c|}\hline
\cellcolor[gray]{0.9}Stundenlohn $x$ in Euro&\cellcolor[gray]{0.9}Anzahl der Beschäftigten\\ \hline
$6\leq x<10$&$20$\\ \hline
$10\leq x<15$&$80$\\ \hline
$15\leq x<20$&$60$\\ \hline
$20\leq x\leq 30$&$40$\\ \hline
\end{tabular}
\end{center}

Der Flächeninhalt eines Rechtecks im unten stehenden Histogramm ist der relative Anteil der Beschäftigten in der jeweiligen Klasse.

Ergänze im nachstehenden Histogramm die fehlende Säule so, dass die obigen Daten dargestellt sind.

\begin{center}
\psset{xunit=0.38cm,yunit=0.3cm,algebraic=true,dimen=middle,dotstyle=o,dotsize=5pt 0,linewidth=1.6pt,arrowsize=3pt 2,arrowinset=0.25}
\begin{pspicture*}(-2.2,-4.5)(35.45454545454545,21.695589622641595)
\multips(0,0)(0,2.0){13}{\psline[linestyle=dashed,linecap=1,dash=1.5pt 1.5pt,linewidth=0.4pt,linecolor=lightgray]{c-c}(0,0)(35.45454545454545,0)}
\multips(0,0)(5.0,0){8}{\psline[linestyle=dashed,linecap=1,dash=1.5pt 1.5pt,linewidth=0.4pt,linecolor=lightgray]{c-c}(0,0)(0,21.695589622641595)}
\psaxes[labelFontSize=\scriptstyle,ylabelFactor=\,\%,xAxis=true,yAxis=true,Dx=2.,Dy=2.,ticksize=-2pt 0,subticks=2]{->}(0,0)(0.,0.)(35.45454545454545,21.695589622641595)
\pspolygon[linewidth=2.pt,fillcolor=black,fillstyle=solid,opacity=0.1](6.,0.)(10.,0.)(10.,2.5)(6.,2.5)
\pspolygon[linewidth=2.pt,fillcolor=black,fillstyle=solid,opacity=0.1](10.,0.)(10.,8.)(15.,8.)(15.,0.)
\pspolygon[linewidth=2.pt,fillcolor=black,fillstyle=solid,opacity=0.1](15.,0.)(20.,0.)(20.,6.)(15.,6.)
\antwort{\pspolygon[linewidth=2.pt,fillcolor=black,fillstyle=solid,opacity=0.1](20.,0.)(30.,0.)(30.,2.)(20.,2.)}
\rput[tl](15.636363636363635,-3){$x$ in Euro}
\psline[linewidth=2.pt](6.,0.)(10.,0.)
\psline[linewidth=2.pt](10.,0.)(10.,2.5)
\psline[linewidth=2.pt](10.,2.5)(6.,2.5)
\psline[linewidth=2.pt](6.,2.5)(6.,0.)
\psline[linewidth=2.pt](10.,0.)(10.,8.)
\psline[linewidth=2.pt](10.,8.)(15.,8.)
\psline[linewidth=2.pt](15.,8.)(15.,0.)
\psline[linewidth=2.pt](15.,0.)(10.,0.)
\psline[linewidth=2.pt](15.,0.)(20.,0.)
\psline[linewidth=2.pt](20.,0.)(20.,6.)
\psline[linewidth=2.pt](20.,6.)(15.,6.)
\psline[linewidth=2.pt](15.,6.)(15.,0.)
\antwort{\psline[linewidth=2.pt](20.,0.)(30.,0.)
\psline[linewidth=2.pt](30.,0.)(30.,2.)
\psline[linewidth=2.pt](30.,2.)(20.,2.)
\psline[linewidth=2.pt](20.,2.)(20.,0.)}
\begin{scriptsize}
\rput[bl](7.272727272727272,1.2){$10\,\%$}
\rput[tl](11.727272727272727,3.5){$40\,\%$}
\rput[tl](16.68181818181818,3){$30\,\%$}
\antwort{\rput[bl](24.409090909090907,0.8){$20\,\%$}}
\end{scriptsize}
\end{pspicture*}
\end{center}
\end{beispiel}