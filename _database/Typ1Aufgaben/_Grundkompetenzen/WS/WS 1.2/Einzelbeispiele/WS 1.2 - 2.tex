\section{WS 1.2 - 2 - Testergebnis - MC - BIFIE}

\begin{beispiel}[WS 1.2]{1} %PUNKTE DES BEISPIELS

				\begin{minipage}[t]{0.5\textwidth}
				Ein Test enthält fünf Aufgaben, die jeweils nur mit einem Punkt (alles richtig) oder keinem Punkt (alles nicht richtig) bewertet werden. Die nebenstehende Grafik zeigt das Ergebnis dieses Tests für eine bestimmte Klasse.\leer
				
				Welches der folgenden Kastenschaubilder (Boxplots) stellt die Ergebnisse des Tests richtig da?\leer
				
				Kreuze das zutreffende Kastenschaubild an.
				\end{minipage}
				\begin{minipage}[t]{0.4\textwidth}
				\begin{center}~
				
				\resizebox{0.9\linewidth}{!}{
\psset{xunit=1.0cm,yunit=1.0cm,algebraic=true,dimen=middle,dotstyle=o,dotsize=5pt 0,linewidth=0.8pt,arrowsize=3pt 2,arrowinset=0.25}
\begin{pspicture*}(-1.6214400331880683,-1.2574818535446604)(5.900313706672241,10.401509941128166)
\multips(0,0)(0,1.0){12}{\psline[linestyle=dashed,linecap=1,dash=1.5pt 1.5pt,linewidth=0.4pt,linecolor=gray]{c-c}(0,0)(6.000313706672241,0)}
\multips(0,0)(1.0,0){8}{\psline[linestyle=dashed,linecap=1,dash=1.5pt 1.5pt,linewidth=0.4pt,linecolor=gray]{c-c}(0,0)(0,10.401509941128166)}
\psaxes[xAxis=true,yAxis=true,labels=y,Dx=1.,Dy=1.,ticks=y,ticksize=-2pt 0,subticks=0]{}(0,0)(0.,0.)(6.000313706672241,10.401509941128166)
\pspolygon[linecolor=darkgray,fillcolor=darkgray,fillstyle=solid,opacity=0.1](1.,0.)(1.,3.)(2.,3.)(2.,0.)
\pspolygon[linecolor=darkgray,fillcolor=darkgray,fillstyle=solid,opacity=0.1](2.,0.)(2.,4.)(3.,4.)(3.,0.)
\pspolygon[linecolor=darkgray,fillcolor=darkgray,fillstyle=solid,opacity=0.1](3.,0.)(3.,3.)(4.,3.)(4.,0.)
\pspolygon[linecolor=darkgray,fillcolor=darkgray,fillstyle=solid,opacity=0.1](4.,0.)(4.,6.)(5.,6.)(5.,0.)
\pspolygon[linecolor=darkgray,fillcolor=darkgray,fillstyle=solid,opacity=0.1](5.,0.)(5.,5.)(6.,5.)(6.,0.)
\psline(1.,0.)(1.,3.)
\psline(1.,3.)(2.,3.)
\psline(2.,3.)(2.,0.)
\psline(2.,0.)(1.,0.)
\psline[linecolor=darkgray](1.,0.)(1.,3.)
\psline[linecolor=darkgray](1.,3.)(2.,3.)
\psline[linecolor=darkgray](2.,3.)(2.,0.)
\psline[linecolor=darkgray](2.,0.)(1.,0.)
\psline[linecolor=darkgray](2.,0.)(2.,4.)
\psline[linecolor=darkgray](2.,4.)(3.,4.)
\psline[linecolor=darkgray](3.,4.)(3.,0.)
\psline[linecolor=darkgray](3.,0.)(2.,0.)
\psline[linecolor=darkgray](3.,0.)(3.,3.)
\psline[linecolor=darkgray](3.,3.)(4.,3.)
\psline[linecolor=darkgray](4.,3.)(4.,0.)
\psline[linecolor=darkgray](4.,0.)(3.,0.)
\psline[linecolor=darkgray](4.,0.)(4.,6.)
\psline[linecolor=darkgray](4.,6.)(5.,6.)
\psline[linecolor=darkgray](5.,6.)(5.,0.)
\psline[linecolor=darkgray](5.,0.)(4.,0.)
\psline[linecolor=darkgray](5.,0.)(5.,5.)
\psline[linecolor=darkgray](5.,5.)(6.,5.)
\psline[linecolor=darkgray](5.9,5.)(5.9,0.)
\psline[linecolor=darkgray](6.,0.)(5.,0.)
\rput[tl](1.9132863389210606,-0.8051807938360951){erzielte Punkte}
\rput[tl](-1,7.713644783879816){$\rotatebox{90}{\text{Anzahl der SchülerInnen}}$}
\rput[tl](0.5,-0.2){0}
\rput[tl](1.5,-0.2){1}
\rput[tl](2.5,-0.2){2}
\rput[tl](3.5,-0.2){3}
\rput[tl](4.5,-0.2){4}
\rput[tl](5.5,-0.2){5}
\end{pspicture*}}
\end{center}
				\end{minipage}\leer
				
				\multiplechoice[6]{  %Anzahl der Antwortmoeglichkeiten, Standard: 5
								L1={\resizebox{0.5\linewidth}{!}{\newrgbcolor{uuuuuu}{0.26666666666666666 0.26666666666666666 0.26666666666666666}
\psset{xunit=1.0cm,yunit=1.0cm,algebraic=true,dimen=middle,dotstyle=o,dotsize=5pt 0,linewidth=0.8pt,arrowsize=3pt 2,arrowinset=0.25}
\begin{pspicture*}(-0.13420674082453773,-0.5427626108048722)(5.927808072632796,1.1570591105362065)
\psaxes[labelFontSize=\scriptstyle,xAxis=true,yAxis=false,Dx=1.,Dy=1.,ticksize=-2pt 0,subticks=2]{}(0,0)(0.,0.)(5.0,1.1570591105362065)
\psframe[linecolor=darkgray,fillcolor=darkgray,fillstyle=solid,opacity=0.1](2.,0.2)(4.5,0.6)
\psline[linecolor=darkgray,fillcolor=darkgray,fillstyle=solid,opacity=0.1](1.,0.2)(1.,0.6)
\psline[linecolor=darkgray,fillcolor=darkgray,fillstyle=solid,opacity=0.1](5.,0.2)(5.,0.6)
\psline[linecolor=darkgray,fillcolor=darkgray,fillstyle=solid,opacity=0.1](2.5,0.2)(2.5,0.6)
\psline[linecolor=darkgray,fillcolor=darkgray,fillstyle=solid,opacity=0.1](1.,0.4)(2.,0.4)
\psline[linecolor=darkgray,fillcolor=darkgray,fillstyle=solid,opacity=0.1](4.5,0.4)(5.,0.4)
\end{pspicture*}}},   %1. Antwortmoeglichkeit 
								L2={\resizebox{0.5\linewidth}{!}{\newrgbcolor{uuuuuu}{0.26666666666666666 0.26666666666666666 0.26666666666666666}
\psset{xunit=1.0cm,yunit=1.0cm,algebraic=true,dimen=middle,dotstyle=o,dotsize=5pt 0,linewidth=0.8pt,arrowsize=3pt 2,arrowinset=0.25}
\begin{pspicture*}(-0.13420674082453773,-0.5427626108048722)(5.927808072632796,1.1570591105362065)
\psaxes[labelFontSize=\scriptstyle,xAxis=true,yAxis=false,Dx=1.,Dy=1.,ticksize=-2pt 0,subticks=2]{}(0,0)(0.,0.)(5.0,1.1570591105362065)
\psframe[linecolor=darkgray,fillcolor=darkgray,fillstyle=solid,opacity=0.1](2.,0.2)(4.5,0.6)
\psline[linecolor=darkgray,fillcolor=darkgray,fillstyle=solid,opacity=0.1](1.,0.2)(1.,0.6)
\psline[linecolor=darkgray,fillcolor=darkgray,fillstyle=solid,opacity=0.1](5.,0.2)(5.,0.6)
\psline[linecolor=darkgray,fillcolor=darkgray,fillstyle=solid,opacity=0.1](3,0.2)(3,0.6)
\psline[linecolor=darkgray,fillcolor=darkgray,fillstyle=solid,opacity=0.1](1.,0.4)(2.,0.4)
\psline[linecolor=darkgray,fillcolor=darkgray,fillstyle=solid,opacity=0.1](4.5,0.4)(5.,0.4)
\end{pspicture*}}},   %2. Antwortmoeglichkeit
								L3={\resizebox{0.5\linewidth}{!}{\newrgbcolor{uuuuuu}{0.26666666666666666 0.26666666666666666 0.26666666666666666}
\psset{xunit=1.0cm,yunit=1.0cm,algebraic=true,dimen=middle,dotstyle=o,dotsize=5pt 0,linewidth=0.8pt,arrowsize=3pt 2,arrowinset=0.25}
\begin{pspicture*}(-0.13420674082453773,-0.5427626108048722)(5.927808072632796,1.1570591105362065)
\psaxes[labelFontSize=\scriptstyle,xAxis=true,yAxis=false,Dx=1.,Dy=1.,ticksize=-2pt 0,subticks=2]{}(0,0)(0.,0.)(5.0,1.1570591105362065)
\psframe[linecolor=darkgray,fillcolor=darkgray,fillstyle=solid,opacity=0.1](2.,0.2)(4.5,0.6)
\psline[linecolor=darkgray,fillcolor=darkgray,fillstyle=solid,opacity=0.1](1.,0.2)(1.,0.6)
\psline[linecolor=darkgray,fillcolor=darkgray,fillstyle=solid,opacity=0.1](5.,0.2)(5.,0.6)
\psline[linecolor=darkgray,fillcolor=darkgray,fillstyle=solid,opacity=0.1](4,0.2)(4,0.6)
\psline[linecolor=darkgray,fillcolor=darkgray,fillstyle=solid,opacity=0.1](1.,0.4)(2.,0.4)
\psline[linecolor=darkgray,fillcolor=darkgray,fillstyle=solid,opacity=0.1](4.5,0.4)(5.,0.4)
\end{pspicture*}}},   %3. Antwortmoeglichkeit
								L4={\resizebox{0.5\linewidth}{!}{\newrgbcolor{uuuuuu}{0.26666666666666666 0.26666666666666666 0.26666666666666666}
\psset{xunit=1.0cm,yunit=1.0cm,algebraic=true,dimen=middle,dotstyle=o,dotsize=5pt 0,linewidth=0.8pt,arrowsize=3pt 2,arrowinset=0.25}
\begin{pspicture*}(-0.13420674082453773,-0.5427626108048722)(5.927808072632796,1.1570591105362065)
\psaxes[labelFontSize=\scriptstyle,xAxis=true,yAxis=false,Dx=1.,Dy=1.,ticksize=-2pt 0,subticks=2]{}(0,0)(0.,0.)(5.0,1.1570591105362065)
\psframe[linecolor=darkgray,fillcolor=darkgray,fillstyle=solid,opacity=0.1](1.,0.2)(5,0.6)
\psline[linecolor=darkgray,fillcolor=darkgray,fillstyle=solid,opacity=0.1](3,0.2)(3,0.6)
\end{pspicture*}}},   %4. Antwortmoeglichkeit
								L5={\resizebox{0.5\linewidth}{!}{\newrgbcolor{uuuuuu}{0.26666666666666666 0.26666666666666666 0.26666666666666666}
\psset{xunit=1.0cm,yunit=1.0cm,algebraic=true,dimen=middle,dotstyle=o,dotsize=5pt 0,linewidth=0.8pt,arrowsize=3pt 2,arrowinset=0.25}
\begin{pspicture*}(-0.13420674082453773,-0.5427626108048722)(5.927808072632796,1.1570591105362065)
\psaxes[labelFontSize=\scriptstyle,xAxis=true,yAxis=false,Dx=1.,Dy=1.,ticksize=-2pt 0,subticks=2]{}(0,0)(0.,0.)(5.0,1.1570591105362065)
\psframe[linecolor=darkgray,fillcolor=darkgray,fillstyle=solid,opacity=0.1](3.5,0.2)(4.5,0.6)
\psline[linecolor=darkgray,fillcolor=darkgray,fillstyle=solid,opacity=0.1](1.,0.2)(1.,0.6)
\psline[linecolor=darkgray,fillcolor=darkgray,fillstyle=solid,opacity=0.1](5.,0.2)(5.,0.6)
\psline[linecolor=darkgray,fillcolor=darkgray,fillstyle=solid,opacity=0.1](4,0.2)(4,0.6)
\psline[linecolor=darkgray,fillcolor=darkgray,fillstyle=solid,opacity=0.1](1.,0.4)(3.5,0.4)
\psline[linecolor=darkgray,fillcolor=darkgray,fillstyle=solid,opacity=0.1](4.5,0.4)(5.,0.4)
\end{pspicture*}}},	 %5. Antwortmoeglichkeit
								L6={\resizebox{0.5\linewidth}{!}{\newrgbcolor{uuuuuu}{0.26666666666666666 0.26666666666666666 0.26666666666666666}
\psset{xunit=1.0cm,yunit=1.0cm,algebraic=true,dimen=middle,dotstyle=o,dotsize=5pt 0,linewidth=0.8pt,arrowsize=3pt 2,arrowinset=0.25}
\begin{pspicture*}(-0.13420674082453773,-0.5427626108048722)(5.927808072632796,1.1570591105362065)
\psaxes[labelFontSize=\scriptstyle,xAxis=true,yAxis=false,Dx=1.,Dy=1.,ticksize=-2pt 0,subticks=2]{}(0,0)(0.,0.)(5.0,1.1570591105362065)
\psframe[linecolor=darkgray,fillcolor=darkgray,fillstyle=solid,opacity=0.1](3.5,0.2)(5,0.6)
\psline[linecolor=darkgray,fillcolor=darkgray,fillstyle=solid,opacity=0.1](1.,0.2)(1.,0.6)
\psline[linecolor=darkgray,fillcolor=darkgray,fillstyle=solid,opacity=0.1](4.,0.2)(4.,0.6)
\psline[linecolor=darkgray,fillcolor=darkgray,fillstyle=solid,opacity=0.1](1.,0.4)(3.5,0.4)
\end{pspicture*}}},	 %6. Antwortmoeglichkeit
								L7={},	 %7. Antwortmoeglichkeit
								L8={},	 %8. Antwortmoeglichkeit
								L9={},	 %9. Antwortmoeglichkeit
								%% LOESUNG: %%
								A1=3,  % 1. Antwort
								A2=0,	 % 2. Antwort
								A3=0,  % 3. Antwort
								A4=0,  % 4. Antwort
								A5=0,  % 5. Antwort
								}
\end{beispiel}