\section{WS 2.3 - 17 - MAT - Alarmanlagen - OA - Matura HT 2016/17}

\begin{beispiel}[WS 2.3]{1} %PUNKTE DES BEISPIELS
Eine bestimmte Alarmanlage löst jeweils mit der Wahrscheinlichkeit 0,9 im Einbruchsfall Alarm aus. Eine Familie lässt zwei dieser Anlagen in ihr Haus so einbauen, dass sie unabhängig voneinander Alarm auslösen. \leer

Berechne die Wahrscheinlichkeit, dass im Einbruchsfall mindestens eine der beiden Anlagen
Alarm auslöst!	

\antwort{Mögliche Berechnung: 

$1-0,1^2=0,99$ \leer

Die Wahrscheinlichkeit, dass im Einbruchsfall mindestens eine der beiden Anlagen Alarm auslöst,
liegt bei 0,99.}	
\end{beispiel}
