\section{WS 2.3 - 2 Wahrscheinlichkeit eines Defekts - OA - BIFIE}

\begin{beispiel}[WS 2.3]{1}
Eine Maschine besteht aus den drei Bauteilen A, B und C. Diese haben die im nachstehenden Modell eingetragenen, voneinander unabh�ngigen Defekth�ufigkeiten. Eine Maschine ist defekt,
wenn mindestens ein Bauteil defekt ist.
\leer

\resizebox{1.0\linewidth}{!}{\Huge
\psset{xunit=1.0cm,yunit=1.0cm,algebraic=true,dimen=middle,dotstyle=o,dotsize=5pt 0,linewidth=0.8pt,arrowsize=3pt 2,arrowinset=0.25}
\begin{pspicture*}(-21.136769952082343,-3.0839761805811605)(14.844736533697363,21.19166204872038)
\psline(-2.,20.)(-10.,14.)
\psline(-10.,14.)(-14.,10.)
\psline(-14.,10.)(-16.,6.)
\psline(6.,14.)(2.,10.)
\psline(2.,10.)(0.,6.)
\psline(10.,10.)(8.,6.)
\psline(-6.,10.)(-8.,6.)
\psline[linestyle=dashed,dash=11pt 11pt](-2.,20.)(6.,14.)
\psline[linestyle=dashed,dash=11pt 11pt](6.,14.)(10.,10.)
\psline[linestyle=dashed,dash=11pt 11pt](10.,10.)(12.,6.)
\psline[linestyle=dashed,dash=11pt 11pt](2.,10.)(4.,6.)
\psline[linestyle=dashed,dash=11pt 11pt](-6.,10.)(-4.,6.)
\psline[linestyle=dashed,dash=11pt 11pt](-14.,10.)(-12.,6.)
\psline[linestyle=dashed,dash=11pt 11pt](-10.,14.)(-6.,10.)
\psline(-20.,20.)(-18.,20.)
\psline(-20.,14.)(-18.,14.)
\psline(-20.,10.)(-18.,10.)
\psline(-20.,6.)(-18.,6.)
\rput[tl](-19.23618617612279,17.26090923934949){A}
\rput[tl](-19.322576347757316,12.250279284547037){B}
\rput[tl](-19.322576347757316,8.276331389358885){C}
\rput[tl](-8.22143929272084,18.5){$\dfrac{95}{100}$}
\rput[tl](2.7933075906811116,18.5){$\dfrac{5}{100}$}
\rput[tl](-14,13.5){$\dfrac{9}{10}$}
\rput[tl](-16.5,9.4){$\dfrac{8}{10}$}
\rput[tl](-7.098367061472013,13.5){$\dfrac{1}{10}$}
\rput[tl](-12.3,9.4){$\dfrac{2}{10}$}
\rput[tl](-8.826170494162515,9.4){$\dfrac{8}{10}$}
\rput[tl](-4.5,9.4){$\dfrac{2}{10}$}
\rput[tl](2.2,13.5){$\dfrac{9}{10}$}
\rput[tl](8.667839261828819,13.5){$\dfrac{1}{10}$}
\rput[tl](-0.7054943605171552,9.4){$\dfrac{8}{10}$}
\rput[tl](7.285596515676416,9.4){$\dfrac{8}{10}$}
\rput[tl](3.5,9.4){$\dfrac{2}{10}$}
\rput[tl](11.6,9.4){$\dfrac{2}{10}$}
\rput[tl](10.266057437067532,20.241370160740605){Teil fehlerfrei}
\psline(8.,20.)(10.,20.)
\psline[linestyle=dashed,dash=11pt 11pt](8.,18.)(10.,18.)
\rput[tl](10.438837780336582,18.29759129896379){Teil defekt}
\begin{scriptsize}
\psdots[dotsize=3pt 0,dotstyle=*](-2.,20.)
\psdots[dotsize=3pt 0,dotstyle=*](-10.,14.)
\psdots[dotsize=3pt 0,dotstyle=*](6.,14.)
\psdots[dotsize=3pt 0,dotstyle=*](-6.,10.)
\psdots[dotsize=3pt 0,dotstyle=*](-14.,10.)
\psdots[dotsize=3pt 0,dotstyle=*](2.,10.)
\psdots[dotsize=3pt 0,dotstyle=*](10.,10.)
\psdots[dotsize=3pt 0,dotstyle=*](-8.,6.)
\psdots[dotsize=3pt 0,dotstyle=*](-4.,6.)
\psdots[dotsize=3pt 0,dotstyle=*](-16.,6.)
\psdots[dotsize=3pt 0,dotstyle=*](-12.,6.)
\psdots[dotsize=3pt 0,dotstyle=*](0.,6.)
\psdots[dotsize=3pt 0,dotstyle=*](4.,6.)
\psdots[dotsize=3pt 0,dotstyle=*](8.,6.)
\psdots[dotsize=3pt 0,dotstyle=*](12.,6.)
\end{scriptsize}
\end{pspicture*}}

Berechne die Wahrscheinlichkeit, dass bei einer Maschine zwei oder mehr Bauteile defekt sind. \leer

$P(X\geq2)=$ \rule{6cm}{0.3pt}


\antwort{\leer

$P(X\geq2)=\frac{95}{100} \cdot \frac{1}{10} \cdot \frac{1}{5} + \frac{5}{100} \cdot \frac{9}{10} \cdot \frac{1}{5} + \frac{5}{100} \cdot \frac{1}{10} \cdot \frac{4}{5} + \frac{5}{100} \cdot \frac{1}{10} \cdot \frac{1}{5}=\frac{33}{1000}=0,033$}
\end{beispiel}