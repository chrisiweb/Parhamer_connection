\section{WS 2.3 - 9 Verschiedenfärbige Kugeln - MC - Matura 2015/16 - Haupttermin}

\begin{beispiel}[WS 2.3]{1} %PUNKTE DES BEISPIELS
Auf einem Tisch steht eine Schachtel mit drei roten und zwölf schwarzen Kugeln. Nach dem
Zufallsprinzip werden nacheinander drei Kugeln aus der Schachtel gezogen, wobei die gezogene
Kugel jeweils wieder zurückgelegt wird. \leer

Gegeben ist der folgende Ausdruck:

$3\cdot 0,8^2 \cdot 0,2$ \leer

Kreuze dasjenige Ereignis an, dessen Wahrscheinlichkeit durch diesen Ausdruck berechnet
wird.

\multiplechoice[6]{  %Anzahl der Antwortmoeglichkeiten, Standard: 5
				L1={Es wird höchstens eine schwarze Kugel gezogen.},   %1. Antwortmoeglichkeit 
				L2={Es werden genau zwei schwarze Kugeln gezogen.},   %2. Antwortmoeglichkeit
				L3={Es werden zwei rote Kugeln und eine schwarze Kugel gezogen.},   %3. Antwortmoeglichkeit
				L4={Es werden nur rote Kugeln gezogen.},   %4. Antwortmoeglichkeit
				L5={Es wird mindestens eine rote Kugel gezogen.},	 %5. Antwortmoeglichkeit
				L6={Es wird keine rote Kugel gezogen.},	 %6. Antwortmoeglichkeit
				L7={},	 %7. Antwortmoeglichkeit
				L8={},	 %8. Antwortmoeglichkeit
				L9={},	 %9. Antwortmoeglichkeit
				%% LOESUNG: %%
				A1=2,  % 1. Antwort
				A2=0,	 % 2. Antwort
				A3=0,  % 3. Antwort
				A4=0,  % 4. Antwort
				A5=0,  % 5. Antwort
				}

\end{beispiel}