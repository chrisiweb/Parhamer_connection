\section{WS 2.3 - 12 Augensumme beim Würfeln - OA - Matura 2014/15 - Nebentermin 1}

\begin{beispiel}[WS 2.3]{1}
Zwei unterscheidbare, faire Würfel mit den Augenzahlen 1, 2, 3, 4, 5, 6 werden gleichzeitig geworfen und die Augensumme wird ermittelt. Das Ereignis, dass die Augensumme durch 5 teilbar
ist, wird mit E bezeichnet. (Ein Würfel ist "`fair"', wenn die Wahrscheinlichkeit, nach einem Wurf nach oben zu zeigen, für alle sechs Seitenflächen gleich groß ist.) \leer

Berechne die Wahrscheinlichkeit des Ereignisses $E$. \leer

$P(E)= \rule{5cm}{0.3pt}$

\antwort{
$P(E)=\dfrac{7}{36}$ \leer

Lösungsschlüssel:\\
Ein Punkt für die richtige Lösung. Andere Schreibweisen des Ergebnisses (als Dezimalzahl oder in Prozent) sind ebenfalls als richtig zu werten.\\
Toleranzintervalle: $[0,19; 0,20]$ bzw. $[19\%; 20\%]$}
\end{beispiel}