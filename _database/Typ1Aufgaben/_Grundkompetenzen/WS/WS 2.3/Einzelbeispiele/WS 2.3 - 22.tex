\section{WS 2.3 - 22 - MAT - Jetons - OA - Matura 2. NT 2017/18}

\begin{beispiel}[WS 2.3]{1}
In zwei Schachteln befindet sich Spielgeld.

In Schachtel I sind fünf 2-Euro-Jetons und zwei 1-Euro-Jetons.

In Schachtel II sind vier 2-Euro-Jetons und fünf 1-Euro-Jetons.

Aus jeder der beiden Schachteln wird unabhängig voneinander je ein Jeton entnommen. Dabei
hat pro Schachtel jeder Jeton die gleiche Wahrscheinlichkeit, entnommen zu werden.

Berechne die Wahrscheinlichkeit, dass nach der Entnahme der beiden Jetons in beiden
Schachteln der gleiche Geldbetrag vorhanden ist!

\antwort{Mögliche Vorgehensweise:

$\frac{2}{7}\cdot \frac{4}{9}\approx 0,127$

Die Wahrscheinlichkeit, dass nach der Entnahme der beiden Jetons in beiden Schachteln der
gleiche Geldbetrag (11 Euro) vorhanden ist, beträgt ca. 12,7\,\%.

Toleranzintervall: [0,12; 0,13]}
\end{beispiel}