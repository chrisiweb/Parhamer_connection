\section{WS 2.3 - 21 - Rot-Gr�n-Sehschw�che - OA - Matura - 1. NT 2017/18}

\begin{beispiel}[WS 2.3]{1}
Eine der bekanntesten Farbfehlsichtigkeiten ist die Rot-Gr�n-Sehschw�che. Wenn jemand davon betroffen ist, dann ist diese Fehlsichtigkeit immer angeboren und verst�rkt oder vermindert sich nicht im Laufe der Zeit. Von ihr sind weltweit etwa $9\,\%$ aller M�nner und etwa $0,8\,\%$ aller Frauen betroffen. Der Anteil von Frauen an der Weltbev�lkerung liegt bei $50,5\,\%$.

Gib die Wahrscheinlichkeit an, dass eine nach dem Zufallsprinzip ausgew�hlte Person eine Rot-Gr�n-Sehschw�che hat!

\antwort{$0,495\cdot 0,09+0,505\cdot 0,008\approx 0,049$

Toleranzintervall: $[0,04; 0,05]$}
\end{beispiel}