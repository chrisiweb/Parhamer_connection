\section{WS 2.3 - 8 - MAT - Zollkontrolle - OA - Matura HT 2015/16}

\begin{beispiel}[WS 2.3]{1} %PUNKTE DES BEISPIELS
Eine Gruppe von zehn Personen überquert eine Grenze zwischen zwei Staaten. Zwei Personen
führen Schmuggelware mit sich. Beim Grenzübertritt werden drei Personen vom Zoll zufällig ausgewählt und kontrolliert.

Berechne die Wahrscheinlichkeit, dass unter den drei kontrollierten Personen die beiden
Schmuggler der Gruppe sind!

\antwort{
$\frac{2}{10} \cdot \frac{1}{9} \cdot 3 =\frac{1}{15}$ \leer

Lösungsschlüssel:

Ein Punkt für die richtige Lösung. Andere Schreibweisen des Ergebnisses (als Dezimalzahl oder in
Prozent) sind ebenfalls als richtig zu werten.\\
Toleranzintervall: $[0,066; 0,07]$ bzw. $[6,6\,\%; 7\,\%]$}
\end{beispiel}