\section{WS 2.3 - 13 Maturaball-Gl�cksspiele - OA - Matura 2014/15 - Nebentermin 2}

\begin{beispiel}[WS 2.3]{1} %PUNKTE DES BEISPIELS
				Bei einem Maturaball werden zwei verschiedene Gl�cksspiele angeboten: ein Gl�cksrad und eine Tombola, bei der $1000$ Lose verkauft werden. Das Gl�cksrad ist in zehn gleich gro�en Sektoren unterteilt, die alle mit der gleichen Wahrscheinlichkeit auftreten k�nnen. Man gewinnt, wenn der Zeiger nach Stillstand des Rades auf das Feld der "`$1$"' oder der "`$6$"' zeigt.
				
				Max hat das Gl�cksrad einmal gedreht und als Erster ein Los der Tombola gekauft. In beiden F�llen hat er gewonnen. Die Maturazeitung berichtet dar�ber: "`Die Wahrscheinlichkeit f�r dieses Ereignis betr�gt $3\,\%$"'. Berechne die Anzahl der Gewinn-Lose.\\
				
				\antwort{$\frac{2}{10}\cdot\frac{x}{1000}=0,03\Rightarrow x=150$.
				
				Es gibt 150 Gewinnlose.}
\end{beispiel}