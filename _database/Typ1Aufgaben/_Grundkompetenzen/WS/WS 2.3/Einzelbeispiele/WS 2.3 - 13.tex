\section{WS 2.3 - 13 - MAT - Maturaball-Glücksspiele - OA - Matura 2. NT 2014/15}

\begin{beispiel}[WS 2.3]{1} %PUNKTE DES BEISPIELS
				Bei einem Maturaball werden zwei verschiedene Glücksspiele angeboten: ein Glücksrad und eine Tombola, bei der $1000$ Lose verkauft werden. Das Glücksrad ist in zehn gleich großen Sektoren unterteilt, die alle mit der gleichen Wahrscheinlichkeit auftreten können. Man gewinnt, wenn der Zeiger nach Stillstand des Rades auf das Feld der "`$1$"' oder der "`$6$"' zeigt.
				
				Max hat das Glücksrad einmal gedreht und als Erster ein Los der Tombola gekauft. In beiden Fällen hat er gewonnen. Die Maturazeitung berichtet darüber: "`Die Wahrscheinlichkeit für dieses Ereignis beträgt $3\,\%$"'. Berechne die Anzahl der Gewinn-Lose.\\
				
				\antwort{$\frac{2}{10}\cdot\frac{x}{1000}=0,03\Rightarrow x=150$.
				
				Es gibt 150 Gewinnlose.}
\end{beispiel}