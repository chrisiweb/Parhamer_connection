\section{WS 2.3 - 19 - MAT - Prüfung - OA - Matura 2. NT 2016/17}

\begin{beispiel}[WS 2.3]{1} %PUNKTE DES BEISPIELS
Um ein Stipendium für einen Auslandsaufenthalt zu erhalten, mussten Studierende entweder in Spanisch oder in Englisch eine Prüfung ablegen.

Im nachstehenden Baumdiagramm sind die Anteile der Studierenden, die sich dieser Prüfung in der jeweiligen Sprache unterzogen haben, angeführt. Zudem gibt das Baumdiagramm Auskunft über die Anteile der positiven bzw. negativen Prüfungsergebnisse.

\resizebox{1\linewidth}{!}{\psset{xunit=1.0cm,yunit=1.0cm,algebraic=true,dimen=middle,dotstyle=o,dotsize=5pt 0,linewidth=0.8pt,arrowsize=3pt 2,arrowinset=0.25}
\begin{pspicture*}(-0.72,-1.8)(18,6.14)
\psline(8.,5.)(3.,3.)
\psline(8.,5.)(13.,3.)
\psline(13.,2.)(11.,0.)
\psline(13.,2.)(15.,0.)
\psline(3.,2.)(5.,0.)
\psline(3.,2.)(1.,0.)
\rput[tl](5,4.48){0,3}
\rput[tl](10.7,4.48){0,7}
\rput[tl](14.46,1.38){0,1}
\rput[tl](11.3,1.36){0,9}
\rput[tl](12,2.75){$\framebox{Englisch}$}
\rput[tl](1.7,2.75){$\framebox{Spanisch}$}
\rput[tl](3.9,-0.2){$\framebox{negativ}$}
\rput[tl](-0.08,-0.2){$\framebox{positiv}$}
\rput[tl](9.78,-0.2){$\framebox{positiv}$}
\rput[tl](14.12,-0.18){$\framebox{negativ}$}
\rput[tl](4.14,1.36){0,2}
\rput[tl](1.3,1.4){0,8}
\end{pspicture*}}

Der Prüfungsakt einer/eines angetretenen Studierenden wird zufällig ausgewählt.

Deute den Ausdruck $0,7\cdot 0,9+(1-0,7)\cdot 0,8$ im gegebenen Kontext!\leer

\antwort{Der Ausdruck beschreibt die Wahrscheinlichkeit, dass der zufällig ausgewählte Prüfungsakt ein positives Prüfungsergebnis aufweist.}
\end{beispiel}