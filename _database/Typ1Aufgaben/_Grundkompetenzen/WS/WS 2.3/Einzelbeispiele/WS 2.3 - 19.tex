\section{WS 2.3 - 19 - MAT - Pr�fung - OA - Matura 2016/17 2. NT}

\begin{beispiel}[WS 2.3]{1} %PUNKTE DES BEISPIELS
Um ein Stipendium f�r einen Auslandsaufenthalt zu erhalten, mussten Studierende entweder in Spanisch oder in Englisch eine Pr�fung ablegen.

Im nachstehenden Baumdiagramm sind die Anteile der Studierenden, die sich dieser Pr�fung in der jeweiligen Sprache unterzogen haben, angef�hrt. Zudem gibt das Baumdiagramm Auskunft �ber die Anteile der positiven bzw. negativen Pr�fungsergebnisse.

\resizebox{1\linewidth}{!}{\psset{xunit=1.0cm,yunit=1.0cm,algebraic=true,dimen=middle,dotstyle=o,dotsize=5pt 0,linewidth=0.8pt,arrowsize=3pt 2,arrowinset=0.25}
\begin{pspicture*}(-0.72,-1.8)(18,6.14)
\psline(8.,5.)(3.,3.)
\psline(8.,5.)(13.,3.)
\psline(13.,2.)(11.,0.)
\psline(13.,2.)(15.,0.)
\psline(3.,2.)(5.,0.)
\psline(3.,2.)(1.,0.)
\rput[tl](5,4.48){0,3}
\rput[tl](10.7,4.48){0,7}
\rput[tl](14.46,1.38){0,1}
\rput[tl](11.3,1.36){0,9}
\rput[tl](12,2.75){$\framebox{Englisch}$}
\rput[tl](1.7,2.75){$\framebox{Spanisch}$}
\rput[tl](3.9,-0.2){$\framebox{negativ}$}
\rput[tl](-0.08,-0.2){$\framebox{positiv}$}
\rput[tl](9.78,-0.2){$\framebox{positiv}$}
\rput[tl](14.12,-0.18){$\framebox{negativ}$}
\rput[tl](4.14,1.36){0,2}
\rput[tl](1.3,1.4){0,8}
\end{pspicture*}}

Der Pr�fungsakt einer/eines angetretenen Studierenden wird zuf�llig ausgew�hlt.

Deute den Ausdruck $0,7\cdot 0,9+(1-0,7)\cdot 0,8$ im gegebenen Kontext!\leer

\antwort{Der Ausdruck beschreibt die Wahrscheinlichkeit, dass der zuf�llig ausgew�hlte Pr�fungsakt ein positives Pr�fungsergebnis aufweist.}
\end{beispiel}