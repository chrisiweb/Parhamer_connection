\section{WS 2.3 - 6 - Laplace-Wahrscheinlichkeit - MC - BIFIE}

\begin{beispiel}[WS 2.3]{1}
In einer Schachtel befinden sich ein roter, ein blauer und ein gelber Wachsmalstift. Ein Stift wird zufällig entnommen, dessen Farbe notiert und der Stift danach zurückgelegt. Dann wird das
Experiment wiederholt.

Beobachtet wird, wie oft bei zweimaligem Ziehen ein gelber Stift entnommen wurde. Die Werte der Zufallsvariablen $X$ beschreiben die Anzahl der gezogenen gelben Stifte.

Kreuze die zutreffende(n) Aussage(n) an.

\multiplechoice[5]{  %Anzahl der Antwortmoeglichkeiten, Standard: 5
				L1={$P(X=0)>P(X=1)$},   %1. Antwortmoeglichkeit 
				L2={$P(X=2)=\frac{1}{9}$},   %2. Antwortmoeglichkeit
				L3={$P(X\leq2)=\frac{8}{9}$},   %3. Antwortmoeglichkeit
				L4={$P(X>0)=\frac{5}{9}$},   %4. Antwortmoeglichkeit
				L5={$P(X<3)=1$},	 %5. Antwortmoeglichkeit
				L6={},	 %6. Antwortmoeglichkeit
				L7={},	 %7. Antwortmoeglichkeit
				L8={},	 %8. Antwortmoeglichkeit
				L9={},	 %9. Antwortmoeglichkeit
				%% LOESUNG: %%
				A1=2,  % 1. Antwort
				A2=4,	 % 2. Antwort
				A3=5,  % 3. Antwort
				A4=0,  % 4. Antwort
				A5=0,  % 5. Antwort
				}
\end{beispiel}