\section{WS 2.3 - 14 Einlasskontrolle - OA - Matura
2015/16 - Nebentermin 1}

\begin{beispiel}{1} %PUNKTE DES BEISPIELS
Beim Einlass zu einer Sportveranstaltung f�hrt eine Person $P$ einen unerlaubten Gegenstand mit
sich. Bei einer Sicherheitskontrolle wird ein unerlaubter Gegenstand mit einer Wahrscheinlichkeit
von $0,9$ entdeckt. Da es sich bei dieser Sportveranstaltung um eine Veranstaltung mit besonders
hohem Risiko handelt, muss jede Person zwei derartige voneinander unabh�ngige Sicherheitskontrollen
durchlaufen. \leer

Berechne die Wahrscheinlichkeit, dass bei der Person $P$ im Zuge der beiden Sicherheitskontrollen
der unerlaubte Gegenstand entdeckt wird.

\antwort{$0,9+0,1\cdot 0,9=0,99$}

\end{beispiel}