\section{WS 2.3 - 24 - MAT - Ziehungswahrscheinlichkeit - OA - Matura 1.NT 2018/19}

\begin{beispiel}[WS 2.3]{1}
In einem Behälter befinden sich fünf Kugeln. Zwei Kugeln werden nacheinander ohne Zurücklegen gezogen (dabei wird angenommen, dass jede Ziehung von zwei Kugeln die gleiche Wahrscheinlichkeit hat). Zwei der fünf Kugeln im Behälter sind blau, die anderen Kugeln sind rot. Mit $p$ wird die Wahrscheinlichkeit bezeichnet, beim zweiten Zug eine blaue Kugel zu ziehen.

Gib die Wahrscheinlichkeit $p$ an.\leer

$p=\antwort[\rule{5cm}{0.3pt}]{\frac{2}{5}\cdot\frac{1}{4}+\frac{3}{5}\cdot\frac{1}{2}=\frac{2}{5}}$
\end{beispiel}