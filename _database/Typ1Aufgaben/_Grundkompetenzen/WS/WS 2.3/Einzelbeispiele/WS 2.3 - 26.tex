\section{WS 2.3 - 26 - MAT - Lieblingsfach - OA - Matura 2019/20 1. HT}

\begin{beispiel}[WS 2.3]{1}
Alle Schulkinder der 1. und der 2. Klassen einer Schule wurden nach ihrem Lieblingsfach befragt. Bei dieser Befragung war genau ein Lieblingsfach anzugeben. Die nachstehende Tabelle fasst die erhobenen Daten zusammen.
\begin{center}
\begin{tabular}{|p{5cm}|C{2cm}|C{2cm}|}\cline{2-3}
\multicolumn{1}{c|}{}&\multicolumn{1}{|c|}{\cellcolor[gray]{0.9}Lieblingsfach Mathematik}&\multicolumn{1}{|c|}{\cellcolor[gray]{0.9}anderes Lieblingsfach}\\ \hline
\cellcolor[gray]{0.9}Schulkinder der 1. Klasse&47&241\\ \hline
\cellcolor[gray]{0.9}Schulkinder der 2. Klasse&33&287\\ \hline
\end{tabular}\end{center}

Ein Schulkind der 1. Klassen wird zufällig ausgewählt. (Dabei haben alle Schulkinder der 1. Klassen die gleiche Wahrscheinlichkeit, ausgewählt zu werden.)

Berechne die Wahrscheinlichkeit, dass dieses Schulkind Mathematik als Lieblingsfach angegeben hat.

\antwort{$\dfrac{47}{47+241}=\dfrac{47}{288}=0,1631\ldots\approx 0,163$

Die Wahrscheinlichkeit, dass dieses Schulkind Mathematik als Lieblingsfach angegeben hat, beträgt ca. 16,3\,\%}
\end{beispiel}