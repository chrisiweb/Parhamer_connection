\section{WS 2.3 - 11 - MAT - Mehrere Wahrscheinlichkeiten - MC - Matura HT 2014/15}

\begin{beispiel}[WS 2.3]{1} %PUNKTE DES BEISPIELS
In einer Unterrichtsstunde sind 15 Schülerinnen und 10 Schüler anwesend. Die Lehrperson wählt
für Überprüfungen nacheinander zufällig drei verschiedene Personen aus dieser Schulklasse aus.
Jeder Prüfling wird nur einmal befragt. \leer

Kreuze die beiden zutreffenden Aussagen an.

\multiplechoice[5]{  %Anzahl der Antwortmoeglichkeiten, Standard: 5
				L1={Die Wahrscheinlichkeit, dass die Lehrperson drei Schülerinnen
auswählt, kann mittels $\frac{15}{25}\cdot \frac{14}{25} \cdot \frac{13}{25}$ berechnet werden.},   %1. Antwortmoeglichkeit 
				L2={Die Wahrscheinlichkeit, dass die Lehrperson als erste Person
einen Schüler auswählt, ist $\frac{10}{25}$.},   %2. Antwortmoeglichkeit
				L3={Die Wahrscheinlichkeit, dass die Lehrperson bei der Wahl von drei
Prüflingen als zweite Person eine Schülerin auswählt, ist $\frac{24}{25}$.},   %3. Antwortmoeglichkeit
				L4={Die Wahrscheinlichkeit, dass die Lehrperson drei Schüler auswählt,
kann mittels $\frac{10}{25}\cdot \frac{9}{24} \cdot \frac{8}{23}$ berechnet werden.},   %4. Antwortmoeglichkeit
				L5={Die Wahrscheinlichkeit, dass sich unter den von der Lehrperson
ausgewählten Personen genau zwei Schülerinnen befinden, kann
mittels $\frac{15}{25}\cdot \frac{14}{24} \cdot \frac{23}{23}$ berechnet werden.},	 %5. Antwortmoeglichkeit
				L6={},	 %6. Antwortmoeglichkeit
				L7={},	 %7. Antwortmoeglichkeit
				L8={},	 %8. Antwortmoeglichkeit
				L9={},	 %9. Antwortmoeglichkeit
				%% LOESUNG: %%
				A1=2,  % 1. Antwort
				A2=4,	 % 2. Antwort
				A3=0,  % 3. Antwort
				A4=0,  % 4. Antwort
				A5=0,  % 5. Antwort
				}
\end{beispiel}