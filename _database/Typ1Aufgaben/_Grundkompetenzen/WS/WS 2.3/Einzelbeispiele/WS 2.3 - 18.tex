\section{WS 2.3 - 18 - MAT - Mensch ärgere Dich nicht - OA - Matura 1. NT 2016/17}

\begin{beispiel}[WS 2.3]{1} %PUNKTE DES BEISPIELS
Um beim Spiel \textit{Mensch ärgere Dich nicht} zu Beginn des Spiels eine Figur auf das Spielfeld setzen zu dürfen, muss mit einem fairen Spielwürfel ein Sechser geworfen werden. (Ein Würfel ist "`fair"', wenn die Wahrscheinlichkeit, nach einem Wurf nach oben zu zeigen für alle sechs Seitenflächen gleich groß ist.)

Die Anzahl der Versuche, einen Sechser zu werfen, ist laut Spielanleitung auf der Versuche beschränkt, bevor die nächste Spielerin/der nächste Spieler an die Reihe kommt.

Berechne die Wahrscheinlichkeit, mit der eine Spielfigur nach maximal drei Versuchen, einen Sechser zur werfen, auf das Spielfeld gesetzt werden darf!

\antwort{$\frac{1}{6}+\frac{5}{6}\cdot\frac{1}{6}+\frac{5}{6}\cdot\frac{5}{6}\cdot\frac{1}{6}\approx 0,42$

Die Wahrscheinlichkeit, eine Spielfigur nach maximal drei Versuchen auf das Spielfeld setzen zu dürfen, beträgt ca. $42\,\%$.

Toleranzintervall: $[0,4;0,45]$ bzw. $[40\,\%;45\,\%]$

Die Aufgabe ist auch dann als richtig gelöst zu werten, wenn bei korrektem Ansatz das Ergebnis aufgrund eines Rechenfehlers nicht richtig ist.}
\end{beispiel}