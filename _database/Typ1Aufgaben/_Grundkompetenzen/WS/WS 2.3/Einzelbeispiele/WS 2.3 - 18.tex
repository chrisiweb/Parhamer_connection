\section{WS 2.3 - 18 Mensch �rgere Dich nicht - OA - Matura NT 1 16/17}

\begin{beispiel}[WS 2.3]{1} %PUNKTE DES BEISPIELS
Um beim Spiel \textit{Mensch �rgere Dich nicht} zu Beginn des Spiels eine Figur auf das Spielfeld setzen zu d�rfen, muss mit einem fairen Spielw�rfel ein Sechser geworfen werden. (Ein W�rfel ist "`fair"', wenn die Wahrscheinlichkeit, nach einem Wurf nach oben zu zeigen f�r alle sechs Seitenfl�chen gleich gro� ist.)

Die Anzahl der Versuche, einen Sechser zu werfen, ist laut Spielanleitung auf der Versuche beschr�nkt, bevor die n�chste Spielerin/der n�chste Spieler an die Reihe kommt.

Berechne die Wahrscheinlichkeit, mit der eine Spielfigur nach maximal drei Versuchen, einen Sechser zur werfen, auf das Spielfeld gesetzt werden darf!

\antwort{$\frac{1}{6}+\frac{5}{6}\cdot\frac{1}{6}+\frac{5}{6}\cdot\frac{5}{6}\cdot\frac{1}{6}\approx 0,42$

Die Wahrscheinlichkeit, eine Spielfigur nach maximal drei Versuchen auf das Spielfeld setzen zu d�rfen, betr�gt ca. $42\,\%$.

Toleranzintervall: $[0,4;0,45]$ bzw. $[40\,\%;45\,\%]$

Die Aufgabe ist auch dann als richtig gel�st zu werten, wenn bei korrektem Ansatz das Ergebnis aufgrund eines Rechenfehlers nicht richtig ist.}
\end{beispiel}