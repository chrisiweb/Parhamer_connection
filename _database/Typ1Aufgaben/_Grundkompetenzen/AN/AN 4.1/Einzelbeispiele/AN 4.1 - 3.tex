\section{AN 4.1 - 3 - MAT - Bestimmtes Integral - MC - Matura 2016/17 2. NT}

\begin{beispiel}[AN 4.1]{1} %PUNKTE DES BEISPIELS
Der Graph einer Funktion $f$ schneidet die $x$-Achse in einem gewissen Bereich an den Stellen $a,b,c,d$ und $e$.

\begin{center}
	\resizebox{0.8\linewidth}{!}{\psset{xunit=1.0cm,yunit=1.0cm,algebraic=true,dimen=middle,dotstyle=o,dotsize=5pt 0,linewidth=1.6pt,arrowsize=3pt 2,arrowinset=0.25}
\begin{pspicture*}(-0.94,-1.5)(13.44,3.78)
\multips(0,-1)(0,1.0){6}{\psline[linestyle=dashed,linecap=1,dash=1.5pt 1.5pt,linewidth=0.4pt,linecolor=lightgray]{c-c}(-0.94,0)(13.44,0)}
\multips(0,0)(1.0,0){15}{\psline[linestyle=dashed,linecap=1,dash=1.5pt 1.5pt,linewidth=0.4pt,linecolor=lightgray]{c-c}(0,-1.5)(0,3.78)}
\psaxes[labelFontSize=\scriptstyle,xAxis=true,yAxis=true,Dx=1.,Dy=1.,ticksize=-2pt 0,subticks=2]{->}(0,0)(-0.94,-1.5)(13.44,3.78)[x,140] [f(x),-40]
\rput[tl](10.36,0.32){e}
\rput[tl](8.34,0.38){d}
\rput[tl](6.16,0.32){c}
\rput[tl](3.68,0.38){b}
\rput[tl](0.3,0.32){a}
\rput[tl](3.1,2.6){f}
\psline[linewidth=2.pt]{->}(3.,4.)(5.14,3.98)
\psplot[linewidth=2.pt,plotpoints=200]{-0.940000000000001}{13.44000000000002}{SIN(1.5*(x-2.14))/(0.5*(x-2.14))-0.020000000000000018}
\end{pspicture*}}
\end{center}

Welche der nachstehend angeführten bestimmten Integrale haben einen Wert, der größer als 0 ist? Kreuze die beiden zutreffenden bestimmten Integrale an!\leer

\multiplechoice[5]{  %Anzahl der Antwortmoeglichkeiten, Standard: 5
				L1={$$\int^c_a f(x)dx$$},   %1. Antwortmoeglichkeit 
				L2={$$\int^c_b f(x)dx$$},   %2. Antwortmoeglichkeit
				L3={$$\int^d_b f(x)dx$$},   %3. Antwortmoeglichkeit
				L4={$$\int^b_a f(x)dx$$},   %4. Antwortmoeglichkeit
				L5={$$\int^e_d f(x)dx$$},	 %5. Antwortmoeglichkeit
				L6={},	 %6. Antwortmoeglichkeit
				L7={},	 %7. Antwortmoeglichkeit
				L8={},	 %8. Antwortmoeglichkeit
				L9={},	 %9. Antwortmoeglichkeit
				%% LOESUNG: %%
				A1=1,  % 1. Antwort
				A2=4,	 % 2. Antwort
				A3=0,  % 3. Antwort
				A4=0,  % 4. Antwort
				A5=0,  % 5. Antwort
				}
\end{beispiel}