\section{AN 4.1 - 4 - MAT - Bestimmtes Integral - OA - Matura 1. NT 2017/18}

\begin{beispiel}[AN 4.1]{1}
In der nachstehenden Abbildung ist der Graph einer abschnittsweise linearen Funktion $f$ dargestellt. Die Koordinaten der Punkt $A, B$ und $C$ des Graphen der Funktion sind ganzzahlig.

\begin{center}
	\resizebox{0.5\linewidth}{!}{
\psset{xunit=1.0cm,yunit=1.0cm,algebraic=true,dimen=middle,dotstyle=o,dotsize=5pt 0,linewidth=1.6pt,arrowsize=3pt 2,arrowinset=0.25}
\begin{pspicture*}(-1.52,-3.62)(8.82,4.62)
\multips(0,-3)(0,1.0){9}{\psline[linestyle=dashed,linecap=1,dash=1.5pt 1.5pt,linewidth=0.4pt,linecolor=darkgray]{c-c}(-1.52,0)(8.82,0)}
\multips(-1,0)(1.0,0){11}{\psline[linestyle=dashed,linecap=1,dash=1.5pt 1.5pt,linewidth=0.4pt,linecolor=darkgray]{c-c}(0,-3.62)(0,4.62)}
\psaxes[labelFontSize=\scriptstyle,xAxis=true,yAxis=true,Dx=1.,Dy=1.,ticksize=-2pt 0,subticks=2]{->}(0,0)(-1.52,-3.62)(8.82,4.62)[x,140] [f(x),-40]
\psplot[linewidth=2.pt]{-1.52}{3.}{(-6.-0.*x)/-3.}
\psline[linewidth=2.pt](7.,-2.)(3.,2.)
\psplot[linewidth=2.pt]{7.}{8.82}{(-6.-0.*x)/3.}
\psdots[dotstyle=*](0.,2.)
\rput[bl](0.08,2.2){$A$}
\psdots[dotstyle=*](3.,2.)
\rput[bl](3.08,2.2){$B$}
\psdots[dotstyle=*](7.,-2.)
\rput[bl](7.08,-1.8){$C$}
\rput[bl](1.44,2.14){$f$}
\end{pspicture*}}
\end{center}

Ermittle den Wert des bestimmten Integrals $\displaystyle\int^7_0{f(x)}$\,d$x$.\leer


$\displaystyle\int^7_0{f(x)}$\,d$x=$\,\antwort[\rule{3cm}{0.3pt}]{$6$}
\end{beispiel}