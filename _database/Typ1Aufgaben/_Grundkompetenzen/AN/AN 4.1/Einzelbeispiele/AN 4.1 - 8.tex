\section{AN 4.1 - 8 - K8 - Mountainbiker - OA - MarStr UNIVIE}

\begin{beispiel}[AN 4.1]{1}
Der Fahrradcomputer eines Mountainbikers misst laufend die Geschwindigkeit in km/min, welche monoton steigt. In folgender Tabelle sind die Daten für ausgewählte Zeitpunkte in einem Zeitraum von 60 Minuten gegeben:

\begin{center}
\begin{tabular}{c|c|c|c|c|c|c|c}
$t$ (in min) & 0 & 10 & 20 & 30 & 40 & 50 & 60 \\ \hline
$v(t)$ (in km/min) & 0,1 & 0,2 & 0,25 & 0,25 & 0,3 & 0,35 & 0,4
\end{tabular}
\end{center}

Gib mithilfe der gegebenen Daten eine möglichst genaue untere und obere Schranke für das Integral $\int\limits_{0}^{60} v(t)\, \text{dt}$ an.

\vspace{0.3cm}
$\antwort{
\text{Untere Schranke:} \\ 
\text{U} = 10 \cdot 0,1 + 10 \cdot 0,2 + 10 \cdot 0,25 + 10 \cdot 0,25 + 10 \cdot 0,3 + 10 \cdot 0,35 = 14,5 \, \text{km}}$

\vspace{0.2cm}
$\antwort{
\text{Obere Schranke:} \\ 
\text{O} = 10 \cdot 0,2 + 10 \cdot 0,25 + 10 \cdot 0,25 + 10 \cdot 0,3 + 10 \cdot 0,35 + 10 \cdot 0,4 = 17,5 \, \text{km}}$

\vspace{0.5cm}
\antwort{
Lösungsschlüssel: \\
Ein Punkt ist genau dann zu geben, wenn die untere und obere Schranke der zurückgelegten Wegstrecke korrekt angegeben werden.}
\end{beispiel}