\section{AN 4.1 - 7 - Obersumme - MC - ThoLin UNIVIE}

\begin{beispiel}[AN 4.1]{1}
In der Abbildung sieht man den Graphen der Funktion $f$ im Intervall [0, 10]. Das Intervall wird in $n$ gleich große Teilintervalle unterteilt. Über jedem Teilintervall wurde ein Rechteck errichtet, wobei der rechte obere Eckpunkt jeweils auf dem Graphen von $f$ liegt. $S_{n}$ beschreibt dabei die Summe der Flächeninhalte von allen Rechtecken, wobei $n \in \mathbb{N}^*$ die Anzahl der Teilintervalle angibt.  
\newline
\newline
\newrgbcolor{zzttqq}{0.6 0.2 0.}
\psset{xunit=1.0cm,yunit=1.0cm,algebraic=true,dimen=middle,dotstyle=o,dotsize=5pt 0,linewidth=1.6pt,arrowsize=3pt 2,arrowinset=0.25}
\begin{pspicture*}(-0.8389,-0.6544)(10.92714,7.49132)
\multips(0,0)(0,1.0){9}{\psline[linestyle=dashed,linecap=1,dash=1.5pt 1.5pt,linewidth=0.4pt,linecolor=gray]{c-c}(0,0)(10.92714,0)}
\multips(0,0)(1.0,0){12}{\psline[linestyle=dashed,linecap=1,dash=1.5pt 1.5pt,linewidth=0.4pt,linecolor=gray]{c-c}(0,0)(0,7.49132)}
\psaxes[labelFontSize=\scriptstyle,xAxis=true,yAxis=true,Dx=1.,Dy=1.,ticksize=-2pt 0,subticks=2]{->}(0,0)(-0.8389,-0.6544)(10.92714,7.49132)[$x$,140] [$f(x)$,-40]
\psframe[linewidth=0.8pt,fillcolor=gray,fillstyle=solid,opacity=0.4](0.,0)(1.,1.0499999928533659)
\psframe[linewidth=0.8pt,fillcolor=gray,fillstyle=solid,opacity=0.4](1.,0)(2.,1.1999999878976215)
\psframe[linewidth=0.8pt,fillcolor=gray,fillstyle=solid,opacity=0.4](2.,0)(3.,1.449999965273762)
\psframe[linewidth=0.8pt,fillcolor=gray,fillstyle=solid,opacity=0.4](3.,0)(4.,1.7999999559428095)
\psframe[linewidth=0.8pt,fillcolor=gray,fillstyle=solid,opacity=0.4](4.,0)(5.,2.249999950405737)
\psframe[linewidth=0.8pt,fillcolor=gray,fillstyle=solid,opacity=0.4](5.,0)(6.,2.7999998876235335)
\psframe[linewidth=0.8pt,fillcolor=gray,fillstyle=solid,opacity=0.4](6.,0)(7.,3.4499998688941216)
\psframe[linewidth=0.8pt,fillcolor=gray,fillstyle=solid,opacity=0.4](7.,0)(8.,4.199999854336559)
\psframe[linewidth=0.8pt,fillcolor=gray,fillstyle=solid,opacity=0.4](8.,0)(9.,5.049999845987634)
\psframe[linewidth=0.8pt,fillcolor=gray,fillstyle=solid,opacity=0.4](9.,0)(10.,5.9999998398296)
\psplot[linewidth=2.pt,plotpoints=200]{0}{10.06}{0.04938271604938271*x^(2.0)+1.0}
\begin{scriptsize}
\rput[bl](0.43886,1.31548){$f$}
\end{scriptsize}
\end{pspicture*}
\subsection{Aufgabenstellung:}
Kreuze die beiden zutreffenden Aussagen an!
\multiplechoice[5]{  %Anzahl der Antwortmoeglichkeiten, Standard: 5
				L1={Es gilt:$\int_{0}^{10}f(x)$\ d$x$ > $S_{10}$},   %1. Antwortmoeglichkeit 
				L2={Es gilt: $S_{10}>S_5$},   %2. Antwortmoeglichkeit
				L3={Je größer man $n$ wählt, desto näher ist $S_n$ dem Wert $\int_{0}^{10}f(x)$\ d$x$. },   %3. Antwortmoeglichkeit
				L4={Je größer $n$ ist, desto größer ist $S_n$.},   %4. Antwortmoeglichkeit
				L5={Es gilt:$\int_{0}^{10}f(x)$\ d$x\leq S_n$ für alle $n \in \mathbb{N}^*$},	 %5. Antwortmoeglichkeit
				L6={},	 %6. Antwortmoeglichkeit
				L7={},	 %7. Antwortmoeglichkeit
				L8={},	 %8. Antwortmoeglichkeit
				L9={},	 %9. Antwortmoeglichkeit
				%% LOESUNG: %%
				A1=3,  % 1. Antwort
				A2=5,	 % 2. Antwort
				A3=0,  % 3. Antwort
				A4=0,  % 4. Antwort
				A5=0,  % 5. Antwort
				}
\antwort{\textbf{Lösungsschlüssel:}\\
Ein Punkt ist zu vergeben, wenn genau die zwei zutreffenden Aussagen angekreuzt wurden.}
\end{beispiel}