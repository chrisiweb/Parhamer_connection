\section{AN 4.3 - 11 - MAT - Halbierung einer Fläche - OA - Matura 2015/16 - Nebentermin 1}

\begin{beispiel}[AN 4.3]{1} %PUNKTE DES BEISPIELS
Gegeben ist die reelle Funktion $f$ mit $f(x)=x^2$. \leer

Berechne die Stelle $b$ so, dass die Fläche zwischen der $x$-Achse und dem Graphen der
Funktion $f$ im Intervall $[2;~4]$ in zwei gleich große Flächen $A_1$ und $A_2$ geteilt wird (siehe Abbildung).\leer

\begin{center}
\resizebox{0.6\linewidth}{!}{
\newrgbcolor{uuuuuu}{0.26666666666666666 0.26666666666666666 0.26666666666666666}
\psset{xunit=5cm,yunit=5cm,algebraic=true,dimen=middle,dotstyle=o,dotsize=5pt 0,linewidth=0.8pt,arrowsize=3pt 2,arrowinset=0.25}
\begin{pspicture*}(-0.06508503852943727,-0.07913862414667742)(1.256863485653584,0.9403714922896743)
\pscustom[linecolor=darkgray,fillcolor=darkgray,fillstyle=solid,opacity=0.2]{\psplot{0.4}{0.66}{x^(2.0)}\lineto(0.66,0)\lineto(0.4,0)\closepath}
\pscustom[linecolor=darkgray,fillcolor=darkgray,fillstyle=solid,opacity=0.2]{\psplot{0.66}{0.8}{x^(2.0)}\lineto(0.8,0)\lineto(0.66,0)\closepath}
\psaxes[labelFontSize=\scriptstyle,xAxis=true,yAxis=true,labels=none,Dx=0.2,Dy=0.2,ticksize=-2pt 0,subticks=0]{->}(0,0)(-0.06508503852943727,-0.07913862414667742)(1.256863485653584,0.9403714922896743)[x,140] [f(x),-40]
\psplot[linewidth=1.2pt,plotpoints=200]{-0.06508503852943727}{1.256863485653584}{x^(2.0)}
\rput[tl](0.385,-0.025){\scriptsize $2$}
\rput[tl](0.785,-0.025){\scriptsize$4$}
\rput[tl](0.5,0.18){\small$A_1$}
\rput[tl](0.69,0.18){\small$A_2$}
\psline(0.4,0.16)(0.4,0.)
\psline(0.66,0.4356)(0.66,-0.0010900027926983445)
\psline(0.8,0.64)(0.8,0.)
\rput[tl](0.645,-0.025){\scriptsize$b$}
\rput[tl](0.7690596021912146,0.7720791524951569){$f$}
\end{pspicture*}}
\end{center}


\antwort{Mögliche Berechnung: \\

$\int_{2}^{b}x^2\,dx=\int_{b}^{4}x^2\,dx \Rightarrow \frac{b^3}{3}-\frac{2^3}{3}=\frac{4^3}{3}-\frac{b^3}{3}$ \\
$b=\sqrt[3]{36}$ \leer

Lösungsschlüssel:

Ein Punkt für die richtige Lösung. Andere Schreibweisen des Ergebnisses sind ebenfalls als richtig zu werten.

Toleranzintervall: $[3,29;~3,31]$

Die Aufgabe ist auch dann als richtig gelöst zu werten, wenn bei korrektem Ansatz das Ergebnis
aufgrund eines Rechenfehlers nicht richtig ist.}
\end{beispiel}