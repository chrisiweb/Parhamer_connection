\section{AN 4.3 - 25 - K8 - E-Scooter - MC - MatKon}

\begin{beispiel}[AN 4.3]{1}
Herr Coseetor ist begeisterter E-Scooter Fahrer. Die Diagramme zeigen die Zeit-Geschwindigkeitsfunktionen seiner letzten fünf Scooter-Touren.

Kreuze an, bei welcher Tour er den längsten Weg zurückgelegt hat.

\langmultiplechoice[6]{  %Anzahl der Antwortmoeglichkeiten, Standard: 5
				L1={\psset{xunit=0.3cm,yunit=0.3cm,algebraic=true,dimen=middle,dotstyle=o,dotsize=5pt 0,linewidth=1pt,arrowsize=3pt 2,arrowinset=0.25}
\begin{pspicture*}(-1.7,-1.7)(17.621951219512216,10.85714285714287)
\multips(0,0)(0,1.0){12}{\psline[linestyle=dashed,linecap=1,dash=1.5pt 1.5pt,linewidth=0.4pt,linecolor=gray]{c-c}(0,0)(17.621951219512216,0)}
\multips(0,0)(1.0,0){19}{\psline[linestyle=dashed,linecap=1,dash=1.5pt 1.5pt,linewidth=0.4pt,linecolor=gray]{c-c}(0,0)(0,10.85714285714287)}
\psaxes[labelFontSize=\scriptstyle,xAxis=true,showorigin=false,yAxis=true,Dx=2.,Dy=2.,ticksize=-2pt 0,subticks=0]{->}(0,0)(0.,0.)(17.621951219512216,10.85714285714287)[$t$,140] [$g(t)$,-40]
\psline[linewidth=1.5pt](0.,0.)(3.,3.)
\psline[linewidth=1.5pt](3.,3.)(6.,3.)
\psline[linewidth=1.5pt](6.,3.)(9.,6.)
\psline[linewidth=1.5pt](9.,6.)(13.,6.)
\psline[linewidth=1.5pt](13.,6.)(16.,0.)
\end{pspicture*}},   %1. Antwortmoeglichkeit 
				L2={\psset{xunit=0.3cm,yunit=0.3cm,algebraic=true,dimen=middle,dotstyle=o,dotsize=5pt 0,linewidth=1pt,arrowsize=3pt 2,arrowinset=0.25}
\begin{pspicture*}(-1.7,-1.7)(17.621951219512216,10.85714285714287)
\multips(0,0)(0,1.0){12}{\psline[linestyle=dashed,linecap=1,dash=1.5pt 1.5pt,linewidth=0.4pt,linecolor=gray]{c-c}(0,0)(17.621951219512216,0)}
\multips(0,0)(1.0,0){19}{\psline[linestyle=dashed,linecap=1,dash=1.5pt 1.5pt,linewidth=0.4pt,linecolor=gray]{c-c}(0,0)(0,10.85714285714287)}
\psaxes[labelFontSize=\scriptstyle,xAxis=true,showorigin=false,yAxis=true,Dx=2.,Dy=2.,ticksize=-2pt 0,subticks=0]{->}(0,0)(0.,0.)(17.621951219512216,10.85714285714287)[$t$,140] [$k(t)$,-40]
\psline[linewidth=1.5pt](0.,0.)(3.,3.)
\psline[linewidth=1.5pt](3.,3.)(10.,3.)
\psline[linewidth=1.5pt](10.,3.)(11.,8.)
\psline[linewidth=1.5pt](11.,8.)(13.,8.)
\psline[linewidth=1.5pt](13.,8.)(16.,0.)
\end{pspicture*}},   %2. Antwortmoeglichkeit
				L3={\psset{xunit=0.3cm,yunit=0.3cm,algebraic=true,dimen=middle,dotstyle=o,dotsize=5pt 0,linewidth=1pt,arrowsize=3pt 2,arrowinset=0.25}
\begin{pspicture*}(-1.7,-1.7)(17.621951219512216,10.85714285714287)
\multips(0,0)(0,1.0){12}{\psline[linestyle=dashed,linecap=1,dash=1.5pt 1.5pt,linewidth=0.4pt,linecolor=gray]{c-c}(0,0)(17.621951219512216,0)}
\multips(0,0)(1.0,0){19}{\psline[linestyle=dashed,linecap=1,dash=1.5pt 1.5pt,linewidth=0.4pt,linecolor=gray]{c-c}(0,0)(0,10.85714285714287)}
\psaxes[labelFontSize=\scriptstyle,xAxis=true,showorigin=false,yAxis=true,Dx=2.,Dy=2.,ticksize=-2pt 0,subticks=0]{->}(0,0)(0.,0.)(17.621951219512216,10.85714285714287)[$t$,140] [$h(t)$,-40]
\psline[linewidth=1.5pt](0.,0.)(3.,8.)
\psline[linewidth=1.5pt](3.,8.)(10.,8.)
\psline[linewidth=1.5pt](10.,8.)(12.,0.)
\end{pspicture*}},   %3. Antwortmoeglichkeit
				L4={\psset{xunit=0.3cm,yunit=0.3cm,algebraic=true,dimen=middle,dotstyle=o,dotsize=5pt 0,linewidth=1pt,arrowsize=3pt 2,arrowinset=0.25}
\begin{pspicture*}(-1.7,-1.7)(17.621951219512216,10.85714285714287)
\multips(0,0)(0,1.0){12}{\psline[linestyle=dashed,linecap=1,dash=1.5pt 1.5pt,linewidth=0.4pt,linecolor=gray]{c-c}(0,0)(17.621951219512216,0)}
\multips(0,0)(1.0,0){19}{\psline[linestyle=dashed,linecap=1,dash=1.5pt 1.5pt,linewidth=0.4pt,linecolor=gray]{c-c}(0,0)(0,10.85714285714287)}
\psaxes[labelFontSize=\scriptstyle,xAxis=true,showorigin=false,yAxis=true,Dx=2.,Dy=2.,ticksize=-2pt 0,subticks=0]{->}(0,0)(0.,0.)(17.621951219512216,10.85714285714287)[$t$,140] [$n(t)$,-40]
\psline[linewidth=1.5pt](0.,6.)(14.,6.)
\psline[linewidth=1.5pt](14.,6.)(16.,0.)
\end{pspicture*}},   %4. Antwortmoeglichkeit
				L5={\psset{xunit=0.3cm,yunit=0.3cm,algebraic=true,dimen=middle,dotstyle=o,dotsize=5pt 0,linewidth=1pt,arrowsize=3pt 2,arrowinset=0.25}
\begin{pspicture*}(-1.7,-1.7)(17.621951219512216,10.85714285714287)
\multips(0,0)(0,1.0){12}{\psline[linestyle=dashed,linecap=1,dash=1.5pt 1.5pt,linewidth=0.4pt,linecolor=gray]{c-c}(0,0)(17.621951219512216,0)}
\multips(0,0)(1.0,0){19}{\psline[linestyle=dashed,linecap=1,dash=1.5pt 1.5pt,linewidth=0.4pt,linecolor=gray]{c-c}(0,0)(0,10.85714285714287)}
\psaxes[labelFontSize=\scriptstyle,xAxis=true,showorigin=false,yAxis=true,Dx=2.,Dy=2.,ticksize=-2pt 0,subticks=0]{->}(0,0)(0.,0.)(17.621951219512216,10.85714285714287)[$t$,140] [$m(t)$,-40]
\psline[linewidth=1.5pt](0.,9.)(10.,9.)
\psline[linewidth=1.5pt](10.,9.)(12.,0.)
\end{pspicture*}},	 %5. Antwortmoeglichkeit
				L6={\psset{xunit=0.3cm,yunit=0.3cm,algebraic=true,dimen=middle,dotstyle=o,dotsize=5pt 0,linewidth=1pt,arrowsize=3pt 2,arrowinset=0.25}
\begin{pspicture*}(-1.7,-1.7)(17.621951219512216,10.85714285714287)
\multips(0,0)(0,1.0){12}{\psline[linestyle=dashed,linecap=1,dash=1.5pt 1.5pt,linewidth=0.4pt,linecolor=gray]{c-c}(0,0)(17.621951219512216,0)}
\multips(0,0)(1.0,0){19}{\psline[linestyle=dashed,linecap=1,dash=1.5pt 1.5pt,linewidth=0.4pt,linecolor=gray]{c-c}(0,0)(0,10.85714285714287)}
\psaxes[labelFontSize=\scriptstyle,xAxis=true,showorigin=false,yAxis=true,Dx=2.,Dy=2.,ticksize=-2pt 0,subticks=0]{->}(0,0)(0.,0.)(17.621951219512216,10.85714285714287)[$t$,140] [$b(t)$,-40]
\psline[linewidth=1.5pt](0.,0.)(4.5,10)
\psline[linewidth=1.5pt](4.5,10)(12.,10.)
\psline[linewidth=1.5pt](12,10)(16.,0.)
\end{pspicture*}},	 %6. Antwortmoeglichkeit
				L7={},	 %7. Antwortmoeglichkeit
				L8={},	 %8. Antwortmoeglichkeit
				L9={},	 %9. Antwortmoeglichkeit
				%% LOESUNG: %%
				A1=6,  % 1. Antwort
				A2=0,	 % 2. Antwort
				A3=0,  % 3. Antwort
				A4=0,  % 4. Antwort
				A5=0,  % 5. Antwort
				}
\end{beispiel}