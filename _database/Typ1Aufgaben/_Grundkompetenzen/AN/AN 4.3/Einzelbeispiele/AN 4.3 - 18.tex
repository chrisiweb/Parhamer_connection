\section{AN 4.3 - 18 - Beschleunigung - MC - Matura - 1. NT 2017/18}

\begin{beispiel}[AN 4.3]{1}
Die Funktion $a$ beschreibt die Beschleunigung eines sich in Bewegung befindlichen Objekts in Abhängigkeit von der Zeit $t$ im Zeitintervall $[t_1;t_1+4]$. Die Beschleunigung $a(t)$ wird in m/s$^2$, die Zeit $t$ in s angegeben.

Es gilt:

$\displaystyle\int^{t_1+4}_{t_1}{a(t)}$\,d$t=2$

Eine der nachstehenden Aussagen interpretiert das angegebene bestimmte Integral korrekt. Kreuze die zutreffende Aussage an!

\multiplechoice[6]{  %Anzahl der Antwortmoeglichkeiten, Standard: 5
				L1={Das Objekt legt im gegebenen Zeitintervall 2\,m zurück.},   %1. Antwortmoeglichkeit 
				L2={Die Geschwindigkeit des Objekts am Ende des gegebenen
Zeitintervalls beträgt 2\,m/s.},   %2. Antwortmoeglichkeit
				L3={Die Beschleunigung des Objekts ist am Ende des gegebenen
Zeitintervalls um 2\,m/s$^2$ höher als am Anfang des Intervalls.},   %3. Antwortmoeglichkeit
				L4={Die Geschwindigkeit des Objekts hat in diesem Zeitintervall
um 2\,m/s zugenommen.},   %4. Antwortmoeglichkeit
				L5={Im Mittel erhöht sich die Geschwindigkeit des Objekts im
gegebenen Zeitintervall pro Sekunde um 2\,m/s.},	 %5. Antwortmoeglichkeit
				L6={Im gegebenen Zeitintervall erhöht sich die Beschleunigung
des Objekts pro Sekunde um $\frac{2}{4}$\,m/s$^2$.},	 %6. Antwortmoeglichkeit
				L7={},	 %7. Antwortmoeglichkeit
				L8={},	 %8. Antwortmoeglichkeit
				L9={},	 %9. Antwortmoeglichkeit
				%% LOESUNG: %%
				A1=4,  % 1. Antwort
				A2=0,	 % 2. Antwort
				A3=0,  % 3. Antwort
				A4=0,  % 4. Antwort
				A5=0,  % 5. Antwort
				}
\end{beispiel}