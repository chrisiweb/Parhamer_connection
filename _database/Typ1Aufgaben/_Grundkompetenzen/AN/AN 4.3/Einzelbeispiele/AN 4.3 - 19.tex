\section{AN 4.3 - 19 - Rotation y-Achse - OA - MatKon}

\begin{beispiel}[AN 4.3]{1}
Ein Glasbehälter mit der Höhe $h=10$\,cm ergibt sich näherungsweise durch Rotation des Graphen der Funktion $f$ mit der Funktionsgleichung $f(x)=0,75x^2$ um die $y$-Achse ($x$ und $y$ in cm). Das Glas soll bis zu 3\,cm unterhalb des Randes mit Wasser gefüllt werden.\\
Berechne, wie viel Liter Wasser ($l$) benötigt werden!

\antwort{$V_y=\pi\cdot\displaystyle\int_3^{10}{\frac{y}{0,75}}\,\text{d}y=60,\dot{6}\,$cm$^3=0,060\dot{6}\,l$}
\end{beispiel}