\section{AN 4.3 - 26 - MAT - Bestimmte Integrale - MC - Matura 2018/19 2. NT}

\begin{beispiel}[AN 4.3]{1}
Nachstehend ist der Graph einer Polynomfunktion $f$ mit den Nullstellen\\
 $x_1=-2$, $x_2=0$, $x_3=2$ und $x_4=4$ dargestellt.

Für die mit $A_1$, $A_2$ und $A_3$ gekennzeichneten Flächeninhalte gilt:\\
$A_1=0,4$, $A_2=1,5$ und $A_3=3,2$.

\begin{center}
\psset{xunit=1.0cm,yunit=0.2cm,algebraic=true,dimen=middle,dotstyle=o,dotsize=5pt 0,linewidth=1.6pt,arrowsize=3pt 2,arrowinset=0.25}
\begin{pspicture*}(-1.74,-8.365628192032684)(5.56,18.389308818522302)
\psaxes[labelFontSize=\scriptstyle,showorigin=false,labels=x,xAxis=true,yAxis=true,Dx=1.,Dy=5.,ticks=x,ticksize=-2pt 0,subticks=0]{->}(0,0)(-1.74,-8.365628192032684)(5.56,18.389308818522302)[$x$,140] [$f(x)$,-40]
\pscustom[linewidth=0.8pt,fillcolor=black,fillstyle=solid,opacity=0.1]{\psplot{-1.}{0.}{-(x+1.0)*(x-2.0)*(x-4.0)*x}\lineto(0.,0)\lineto(-1.,0)\closepath}
\pscustom[linewidth=0.8pt,fillcolor=black,fillstyle=solid,opacity=0.1]{\psplot{0.}{2.}{-(x+1.0)*(x-2.0)*(x-4.0)*x}\lineto(2.,0)\lineto(0.,0)\closepath}
\pscustom[linewidth=0.8pt,fillcolor=black,fillstyle=solid,opacity=0.1]{\psplot{2.}{4.}{-(x+1.0)*(x-2.0)*(x-4.0)*x}\lineto(4.,0)\lineto(2.,0)\closepath}
\psplot[linewidth=2.pt,plotpoints=200]{-1.7400000000000009}{5.560000000000002}{-(x+1.0)*(x-2.0)*(x-4.0)*x}
\begin{scriptsize}
\rput[bl](-1.16,-7.988798093292473){$f$}
\rput[bl](-0.65,0.6){$A_1$}
\rput[bl](0.92,-4.5){$A_2$}
\rput[bl](2.96,4.295863125638409){$A_3$}
\end{scriptsize}
\end{pspicture*}
\end{center}

Kreuze die beiden Gleichungen an, die wahre Aussagen sind.

\multiplechoice[5]{  %Anzahl der Antwortmoeglichkeiten, Standard: 5
				L1={$\displaystyle\int^2_{-1} f(x)\,\text{d}x=1,9$},   %1. Antwortmoeglichkeit 
				L2={$\displaystyle\int^4_{0} f(x)\,\text{d}x=1,7$},   %2. Antwortmoeglichkeit
				L3={$\displaystyle\int^4_{-1} f(x)\,\text{d}x=5,1$},   %3. Antwortmoeglichkeit
				L4={$\displaystyle\int^2_{0} f(x)\,\text{d}x=1,5$},   %4. Antwortmoeglichkeit
				L5={$\displaystyle\int^4_{2} f(x)\,\text{d}x=3,2$},	 %5. Antwortmoeglichkeit
				L6={},	 %6. Antwortmoeglichkeit
				L7={},	 %7. Antwortmoeglichkeit
				L8={},	 %8. Antwortmoeglichkeit
				L9={},	 %9. Antwortmoeglichkeit
				%% LOESUNG: %%
				A1=2,  % 1. Antwort
				A2=5,	 % 2. Antwort
				A3=0,  % 3. Antwort
				A4=0,  % 4. Antwort
				A5=0,  % 5. Antwort
				}
\end{beispiel}