\section{AN 4.3 - 7 Arbeit beim Verschieben eines Massestücks - OA - Matura 2015/16 - Haupttermin}

\begin{beispiel}[AN 4.3]{1} %PUNKTE DES BEISPIELS
Ein Massestück wird durch die Einwirkung einer Kraft geradlinig bewegt. Die dazu erforderliche
Kraftkomponente in Wegrichtung ist als Funktion des zurückgelegten Weges in der nachstehenden
Abbildung dargestellt. Der Weg $s$ wird in Metern (m), die Kraft $F(s)$ in Newton (N) gemessen.\leer

Im ersten Wegabschnitt wird $F(s)$ durch $f_1$ mit $f_1(s) = \frac{5}{16}\cdot s^2$ beschrieben. Im zweiten Abschnitt ($f_2$) nimmt sie linear auf den Wert null ab. \leer

Koordinaten der hervorgehobenen Punkte des Graphen der Funktion sind ganzzahlig.


\begin{center}
\resizebox{1\linewidth}{!}{
\psset{xunit=1.0cm,yunit=1.0cm,algebraic=true,dimen=middle,dotstyle=o,dotsize=5pt 0,linewidth=0.8pt,arrowsize=3pt 2,arrowinset=0.25}
\begin{pspicture*}(-0.5852276728024793,-0.6198401700150555)(17.432717595263284,6.526836966741394)
\multips(0,0)(0,1.0){8}{\psline[linestyle=dashed,linecap=1,dash=1.5pt 1.5pt,linewidth=0.4pt,linecolor=gray]{c-c}(0,0)(17.432717595263284,0)}
\multips(0,0)(1.0,0){19}{\psline[linestyle=dashed,linecap=1,dash=1.5pt 1.5pt,linewidth=0.4pt,linecolor=gray]{c-c}(0,0)(0,6.526836966741394)}
\psaxes[labelFontSize=\scriptstyle,xAxis=true,yAxis=true,Dx=1.,Dy=1.,ticksize=-2pt 0,subticks=2]{->}(0,0)(0.,0.)(17.432717595263284,6.526836966741394)[$s$ in m,140] [$F(s)$ in N,-40]
\psplot[linewidth=1.2pt,plotpoints=200]{0}{4}{5.0/16.0*x^(2.0)}
\psplot{4}{15}{(--75.-5.*x)/11.}
\rput[tl](2.118409445070977,2.877872185275799){$f_1$}
\rput[tl](8.376478578050795,3.690853975965025){$f_2$}
\begin{scriptsize}
\psdots[dotstyle=*](4.,5.)
\psdots[dotstyle=*](15.,0.)
\end{scriptsize}
\end{pspicture*}}
\end{center}

Ermittle die Arbeit $W$ in Joule (J), die diese Kraft an dem Massestück verrichtet, wenn es von $s = 0$\,m bis zu $s = 15$\,m bewegt wird. \leer


$W=\rule{3cm}{0.3pt}$\,J

\antwort{
$W=\int_{0}^{4} \! \dfrac{5}{16} \cdot s^2 \, \mathrm{d}s + \dfrac{5 \cdot 11}{2}$\\

$W \approx 34,17\,$J 

Lösungsschlüssel:

Ein Punkt für die richtige Lösung. Andere Schreibweisen des Ergebnisses sind ebenfalls als richtig zu werten.

Toleranzintervall: $[34\,J;~35\,J]$
}
\end{beispiel}