\section{AN 4.3 - 12 Tachograph - OA - Matura NT 2 15/16}

\begin{beispiel}[AN 4.3]{1} %PUNKTE DES BEISPIELS
Mithilfe eines Tachographen kann die Geschwindigkeit eines Fahrzeugs in Abh�ngigkeit von der Zeit aufgezeichnet werden. Es sei $v(t)$ die Geschwindigkeit zum Zeitpunkt $t$. Die Zeit wird in Stunden (h) angegeben, die Geschwindigkeit in Kilometern pro Stunde (km/h).

Ein Fahrzeug startet zum Zeitpunkt $t=0$.

Gib die Bedeutung der Gleichung $$\int^{0,5}_0{v(t)}dt=40$$ unter Verwendung der korrekten Einheiten im gegebenen Kontext an!\leer

\antwort{Diese Gleichung sagt aus, dass das Fahrzeug in der ersten halben Stunde (bzw. im Zeitintervall $[0\,h;0,5\,h]$) 40 km zur�ckgelegt hat.}
\end{beispiel}