\section{AN 4.3 - 10 - Bremsweg - OA - Matura 2014/15 Kompensationsprüfung}

\begin{beispiel}[AN 4.3]{1} %PUNKTE DES BEISPIELS
Ein PKW beginnt zum Zeitpunkt $t=0$ gleichmäßig zu bremsen.\\
Die Funktion $v$ beschreibt die Geschwindigkeit $v(t)$ des PKW zum Zeitpunkt $t$ ($v(t)$ in Metern pro Sekunde, $t$ in Sekunden). Es gilt: $v(t)=20-8t$.

Berechne die Länge desjenigen Weges, den der PKW während des gleichmäßigen Bremsvorgangs bis zum Stillstand zurücklegt.

\antwort{Mögliche Berechnung:\\
$v(t)=0\Rightarrow t=2,5$\\
$$\int^{2,5}_{0}{(20-8t)}\,dt=\left. (20t-4t^2) \right|_{0}^{2,5}=25$$\\
Die Länge des Bremsweges beträgt 25m.}
\end{beispiel}