\section{AN 4.3 - 6 Flächenberechnung - MC - BIFIE}

\begin{beispiel}[AN 4.3]{1} %PUNKTE DES BEISPIELS
				Die Summe A der Inhalte der beiden von den Graphen der Funktionen $f$ und $g$ eingeschlossenen Flächen soll berechnet werden.
				\begin{center}
					\resizebox{0.8\linewidth}{!}{\psset{xunit=1.0cm,yunit=1.0cm,algebraic=true,dimen=middle,dotstyle=o,dotsize=5pt 0,linewidth=0.8pt,arrowsize=3pt 2,arrowinset=0.25}
\begin{pspicture*}(-1.7761488644641417,-2.938334224369601)(11.80860741102746,8.927101335005048)
\pscustom[fillcolor=black!30,fillstyle=solid,opacity=0.1]{\psplot{1.}{8.}{-0.09690716815660791*x^(3.0)+0.9628860178792948*x^(2.0)-0.5917508854812766*x-2.2742279642414105}\lineto(8.,5.)\psplot{8.}{1.}{0.2*x^(3.0)-2.6*x^(2.0)+9.8*x-9.4}\lineto(1.,-2.)\closepath}
\multips(0,-2)(0,1.0){12}{\psline[linestyle=dashed,linecap=1,dash=1.5pt 1.5pt,linewidth=0.4pt,linecolor=darkgray]{c-c}(-1.7761488644641417,0)(11.80860741102746,0)}
\multips(-1,0)(1.0,0){14}{\psline[linestyle=dashed,linecap=1,dash=1.5pt 1.5pt,linewidth=0.4pt,linecolor=darkgray]{c-c}(0,-2.938334224369601)(0,8.927101335005048)}
\psaxes[labelFontSize=\scriptstyle,xAxis=true,yAxis=true,Dx=1.,Dy=1.,ticksize=-2pt 0,subticks=2]{->}(0,0)(-1.7761488644641417,-2.938334224369601)(11.80860741102746,8.927101335005048)[x,140] [y,-40]
\psplot[linewidth=1.2pt,plotpoints=200]{-1.7761488644641417}{11.80860741102746}{-0.09690716815660791*x^(3.0)+0.9628860178792948*x^(2.0)-0.5917508854812766*x-2.2742279642414105}
\psplot[linewidth=1.2pt,plotpoints=200]{-1.7761488644641417}{11.80860741102746}{0.2*x^(3.0)-2.6*x^(2.0)+9.8*x-9.4}
\rput[tl](4.464348549589812,6.613447531835393){g}
\rput[tl](4.336991459507078,1.2644497483605968){f}
\end{pspicture*}}
				\end{center}
				\leer
				
				Kreuze die zutreffende(n) Formel(n) an!
				\multiplechoice[5]{  %Anzahl der Antwortmoeglichkeiten, Standard: 5
								L1={$$A=\int_1^8{(f(x)-g(x))}dx$$},   %1. Antwortmoeglichkeit 
								L2={$$A=\int_1^3{(f(x)-g(x))}dx+\int_3^8{(g(x)-f(x))}dx$$},   %2. Antwortmoeglichkeit
								L3={$$A=\left|\int_1^8{(f(x)-g(x))}dx\right|$$},   %3. Antwortmoeglichkeit
								L4={$$A=\int_1^3{(f(x)-g(x))}dx-\int_3^8{(f(x)-g(x))}dx$$},   %4. Antwortmoeglichkeit
								L5={$$A=\left|\int_1^3{(f(x)-g(x))}dx\right|+\left|\int_3^8{(f(x)-g(x))}dx\right|$$},	 %5. Antwortmoeglichkeit
								L6={},	 %6. Antwortmoeglichkeit
								L7={},	 %7. Antwortmoeglichkeit
								L8={},	 %8. Antwortmoeglichkeit
								L9={},	 %9. Antwortmoeglichkeit
								%% LOESUNG: %%
								A1=2,  % 1. Antwort
								A2=4,	 % 2. Antwort
								A3=5,  % 3. Antwort
								A4=0,  % 4. Antwort
								A5=0,  % 5. Antwort
								}
\end{beispiel}