\section{AN 4.3 - 27 - MAT - Vergleich bestimmter Integrale - MC - Matura 2019/20 1. HT}

\begin{beispiel}[AN 4.3]{1}
Gegeben sind fünf Abbildungen mit Graphen von Polynomfunktionen.

Kreuze die beiden Abbildungen an, für die gilt: $\displaystyle\int^{-1}_{-5}f(x)\dx>\displaystyle\int^{1}_{-5}\dx$.

\langmultiplechoice[5]{  %Anzahl der Antwortmoeglichkeiten, Standard: 5
				L1={\psset{xunit=0.6cm,yunit=0.6cm,algebraic=true,dimen=middle,dotstyle=o,dotsize=5pt 0,linewidth=1.pt,arrowsize=3pt 2,arrowinset=0.25}
\begin{pspicture*}(-6.8,-3.42)(3.62,4.56)
\psaxes[labelFontSize=\scriptstyle,xAxis=true,yAxis=true,showorigin=false,Dx=1.,Dy=1.,ticksize=-2pt 0,subticks=0]{->}(0,0)(-6.8,-3.42)(3.62,4.56)[$x$,140] [$f(x)$,-40]
\psplot[linewidth=2.pt,plotpoints=200]{-6.8000000000000025}{3.6200000000000014}{0.19864425297345617*x^(3.0)+0.9932212648672808*x^(2.0)-0.19864425297345617*x-0.9932212648672808}
\begin{scriptsize}
\rput[bl](-5.8,-1.72){$f$}
\end{scriptsize}
\end{pspicture*}},   %1. Antwortmoeglichkeit 
				L2={\psset{xunit=0.6cm,yunit=0.6cm,algebraic=true,dimen=middle,dotstyle=o,dotsize=5pt 0,linewidth=1.pt,arrowsize=3pt 2,arrowinset=0.25}
\begin{pspicture*}(-6.8,-3.42)(3.62,4.56)
\psaxes[labelFontSize=\scriptstyle,xAxis=true,yAxis=true,showorigin=false,Dx=1.,Dy=1.,ticksize=-2pt 0,subticks=0]{->}(0,0)(-6.8,-3.42)(3.62,4.56)[$x$,140] [$f(x)$,-40]
\psplot[linewidth=2.pt,plotpoints=200]{-6.400000000000003}{3.780000000000004}{-0.05139163908597688*x^(5.0)-0.5673194920076424*x^(4.0)-1.768749922106076*x^(3.0)-0.7743558995086735*x^(2.0)+1.820141561192053*x+1.3416753915163158}
\begin{scriptsize}
\rput[bl](-6.,2.){$f$}
\end{scriptsize}
\end{pspicture*}},   %2. Antwortmoeglichkeit
				L3={\psset{xunit=0.6cm,yunit=0.6cm,algebraic=true,dimen=middle,dotstyle=o,dotsize=5pt 0,linewidth=1.pt,arrowsize=3pt 2,arrowinset=0.25}
\begin{pspicture*}(-6.8,-3.42)(3.62,4.56)
\psaxes[labelFontSize=\scriptstyle,xAxis=true,yAxis=true,showorigin=false,Dx=1.,Dy=1.,ticksize=-2pt 0,subticks=0]{->}(0,0)(-6.8,-3.42)(3.62,4.56)[$x$,140] [$f(x)$,-40]
\psplot[linewidth=2.pt,plotpoints=200]{-6.400000000000003}{3.780000000000004}{-0.2*x^(2.0)-0.8*x+1.0}
\begin{scriptsize}
\rput[bl](-5.76,-1.45){$f$}
\end{scriptsize}
\end{pspicture*}},   %3. Antwortmoeglichkeit
				L4={\psset{xunit=0.6cm,yunit=0.6cm,algebraic=true,dimen=middle,dotstyle=o,dotsize=5pt 0,linewidth=1.pt,arrowsize=3pt 2,arrowinset=0.25}
\begin{pspicture*}(-6.8,-3.42)(3.62,4.56)
\psaxes[labelFontSize=\scriptstyle,xAxis=true,yAxis=true,showorigin=false,Dx=1.,Dy=1.,ticksize=-2pt 0,subticks=0]{->}(0,0)(-6.8,-3.42)(3.62,4.56)[$x$,140] [$f(x)$,-40]
\psplot[linewidth=2.pt,plotpoints=200]{-6.400000000000003}{3.780000000000004}{-0.1986442529734627*x^(3.0)-0.9932212648672947*x^(2.0)+0.1986442529734533*x+0.9932212648675292}
\begin{scriptsize}
\rput[bl](-5.14,1.94){$f$}
\end{scriptsize}
\end{pspicture*}},   %4. Antwortmoeglichkeit
				L5={\psset{xunit=0.6cm,yunit=0.6cm,algebraic=true,dimen=middle,dotstyle=o,dotsize=5pt 0,linewidth=1.pt,arrowsize=3pt 2,arrowinset=0.25}
\begin{pspicture*}(-6.8,-3.42)(3.62,4.56)
\psaxes[labelFontSize=\scriptstyle,xAxis=true,yAxis=true,showorigin=false,Dx=1.,Dy=1.,ticksize=-2pt 0,subticks=0]{->}(0,0)(-6.8,-3.42)(3.62,4.56)[$x$,140] [$f(x)$,-40]
\psplot[linewidth=2.pt,plotpoints=200]{-6.400000000000003}{3.780000000000004}{0.05139163908597292*x^(5.0)+0.5673194920075832*x^(4.0)+1.7687499221057537*x^(3.0)+0.7743558995079038*x^(2.0)-1.820141561192802*x-1.341675391516498}
\begin{scriptsize}
\rput[bl](-5.46,-2.72){$f$}
\end{scriptsize}
\end{pspicture*}},	 %5. Antwortmoeglichkeit
				L6={},	 %6. Antwortmoeglichkeit
				L7={},	 %7. Antwortmoeglichkeit
				L8={},	 %8. Antwortmoeglichkeit
				L9={},	 %9. Antwortmoeglichkeit
				%% LOESUNG: %%
				A1=1,  % 1. Antwort
				A2=5,	 % 2. Antwort
				A3=0,  % 3. Antwort
				A4=0,  % 4. Antwort
				A5=0,  % 5. Antwort
				}
\end{beispiel}