\section{AN 4.3 - 3 Aussagen �ber bestimmte Integrale - MC - BIFIE}

\begin{beispiel}[AN 4.3]{1} %PUNKTE DES BEISPIELS
				Die stetige reelle Funktion $f$ mit dem abgebildeten Graphen hat Nullstellen bei $x_1=1,x_2=3$ und $x_3=6$.
				\begin{center}
					\resizebox{0.8\linewidth}{!}{\psset{xunit=1.0cm,yunit=1.0cm,algebraic=true,dimen=middle,dotstyle=o,dotsize=5pt 0,linewidth=0.8pt,arrowsize=3pt 2,arrowinset=0.25}
\begin{pspicture*}(-0.8064027241051648,-2.83200511447316)(6.750476169359091,2.8625149995591515)
\multips(0,-2)(0,1.0){6}{\psline[linestyle=dashed,linecap=1,dash=1.5pt 1.5pt,linewidth=0.4pt,linecolor=lightgray]{c-c}(-0.8064027241051648,0)(6.750476169359091,0)}
\multips(0,0)(1.0,0){8}{\psline[linestyle=dashed,linecap=1,dash=1.5pt 1.5pt,linewidth=0.4pt,linecolor=lightgray]{c-c}(0,-2.83200511447316)(0,2.8625149995591515)}
\psaxes[labelFontSize=\scriptstyle,xAxis=true,yAxis=true,Dx=1.,Dy=1.,ticksize=-2pt 0,subticks=2]{->}(0,0)(-0.8064027241051648,-2.83200511447316)(6.750476169359091,2.8625149995591515)[x,140] [f(x),-40]
\psplot[linewidth=1.2pt,plotpoints=200]{-0.8064027241051648}{6.750476169359091}{0.2992347613554581*x^(3.0)-2.992347613554581*x^(2.0)+8.079338556597369*x-5.386225704398246}
\end{pspicture*}}
				\end{center}
				Welche der folgenden Aussagen ist/sind zutreffend? Kreuze die zutreffende(n) Aussage(n) an!
				
				\multiplechoice[5]{  %Anzahl der Antwortmoeglichkeiten, Standard: 5
								L1={$$\int_1^3{f(x)}dx<2$$},   %1. Antwortmoeglichkeit 
								L2={$$\int_1^6{f(x)}dx<0$$},   %2. Antwortmoeglichkeit
								L3={$$\left|\int_3^6{f(x)}dx\right|<6$$},   %3. Antwortmoeglichkeit
								L4={$$\int_1^3{f(x)}dx+\int_3^6{f(x)}dx>0$$},   %4. Antwortmoeglichkeit
								L5={$$\int_1^3{f(x)}dx>0$$ und $$\int_3^6{f(x)}dx<0$$},	 %5. Antwortmoeglichkeit
								L6={},	 %6. Antwortmoeglichkeit
								L7={},	 %7. Antwortmoeglichkeit
								L8={},	 %8. Antwortmoeglichkeit
								L9={},	 %9. Antwortmoeglichkeit
								%% LOESUNG: %%
								A1=1,  % 1. Antwort
								A2=2,	 % 2. Antwort
								A3=3,  % 3. Antwort
								A4=5,  % 4. Antwort
								A5=0,  % 5. Antwort
								}
\end{beispiel}