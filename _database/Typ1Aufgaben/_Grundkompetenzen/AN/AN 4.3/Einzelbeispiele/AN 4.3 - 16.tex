\section{AN 4.3 - 16 - MAT - Schadstoffaussto� - OA - Matura 2016/17 2. NT}

\begin{beispiel}{1} %PUNKTE DES BEISPIELS
An einem Wintertag wird der Schadstoffaussto� eines Kamins gemessen. Die Funktion $A:~\mathbb R^+ \to \mathbb R^+$ beschreibt in Abh�ngigkeit von der Zeit $t$ den momentanen Schadstoffaussto� $A(t)$, wobei $A(t)$ in Gramm pro Stunde und $t$ in Stunden ($t=0$ entspricht 0 Uhr) gemessen wird. \leer

Deute den Ausdruck $\displaystyle\int^{15}_7\! A(t)\,dt$ im gegebenen Kontext!

\antwort{Der Ausdruck gibt den gesamten Schadstoffaussto� (in Gramm) von 7 Uhr bis 15 Uhr an.}
\end{beispiel}