\section{AN 4.3 - 9 - MAT - Durchflussrate - OA - Matura 2014/15 - Nebentermin 1}

\begin{beispiel}[AN 4.3]{1} %PUNKTE DES BEISPIELS
In einem Wasserrohr wird durch einen Sensor die Durchflussrate (= Durchflussmenge pro Zeiteinheit) gemessen. Die Funktion $D$ ordnet jedem Zeitpunkt $t$ die Durchflussrate $D(t)$ zu. Dabei wird $t$ in Minuten und $D(t)$ in Litern pro Minute angegeben. \leer

Gib die Bedeutung der Zahl $\displaystyle\int_{60}^{120}{D(t)\,dt}$ im vorliegenden Kontext an.

\antwort{
Der Ausdruck beschreibt die durch das Rohr geflossene Wassermenge (in Litern) vom Zeitpunkt $t=60$ bis zum Zeitpunkt $t=120$.}
\end{beispiel}