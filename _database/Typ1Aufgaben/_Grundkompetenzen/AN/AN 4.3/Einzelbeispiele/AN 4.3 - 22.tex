\section{AN 4.3 - 22 - MAT - Wurfhöhe eines Körpers - OA - Matura 1.NT 2018/19}

\begin{beispiel}[AN 4.3]{1}
Ein Körper wird aus einer Höhe von 1\,m über dem Erdboden senkrecht nach oben geworfen. Die Geschwindigkeit des Körpers nach $t$ Sekunden wird modellhaft durch die Funktion $v$ mit $v(t)=15-10\cdot t$ beschrieben ($v(t)$ in Metern pro Sekunde, $t$ in Sekunden).

Gib diejenige Höhe (in Metern) über dem Erdboden an, in der sich der Körper nach 2\,s befindet.

\antwort{mögliche Vorgehensweise:\\
$v(t)=15-10\cdot t$\\
$s(t)=15\cdot t-5\cdot t^2+h_0$\\
$s(0)=1=h_0$\\
$s(t)=15\cdot t-5\cdot t^2+1$\\
$s(2)=30-20+1=11$

Der Körper befindet sich nach 2\,s in einer Höhe von 11\,m über dem Erdboden.

Ein Punkt für die richtige Lösung, wobei die Einheit „m“ nicht angeführt sein muss.
Die Aufgabe ist auch dann als richtig gelöst zu werten, wenn bei korrektem Ansatz das Ergebnis
aufgrund eines Rechenfehlers nicht richtig ist.}
\end{beispiel}