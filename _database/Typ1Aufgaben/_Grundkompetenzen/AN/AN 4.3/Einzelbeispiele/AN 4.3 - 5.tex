\section{AN 4.3 - 5 Fl�che zwischen zwei Kurven - MC - BIFIE}

\begin{beispiel}[AN 4.3]{1} %PUNKTE DES BEISPIELS
Die Funktionsgraphen von $f$ und $g$ schlie�en ein gemeinsames Fl�chenst�ck ein. 

\begin{center}
\newrgbcolor{uququq}{0.25098039215686274 0.25098039215686274 0.25098039215686274}
\psset{xunit=1.0cm,yunit=1.0cm,algebraic=true,dimen=middle,dotstyle=o,dotsize=5pt 0,linewidth=0.8pt,arrowsize=3pt 2,arrowinset=0.25}
\begin{pspicture*}(-2.4759718077494743,-3.384122420216285)(6.6618232318926704,2.6150386710270324)
\multips(0,-3)(0,1.0){6}{\psline[linestyle=dashed,linecap=1,dash=1.5pt 1.5pt,linewidth=0.4pt,linecolor=gray]{c-c}(-2.4759718077494743,0)(6.6618232318926704,0)}
\multips(-2,0)(1.0,0){10}{\psline[linestyle=dashed,linecap=1,dash=1.5pt 1.5pt,linewidth=0.4pt,linecolor=gray]{c-c}(0,-3.384122420216285)(0,2.6150386710270324)}
\psaxes[labelFontSize=\scriptstyle,xAxis=true,yAxis=true,Dx=1.,Dy=1.,ticksize=-2pt 0,subticks=2]{->}(0,0)(-2.4759718077494743,-3.384122420216285)(6.6618232318926704,2.6150386710270324)[x,140] [y,-40]
\pscustom[linecolor=uququq,fillcolor=uququq,fillstyle=solid,opacity=0.25]{\psplot{-1.}{6.}{0.25*x^(2.0)-x-1.25}\lineto(6.,1.760295797992526)\psplot{6.}{-1.}{0.015721239059366534*x^(3.0)-0.1853729697133834*x^(2.0)+0.6909772660112011*x+0.892071474783951}\lineto(-1.,0.)\closepath}
\psplot[plotpoints=200]{-2.4759718077494743}{6.6618232318926704}{0.25*x^(2.0)-x-1.25}
\psplot[plotpoints=200]{-2.4759718077494743}{6.6618232318926704}{0.015721239059366534*x^(3.0)-0.1853729697133834*x^(2.0)+0.6909772660112011*x+0.892071474783951}
\begin{scriptsize}
\psdots[dotsize=3pt 0,dotstyle=*](-1.,0.)
\rput[bl](-2.3567831767976206,1.5){$g$}
\psdots[dotsize=3pt 0,dotstyle=*](6.005606333326987,1.7612205243973174)
\rput[bl](-2.2971888613216933,-1.4){$f$}
\end{scriptsize}
\end{pspicture*}
\end{center}

Mit welchen der nachstehenden Berechnungsvorschriften kann man den Fl�cheninhalt des gekennzeichneten Fl�chenst�cks ermitteln? \leer

Kreuze die beiden zutreffenden Berechnungsvorschriften an.

\multiplechoice[5]{  %Anzahl der Antwortmoeglichkeiten, Standard: 5
				L1={$$\int_{-1}^{6}{[g(x)-f(x)]dx}$$},   %1. Antwortmoeglichkeit 
				L2={$$\int_{-1}^{6}{[f(x)-g(x)]dx}$$},   %2. Antwortmoeglichkeit
				L3={$$\int_{-1}^{6}{f(x)dx}+\int_{5}^{6}{g(x)dx}-\int_{-1}^{5}{g(x)dx}$$},   %3. Antwortmoeglichkeit
				L4={$$\int_{-1}^{6}{f(x)dx}+\int_{-1}^{6}{g(x)dx}$$},   %4. Antwortmoeglichkeit
				L5={$$\int_{-1}^{6}{f(x)dx}-\int_{5}^{6}{g(x)dx}+\left|\int_{-1}^{5}{g(x)dx}\right|$$},	 %5. Antwortmoeglichkeit
				L6={},	 %6. Antwortmoeglichkeit
				L7={},	 %7. Antwortmoeglichkeit
				L8={},	 %8. Antwortmoeglichkeit
				L9={},	 %9. Antwortmoeglichkeit
				%% LOESUNG: %%
				A1=2,  % 1. Antwort
				A2=5,	 % 2. Antwort
				A3=0,  % 3. Antwort
				A4=0,  % 4. Antwort
				A5=0,  % 5. Antwort
				}

\end{beispiel}