\section{AN 1.2 - 9 - MAT - Differenzenquotient und Differentialquotient - MC - Matura 2018/19 2. NT}

\begin{beispiel}[AN 1.2]{1}
Nachstehend ist der Graph einer Polynomfunktion $f$ zweiten Grades abgebildet. Zusätzlich sind vier Punkte auf dem Graphen mit den $x$-Koordinaten $x_1, x_2, x_3$ und $x_4$ eingezeichnet.

\begin{center}
\psset{xunit=1.0cm,yunit=1.0cm,algebraic=true,dimen=middle,dotstyle=o,dotsize=5pt 0,linewidth=1.6pt,arrowsize=3pt 2,arrowinset=0.25}
\begin{pspicture*}(-0.66,-1.)(8.98,6.4)
\psaxes[labelFontSize=\scriptstyle,xAxis=true,yAxis=true,labels=none,Dx=1.,Dy=1.,ticks=none]{->}(0,0)(-0.66,-1.)(8.98,6.4)[$x$,140] [$f(x)$,-40]
\psplot[linewidth=2.pt,plotpoints=200]{-0.6600000000000003}{8.980000000000004}{0.2*(x-2.0)^(2.0)+1.0}
\psline[linewidth=2.pt](0.7227653404212884,0.)(0.7227653404212884,1.3262656751258295)
\psline[linewidth=2.pt](2.5680578014886875,0.)(2.5680578014886875,1.0645379331664322)
\psline[linewidth=2.pt](4.248200471379322,0.)(4.248200471379322,2.010881071902041)
\psline[linewidth=2.pt](6.516517766805217,0.)(6.516517766805217,5.079786547573437)
\rput[tl](7.04,5.76){$f$}
\rput[tl](0.65,-0.15){$x_1$}
\rput[tl](2.45,-0.15){$x_2$}
\rput[tl](4.15,-0.15){$x_3$}
\rput[tl](6.4,-0.15){$x_4$}
\begin{scriptsize}
\psdots[dotsize=6pt 0,dotstyle=*](0.7227653404212884,1.3262656751258295)
\psdots[dotsize=6pt 0,dotstyle=*](2.5680578014886875,1.0645379331664322)
\psdots[dotsize=6pt 0,dotstyle=*](4.248200471379322,2.010881071902041)
\psdots[dotsize=6pt 0,dotstyle=*](6.516517766805217,5.079786547573437)
\end{scriptsize}
\end{pspicture*}
\end{center}

Kreuze die beiden auf die Funktion $f$ zutreffenden Aussagen an.

\multiplechoice[5]{  %Anzahl der Antwortmoeglichkeiten, Standard: 5
				L1={Der Differenzenquotient für das Intervall $[x_1;x_2]$ ist kleiner als der Differentialquotient an der Stelle $x_1$.},   %1. Antwortmoeglichkeit 
				L2={Der Differenzenquotient für das Intervall $[x_1;x_3]$ ist kleiner als der Differentialquotient an der Stelle $x_3$.},   %2. Antwortmoeglichkeit
				L3={Der Differenzenquotient für das Intervall $[x_1;x_4]$ ist kleiner als der Differentialquotient an der Stelle $x_2$.},   %3. Antwortmoeglichkeit
				L4={Der Differenzenquotient für das Intervall $[x_2;x_4]$ ist größer als der Differentialquotient an der Stelle $x_2$.},   %4. Antwortmoeglichkeit
				L5={Der Differenzenquotient für das Intervall $[x_3;x_4]$ ist größer als der Differentialquotient an der Stelle $x_4$.},	 %5. Antwortmoeglichkeit
				L6={},	 %6. Antwortmoeglichkeit
				L7={},	 %7. Antwortmoeglichkeit
				L8={},	 %8. Antwortmoeglichkeit
				L9={},	 %9. Antwortmoeglichkeit
				%% LOESUNG: %%
				A1=2,  % 1. Antwort
				A2=4,	 % 2. Antwort
				A3=0,  % 3. Antwort
				A4=0,  % 4. Antwort
				A5=0,  % 5. Antwort
				}
\end{beispiel}