\section{AN 1.2 - 5 - MAT - Änderungsraten einer Polynomfunktion - MC - Matura NT 2 15/16}

\begin{beispiel}[AN 1.2]{1} %PUNKTE DES BEISPIELS
Gegeben ist der Graph einer Polynomfunktion $f$.

\begin{center}
\psset{xunit=0.6cm,yunit=0.6cm,algebraic=true,dimen=middle,dotstyle=o,dotsize=5pt 0,linewidth=0.8pt,arrowsize=3pt 2,arrowinset=0.25}
\begin{pspicture*}(-4.62,-3.72)(11.66,6.94)
\multips(0,-3)(0,1.0){11}{\psline[linestyle=dashed,linecap=1,dash=1.5pt 1.5pt,linewidth=0.4pt,linecolor=gray]{c-c}(-4.62,0)(11.66,0)}
\multips(-4,0)(1.0,0){17}{\psline[linestyle=dashed,linecap=1,dash=1.5pt 1.5pt,linewidth=0.4pt,linecolor=gray]{c-c}(0,-3.72)(0,6.94)}
\begin{scriptsize}
\psaxes[xAxis=true,yAxis=true,showorigin=false,Dx=1.,Dy=1.,ticksize=-2pt 0,subticks=0]{->}(0,0)(-4.62,-3.72)(11.66,6.94)[$x$,140] [$f(x)$,-40]
\psplot[linewidth=1.pt,plotpoints=200]{-4.62}{11.659999999999991}{-0.02263888888888889*x^(4.0)+0.25055555555555553*x^(3.0)-0.35180555555555554*x^(2.0)-2.255*x+5.0}
\rput[bl](-3.56,-2.12){$f$}
\end{scriptsize}
\end{pspicture*}
\end{center}

Kreuze die beiden zutreffenden Aussagen an!

\multiplechoice[5]{  %Anzahl der Antwortmoeglichkeiten, Standard: 5
				L1={Der Differenzialquotient an der Stelle $x=6$ ist größer als der Differenzialquotient an der Stelle $x=-3$.},   %1. Antwortmoeglichkeit 
				L2={Der Differenzialquotient an der Stelle $x=1$ ist negativ.},   %2. Antwortmoeglichkeit
				L3={Der Differenzialquotient im Intervall $[-3;0]$ ist 1.},   %3. Antwortmoeglichkeit
				L4={Die mittlere Änderungsrate ist in keinem Intervall gleich 0.},   %4. Antwortmoeglichkeit
				L5={Der Differenzialquotient im Intervall $[3;6]$ ist nicht negativ.},	 %5. Antwortmoeglichkeit
				L6={},	 %6. Antwortmoeglichkeit
				L7={},	 %7. Antwortmoeglichkeit
				L8={},	 %8. Antwortmoeglichkeit
				L9={},	 %9. Antwortmoeglichkeit
				%% LOESUNG: %%
				A1=2,  % 1. Antwort
				A2=5,	 % 2. Antwort
				A3=0,  % 3. Antwort
				A4=0,  % 4. Antwort
				A5=0,  % 5. Antwort
				}
\end{beispiel}