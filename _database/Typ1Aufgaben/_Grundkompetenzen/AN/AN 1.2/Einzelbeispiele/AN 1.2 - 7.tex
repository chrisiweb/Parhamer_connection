\section{AN 1.2 - 7 - Wasserstand eines Flusses - OA - Matura - 1. NT 2017/18}

\begin{beispiel}[AN 1.2]{1}
Die Funktion $W\!:[0;24]\rightarrow\mathbb{R}^+_0$ ordnet jedem Zeitpunkt $t$ den Wasserstand $W(t)$ eines Flusses an einer bestimmten Messstelle zu. Dabei wird $t$ in Stunden und $W(t)$ in Metern angegeben.

Interpretiere den nachstehenden Ausdruck im Hinblick auf den Wasserstand $W(t)$ des Flusses!\leer

$\lim\limits_{\Delta t\rightarrow 0}\dfrac{W(6+\Delta t)-W(6)}{\Delta t}$

\antwort{Der Ausdruck beschreibt die Änderungsgeschwindigkeit (momentane Änderungsrate) in m/h des Wasserstands $W(t)$ zum Zeitpunkt $t=6$ an dieser Messstelle des Flusses.}
\end{beispiel}