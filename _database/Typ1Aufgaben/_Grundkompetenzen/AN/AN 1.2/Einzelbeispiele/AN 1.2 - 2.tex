\section{AN 1.2 - 2 Bewegung eines Körpers - LT - BIFIE}


\begin{beispiel}[AN 1.2]{1} %PUNKTE DES BEISPIELS
Bei der Bewegung eines Körpers gibt die Zeit-Weg-Funktion seine Entfernung $s$ (in m) vom Ausgangspunkt seiner Bewegung nach $t$ Sekunden an. 
Der Differenzenquotient $\dfrac{s(t_2)-s(t_1)}{t_2-t_1}$ gibt seine mittlere Geschwindigkeit im Zeitintervall $[t_1;t_2]$ an.

\lueckentext{
				text={Der Ausdruck $\lim\limits_{t_2\rightarrow t_1}{\dfrac{s(t_2)-s(t_1)}{t_2-t_1}}$ gibt dir \gap \gap an.}, 	%Lueckentext Luecke=\gap
				L1={Momentangeschwindigkeit }, 		%1.Moeglichkeit links  
				L2={Momentanbeschleunigung }, 		%2.Moeglichkeit links
				L3={durchschnittliche Beschleunigung }, 		%3.Moeglichkeit links
				R1={zwischen den Zeitpunkten $t_1$ und $t_2$}, 		%1.Moeglichkeit rechts 
				R2={zum Zeitpunkt $t_1$ }, 		%2.Moeglichkeit rechts
				R3={zum Zeitpunkt $t_2$ }, 		%3.Moeglichkeit rechts
				%% LOESUNG: %%
				A1=1,   % Antwort links
				A2=2		% Antwort rechts 
				}
\end{beispiel}