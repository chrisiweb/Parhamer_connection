\section{AN 1.2 - 6 - MAT - Differenzenquotient - Differentialquotient - MC - Matura 2013/14 1. Nebentermin}

\begin{beispiel}[AN 1.2]{1} %PUNKTE DES BEISPIELS
				Gegeben ist der Graph einer Polynomfunktion $f$:
				
				\begin{center}\resizebox{0.7\linewidth}{!}{\psset{xunit=1.0cm,yunit=1.0cm,algebraic=true,dimen=middle,dotstyle=o,dotsize=5pt 0,linewidth=0.8pt,arrowsize=3pt 2,arrowinset=0.25}
\begin{pspicture*}(-5.832548135104837,-3.771792653966293)(5.891911065818181,5.707557338269345)
\multips(0,-3)(0,1.0){10}{\psline[linestyle=dashed,linecap=1,dash=1.5pt 1.5pt,linewidth=0.4pt,linecolor=darkgray]{c-c}(-5.832548135104837,0)(5.891911065818181,0)}
\multips(-5,0)(1.0,0){12}{\psline[linestyle=dashed,linecap=1,dash=1.5pt 1.5pt,linewidth=0.4pt,linecolor=darkgray]{c-c}(0,-3.771792653966293)(0,5.707557338269345)}
\psaxes[labelFontSize=\scriptstyle,xAxis=true,yAxis=true,Dx=1.,Dy=1.,ticksize=-2pt 0,subticks=2]{->}(0,0)(-5.832548135104837,-3.771792653966293)(5.891911065818181,5.707557338269345)[x,140] [f(x),-40]
\psplot[linewidth=1.2pt,plotpoints=200]{-5.832548135104837}{5.891911065818181}{0.09804004429333255*x^(4.0)+0.040538292112993014*x^(3.0)-1.404194081776737*x^(2.0)+0.008372796173848979*x+2.9967958492616105}
\rput[tl](-3.886786820909102,4.609948391799956){f}
\end{pspicture*}}\end{center}

Kreuze die beiden zutreffenden Aussagen an!\leer

\multiplechoice[5]{  %Anzahl der Antwortmoeglichkeiten, Standard: 5
				L1={$\frac{f(3)-f(-3)}{6}=0$},   %1. Antwortmoeglichkeit 
				L2={$\frac{f(3)-f(0)}{3}<0$},   %2. Antwortmoeglichkeit
				L3={$f'(3)=0$},   %3. Antwortmoeglichkeit
				L4={$f'(-2)>0$},   %4. Antwortmoeglichkeit
				L5={$f'(-1)=f'(1)$},	 %5. Antwortmoeglichkeit
				L6={},	 %6. Antwortmoeglichkeit
				L7={},	 %7. Antwortmoeglichkeit
				L8={},	 %8. Antwortmoeglichkeit
				L9={},	 %9. Antwortmoeglichkeit
				%% LOESUNG: %%
				A1=2,  % 1. Antwort
				A2=4,	 % 2. Antwort
				A3=0,  % 3. Antwort
				A4=0,  % 4. Antwort
				A5=0,  % 5. Antwort
				}
\end{beispiel}