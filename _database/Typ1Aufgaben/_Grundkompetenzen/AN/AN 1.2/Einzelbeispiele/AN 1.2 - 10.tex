\section{AN 1.2 - 10 - Krafteinwirkung auf ein Teilchen - OA - MarPar UNIVIE}

\begin{beispiel}[AN 1.2]{1}
Die Funktion $F\!:[0;60]\rightarrow\mathbb{R}$ beschreibt die Krafteinwirkung auf ein elektrisch geladenes Teilchen, welches sich durch ein magnetisches Feld bewegt. Sie ordnet jedem Zeitpunkt $t$ die Kraft $F(t)$ an einer bestimmten Messstelle zu. Dabei wird $t$ in Sekunden und $F(t)$ in Newton angegeben.

Interpretiere den nachstehenden Ausdruck im Hinblick auf die Krafteinwirkung $F(t)$ auf das Teilchen!\leer

$\lim\limits_{\Delta t\rightarrow 0}\dfrac{F(10+\Delta t)-F(10)}{\Delta t}$

\antwort{Der Ausdruck beschreibt die momentane Änderungsrate der Kraft auf das Teilchen zum Zeitpunkt $t=10$ in $\frac{N}{s}$.

Der Punkt kann auch dann gegeben werden, wenn die Einheit in der Interpretation fehlt.}
\end{beispiel}