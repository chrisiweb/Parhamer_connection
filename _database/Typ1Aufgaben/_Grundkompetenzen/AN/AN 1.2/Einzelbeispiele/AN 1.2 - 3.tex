\section{AN 1.2 - 3 Mittlere �nderungsrate interpretieren - MC - Matura 2015/16 - Haupttermin}

\begin{beispiel}[AN 1.2]{1} %PUNKTE DES BEISPIELS
Gegeben ist eine Polynomfunktion $f$ dritten Grades. Die mittlere �nderungsrate von $f$ hat im Intervall $[x_1;~ x_2]$ den Wert 5. \leer

Welche der nachstehenden Aussagen k�nnen �ber die Funktion $f$ sicher getroffen werden?
Kreuze die beiden zutreffenden Aussagen an.

\multiplechoice[5]{  %Anzahl der Antwortmoeglichkeiten, Standard: 5
				L1={Im Intervall $[x_1;~x_2]$ gibt es mindestens eine Stelle $x$ mit $f(x) = 5$.},   %1. Antwortmoeglichkeit 
				L2={$f(x_2)>f(x_1)$},   %2. Antwortmoeglichkeit
				L3={Die Funktion $f$ ist im Intervall $[x_1; ~x_2]$ monoton steigend.},   %3. Antwortmoeglichkeit
				L4={$f'(x) = 5$ f�r alle $x \in [x_1; ~x_2]$},   %4. Antwortmoeglichkeit
				L5={$f(x_2)-f(x_1)=5\cdot (x_2-x_1)$},	 %5. Antwortmoeglichkeit
				L6={},	 %6. Antwortmoeglichkeit
				L7={},	 %7. Antwortmoeglichkeit
				L8={},	 %8. Antwortmoeglichkeit
				L9={},	 %9. Antwortmoeglichkeit
				%% LOESUNG: %%
				A1=2,  % 1. Antwort
				A2=5,	 % 2. Antwort
				A3=0,  % 3. Antwort
				A4=0,  % 4. Antwort
				A5=0,  % 5. Antwort
				}
\end{beispiel}