\section{AN 1.3 - 5 Differenzenquotient - OA - BIFIE}


\begin{beispiel}[AN 1.3]{1} %PUNKTE DES BEISPIELS
Eine Funktion $s:[0;6]\rightarrow \mathbb{R}$ beschreibt den von einem Radfahrer innerhalb von $t$ Sekunden zur�ckgelegten Weg.

\leer

Es gilt: $s(t)=\frac{1}{2}t^2+2t$.

Der zur�ckgelegte Weg wird dabei in Metern angegeben, die Zeit wird ab dem Zeitpunkt $t_0=0$ in Sekunden gemessen.

\leer

Ermittle den Differenzenquotienten der Funktion $s$ im Intervall $[0; 6]$ und deute das Ergebnis.

\antwort{$\dfrac{s(6)-s(0)}{6-0}=\dfrac{30-0}{6}=5$

Das Ergebnis bedeutet, dass die mittlere Geschwindigkeit (auch Durchschnittsgeschwindigkeit) des Radfahrers im Zeitintervall $[0; 6]$ 5\,m/s betr�gt.} 
\end{beispiel}