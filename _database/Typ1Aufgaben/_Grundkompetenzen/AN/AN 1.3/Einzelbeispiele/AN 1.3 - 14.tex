\section{AN 1.3 - 14 Abkühlungsprozess - OA - Matura 17/18}

\begin{beispiel}[AN 1.3]{1} %PUNKTE DES BEISPIELS
Eine Flüssigkeit wird abgekühlt. Die Funktion $T$ beschreibt modellhaft den Temperaturverlauf. Dabei gibt $T(t)$ die Temperatur der Flüssigkeit zum Zeitpunkt $t\geq 0$ an ($T(t)$ in $^\circ$C, $t$ in Minuten). Der Abkühlungsprozess startet zum Zeitpunkt $t=0$.

Interpretiere die Gleichung $T'(20)=-0,97$ im gegebenen Kontext unter Angabe der korrekten Einheiten!

\antwort{Die momentane Abnahme der Temperatur der Flüssigkeit beträgt 20 Minuten nach dem Start des Abkühlungsprozesses 0,97\,$^\circ$C pro Minute.}
\end{beispiel}