\section{AN 1.3 - 17 - MAT - Bewegung - OA - Matura 2018/19 2. NT}

\begin{beispiel}[AN 1.3]{1}
Ein Körper startet seine geradlinige Bewegung zum Zeitpunkt $t=0$.

Die Funktion $v$ ordnet jedem Zeitpunkt $t$ die Geschwindigkeit $v(t)$ des Körpers zum Zeitpunkt $t$ zu ($t$ in s, $v(t)$ in m/s).

Interpretiere die Gleichung $v'(3)=1$ im gegebenen Kontext unter Verwendung der entsprechenden Einheit.

\antwort{mögliche Interpretation:

Zum Zeitpunkt $t=3$ beträgt die Beschleunigung des Körpers 1\,m/s$^2$.}
\end{beispiel}