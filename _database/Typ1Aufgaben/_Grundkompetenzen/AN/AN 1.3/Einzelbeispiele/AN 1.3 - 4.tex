\section{AN 1.3 - 4 - Differenzenquotient - LT - BIFIE}

\begin{beispiel}[AN 1.3]{1} %PUNKTE DES BEISPIELS
Die nachstehende Abbildung zeigt den Graphen einer Funktion $f$ mit einer Sekante.

\begin{center}
\newrgbcolor{xdxdff}{0.49019607843137253 0.49019607843137253 1.}
\psset{xunit=1.0cm,yunit=1.0cm,algebraic=true,dimen=middle,dotstyle=o,dotsize=5pt 0,linewidth=0.8pt,arrowsize=3pt 2,arrowinset=0.25}
\begin{pspicture*}(-0.66845823698629,-0.7198611834877261)(7,4.6631148574428485)
\psaxes[labelFontSize=\scriptstyle,xAxis=true,yAxis=true,labels=none,Dx=1.,Dy=1.,ticksize=0pt 0,subticks=0]{->}(0,0)(-0.66845823698629,-0.7198611834877261)(6.623496527724776,4.6631148574428485)[x,140] [f(x),-40]
\psplot{-0.66845823698629}{6.623496527724776}{(--1.8271604938271606--2.962962962962963*x)/4.}
\psline(1.,1.1975308641975309)(1.,0.)
\psline(5.,4.160493827160494)(5.,0.)
\rput[tl](1.1,0.7){$f(x_0)$}
\rput[tl](5.041279927834639,2.5){$f(x_0+h)$}
\rput[tl](2.9,1.3){$f$}
\rput[tl](0.7933723172600322,-0.15){$x_0$}
\rput[tl](4.559735745259379,-0.1){$x_0+h$}
\rput[tl](3,-0.13){$h$}
\psplot[linewidth=1.pt,plotpoints=200]{-4.3}{10.06}{0.32176578540460665*x^(2.0)-1.1898539716868992*x+2.0656190504798233}

\begin{scriptsize}
\psdots[dotsize=3pt 0,dotstyle=*,linecolor=darkgray](1.,1.1975308641975309)
\psdots[dotsize=3pt 0,dotstyle=*,linecolor=darkgray](5.,4.160493827160494)
\psdots[dotsize=3pt 0,dotstyle=*,linecolor=darkgray](1.,0.)
\psdots[dotsize=3pt 0,dotstyle=*,linecolor=darkgray](5.,0.)
\end{scriptsize}
\end{pspicture*}
\end{center}

\lueckentext{
				text={Der Ausdruck \gap beschriebt die \gap.}, 	%Lueckentext Luecke=\gap
				L1={$\dfrac{f(x)-f(x_0)}{h}$}, 		%1.Moeglichkeit links  
				L2={$\dfrac{f(x_0+h)-f(x_0)}{h}$}, 		%2.Moeglichkeit links
				L3={$\dfrac{f(x_0+h)-f(x_0)}{x_0}$}, 		%3.Moeglichkeit links
				R1={die Steigung von $f$ an der Stelle $x$}, 		%1.Moeglichkeit rechts 
				R2={die 1. Ableitung der Funktion $f$}, 		%2.Moeglichkeit rechts
				R3={die mittlere Änderungsrate im
Intervall $[x_0; x_0 + h]$}, 		%3.Moeglichkeit rechts
				%% LOESUNG: %%
				A1=2,   % Antwort links
				A2=3		% Antwort rechts 
				}

\end{beispiel}