\section{AN 1.3 - 9 - MAT - Mittlere Änderungsrate der Temperatur - OA - Matura 2014/15 Haupttermin}

\begin{beispiel}[AN 1.3]{1} %PUNKTE DES BEISPIELS
Ein bestimmter Temperaturverlauf wird modellhaft durch eine Funktion $T$ beschrieben.
Die Funktion $T:~[0; 60] \rightarrow \mathbb{R}$ ordnet jedem Zeitpunkt $t$ eine Temperatur $T(t)$ zu. Dabei wird $t$ in Minuten und $T(t)$ in Grad Celsius angegeben.

Stelle die mittlere Änderungsrate $D$ der Temperatur im Zeitintervall $[20;30]$ durch den Term dar. \leer

$D= \antwort[\rule{6cm}{0.3pt}]{\dfrac{T(39)-T(20)}{10}}$ $^\circ$\,C/min
\end{beispiel}