\section{AN 1.3 - 12 Finanzschulden - OA - Matura 2016/17 - Haupttermin}

\begin{beispiel}[AN 1.3]{1} %PUNKTE DES BEISPIELS
Die Finanzschulden Österreichs haben im Zeitraum 2000 bis 2010 zugenommen. Im Jahr 2000
betrugen die Finanzschulden Österreichs $F_0$, zehn Jahre später betrugen sie $F_1$ (jeweils in Milliarden Euro). \leer

Interpretieren Sie den Ausdruck $\frac{F_1-F_0}{10}$ im Hinblick auf die Entwicklung der Finanzschulden Österreichs!

\antwort{Der Ausdruck beschreibt die durchschnittliche jährliche Zunahme (durchschnittliche jährliche Änderung) der Finanzschulden Österreichs (in Milliarden Euro im angegebenen Zeitraum).}
\end{beispiel}
