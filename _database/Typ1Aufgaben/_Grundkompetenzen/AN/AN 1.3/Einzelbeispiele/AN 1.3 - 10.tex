\section{AN 1.3 - 10 - MAT - Aktienkurs - OA - Matura 2015/16 Nebentermin 1}

\begin{beispiel}[AN 1.3]{1} %PUNKTE DES BEISPIELS
Ab dem Zeitpunkt $t=0$ wird der Kurs einer Aktie (in Euro) beobachtet und dokumentiert. $A(t)$ beschreibt den Kurs der Aktie nach $t$ Tagen. 

Es wird folgender Wert berechnet:

$\dfrac{A(10)-A(0)}{10}=2$

Gib an, was dieser Wert im Hinblick auf die Entwicklung des Aktienkurses aussagt. 

\antwort{Der Kurs der Aktie ist in den (ersten) 10 Tagen um durchschnittlich 2 Euro pro Tag gestiegen.}
\end{beispiel}