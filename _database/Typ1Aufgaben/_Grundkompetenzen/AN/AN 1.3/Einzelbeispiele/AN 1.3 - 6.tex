\section{AN 1.3 - 6 - Freier Fall eines Körpers - MC - BIFIE}

\begin{beispiel}[AN 1.3]{1} %PUNKTE DES BEISPIELS
Die Funktion $s$ mit $s(t)=\frac{g}{2}\cdot t^2 ~ (g\approx 10\,\text{m/s}^2)$ beschreibt annähernd den von einem Körper in der Zeit $t$ (in Sekunden) im freien Fall zurückgelegten Weg $s(t)$ (in m).

Kreuze die zutreffende(n) Aussage(n) an.

\multiplechoice[5]{  %Anzahl der Antwortmoeglichkeiten, Standard: 5
				L1={Die erste Ableitung $s'$ der Funktion $s$ an der Stelle $t_1$ beschreibt die Momentangeschwindigkeit
des Körpers zum Zeitpunkt $t_1$.},   %1. Antwortmoeglichkeit 
				L2={Die zweite Ableitung $s''$ der Funktion $s$ an der Stelle $t_1$ beschreibt die momentane
Änderungsrate der Geschwindigkeit zum Zeitpunkt $t_1$. 
},   %2. Antwortmoeglichkeit
				L3={Der Differenzenquotient der Funktion $s$ im Intervall $[t_1; t_2]$ gibt den in diesem Intervall
zurückgelegten Weg an.},   %3. Antwortmoeglichkeit
				L4={Der Differenzialquotient der Funktion $s$ an einer Stelle $t$ gibt den Winkel an, den die
Tangente an den Graphen im Punkt $P = (t|s(t))$ mit der positiven $x$-Achse einschließt},   %4. Antwortmoeglichkeit
				L5={Der Differenzenquotient der Funktion $s'$ im Intervall $[t_1; t_2]$ gibt die mittlere Änderung
der Geschwindigkeit pro Sekunde im Intervall $[t_1; t_2]$ an.},	 %5. Antwortmoeglichkeit
				L6={},	 %6. Antwortmoeglichkeit
				L7={},	 %7. Antwortmoeglichkeit
				L8={},	 %8. Antwortmoeglichkeit
				L9={},	 %9. Antwortmoeglichkeit
				%% LOESUNG: %%
				A1=1,  % 1. Antwort
				A2=2,	 % 2. Antwort
				A3=5,  % 3. Antwort
				A4=0,  % 4. Antwort
				A5=0,  % 5. Antwort
				}
\end{beispiel}