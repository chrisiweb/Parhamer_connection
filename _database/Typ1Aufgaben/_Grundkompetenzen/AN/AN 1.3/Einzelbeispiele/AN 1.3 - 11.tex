\section{AN 1.3 - 11 - MAT - Ableitungswerte ordnen - OA - Matura 2013/14 Haupttermin}

\begin{beispiel}[AN 1.3]{1} %PUNKTE DES BEISPIELS
				Gegeben ist der Graph einer Polynomfunktion $f$.\leer
				
				\begin{center}\resizebox{0.5\linewidth}{!}{\psset{xunit=1.0cm,yunit=1.0cm,algebraic=true,dimen=middle,dotstyle=o,dotsize=5pt 0,linewidth=0.8pt,arrowsize=3pt 2,arrowinset=0.25}
\begin{pspicture*}(-2.8,-1.4)(4.48,5.54)
\multips(0,-1)(0,1.0){7}{\psline[linestyle=dashed,linecap=1,dash=1.5pt 1.5pt,linewidth=0.4pt,linecolor=lightgray]{c-c}(-2.8,0)(4.48,0)}
\multips(-2,0)(1.0,0){8}{\psline[linestyle=dashed,linecap=1,dash=1.5pt 1.5pt,linewidth=0.4pt,linecolor=lightgray]{c-c}(0,-1.4)(0,5.54)}
\psaxes[labelFontSize=\scriptstyle,xAxis=true,yAxis=true,Dx=1.,Dy=1.,ticksize=-2pt 0,subticks=2]{->}(0,0)(-2.8,-1.4)(4.48,5.54)[x,140] [f(x),-40]
\psplot[linewidth=1.2pt,plotpoints=200]{-2.8000000000000007}{4.480000000000007}{0.1450365624797774*x^(3.0)-0.4450365624797774*x^(2.0)+0.02978062512133562*x+2.0}
\begin{scriptsize}
\rput[tl](0.94,2.05){f}
\end{scriptsize}
\end{pspicture*}}\end{center}\leer

Ordne die Werte $f'(0),f'(1),f'(3)$ und $f'(4)$ der Größe nach, beginnend mit dem kleinsten Wert!\\
(Die konkreten Werte von $f'(0),f'(1),f'(3)$ und $f'(4)$ sind dabei nicht anzugeben.)\leer

\antwort{$f'(1)<f'(0)<f'(3)<f'(4)$\\
Auch zu werten wenn das "`Kleiner"'-Zeichen fehlt aber die Reihenfolge stimmt.}
\end{beispiel}