\section{AN 1.3 - 1 - Änderungsmaße - MC - BIFIE}

\begin{beispiel}[AN 1.3]{1} %PUNKTE DES BEISPIELS
Die nachstehende Abbildung zeigt den Graphen der Funktion $f$ mit der Gleichung $f(x) = 0,1x^2$.

\leer

\begin{center}
\psset{xunit=0.8cm,yunit=0.8cm,algebraic=true,dimen=middle,dotstyle=o,dotsize=5pt 0,linewidth=0.8pt,arrowsize=3pt 2,arrowinset=0.25}
\begin{pspicture*}(-0.4926028799460468,-0.55)(8.57131234392381,6.638523769581398)
\multips(0,0)(0,1.0){8}{\psline[linestyle=dashed,linecap=1,dash=1.5pt 1.5pt,linewidth=0.4pt,linecolor=gray]{c-c}(0,0)(8.57131234392381,0)}
\multips(0,0)(1.0,0){10}{\psline[linestyle=dashed,linecap=1,dash=1.5pt 1.5pt,linewidth=0.4pt,linecolor=gray]{c-c}(0,0)(0,6.638523769581398)}
\begin{scriptsize}
\psaxes[xAxis=true,yAxis=true,Dx=1.,Dy=1.,ticksize=-2pt 0,subticks=0]{->}(0,0)(-0.4926028799460468,-0.4731635599164806)(8.57131234392381,6.638523769581398)[$x$,140] [$f(x)$,-40]
\psplot[plotpoints=200]{-0.4926028799460468}{8.57131234392381}{0.1*x^(2.0)}
\rput[tl](7.595198396737825,5.781933979193698){$f$}
\end{scriptsize}
\end{pspicture*}
\end{center}

Kreuze die beiden Aussagen an, die für die gegebene Funktion $f$ zutreffend sind.

\multiplechoice[5]{  %Anzahl der Antwortmoeglichkeiten, Standard: 5
				L1={Die absolute Änderung in den Intervallen $[0; 3]$ und $[4; 5]$ ist gleich groß.},   %1. Antwortmoeglichkeit 
				L2={Die mittlere Änderungsrate der Funktion f in den Intervallen $[0; 2]$
und $[2; 4]$ ist gleich.},   %2. Antwortmoeglichkeit
				L3={Die momentane Änderungsrate an der Stelle $x = 5$ hat den Wert $2,5$.},   %3. Antwortmoeglichkeit
				L4={Die momentane Änderungsrate an der Stelle $x = 2$ ist größer als die momentane Änderungsrate an der Stelle $x = 6$.},   %4. Antwortmoeglichkeit
				L5={Die Steigung der Sekante durch die Punkte $A = (3\mid f(3))$ und \mbox{$B = (6\mid f(6))$}
ist größer als die momentane Änderungsrate an der Stelle $x = 3$.},	 %5. Antwortmoeglichkeit
				L6={},	 %6. Antwortmoeglichkeit
				L7={},	 %7. Antwortmoeglichkeit
				L8={},	 %8. Antwortmoeglichkeit
				L9={},	 %9. Antwortmoeglichkeit
				%% LOESUNG: %%
				A1=1,  % 1. Antwort
				A2=5,	 % 2. Antwort
				A3=0,  % 3. Antwort
				A4=0,  % 4. Antwort
				A5=0,  % 5. Antwort
				}

\end{beispiel}