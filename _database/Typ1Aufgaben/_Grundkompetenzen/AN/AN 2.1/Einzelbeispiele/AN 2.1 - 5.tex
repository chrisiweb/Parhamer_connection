\section{AN 2.1 - 5 Ableitung von Funktionen - ZO - BIFIE}

\begin{beispiel}[AN 2.1]{1} %PUNKTE DES BEISPIELS
				Die Ableitungsfunktion einer Funktion kann mithilfe einfacher Regeln des Differenzierens ermittelt werden.

Ordne den gegebenen Funktionen jeweils die entsprechende Ableitungsfunktion zu!

\zuordnen{
				title1={Ableitungsfunktionen}, 		%Titel Antwortmoeglichkeiten
				A={$f'(x)=-4x+2$}, 				%Moeglichkeit A  
				B={$f'(x)=\dfrac{1}{\sqrt{2x}}$}, 				%Moeglichkeit B  
				C={$f'(x)=\dfrac{2}{\sqrt{2x}}$}, 				%Moeglichkeit C  
				D={$f'(x)=-\dfrac{2}{x^4}$}, 				%Moeglichkeit D  
				E={$f'(x)=-\dfrac{2}{x^3}$}, 				%Moeglichkeit E  
				F={$f'(x)=-\dfrac{2}{x^2}$}, 				%Moeglichkeit F  
				title2={Funktionen},		%Titel Zuordnung
				R1={$f_1(x)=\dfrac{2}{x}$},				%1. Antwort rechts
				R2={$f_2(x)=-2x^2+2x-2$},				%2. Antwort rechts
				R3={$f_3(x)=\dfrac{1}{x^2}$},				%3. Antwort rechts
				R4={$f_4(x)=\sqrt{2x}$},				%4. Antwort rechts
				%% LOESUNG: %%
				A1={F},				% 1. richtige Zuordnung
				A2={A},				% 2. richtige Zuordnung
				A3={E},				% 3. richtige Zuordnung
				A4={B},				% 4. richtige Zuordnung
				}
\end{beispiel}