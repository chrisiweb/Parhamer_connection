\section{AN 2.1 - 16 - MAT - Werte einer Ableitungsfunktion - MC - Matura-HT-18/19}

\begin{beispiel}[AN 2.1]{1}
Gegeben ist die Funktion $f\!:\mathbb{R}\rightarrow\mathbb{R}$ mit $f(x)=3\cdot e^x$.

Die nachstehenden Aussagen beziehen sich auf Eigenschaften der Funktion $f$ bzw. deren Ableitungsfunktion $f'$.\\
Kreuze die beiden zutreffenden Aussagen an!

\multiplechoice[5]{  %Anzahl der Antwortmoeglichkeiten, Standard: 5
				L1={Es gibt eine Stelle $x\in\mathbb{R}$ mit $f'(x)=2$.},   %1. Antwortmoeglichkeit 
				L2={Für alle $x\in\mathbb{R}$ gilt: $f'(x)>f'(x+1)$.},   %2. Antwortmoeglichkeit
				L3={Für alle $x\in\mathbb{R}$ gilt: $f'(x)=3\cdot f(x)$.},   %3. Antwortmoeglichkeit
				L4={Es gibt eine Stelle $x\in\mathbb{R}$ mit $f'(x)=0$.},   %4. Antwortmoeglichkeit
				L5={Für alle $x\in\mathbb{R}$ gilt: $f'(x)\geq 0$.},	 %5. Antwortmoeglichkeit
				L6={},	 %6. Antwortmoeglichkeit
				L7={},	 %7. Antwortmoeglichkeit
				L8={},	 %8. Antwortmoeglichkeit
				L9={},	 %9. Antwortmoeglichkeit
				%% LOESUNG: %%
				A1=1,  % 1. Antwort
				A2=5,	 % 2. Antwort
				A3=0,  % 3. Antwort
				A4=0,  % 4. Antwort
				A5=0,  % 5. Antwort
				}
\end{beispiel}