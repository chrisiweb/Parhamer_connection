\section{AN 3.2 - 10 Eigenschaften der Ableitungsfunktion einer Polynomfunktion 3.Grades - OA - Matura 2014/15 - Nebentermin 2}

\begin{beispiel}[AN 3.2]{1} %PUNKTE DES BEISPIELS
				Die nachstehende Abbildung zeigt den Graphen einer Polynomfunktion $f$ dritten Grades. Die Koordinaten der hervorgehobenen Punkte des Graphen der Funktion sind ganzzahlig.
				\begin{center}
					\resizebox{0.6\linewidth}{!}{\psset{xunit=1.0cm,yunit=1.0cm,algebraic=true,dimen=middle,dotstyle=o,dotsize=5pt 0,linewidth=0.8pt,arrowsize=3pt 2,arrowinset=0.25}
\begin{pspicture*}(-3.3,-1.6)(3.8,5.7)
\multips(0,-1)(0,1.0){8}{\psline[linestyle=dashed,linecap=1,dash=1.5pt 1.5pt,linewidth=0.4pt,linecolor=lightgray]{c-c}(-3.3,0)(3.8,0)}
\multips(-3,0)(1.0,0){8}{\psline[linestyle=dashed,linecap=1,dash=1.5pt 1.5pt,linewidth=0.4pt,linecolor=lightgray]{c-c}(0,-1.6)(0,5.7)}
\psaxes[labelFontSize=\scriptstyle,xAxis=true,yAxis=true,Dx=1.,Dy=1.,ticksize=-2pt 0,subticks=2]{->}(0,0)(-3.3,-1.6)(3.8,5.7)[x,140] [f(x),-40]
\psplot[linewidth=1.2pt,plotpoints=200]{-3.300000000000001}{3.8000000000000016}{x^(3.0)-3.0*x^(2.0)+4.0}
\begin{scriptsize}
\psdots[dotsize=3pt 0,dotstyle=*](-1.,0.)
\psdots[dotsize=3pt 0,dotstyle=*](0.,4.)
\psdots[dotsize=3pt 0,dotstyle=*](1.,2.)
\psdots[dotsize=3pt 0,dotstyle=*](2.,0.)
\rput[bl](-1.28,-0.04){$f$}
\end{scriptsize}
\end{pspicture*}}
				\end{center}
				
				Welche der folgenden Aussagen treffen auf die Ableitungsfunktion $f'$ der Funktion $f$ zu? Kreuze die beiden zutreffenden Aussagen an.
				
				\multiplechoice[5]{  %Anzahl der Antwortmoeglichkeiten, Standard: 5
								L1={Die Funktionswerte der Funktion $f'$ sind im Intervall $(0;2)$ negativ.},   %1. Antwortmoeglichkeit 
								L2={Die Funktion $f'$ ist im Intervall $(-1;0)$ streng monoton steigend.},   %2. Antwortmoeglichkeit
								L3={Die Funktion $f'$ hat an der Stelle $x=2$ eine Wendestelle.},   %3. Antwortmoeglichkeit
								L4={Die Funktion $f'$ hat an der Stelle $x=1$ ein lokales Maximum.},   %4. Antwortmoeglichkeit
								L5={Die Funktion $f'$ hat an der Stelle $x=0$ eine Nullstelle.},	 %5. Antwortmoeglichkeit
								L6={},	 %6. Antwortmoeglichkeit
								L7={},	 %7. Antwortmoeglichkeit
								L8={},	 %8. Antwortmoeglichkeit
								L9={},	 %9. Antwortmoeglichkeit
								%% LOESUNG: %%
								A1=1,  % 1. Antwort
								A2=5,	 % 2. Antwort
								A3=0,  % 3. Antwort
								A4=0,  % 4. Antwort
								A5=0,  % 5. Antwort
								}
\end{beispiel}