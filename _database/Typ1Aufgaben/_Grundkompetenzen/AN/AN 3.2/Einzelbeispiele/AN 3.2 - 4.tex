\section{AN 3.2 - 4 - Gleiche Ableitungsfunktion - OA - BIFIE}

\begin{beispiel}[AN 3.2]{1} %PUNKTE DES BEISPIELS
				In der unten stehenden Abbildung ist der Graph der Funktion $g$ dargestellt.
				
				Zeichen im vorgegebenen Koordinatensystem den Graphen einer Funktion $f\, (f\neq g)$ ein, die die gleiche Ableitungsfunktion wie die Funktion g hat!
				\leer
				
				\begin{center}
					\resizebox{0.8\linewidth}{!}{\psset{xunit=1.0cm,yunit=1.0cm,algebraic=true,dimen=middle,dotstyle=o,dotsize=5pt 0,linewidth=0.8pt,arrowsize=3pt 2,arrowinset=0.25}
\begin{pspicture*}(-4.2443125541125575,-1.549818101486145)(4.254042424242424,7.682736011067972)
\multips(0,-1)(0,1.0){10}{\psline[linestyle=dashed,linecap=1,dash=1.5pt 1.5pt,linewidth=0.4pt,linecolor=gray]{c-c}(-4.2443125541125575,0)(4.254042424242424,0)}
\multips(-4,0)(1.0,0){9}{\psline[linestyle=dashed,linecap=1,dash=1.5pt 1.5pt,linewidth=0.4pt,linecolor=gray]{c-c}(0,-1.549818101486145)(0,7.682736011067972)}
\psaxes[labelFontSize=\scriptstyle,xAxis=true,yAxis=true,Dx=1.,Dy=1.,showorigin=false,ticksize=-2pt 0,subticks=0]{->}(0,0)(-4.2443125541125575,-1.549818101486145)(4.254042424242424,7.682736011067972)[$x$,140] [$f(x)$,-40]
\psplot[linewidth=1.2pt,plotpoints=200]{-4.2443125541125575}{4.254042424242424}{x^(2.0)+2.0}
\antwort{\psplot[linewidth=1.2pt,linecolor=red,plotpoints=200]{-4.2443125541125575}{4.254042424242424}{x^(2.0)+3.0}}
\begin{scriptsize}
\rput[bl](-2.335394805194808,6.012432980764941){$f$}
\end{scriptsize}
\end{pspicture*}}
				\end{center}
				\leer
				
				\antwort{Die Aufgabe gilt nur dann als richtig gelöst, wenn der Graph von $f$ erkennbar durch eine Verschiebung in Richtung der y-Achse aus dem Graphen von $g$ entsteht.}
\end{beispiel}