\section{AN 3.2 - 14 - MAT - Grafisch differenzieren - OA - Matura 2016/17 - Haupttermin}

\begin{beispiel}[AN 3.2]{1} %PUNKTE DES BEISPIELS
Gegeben ist der Graph einer Polynomfunktion dritten Grades $f$.


\begin{center}
\psset{xunit=0.8cm,yunit=0.8cm,algebraic=true,dimen=middle,dotstyle=o,dotsize=5pt 0,linewidth=0.8pt,arrowsize=3pt 2,arrowinset=0.25}
\begin{pspicture*}(-8.62,-5.14)(6.88,8.62)
\multips(0,-5)(0,1.0){14}{\psline[linestyle=dashed,linecap=1,dash=1.5pt 1.5pt,linewidth=0.4pt,linecolor=gray]{c-c}(-8.62,0)(6.88,0)}
\multips(-8,0)(1.0,0){16}{\psline[linestyle=dashed,linecap=1,dash=1.5pt 1.5pt,linewidth=0.4pt,linecolor=gray]{c-c}(0,-5.14)(0,8.62)}
\psaxes[labelFontSize=\scriptstyle,xAxis=true,yAxis=true,labels=none,Dx=1.,Dy=1.,ticksize=-2pt 0,subticks=0]{->}(0,0)(-8.62,-5.14)(6.88,8.62)[x,140] [f(x),-40]
\antwort{\psplot[linecolor=red,plotpoints=200]{-7}{2}{0.31003100310031*x^(2.0)+1.24012401240124*x}}
\psplot[plotpoints=200]{-8.620000000000006}{6.88}{0.1034*x^(3.0)+0.62*x^(2.0)+1.7}
\rput[tl](1.66,5.96){$f$}
\rput[tl](-7.24,-0.18){$x_1$}
\rput[tl](1.8,-0.2){$x_2$}
\antwort{
\rput[tl](1.68,2.6){$f'$}
\psdots[dotsize=5pt 0,dotstyle=*,linecolor=red](-4.,0.)
\psdots[dotsize=5pt 0,dotstyle=*,linecolor=red](0.,0.)}
\end{pspicture*}
\end{center}


Skizziere in der gegebenen Grafik den Graphen der Ableitungsfunktion $f'$ im Intervall $[x_1; x_2]$ und markiere gegebenenfalls die Nullstellen!

\antwort{\leer

Lösungsschlüssel:

Ein Punkt für eine korrekte Darstellung der Ableitungsfunktion $f'$. Der Graph der Funktion $f'$ muss erkennbar die Form einer nach oben offenen Parabel haben und die $x$-Achse an den beiden Stellen schneiden, bei denen die Funktion $f$ die Extremstellen hat. Der Graph einer entsprechenden Funktion $f'$, der über das Intervall $[x_1; x_2]$ hinaus gezeichnet ist, ist ebenfalls als richtig zu werten.}
\end{beispiel}