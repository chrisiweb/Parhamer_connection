\section{AN 3.2 - 18 - MAT - Graphen von Ableitungsfunktionen - ZO - Matura 2018/19 2. NT}

\begin{beispiel}[AN 3.2]{1}
Unten stehend sind die vier Graphen der Funktionen $f_1$ bis $f_4$ sowie die Graphen von sechs Funktionen (A bis F) abgebildet.

Ordne den vier Graphen der Funktionen $f_1$ bis $f_4$ jeweils denjenigen Graphen (aus A bis F) zu, der die Ableitung dieser Funktion darstellt.

\zuordnen{
				R1={\psset{xunit=0.45cm,yunit=0.45cm,algebraic=true,dimen=middle,dotstyle=o,dotsize=5pt 0,linewidth=0.8pt,arrowsize=3pt 2,arrowinset=0.25}
\begin{pspicture*}(-4.8,-3.38)(4.84,4.02)
\psaxes[labelFontSize=\scriptstyle,xAxis=true,yAxis=true,labels=none,Dx=1.,Dy=1.,ticks=none]{->}(0,0)(-4.8,-3.38)(4.84,4.02)
\psplot[linewidth=1.pt,plotpoints=200]{-4.8}{-0.1}{x^(-1.0)}
\psplot[linewidth=1.pt,plotpoints=200]{0.1}{4.84}{x^(-1.0)}
\begin{scriptsize}
\rput[tl](0.2,-0.2){0}
\rput[tl](4.3,-0.3){$x$}
\rput[tl](-1.8,3.82){$f_1(x)$}
\rput[tl](1.3,1.8){$f_1$}
\end{scriptsize}
\end{pspicture*}},				% Response 1
				R2={\psset{xunit=0.45cm,yunit=0.45cm,algebraic=true,dimen=middle,dotstyle=o,dotsize=5pt 0,linewidth=0.8pt,arrowsize=3pt 2,arrowinset=0.25}
\begin{pspicture*}(-4.8,-3.38)(4.84,4.02)
\psaxes[labelFontSize=\scriptstyle,xAxis=true,yAxis=true,labels=none,Dx=1.,Dy=1.,ticks=none]{->}(0,0)(-4.8,-3.38)(4.84,4.02)
\psplot[linewidth=1.pt,plotpoints=200]{-4.8}{-0.1}{-2*x^(-2.0)}
\psplot[linewidth=1.pt,plotpoints=200]{0.1}{4.84}{-2*x^(-2.0)}
\begin{scriptsize}
\rput[tl](0.2,-0.2){0}
\rput[bl](4.3,0.3){$x$}
\rput[tl](-1.8,3.82){$f_2(x)$}
\rput[tl](1.3,-1.8){$f_2$}
\end{scriptsize}
\end{pspicture*}},				% Response 2
				R3={\psset{xunit=0.45cm,yunit=0.45cm,algebraic=true,dimen=middle,dotstyle=o,dotsize=5pt 0,linewidth=0.8pt,arrowsize=3pt 2,arrowinset=0.25}
\begin{pspicture*}(-4.8,-3.38)(4.84,4.02)
\psaxes[labelFontSize=\scriptstyle,xAxis=true,yAxis=true,labels=none,Dx=1.,Dy=1.,ticks=none]{->}(0,0)(-4.8,-3.38)(4.84,4.02)
\psplot[linewidth=1.pt,plotpoints=200]{0.01}{4.84}{ln(x)}
\begin{scriptsize}
\rput[tl](-0.5,-0.2){0}
\rput[bl](4.3,0.3){$x$}
\rput[tl](-1.8,3.82){$f_3(x)$}
\rput[tl](1.3,1.8){$f_3$}
\end{scriptsize}
\end{pspicture*}},				% Response 3
				R4={\psset{xunit=0.45cm,yunit=0.45cm,algebraic=true,dimen=middle,dotstyle=o,dotsize=5pt 0,linewidth=0.8pt,arrowsize=3pt 2,arrowinset=0.25}
\begin{pspicture*}(-4.8,-3.38)(4.84,4.02)
\psaxes[labelFontSize=\scriptstyle,xAxis=true,yAxis=true,labels=none,Dx=1.,Dy=1.,ticks=none]{->}(0,0)(-4.8,-3.38)(4.84,4.02)
\psplot[linewidth=1.pt,plotpoints=200]{-4.8}{4.8}{2.718281828459045^(x)}
\begin{scriptsize}
\rput[tl](0.2,-0.2){0}
\rput[bl](4.3,0.3){$x$}
\rput[tl](-1.8,3.82){$f_4(x)$}
\rput[tl](1.3,2){$f_4$}
\end{scriptsize}
\end{pspicture*}},				% Response 4
				%% Moegliche Zuordnungen: %%
				A={\psset{xunit=0.45cm,yunit=0.4cm,algebraic=true,dimen=middle,dotstyle=o,dotsize=5pt 0,linewidth=0.8pt,arrowsize=3pt 2,arrowinset=0.25}
\begin{pspicture*}(-4.8,-3.38)(4.84,4.02)
\psaxes[labelFontSize=\scriptstyle,xAxis=true,yAxis=true,labels=none,Dx=1.,Dy=1.,ticks=none]{->}(0,0)(-4.8,-3.38)(4.84,4.02)
\psplot[linewidth=1.pt,plotpoints=200]{-4.8}{4.8}{2.718281828459045^(x)}
\begin{scriptsize}
\rput[tl](0.2,-0.2){0}
\rput[bl](4.3,0.3){$x$}
\rput[tl](-1.8,3.82){$f_i'(x)$}
\rput[tl](1.1,2){$f_i'$}
\end{scriptsize}
\end{pspicture*}}, 				%Moeglichkeit A  
				B={\psset{xunit=0.45cm,yunit=0.4cm,algebraic=true,dimen=middle,dotstyle=o,dotsize=5pt 0,linewidth=0.8pt,arrowsize=3pt 2,arrowinset=0.25}
\begin{pspicture*}(-4.8,-3.38)(4.84,4.02)
\psaxes[labelFontSize=\scriptstyle,xAxis=true,yAxis=true,labels=none,Dx=1.,Dy=1.,ticks=none]{->}(0,0)(-4.8,-3.38)(4.84,4.02)
\psplot[linewidth=1.pt,plotpoints=200]{-4.8}{-0.01}{1.0/x^(2.0)}
\psplot[linewidth=1.pt,plotpoints=200]{0.01}{4.84}{1.0/x^(2.0)}
\begin{scriptsize}
\rput[tl](0.2,-0.2){0}
\rput[bl](4.3,0.3){$x$}
\rput[tl](-2.2,3.82){$f_i'(x)$}
\rput[tl](1.3,1.6){$f_i'$}
\end{scriptsize}
\end{pspicture*}}, 				%Moeglichkeit B  
				C={\psset{xunit=0.45cm,yunit=0.4cm,algebraic=true,dimen=middle,dotstyle=o,dotsize=5pt 0,linewidth=0.8pt,arrowsize=3pt 2,arrowinset=0.25}
\begin{pspicture*}(-4.8,-3.38)(4.84,4.02)
\psaxes[labelFontSize=\scriptstyle,xAxis=true,yAxis=true,labels=none,Dx=1.,Dy=1.,ticks=none]{->}(0,0)(-4.8,-3.38)(4.84,4.02)
\psplot[linewidth=1.pt,plotpoints=200]{-4.8}{-0.01}{4.0/x^(3.0)}
\psplot[linewidth=1.pt,plotpoints=200]{0.01}{4.84}{4.0/x^(3.0)}
\begin{scriptsize}
\rput[tl](0.2,-0.2){0}
\rput[bl](4.3,0.3){$x$}
\rput[tl](-1.8,3.82){$f_i'(x)$}
\rput[tl](1.65,2){$f_i'$}
\end{scriptsize}
\end{pspicture*}}, 				%Moeglichkeit C  
				D={\psset{xunit=0.45cm,yunit=0.4cm,algebraic=true,dimen=middle,dotstyle=o,dotsize=5pt 0,linewidth=0.8pt,arrowsize=3pt 2,arrowinset=0.25}
\begin{pspicture*}(-4.8,-3.38)(4.84,4.02)
\psaxes[labelFontSize=\scriptstyle,xAxis=true,yAxis=true,labels=none,Dx=1.,Dy=1.,ticks=none]{->}(0,0)(-4.8,-3.38)(4.84,4.02)
\psplot[linewidth=1.pt,plotpoints=200]{-4.8}{-0.01}{-1.0/x^(2.0)}
\psplot[linewidth=1.pt,plotpoints=200]{0.01}{4.84}{-1.0/x^(2.0)}
\begin{scriptsize}
\rput[tl](0.2,-0.2){0}
\rput[bl](4.3,0.3){$x$}
\rput[tl](-1.8,3.82){$f_i'(x)$}
\rput[tl](1.3,-1.4){$f_i'$}
\end{scriptsize}
\end{pspicture*}}, 				%Moeglichkeit D  
				E={\psset{xunit=0.45cm,yunit=0.4cm,algebraic=true,dimen=middle,dotstyle=o,dotsize=5pt 0,linewidth=0.8pt,arrowsize=3pt 2,arrowinset=0.25}
\begin{pspicture*}(-4.8,-3.38)(4.84,4.02)
\psaxes[labelFontSize=\scriptstyle,xAxis=true,yAxis=true,labels=none,Dx=1.,Dy=1.,ticks=none]{->}(0,0)(-4.8,-3.38)(4.84,4.02)
\psplot[linewidth=1.pt,plotpoints=200]{0.01}{4.84}{-ln(x)}
\begin{scriptsize}
\rput[tl](-0.5,-0.2){0}
\rput[bl](4.3,0.3){$x$}
\rput[tl](-1.8,3.82){$f_i'(x)$}
\rput[tl](0.5,1.8){$f_i'$}
\end{scriptsize}
\end{pspicture*}}, 				%Moeglichkeit E  
				F={\psset{xunit=0.45cm,yunit=0.4cm,algebraic=true,dimen=middle,dotstyle=o,dotsize=5pt 0,linewidth=0.8pt,arrowsize=3pt 2,arrowinset=0.25}
\begin{pspicture*}(-4.8,-3.38)(4.84,4.02)
\psaxes[labelFontSize=\scriptstyle,xAxis=true,yAxis=true,labels=none,Dx=1.,Dy=1.,ticks=none]{->}(0,0)(-4.8,-3.38)(4.84,4.02)
\psplot[linewidth=1.pt,plotpoints=200]{0.01}{4.8}{1.0/x}
\begin{scriptsize}
\rput[tl](0.2,-0.2){0}
\rput[bl](4.3,0.3){$x$}
\rput[tl](-1.8,3.82){$f_i'(x)$}
\rput[tl](1.3,1.8){$f_i'$}
\end{scriptsize}
\end{pspicture*}}, 				%Moeglichkeit F  
				%% LOESUNG: %%
				A1={D},				% 1. richtige Zuordnung
				A2={C},				% 2. richtige Zuordnung
				A3={F},				% 3. richtige Zuordnung
				A4={A},				% 4. richtige Zuordnung
				}
\end{beispiel}