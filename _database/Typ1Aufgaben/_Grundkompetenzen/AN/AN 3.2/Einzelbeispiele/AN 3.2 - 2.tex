\section{AN 3.2 - 2 Graph der ersten Ableitungsfunktion - MC - BIFIE}

\begin{beispiel}[AN 3.2]{1} %PUNKTE DES BEISPIELS
				Gegeben ist der Graph der Funktion $f$.
\begin{center}\resizebox{0.25\linewidth}{!}{\psset{xunit=1.0cm,yunit=1.0cm,algebraic=true,dimen=middle,dotstyle=o,dotsize=5pt 0,linewidth=0.8pt,arrowsize=3pt 2,arrowinset=0.25}
\begin{pspicture*}(-3.4052929250927324,-3.580987511391765)(1.8952898818965822,3.491592454910838)
\multips(0,-3)(0,1.0){8}{\psline[linestyle=dashed,linecap=1,dash=1.5pt 1.5pt,linewidth=0.4pt,linecolor=lightgray]{c-c}(-3.4052929250927324,0)(1.8952898818965822,0)}
\multips(-3,0)(1.0,0){6}{\psline[linestyle=dashed,linecap=1,dash=1.5pt 1.5pt,linewidth=0.4pt,linecolor=lightgray]{c-c}(0,-3.580987511391765)(0,3.491592454910838)}
\psaxes[labelFontSize=\scriptstyle,xAxis=true,yAxis=true,Dx=1.,Dy=1.,ticksize=-2pt 0,subticks=2]{->}(0,0)(-3.4052929250927324,-3.580987511391765)(1.8952898818965822,3.491592454910838)[x,140] [f(x),-40]
\psplot[linewidth=1.2pt,plotpoints=200]{-3.4052929250927324}{1.8952898818965822}{-0.75*x^(3.0)-2.25*x^(2.0)+4.0E-50*x+2.0}
\end{pspicture*}}\end{center}
\scriptsize{Welche der nachstehenden Abbildungen beschreibt den Graphen der ersten Ableitungsfunktion der Funktion $f$? Kreuze die zutreffende Abbildung an!}

\langmultiplechoice[6]{  %Anzahl der Antwortmoeglichkeiten, Standard: 5
				L1={\resizebox{0.6\linewidth}{!}{\newrgbcolor{ttqqqq}{0.2 0. 0.}
\psset{xunit=1.0cm,yunit=1.0cm,algebraic=true,dimen=middle,dotstyle=o,dotsize=5pt 0,linewidth=0.8pt,arrowsize=3pt 2,arrowinset=0.25}
\begin{pspicture*}(-3.020076151328974,-3.596396182342316)(3.5902436864571214,3.4761837839602894)
\multips(0,-3)(0,1.0){8}{\psline[linestyle=dashed,linecap=1,dash=1.5pt 1.5pt,linewidth=0.4pt,linecolor=lightgray]{c-c}(-3.020076151328974,0)(3.5902436864571214,0)}
\multips(-3,0)(1.0,0){7}{\psline[linestyle=dashed,linecap=1,dash=1.5pt 1.5pt,linewidth=0.4pt,linecolor=lightgray]{c-c}(0,-3.596396182342316)(0,3.4761837839602894)}
\psaxes[labelFontSize=\scriptstyle,xAxis=true,yAxis=true,Dx=1.,Dy=1.,ticksize=-2pt 0,subticks=2]{->}(0,0)(-3.020076151328974,-3.596396182342316)(3.5902436864571214,3.4761837839602894)[x,140] [f(x),-40]
\psplot[linewidth=1.2pt,linecolor=ttqqqq,plotpoints=200]{-3.020076151328974}{3.5902436864571214}{-2.2130955546979934*x^(2.0)+4.426191109395987*x}
\end{pspicture*}}},   %1. Antwortmoeglichkeit 
				L2={\resizebox{0.6\linewidth}{!}{\psset{xunit=1.0cm,yunit=1.0cm,algebraic=true,dimen=middle,dotstyle=o,dotsize=5pt 0,linewidth=0.8pt,arrowsize=3pt 2,arrowinset=0.25}
\begin{pspicture*}(-3.6287932658462516,-3.6564760974760597)(2.8990384294437543,3.3278473247573155)
\multips(0,-3)(0,1.0){7}{\psline[linestyle=dashed,linecap=1,dash=1.5pt 1.5pt,linewidth=0.4pt,linecolor=lightgray]{c-c}(-3.6287932658462516,0)(2.8990384294437543,0)}
\multips(-3,0)(1.0,0){7}{\psline[linestyle=dashed,linecap=1,dash=1.5pt 1.5pt,linewidth=0.4pt,linecolor=lightgray]{c-c}(0,-3.6564760974760597)(0,3.3278473247573155)}
\psaxes[labelFontSize=\scriptstyle,xAxis=true,yAxis=true,Dx=1.,Dy=1.,ticksize=-2pt 0,subticks=2]{->}(0,0)(-3.6287932658462516,-3.6564760974760597)(2.8990384294437543,3.3278473247573155)[x,140] [f(x),-40]
\psplot[linewidth=1.2pt,plotpoints=200]{-3.6287932658462516}{2.8990384294437543}{-1.0000884807074706*x^(4.0)-6.036615308075288*x^(3.0)-12.1813954069345*x^(2.0)-8.217037427227613*x}
\end{pspicture*}}},   %2. Antwortmoeglichkeit
				L3={\resizebox{0.6\linewidth}{!}{\psset{xunit=1.0cm,yunit=1.0cm,algebraic=true,dimen=middle,dotstyle=o,dotsize=5pt 0,linewidth=0.8pt,arrowsize=3pt 2,arrowinset=0.25}
\begin{pspicture*}(-3.6287932658462516,-3.6564760974760597)(2.8990384294437543,3.3278473247573155)
\multips(0,-3)(0,1.0){7}{\psline[linestyle=dashed,linecap=1,dash=1.5pt 1.5pt,linewidth=0.4pt,linecolor=lightgray]{c-c}(-3.6287932658462516,0)(2.8990384294437543,0)}
\multips(-3,0)(1.0,0){7}{\psline[linestyle=dashed,linecap=1,dash=1.5pt 1.5pt,linewidth=0.4pt,linecolor=lightgray]{c-c}(0,-3.6564760974760597)(0,3.3278473247573155)}
\psaxes[labelFontSize=\scriptstyle,xAxis=true,yAxis=true,Dx=1.,Dy=1.,ticksize=-2pt 0,subticks=2]{->}(0,0)(-3.6287932658462516,-3.6564760974760597)(2.8990384294437543,3.3278473247573155)[x,140] [f(x),-40]
\psplot[linewidth=1.2pt,plotpoints=200]{-3.6287932658462516}{2.8990384294437543}{-2.201863609943576*x^(2.0)-4.403727219887152*x}
\end{pspicture*}}},   %3. Antwortmoeglichkeit
				L4={\resizebox{0.6\linewidth}{!}{\psset{xunit=1.0cm,yunit=1.0cm,algebraic=true,dimen=middle,dotstyle=o,dotsize=5pt 0,linewidth=0.8pt,arrowsize=3pt 2,arrowinset=0.25}
\begin{pspicture*}(-3.6287932658462516,-3.6564760974760597)(2.8990384294437543,3.3278473247573155)
\multips(0,-3)(0,1.0){7}{\psline[linestyle=dashed,linecap=1,dash=1.5pt 1.5pt,linewidth=0.4pt,linecolor=lightgray]{c-c}(-3.6287932658462516,0)(2.8990384294437543,0)}
\multips(-3,0)(1.0,0){7}{\psline[linestyle=dashed,linecap=1,dash=1.5pt 1.5pt,linewidth=0.4pt,linecolor=lightgray]{c-c}(0,-3.6564760974760597)(0,3.3278473247573155)}
\psaxes[labelFontSize=\scriptstyle,xAxis=true,yAxis=true,Dx=1.,Dy=1.,ticksize=-2pt 0,subticks=2]{->}(0,0)(-3.6287932658462516,-3.6564760974760597)(2.8990384294437543,3.3278473247573155)[x,140] [f(x),-40]
\psplot{-3.6287932658462516}{2.8990384294437543}{(-3.0021712888572467-3.0021712888572467*x)/0.6883799499421429}
\end{pspicture*}}},   %4. Antwortmoeglichkeit
				L5={\resizebox{0.6\linewidth}{!}{\psset{xunit=1.0cm,yunit=1.0cm,algebraic=true,dimen=middle,dotstyle=o,dotsize=5pt 0,linewidth=0.8pt,arrowsize=3pt 2,arrowinset=0.25}
\begin{pspicture*}(-3.6287932658462516,-3.6564760974760597)(2.8990384294437543,3.3278473247573155)
\multips(0,-3)(0,1.0){7}{\psline[linestyle=dashed,linecap=1,dash=1.5pt 1.5pt,linewidth=0.4pt,linecolor=lightgray]{c-c}(-3.6287932658462516,0)(2.8990384294437543,0)}
\multips(-3,0)(1.0,0){7}{\psline[linestyle=dashed,linecap=1,dash=1.5pt 1.5pt,linewidth=0.4pt,linecolor=lightgray]{c-c}(0,-3.6564760974760597)(0,3.3278473247573155)}
\psaxes[labelFontSize=\scriptstyle,xAxis=true,yAxis=true,Dx=1.,Dy=1.,ticksize=-2pt 0,subticks=2]{->}(0,0)(-3.6287932658462516,-3.6564760974760597)(2.8990384294437543,3.3278473247573155)[x,140] [f(x),-40]
\psplot[linewidth=1.2pt,plotpoints=200]{-3.6287932658462516}{2.8990384294437543}{(x+0.9)^(2.0)-2.97}
\end{pspicture*}}},	 %5. Antwortmoeglichkeit
				L6={\resizebox{0.6\linewidth}{!}{\psset{xunit=1.0cm,yunit=1.0cm,algebraic=true,dimen=middle,dotstyle=o,dotsize=5pt 0,linewidth=0.8pt,arrowsize=3pt 2,arrowinset=0.25}
\begin{pspicture*}(-3.6287932658462516,-3.6564760974760597)(2.8990384294437543,3.3278473247573155)
\multips(0,-3)(0,1.0){7}{\psline[linestyle=dashed,linecap=1,dash=1.5pt 1.5pt,linewidth=0.4pt,linecolor=lightgray]{c-c}(-3.6287932658462516,0)(2.8990384294437543,0)}
\multips(-3,0)(1.0,0){7}{\psline[linestyle=dashed,linecap=1,dash=1.5pt 1.5pt,linewidth=0.4pt,linecolor=lightgray]{c-c}(0,-3.6564760974760597)(0,3.3278473247573155)}
\psaxes[labelFontSize=\scriptstyle,xAxis=true,yAxis=true,Dx=1.,Dy=1.,ticksize=-2pt 0,subticks=2]{->}(0,0)(-3.6287932658462516,-3.6564760974760597)(2.8990384294437543,3.3278473247573155)[x,140] [f(x),-40]
\psplot[linewidth=1.2pt,plotpoints=200]{-3.6287932658462516}{2.8990384294437543}{2.180778313501426*x^(2.0)+4.361556627002852*x}
\end{pspicture*}}},	 %6. Antwortmoeglichkeit
				L7={},	 %7. Antwortmoeglichkeit
				L8={},	 %8. Antwortmoeglichkeit
				L9={},	 %9. Antwortmoeglichkeit
				%% LOESUNG: %%
				A1=3,  % 1. Antwort
				A2=0,	 % 2. Antwort
				A3=0,  % 3. Antwort
				A4=0,  % 4. Antwort
				A5=0,  % 5. Antwort
				}

\end{beispiel}