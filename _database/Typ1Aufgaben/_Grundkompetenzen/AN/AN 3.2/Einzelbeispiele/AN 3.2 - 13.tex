\section{AN 3.2 - 13 Ableitung - OA - Matura 2013/14 1. Nebentermin}

\begin{beispiel}[AN 3.2]{1} %PUNKTE DES BEISPIELS
				In der nachstehenden Abbildung ist der Graph der 1. Ableitungsfunktion $f'$ einer Polynomfunktion $f$ dargestellt.
				
				\begin{center}\resizebox{0.7\linewidth}{!}{\psset{xunit=1.0cm,yunit=1.0cm,algebraic=true,dimen=middle,dotstyle=o,dotsize=5pt 0,linewidth=0.8pt,arrowsize=3pt 2,arrowinset=0.25}
\begin{pspicture*}(-5.76,-2.48)(6.82,5.98)
\multips(0,-2)(0,1.0){9}{\psline[linestyle=dashed,linecap=1,dash=1.5pt 1.5pt,linewidth=0.4pt,linecolor=lightgray]{c-c}(-5.76,0)(6.82,0)}
\multips(-5,0)(1.0,0){13}{\psline[linestyle=dashed,linecap=1,dash=1.5pt 1.5pt,linewidth=0.4pt,linecolor=lightgray]{c-c}(0,-2.48)(0,5.98)}
\psaxes[labelFontSize=\scriptstyle,xAxis=true,yAxis=true,Dx=1.,Dy=1.,ticksize=-2pt 0,subticks=2]{->}(0,0)(-5.76,-2.48)(6.82,5.98)[x,140] [f'(x),-40]
\psplot[linewidth=1.2pt,plotpoints=200]{-5.760000000000003}{6.820000000000001}{-0.25*x^(2.0)+4.0}
\rput[tl](2.56,2.8){f'}
\end{pspicture*}}\end{center}

Bestimme, an welchen Stellen die Funktion $f$ im Intervall $(-5; 5)$ jedenfalls lokale Extrema hat! Die f�r die Bestimmung relevanten Punkte mit ganzzahligen Koordinaten k�nnen der Abbildung entnommen werden.\leer

\antwort{An den Stellen $x_1=-4$ und $x_2=4$ hat $f$ lokale Extrema.}
\end{beispiel}