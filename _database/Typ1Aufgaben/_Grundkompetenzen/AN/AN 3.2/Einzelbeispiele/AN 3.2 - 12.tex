\section{AN 3.2 - 12 - MAT - Eigenschaften der zweiten Ableitung - MC - Matura NT 2 15/16}

\begin{beispiel}[AN 3.2]{1} %PUNKTE DES BEISPIELS
Gegeben sind die Graphen von fünf reellen Funktionen.

Für welche der angegebenen Funktionen gilt $f''(x)>0$ im Intervall $[-1;1]$? Kreuze die beiden zutreffenden Graphen an!\leer

\langmultiplechoice[5]{  %Anzahl der Antwortmoeglichkeiten, Standard: 5
				L1={\resizebox{0.7\linewidth}{!}{\psset{xunit=1.0cm,yunit=1.0cm,algebraic=true,dimen=middle,dotstyle=o,dotsize=5pt 0,linewidth=0.8pt,arrowsize=3pt 2,arrowinset=0.25}
\begin{pspicture*}(-3.8,-2.54)(3.6,2.66)
\multips(0,-2)(0,1.0){6}{\psline[linestyle=dashed,linecap=1,dash=1.5pt 1.5pt,linewidth=0.4pt,linecolor=lightgray]{c-c}(-3.8,0)(3.6,0)}
\multips(-3,0)(1.0,0){8}{\psline[linestyle=dashed,linecap=1,dash=1.5pt 1.5pt,linewidth=0.4pt,linecolor=lightgray]{c-c}(0,-2.54)(0,2.66)}
\psaxes[labelFontSize=\scriptstyle,xAxis=true,yAxis=true,Dx=1.,Dy=1.,ticksize=-2pt 0,subticks=2]{->}(0,0)(-3.8,-2.54)(3.6,2.66)[x,140] [f(x),-40]
\psplot[linewidth=2.pt,plotpoints=200]{-3.8000000000000016}{3.5999999999999948}{0.4571364969828991*x^(3.0)-1.8285459879315964*x}
\rput[bl](-2.5,-0.86){$f$}
\end{pspicture*}}},   %1. Antwortmoeglichkeit 
				L2={\resizebox{0.7\linewidth}{!}{\psset{xunit=1.0cm,yunit=1.0cm,algebraic=true,dimen=middle,dotstyle=o,dotsize=5pt 0,linewidth=0.8pt,arrowsize=3pt 2,arrowinset=0.25}
\begin{pspicture*}(-3.8,-2.54)(3.6,2.66)
\multips(0,-2)(0,1.0){6}{\psline[linestyle=dashed,linecap=1,dash=1.5pt 1.5pt,linewidth=0.4pt,linecolor=lightgray]{c-c}(-3.8,0)(3.6,0)}
\multips(-3,0)(1.0,0){8}{\psline[linestyle=dashed,linecap=1,dash=1.5pt 1.5pt,linewidth=0.4pt,linecolor=lightgray]{c-c}(0,-2.54)(0,2.66)}
\psaxes[labelFontSize=\scriptstyle,xAxis=true,yAxis=true,Dx=1.,Dy=1.,ticksize=-2pt 0,subticks=2]{->}(0,0)(-3.8,-2.54)(3.6,2.66)[x,140] [f(x),-40]
\psplot[linewidth=2.pt,plotpoints=200]{-3.8000000000000016}{3.5999999999999948}{0.77*x^2+0.5}
\rput[bl](-1.7,0.86){$f$}
\end{pspicture*}}},   %2. Antwortmoeglichkeit
				L3={\resizebox{0.7\linewidth}{!}{\psset{xunit=1.0cm,yunit=1.0cm,algebraic=true,dimen=middle,dotstyle=o,dotsize=5pt 0,linewidth=0.8pt,arrowsize=3pt 2,arrowinset=0.25}
\begin{pspicture*}(-3.8,-2.54)(3.6,2.66)
\multips(0,-2)(0,1.0){6}{\psline[linestyle=dashed,linecap=1,dash=1.5pt 1.5pt,linewidth=0.4pt,linecolor=lightgray]{c-c}(-3.8,0)(3.6,0)}
\multips(-3,0)(1.0,0){8}{\psline[linestyle=dashed,linecap=1,dash=1.5pt 1.5pt,linewidth=0.4pt,linecolor=lightgray]{c-c}(0,-2.54)(0,2.66)}
\psaxes[labelFontSize=\scriptstyle,xAxis=true,yAxis=true,Dx=1.,Dy=1.,ticksize=-2pt 0,subticks=2]{->}(0,0)(-3.8,-2.54)(3.6,2.66)[x,140] [f(x),-40]
\psplot[linewidth=2.pt,plotpoints=200]{-3.8000000000000016}{3.5999999999999948}{-0.16*x^2-0.32*x-0.8}
\rput[bl](-2.5,-0.86){$f$}
\end{pspicture*}}},   %3. Antwortmoeglichkeit
				L4={\resizebox{0.7\linewidth}{!}{\psset{xunit=1.0cm,yunit=1.0cm,algebraic=true,dimen=middle,dotstyle=o,dotsize=5pt 0,linewidth=0.8pt,arrowsize=3pt 2,arrowinset=0.25}
\begin{pspicture*}(-3.8,-2.54)(3.6,2.66)
\multips(0,-2)(0,1.0){6}{\psline[linestyle=dashed,linecap=1,dash=1.5pt 1.5pt,linewidth=0.4pt,linecolor=lightgray]{c-c}(-3.8,0)(3.6,0)}
\multips(-3,0)(1.0,0){8}{\psline[linestyle=dashed,linecap=1,dash=1.5pt 1.5pt,linewidth=0.4pt,linecolor=lightgray]{c-c}(0,-2.54)(0,2.66)}
\psaxes[labelFontSize=\scriptstyle,xAxis=true,yAxis=true,Dx=1.,Dy=1.,ticksize=-2pt 0,subticks=2]{->}(0,0)(-3.8,-2.54)(3.6,2.66)[x,140] [f(x),-40]
\psplot[linewidth=2.pt,plotpoints=200]{-3.8000000000000016}{3.5999999999999948}{1.15}
\rput[bl](-1.7,0.66){$f$}
\end{pspicture*}}},   %4. Antwortmoeglichkeit
				L5={\resizebox{0.7\linewidth}{!}{\psset{xunit=1.0cm,yunit=1.0cm,algebraic=true,dimen=middle,dotstyle=o,dotsize=5pt 0,linewidth=0.8pt,arrowsize=3pt 2,arrowinset=0.25}
\begin{pspicture*}(-3.8,-2.54)(3.6,2.66)
\multips(0,-2)(0,1.0){6}{\psline[linestyle=dashed,linecap=1,dash=1.5pt 1.5pt,linewidth=0.4pt,linecolor=lightgray]{c-c}(-3.8,0)(3.6,0)}
\multips(-3,0)(1.0,0){8}{\psline[linestyle=dashed,linecap=1,dash=1.5pt 1.5pt,linewidth=0.4pt,linecolor=lightgray]{c-c}(0,-2.54)(0,2.66)}
\psaxes[labelFontSize=\scriptstyle,xAxis=true,yAxis=true,Dx=1.,Dy=1.,ticksize=-2pt 0,subticks=2]{->}(0,0)(-3.8,-2.54)(3.6,2.66)[x,140] [f(x),-40]
\psplot[linewidth=2.pt,plotpoints=200]{-3.8000000000000016}{3.5999999999999948}{1/2.718^x}
\rput[bl](-0.8,0.86){$f$}
\end{pspicture*}}},	 %5. Antwortmoeglichkeit
				L6={},	 %6. Antwortmoeglichkeit
				L7={},	 %7. Antwortmoeglichkeit
				L8={},	 %8. Antwortmoeglichkeit
				L9={},	 %9. Antwortmoeglichkeit
				%% LOESUNG: %%
				A1=2,  % 1. Antwort
				A2=5,	 % 2. Antwort
				A3=0,  % 3. Antwort
				A4=0,  % 4. Antwort
				A5=0,  % 5. Antwort
				}
\end{beispiel}