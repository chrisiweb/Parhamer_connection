\section{AN 3.2 - 3 Funktion - Ableitungsfunktion - MC - BIFIE}

\begin{beispiel}[AN 3.2]{1} %PUNKTE DES BEISPIELS
				In der untenstehenden Abbildung ist der Graph der Ableitungsfunktion $f'$ einer Funktion $f$ dargestellt.
				\leer
				
				\begin{center}\resizebox{0.5\linewidth}{!}{\psset{xunit=1.0cm,yunit=1.0cm,algebraic=true,dimen=middle,dotstyle=o,dotsize=5pt 0,linewidth=0.8pt,arrowsize=3pt 2,arrowinset=0.25}
\begin{pspicture*}(-5.406875760298069,-1.8426959226203188)(6.927208122057785,11.556967560803026)
\psaxes[labelFontSize=\scriptstyle,xAxis=true,yAxis=true,labels=none,Dx=1.,Dy=1.,ticksize=-2pt 0,subticks=2]{->}(0,0)(-5.406875760298069,-1.8426959226203188)(6.927208122057785,11.556967560803026)[x,140] [f'(x),-40]
\psplot[linewidth=1.2pt,plotpoints=200]{-5.406875760298069}{6.927208122057785}{0.0011904761904761906*x^(4.0)-0.13452380952380952*x^(3.0)+0.39166666666666666*x^(2.0)+1.1345238095238095*x+0.6071428571428571}
\psline[linewidth=1.6pt,linestyle=dashed,dash=3pt 3pt](-3.9716439012649727,11.003281636345033)(-4.,0.)
\psline[linewidth=1.6pt,linestyle=dashed,dash=3pt 3pt](-2.,1.)(-2.,0.)
\psline[linewidth=1.6pt,linestyle=dashed,dash=3pt 3pt](1.,0.)(1.,2.)
\psline[linewidth=1.6pt,linestyle=dashed,dash=3pt 3pt](3.,0.)(3.,4.)
\rput[tl](-4.1015407489904305,-0.19104754096575155){$x_1$}
\rput[tl](-2.2101369570956884,-0.19104754096575155){$x_2$}
\rput[tl](-1.144557356028228,-0.19104754096575155){$x_3$}
\rput[tl](0.8534043959732601,-0.1377685609123784){$x_4$}
\rput[tl](2.824726657948062,-0.1377685609123784){$x_5$}
\begin{scriptsize}
\rput[bl](-3.5687509484567004,9.985237649228518){$f'$}
\end{scriptsize}
\end{pspicture*}}\end{center}

Kreuze die beiden zutreffenden Aussagen an!

\multiplechoice[5]{  %Anzahl der Antwortmoeglichkeiten, Standard: 5
				L1={Jede Funktion $f$ mit der Ableitungsfunktion $f'$ hat an der Stelle $x_5$ eine horizontale Tangente.},   %1. Antwortmoeglichkeit 
				L2={Es gibt eine Funktion $f$ mit der Ableitungsfunktion $f'$, deren Graph durch den Punkt $P=(0/0)$ verläuft.},   %2. Antwortmoeglichkeit
				L3={Jede Funktion $f$ mit der Ableitungsfunktion $f'$ ist im Intervall $[x_1;x_2]$ streng monoton fallend.},   %3. Antwortmoeglichkeit
				L4={Jede Funktion $f$ mit der Ableitungsfunktion $f'$ ist im Intervall $[x_3;x_4]$ streng monoton steigend.},   %4. Antwortmoeglichkeit
				L5={Die Funktionswerte $f(x)$ jeder Funktion $f$ mit der Ableitungsfunktion $f'$ sind für $x\in[x_3;x_5]$ stets positiv.},	 %5. Antwortmoeglichkeit
				L6={},	 %6. Antwortmoeglichkeit
				L7={},	 %7. Antwortmoeglichkeit
				L8={},	 %8. Antwortmoeglichkeit
				L9={},	 %9. Antwortmoeglichkeit
				%% LOESUNG: %%
				A1=2,  % 1. Antwort
				A2=4,	 % 2. Antwort
				A3=0,  % 3. Antwort
				A4=0,  % 4. Antwort
				A5=0,  % 5. Antwort
				}
\end{beispiel}