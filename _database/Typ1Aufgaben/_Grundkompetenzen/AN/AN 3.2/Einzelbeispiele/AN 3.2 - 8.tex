\section{AN 3.2 - 8 - MAT - Zusammenhang zwischen Funktion und Ableitungsfunktion - LT - Matura 2014/15 Haupttermin}

\begin{beispiel}[AN 3.2]{1} %PUNKTE DES BEISPIELS
In der folgenden Abbildung ist der Graph einer Polynomfunktion $f$ dargestellt:

\begin{center}

\psset{xunit=1.0cm,yunit=0.8cm,algebraic=true,dimen=middle,dotstyle=o,dotsize=5pt 0,linewidth=0.8pt,arrowsize=3pt 2,arrowinset=0.25}
\begin{pspicture*}(-3.38,-3.48)(3.52,4.54)
\multips(0,-3)(0,1.0){9}{\psline[linestyle=dashed,linecap=1,dash=1.5pt 1.5pt,linewidth=0.4pt,linecolor=gray]{c-c}(-3.38,0)(3.52,0)}
\multips(-3,0)(1.0,0){7}{\psline[linestyle=dashed,linecap=1,dash=1.5pt 1.5pt,linewidth=0.4pt,linecolor=gray]{c-c}(0,-3.48)(0,4.54)}
\psaxes[labelFontSize=\scriptstyle,xAxis=true,yAxis=true,Dx=1.,Dy=1.,showorigin=false,ticksize=-2pt 0,subticks=0]{->}(0,0)(-3.38,-3.48)(3.52,4.54)[$x$,140] [$f(x)$,-40]
\psplot[linewidth=1.2pt,plotpoints=200]{-3.3799999999999994}{3.519999999999997}{-x^(3.0)+4.0*x}
\begin{scriptsize}
\rput[bl](-2.1,4){$f$}
\end{scriptsize}
\end{pspicture*}
\end{center}
 \leer



\lueckentext{
				text={Die erste Ableitung der Funktion $f$ ist \gap, und daraus folgt: \gap.}, 	%Lueckentext Luecke=\gap
				L1={im Intervall $[-1; 1]$ negativ}, 		%1.Moeglichkeit links  
				L2={im Intervall $[-1; 1]$ gleich null}, 		%2.Moeglichkeit links
				L3={im Intervall $[-1; 1]$ positiv}, 		%3.Moeglichkeit links
				R1={$f$ hat im Intervall $[-1; 1]$ eine Nullstelle}, 		%1.Moeglichkeit rechts 
				R2={$f$ ist im Intervall $[-1; 1]$ streng monoton steigend}, 		%2.Moeglichkeit rechts
				R3={$f$ hat im Intervall $[-1; 1]$ eine Wendestelle}, 		%3.Moeglichkeit rechts
				%% LOESUNG: %%
				A1=3,   % Antwort links
				A2=2		% Antwort rechts 
				}

\end{beispiel}