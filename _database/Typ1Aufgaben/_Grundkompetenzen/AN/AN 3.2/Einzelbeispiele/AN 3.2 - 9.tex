\section{AN 3.2 - 9 Stammfunktion einer konstanten Funktion - OA - Matura 2014/15 - Nebentermin 1}

\begin{beispiel}[AN 3.2]{1} %PUNKTE DES BEISPIELS
In der nachstehenden Abbildung ist der Graph einer konstanten Funktion $f$ dargestellt. 

\begin{center}
\resizebox{0.5\linewidth}{!}{
\psset{xunit=1.0cm,yunit=1.0cm,algebraic=true,dimen=middle,dotstyle=o,dotsize=5pt 0,linewidth=0.8pt,arrowsize=3pt 2,arrowinset=0.25}
\begin{pspicture*}(-4.5,-4.5)(4.5,4.5)
\multips(0,-4)(0,1.0){10}{\psline[linestyle=dashed,linecap=1,dash=1.5pt 1.5pt,linewidth=0.4pt,linecolor=gray]{c-c}(-4.5,0)(4.5,0)}
\multips(-4,0)(1.0,0){10}{\psline[linestyle=dashed,linecap=1,dash=1.5pt 1.5pt,linewidth=0.4pt,linecolor=gray]{c-c}(0,-4.5)(0,4.5)}
\psaxes[labelFontSize=\scriptstyle,xAxis=true,yAxis=true,Dx=1.,Dy=1.,ticksize=-2pt 0,subticks=2]{->}(0,0)(-4.5,-4.5)(4.5,4.5)[$x$,140] [$f(x)$,-40]
\psplot{-4.5}{4.5}{(-2.-0.*x)/1.}
\rput[tl](3.0532804384030374,-1.5){$f$}
\end{pspicture*}}
\end{center}

\leer

Der Graph einer Stammfunktion $F$ von $f$ verläuft durch den Punkt $P=(1|1)$. Zeichne den Graphen der Stammfunktion $F$ im nachstehenden Koordinatensystem.

\begin{center}
\resizebox{0.5\linewidth}{!}{
\psset{xunit=1.0cm,yunit=1.0cm,algebraic=true,dimen=middle,dotstyle=o,dotsize=5pt 0,linewidth=0.8pt,arrowsize=3pt 2,arrowinset=0.25}
\begin{pspicture*}(-4.5,-4.5)(4.5,4.5)
\multips(0,-4)(0,1.0){10}{\psline[linestyle=dashed,linecap=1,dash=1.5pt 1.5pt,linewidth=0.4pt,linecolor=gray]{c-c}(-4.5,0)(4.5,0)}
\multips(-4,0)(1.0,0){10}{\psline[linestyle=dashed,linecap=1,dash=1.5pt 1.5pt,linewidth=0.4pt,linecolor=gray]{c-c}(0,-4.5)(0,4.5)}
\psaxes[labelFontSize=\scriptstyle,xAxis=true,yAxis=true,Dx=1.,Dy=1.,ticksize=-2pt 0,subticks=2]{->}(0,0)(-4.5,-4.5)(4.5,4.5)[$x$,140] [$F(x)$,-40]
\antwort{\psplot[linecolor=red]{-4.5}{4.5}{-2*x+3}
\rput[tl](3.0532804384030374,-1.827770670297025){\red{$F$}}}
\end{pspicture*}}
\end{center}

\antwort{Lösungsschlüssel:\\
Ein Punkt ist genau dann zu geben, wenn die lineare Stammfunktion F durch den Punkt $P = (1|1)$ verläuft und die Steigung $-2$ hat.}
\end{beispiel}