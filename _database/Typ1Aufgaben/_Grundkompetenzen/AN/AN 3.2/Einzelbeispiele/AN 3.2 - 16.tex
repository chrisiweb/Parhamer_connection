\section{AN 3.2 - 16 - MAT - Flächeninhalt - OA - Matura 2016/17 2. NT}

\begin{beispiel}[AN 3.2]{1} %PUNKTE DES BEISPIELS
In der nachstehenden Abbildung sind der Graph einer Polynomfunktion $f$ dritten Grades und der Graph einer ihrer Stammfunktionen $F$ dargestellt.

\begin{center}
\psset{xunit=0.8cm,yunit=0.8cm,algebraic=true,dimen=middle,dotstyle=o,dotsize=5pt 0,linewidth=0.6pt,arrowsize=3pt 2,arrowinset=0.25}
\begin{pspicture*}(-3.3,-1.3)(6.3,7.3)
\multips(0,-1)(0,0.5){19}{\psline[linestyle=dashed,linecap=1,dash=1.5pt 1.5pt,linewidth=0.4pt,linecolor=gray]{c-c}(-3.3,0)(6.3,0)}
\multips(-3,0)(0.5,0){20}{\psline[linestyle=dashed,linecap=1,dash=1.5pt 1.5pt,linewidth=0.4pt,linecolor=gray]{c-c}(0,-1.3)(0,7.3)}
\begin{scriptsize}
\psaxes[xAxis=true,yAxis=true,Dx=1.,Dy=1.,showorigin=false,ticksize=-2pt 0,subticks=0]{->}(0,0)(-3.3,-1.3)(6.3,7.3)[$x$,140] [\text{$f(x), F(x)$},-40]
\psplot[linewidth=1.5pt,plotpoints=200]{-3.000000000000002}{6.500000000000001}{-0.27802082942411105*x^(3.0)+1.1222222222222222*x^(2.0)-0.024453133795750107*x}
\psplot[linewidth=1.pt,plotpoints=200]{-3.000000000000002}{6.500000000000001}{-0.06950520735602776*x^(4.0)+0.37407407407407406*x^(3.0)-0.012226566897875053*x^(2.0)+1.0}
\rput[tl](4.64,6.86){$F$}
\rput[tl](3.62,2.14){$f$}
\end{scriptsize}
\end{pspicture*}
\end{center}

Der Graph von $f$ und die positive x-Achse begrenzen im Intervall $[0;4]$ ein endliches Flächenstück. Ermittle den Flächeninhalt dieses Flächenstücks!\leer

\antwort{$F(4)-F(0)=7-1=6$, Toleranzintervall: $[5,8;6,2]$}
\end{beispiel}