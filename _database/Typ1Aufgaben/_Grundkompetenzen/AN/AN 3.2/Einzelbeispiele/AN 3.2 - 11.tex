\section{AN 3.2 - 11 Graphen von Ableitungsfunktionen - MC - Matura 2015/16 - Nebentermin 1}

\begin{beispiel}[AN 3.2]{1} %PUNKTE DES BEISPIELS
In den unten stehenden Abbildungen sind jeweils die Graphen der Funktionen $f$, $g$ und $h$ dargestellt. \leer

In einer der sechs Abbildungen ist $g$ die erste Ableitung von $f$ und $h$ die zweite Ableitung von $f$. Kreuze diese Abbildung an.

\langmultiplechoice[6]{  %Anzahl der Antwortmoeglichkeiten, Standard: 5
				L1={\resizebox{0.9\linewidth}{!}{\psset{xunit=1.0cm,yunit=1.0cm,algebraic=true,dimen=middle,dotstyle=o,dotsize=5pt 0,linewidth=0.8pt,arrowsize=3pt 2,arrowinset=0.25}
\begin{pspicture*}(-3.619791766083621,-4.703762574277409)(9.543292745697165,5.625418430942933)
\multips(0,-4)(0,1.0){11}{\psline[linestyle=dashed,linecap=1,dash=1.5pt 1.5pt,linewidth=0.4pt,linecolor=black!60]{c-c}(-3.619791766083621,0)(9.543292745697165,0)}
\multips(-3,0)(1.0,0){14}{\psline[linestyle=dashed,linecap=1,dash=1.5pt 1.5pt,linewidth=0.4pt,linecolor=black!50]{c-c}(0,-4.703762574277409)(0,5.625418430942933)}
\psaxes[labelFontSize=\scriptstyle,xAxis=true,yAxis=true,labels=none,Dx=1.,Dy=1.,ticksize=-2pt 0,subticks=2]{->}(0,0)(-3.619791766083621,-4.703762574277409)(9.543292745697165,5.625418430942933)[$x$,140] [\mbox{$f(x), g(x), h(x)$},-40]
\psplot[linestyle=dashed,dash=2pt 2pt]{-3.619791766083621}{9.543292745697165}{-.4*x+1.2}
\psplot[linewidth=1.2pt,linestyle=dashed,dash=5pt 5pt,plotpoints=200]{-3.619791766083621}{9.543292745697165}{x^(2.0)/5.0-6.0/5.0*x}
\psplot[linewidth=1.2pt,plotpoints=200]{-3.619791766083621}{9.543292745697165}{1.0/15.0*x^(3.0)-3.0/5.0*x^(2.0)+3.0}
\rput[tl](-3,2){$h$}
\rput[tl](6,-0.1946777893062211){$g$}
\rput[tl](1.6507390545288063,2.68557460638022){$f$}
\end{pspicture*}}},   %1. Antwortmoeglichkeit 
				L2={\resizebox{0.9\linewidth}{!}{\psset{xunit=1.0cm,yunit=1.0cm,algebraic=true,dimen=middle,dotstyle=o,dotsize=5pt 0,linewidth=0.8pt,arrowsize=3pt 2,arrowinset=0.25}
\begin{pspicture*}(-3.619791766083621,-4.703762574277409)(9.543292745697165,5.625418430942933)
\multips(0,-4)(0,1.0){11}{\psline[linestyle=dashed,linecap=1,dash=1.5pt 1.5pt,linewidth=0.4pt,linecolor=black!60]{c-c}(-3.619791766083621,0)(9.543292745697165,0)}
\multips(-3,0)(1.0,0){14}{\psline[linestyle=dashed,linecap=1,dash=1.5pt 1.5pt,linewidth=0.4pt,linecolor=black!50]{c-c}(0,-4.703762574277409)(0,5.625418430942933)}
\psaxes[labelFontSize=\scriptstyle,xAxis=true,yAxis=true,labels=none,Dx=1.,Dy=1.,ticksize=-2pt 0,subticks=2]{->}(0,0)(-3.619791766083621,-4.703762574277409)(9.543292745697165,5.625418430942933)[$x$,140] [\mbox{$f(x), g(x), h(x)$},-40]
\psplot[linestyle=dashed,dash=2pt 2pt]{-3.619791766083621}{9.543292745697165}{.4*x-1.2}
\psplot[linewidth=1.2pt,linestyle=dashed,dash=5pt 5pt,plotpoints=200]{-3.619791766083621}{9.543292745697165}{x^(2.0)/5.0-6.0/5.0*x}
\psplot[linewidth=1.2pt,plotpoints=200]{-3.619791766083621}{9.543292745697165}{1.0/15.0*x^(3.0)-3.0/5.0*x^(2.0)+3.0}
\rput[tl](5.252709866605641,1.5){$h$}
\rput[tl](6,-0.1946777893062211){$g$}
\rput[tl](1.6507390545288063,2.68557460638022){$f$}
\end{pspicture*}}},   %2. Antwortmoeglichkeit
				L3={\resizebox{0.9\linewidth}{!}{\psset{xunit=1.0cm,yunit=1.0cm,algebraic=true,dimen=middle,dotstyle=o,dotsize=5pt 0,linewidth=0.8pt,arrowsize=3pt 2,arrowinset=0.25}
\begin{pspicture*}(-3.619791766083621,-4.703762574277409)(9.543292745697165,5.625418430942933)
\multips(0,-4)(0,1.0){11}{\psline[linestyle=dashed,linecap=1,dash=1.5pt 1.5pt,linewidth=0.4pt,linecolor=black!60]{c-c}(-3.619791766083621,0)(9.543292745697165,0)}
\multips(-3,0)(1.0,0){14}{\psline[linestyle=dashed,linecap=1,dash=1.5pt 1.5pt,linewidth=0.4pt,linecolor=black!50]{c-c}(0,-4.703762574277409)(0,5.625418430942933)}
\psaxes[labelFontSize=\scriptstyle,xAxis=true,yAxis=true,labels=none,Dx=1.,Dy=1.,ticksize=-2pt 0,subticks=2]{->}(0,0)(-3.619791766083621,-4.703762574277409)(9.543292745697165,5.625418430942933)[$x$,140] [\mbox{$f(x), g(x), h(x)$},-40]
\psplot[linestyle=dashed,dash=2pt 2pt]{-3.619791766083621}{9.543292745697165}{-.4*x+1.2}
\psplot[linewidth=1.2pt,linestyle=dashed,dash=5pt 5pt,plotpoints=200]{-3.619791766083621}{9.543292745697165}{-x^(2.0)/5.0+6.0/5.0*x}
\psplot[linewidth=1.2pt,plotpoints=200]{-3.619791766083621}{9.543292745697165}{1.0/15.0*x^(3.0)-3.0/5.0*x^(2.0)+3.0}
\rput[tl](-3,2){$h$}
\rput[tl](5.5,1.3){$g$}
\rput[tl](1.6507390545288063,2.68557460638022){$f$}
\end{pspicture*}}},   %3. Antwortmoeglichkeit
				L4={\resizebox{0.9\linewidth}{!}{\psset{xunit=1.0cm,yunit=1.0cm,algebraic=true,dimen=middle,dotstyle=o,dotsize=5pt 0,linewidth=0.8pt,arrowsize=3pt 2,arrowinset=0.25}
\begin{pspicture*}(-3.619791766083621,-4.703762574277409)(9.543292745697165,5.625418430942933)
\multips(0,-4)(0,1.0){11}{\psline[linestyle=dashed,linecap=1,dash=1.5pt 1.5pt,linewidth=0.4pt,linecolor=black!60]{c-c}(-3.619791766083621,0)(9.543292745697165,0)}
\multips(-3,0)(1.0,0){14}{\psline[linestyle=dashed,linecap=1,dash=1.5pt 1.5pt,linewidth=0.4pt,linecolor=black!50]{c-c}(0,-4.703762574277409)(0,5.625418430942933)}
\psaxes[labelFontSize=\scriptstyle,xAxis=true,yAxis=true,labels=none,Dx=1.,Dy=1.,ticksize=-2pt 0,subticks=2]{->}(0,0)(-3.619791766083621,-4.703762574277409)(9.543292745697165,5.625418430942933)[$x$,140] [\mbox{$f(x), g(x), h(x)$},-40]
\psplot[linestyle=dashed,dash=2pt 2pt]{-3.619791766083621}{9.543292745697165}{-.4*x+1.2}
\psplot[linewidth=1.2pt,linestyle=dashed,dash=5pt 5pt,plotpoints=200]{-3.619791766083621}{9.543292745697165}{x^(2.0)/5.0-6.0/5.0*x}
\psplot[linewidth=1.2pt,plotpoints=200]{-3.619791766083621}{9.543292745697165}{-1.0/15.0*x^(3.0)+3.0/5.0*x^(2.0)-3.0}
\rput[tl](-3.3,2.4){$h$}
\rput[tl](6,-0.1946777893062211){$g$}
\rput[tl](3.5,2.68557460638022){$f$}
\end{pspicture*}}},   %4. Antwortmoeglichkeit
				L5={\resizebox{0.9\linewidth}{!}{\psset{xunit=1.0cm,yunit=1.0cm,algebraic=true,dimen=middle,dotstyle=o,dotsize=5pt 0,linewidth=0.8pt,arrowsize=3pt 2,arrowinset=0.25}
\begin{pspicture*}(-3.619791766083621,-4.703762574277409)(9.543292745697165,5.625418430942933)
\multips(0,-4)(0,1.0){11}{\psline[linestyle=dashed,linecap=1,dash=1.5pt 1.5pt,linewidth=0.4pt,linecolor=black!60]{c-c}(-3.619791766083621,0)(9.543292745697165,0)}
\multips(-3,0)(1.0,0){14}{\psline[linestyle=dashed,linecap=1,dash=1.5pt 1.5pt,linewidth=0.4pt,linecolor=black!50]{c-c}(0,-4.703762574277409)(0,5.625418430942933)}
\psaxes[labelFontSize=\scriptstyle,xAxis=true,yAxis=true,labels=none,Dx=1.,Dy=1.,ticksize=-2pt 0,subticks=2]{->}(0,0)(-3.619791766083621,-4.703762574277409)(9.543292745697165,5.625418430942933)[$x$,140] [\mbox{$f(x), g(x), h(x)$},-40]
\psplot[linestyle=dashed,dash=2pt 2pt]{-3.619791766083621}{9.543292745697165}{.4*x-1.2}
\psplot[linewidth=1.2pt,linestyle=dashed,dash=5pt 5pt,plotpoints=200]{-3.619791766083621}{9.543292745697165}{-x^(2.0)/5.0+6.0/5.0*x}
\psplot[linewidth=1.2pt,plotpoints=200]{-3.619791766083621}{9.543292745697165}{-1.0/15.0*x^(3.0)+3.0/5.0*x^(2.0)-3.0}
\rput[tl](5.252709866605641,1.5){$h$}
\rput[tl](5.5,1.3){$g$}
\rput[tl](3.5,2.68557460638022){$f$}
\end{pspicture*}}},	 %5. Antwortmoeglichkeit
				L6={\resizebox{0.9\linewidth}{!}{\psset{xunit=1.0cm,yunit=1.0cm,algebraic=true,dimen=middle,dotstyle=o,dotsize=5pt 0,linewidth=0.8pt,arrowsize=3pt 2,arrowinset=0.25}
\begin{pspicture*}(-3.619791766083621,-4.703762574277409)(9.543292745697165,5.625418430942933)
\multips(0,-4)(0,1.0){11}{\psline[linestyle=dashed,linecap=1,dash=1.5pt 1.5pt,linewidth=0.4pt,linecolor=black!60]{c-c}(-3.619791766083621,0)(9.543292745697165,0)}
\multips(-3,0)(1.0,0){14}{\psline[linestyle=dashed,linecap=1,dash=1.5pt 1.5pt,linewidth=0.4pt,linecolor=black!50]{c-c}(0,-4.703762574277409)(0,5.625418430942933)}
\psaxes[labelFontSize=\scriptstyle,xAxis=true,yAxis=true,labels=none,Dx=1.,Dy=1.,ticksize=-2pt 0,subticks=2]{->}(0,0)(-3.619791766083621,-4.703762574277409)(9.543292745697165,5.625418430942933)[$x$,140] [\mbox{$f(x), g(x), h(x)$},-40]
\psplot[linestyle=dashed,dash=2pt 2pt]{-3.619791766083621}{9.543292745697165}{.4*x-1.2}
\psplot[linewidth=1.2pt,linestyle=dashed,dash=5pt 5pt,plotpoints=200]{-3.619791766083621}{9.543292745697165}{x^(2.0)/5.0-6.0/5.0*x}
\psplot[linewidth=1.2pt,plotpoints=200]{-3.619791766083621}{9.543292745697165}{-1.0/15.0*x^(3.0)+3.0/5.0*x^(2.0)-3.0}
\rput[tl](5.252709866605641,1.5){$h$}
\rput[tl](6,-0.1946777893062211){$g$}
\rput[tl](3.5,2.68557460638022){$f$}
\end{pspicture*}}},	 %6. Antwortmoeglichkeit
				L7={},	 %7. Antwortmoeglichkeit
				L8={},	 %8. Antwortmoeglichkeit
				L9={},	 %9. Antwortmoeglichkeit
				%% LOESUNG: %%
				A1=2,  % 1. Antwort
				A2=0,	 % 2. Antwort
				A3=0,  % 3. Antwort
				A4=0,  % 4. Antwort
				A5=0,  % 5. Antwort
				}

\end{beispiel}