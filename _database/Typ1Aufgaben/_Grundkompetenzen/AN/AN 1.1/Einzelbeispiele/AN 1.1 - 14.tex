\section{AN 1.1 - 14 - Änderungsraten der Anzahl der Mindestsicherungsbeziehenden - MC - AngMac UNIVIE}

\begin{beispiel}[AN 1.1]{1}
Die Wiener Mindestsicherung (WMS) ist eine finanzielle Unterstützung für Menschen mit geringem oder keinem Einkommen um die Lebenshaltungskosten und Mietkosten zu decken. Die Stadt Wien veröffentlichte zur Anzahl der Personen, die die WMS beziehen, folgende Statistik: \newline

\begin{tabular}{l|c|c} 
\rowcolor{lightgray}
Anzahl der WMS-Beziehenden & 2016 & 2017 \\ \hline
\textbf{Jahresdurchschnitt} & \textbf{ 146 597} & \textbf{150 150} \\ 
April & 147 735 & 153 384 \\ 
November & 149 502 & 148 143  
\end{tabular} \newline

\begin{singlespace}\scriptsize Quelle: Stadt Wien MA 40 https://www.wien.gv.at/kontakte/ma40/pdf/jahresstatistik-mindestsicherung-2017.pdf, Seite 14 [Zugriff: 29.11.2019]\end{singlespace}
\normalsize


\subsubsection{Aufgabenstellung:}
Welche Aussagen über die Änderung der Anzahl der WMS-Beziehenden sind korrekt? \\
Kreuze die beiden zutreffenden Aussagen an!
\multiplechoice[5]{  %Anzahl der Antwortmoeglichkeiten, Standard: 5
				L1={Die relative Änderungsrate der Anzahl der WMS-Beziehenden von November 2016 auf November 2017 beträgt circa 0,9 \%. },   %1. Antwortmoeglichkeit 
				L2={Die absolute Änderung der Anzahl der WMS-Beziehenden von April 2016 auf April 2017 beträgt circa 3,7 \%. },   %2. Antwortmoeglichkeit
				L3={Die relative Änderungsrate der Anzahl der WMS-Beziehenden von November 2016 auf November 2017 hat ein negatives Vorzeichen.},   %3. Antwortmoeglichkeit
				L4={Die absolute Änderung der Anzahl der WMS-Beziehenden von April 2016 auf April 2017 beträgt 153 384 Personen.},   %4. Antwortmoeglichkeit
				L5={Die Anzahl der WMS-Beziehenden im Jahresdurchschnitt 2017 ist im Vergleich zum Jahresdurchschnitt 2016 um circa 2,4 \% gestiegen.},	 %5. Antwortmoeglichkeit
				L6={},	 %6. Antwortmoeglichkeit
				L7={},	 %7. Antwortmoeglichkeit
				L8={},	 %8. Antwortmoeglichkeit
				L9={},	 %9. Antwortmoeglichkeit
				%% LOESUNG: %%
				A1=3,  % 1. Antwort
				A2=5,	 % 2. Antwort
				A3=0,  % 3. Antwort
				A4=0,  % 4. Antwort
				A5=0,  % 5. Antwort
				}
\end{beispiel}