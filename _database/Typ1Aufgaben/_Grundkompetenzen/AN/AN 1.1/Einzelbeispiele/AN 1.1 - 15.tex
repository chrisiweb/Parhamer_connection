\section{AN 1.1 - 15 - K5 - Sammelkartenspiel - OA - CleTur}

\begin{beispiel}[AN 1.1]{1} %PUNKTE DES BEISPIELS
Die Preise von Karten eines Sammelkartenspiels werden in einer Datenbank erfasst.\\ 
Der Preis einer begehrten Karte lag am 1.1.2015 bei \euro 42. Die absolute Preisänderung der Karte vom 1.1.2015 bis zum 1.1.2019 betrug \euro 33. Vom 1.1.2018 bis zum 1.1.2019 verzeichnete die Karte eine Preiserhöhung von 189 \%

Zeichne einen möglichen Verlauf des Preises für diese Karte in das Koordinatensystem!\leer

\psset{xunit=1.0cm,yunit=0.1cm,algebraic=true,dimen=middle,dotstyle=o,dotsize=5pt 0,linewidth=1.6pt,arrowsize=3pt 2,arrowinset=0.25}
\begin{pspicture*}(-1.22,-7.04)(13.88,91.36)
\psaxes[labelFontSize=\scriptstyle,xAxis=true,yAxis=true,labels=y,Dx=2.,Dy=10.,ticksize=-2pt 0,subticks=0]{->}(0,0)(0.,0.)(13.88,91.36)[Datum,140] [Preis in \euro,-40]
\rput[tl](1.65,-2.5){1.1.15}
\rput[tl](3.65,-2.5){1.1.16}
\rput[tl](5.65,-2.5){1.1.17}
\rput[tl](7.65,-2.5){1.1.18}
\rput[tl](9.65,-2.5){1.1.19}
\rput[tl](11.65,-2.5){1.1.20}
\rput[tl](-0.35,-2.5){1.1.14}
\end{pspicture*}

\antwort{Ein Punkt ist zu geben, wenn alle 3 Punkte im Koordinatensystem korrekt eingezeichnet sind. Der Verlauf zwischen den Punkten ist frei zu wählen. Die 3 Punkte lauten wie folgt:
\\ (1.1.15/42)
\\ (1.1.18/26)
\\ (1.1.19/75)}
\end{beispiel}