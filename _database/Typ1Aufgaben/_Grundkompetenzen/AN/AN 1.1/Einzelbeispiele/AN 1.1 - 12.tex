\section{AN 1.1 - 12 - MAT - Nächtigungen in österreichischen Jugendherbergen - OA - Matura 2. NT 2017/18}

\begin{beispiel}[AN 1.1]{1}
Der Wert $N_{12}$ gibt die Anzahl der Nächtigungen in österreichischen Jugendherbergen im Jahr 2012 an, der Wert $N_{13}$ jene im Jahr 2013.

Gib die Bedeutung der Gleichung $\dfrac{N_{12}}{N_{13}}=1,012$ für die Veränderung der Anzahl der Nächtigungen in österreichischen Jugendherbergen an!

\antwort{Im Jahr 2013 gab es um $1,2\,\%$ mehr Nächtigungen in österreichischen Jugendherbergen als im Jahr 2012.}
\end{beispiel}