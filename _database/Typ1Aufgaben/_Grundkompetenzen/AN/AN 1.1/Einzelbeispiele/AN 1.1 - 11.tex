\section{AN 1.1 - 11 - MAT - Wertschöpfung - OA - Matura HT 2017/18}

\begin{beispiel}[AN 1.1]{1} %PUNKTE DES BEISPIELS
Gegeben ist eine Grafik der Arbeiterkammer.

\begin{center}
	
	\textbf{AK-Wertschöpfungsbarometer}
	
	Überschuss pro Beschäftigtem 2003 bis 2009
	
	\resizebox{1\linewidth}{!}{\begin{large}\newrgbcolor{wqwqwq}{0.3764705882352941 0.3764705882352941 0.3764705882352941}
\newrgbcolor{aqaqaq}{0.6274509803921569 0.6274509803921569 0.6274509803921569}
\psset{xunit=1.5cm,yunit=0.1cm,algebraic=true,dimen=middle,dotstyle=o,dotsize=5pt 0,linewidth=1.6pt,arrowsize=3pt 2,arrowinset=0.25}
\begin{pspicture*}(-1.1052501442819616,-34.43225559566844)(15.087827240577793,107.09537851604473)
\multips(0,0)(0,10.0){16}{\psline[linestyle=dashed,linecap=1,dash=1.5pt 1.5pt,linewidth=0.4pt,linecolor=darkgray]{c-c}(0,0)(15.087827240577793,0)}
\multips(0,0)(100.0,0){1}{\psline[linestyle=dashed,linecap=1,dash=1.5pt 1.5pt,linewidth=0.4pt,linecolor=darkgray]{c-c}(0,0)(0,107.09537851604473)}
\psaxes[labelFontSize=\scriptstyle,xAxis=true,yAxis=true,labels=y,Dx=2.,Dy=10.,ticksize=-2pt 0,subticks=2](0,0)(0.,0.)(15.087827240577793,107.09537851604473)
\pspolygon[linewidth=0.pt,linecolor=wqwqwq,fillcolor=wqwqwq,fillstyle=solid,opacity=0.7](0.5,0.)(0.5,73.634)(1.,73.634)(1.,0.)
\pspolygon[linewidth=0.pt,linecolor=aqaqaq,fillcolor=aqaqaq,fillstyle=solid,opacity=0.5](1.,0.)(1.,49.416)(2.,49.416)(2.,0.)
\pspolygon[linewidth=0.pt,fillcolor=black,fillstyle=solid,opacity=0.7](1.,49.416)(1.,73.634)(2.,73.634)(2.,49.416)
\pspolygon[linewidth=0.pt,linecolor=wqwqwq,fillcolor=wqwqwq,fillstyle=solid,opacity=0.7](2.5,0.)(2.5,76.906)(3.,76.906)(3.,0.)
\pspolygon[linewidth=0.pt,linecolor=aqaqaq,fillcolor=aqaqaq,fillstyle=solid,opacity=0.5](3.,0.)(3.,50.568)(4.,50.568)(4.,0.)
\pspolygon[linewidth=0.pt,fillcolor=black,fillstyle=solid,opacity=0.7](3.,50.568)(3.,76.906)(4.,76.906)(4.,50.568)
\pspolygon[linewidth=0.pt,linecolor=wqwqwq,fillcolor=wqwqwq,fillstyle=solid,opacity=0.7](4.5,0.)(4.5,80.464)(5.,80.464)(5.,0.)
\pspolygon[linewidth=0.pt,linecolor=aqaqaq,fillcolor=aqaqaq,fillstyle=solid,opacity=0.5](5.,0.)(5.,52.168)(6.,52.168)(6.,0.)
\pspolygon[linewidth=0.pt,fillcolor=black,fillstyle=solid,opacity=0.7](5.,52.168)(5.,80.464)(6.,80.464)(6.,52.168)
\pspolygon[linewidth=0.pt,linecolor=wqwqwq,fillcolor=wqwqwq,fillstyle=solid,opacity=0.7](6.5,0.)(6.5,85.252)(7.,85.252)(7.,0.)
\pspolygon[linewidth=0.pt,linecolor=aqaqaq,fillcolor=aqaqaq,fillstyle=solid,opacity=0.5](7.,0.)(7.,53.834)(8.,53.834)(8.,0.)
\pspolygon[linewidth=0.pt,fillcolor=black,fillstyle=solid,opacity=0.7](7.,85.252)(7.,53.834)(8.,53.834)(8.,85.252)
\pspolygon[linewidth=0.pt,linecolor=wqwqwq,fillcolor=wqwqwq,fillstyle=solid,opacity=0.7](8.5,0.)(8.5,92.258)(9.,92.258)(9.,0.)
\pspolygon[linewidth=0.pt,linecolor=aqaqaq,fillcolor=aqaqaq,fillstyle=solid,opacity=0.5](9.,0.)(9.,55.125)(10.,55.125)(10.,0.)
\pspolygon[linewidth=0.pt,fillcolor=black,fillstyle=solid,opacity=0.7](9.,92.258)(9.,55.125)(10.,55.125)(10.,92.258)
\pspolygon[linewidth=0.pt,linecolor=wqwqwq,fillcolor=wqwqwq,fillstyle=solid,opacity=0.7](10.5,0.)(10.5,94.282)(11.,94.282)(11.,0.)
\pspolygon[linewidth=0.pt,linecolor=aqaqaq,fillcolor=aqaqaq,fillstyle=solid,opacity=0.5](11.,0.)(11.,57.321)(12.,57.321)(12.,0.)
\pspolygon[linewidth=0.pt,fillcolor=black,fillstyle=solid,opacity=0.7](11.,94.282)(11.,57.321)(12.,57.321)(12.,94.282)
\pspolygon[linewidth=0.pt,linecolor=wqwqwq,fillcolor=wqwqwq,fillstyle=solid,opacity=0.7](12.5,0.)(12.5,92.006)(13.,92.006)(13.,0.)
\pspolygon[linewidth=0.pt,linecolor=aqaqaq,fillcolor=aqaqaq,fillstyle=solid,opacity=0.5](13.,0.)(13.,55.063)(14.,55.063)(14.,0.)
\pspolygon[linewidth=0.pt,fillcolor=black,fillstyle=solid,opacity=0.7](13.,55.063)(13.,92.006)(14.,92.006)(14.,55.063)
\pspolygon[linewidth=0.pt,fillcolor=wqwqwq,fillstyle=solid,opacity=0.7](0.5,-13.6)(0.5,-18.6)(0.8,-18.6)(0.8,-13.6)
\rput[tl](1.,-14.5){durchschnittliche Wertschöpfung pro Beschäftigtem}
\pspolygon[linewidth=0.pt,fillcolor=aqaqaq,fillstyle=solid,opacity=0.7](0.5,-20.6)(0.5,-25.6)(0.8,-25.6)(0.8,-20.6)
\rput[tl](1.,-21.5){durchschnittlicher Personalaufwand pro Beschäftigtem}
\pspolygon[linewidth=0.pt,fillcolor=black,fillstyle=solid,opacity=0.7](0.5,-27.6)(0.5,-32.6)(0.8,-32.6)(0.8,-27.6)
\rput[tl](1.,-28.5){Überschuss pro Beschäftigtem}
\rput[tl](0.84,-3){2003}
\rput[tl](2.84,-3){2004}
\rput[tl](4.84,-3){2005}
\rput[tl](6.84,-3){2006}
\rput[tl](8.84,-3){2007}
\rput[tl](10.84,-3){2008}
\rput[tl](12.84,-3){2009}
\rput[tl](1.05,64.85){$24\,218$}
\rput[tl](3.05,64.85){$26\,338$}
\rput[tl](5.05,64.85){$28\,296$}
\rput[tl](7.05,64.85){$31\,418$}
\rput[tl](9.05,64.85){$37\,133$}
\rput[tl](11.05,64.85){$36\,961$}
\rput[tl](13.05,64.85){$36\,943$}
\rput[tl](1.1,44.86){$49\,416$}
\rput[tl](3.1,44.86){$50\,568$}
\rput[tl](5.1,44.86){$52\,168$}
\rput[tl](7.1,44.86){$53\,834$}
\rput[tl](9.1,44.86){$55\,125$}
\rput[tl](11.1,44.86){$57\,321$}
\rput[tl](13.1,44.86){$55\,063$}
\rput[tl](0.65,55){$\rotatebox{90}{73\,634}$}
\rput[tl](2.65,55){$\rotatebox{90}{76\,906}$}
\rput[tl](4.65,55){$\rotatebox{90}{80\,464}$}
\rput[tl](6.65,55){$\rotatebox{90}{85\,252}$}
\rput[tl](8.65,55){$\rotatebox{90}{92\,258}$}
\rput[tl](10.65,55){$\rotatebox{90}{94\,282}$}
\rput[tl](12.65,55){$\rotatebox{90}{92\,006}$}
\end{pspicture*}\end{large}}
\end{center}

Der AK-Wertschöpfungsbarometer zeigt die Entwicklung desjenigen Wertes auf, den österreichische Mittel- und Großbetriebe im Durchschnitt an jeder Mitarbeiterin/jedem Mitarbeiter pro Jahr verdienen.

Konkret ermittelt wird dabei der Überschuss pro Beschäftigtem, also die Differenz zwischen der durchschnittlichen Wertschöpfung pro Beschäftigtem und dem durchschnittlichen Personalaufwand pro Beschäftigtem.

Berechnen Sie für das Jahr 2007 den Anteil dieses Überschusses (in Prozent) gemessen an der Pro-Kopf-Wertschöpfung!

\antwort{Anteil des Überschusses im Jahr 2007: $\frac{37\,133}{92\,258}\approx 0,4025=40,25\,\%$}

\end{beispiel}