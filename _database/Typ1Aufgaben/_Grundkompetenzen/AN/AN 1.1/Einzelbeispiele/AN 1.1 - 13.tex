\section{AN 1.1 - 13 - MAT - Kriminalstatistik 2010-2011 - OA - Matura-HT-18/19}

\begin{beispiel}[AN 1.1]{1}
Die nachstehende Tabelle gibt an, wie viele Kriminalfälle in jedem Bundesland in Österreich in den Jahren 2010 und 2011 angezeigt wurden.

\begin{tabular}{|l|r|r|}\hline
\cellcolor[gray]{0.9}Bundesland&\cellcolor[gray]{0.9}angezeigte Kriminalfälle 2010&\cellcolor[gray]{0.9}angezeigte Kriminalfälle 2011\\ \hline
Burgenland&9\,306&10\,391\\ \hline
Kärnten&30\,192&29\,710\\ \hline
Niederösterreich&73\,146&78\,634\\ \hline
Oberösterreich&66\,141&67\,477\\ \hline
Salzburg&29\,382&30\,948\\ \hline
Steiermark&55\,167&55\,472\\ \hline
Tirol&44\,185&45\,944\\ \hline
Vorarlberg&20\,662&20\,611\\ \hline
Wien&207\,564&200\,820\\ \hline
\end{tabular}

Gib für das Burgenland die relative Änderung der angezeigten Kriminalfälle im Jahr 2011 im Vergleich zum Jahr 2010 an!\leer

\antwort{$\dfrac{10\,391-9\,306}{9\,306}\approx 0,117$

Die relative Änderung beträgt ca. 0,117.}
\end{beispiel}