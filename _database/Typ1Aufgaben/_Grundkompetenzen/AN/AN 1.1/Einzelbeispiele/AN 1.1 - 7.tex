\section{AN 1.1 - 7 Prozente - MC - Matura 2013/14 Haupttermin}

\begin{beispiel}{1} %PUNKTE DES BEISPIELS
			Zahlenangaben in Prozent $(\%)$ machen Anteile unterschiedlicher Größen vergleichbar.
			
			Kreuze beide zutreffenden Aussagen!\leer
			
			\multiplechoice[5]{  %Anzahl der Antwortmoeglichkeiten, Standard: 5
							L1={Peters monatliches Taschengeld wurde von \EUR{80} auf \EUR{100} erhöht. Somit bekommt er jetzt um 20\,\% mehr als vorher.},   %1. Antwortmoeglichkeit 
							L2={Ein Preis ist im Laufe der letzten fünf Jahre um 10\,\% gestiegen. Das bedeutet in jedem Jahr eine Steigerung von 2\,\% gegenüber dem Vorjahr.},   %2. Antwortmoeglichkeit
							L3={Wenn die Inflationsrate in den letzten Monaten von 2\,\% auf 1,5\,\% gesunken ist, bedeutet das eine relative Abnahme der Inflationsrate um $25\,\%.$},   %3. Antwortmoeglichkeit
							L4={Wenn ein Preis zunächst um 20\,\% gesenkt und kurze Zeit darauf wieder um 5\,\% erhöht wurde, dann ist er jetzt um 15\,\% niedriger als ursprünglich.},   %4. Antwortmoeglichkeit
							L5={Eine Zunahme um 200\,\% bedeutet eine Steigerung auf das Dreifache.},	 %5. Antwortmoeglichkeit
							L6={},	 %6. Antwortmoeglichkeit
							L7={},	 %7. Antwortmoeglichkeit
							L8={},	 %8. Antwortmoeglichkeit
							L9={},	 %9. Antwortmoeglichkeit
							%% LOESUNG: %%
							A1=3,  % 1. Antwort
							A2=5,	 % 2. Antwort
							A3=0,  % 3. Antwort
							A4=0,  % 4. Antwort
							A5=0,  % 5. Antwort
							}
\end{beispiel}
