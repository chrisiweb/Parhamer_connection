\section{AN 1.1 - 5 Preis�nderungen - LT - Matura 2014/15 - Haupttermin}

\begin{beispiel}[AN 1.1]{1} %PUNKTE DES BEISPIELS
Ein Fernsehger�t wurde im Jahr 2012 zum Preis $P_0$ verkauft, das gleiche Ger�t wurde im Jahr
2014 zum Preis $P_2$ verkauft.

\lueckentext{
				text={Der Term \gap gibt die absolute Preis�nderung von 2012 auf 2014 an, der Term \gap die relative Preis�nderung von 2012 auf 2014.}, 	%Lueckentext Luecke=\gap
				L1={$\dfrac{P_0}{P_2}$}, 		%1.Moeglichkeit links  
				L2={$P_2-P_0$}, 		%2.Moeglichkeit links
				L3={$\dfrac{P_2-P_0}{2}$}, 		%3.Moeglichkeit links
				R1={$\dfrac{P_2}{P_0}$}, 		%1.Moeglichkeit rechts 
				R2={$\dfrac{P_0-P_2}{2}$}, 		%2.Moeglichkeit rechts
				R3={$\dfrac{P_2-P_0}{P_0}$}, 		%3.Moeglichkeit rechts
				%% LOESUNG: %%
				A1=2,   % Antwort links
				A2=3		% Antwort rechts 
				}
\end{beispiel}