\section{AN 1.1 - 8 - MAT - Leistungsverbesserung - OA - Matura 2016/17 - Haupttermin}

\begin{beispiel}[AN 1.1]{1} %PUNKTE DES BEISPIELS
Drei Personen $A$, $B$ und $C$ absolvieren jeweils vor und nach einem Spezialtraining denselben
Koordinationstest. In der nachstehenden Tabelle sind die dabei erreichten Punkte angeführt.

\begin{tabular}{|l|c|c|c|}\hhline{~|---}
\multicolumn{1}{c|}{}& \cellcolor{black!20} Person $A$ & \cellcolor{black!20}Person $B$ & \cellcolor{black!20}Person $C$ \\ \hline
\cellcolor{black!20}erreichte Punkte vor dem Spezialtraining & 5 & 15 & 20 \\ \hline
\cellcolor{black!20}erreichte Punkte nach dem Spezialtraining & 8 & 19 & 35 \\ \hline
\end{tabular}\leer

Gute Leistungen sind durch hohe Punktezahlen gekennzeichnet. Wie aus der Tabelle ersichtlich
ist, erreichen alle drei Personen nach dem Spezialtraining mehr Punkte als vorher. \leer

Wähle aus den Personen $A$, $B$ und $C$ die beiden aus, die die nachstehenden Bedingungen
erfüllen! 

\begin{itemize}
	\item Bei der ersten Person ist die absolute Änderung der Punktezahl größer als bei der zweiten.
	\item Bei der zweiten Person ist die relative Änderung der Punktezahl größer als bei der ersten Person.
\end{itemize}

erste Person: \antwort[\rule{5cm}{0.3pt}]{Person $B$}

zweite Person: \antwort[\rule{5cm}{0.3pt}]{Person $A$}
\end{beispiel}