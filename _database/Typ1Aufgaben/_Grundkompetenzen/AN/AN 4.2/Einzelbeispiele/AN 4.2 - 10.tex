\section{AN 4.2 - 10 - MAT - Bestimmen eines Koeffizienten - OA - Matura 1.NT 2018/19}

\begin{beispiel}[AN 4.2]{1}
Gegeben ist die Funktion $f$: $\mathbb{R}\rightarrow\mathbb{R}$ mit $f(x)=a\cdot x^2+2$ mit $a\in\mathbb{R}$.

Gib den Wert des Koeffizienten $a$ so an, dass die Gleichung $\displaystyle\int^1_0 f(x)\,\text{d}x=1$ erfüllt ist.\leer

$a=\antwort[\rule{5cm}{0.3pt}]{-3}$
\end{beispiel}