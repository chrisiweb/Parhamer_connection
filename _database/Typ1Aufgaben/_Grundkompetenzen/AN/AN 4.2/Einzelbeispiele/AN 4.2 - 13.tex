\section{AN 4.2 - 13 - MAT - Fläche zwischen zwei Funktionen - OA - FilJan UNIVIE}

\begin{beispiel}[AN 4.2]{1}
Die folgende Abbildung zeigt den Graphen der Funktion $g$ mit $g(x)=-2x^2+1$. Weiteres sei die Polynomfunktion $f$ mit $f(x)=4x^3-4x$ gegeben.
\begin{center}
\newrgbcolor{ttqqtt}{0.2 0 0.2}
\psset{showorigin = false, xunit=1cm,yunit=1cm,algebraic=true,dimen=middle,dotstyle=o,dotsize=5pt 0,linewidth=0.8pt,arrowsize=5pt 2,arrowinset=0.25}
\begin{pspicture*}(-2.5,-2.5)(3,3.3)
\multips(0,-2)(0,1){5}{\psline[linestyle=dashed,linecap=1,dash=1.5pt 1.5pt,linewidth=0.4pt,linecolor=darkgray]{c-c}(-4.011783136876828,0)(2.5,0)}
\multips(-2,0)(1,0){5}{\psline[linestyle=dashed,linecap=1,dash=1.5pt 1.5pt,linewidth=0.4pt,linecolor=darkgray]{c-c}(0,-4.18658121324868)(0,2.5)}
\psaxes[labelFontSize=\scriptstyle,xAxis=true,yAxis=true,Dx=1,Dy=1,ticksize=-2pt 0,subticks=0]{->}(0,0)(-4.011783136876828,-4.18658121324868)(2.5,2.5)
\psplot[linewidth=2pt,linecolor=ttqqtt,plotpoints=200]{-4.011783136876828}{8.47768410514725}{-2*x^(2)+1}
\begin{scriptsize}
\rput[bl](0.6,0.7){\large $g$} 
\rput[bl](2.55,-0.1){\normalsize $x$} 
\rput[bl](-0.9,2.6){\normalsize $F(x),g(x)$} 
\end{scriptsize}
\end{pspicture*}
\end{center}

Bestimme die Stammfunktion $F$ von $f$, die durch den Punkt $P(0\mid 0)$ geht und zeichne die eingeschlossene Fläche zwischen $g$ und $F$ mit den gegebenen Schnittpunkten $S_1(-1\mid -1)$ und $S_2(1\mid -1)$ ein!

\antwort{
Lösungserwartung:

$F(x)=\int (4x^3-4x)$ $dx = x^4-2x^2+c$ $(c \in \mathbb{R})$

Punkt $P(0\mid 0)$ in $F$ einsetzen: $c=0$

Stammfunktion: $F(x)=x^4-2x^2$

\begin{center}
\newrgbcolor{ttqqtt}{0.2 0 0.2}
\psset{showorigin = false, xunit=1cm,yunit=1cm,algebraic=true,dimen=middle,dotstyle=o,dotsize=5pt 0,linewidth=0.5pt,arrowsize=5pt 2,arrowinset=0.25}
\begin{pspicture*}(-2.5,-2.5)(3,3.3)
\multips(0,-4)(0,1){12}{\psline[linestyle=dashed,linecap=1,dash=1.5pt 1.5pt,linewidth=0.4pt,linecolor=darkgray]{c-c}(-4.011783136876828,0)(8.47768410514725,0)}
\multips(-4,0)(1,0){13}{\psline[linestyle=dashed,linecap=1,dash=1.5pt 1.5pt,linewidth=0.4pt,linecolor=darkgray]{c-c}(0,-4.18658121324868)(0,6.8389191725674925)}
\psaxes[labelFontSize=\scriptstyle,xAxis=true,yAxis=true,Dx=1,Dy=1,ticksize=-2pt 0,subticks=0]{->}(0,0)(-4.011783136876828,-4.18658121324868)(2.5,2.5)
\pscustom[linewidth=0.8pt,linecolor=red,hatchcolor=red,fillstyle=hlines,hatchangle=45,hatchsep=0.2]{\psplot{-1}{1}{-2*x^(2)+1}\lineto(1,-1)\psplot{1}{-1}{x^(4)-2*x^(2)}\lineto(-1,-1)\closepath}
\psplot[linewidth=2pt,linecolor=ttqqtt, plotpoints=200]{-4.011783136876828}{8.47768410514725}{-2*x^(2)+1}
\psplot[linewidth=2pt,linecolor=red,plotpoints=200]{-4.445046623369621}{4.361628996006329}{x^(4)-2*x^(2)}
\begin{scriptsize}
\rput[bl](0.6,0.6){\large $g$} 
\rput[bl](2.55,-0.1){\normalsize $x$} 
\rput[bl](-1.7,-1.2){\normalsize \red{$S_1$}}
\psdots[dotsize=6pt 0,dotstyle=*,linecolor=red](-1,-1)
\rput[bl](1.3,-1.2){\normalsize \red{$S_2$}}
\psdots[dotsize=6pt 0,dotstyle=*,linecolor=red](1,-1)
\rput[bl](-0.9,2.6){\normalsize \black{$F(x),g(x)$}} 
\rput[bl](1.8,1.3){\large $F$} 
\end{scriptsize}
\end{pspicture*}
\end{center}

Lösungsschlüssel:

Die Aufgabe gilt nur dann als richtig gelöst, wenn die Stammfunktion und die eingeschlossene Fläche klar erkennbar sind. 
}

\end{beispiel}