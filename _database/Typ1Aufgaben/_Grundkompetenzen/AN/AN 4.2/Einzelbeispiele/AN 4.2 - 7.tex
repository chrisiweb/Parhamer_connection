\section{AN 4.2 - 7 - MAT - Integral - MC - Matura 2015/16 - Nebentermin 1}

\begin{beispiel}[AN 4.2]{1} %PUNKTE DES BEISPIELS
Gegeben ist das bestimmte Integral

$$I=\int_{0}^{a}(25\cdot x^2+3)\,dx \text{ mit }a\in \mathbb{R}^+.$$

Kreuze die beiden Ausdrücke an, die für alle $a>0$ denselben Wert wie $I$ haben.

\multiplechoice[5]{  %Anzahl der Antwortmoeglichkeiten, Standard: 5
				L1={$$25\cdot \int_{0}^{a}\! x^2\,dx + \int_{0}^{a}\!3\,dx$$},   %1. Antwortmoeglichkeit 
				L2={$$\int_{0}^{a}\!25\,dx \cdot \int_{0}^{a}\!x^2\,dx+\int_{0}^{a}\!3\,dx$$},   %2. Antwortmoeglichkeit
				L3={$$\int_{0}^{a}\!25\cdot x^2 \,dx+3$$},   %3. Antwortmoeglichkeit
				L4={$\dfrac{25\cdot a^3}{3}+3\cdot a$},   %4. Antwortmoeglichkeit
				L5={$50\cdot a$},	 %5. Antwortmoeglichkeit
				L6={},	 %6. Antwortmoeglichkeit
				L7={},	 %7. Antwortmoeglichkeit
				L8={},	 %8. Antwortmoeglichkeit
				L9={},	 %9. Antwortmoeglichkeit
				%% LOESUNG: %%
				A1=1,  % 1. Antwort
				A2=4,	 % 2. Antwort
				A3=0,  % 3. Antwort
				A4=0,  % 4. Antwort
				A5=0,  % 5. Antwort
				}
\end{beispiel}