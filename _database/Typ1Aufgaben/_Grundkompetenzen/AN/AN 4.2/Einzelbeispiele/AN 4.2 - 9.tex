\section{AN 4.2 - 9 - MAT - Schnitt zweier Funktionen - OA - Matura 2013/14 Haupttermin}

\begin{beispiel}[AN 4.2]{1} %PUNKTE DES BEISPIELS
				Gegeben sind die beiden reellen Funktionen $f$ und $g$ mit den Gleichungen \mbox{$f(x)=x^2$} und \mbox{$g(x)=-x^2+8$}.
				
				Im nachstehenden Koordinatensystem sind die Graphen der beiden Funktionen $f$ und $g$ dargestellt. Schraffiere jene Fläche, deren Größe $A$ mit\\ 
				$A=\displaystyle\int^1_0{g(x)}\,$d$x-\displaystyle\int^1_0{f(x)}\,$d$x$ berechnet werden kann!
				
				\begin{center}\psset{xunit=0.8cm,yunit=0.4cm,algebraic=true,dimen=middle,dotstyle=o,dotsize=5pt 0,linewidth=0.8pt,arrowsize=3pt 2,arrowinset=0.25}
\begin{pspicture*}(-4.573832520992393,-9.131550337803684)(5.618573368066927,9.459644265284513)
\multips(0,-8)(0,2.0){10}{\psline[linestyle=dashed,linecap=1,dash=1.5pt 1.5pt,linewidth=0.4pt,linecolor=gray]{c-c}(-4.573832520992393,0)(5.618573368066927,0)}
\multips(-4,0)(1.0,0){11}{\psline[linestyle=dashed,linecap=1,dash=1.5pt 1.5pt,linewidth=0.4pt,linecolor=gray]{c-c}(0,-9.131550337803684)(0,9.459644265284513)}
\begin{scriptsize}
\psaxes[xAxis=true,yAxis=true,showorigin=false,Dx=1.,Dy=2.,ticksize=-2pt 0,subticks=0]{->}(0,0)(-4.573832520992393,-9.131550337803684)(5.618573368066927,9.459644265284513)[$x$,140] [\mbox{$f(x), g(x)$},-40]
\antwort{\pscustom[hatchcolor=black,fillstyle=hlines,hatchangle=45.0,hatchsep=0.3703425219738694]{\psplot{0.}{1.}{x^(2.0)}\lineto(1.,7.)\psplot{1.}{0.}{-x^(2.0)+8.0}\lineto(0.,0.)\closepath}}
\psplot[linewidth=1.2pt,plotpoints=200]{-4.573832520992393}{5.618573368066927}{x^(2.0)}
\psplot[linewidth=1.2pt,plotpoints=200]{-4.573832520992393}{5.618573368066927}{-x^(2.0)+8.0}

\rput[bl](-2.9,5.97842455873015){$f$}
\rput[bl](-3.599691073188493,-6.094741657617963){$g$}
\end{scriptsize}
\end{pspicture*}\end{center}
\end{beispiel}