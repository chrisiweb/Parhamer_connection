\section{AN 4.2 - 6 Integrationsregeln - MC - Matura 2014/15 - Nebentermin 1}

\begin{beispiel}[AN 4.2]{1} %PUNKTE DES BEISPIELS
Zwei nachstehend angeführt Gleichungen sind für alle Polynomfunktionen $f$ und bei beliebiger Wahl der Integrationsgrenzen $a$ und $b$ (mit $a<b$) richtig. \leer

Kreuze die beiden zutreffenden Gleichungen an. 

\multiplechoice[5]{  %Anzahl der Antwortmoeglichkeiten, Standard: 5
				L1={$$\int_{a}^{b}{\left(f(x)+x\right)}\,dx =\int_{a}^{b}{f(x)}\,dx+\int_{a}^{b}{x}\,dx$$},   %1. Antwortmoeglichkeit 
				L2={$$\int_{a}^{b}{f(2\cdot x)\,dx}=\frac{1}{2}\cdot \int_{a}^{b}{f(x)\,dx}$$},   %2. Antwortmoeglichkeit
				L3={$$\int_{a}^{b}{(1-f(x))\,dx}=x-\int_{a}^{b}{f(x)}\,dx$$},   %3. Antwortmoeglichkeit
				L4={$$\int_{a}^{b}{(f(x)+2)\,dx}=\int_{a}^{b}{f(x)}\,dx + 2$$},   %4. Antwortmoeglichkeit
				L5={$$\int_{a}^{b}{(3\cdot f(x))}\,dx=3\cdot \int_{a}^{b}{f(x)}\,dx$$},	 %5. Antwortmoeglichkeit
				L6={},	 %6. Antwortmoeglichkeit
				L7={},	 %7. Antwortmoeglichkeit
				L8={},	 %8. Antwortmoeglichkeit
				L9={},	 %9. Antwortmoeglichkeit
				%% LOESUNG: %%
				A1=1,  % 1. Antwort
				A2=5,	 % 2. Antwort
				A3=0,  % 3. Antwort
				A4=0,  % 4. Antwort
				A5=0,  % 5. Antwort
				}
\end{beispiel}