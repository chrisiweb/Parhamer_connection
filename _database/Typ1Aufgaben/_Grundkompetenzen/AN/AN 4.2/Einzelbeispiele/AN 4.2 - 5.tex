\section{AN 4.2 - 5 Integral einer Funktion $f$ - OA - Matura 2014/15 - Haupttermin}

\begin{beispiel}[AN 4.2]{1} %PUNKTE DES BEISPIELS
Die nachstehende Abbildung zeigt den Graphen der Polynomfunktion $f$. Alle Nullstellen sind ganzzahlig. Die Fläche, die vom Graphen der Funktion $f$ und der $x$-Achse begrenzt wird, ist schraffiert dargestellt. $A$ bezeichnet die Summe der beiden schraffierten Flächeninhalte. \leer

\begin{center}
\psset{xunit=1.0cm,yunit=1.0cm,algebraic=true,dimen=middle,dotstyle=o,dotsize=5pt 0,linewidth=0.8pt,arrowsize=3pt 2,arrowinset=0.25}
\begin{pspicture*}(-3.399092081089317,-3.552163075525553)(3.4002745941473695,3.4931755207791317)
\multips(0,-3)(0,1.0){8}{\psline[linestyle=dashed,linecap=1,dash=1.5pt 1.5pt,linewidth=0.4pt,linecolor=lightgray]{c-c}(-3.399092081089317,0)(3.4002745941473695,0)}
\multips(-3,0)(1.0,0){7}{\psline[linestyle=dashed,linecap=1,dash=1.5pt 1.5pt,linewidth=0.4pt,linecolor=lightgray]{c-c}(0,-3.552163075525553)(0,3.4931755207791317)}
\psaxes[labelFontSize=\scriptstyle,xAxis=true,yAxis=true,Dx=1.,Dy=1.,ticksize=-2pt 0,subticks=2]{->}(0,0)(-3.399092081089317,-3.552163075525553)(3.4002745941473695,3.4931755207791317)[$x$,140] [$f(x)$,-40]
\pscustom[hatchcolor=black,fillstyle=hlines,hatchangle=45.0,hatchsep=0.17569422933428136]{\psplot{-2.}{1.}{-0.5*x^(3.0)+0.5*x^(2.0)+2.0*x-2.0}\lineto(1.,0)\lineto(-2.,0)\closepath}
\pscustom[hatchcolor=black,fillstyle=hlines,hatchangle=45.0,hatchsep=0.17569422933428136]{\psplot{1.}{2.}{-0.5*x^(3.0)+0.5*x^(2.0)+2.0*x-2.0}\lineto(2.,0)\lineto(1.,0)\closepath}
\psplot[plotpoints=200]{-3.399092081089317}{3.4002745941473695}{-0.5*x^(3.0)+0.5*x^(2.0)+2.0*x-2.0}
\rput[tl](-2.678745740818764,2.684982065841437){$f$}
\end{pspicture*}
\end{center} \leer

Gib einen korrekten Ausdruck für $A$ mithilfe der Integralschreibweise an.\leer

$A= \rule{6cm}{0.3pt}$

\antwort{
$$A=\int_1^2{f(x)\,\mathrm{d}x} - \int_{-2}^1{f(x)\,\mathrm{d}x}$$ \leer

oder: \\

$$A=\int_{-2}^2\left|f(x)\right|\,\mathrm{d}x $$
}


\end{beispiel}