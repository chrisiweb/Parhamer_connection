\section{AN 4.2 - 1 Unbestimmtes Integral - MC - BIFIE}

\begin{beispiel}[AN 4.2]{1} %PUNKTE DES BEISPIELS
				Gegeben sind Aussagen �ber die L�sung eines unbestimmten Integrals. Nur eine Rechnung ist richtig. Die Integrationskonstante wird in allen F�llen mit $c=0$ angenommen.
				
				Kreuze die korrekte Rechnung an!
				\multiplechoice[6]{  %Anzahl der Antwortmoeglichkeiten, Standard: 5
								L1={$$\int{3\cdot(2x+5)dx=(6x+5)�}$$},   %1. Antwortmoeglichkeit 
								L2={$$\int{3\cdot(2x+5)dx=3x�+5x}$$},   %2. Antwortmoeglichkeit
								L3={$$\int{3\cdot(2x+5)dx=(6x+15)�}$$},   %3. Antwortmoeglichkeit
								L4={$$\int{3\cdot(2x+5)dx=3\cdot(x�+5x)}$$},   %4. Antwortmoeglichkeit
								L5={$$\int{3\cdot(2x+5)dx=3x�+15}$$},	 %5. Antwortmoeglichkeit
								L6={$$\int{3\cdot(2x+5)dx=6x�+15x}$$},	 %6. Antwortmoeglichkeit
								L7={},	 %7. Antwortmoeglichkeit
								L8={},	 %8. Antwortmoeglichkeit
								L9={},	 %9. Antwortmoeglichkeit
								%% LOESUNG: %%
								A1=4,  % 1. Antwort
								A2=0,	 % 2. Antwort
								A3=0,  % 3. Antwort
								A4=0,  % 4. Antwort
								A5=0,  % 5. Antwort
								}
\end{beispiel}