\section{AN 3.1 - 12 - Zusammenhang zwischen drei Funktionen - MC - MarNeu UNIVIE}

\begin{beispiel}[AN 3.1]{1}
Gegeben sind die Funktionen $f$, $g$ und $h$.\\
Zwischen diesen Funktionen bestehen folgende Zusammenhänge:
\begin{itemize}
\item $f \text{ ist eine Stammfunktion von } h$ 
\item $g''=h$ 
\end{itemize}
\vspace{0,5 cm}
Kreuze die beiden jedenfalls zutreffenden Aussagen an! \\
\multiplechoice[5]{  %Anzahl der Antwortmoeglichkeiten, Standard: 5
				L1={Es gibt eine Konstante $C \in \mathbb{R}$, sodass $f+C$ die Ableitungsfunktion der Funktion $g$ ist.},   %1. Antwortmoeglichkeit 
				L2={Die Funktion $f$ ist eine Stammfunktion der Funktion $g$.},   %2. Antwortmoeglichkeit
				L3={Es gibt eine Konstante $C \in \mathbb{R}$, sodass $h+C$ die Ableitungsfunktion der Funktion $f$ ist.},   %3. Antwortmoeglichkeit
				L4={Die Funktion $g$ ist eine Stammfunktion der Funktion $h$.},   %4. Antwortmoeglichkeit
				L5={Es gibt eine Konstante $C \in \mathbb{R}$, sodass $f+C$ die Ableitungsfunktion der Funktion $h$ ist.},	 %5. Antwortmoeglichkeit
				L6={},	 %6. Antwortmoeglichkeit
				L7={},	 %7. Antwortmoeglichkeit
				L8={},	 %8. Antwortmoeglichkeit
				L9={},	 %9. Antwortmoeglichkeit
				%% LOESUNG: %%
				A1=1,  % 1. Antwort
				A2=3,	 % 2. Antwort
				A3=0,  % 3. Antwort
				A4=0,  % 4. Antwort
				A5=0,  % 5. Antwort
				}
\end{beispiel}