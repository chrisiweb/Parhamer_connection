\section{AN 3.1 - 2 Stammfunktion - LT - BIFIE}

\begin{beispiel}[AN 3.1]{1} %PUNKTE DES BEISPIELS
Es gilt die Aussage:
"`Besitzt eine Funktion $f$ eine Stammfunktion, so besitzt sie sogar unendlich viele. Ist n�mlich $F$ eine Stammfunktion von $f$, so ist f�r jede beliebige reelle Zahl $c$ auch die durch $G(x) = F(x) + c$ definierte Funktion $G$ eine Stammfunktion von $f$."'
\begin{flushright}
\tiny{Quelle: Wikipedia}
\end{flushright}

\lueckentext{
				text={Ist die Funktion $F$ eine Stammfunktion $f$, dann gilt \gap. Gilt zudem \gap, dann ist auch die Funktion G eine Stammfunktion von $f$.}, 	%Lueckentext Luecke=\gap
				L1={$F(x)=f(x)$}, 		%1.Moeglichkeit links  
				L2={$F(x)=f'(x)$}, 		%2.Moeglichkeit links
				L3={$F'(x)=f(x)$}, 		%3.Moeglichkeit links
				R1={$G'(x)=F'(x)=f(x)$}, 		%1.Moeglichkeit rechts 
				R2={$G(x)=F(x)=f'(x)$}, 		%2.Moeglichkeit rechts
				R3={$G'(x)=F(x)=f'(x)$}, 		%3.Moeglichkeit rechts
				%% LOESUNG: %%
				A1=3,   % Antwort links
				A2=1		% Antwort rechts 
				}

 \end{beispiel}