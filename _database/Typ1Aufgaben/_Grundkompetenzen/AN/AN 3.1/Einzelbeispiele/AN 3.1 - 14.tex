\section{AN 3.1 - 14 - MAT - Wachstum einer Pflanze - OA - Matura 2019/20 1. HT}

\begin{beispiel}[AN 3.1]{1}
Zu Beginn eines dreiwöchigen Beobachtungszeitraums ist eine bestimmte Pflanze 15\,cm hoch.\\
Die momentane Änderungsrate der Höhe dieser Pflanze wird durch die Funktion $v$ in Abhängigkeit von der Zeit $t$ beschrieben.\\
Dabei gilt:\\
$v(t)=3-0,3\cdot t^2$ mit $t\in[0;3]$ die Höhe $h(t)$ der Pflanze zu ($t$ in Wochen, $h(t)$ in cm).

Gib $h(t)$ an.\leer

$h(t)=\,\antwort[\rule{5cm}{0.3pt}]{-0,1\cdot t^3+3\cdot t+15}$
\end{beispiel}