\section{AN 3.1 - 10 - Besonderheiten von Kreis und Kugel - MC - BarTri UNIVIE}

\begin{beispiel}[AN 3.1]{1}
Die Funktion des Umfangs $U\!:r \rightarrow U(r)$ und die Funktion der Fläche\\ 
$A\!:r \rightarrow A(r)$ eines Kreises mit Radius $r \in \mathbb{R}^+$ und die Funktion der Oberfläche\\ $O\!:r \rightarrow O(r)$ und die Funktion des Volumens $V\!:r \rightarrow V(r)$ einer Kugel mit Radius $r \in \mathbb{R}^+$ stehen in Beziehung zueinander.

Kreuze die beiden richtigen Aussagen an.

\multiplechoice[5]{  %Anzahl der Antwortmoeglichkeiten, Standard: 5
				L1={Die Ableitungsfunktion $A'$ des Flächeninhalts eines Kreises, beschreibt die Funktion $U$ des Kreisumfangs.},   %1. Antwortmoeglichkeit 
				L2={Für den Flächeninhalt $A(r)=r^2 \pi$ und den Umfang $U(r)=2r \pi$ gilt folgende Beziehung: $U'(r)=A(r) \quad \text{für} \quad r \in \mathbb{R}^+$.},   %2. Antwortmoeglichkeit
				L3={Eine Stammfunktion $V$ der Kugeloberfläche $O(r)=4 \pi r^2$ ist gegeben durch $V(r)= \frac{4}{3} r^3 \pi \quad \text{für} \quad r \in \mathbb{R}^+$. },   %3. Antwortmoeglichkeit
				L4={Die Funktion $O$, die zur Beschreibung der Kugeloberfläche dient, ist eine Stammfunktion von $V$.},   %4. Antwortmoeglichkeit
				L5={Es gilt: $A'(r)=U(r)$ und  $V(r)=O'(r)$ für $r \in \mathbb{R}^+ $.},	 %5. Antwortmoeglichkeit
				L6={},	 %6. Antwortmoeglichkeit
				L7={},	 %7. Antwortmoeglichkeit
				L8={},	 %8. Antwortmoeglichkeit
				L9={},	 %9. Antwortmoeglichkeit
				%% LOESUNG: %%
				A1=1,  % 1. Antwort
				A2=3,	 % 2. Antwort
				A3=0,  % 3. Antwort
				A4=0,  % 4. Antwort
				A5=0,  % 5. Antwort
				}
\end{beispiel}