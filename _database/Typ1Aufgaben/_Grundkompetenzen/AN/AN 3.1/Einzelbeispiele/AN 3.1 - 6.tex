\section{AN 3.1 - 6 - MAT - Eigenschaften von Stammfunktionen - MC - Matura 1. NT 2017/18}

\begin{beispiel}[AN 3.1]{1}
In der nachstehenden Abbildung ist der Graph einer linearen Funktion $g$ dargestellt.

\begin{center}
	\resizebox{0.5\linewidth}{!}{\psset{xunit=1.0cm,yunit=1.0cm,algebraic=true,dimen=middle,dotstyle=o,dotsize=5pt 0,linewidth=1.6pt,arrowsize=3pt 2,arrowinset=0.25}
\begin{pspicture*}(-6.84,-6.48)(6.78,6.7)
\multips(0,-6)(0,1.0){14}{\psline[linestyle=dashed,linecap=1,dash=1.5pt 1.5pt,linewidth=0.4pt,linecolor=darkgray]{c-c}(-6.84,0)(6.78,0)}
\multips(-6,0)(1.0,0){14}{\psline[linestyle=dashed,linecap=1,dash=1.5pt 1.5pt,linewidth=0.4pt,linecolor=darkgray]{c-c}(0,-6.48)(0,6.7)}
\psaxes[labelFontSize=\scriptstyle,xAxis=true,yAxis=true,Dx=1.,Dy=1.,ticksize=-2pt 0,subticks=2]{->}(0,0)(-6.84,-6.48)(6.78,6.7)[x,140] [g(x),-40]
\psplot[linewidth=2.pt]{-6.84}{6.78}{(--2.--1.*x)/-2.}
\begin{scriptsize}
\rput[bl](-5.8,2.16){$g$}
\end{scriptsize}
\end{pspicture*}}
\end{center}

Kreuze die beiden für die Funktion $g$ zutreffenden Aussagen an!

\multiplechoice[5]{  %Anzahl der Antwortmoeglichkeiten, Standard: 5
				L1={Jede Stammfunktion von $g$ ist eine Polynomfunktion zweiten Grades.},   %1. Antwortmoeglichkeit 
				L2={Jede Stammfunktion von $g$ hat an der Stelle $x=-2$ ein lokales Minimum.},   %2. Antwortmoeglichkeit
				L3={Jede Stammfunktion von $g$ ist im Intervall $(0;2)$ streng monoton fallend.},   %3. Antwortmoeglichkeit
				L4={Die Funktion $G$ mit $G(x)=-0,5$ ist eine Stammfunktion von $g$.},   %4. Antwortmoeglichkeit
				L5={Jede Stammfunktion von $g$ hat mindestens eine Nullstelle.},	 %5. Antwortmoeglichkeit
				L6={},	 %6. Antwortmoeglichkeit
				L7={},	 %7. Antwortmoeglichkeit
				L8={},	 %8. Antwortmoeglichkeit
				L9={},	 %9. Antwortmoeglichkeit
				%% LOESUNG: %%
				A1=1,  % 1. Antwort
				A2=3,	 % 2. Antwort
				A3=0,  % 3. Antwort
				A4=0,  % 4. Antwort
				A5=0,  % 5. Antwort
				}
\end{beispiel}