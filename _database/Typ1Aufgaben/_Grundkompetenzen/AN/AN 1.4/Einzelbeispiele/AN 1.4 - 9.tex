\section{AN 1.4 - 9 - MAT - Kapitalwachstum - OA - Matura-HT-18/19}

\begin{beispiel}[AN 1.4]{1}
Ein Kapital von \EUR{100.000} wird mit einem fixen jährlichen Zinssatz angelegt. Die nachstehende Tabelle gibt Auskunft über den Verlauf des Kapitals in den ersten drei Jahren. Dabei beschreibt $x_n$ das Kapital nach $n$ Jahren $(n\in\mathbb{N})$.
\begin{center}
\begin{tabular}{|c|c|}\hline
\cellcolor[gray]{0.9}$n$ in Jahren&\cellcolor[gray]{0.9}$x_n$ in Euro\\ \hline
0&100\,000\\ \hline
1&103\,000\\ \hline
2&106\,090\\ \hline
3&109\,272,7\\ \hline
\end{tabular}
\end{center}

Stelle eine Gleichung zur Bestimmung des Kapitals $x_{n+1}$ aus dem Kapital $x_n$ auf!\leer

$x_{n+1}=$\,\antwort[\rule{3cm}{0.3pt}]{$1,03\cdot x_n$}
\end{beispiel}