\section{AN 1.4 - 2 Wirkstoffe im K�rper - LT - BIFIE}


\begin{beispiel}[AN 1.4]{1} %PUNKTE DES BEISPIELS
Ein Patient, der an Bluthochdruck leidet, muss auf �rztliche Empfehlung ab sofort t�glich am Morgen eine Tablette mit Wirkstoffgehalt 100\,mg zur Therapie einnehmen. Der K�rper scheidet im Laufe eines Tages 80\,\% des Wirkstoffs wieder aus. Die Wirkstoffmenge $W_n$ im K�rper des Patienten nach $n$ Tagen kann daher (rekursiv) aus der Menge des Vortags $W_{n-1}$ nach folgender Beziehung bestimmt werden: 
\vspace{-0.3cm}
\[W_n = 0,2 \cdot W_{n-1}+100, ~ W_0=100 ~ \text{($W_i$ in mg)}\]

In welcher Weise wird sich die Wirkstoffmenge im K�rper des Patienten langfristig entwickeln?

\lueckentext[-0.09]{
				text={Die Wirkstoffmenge im K�rper des Patienten wird langfristig \gap, weil \gap.}, 	%Lueckentext Luecke=\gap
				L1={unbeschr�nkt wachsen}, 		%1.Moeglichkeit links  
				L2={beschr�nkt wachsen}, 		%2.Moeglichkeit links
				L3={wieder sinken}, 		%3.Moeglichkeit links
				R1={der K�rper des Patienten mit steigendem Wirkstoffgehalt im K�rper absolut immer mehr abbaut und damit der Abbau letztlich die Zufuhr �bersteigt}, 		%1.Moeglichkeit rechts 
				R2={dem K�rper t�glich zus�tzlicher Wirkstoff zugef�hrt wird, dernur zu 80\,\% abgebaut werden kann, und somit die Zufuhr im Vergleich zum Abbau �berwiegt}, 		%2.Moeglichkeit rechts
				R3={der K�rper des Patienten mit steigendem Wirkstoffgehalt im K�rper absolut immer mehr davon abbaut, auch wenn der Prozentsatz gleich bleibt}, 		%3.Moeglichkeit rechts
				%% LOESUNG: %%
				A1=2,   % Antwort links
				A2=3		% Antwort rechts 
				}

\end{beispiel}