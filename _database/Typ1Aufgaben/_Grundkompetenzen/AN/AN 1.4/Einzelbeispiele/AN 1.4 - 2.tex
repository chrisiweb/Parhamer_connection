\section{AN 1.4 - 2 - Wirkstoffe im Körper - LT - BIFIE}


\begin{beispiel}[AN 1.4]{1} %PUNKTE DES BEISPIELS
Ein Patient, der an Bluthochdruck leidet, muss auf ärztliche Empfehlung ab sofort täglich am Morgen eine Tablette mit Wirkstoffgehalt 100\,mg zur Therapie einnehmen. Der Körper scheidet im Laufe eines Tages 80\,\% des Wirkstoffs wieder aus. Die Wirkstoffmenge $W_n$ im Körper des Patienten nach $n$ Tagen kann daher (rekursiv) aus der Menge des Vortags $W_{n-1}$ nach folgender Beziehung bestimmt werden: 
\vspace{-0.3cm}
\[W_n = 0,2 \cdot W_{n-1}+100, ~ W_0=100 ~ \text{($W_i$ in mg)}\]

In welcher Weise wird sich die Wirkstoffmenge im Körper des Patienten langfristig entwickeln?

\lueckentext[-0.09]{
				text={Die Wirkstoffmenge im Körper des Patienten wird langfristig \gap, weil \gap.}, 	%Lueckentext Luecke=\gap
				L1={unbeschränkt wachsen}, 		%1.Moeglichkeit links  
				L2={beschränkt wachsen}, 		%2.Moeglichkeit links
				L3={wieder sinken}, 		%3.Moeglichkeit links
				R1={der Körper des Patienten mit steigendem Wirkstoffgehalt im Körper absolut immer mehr abbaut und damit der Abbau letztlich die Zufuhr übersteigt}, 		%1.Moeglichkeit rechts 
				R2={dem Körper täglich zusätzlicher Wirkstoff zugeführt wird, dernur zu 80\,\% abgebaut werden kann, und somit die Zufuhr im Vergleich zum Abbau überwiegt}, 		%2.Moeglichkeit rechts
				R3={der Körper des Patienten mit steigendem Wirkstoffgehalt im Körper absolut immer mehr davon abbaut, auch wenn der Prozentsatz gleich bleibt}, 		%3.Moeglichkeit rechts
				%% LOESUNG: %%
				A1=2,   % Antwort links
				A2=3		% Antwort rechts 
				}

\end{beispiel}