\section{AN 1.4 - 4 Wirkstoff - MC - BIFIE}

\begin{beispiel}[AN 1.4]{1} %PUNKTE DES BEISPIELS
Eine Person beginnt mit der Einnahme eines Medikaments und wiederholt die Einnahme alle 24 Stunden. Sie f�hrt dem K�rper dabei jeweils 125\,$\mu$g eines Wirkstoffs zu. Innerhalb eines Tages werden jeweils 70\,\% der im K�rper vorhandenen Menge des Wirkstoffs abgebaut. 
\leer

Die Wirkstoffmenge $x_n$ (in $\mu$g) gibt die vorhandene Menge des Wirkstoffs im K�rper dieser Person nach $n$ Tagen unmittelbar nach Einnahme des Wirkstoffs an und kann modellhaft durch eine Differenzengleichung beschrieben werden.\\
Kreuze die entsprechende Gleichung an. 

\multiplechoice[6]{  %Anzahl der Antwortmoeglichkeiten, Standard: 5
				L1={$x_{n+1}=x_n+125)\cdot 0,3$},   %1. Antwortmoeglichkeit 
				L2={$x_{n+1}=0,3\cdot x_n +125$},   %2. Antwortmoeglichkeit
				L3={$x_{n+1}=1,3 \cdot x_n -125$},   %3. Antwortmoeglichkeit
				L4={$x_{n+1}=x_n +125 \cdot 0,7$},   %4. Antwortmoeglichkeit
				L5={$x_{n+1}=(x_n-125)\cdot 0,7$},	 %5. Antwortmoeglichkeit
				L6={$x_{n+1}=(x_n-0,3)\cdot 125$},	 %6. Antwortmoeglichkeit
				L7={},	 %7. Antwortmoeglichkeit
				L8={},	 %8. Antwortmoeglichkeit
				L9={},	 %9. Antwortmoeglichkeit
				%% LOESUNG: %%
				A1=2,  % 1. Antwort
				A2=0,	 % 2. Antwort
				A3=0,  % 3. Antwort
				A4=0,  % 4. Antwort
				A5=0,  % 5. Antwort
				}
\end{beispiel}