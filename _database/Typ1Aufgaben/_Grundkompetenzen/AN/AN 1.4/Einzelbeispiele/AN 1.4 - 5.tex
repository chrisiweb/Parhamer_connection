\section{AN 1.4 - 5 Holzbestand - OA - Matura 2014/15 - Kompensationsprüfung}

\begin{beispiel}[AN 1.4]{1} %PUNKTE DES BEISPIELS
				Der Holzbestand eines Waldes wird in Kubikmetern ($m³$) angegeben. Zu Beginn eines bestimmten Jahres beträgt der Holzbestand $10\,000\,m³$. Jedes Jahr wächst der Holzbestand um $3\,\%$. Am Jahresende werden jeweils $500\,m³$ Holz geschlägert. Dabei gibt $a_{n}$ die Holzmenge am Ende des n-ten Jahres an.
				
				Stelle die Entwicklung des Holzbestandes durch eine Differenzengleichung dar.\\
				
				\antwort{$a_{0}=10\,000$\\
				$a_{n+1}=1,03\cdot a_{n}-500$\\
				
				$a_{0}$ ... Holzbestand zu Beginn\\
				$n$ ... Jahre nach Beginn\\
				$a_{n+1}$ ... Holzbestand am Ende des $(n+1)$-ten Jahres}
\end{beispiel}