\section{AN 1.4 - 10 - MAT - Konzentration eines Arzneistoffs - OA - Matura 2018/19 2. NT}

\begin{beispiel}[AN 1.4]{1}
Eine Patientin wird täglich um 8:00 Uhr ein Arzneistoff intravenös verabreicht. Die Konzentration des Arzneistoffs im Blut der Patientin am Tag $t$ unmittelbar vor der Verabreichung des Arzneistoffs wird mit $c_t$ bezeichnet ($c_t$ in Milligramm/Liter).

Für $t\in\mathbb{R}$ gilt: $c_{t+1}=0,3\cdot(c_t+4)$

Interpretiere den in der Gleichung auftretenden Zahlenwert 4 im gegebenen Kontext unter Verwendung der entsprechenden Einheit.

\antwort{mögliche Interpretation:

Durch die Verabreichung des Arzneistoffs erhöht sich dessen Konzentration im Blut der Patientin um 4\,mg/L.}
\end{beispiel}