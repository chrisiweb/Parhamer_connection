\section{AN 1.4 - 8 Kredittilgung - OA - Matura 17/18}

\begin{beispiel}[AN 1.4]{1} %PUNKTE DES BEISPIELS
Jemand hat bei einer Bank einen Wohnbaukredit zur Finanzierung einer Eigentumswohnung aufgenommen. Am Ende eines jeden Monats erh�ht sich der Schuldenstand aufgrund der Kreditzinsen um $0,4\,\%$ und anschlie�end wird die monatliche Rate von \EUR{450} zur�ckgezahlt.

Der Schuldenstand am Ende von $t$ Monaten wird durch $S(t)$ beschrieben.

Geben Sie eine Differenzengleichung an, mit deren Hilfe man bei Kenntnis des Schuldenstands am Ende eines Monats den Schuldenstand am Ende des darauffolgenden Monats berechnen kann!

\antwort{m�gliche Differenzengleichung: $S(t+1)-S(t)=S(t)\cdot 0,004-450$}
\end{beispiel}