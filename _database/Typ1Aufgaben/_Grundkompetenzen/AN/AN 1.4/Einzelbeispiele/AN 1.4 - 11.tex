\section{AN 1.4 - 11 - MAT - Population - OA - Matura 2019/20 1. HT}

\begin{beispiel}[AN 1.4]{1}
Die Anzahl der Rehe in einem Wald am Ende eines Jahres $i$ ($i=1,2,3$) wird mit $R_i$ bezeichnet.\\
Am Ende des ersten Jahres gibt es 60 Rehe in diesem Wald.

Die nachstehende Gleichung beschreibt die Entwicklung der Population der Rehe.

$R_{i+1}=1,2\cdot R_i-2$ für $i=1,2$

Bestimme die Anzahl der Rehe in diesem Wald am Ende des dritten Jahres.

Die Anzahl der Rehe am Ende des dritten Jahres beträgt\,\,\antwort[\rule{3cm}{0.3pt}]{82}.

\antwort{mögliche Vorgehensweise:

$R_1=60$\\
$R_2=1,2\cdot 60-2=70$\\
$R_3=1,2\cdot 70-2=82$}
\end{beispiel}