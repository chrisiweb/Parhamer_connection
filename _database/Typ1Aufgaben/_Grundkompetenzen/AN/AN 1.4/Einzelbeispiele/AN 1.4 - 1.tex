\section{AN 1.4 - 1 Wachstum - MC - BIFIE}


\begin{beispiel}[AN 1.4]{1} %PUNKTE DES BEISPIELS
Wachstum tritt in der Natur fast nie unbegrenzt auf, es erreicht einmal eine gewisse Grenze (Sättigung). Diese Sättigungsgrenze sei $K$. Der vorhandene Bestand zum Zeitpunkt $n$ sei $xn$.

Zur Beschreibung vieler Vorgänge (Wachstum von Populationen, Ausbreitung von Krankheiten
oder Informationen, Erwärmung etc.) verwendet man folgendes mathematisches Modell:

\[x_{n+1} - x_n = r \cdot (K - x_n)~ \text {mit } r \in \mathbb{R}^+,~ 0 < r < 1 \text{ ($r$ ist ein Proportionalitätsfaktor)} \]

Kreuze die auf dieses Modell zutreffende(n) Aussage(n) an.

\multiplechoice[5]{  %Anzahl der Antwortmoeglichkeiten, Standard: 5
				L1={Diese Gleichung kann als eine lineare Differenzengleichung der Form $x_{n+1} = a \cdot x_n +b$ gedeutet werden.},   %1. Antwortmoeglichkeit 
				L2={Der Zuwachs pro Zeiteinheit ist proportional zum momentanen Bestand.
},   %2. Antwortmoeglichkeit
				L3={Es liegt ein kontinuierliches Wachstumsmodell vor, d.h., man kann zu
jedem beliebigen Zeitpunkt die Größe des Bestands errechnen.},   %3. Antwortmoeglichkeit
				L4={Der Zuwachs bei diesem Wachstum ist proportional zur noch verfügbaren
Restkapazität (= Freiraum).},   %4. Antwortmoeglichkeit
				L5={Mit zunehmender Zeit wird der Zuwachs immer geringer.},	 %5. Antwortmoeglichkeit
				L6={},	 %6. Antwortmoeglichkeit
				L7={},	 %7. Antwortmoeglichkeit
				L8={},	 %8. Antwortmoeglichkeit
				L9={},	 %9. Antwortmoeglichkeit
				%% LOESUNG: %%
				A1=1,  % 1. Antwort
				A2=4,	 % 2. Antwort
				A3=5,  % 3. Antwort
				A4=0,  % 4. Antwort
				A5=0,  % 5. Antwort
				}
\end{beispiel}