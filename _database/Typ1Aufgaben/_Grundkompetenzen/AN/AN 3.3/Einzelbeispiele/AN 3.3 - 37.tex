\section{AN 3.3 - 37 - MAT - Eigenschaften einer Polynomfunktion dritten Grades - MC - Matura 1.NT 2018/19}

\begin{beispiel}[AN 3.3]{1}
Gegeben ist eine Polynomfunktion $f$ dritten Grades. An den beiden Stellen $x_1$ und $x_2$ mit $x_1<x_2$ gelten folgende Bedingungen:\leer

$f'(x_1)=0$ und $f''(x_1)<0$\\
$f'(x_2)=0$ und $f''(x_2)>0$\leer

Kreuze die beiden Aussagen an, die für die Funktion $f$ auf jeden Fall zutreffen.

\multiplechoice[5]{  %Anzahl der Antwortmoeglichkeiten, Standard: 5
				L1={$f(x_1)>f(x_2)$},   %1. Antwortmoeglichkeit 
				L2={Es gibt eine weitere Stelle $x_3$ mit $f'(x_3)=0$.},   %2. Antwortmoeglichkeit
				L3={Im Intervall $[x_1;x_2]$ gibt es eine Stelle $x_3$ mit $f(x_3)>f(x_1)$.},   %3. Antwortmoeglichkeit
				L4={Im Intervall $[x_1;x_2]$ gibt es eine Stelle $x_3$ mit $f''(x_3)=0$.},   %4. Antwortmoeglichkeit
				L5={Im Intervall $[x_1;x_2]$ gibt es eine Stelle $x_3$ mit $f'(x_3)>0$.},	 %5. Antwortmoeglichkeit
				L6={},	 %6. Antwortmoeglichkeit
				L7={},	 %7. Antwortmoeglichkeit
				L8={},	 %8. Antwortmoeglichkeit
				L9={},	 %9. Antwortmoeglichkeit
				%% LOESUNG: %%
				A1=1,  % 1. Antwort
				A2=4,	 % 2. Antwort
				A3=0,  % 3. Antwort
				A4=0,  % 4. Antwort
				A5=0,  % 5. Antwort
				}
\end{beispiel}