\section{AN 3.3 - 28 Extremstelle - MC - Matura 2013/14 1. Nebentermin}

\begin{beispiel}[AN 3.3]{1} %PUNKTE DES BEISPIELS
				Die Ermittlung lokaler Extremstellen einer Polynomfunktion $f$ erfolgt häufig mithilfe der Differenzialrechnung.
				
				Kreuze die beiden Aussagen an, die stets zutreffend sind!\leer
				
				\multiplechoice[5]{  %Anzahl der Antwortmoeglichkeiten, Standard: 5
								L1={Wenn $x_0$ eine lokale Extremstelle von $f$ ist, dann wechselt die Funktion an der Stelle $x_0$ das Krümmungsverhalten.},   %1. Antwortmoeglichkeit 
								L2={Wenn $x_0$ eine lokale Extremstelle von $f$ ist, dann ist $f''(x_0)=0$.},   %2. Antwortmoeglichkeit
								L3={Wenn die Funktion $f$ bei $x_0$ das Monotonieverhalten ändert, dann liegt bei $x_0$ eine lokale Extremstelle von $f$.},   %3. Antwortmoeglichkeit
								L4={Wenn $x_0$ eine lokale Extremstelle von $f$ ist, dann ist $f'(x_0)=0$.},   %4. Antwortmoeglichkeit
								L5={Wenn $x_0$ eine lokale Extremstelle von $f$ ist, dann ist $f'(x)$ für $x<x_0$ immer negativ und für $x>x_0$ immer positiv.},	 %5. Antwortmoeglichkeit
								L6={},	 %6. Antwortmoeglichkeit
								L7={},	 %7. Antwortmoeglichkeit
								L8={},	 %8. Antwortmoeglichkeit
								L9={},	 %9. Antwortmoeglichkeit
								%% LOESUNG: %%
								A1=3,  % 1. Antwort
								A2=4,	 % 2. Antwort
								A3=0,  % 3. Antwort
								A4=0,  % 4. Antwort
								A5=0,  % 5. Antwort
								}
\end{beispiel}