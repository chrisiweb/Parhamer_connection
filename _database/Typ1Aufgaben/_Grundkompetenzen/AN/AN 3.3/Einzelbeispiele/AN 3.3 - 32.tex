\section{AN 3.3 - 32 - MAT - Wendestelle - OA - Matura 2016/17 2. NT}

\begin{beispiel}{1} %PUNKTE DES BEISPIELS
Eine Polynomfunktion dritten Grades $f$ hat die Ableitungsfunktion $f'$ mit $f'(x)=12\cdot x^2 - 4 \cdot x - 8$. \leer

Gib an, ob die Funktion $f$ an der Stelle $x=6$ eine Wendestelle hat, und begründe deine Entscheidung.

\antwort{Die Funktion $f$ hat an der Stelle $x=6$ keine Wendestelle.

$f''(x)=24\cdot x -4$ 

$f''(6)=140 \neq 0 \Rightarrow$ Die Funktion $f$ kann an der Stelle $x=6$ keine Wendestelle haben.}

\end{beispiel}