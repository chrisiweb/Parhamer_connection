\section{AN 3.3 - 19 Wachstumsgeschwindigkeit - LT - BIFIE}

\begin{beispiel}[AN 3.3]{1} %PUNKTE DES BEISPIELS
				Das Wachstum einer Bakterienkultur wird durch eine Funktion $N$ beschrieben. Dabei gibt $N(t)$ die Anzahl der Bakterien zum Zeitpunkt $t$ ($t$ in Stunden) an.
	
	\lueckentext[0.1]{
								text={Wenn \gap positiv sind, erfolgt das Bakterienwachstum im Intervall [a;b] \gap.}, 	%Lueckentext Luecke=\gap
								L1={die Funktionswerte $N(t)$ für $t\in[a;b]$}, 		%1.Moeglichkeit links  
								L2={die Funktionswerte $N'(t)$ für $t\in[a;b]$}, 		%2.Moeglichkeit links
								L3={die Funktionswerte $N''(t)$ für $t\in[a;b]$}, 		%3.Moeglichkeit links
								R1={immer schneller}, 		%1.Moeglichkeit rechts 
								R2={immer langsamer}, 		%2.Moeglichkeit rechts
								R3={gleich schnell}, 		%3.Moeglichkeit rechts
								%% LOESUNG: %%
								A1=3,   % Antwort links
								A2=1		% Antwort rechts 
								}
\end{beispiel}