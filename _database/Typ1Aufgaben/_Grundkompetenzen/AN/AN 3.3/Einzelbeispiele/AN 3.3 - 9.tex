\section{AN 3.3 - 9 Wendestelle - MC - BIFIE}

\begin{beispiel}[AN 3.3]{1} %PUNKTE DES BEISPIELS
				Ein Becken wird mit Wasser gefüllt. Die in das Becken zufließende Wassermenge, angegeben in $m³$ pro Stunde, kann im Intervall $[0;8)$ durch die Funktion $f$ beschrieben werden. Die Funktion $f$ hat an der Stelle $t=4$ eine Wendestelle.
				\begin{center}
				\resizebox{0.8\linewidth}{!}{\psset{xunit=1.0cm,yunit=1.0cm,algebraic=true,dimen=middle,dotstyle=o,dotsize=5pt 0,linewidth=0.8pt,arrowsize=3pt 2,arrowinset=0.25}
\begin{pspicture*}(-0.5779223140495889,-0.7518020448155451)(8.982408264462818,5.976131839481978)
\psaxes[labelFontSize=\scriptstyle,xAxis=true,yAxis=true,Dx=1.,Dy=1.,ticksize=-2pt 0,subticks=2]{->}(0,0)(0.,0.)(8.982408264462818,5.976131839481978)[t in h,140] [$f(t)$ in $m³/h$,-40]
\psplot[linewidth=1.2pt,plotpoints=200]{0}{8}{0.014970220434915651*x^(3.0)-0.17913968982498327*x^(2.0)-0.02502066476408853*x+5.017520661157025}
\end{pspicture*}}\end{center}
\leer

Kreuze die für die Funktion $f$ zutreffende(n) Aussage(n) an!
\multiplechoice[5]{  %Anzahl der Antwortmoeglichkeiten, Standard: 5
				L1={An der Stelle $t=4$ geht die Linkskrümmung $(f''(t)>0)$ in eine Rechtskrümmung $(f''(t)<0)$ über.},   %1. Antwortmoeglichkeit 
				L2={An der Stelle $t=4$ geht die Rechtskrümmung $(f''(t)<0)$ in eine Linkskrümmung $(f''(t)>0)$ über.},   %2. Antwortmoeglichkeit
				L3={Der Wert der zweiten Ableitung der Funktion $f$ an der Stelle 4 ist null.},   %3. Antwortmoeglichkeit
				L4={Es gilt $f''(t)>0$ für $t>4$.},   %4. Antwortmoeglichkeit
				L5={Für $t>4$ sinkt die pro Stunde zufließende Wassermenge.},	 %5. Antwortmoeglichkeit
				L6={},	 %6. Antwortmoeglichkeit
				L7={},	 %7. Antwortmoeglichkeit
				L8={},	 %8. Antwortmoeglichkeit
				L9={},	 %9. Antwortmoeglichkeit
				%% LOESUNG: %%
				A1=2,  % 1. Antwort
				A2=3,	 % 2. Antwort
				A3=4,  % 3. Antwort
				A4=5,  % 4. Antwort
				A5=0,  % 5. Antwort
				}
\end{beispiel}