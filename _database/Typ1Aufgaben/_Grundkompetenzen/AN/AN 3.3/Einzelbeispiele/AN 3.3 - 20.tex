\section{AN 3.3 - 20 - Ableitungsfunktionen - MC - BIFIE}

\begin{beispiel}[AN 3.3]{1} %PUNKTE DES BEISPIELS
Die nachstehenden Abbildungen zeigen die Graphen von drei Funktionen $f_1$, $f_2$, $f_3$ im Intervall $[0; 160]$.

\meinlr{\psset{xunit=0.04cm,yunit=0.05cm,algebraic=true,dimen=middle,dotstyle=o,dotsize=5pt 0,linewidth=0.8pt,arrowsize=3pt 2,arrowinset=0.25}
\begin{pspicture*}(-8.721096172049721,-13.67475458356514)(170,95.16587686451683)
\begin{scriptsize}
\psaxes[xAxis=true,yAxis=true,labels=none,Dx=50.,Dy=10.,ticks=x,ticksize=-2pt 0,subticks=0]{->}(0,0)(-8.721096172049721,-13.67475458356514)(170,95.16587686451683)[$x$,140] [$f_1(x)$,-40]
\psplot[linewidth=1.2pt,plotpoints=200] {0}{160}{15.0*ln(x+20.0)-30.0}
\rput[tl](48,-3){$\text{50}$}
\rput[tl](95,-3){$\text{100}$}
\rput[tl](155,-3){$\text{160}$}
\rput[tl](51.30131087652506,50){$f_1$}
\end{scriptsize}
\end{pspicture*}}
{\psset{xunit=0.04cm,yunit=0.05cm,algebraic=true,dimen=middle,dotstyle=o,dotsize=5pt 0,linewidth=0.8pt,arrowsize=3pt 2,arrowinset=0.25}
\begin{pspicture*}(-8.721096172049721,-13.67475458356514)(170,95.16587686451683)
\begin{scriptsize}
\psaxes[xAxis=true,yAxis=true,labels=none,Dx=50.,Dy=10.,ticks=x,ticksize=-2pt 0,subticks=0]{->}(0,0)(-8.721096172049721,-13.67475458356514)(170,95.16587686451683)[$x$,140] [$f_2(x)$,-40]
\psplot[linewidth=1.2pt,plotpoints=200]{0}{160}{-1.989065674500714E-5*x^(3.0)+0.005568228690189832*x^(2.0)-0.053735655918526345*x+5.936048947883409}
\rput[tl](48,-3){$\text{50}$}
\rput[tl](95,-3){$\text{100}$}
\rput[tl](155,-3){$\text{160}$}
\rput[tl](51.30131087652506,35){$f_2$}
\end{scriptsize}
\end{pspicture*}}

\begin{center}
\psset{xunit=0.04cm,yunit=0.05cm,algebraic=true,dimen=middle,dotstyle=o,dotsize=5pt 0,linewidth=0.8pt,arrowsize=3pt 2,arrowinset=0.25}
\begin{pspicture*}(-8.721096172049721,-13.67475458356514)(170,95.16587686451683)
\begin{scriptsize}
\psaxes[xAxis=true,yAxis=true,labels=none,Dx=50.,Dy=10.,showorigin=false,ticks=x,ticksize=-2pt 0,subticks=0]{->}(0,0)(-8.721096172049721,-13.67475458356514)(170,95.16587686451683)[$x$,140] [$f_3(x)$,-40]
\psplot[linewidth=1.2pt,plotpoints=200]{0}{160}{x^(2.0)/450.0}
\rput[tl](48,-3){$\text{50}$}
\rput[tl](95,-3){$\text{100}$}
\rput[tl](155,-3){$\text{160}$}
\rput[tl](60,25){$f_3$}
\end{scriptsize}
\end{pspicture*}
\end{center}

Kreuze die zutreffende(n) Aussage(n) an.
\multiplechoice[5]{  %Anzahl der Antwortmoeglichkeiten, Standard: 5
				L1={Die Funktionswerte von $f_1'$ sind im Intervall $[0; 160]$ negativ.},   %1. Antwortmoeglichkeit 
				L2={Der Wert des Differenzialquotienten von $f_3$ wächst im Intervall $[0; 160]$ mit
wachsendem $x$.},   %2. Antwortmoeglichkeit
				L3={Die Funktion $f_2''$ hat im Intervall $(0; 160)$ genau eine Nullstelle.},   %3. Antwortmoeglichkeit
				L4={Die Funktionswerte von $f_3''$ sind im Intervall $[0; 160]$ negativ.},   %4. Antwortmoeglichkeit
				L5={Die Funktion $f_1'$ ist im Intervall $[0; 160]$ streng monoton fallend.},	 %5. Antwortmoeglichkeit
				L6={},	 %6. Antwortmoeglichkeit
				L7={},	 %7. Antwortmoeglichkeit
				L8={},	 %8. Antwortmoeglichkeit
				L9={},	 %9. Antwortmoeglichkeit
				%% LOESUNG: %%
				A1=2,  % 1. Antwort
				A2=3,	 % 2. Antwort
				A3=5,  % 3. Antwort
				A4=0,  % 4. Antwort
				A5=0,  % 5. Antwort
				}
\end{beispiel}