\section{AN 3.3 - 27 Eigenschaften einer Funktion - MC - Matura 2013/14 Haupttermin}

\begin{beispiel}[AN 3.3]{1} %PUNKTE DES BEISPIELS
				Von einer rellen Polynomfunktion $f$ sind der Graph und die Funktionsgleichung der Ableitungsfunktion $f'$ gegeben: $f'(x)=-x+2$.
				
				\begin{center}\resizebox{0.5\linewidth}{!}{\psset{xunit=1.0cm,yunit=1.0cm,algebraic=true,dimen=middle,dotstyle=o,dotsize=5pt 0,linewidth=0.8pt,arrowsize=3pt 2,arrowinset=0.25}
\begin{pspicture*}(-2.78,-2.42)(4.5,4.52)
\multips(0,-2)(0,1.0){7}{\psline[linestyle=dashed,linecap=1,dash=1.5pt 1.5pt,linewidth=0.4pt,linecolor=lightgray]{c-c}(-2.78,0)(4.5,0)}
\multips(-2,0)(1.0,0){8}{\psline[linestyle=dashed,linecap=1,dash=1.5pt 1.5pt,linewidth=0.4pt,linecolor=lightgray]{c-c}(0,-2.42)(0,4.52)}
\psaxes[labelFontSize=\scriptstyle,xAxis=true,yAxis=true,Dx=1.,Dy=1.,ticksize=-2pt 0,subticks=2]{->}(0,0)(-2.78,-2.42)(4.5,4.52)[x,140] [f'(x),-40]
\begin{scriptsize}
\rput[tl](0.78,1.7){f'}
\end{scriptsize}
\psplot{-2.78}{4.5}{(--4.-2.*x)/2.}
\end{pspicture*}}\end{center}\leer

Kreuze die beiden zutreffenden Aussagen an!\leer

\multiplechoice[5]{  %Anzahl der Antwortmoeglichkeiten, Standard: 5
				L1={Die Stelle $x_1=0$ ist eine Wendestelle von $f$.},   %1. Antwortmoeglichkeit 
				L2={Im Intervall $[0;1]$ ist $f$ streng monoton fallend.},   %2. Antwortmoeglichkeit
				L3={Die Tangente an den Graphen der Funktion $f$ im Punkt $(0|f(0))$ hat die Steigung 2.},   %3. Antwortmoeglichkeit
				L4={Die Stelle $x_2=2$ ist eine lokale Maximumstelle von $f$.},   %4. Antwortmoeglichkeit
				L5={Der Graph der Funktion $f$ weist im Intervall $[2;3]$ eine Linkskrümmung (positive Krümmung) auf.},	 %5. Antwortmoeglichkeit
				L6={},	 %6. Antwortmoeglichkeit
				L7={},	 %7. Antwortmoeglichkeit
				L8={},	 %8. Antwortmoeglichkeit
				L9={},	 %9. Antwortmoeglichkeit
				%% LOESUNG: %%
				A1=3,  % 1. Antwort
				A2=4,	 % 2. Antwort
				A3=0,  % 3. Antwort
				A4=0,  % 4. Antwort
				A5=0,  % 5. Antwort
				}
\end{beispiel}