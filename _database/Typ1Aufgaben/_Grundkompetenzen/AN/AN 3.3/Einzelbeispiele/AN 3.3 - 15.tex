\section{AN 3.3 - 15 - Monotonie - LT - BIFIE}

\begin{beispiel}[AN 3.3]{1} %PUNKTE DES BEISPIELS
				Gegeben ist die reelle Funktion $f$ mit $f(x)=x^2-2x+3$.
				
				\lueckentext[-0.05]{
								text={Die Funktion $f$ ist im Intervall [2;3] \gap, weil \gap.}, 	%Lueckentext Luecke=\gap
								L1={streng monoton fallend}, 		%1.Moeglichkeit links  
								L2={konstant}, 		%2.Moeglichkeit links
								L3={streng monoton steigend}, 		%3.Moeglichkeit links
								R1={für alle $x\in[2;3]\,f''(x)>0$ gilt}, 		%1.Moeglichkeit rechts 
								R2={für alle $x\in[2;3]\,f'(x)>0$ gilt}, 		%2.Moeglichkeit rechts
								R3={es ein $x\in[2;3]$ mit $f'(x)=0$ gibt}, 		%3.Moeglichkeit rechts
								%% LOESUNG: %%
								A1=3,   % Antwort links
								A2=2		% Antwort rechts 
								}
\end{beispiel}