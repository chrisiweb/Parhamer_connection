\section{AN 3.3 - 11 Pflanzenwachstum - MC - BIFIE}

\begin{beispiel}[AN 3.3]{1} %PUNKTE DES BEISPIELS
				Die Höhe $h$ (in cm) von drei verschiedenen Pflanzen in Abhängigkeit von der Zeit $t$ (in Tagen) wurde über einen läneren Zeitraum beobachtet und mittels geeigneter Funktionen $h_1$ (für die Pflanze 1), $h_2$ (für die Pflanze 2) und $h_3$ (für die Pflanze 3) modelliert. Die nachstehende Abbildung zeigt die Graphen der drei Funktionen $h_1,h_2$ und $h_3$.
				\begin{center}
					\psset{xunit=0.6cm,yunit=0.8cm,algebraic=true,dimen=middle,dotstyle=o,dotsize=5pt 0,linewidth=0.8pt,arrowsize=3pt 2,arrowinset=0.25}
\begin{pspicture*}(-1.068140920862352,-1.3044971034249697)(20.092369996730643,15.495666109936746)
\multips(0,0)(0,1.0){17}{\psline[linestyle=dashed,linecap=1,dash=1.5pt 1.5pt,linewidth=0.4pt,linecolor=gray]{c-c}(0,0)(20.092369996730643,0)}
\multips(0,0)(1.0,0){22}{\psline[linestyle=dashed,linecap=1,dash=1.5pt 1.5pt,linewidth=0.4pt,linecolor=gray]{c-c}(0,0)(0,15.495666109936746)}
\psaxes[labelFontSize=\scriptstyle,xAxis=true,yAxis=true,Dx=1.,Dy=2.,ticksize=-2pt 0,subticks=0]{->}(0,0)(0.,0.)(20.092369996730643,15.495666109936746)[Zeit t (in Tagen),140] [Höhe h (in cm),-40]
\psplot[linewidth=1.2pt,plotpoints=200]{0}{17}{-0.04210203007154861*x^(2.0)+1.4097473919609318*x+0.5871243270871337}
\psplot[linewidth=1.2pt,plotpoints=200]{0}{17}{-0.004513999874025487*x^(3.0)+0.12102207500187186*x^(2.0)-0.05864538379486307*x+0.5871243270871337}
\psplot[linewidth=1.2pt,plotpoints=200]{0}{17}{0.0418190236099568*x^(2.0)-0.015434862840929712*x+0.5871243270871337}
\rput[tl](8.58233451276718,10.84676598410192){$h_3$}
\rput[tl](9.704482819003172,8.76277627252079){$h_2$}
\rput[tl](13.199173258423832,9.179574214837016){$h_1$}
\end{pspicture*}
				\end{center}
				
				Kreuze die beiden zutreffenden Aussagen an!
				\multiplechoice[5]{  %Anzahl der Antwortmoeglichkeiten, Standard: 5
								L1={Der Graph der Funktion $h_1$ ist im Intervall [1;5] links gekrümmt.},   %1. Antwortmoeglichkeit 
								L2={Die Wachstumsgeschwindigkeit von Pflanze 1 nimmt im Intervall [11;13] ab.},   %2. Antwortmoeglichkeit
								L3={Während des Beobachtungszeitraums [0;17] nimmt die Wachstumsgeschwindigkeit von Pflanze 2 ständig zu.},   %3. Antwortmoeglichkeit
								L4={Für alle Werte $t\in[0;17]$ gilt $h_3''(t)\leq 0$},   %4. Antwortmoeglichkeit
								L5={Für alle Werte $t\in[3;8]$ gilt $h_1'(t)<0$},	 %5. Antwortmoeglichkeit
								L6={},	 %6. Antwortmoeglichkeit
								L7={},	 %7. Antwortmoeglichkeit
								L8={},	 %8. Antwortmoeglichkeit
								L9={},	 %9. Antwortmoeglichkeit
								%% LOESUNG: %%
								A1=1,  % 1. Antwort
								A2=4,	 % 2. Antwort
								A3=0,  % 3. Antwort
								A4=0,  % 4. Antwort
								A5=0,  % 5. Antwort
								}
\end{beispiel}