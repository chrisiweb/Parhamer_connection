\section{AN 3.3 - 2 Zweite Ableitung einer Funktion - MC - BIFIE}

\begin{beispiel}[AN 3.3]{1} %PUNKTE DES BEISPIELS
In der nachstehenden Abbildung ist der Graph der Funktion $f''$ einer Polynomfunktion $f$ dargestellt:

\begin{center}
\psset{xunit=1.0cm,yunit=1.0cm,algebraic=true,dimen=middle,dotstyle=o,dotsize=5pt 0,linewidth=0.8pt,arrowsize=3pt 2,arrowinset=0.25}
\begin{pspicture*}(-2.4049893887185867,-2.9034880814815796)(6.658925835151265,3.61057846402488)
\multips(0,-2)(0,1.0){7}{\psline[linestyle=dashed,linecap=1,dash=1.5pt 1.5pt,linewidth=0.4pt,linecolor=gray]{c-c}(-2.4049893887185867,0)(6.658925835151265,0)}
\multips(-2,0)(1.0,0){10}{\psline[linestyle=dashed,linecap=1,dash=1.5pt 1.5pt,linewidth=0.4pt,linecolor=gray]{c-c}(0,-2.9034880814815796)(0,3.61057846402488)}
\psaxes[labelFontSize=\scriptstyle,xAxis=true,yAxis=true,Dx=1.,Dy=1.,ticksize=-2pt 0,subticks=2]{->}(0,0)(-2.4049893887185867,-2.9034880814815796)(6.658925835151265,3.61057846402488)[x,140] [$f''(x)$,-40]
\psplot[linewidth=1.2pt,plotpoints=200]{-2.4049893887185867}{6.658925835151265}{4.0/5.0*x-2.0}
\rput[tl](3.3720115231984615,2.0766851184469064){$f''$}
\end{pspicture*}
\end{center}

Welche Aussage l�sst sich aus dieser Information eindeutig schlie�en?\\

Kreuze die zutreffende Aussage an.

\multiplechoice[6]{  %Anzahl der Antwortmoeglichkeiten, Standard: 5
				L1={Die Funktion $f$ hat im Intervall $[-1;1]$ eine Nullstelle. },   %1. Antwortmoeglichkeit 
				L2={Die Funktion $f$ hat im Intervall $[-1;1]$ eine lokale Extremstelle.},   %2. Antwortmoeglichkeit
				L3={Die Funktion $f$ hat im Intervall $[-1;1]$ eine Wendestelle.},   %3. Antwortmoeglichkeit
				L4={Die Funktion $f$ ist im Intervall $[-1;1]$ streng monoton steigend.},   %4. Antwortmoeglichkeit
				L5={Die Funktion $f$ �ndert im Intervall $[-1;1]$ ihr Monotonieverhalten.},	 %5. Antwortmoeglichkeit
				L6={Der Graph der Funktion $f$ ist im Intervall $[-1; 1]$ rechts gekr�mmt \mbox{(negativ gekr�mmt)}.},	 %6. Antwortmoeglichkeit
				L7={},	 %7. Antwortmoeglichkeit
				L8={},	 %8. Antwortmoeglichkeit
				L9={},	 %9. Antwortmoeglichkeit
				%% LOESUNG: %%
				A1=6,  % 1. Antwort
				A2=0,	 % 2. Antwort
				A3=0,  % 3. Antwort
				A4=0,  % 4. Antwort
				A5=0,  % 5. Antwort
				}
\end{beispiel}