\section{AN 3.3 - 35 - MAT - Eigenschaften einer Polynomfunktion dritten Grades - MC - Matura 2. NT 2017/18}

\begin{beispiel}[AN 3.3]{1}
Gegeben ist der Graph einer Polynomfunktion dritten Grades $f$. Die Stellen $x=-2$ und $x=2$ sind Extremstellen von $f$.

\begin{center}
\psset{xunit=1.0cm,yunit=0.2cm,algebraic=true,dimen=middle,dotstyle=o,dotsize=5pt 0,linewidth=0.8pt,arrowsize=3pt 2,arrowinset=0.25}
\begin{pspicture*}(-4.9,-14.9)(4.9,28)
\multips(0,-15)(0,5.0){9}{\psline[linestyle=dashed,linecap=1,dash=1.5pt 1.5pt,linewidth=0.4pt,linecolor=gray]{c-c}(-4.9,0)(4.9,0)}
\multips(-5,0)(1.0,0){10}{\psline[linestyle=dashed,linecap=1,dash=1.5pt 1.5pt,linewidth=0.4pt,linecolor=gray]{c-c}(0,-14.9)(0,28)}
\begin{scriptsize}
\psaxes[xAxis=true,yAxis=true,showorigin=false,Dx=1.,Dy=5.,ticksize=-2pt 0,subticks=0]{->}(0,0)(-4.9,-14.9)(4.9,28)[$x$,140] [$f(x)$,-40]
\psplot[linewidth=0.8pt,plotpoints=200]{-4.9}{4.9}{x^3-12*x+8}
\rput[bl](-3.6,12){$f$}
\end{scriptsize}
\end{pspicture*}
\end{center}

Kreuze die beiden zutreffenden Aussagen an!

\multiplechoice[5]{  %Anzahl der Antwortmoeglichkeiten, Standard: 5
				L1={$f'(0)=0$},   %1. Antwortmoeglichkeit 
				L2={$f''(1)>0$},   %2. Antwortmoeglichkeit
				L3={$f'(-3)<0$},   %3. Antwortmoeglichkeit
				L4={$f'(2)=0$},   %4. Antwortmoeglichkeit
				L5={$f''(-2)>0$},	 %5. Antwortmoeglichkeit
				L6={},	 %6. Antwortmoeglichkeit
				L7={},	 %7. Antwortmoeglichkeit
				L8={},	 %8. Antwortmoeglichkeit
				L9={},	 %9. Antwortmoeglichkeit
				%% LOESUNG: %%
				A1=2,  % 1. Antwort
				A2=4,	 % 2. Antwort
				A3=0,  % 3. Antwort
				A4=0,  % 4. Antwort
				A5=0,  % 5. Antwort
				}
\end{beispiel}