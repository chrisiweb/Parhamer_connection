\section{AN 3.3 - 6 Kostenkehre - OA - BIFIE}

\begin{beispiel}[AN 3.3]{1} %PUNKTE DES BEISPIELS
In einem Betrieb können die Kosten zur Herstellung eines Produkts in einem bestimmten Intervall näherungsweise durch die Funktion $K$ mit der Gleichung \mbox{$K(x) = a \cdot x³ + b\cdot x² + c \cdot x + d$}
mit $a, b, c, d \in \mathbb{R}$ und $a > 0$ beschrieben werden ($K(x)$ in \euro, $x$ in mg). \leer

Begründe, warum es bei dieser Modellierung durch eine Polynomfunktion dritten Grades genau eine Stelle gibt, bei der die Funktion von einem degressiven Kostenverlauf in einen progressiven Kostenverlauf übergeht.

\antwort{Der Übergang von einem degressiven in einen progressiven Kostenverlauf (die Kostenkehre)
der Funktion $K$ wird durch $K''(x) = 6 \cdot a \cdot x + 2 \cdot b = 0$ berechnet.\\
$6 \cdot a \cdot x + 2 \cdot b = 0$ ist (für $a>0$) eine lineare Gleichung mit genau einer Lösung bei $x=-\dfrac{b}{3\cdot a}$, wobei $K'''\left(-\dfrac{b}{3\cdot a}\right) = 6 \cdot a \neq 0$. \\
Daraus folgt, dass es nur eine Kostenkehre gibt.}
\end{beispiel}