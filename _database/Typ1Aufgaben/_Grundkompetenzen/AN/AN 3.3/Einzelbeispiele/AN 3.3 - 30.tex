\section{AN 3.3 - 30 - MAT - Zeit-Weg-Funktion - MC - Matura NT 1 16/17}

\begin{beispiel}[AN 3.3]{1} %PUNKTE DES BEISPIELS
Die geradlinige Bewegung eines Autos wird mithilfe der Zeit-Weg-Funktion $s$ beschrieben. Innerhalb des Beobachtungszeitraums ist die Funktion $s$ streng monoton wachsend und rechtsgekrümmt.

Kreuze die beiden für diesen Beobachtungszeitraum zutreffenden Aussagen an!

\multiplechoice[5]{  %Anzahl der Antwortmoeglichkeiten, Standard: 5
				L1={Die Geschwindigkeit des Autos wird immer größer.},   %1. Antwortmoeglichkeit 
				L2={Die Funktionswerte von $s'$ sind negativ.},   %2. Antwortmoeglichkeit
				L3={Die Funktionswerte von $s''$ sind negativ.},   %3. Antwortmoeglichkeit
				L4={Der Wert des Differenzenquotienten von $s$ im Beobachtungszeitraum ist negativ.},   %4. Antwortmoeglichkeit
				L5={Der Wert des Differenzialquotienten von $s$ wird immer kleiner.},	 %5. Antwortmoeglichkeit
				L6={},	 %6. Antwortmoeglichkeit
				L7={},	 %7. Antwortmoeglichkeit
				L8={},	 %8. Antwortmoeglichkeit
				L9={},	 %9. Antwortmoeglichkeit
				%% LOESUNG: %%
				A1=3,  % 1. Antwort
				A2=5,	 % 2. Antwort
				A3=0,  % 3. Antwort
				A4=0,  % 4. Antwort
				A5=0,  % 5. Antwort
				}
\end{beispiel}