\section{AN 3.3 - 21 Nachweis eines lokalen Minimums - OA - Matura 2015/16 - Haupttermin}

\begin{beispiel}[AN 3.3]{1} %PUNKTE DES BEISPIELS
Gegeben ist eine Polynomfunktion $p$ mit $p(x) = x^3 - 3 \cdot x + 2$. Die erste Ableitung $p'$ mit $p'(x) = 3 \cdot x^2 - 3$ hat an der Stelle $x = 1$ den Wert null. \leer

Zeige rechnerisch, dass $p$ an dieser Stelle ein lokales Minimum (d. h. ihr Graph dort einen
Tiefpunkt) hat.

\antwort{\leer

Möglicher rechnerischer Nachweis:

$p''(x) = 6x$

$p''(1) = 6 > 0 \Rightarrow$ An der Stelle 1 liegt ein lokales Minimum vor.}
\end{beispiel}