\section{AN 3.3 - 38 - MAT - Eigenschaften einer Polynomfunktion - LT - Matura 2018/19 2. NT}

\begin{beispiel}[AN 3.3]{1}
Es sei $f$: $\mathbb{R}\rightarrow\mathbb{R}$ eine Polynomfunktion und $a,b\in\mathbb{R}$ mit $a<b$.

\lueckentext{
				text={Wenn für alle $a\in(a;b)$ \gap gilt, dann ist die Funktion $f$ im Intervall $(a;b)$ \gap.}, 	%Lueckentext Luecke=\gap
				L1={$f(x)>0$}, 		%1.Moeglichkeit links  
				L2={$f'(x)<0$}, 		%2.Moeglichkeit links
				L3={$f''(x)>0$}, 		%3.Moeglichkeit links
				R1={streng monoton fallend}, 		%1.Moeglichkeit rechts 
				R2={rechtsgekrümmt (negativ gekrümmt)}, 		%2.Moeglichkeit rechts
				R3={streng monoton steigend}, 		%3.Moeglichkeit rechts
				%% LOESUNG: %%
				A1=2,   % Antwort links
				A2=1		% Antwort rechts 
				}
\end{beispiel}