\section{AN 3.3 - 7 - Wendepunkt - OA - BIFIE}

\begin{beispiel}[AN 3.3]{1} %PUNKTE DES BEISPIELS
	Gegeben sind die Funktion $f$ mit der Gleichung \mbox{$f(x)=\frac{1}{4}x^3+\frac{3}{2}x^2+4x+5$} sowie die Gleichung der dritten Ableitungsfunktion $f'''(x)=\frac{3}{2}\neq 0$.
	
Berechne die Koordinaten des Wendepunktes der Funktion $f$.	

\antwort{$f''(x)=\frac{3}{2}x+3=0 \Rightarrow x=-2$

$f(-2)=\frac{1}{4}\cdot (-8)+\frac{3}{2}\cdot 4 + 4 \cdot (-2)+5=1 \Rightarrow$

Die Koordinaten des Wendepunktes lauten daher $W=(-2|1)$.

Die Aufgabe gilt nur dann als gelöst, wenn beide Koordinaten des Wendepunktes korrekt angegeben
sind.}


\end{beispiel}