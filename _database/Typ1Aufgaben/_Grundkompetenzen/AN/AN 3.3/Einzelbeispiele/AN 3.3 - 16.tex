\section{AN 3.3 - 16 - Ableitungsfunktion - LT - BIFIE}

\begin{beispiel}[AN 3.3]{1} %PUNKTE DES BEISPIELS
In der nachstehenden Abbildung ist der Graph der Ableitungsfunktion $f'$ einer Funktion $f$ dargestellt.

\begin{center}
\psset{xunit=1.0cm,yunit=1.0cm,algebraic=true,dimen=middle,dotstyle=o,dotsize=5pt 0,linewidth=0.8pt,arrowsize=3pt 2,arrowinset=0.25}
\begin{pspicture*}(-4.48924462184089,-4.313210198857752)(4.446627650059073,5.384783375549648)
\multips(0,-4)(0,1.0){10}{\psline[linestyle=dashed,linecap=1,dash=1.5pt 1.5pt,linewidth=0.4pt,linecolor=gray]{c-c}(-4.48924462184089,0)(4.446627650059073,0)}
\multips(-4,0)(1.0,0){9}{\psline[linestyle=dashed,linecap=1,dash=1.5pt 1.5pt,linewidth=0.4pt,linecolor=gray]{c-c}(0,-4.313210198857752)(0,5.384783375549648)}
\begin{scriptsize}
\psaxes[xAxis=true,yAxis=true,showorigin=false,Dx=1.,Dy=1.,ticksize=-2pt 0,subticks=0]{->}(0,0)(-4.48924462184089,-4.313210198857752)(4.446627650059073,5.384783375549648)[$x$,140] [$f'(x)$,-40]
\psplot[linewidth=1.2pt,plotpoints=200]{-4.48924462184089}{4.446627650059073}{0.2904063017168803*x^(3.0)-2.613656715451923*x}
\rput[tl](2.922385045044154,3.0984194680272745){$f'$}
\end{scriptsize}
\end{pspicture*}
\end{center}

Kreuze die zutreffende(n) Aussage(n) an.

\multiplechoice[5]{  %Anzahl der Antwortmoeglichkeiten, Standard: 5
				L1={Die Funktion $f$ hat im Intervall $[-4; 4]$ drei lokale Extremstellen.},   %1. Antwortmoeglichkeit 
				L2={Die Funktion $f$ ist im Intervall $(2; 3)$ streng monoton steigend.},   %2. Antwortmoeglichkeit
				L3={Die Funktion $f$ hat im Intervall $[-3; 0]$ eine Wendestelle.},   %3. Antwortmoeglichkeit
				L4={Die Funktion $f''$ hat im Intervall $[-3; 3]$ zwei Nullstellen.},   %4. Antwortmoeglichkeit
				L5={Die Funktion $f$ hat an der Stelle $x = 0$ ein lokales Minimum.},	 %5. Antwortmoeglichkeit
				L6={},	 %6. Antwortmoeglichkeit
				L7={},	 %7. Antwortmoeglichkeit
				L8={},	 %8. Antwortmoeglichkeit
				L9={},	 %9. Antwortmoeglichkeit
				%% LOESUNG: %%
				A1=1,  % 1. Antwort
				A2=3,	 % 2. Antwort
				A3=4,  % 3. Antwort
				A4=0,  % 4. Antwort
				A5=0,  % 5. Antwort
				}
\end{beispiel}