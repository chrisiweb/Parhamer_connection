\section{AN 3.3 - 39 - MAT - Kurvenverlauf - ZO - Matura 2019/20 1. HT}

\begin{beispiel}[AN 3.3]{1}
Die unten links stehenden Abbildungen zeigen jeweils die Tangente $t$ in einem Punkt $P=(x_p\mid f(x_p))$ des Graphen einer Polynomfunktion $f$. Dabei ist $P$ der einzige gemeinsame Punkt des Graphen von $f$ und der Tangente $t$. In der unten stehenden Tabelle sind Aussagen über $f'(x_p)$ und $f''(x_p)$ gegeben.

Ordne den vier Abbildungen jeweils die zutreffende Aussage (aus A bis F) zu.

\zuordnen{
				R1={\psset{xunit=0.6cm,yunit=0.6cm,algebraic=true,dimen=middle,dotstyle=o,dotsize=5pt 0,linewidth=0.8pt,arrowsize=3pt 2,arrowinset=0.25}
\begin{pspicture*}(-0.64,-0.52)(6.7,6.48)
\psaxes[labelFontSize=\scriptstyle,xAxis=true,yAxis=true,labels=none,Dx=1.,Dy=1.,ticks=none]{->}(0,0)(-0.64,-0.52)(6.7,6.48)[$x$,140] [$f(x)$,-40]
\psplot[linewidth=1.7pt,plotpoints=200]{-0.6399999999999996}{6.699999999999997}{0.1*(x-3.0)^(3.0)+0.3*x+2.0}
\psplot[linewidth=1pt]{-0.64}{6.7}{(--2.--0.3*x)/1.}
\rput[tl](5.74,5.46){$f$}
\psdots[dotsize=6pt 0,dotstyle=*](3.,2.9)
\rput[bl](3,3.2){$P$}
\rput[bl](0.7,2.7){$t$}
\end{pspicture*}},				% Response 1
				R2={\psset{xunit=0.6cm,yunit=0.6cm,algebraic=true,dimen=middle,dotstyle=o,dotsize=5pt 0,linewidth=0.8pt,arrowsize=3pt 2,arrowinset=0.25}
\begin{pspicture*}(-0.64,-0.52)(6.7,6.48)
\psaxes[labelFontSize=\scriptstyle,xAxis=true,yAxis=true,labels=none,Dx=1.,Dy=1.,ticks=none]{->}(0,0)(-0.64,-0.52)(6.7,6.48)[$x$,140] [$f(x)$,-40]
\psplot[linewidth=1.7pt,plotpoints=200]{-5.919999999999997}{20.75999999999999}{-0.04416874320547999*x^(3.0)+0.10154658523834101*x^(2.0)+0.14384587661914286*x+2.65947185145219}
\psplot[linewidth=1.pt]{-5.92}{20.76}{(--4.772839256238768-0.6424214627225209*x)/1.}
\rput[tl](3.9,1){$f$}
\psdots[dotsize=6pt 0,dotstyle=*](3.32,2.64)
\rput[bl](3.4,2.88){$P$}
\rput[bl](0.7,3.5){$t$}
\end{pspicture*}},				% Response 2
				R3={\psset{xunit=0.6cm,yunit=0.6cm,algebraic=true,dimen=middle,dotstyle=o,dotsize=5pt 0,linewidth=0.8pt,arrowsize=3pt 2,arrowinset=0.25}
\begin{pspicture*}(-0.64,-0.52)(6.7,6.48)
\psaxes[labelFontSize=\scriptstyle,xAxis=true,yAxis=true,labels=none,Dx=1.,Dy=1.,ticks=none]{->}(0,0)(-0.64,-0.52)(6.7,6.48)[$x$,140] [$f(x)$,-40]
\psplot[linewidth=1.7pt,plotpoints=200]{-10.769945393364655}{22.974321151658057}{-0.007811324562222329*x^(3.0)-0.0015258252373834913*x^(2.0)+0.5667488016253669*x+1.7508478003560068}
\psplot[linewidth=1.pt]{-10.769945393364655}{22.974321151658057}{(--2.3800725739937487--0.28619644359351004*x)/1.}
\rput[tl](0.3,1.5){$f$}
\psdots[dotsize=6pt 0,dotstyle=*](3.3955638009477287,3.35187085781985)
\rput[bl](3.496746009478531,3.655417483412258){$P$}
\rput[bl](0.7,3){$t$}
\end{pspicture*}},				% Response 3
				R4={\psset{xunit=0.6cm,yunit=0.6cm,algebraic=true,dimen=middle,dotstyle=o,dotsize=5pt 0,linewidth=0.8pt,arrowsize=3pt 2,arrowinset=0.25}
\begin{pspicture*}(-0.64,-0.52)(6.7,6.48)
\psaxes[labelFontSize=\scriptstyle,xAxis=true,yAxis=true,labels=none,Dx=1.,Dy=1.,ticks=none]{->}(0,0)(-0.64,-0.52)(6.7,6.48)[$x$,140] [$f(x)$,-40]
\psplot[linewidth=1.7pt,plotpoints=200]{-10.769945393364655}{22.974321151658057}{-0.09999999999999998*x^(3.0)+0.8999999999999998*x^(2.0)-2.999999999999999*x+6.699999999999999}
\psplot[linewidth=1.pt]{-10.769945393364655}{22.974321151658057}{(--4.-0.3*x)/1.}
\rput[tl](4.3,1.5){$f$}
\psdots[dotsize=6pt 0,dotstyle=*](3.,3.1)
\rput[bl](3.0920171753553203,3.402461962085251){$P$}
\rput[bl](0.7,3){$t$}
\end{pspicture*}},				% Response 4
				%% Moegliche Zuordnungen: %%
				A={$f'(x_p)>0$ und $f''(x_p)>0$}, 				%Moeglichkeit A  
				B={$f'(x_p)>0$ und $f''(x_p)<0$}, 				%Moeglichkeit B  
				C={$f'(x_p)<0$ und $f''(x_p)>0$}, 				%Moeglichkeit C  
				D={$f'(x_p)<0$ und $f''(x_p)<0$}, 				%Moeglichkeit D  
				E={$f'(x_p)>0$ und $f''(x_p)=0$}, 				%Moeglichkeit E  
				F={$f'(x_p)<0$ und $f''(x_p)=0$}, 				%Moeglichkeit F  
				%% LOESUNG: %%
				A1={E},				% 1. richtige Zuordnung
				A2={D},				% 2. richtige Zuordnung
				A3={B},				% 3. richtige Zuordnung
				A4={F},				% 4. richtige Zuordnung
				}
				
\antwort{\textbf{Lösungsschlüssel:}\\
Bei zwei oder drei richtigen Zuordnungen ist ein halber Punkt zu geben.}
\end{beispiel}