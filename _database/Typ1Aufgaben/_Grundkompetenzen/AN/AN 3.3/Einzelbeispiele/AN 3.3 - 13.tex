\section{AN 3.3 - 13 Ableitungsfunktion - MC - BIFIE}

\begin{beispiel}[AN 3.3]{1} %PUNKTE DES BEISPIELS
				Die Abbildung zeigt den Graphen der Ableitungsfunktion $f'$ einer Polynomfunktion $f$.
				\begin{center}
					\psset{xunit=1.0cm,yunit=1.0cm,algebraic=true,dimen=middle,dotstyle=o,dotsize=5pt 0,linewidth=0.8pt,arrowsize=3pt 2,arrowinset=0.25}
\begin{pspicture*}(-3.519015652326958,-3.783581329740892)(5.011717602655804,6.659536083501739)
\multips(0,-3)(0,1.0){11}{\psline[linestyle=dashed,linecap=1,dash=1.5pt 1.5pt,linewidth=0.4pt,linecolor=lightgray]{c-c}(-3.519015652326958,0)(5.011717602655804,0)}
\multips(-3,0)(1.0,0){9}{\psline[linestyle=dashed,linecap=1,dash=1.5pt 1.5pt,linewidth=0.4pt,linecolor=lightgray]{c-c}(0,-3.783581329740892)(0,6.659536083501739)}
\psaxes[labelFontSize=\scriptstyle,xAxis=true,yAxis=true,Dx=1.,Dy=1.,ticksize=-2pt 0,subticks=2]{->}(0,0)(-3.519015652326958,-3.783581329740892)(5.011717602655804,6.659536083501739)[x,140] [y,-40]
\psplot[linewidth=1.2pt,plotpoints=200]{-3.519015652326958}{5.011717602655804}{-0.32497368781752023*x^(3.0)+1.004188921320476*x^(2.0)-0.08780357360374604*x}
\begin{scriptsize}
\rput[bl](-1.4941383082870934,5.215873532658502){$f'$}
\end{scriptsize}
\end{pspicture*}
				\end{center}
				
				Kreuze die beiden zutreffenden Aussagen an!
				
				\multiplechoice[5]{  %Anzahl der Antwortmoeglichkeiten, Standard: 5
								L1={Die Funktion $f$ hat an der Stelle $x=3$ einen lokalen Hochpunkt.},   %1. Antwortmoeglichkeit 
								L2={Die Funktion $f$ ist im Intervall [2;5] streng monoton fallend.},   %2. Antwortmoeglichkeit
								L3={Die Funktion $f$ hat an der Stelle $x=0$ einen Wendepunkt.},   %3. Antwortmoeglichkeit
								L4={Die Funktion $f$ hat an der Stelle $x=0$ eine lokale Extremstelle.},   %4. Antwortmoeglichkeit
								L5={Die Funktion $f$ ist im Intervall [-2;0] links gekr�mmt.},	 %5. Antwortmoeglichkeit
								L6={},	 %6. Antwortmoeglichkeit
								L7={},	 %7. Antwortmoeglichkeit
								L8={},	 %8. Antwortmoeglichkeit
								L9={},	 %9. Antwortmoeglichkeit
								%% LOESUNG: %%
								A1=1,  % 1. Antwort
								A2=3,	 % 2. Antwort
								A3=0,  % 3. Antwort
								A4=0,  % 4. Antwort
								A5=0,  % 5. Antwort
								}
\end{beispiel}