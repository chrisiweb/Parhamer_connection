\section{AN 3.3 - 36 - MAT - Polynomfunktion - MC - Matura-HT-18/19}

\begin{beispiel}[AN 3.3]{1}
In der nachstehenden Abbildung ist der Graph einer Polynomfunktion $f\!:\mathbb{R}\rightarrow\mathbb{R}$ vom Grad 3 im Intervall $[-1;7]$ dargestellt. Alle lokalen Extremstellen sowie die Wendestelle von $f$ im Intervall $[-1;7]$ sing ganzzahlig und können aus der Abbildung abgelesen werden.

\begin{center}
\psset{xunit=1.0cm,yunit=1.0cm,algebraic=true,dimen=middle,dotstyle=o,dotsize=5pt 0,linewidth=1.6pt,arrowsize=3pt 2,arrowinset=0.25}
\begin{pspicture*}(-1.72,-0.52)(7.66,4.8)
\multips(0,0)(0,1.0){6}{\psline[linestyle=dashed,linecap=1,dash=1.5pt 1.5pt,linewidth=0.4pt,linecolor=gray]{c-c}(-1.72,0)(7.66,0)}
\multips(-1,0)(1.0,0){10}{\psline[linestyle=dashed,linecap=1,dash=1.5pt 1.5pt,linewidth=0.4pt,linecolor=gray]{c-c}(0,0)(0,4.8)}
\begin{scriptsize}
\psaxes[xAxis=true,yAxis=true,Dx=1.,Dy=1.,ticksize=-2pt 0,subticks=0]{->}(0,0)(-1.72,-0.52)(7.66,4.8)[$x$,140] [$f(x)$,-40]
\psplot[linewidth=2.pt,plotpoints=200]{-1.}{7}{-0.034968884461000896*x^(3.0)+0.31471996014900805*x^(2.0)+0.03722674636865053*x}
\rput[tl](4.34,2.86){$f$}
\end{scriptsize}
\end{pspicture*}
\end{center}

Kreuze die beiden auf die Funktion $f$ zutreffenden Aussagen an!
\multiplechoice[5]{  %Anzahl der Antwortmoeglichkeiten, Standard: 5
				L1={$f''(3)=0$},   %1. Antwortmoeglichkeit 
				L2={$f'(1)>f'(3)$},   %2. Antwortmoeglichkeit
				L3={$f''(1)=f''(5)$},   %3. Antwortmoeglichkeit
				L4={$f''(1)>f''(4)$},   %4. Antwortmoeglichkeit
				L5={$f'(3)=0$},	 %5. Antwortmoeglichkeit
				L6={},	 %6. Antwortmoeglichkeit
				L7={},	 %7. Antwortmoeglichkeit
				L8={},	 %8. Antwortmoeglichkeit
				L9={},	 %9. Antwortmoeglichkeit
				%% LOESUNG: %%
				A1=1,  % 1. Antwort
				A2=4,	 % 2. Antwort
				A3=0,  % 3. Antwort
				A4=0,  % 4. Antwort
				A5=0,  % 5. Antwort
				}
\end{beispiel}