\section{AG-L 5.1 - 1 - K7 - Hyperbelgleichung - OA - MatKon}

\begin{beispiel}[AG-L 5.1]{1}
Gegeben ist die Abbildung einer Hyperbel in 1. Hauptlage.
			\begin{center}
				\psset{xunit=0.8cm,yunit=0.8cm,algebraic=true,dimen=middle,dotstyle=o,dotsize=4pt 0,linewidth=0.8pt,arrowsize=3pt 2,arrowinset=0.25}
\begin{pspicture*}(-6.6,-5.62)(7.,5.92)
\multips(0,-5)(0,1.0){12}{\psline[linestyle=dashed,linecap=1,dash=1.5pt 1.5pt,linewidth=0.4pt,linecolor=gray]{c-c}(-6.6,0)(7.,0)}
\multips(-6,0)(1.0,0){13}{\psline[linestyle=dashed,linecap=1,dash=1.5pt 1.5pt,linewidth=0.4pt,linecolor=gray]{c-c}(0,-5.62)(0,5.92)}
\psaxes[labelFontSize=\scriptstyle,xAxis=true,yAxis=true,Dx=1.,Dy=1.,showorigin=false,ticksize=-2pt 0,subticks=0]{->}(0,0)(-6.6,-5.62)(7.,5.92)[x,140] [y,-40]
\rput{90}(0,0){\parametricplot[linewidth=2pt]{-0.99}{0.99}{3*(1+t^2)/(1-t^2)|4*2*t/(1-t^2)}}
\rput{90}(0,0){\parametricplot[linewidth=2pt]{-0.99}{0.99}{3*(-1-t^2)/(1-t^2)|4*(-2)*t/(1-t^2)}}
\begin{scriptsize}
\psdots[dotstyle=*](0,-5)
\rput[bl](0.16,-5.2){$F_1$}
\psdots[dotstyle=*](0,-3)
\rput[bl](0.16,-2.8){$A(0\mid -3)$}
\psdots[dotstyle=*](0,3)
\rput[bl](0.16,2.6){B}
\psdots[dotstyle=*](0,5)
\rput[bl](0.16,5.){$F_2(0\mid 5)$}
\psdots[dotstyle=*](0.,0.)
\rput[bl](0.16,0.2){M}
\end{scriptsize}
\end{pspicture*}
			\end{center}
			
			Stelle die Gleichung dieser Hyperbel auf!
			
			\antwort{$a=3$ und $e=5 \Rightarrow b=\sqrt{e^2-a^2}=4$
			
			$\text{hyp: } -\frac{x^2}{16}+\frac{y^2}{9}=1$ bzw. $-9x^2+16y^2=144$}
\end{beispiel}