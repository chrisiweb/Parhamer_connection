\section{AG 1.1 - 18 - MAT - Zahlenmengen - MC - Matura 1.NT 2018/19}

\begin{beispiel}[AG 1.1]{1}
Zwischen Zahlenmengen bestehen bestimmte Beziehungen.

Kreuzen Sie die beiden wahren Aussagen an.

\multiplechoice[5]{  %Anzahl der Antwortmoeglichkeiten, Standard: 5
				L1={$\mathbb{Z}^+\subseteq\mathbb{N}$},   %1. Antwortmoeglichkeit 
				L2={$\mathbb{C}\subseteq\mathbb{Z}$},   %2. Antwortmoeglichkeit
				L3={$\mathbb{N}\subseteq\mathbb{R}^-$},   %3. Antwortmoeglichkeit
				L4={$\mathbb{R}^+\subseteq\mathbb{Q}$},   %4. Antwortmoeglichkeit
				L5={$\mathbb{Q}\subseteq\mathbb{C}$},	 %5. Antwortmoeglichkeit
				L6={},	 %6. Antwortmoeglichkeit
				L7={},	 %7. Antwortmoeglichkeit
				L8={},	 %8. Antwortmoeglichkeit
				L9={},	 %9. Antwortmoeglichkeit
				%% LOESUNG: %%
				A1=1,  % 1. Antwort
				A2=5,	 % 2. Antwort
				A3=0,  % 3. Antwort
				A4=0,  % 4. Antwort
				A5=0,  % 5. Antwort
				}
\end{beispiel}