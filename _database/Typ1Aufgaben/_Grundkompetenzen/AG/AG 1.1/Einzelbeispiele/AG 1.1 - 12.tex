\section{AG 1.1 - 12 - MAT - Ganze Zahlen - MC - Matura 2016/17 - Haupttermin}

\begin{beispiel}[AG 1.1]{1} %PUNKTE DES BEISPIELS
	Es sei $a$ eine positive ganze Zahl.
			
Welche der nachstehenden Ausdrücke ergeben für $a\in \mathbb{Z}^+$ stets eine ganze Zahl? 

Kreuze die beiden zutreffenden Ausdrücke an.

\multiplechoice[5]{  %Anzahl der Antwortmoeglichkeiten, Standard: 5
				L1={$a^{-1}$},   %1. Antwortmoeglichkeit 
				L2={$a^2$},   %2. Antwortmoeglichkeit
				L3={$a^{\frac{1}{2}}$},   %3. Antwortmoeglichkeit
				L4={$3\cdot a$},   %4. Antwortmoeglichkeit
				L5={$\frac{a}{2}$},	 %5. Antwortmoeglichkeit
				L6={},	 %6. Antwortmoeglichkeit
				L7={},	 %7. Antwortmoeglichkeit
				L8={},	 %8. Antwortmoeglichkeit
				L9={},	 %9. Antwortmoeglichkeit
				%% LOESUNG: %%
				A1=2,  % 1. Antwort
				A2=4,	 % 2. Antwort
				A3=0,  % 3. Antwort
				A4=0,  % 4. Antwort
				A5=0,  % 5. Antwort
				}			
\end{beispiel}