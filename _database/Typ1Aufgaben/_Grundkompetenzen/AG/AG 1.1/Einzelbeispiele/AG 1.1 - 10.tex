\section{AG 1.1 - 10 - MAT - Positive rationale Zahlen - MC - Matura 2013/14 Haupttermin}

\begin{beispiel}[AG 1.1]{1} %PUNKTE DES BEISPIELS
				Gegeben ist die Zahlenmenge $\mathbb{Q}^+$.
				
				Kreuze jene beiden Zahlen an, die Elemente dieser Zahlenmenge sind!\leer
				
				\multiplechoice[5]{  %Anzahl der Antwortmoeglichkeiten, Standard: 5
								L1={$\sqrt{5}$},   %1. Antwortmoeglichkeit 
								L2={$0,9\cdot 10^{-3}$},   %2. Antwortmoeglichkeit
								L3={$\sqrt{0,01}$},   %3. Antwortmoeglichkeit
								L4={$\frac{\pi}{4}$},   %4. Antwortmoeglichkeit
								L5={$-1,41\cdot 10^3$},	 %5. Antwortmoeglichkeit
								L6={},	 %6. Antwortmoeglichkeit
								L7={},	 %7. Antwortmoeglichkeit
								L8={},	 %8. Antwortmoeglichkeit
								L9={},	 %9. Antwortmoeglichkeit
								%% LOESUNG: %%
								A1=2,  % 1. Antwort
								A2=3,	 % 2. Antwort
								A3=0,  % 3. Antwort
								A4=0,  % 4. Antwort
								A5=0,  % 5. Antwort
								}

\end{beispiel}