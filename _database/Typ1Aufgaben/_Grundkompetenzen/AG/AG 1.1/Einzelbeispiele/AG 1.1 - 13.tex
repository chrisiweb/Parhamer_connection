\section{AG 1.1 - 13 - MAT - Zahlenmengen - MC - Matura NT 1 16/17}

\begin{beispiel}[AG 1.1]{1} %PUNKTE DES BEISPIELS
Untenstehend werden Aussagen über Zahlen aus den Zahlenmengen $\mathbb{N}, \mathbb{Z}, \mathbb{Q}, \mathbb{R}$ und $\mathbb{C}$ getroffen.

Kreuze die zutreffende(n) Aussage(n) an!
\multiplechoice[5]{  %Anzahl der Antwortmoeglichkeiten, Standard: 5
				L1={Jede reelle Zahl ist eine rationale Zahl.},   %1. Antwortmoeglichkeit 
				L2={Jede natürliche Zahl ist eine rationale Zahl.},   %2. Antwortmoeglichkeit
				L3={Jede ganze Zahl ist eine reelle Zahl.},   %3. Antwortmoeglichkeit
				L4={Jede rationale Zahl ist eine reelle Zahl.},   %4. Antwortmoeglichkeit
				L5={Jede komplexe Zahl ist eine reelle Zahl.},	 %5. Antwortmoeglichkeit
				L6={},	 %6. Antwortmoeglichkeit
				L7={},	 %7. Antwortmoeglichkeit
				L8={},	 %8. Antwortmoeglichkeit
				L9={},	 %9. Antwortmoeglichkeit
				%% LOESUNG: %%
				A1=2,  % 1. Antwort
				A2=3,	 % 2. Antwort
				A3=4,  % 3. Antwort
				A4=0,  % 4. Antwort
				A5=0,  % 5. Antwort
				}
\end{beispiel}