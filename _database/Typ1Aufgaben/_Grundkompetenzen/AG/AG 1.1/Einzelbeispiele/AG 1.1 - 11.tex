\section{AG 1.1 - 11 - MAT - Aussagen über Zahlenmengen - MC - Matura 2013/14 1. Nebentermin}

\begin{beispiel}[AG 1.1]{1} %PUNKTE DES BEISPIELS
				Untenstehend sind fünf Aussagen über Zahlen aus den Zahlenmengen $\mathbb{N}, \mathbb{Z}, \mathbb{Q}$ und $\mathbb{R}$ angeführt.
				
				Kreuze die beiden Aussagen an, die korrekt sind!\leer
				
				\multiplechoice[5]{  %Anzahl der Antwortmoeglichkeiten, Standard: 5
								L1={Reelle Zahlen mit periodischer oder endlicher Dezimaldarstellung sind rationale Zahlen.},   %1. Antwortmoeglichkeit 
								L2={Die Differenz zweier natürlicher Zahlen ist stets eine  natürliche Zahl. },   %2. Antwortmoeglichkeit
								L3={Alle Wurzelausdrücke der Form $\sqrt{a}$ für $a\in\mathbb{R}$ und $a>0$ sind stets irrationale Zahlen.},   %3. Antwortmoeglichkeit
								L4={Zwischen zwei verschiedenen rationalen Zahlen $a, b$ existiert stets eine weitere rationale Zahl. },   %4. Antwortmoeglichkeit
								L5={Der Quotient zweier negativer ganzer Zahlen ist stets eine positive ganze Zahl.},	 %5. Antwortmoeglichkeit
								L6={},	 %6. Antwortmoeglichkeit
								L7={},	 %7. Antwortmoeglichkeit
								L8={},	 %8. Antwortmoeglichkeit
								L9={},	 %9. Antwortmoeglichkeit
								%% LOESUNG: %%
								A1=1,  % 1. Antwort
								A2=4,	 % 2. Antwort
								A3=0,  % 3. Antwort
								A4=0,  % 4. Antwort
								A5=0,  % 5. Antwort
								}
\end{beispiel}