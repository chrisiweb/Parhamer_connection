\section{AG 1.1 - 8 - Abgeschlossene Zahlenmengen - MC - MatKon}

\begin{beispiel}[AG 1.1]{1} %PUNKTE DES BEISPIELS
				Eine Zahlenmenge M heißt abgeschlossen bezüglich der Addition (Multiplikation), wenn die Summe (das Produkt) zweier Zahlen aus M wieder in M liegt. Welche der folgenden Mengen sind abgeschlossen gegenüber der Addition? Kreuze die entsprechende(n) Zahlenmenge(n) an.

\multiplechoice[5]{  %Anzahl der Antwortmoeglichkeiten, Standard: 5
				L1={$\mathbb{Z}^{+}$},   %1. Antwortmoeglichkeit 
				L2={$\mathbb{Q}$},   %2. Antwortmoeglichkeit
				L3={$\mathbb{N}_{g}$},   %3. Antwortmoeglichkeit
				L4={$\mathbb{R}^{+}$},   %4. Antwortmoeglichkeit
				L5={$[0;1]$},	 %5. Antwortmoeglichkeit
				L6={},	 %6. Antwortmoeglichkeit
				L7={},	 %7. Antwortmoeglichkeit
				L8={},	 %8. Antwortmoeglichkeit
				L9={},	 %9. Antwortmoeglichkeit
				%% LOESUNG: %%
				A1=1,  % 1. Antwort
				A2=2,	 % 2. Antwort
				A3=3,  % 3. Antwort
				A4=4,  % 4. Antwort
				A5=0,  % 5. Antwort
				}
\end{beispiel}