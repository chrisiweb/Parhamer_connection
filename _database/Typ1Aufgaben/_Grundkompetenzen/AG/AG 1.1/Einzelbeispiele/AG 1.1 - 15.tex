\section{AG 1.1 - 15 - Zahlenmengen - MC - Matura - 1. NT 2017/18}

\begin{beispiel}[AG 1.1]{1}
Nachstehend sind Aussagen �ber Zahlen aus den Mengen $\mathbb{Z}, \mathbb{Q} und \mathbb{R}$ angef�hrt.

Kreuze die beiden zutreffenden Aussagen an!

\multiplechoice[5]{  %Anzahl der Antwortmoeglichkeiten, Standard: 5
				L1={Irrationale Zahlen lassen sich in der Form $\frac{a}{b}$ mit $a,b\in\mathbb{Z}$ und $b\neq 0$ darstellen.},   %1. Antwortmoeglichkeit 
				L2={Jede rationale Zahl kann in endlicher oder periodischer Dezimalschreibweise geschrieben werden.},   %2. Antwortmoeglichkeit
				L3={Jede Bruchzahl ist eine reelle Zahl.},   %3. Antwortmoeglichkeit
				L4={Die Menge der rationalen Zahlen besteht ausschlie�lich aus positiven Bruchzahlen.},   %4. Antwortmoeglichkeit
				L5={Jede reelle Zahl ist auch eine rationale Zahl.},	 %5. Antwortmoeglichkeit
				L6={},	 %6. Antwortmoeglichkeit
				L7={},	 %7. Antwortmoeglichkeit
				L8={},	 %8. Antwortmoeglichkeit
				L9={},	 %9. Antwortmoeglichkeit
				%% LOESUNG: %%
				A1=2,  % 1. Antwort
				A2=3,	 % 2. Antwort
				A3=0,  % 3. Antwort
				A4=0,  % 4. Antwort
				A5=0,  % 5. Antwort
				}
\end{beispiel}