\section{AG 1.1 - 5 Menge von Zahlen - MC - Matura 2015/16 - Haupttermin}

\begin{beispiel}[AG 1.1]{1} %PUNKTE DES BEISPIELS
Die Menge $M = \{ x \in \mathbb{Q}\,|\,2 < x < 5\}$ ist eine Teilmenge der rationalen Zahlen. \leer

Kreuze die beiden zutreffenden Aussagen an.

\multiplechoice[5]{  %Anzahl der Antwortmoeglichkeiten, Standard: 5
				L1={4,99 ist die größte Zahl, die zur Menge $M$ gehört.},   %1. Antwortmoeglichkeit 
				L2={Es gibt unendlich viele Zahlen in der Menge $M$, die kleiner als 2,1 sind.},   %2. Antwortmoeglichkeit
				L3={Jede reelle Zahl, die größer als 2 und kleiner als 5 ist, ist in der Menge $M$
enthalten.},   %3. Antwortmoeglichkeit
				L4={Alle Elemente der Menge M können in der Form $\frac{a}{b}$
geschrieben werden, wobei $a$ und $b$ ganze Zahlen sind und $b \neq 0$ ist.},   %4. Antwortmoeglichkeit
				L5={Die Menge $M$ enthält keine Zahlen aus der Menge der komplexen Zahlen.},	 %5. Antwortmoeglichkeit
				L6={},	 %6. Antwortmoeglichkeit
				L7={},	 %7. Antwortmoeglichkeit
				L8={},	 %8. Antwortmoeglichkeit
				L9={},	 %9. Antwortmoeglichkeit
				%% LOESUNG: %%
				A1=2,  % 1. Antwort
				A2=4,	 % 2. Antwort
				A3=0,  % 3. Antwort
				A4=0,  % 4. Antwort
				A5=0,  % 5. Antwort
				}
\end{beispiel}
