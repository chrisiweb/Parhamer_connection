\section{AG 1.1 - 16 - MAT - Zahlen und Zahlenmengen - MC - Matura 2. NT 2017/18}

\begin{beispiel}[AG 1.1]{1}
Nachstehend sind Aussagen über Zahlen und Zahlenmengen angeführt.

Kreuze die beiden zutreffenden Aussagen an!

\multiplechoice[5]{  %Anzahl der Antwortmoeglichkeiten, Standard: 5
				L1={Es gibt mindestens eine Zahl, die in $\mathbb{N}$ enthalten ist, nicht aber in $\mathbb{Z}$.},   %1. Antwortmoeglichkeit 
				L2={$-\sqrt{9}$ ist eine irrationale Zahl.},   %2. Antwortmoeglichkeit
				L3={Die Zahl 3 ist ein Element der Menge $\mathbb{Q}$.},   %3. Antwortmoeglichkeit
				L4={$\sqrt{-2}$ ist in $\mathbb{C}$ enthalten, nicht aber in $\mathbb{R}$.},   %4. Antwortmoeglichkeit
				L5={Die periodische Zahl $1,\dot{5}$ ist in $\mathbb{R}$ enthalten, nicht aber in $\mathbb{Q}$},	 %5. Antwortmoeglichkeit
				L6={},	 %6. Antwortmoeglichkeit
				L7={},	 %7. Antwortmoeglichkeit
				L8={},	 %8. Antwortmoeglichkeit
				L9={},	 %9. Antwortmoeglichkeit
				%% LOESUNG: %%
				A1=3,  % 1. Antwort
				A2=4,	 % 2. Antwort
				A3=0,  % 3. Antwort
				A4=0,  % 4. Antwort
				A5=0,  % 5. Antwort
				}
\end{beispiel}