\section{AG 1.1 - 9 - MAT - Eigenschaften von Zahlen - MC - Matura 2015/16-Nebentermin 1}

\begin{beispiel}[AG 1.1]{1} %PUNKTE DES BEISPIELS
				Nachstehend sind Aussagen über Zahlen und Zahlenmengen angeführt.
				
				Kreuze die beiden zutreffenden Aussagen an.
				
				\multiplechoice[5]{  %Anzahl der Antwortmoeglichkeiten, Standard: 5
								L1={Die Quadratwurzel jeder natürlichen Zahl ist eine irrationale Zahl},   %1. Antwortmoeglichkeit 
								L2={Jede natürliche Zahl kann als Bruch in der Form $\frac{a}{b}$ mit $a \in \mathbb{Z}$ und $b \in \mathbb{Z}\backslash\{0\}$ dargestellt werden.},   %2. Antwortmoeglichkeit
								L3={Das Produkt zweier rationalen Zahlen kann eine natürliche Zahl sein.},   %3. Antwortmoeglichkeit
								L4={Jede reelle Zahl kann als Bruch in der Form $\frac{a}{b}$ mit $a \in \mathbb{Z}$ und $b \in \mathbb{Z}\backslash\{0\}$ dargestellt werden.},   %4. Antwortmoeglichkeit
								L5={Es gibt eine kleinste ganze Zahl.},	 %5. Antwortmoeglichkeit
								L6={},	 %6. Antwortmoeglichkeit
								L7={},	 %7. Antwortmoeglichkeit
								L8={},	 %8. Antwortmoeglichkeit
								L9={},	 %9. Antwortmoeglichkeit
								%% LOESUNG: %%
								A1=2,  % 1. Antwort
								A2=3,	 % 2. Antwort
								A3=0,  % 3. Antwort
								A4=0,  % 4. Antwort
								A5=0,  % 5. Antwort
								}
\end{beispiel}