\section{AG 1.1 - 20 - MAT - Zahlen und Zahlenmengen - MC - Matura 2019/20 1. HT}

\begin{beispiel}[AG 1.1]{1}
Gegeben sind fünf Aussagen zu Zahlen und Zahlenmengen.

\multiplechoice[5]{  %Anzahl der Antwortmoeglichkeiten, Standard: 5
				L1={$\sqrt{\dfrac{9}{2}}$ ist eine rationale Zahl.},   %1. Antwortmoeglichkeit 
				L2={$-\sqrt{100}$ ist eine ganze Zahl.},   %2. Antwortmoeglichkeit
				L3={$\sqrt{15}$ hat eine endliche Dezimaldarstellung.},   %3. Antwortmoeglichkeit
				L4={$\sqrt{2}$ ist eine rationale Zahl.},   %4. Antwortmoeglichkeit
				L5={$-4$ ist kein Quadrat einer reellen Zahl.},	 %5. Antwortmoeglichkeit
				L6={},	 %6. Antwortmoeglichkeit
				L7={},	 %7. Antwortmoeglichkeit
				L8={},	 %8. Antwortmoeglichkeit
				L9={},	 %9. Antwortmoeglichkeit
				%% LOESUNG: %%
				A1=2,  % 1. Antwort
				A2=5,	 % 2. Antwort
				A3=0,  % 3. Antwort
				A4=0,  % 4. Antwort
				A5=0,  % 5. Antwort
				}
\end{beispiel}