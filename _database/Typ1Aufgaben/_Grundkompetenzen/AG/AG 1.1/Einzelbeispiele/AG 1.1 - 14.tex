\section{AG 1.1 - 14 - MAT - Zusammenhang zweier Variablen - MC - Matura HT 2017/18}

\begin{beispiel}[AG 1.1]{1} %PUNKTE DES BEISPIELS
Für $a,b\in\mathbb{R}$ gilt der Zusammenhang $a\cdot b=1$.

Zwei der fünf nachstehenden Aussagen treffen in jedem Fall zu. Kreuze die beiden zutreffenden Aussagen an!\leer

\multiplechoice[5]{  %Anzahl der Antwortmoeglichkeiten, Standard: 5
				L1={Wenn $a$ kleiner als null ist, dann ist auch $b$ kleiner als null.},   %1. Antwortmoeglichkeit 
				L2={Die Vorzeichen von $a$ und $b$ können unterschiedlich sein.},   %2. Antwortmoeglichkeit
				L3={Für jedes $n\in\mathbb{N}$ gilt: $(a-n)\cdot(b+n)=1$.},   %3. Antwortmoeglichkeit
				L4={Für jedes $n\in\mathbb{N}\backslash\{0\}$ gilt: $(a\cdot n)\cdot\left(\frac{b}{n}\right)=1$.},   %4. Antwortmoeglichkeit
				L5={Es gilt: $a\neq b$.},	 %5. Antwortmoeglichkeit
				L6={},	 %6. Antwortmoeglichkeit
				L7={},	 %7. Antwortmoeglichkeit
				L8={},	 %8. Antwortmoeglichkeit
				L9={},	 %9. Antwortmoeglichkeit
				%% LOESUNG: %%
				A1=1,  % 1. Antwort
				A2=4,	 % 2. Antwort
				A3=0,  % 3. Antwort
				A4=0,  % 4. Antwort
				A5=0,  % 5. Antwort
				}
\end{beispiel}