\section{AG 4.1 - 6 Sonnenradius - OA - BIFIE}

\begin{beispiel}[AG 4.1]{1} %PUNKTE DES BEISPIELS
Die Sonne erscheint von der Erde aus unter einem Sehwinkel von $\alpha\approx0,52^\circ$.
Die Entfernung der Erde vom Mittelpunkt der Sonne betr�gt ca. $150\cdot 10^{6}\,km$.

\newrgbcolor{qqwuqq}{0. 0.39215686274509803 0.}
\psset{xunit=1.0cm,yunit=1.0cm,algebraic=true,dimen=middle,dotstyle=o,dotsize=5pt 0,linewidth=0.8pt,arrowsize=3pt 2,arrowinset=0.25}
\begin{pspicture*}(-2.68,-0.1)(11.62,6.08)
\psline[linewidth=1.6pt](-2.,3.)(9.,3.)
\pscircle(9.,3.){2.23606797749979}
\psline[linewidth=1.6pt](9.,3.)(8.545454545454545,5.18938083250769)
\psline[linewidth=1.6pt](8.545454545454545,5.18938083250769)(-2.,3.)
\psline[linewidth=1.6pt](-2.,3.)(8.545454545454545,0.8106191674923098)
\psline[linewidth=1.6pt](8.545454545454545,0.8106191674923098)(9.,3.)
\pscustom[linecolor=qqwuqq,fillcolor=qqwuqq,fillstyle=solid,opacity=0.1]{
\parametricplot{-2.9368870622587355}{-1.3660907354638387}{0.6*cos(t)+8.545454545454545|0.6*sin(t)+5.18938083250769}
\lineto(8.545454545454545,5.18938083250769)\closepath}
\psellipse*[linecolor=qqwuqq,fillcolor=qqwuqq,fillstyle=solid,opacity=0.9](8.351829916285142,4.894292749001334)(0.04,0.04)
\pscustom[linecolor=qqwuqq,fillcolor=qqwuqq,fillstyle=solid,opacity=0.1]{
\parametricplot{1.3660907354638387}{2.9368870622587355}{0.6*cos(t)+8.545454545454545|0.6*sin(t)+0.8106191674923098}
\lineto(8.545454545454545,0.8106191674923098)\closepath}
\psellipse*[linecolor=qqwuqq,fillcolor=qqwuqq,fillstyle=solid,opacity=0.9](8.351829916285142,1.105707250998666)(0.04,0.04)
\begin{scriptsize}
\psdots[dotsize=3pt 0,dotstyle=*](-2.,3.)
\rput[bl](-2.3,3.32){Erde}
\psdots[dotsize=3pt 0,dotstyle=*](9.,3.)
\rput[bl](9.08,3.12){Sonne}
\end{scriptsize}
\end{pspicture*}

Gib eine Formel zur Berechnung des Sonnenradius an und berechne den Radius!
\leer

$r=$ \antwort[\rule{3cm}{0.3pt}]{$150\cdot 10^{6}\cdot sin(0,26^\circ)$}
\leer

$r=$ \antwort[\rule{3cm}{0.3pt}$km$]{$6,8\cdot10^{5}\, km$}

\antwort{Die Ma�zahl f�r den Radius muss aus dem Intervall $[6\cdot10^{5};7\cdot10^{5}]$ sein.}
\end{beispiel}