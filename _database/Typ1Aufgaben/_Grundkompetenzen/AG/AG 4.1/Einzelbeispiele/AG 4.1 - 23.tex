\section{AG 4.1 - 23 - MAT - Bahntrasse - OA - Matura 2019/20 1. HT}

\begin{beispiel}[AG 4.1]{1}
Die Steigung einer geradlinigen Bahntrasse wird in Promille ($\permil$) angegeben. Beispielsweise ist bei einem Höhenunterschied von 1\,m pro 1\,000\,m zurückgelegter Distanz in horizontaler Richtung die Steigung 1\,\permil.

Gib eine Gleichung an, mit der für eine geradlinige Bahntrasse mit der Steigung 30\,\permil der Steigungswinkel $\alpha$ exakt berechnet werden kann ($\alpha>0$).\leer

\antwort{$\tan^{-1}\left(\dfrac{30}{1000}\right)$}
\end{beispiel}