\section{AG 4.1 - 1 Rechtwinkliges Dreieck - MC - BIFIE}

\begin{beispiel}[AG 4.1]{1} %PUNKTE DES BEISPIELS
Gegeben ist ein rechtwinkliges Dreieck wie in nachstehender Skizze.
\begin{center}
\newrgbcolor{zzttqq}{0.6 0.2 0.}
\newrgbcolor{qqwuqq}{0. 0.39215686274509803 0.}
\psset{xunit=0.2cm,yunit=0.2cm,algebraic=true,dimen=middle,dotstyle=o,dotsize=5pt 0,linewidth=0.8pt,arrowsize=3pt 2,arrowinset=0.25}
\begin{pspicture*}(-6.360647481963261,-7.948158601769999)(24.238977324744106,39.568331951734855)
\pspolygon[linecolor=zzttqq,fillcolor=zzttqq,fillstyle=solid,opacity=0.1](-0.36,-1.66)(14.64,-1.66)(14.64,34.34)
\psline(-0.36,-1.66)(14.64,-1.66)
\psline(-0.36,-1.66)(14.64,34.34)
\psline(14.64,34.34)(14.64,-1.66)
\psline[linecolor=zzttqq](-0.36,-1.66)(14.64,-1.66)
\psline[linecolor=zzttqq](14.64,-1.66)(14.64,34.34)
\psline[linecolor=zzttqq](14.64,34.34)(-0.36,-1.66)
\pscustom[linecolor=qqwuqq,fillcolor=qqwuqq,fillstyle=solid,opacity=0.1]{
\parametricplot{0.0}{1.1760052070951352}{3.7316615617935818*cos(t)+-0.36|3.7316615617935818*sin(t)+-1.66}
\lineto(-0.36,-1.66)\closepath}
\pscustom[linecolor=qqwuqq,fillcolor=qqwuqq,fillstyle=solid,opacity=0.1]{
\parametricplot{1.5707963267948966}{3.141592653589793}{3.7316615617935818*cos(t)+14.64|3.7316615617935818*sin(t)+-1.66}
\lineto(14.64,-1.66)\closepath}
\psellipse*[linecolor=qqwuqq,fillcolor=qqwuqq,fillstyle=solid,opacity=1](13.087833414448575,-0.10783341444857375)(0.24877743745290543,0.24877743745290543)
\pscustom[linecolor=qqwuqq,fillcolor=qqwuqq,fillstyle=solid,opacity=0.1]{
\parametricplot{-1.965587446494658}{-1.5707963267948966}{6.219435936322636*cos(t)+14.64|6.219435936322636*sin(t)+34.34}
\lineto(14.64,34.34)\closepath}
\begin{scriptsize}
\psdots[dotsize=3pt 0,dotstyle=*,linecolor=darkgray](-0.36,-1.66)
\rput[bl](-1.7582648890845096,-4.216497039976424){\darkgray{$A$}}
\psdots[dotsize=3pt 0,dotstyle=*,linecolor=darkgray](14.64,-1.66)
\rput[bl](15.15860085771306,-4.216497039976424){\darkgray{$B$}}
\rput[bl](5.4562807970497476,-4.216497039976424){$c = 15$}
\psdots[dotsize=3pt 0,dotstyle=*,linecolor=darkgray](14.64,-37.66)
\psdots[dotsize=3pt 0,dotstyle=*,linecolor=darkgray](14.64,34.34)
\rput[bl](15.15860085771306,35.090338077582565){\darkgray{$C$}}
\rput[bl](0.107565891812281,15.312531800076618){$b = 39$}
\rput[bl](16.153710607524683,15.934475393708881){$a = 36$}
\rput[bl](0.97828692289745,-0.9823903530886592){\qqwuqq{$\alpha$}}
\rput[bl](13.29277007681627,29.2440682974393){\qqwuqq{$\gamma$}}
\end{scriptsize}
\end{pspicture*}
\end{center}

Welche der nachfolgenden Aussagen sind f�r das abgebildete Dreieck zutreffend?

Kreuz die beiden zutreffenden Aussagen an!
\multiplechoice[5]{  %Anzahl der Antwortmoeglichkeiten, Standard: 5
				L1={$tan(\alpha)=\frac{5}{13}$},   %1. Antwortmoeglichkeit 
				L2={$cos(\alpha)=\frac{13}{12}$},   %2. Antwortmoeglichkeit
				L3={$sin(\gamma)=\frac{5}{13}$},   %3. Antwortmoeglichkeit
				L4={$cos(\gamma)=\frac{12}{13}$},   %4. Antwortmoeglichkeit
				L5={$tan(\gamma)=\frac{12}{5}$},	 %5. Antwortmoeglichkeit
				L6={},	 %6. Antwortmoeglichkeit
				L7={},	 %7. Antwortmoeglichkeit
				L8={},	 %8. Antwortmoeglichkeit
				L9={},	 %9. Antwortmoeglichkeit
				%% LOESUNG: %%
				A1=3,  % 1. Antwort
				A2=4,	 % 2. Antwort
				A3=0,  % 3. Antwort
				A4=0,  % 4. Antwort
				A5=0,  % 5. Antwort
				}
\end{beispiel}
