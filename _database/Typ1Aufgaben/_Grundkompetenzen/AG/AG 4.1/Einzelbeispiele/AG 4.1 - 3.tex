\section{AG 4.1 - 3 - Rechtwinkeliges Dreieck - OA - BIFIE}

\begin{beispiel}[AG 4.1]{1} %PUNKTE DES BEISPIELS
				Von einem rechtwinkligen Dreieck $ABC$ sind die Längen der Seiten $a$ und $c$ gegeben.

\begin{center}
\newrgbcolor{qqwuqq}{0. 0.39215686274509803 0.}
\psset{xunit=1.0cm,yunit=1.0cm,algebraic=true,dimen=middle,dotstyle=o,dotsize=5pt 0,linewidth=0.8pt,arrowsize=3pt 2,arrowinset=0.25}
\begin{pspicture*}(-2.6792957269168296,-1.7830240399474664)(7.839433728689722,6.030889269931665)
\psline(-2.,1.)(7.,-1.)
\psline(-1.0119423187427805,5.4462595656574875)(7.,-1.)
\psline(-1.0119423187427805,5.4462595656574875)(-2.,1.)
\pscustom[fillstyle=solid,opacity=0.25]{
\parametricplot{-0.21866894587394195}{1.3521273809209546}{0.5303561070053724*cos(t)+-2.|0.5303561070053724*sin(t)+1.}
\lineto(-2.,1.)\closepath}
\psellipse*[fillstyle=solid,opacity=1](-1.736799398668258,1.167491291756563)(0.03535707380035816,0.03535707380035816)
\pscustom[fillstyle=solid,opacity=0.25]{
\parametricplot{2.4640644264507188}{2.9229237077158516}{1.0607122140107448*cos(t)+7.|1.0607122140107448*sin(t)+-1.}
\lineto(7.,-1.)\closepath}
\psdots[dotsize=3pt 0,dotstyle=*](-2.,1.)
\rput[bl](-2.308046452013069,0.5858999046765238){$B$}
\psdots[dotsize=3pt 0,dotstyle=*](7.,-1.)
\rput[bl](7.132292252682559,-1.2526679329420953){$A$}
\rput[bl](1.6873028874274028,-0.20963425583153264){$c$}
\psdots[dotsize=3pt 0,dotstyle=*](-1.0119423187427805,5.4462595656574875)
\rput[bl](-1.0467991106992799,5.753568773626831){$C$}
\rput[bl](2.9248004704399384,2.6189316481971123){$b$}
\rput[bl](-1.7953688819078757,3.2907160504039155){$a$}
\rput[bl](6.15997272317271,-0.7223118259367245){$\alpha$}
\end{pspicture*}
\end{center}
Gib eine Formel für die Berechnung des Winkels $\alpha$ an!
\leer

\antwort{$\alpha=tan^{-1}(\frac{a}{c})$ oder $\alpha=arctan(\frac{a}{c})$ oder $tan(\alpha)=\frac{a}{c}$\\

Lösungsschlüssel:

Als nicht richtig zu werten sind Umformungsketten, die die Gleichheit verletzen, wie z.B.: $\alpha=tan(\alpha)=\frac{a}{c}=tan^{-1}(\frac{a}{c})$.

Formeln, bei denen $b$ durch $a$ und $c$ ausgedrückt wird, sind ebenso als richtig zu werten, wie z.B.: $sin(\alpha)=\dfrac{a}{\sqrt{a^2+c^2}}$.}
\end{beispiel}