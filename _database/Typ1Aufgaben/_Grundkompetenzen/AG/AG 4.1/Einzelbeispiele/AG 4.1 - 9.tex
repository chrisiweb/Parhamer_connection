\section{AG 4.1 - 9 Sehwinkel - OA - Matura 2014/15 - Haupttermin}

\begin{beispiel}[AG 4.1]{1} %PUNKTE DES BEISPIELS
Der Sehwinkel ist derjenige Winkel, unter dem ein Objekt von einem Beobachter wahrgenommen
wird. Die nachstehende Abbildung verdeutlicht den Zusammenhang zwischen dem Sehwinkel $\alpha$,
der Entfernung $r$ und der realen ("`wahren"') Ausdehnung $g$ eines Objekts in zwei Dimensionen.
\leer


\begin{center}
\resizebox{0.8\linewidth}{!}{
\psset{xunit=1.0cm,yunit=1.0cm,algebraic=true,dimen=middle,dotstyle=o,dotsize=5pt 0,linewidth=0.8pt,arrowsize=3pt 2,arrowinset=0.25}
\begin{pspicture*}(2.8044671336415026,-0.1517944668121856)(9.331484498270326,5.164757859285479)
\pspolygon[linecolor=gray,fillcolor=gray,fillstyle=solid,opacity=0.5](8.,4.5)(8.5,4.5)(8.5,0.5)(8.,0.5)
\psline(4.,3.)(3.,2.5)
\psline(3.,2.5)(4.,2.)
\parametricplot{-1.051650212548374}{0.896055384571344}{1.*0.8062257748298549*cos(t)+0.*0.8062257748298549*sin(t)+3.|0.*0.8062257748298549*cos(t)+1.*0.8062257748298549*sin(t)+2.5}
\parametricplot{2.556543767907307}{3.730340846977759}{1.*0.5412441528494601*cos(t)+0.*0.5412441528494601*sin(t)+4.2|0.*0.5412441528494601*cos(t)+1.*0.5412441528494601*sin(t)+2.5}
\parametricplot[fillcolor=black,fillstyle=solid,opacity=1.0]{2.8032685141037055}{3.4831451519557395}{1.*0.43099917287242917*cos(t)+0.*0.43099917287242917*sin(t)+4.2|0.*0.43099917287242917*cos(t)+1.*0.43099917287242917*sin(t)+2.5}
\parametricplot[fillcolor=black,fillstyle=solid,opacity=1.0]{-0.1803677864212867}{0.17837802046432993}{1.*0.806225774829855*cos(t)+0.*0.806225774829855*sin(t)+3.|0.*0.806225774829855*cos(t)+1.*0.806225774829855*sin(t)+2.5}
\psline(3.805,2.5)(8.,2.5)
\psline[linewidth=2.pt](8.,0.5)(8.,4.5)
\psline(8.,4.5)(8.5,4.5)
\psline(8.5,4.5)(8.5,0.5)
\psline(8.5,0.5)(8.,0.5)
\psline(8.9,4.5)(8.9,0.50091)
\psline[linestyle=dashed,dash=2pt 2pt](8.5,4.5)(8.9,4.5)
\psline[linestyle=dashed,dash=2pt 2pt](8.5,0.5)(8.9,0.50091)
\rput[tl](8.075,1.4621589178960341){\tiny Objekt}
\rput[tl](2.9,1.7){\tiny Beobachter}
\psline(3.805,2.5)(9.,5.)
\psline(3.805,2.5)(9.,0.)
\parametricplot{1.5707963267948966}{3.141592653589793}{0.356019128979754*cos(t)+8.|0.356019128979754*sin(t)+2.5}
\psellipse*[fillcolor=black,fillstyle=solid,opacity=1](7.851915564509581,2.6480844354904187)(0.02373460859865027,0.02373460859865027)
\parametricplot{-0.44488176595303497}{0.44488176595303497}{0.712038257959508*cos(t)+3.805|0.712038257959508*sin(t)+2.5}
\parametricplot{0.0}{0.44488176595303497}{1.1867304299325134*cos(t)+3.805|1.1867304299325134*sin(t)+2.5}
\rput[tl](4.643899300036899,2.85){\tiny $\frac{\alpha}{2}$}
\begin{scriptsize}
\rput[bl](6.222250771847142,2.6251547392298984){$r$}
\rput[bl](8.987332673589897,2.4){$g$}
\rput[bl](4.15,2.38){\scalebox{0.8}{\colorbox[rgb]{1,1,1}{$\alpha$}}}
\end{scriptsize}
\end{pspicture*}}
\end{center}
\tiny Quelle: http://upload.wikimedia.org/wikipedia/commons/d/d3/ScheinbareGroesse.png [22.01.2015] (adaptiert) 
\leer


\normalsize

Gib eine Formel an,  mit der die reale Ausdehnung $g$ dieses Objekts mithilfe von $\alpha$ und $r$ berechnet werden kann. \leer

$g=$ \antwort[\rule{3cm}{0.3pt}km]{$2\cdot r \cdot \tan\left( \frac{\alpha}{2}\right)$ mit $\alpha \in (0;180^\circ)$ bzw. $\alpha \in (0;\pi)$}
\end{beispiel}