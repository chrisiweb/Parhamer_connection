\section{AG 4.1 - 22 - MAT - Räumliches Sehen - OA - Matura 2018/19 2. NT}

\begin{beispiel}[AG 4.1]{1}
Betrachtet man einen Gegenstand, so schließen die Blickrichtungen der beiden Augen einen Winkel $\varepsilon$ ein. In der nachstehend dargestellten Situation hat der Gegenstand $G$ zu den beiden Augen $A_1$ und $A_2$ den gleichen Abstand $g$. Der Augenabstand wird mit $d$ bezeichnet.

\begin{center}
\psset{xunit=1cm,yunit=0.8cm,algebraic=true,dimen=middle,dotstyle=o,dotsize=5pt 0,linewidth=1.6pt,arrowsize=3pt 2,arrowinset=0.25}
\begin{pspicture*}(-2.72,-3.68)(9.62,3.7)
\psline[linewidth=2.pt](-2.,3.)(-2.,-3.)
\psline[linewidth=2.pt](9.,0.)(-2.,3.)
\psline[linewidth=2.pt](-2.,-3.)(9.,0.)
\psline[linewidth=2.pt,linestyle=dashed,dash=5pt 3pt](9.,0.)(-2.,0.)
\parametricplot{0.0}{1.5707963267948966}{0.6*cos(t)+-2.|0.6*sin(t)+0.}
\psellipse*[linewidth=2.pt,fillcolor=black,fillstyle=solid,opacity=1](-1.7504329007576889,0.24956709924231116)(0.07,0.07)
\parametricplot{2.875340604438868}{3.407844702740719}{2*cos(t)+9.|2*sin(t)+0.}
\psdots[dotsize=5pt 0,dotstyle=*](-2.,3.)
\rput[bl](-2.3,3.16){$A_2$}
\psdots[dotsize=5pt 0,dotstyle=*](-2.,-3.)
\rput[bl](-2.3,-3.7){$A_1$}
\rput[bl](-2.4,0.02){$d$}
\psdots[dotsize=5pt 0,dotstyle=*](9.,0.)
\rput[bl](9.08,0.16){$G$}
\rput[bl](3.58,1.82){$g$}
\rput[bl](3.58,-2){$g$}
\rput[bl](7.2,-0.3){$\epsilon$}
\end{pspicture*}
\end{center}

Gib den Abstand $g$ in Abhängigkeit vom Augenabstand $d$ und vom Winkel $\varepsilon$ an.\leer

$g=\,\antwort[\rule{3cm}{0.3pt}]{\dfrac{d}{2\cdot\sin\left(\frac{\varepsilon}{2}\right)}}$
\end{beispiel}