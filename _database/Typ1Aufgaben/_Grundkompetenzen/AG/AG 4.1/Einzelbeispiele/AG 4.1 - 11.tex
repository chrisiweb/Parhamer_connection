\section{AG 4.1 - 11 - MAT - Festungsbahn Salzburg - OA - Matura 2014/15 - Nebentermin 2}

\begin{beispiel}[AG 4.1]{1} %PUNKTE DES BEISPIELS
				Die Festungsbahn Salzburg ist eine Standseilbahn in der Stadt Salzburg mit konstanter Steigung. Die Bahn auf den dortigen Festungsberg ist die älteste in Betrieb befindliche Seilbahn dieser Art in Österreich. Die Standseilbahn legt eine Wegstrecke von $198,5\,m$ zurück und überwindet dabei einen Höhenunterschied von $96,6\,m$.
				
				Berechne den Winkel $\alpha$, unter dem die Gleise der Bahn gegen die Horizontale geneigt sind.\\
				
				\antwort{$\sin(\alpha)=\frac{96,6}{198,5}\Rightarrow\alpha\approx 29,12^\circ$
				
				Toleranzintervall: $\left[29^\circ;30^\circ\right]$}
\end{beispiel}