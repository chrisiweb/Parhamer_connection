\section{AG 4.1 - 7 Aufwölbung des Bodensees - OA - Matura 2015/16
- Nebentermin 1}

\begin{beispiel}[AG 4.1]{1} %PUNKTE DES BEISPIELS

Aufgrund der Erdkrümmung ist die Oberfläche des Bodensees gewölbt. Wird die Erde modellhaft
als Kugel mit dem Radius $R = 6\,370$\,km und dem Mittelpunkt $M$ angenommen und aus der Länge
der Südost-Nordwest-Ausdehnung des Bodensees der Winkel $\phi = 0,5846^\circ$ ermittelt, so lässt
sich die Aufwölbung des Bodensees näherungsweise berechnen. \leer

\begin{center}
\resizebox{0.9\linewidth}{!}{\newrgbcolor{qqwuqq}{0. 0. 0.}
\psset{xunit=1.0cm,yunit=1.0cm,algebraic=true,dimen=middle,dotstyle=o,dotsize=5pt 0,linewidth=0.8pt,arrowsize=3pt 2,arrowinset=0.25}
\begin{pspicture*}(-6.105219107389741,-0.45986986771741695)(6,6.105219107389741)
\psline(-2.,5.)(0.,0.)
\psline(-2.,5.)(2.,5.)
\psline(2.,5.)(0.,0.)
\parametricplot[linewidth=3.2pt]{1.1902899496825317}{1.9513027039072617}{1.*5.385164807134505*cos(t)+0.*5.385164807134505*sin(t)+0.|0.*5.385164807134505*cos(t)+1.*5.385164807134505*sin(t)+0.}
\pscustom[linecolor=qqwuqq,fillcolor=qqwuqq,fillstyle=solid,opacity=0]{
\parametricplot{1.1902899496825317}{1.9513027039072615}{1.1936525409285728*cos(t)+0.|1.1936525409285728*sin(t)+0.}
\lineto(0.,0.)\closepath}
\psline(0.,5.385164807134505)(0.,0.)
\rput[tl](-0.2,1.1){\colorbox[rgb]{1,1,1}{\small $\varphi$}}
\rput[tl](1.0399299530591761,2.6095223803846306){$R$}
\rput[tl](-1.4155838454224596,2.626574559540753){$R$}
\psline[linewidth=3.2pt](0.,5.385164807134505)(0.,5.)
\psline{->}(2.1994781356755033,5.576601553549939)(0.03385138284795142,5.184401432959122)
\rput[tl](2.3188433897683614,5.713018986798923){Aufwölbung}
\rput[tl](-0.955175008207153,5.951749494984639){Bodensee}
\pscustom[linecolor=qqwuqq,fillcolor=qqwuqq,fillstyle=solid,opacity=0.2]{
\parametricplot{3.141592653589793}{4.71238898038469}{0.4263044789030617*cos(t)+0.|0.4263044789030617*sin(t)+5.}
\lineto(0.,5.)\closepath}
\psellipse*[linecolor=qqwuqq,fillcolor=qqwuqq,fillstyle=solid,opacity=1](-0.17731928698973679,4.822680713010263)(0.03410435831224494,0.03410435831224494)
\begin{scriptsize}
\psdots[dotsize=3pt 0,dotstyle=*,linecolor=darkgray](0.,0.)
\rput[rl](-0.0684616920887845,-0.323452434468437){\normalsize $M$}
\end{scriptsize}
\end{pspicture*}}
\end{center}

Berechne die Aufwölbung des Bodensees (siehe obige Abbildung) in Metern. \leer

Aufwölbung: \rule{4cm}{0.3pt} Meter

\antwort{Mögliche Berechnung: \leer

$6\,370-6370\cdot \cos\left(\frac{0,5846}{2}\right)\approx 0,083\,km ~\widehat{=} ~83\,m$

Toleranzintervall: $[82\,m;~ 84\,m]$}
\end{beispiel}