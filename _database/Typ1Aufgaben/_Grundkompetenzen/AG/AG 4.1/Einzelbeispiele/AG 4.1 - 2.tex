\section{AG 4.1 - 2 Winkelfunktion - OA - BIFIE}

\begin{beispiel}[AG 4.1]{1} %PUNKTE DES BEISPIELS
Gegeben ist ein rechtwinkliges Dreieck:

\psset{xunit=1.0cm,yunit=1.0cm,algebraic=true,dimen=middle,dotstyle=o,dotsize=5pt 0,linewidth=0.8pt,arrowsize=3pt 2,arrowinset=0.25}
\begin{pspicture*}(-4.3,-0.1)(3.74,6.3)
\psline(-3.,3.)(-1.,1.)
\psline(-3.,3.)(3.,5.)
\psline(-1.,1.)(3.,5.)
\rput[tl](-1.18,1.66){90$^\circ$}
\rput[tl](-0.64,4.24){w}
\rput[tl](0.9,2.58){u}
\rput[tl](-2.42,2.06){v}
\rput[tl](-2.42,2.9){$\varphi$}
\rput[tl](1.86,4.54){$\psi$}
\begin{scriptsize}
\psdots[dotsize=3pt 0,dotstyle=*](-3.,3.)
\psdots[dotsize=3pt 0,dotstyle=*](-1.,1.)
\psdots[dotsize=3pt 0,dotstyle=*](3.,5.)
\end{scriptsize}
\end{pspicture*}

Gib $\tan(\psi)$ in Abhängigkeit von den Seitenlängen $u,v$ und $w$ an!
\leer

$tan(\psi)=$ \antwort[\rule{3cm}{0.3pt}]{$\frac{v}{u}$}
\end{beispiel}