\section{AG 4.1 - 12 - MAT - Rhombus - OA - Matura NT 2 15/16}

\begin{beispiel}[AG 4.1]{1} %PUNKTE DES BEISPIELS
In einem Rhombus mit der Seite $a$ halbieren die Diagonalen $e=AC$ und $f=BD$ einander. Die Diagonale $e$ halbiert den Winkel $\alpha=\angle DAB$ und die Diagonale $f$ halbiert den Winkel $\beta=\angle ABC$.

\resizebox{1\linewidth}{!}{\newrgbcolor{uuuuuu}{0.26666666666666666 0.26666666666666666 0.26666666666666666}
\newrgbcolor{srsrsr}{0.12941176470588237 0.12941176470588237 0.12941176470588237}
\psset{xunit=1.0cm,yunit=1.0cm,algebraic=true,dimen=middle,dotstyle=o,dotsize=5pt 0,linewidth=0.8pt,arrowsize=3pt 2,arrowinset=0.25}
\begin{pspicture*}(-2.98,-2.3)(12.48,6.14)
\psline(-2.,-1.)(6.,-1.)
\pscustom[linecolor=srsrsr,fillcolor=srsrsr,fillstyle=solid,opacity=0.2]{
\parametricplot{0.0}{0.7853981633974483}{1.*cos(t)+-2.|1.*sin(t)+-1.}
\lineto(-2.,-1.)\closepath}
\psline(-2.,-1.)(3.6568542494923806,4.65685424949238)
\psline(6.,-1.)(11.656854249492381,4.65685424949238)
\psline(11.656854249492381,4.65685424949238)(3.6568542494923806,4.65685424949238)
\psline(-2.,-1.)(11.656854249492381,4.65685424949238)
\psline(6.,-1.)(3.6568542494923806,4.65685424949238)
\pscustom[linecolor=srsrsr,fillcolor=srsrsr,fillstyle=solid,opacity=0.1]{
\parametricplot{0.3926990816987241}{1.963495408493621}{0.6*cos(t)+4.828427124746191|0.6*sin(t)+1.8284271247461896}
\lineto(4.828427124746191,1.8284271247461896)\closepath}
\psellipse*[linecolor=srsrsr,fillcolor=srsrsr,fillstyle=solid,opacity=1](4.963491865580928,2.15450225386782)(0.04,0.04)
\pscustom[linecolor=srsrsr,fillcolor=srsrsr,fillstyle=solid,opacity=0.2]{
\parametricplot{0.7853981633974482}{3.141592653589793}{1.*cos(t)+6.|1.*sin(t)+-1.}
\lineto(6.,-1.)\closepath}
\psdots[dotsize=3pt 0,dotstyle=*](-2.,-1.)
\rput[bl](-2.2,-1.58){$A$}
\psdots[dotsize=3pt 0,dotstyle=*](6.,-1.)
\rput[bl](6.18,-1.42){$B$}
\rput[bl](2.,-1.32){$a$}
\psdots[dotsize=3pt 0,dotstyle=*](3.6568542494923806,4.65685424949238)
\rput[bl](3.74,4.78){$D$}
\rput[bl](-1.48,-0.82){$\alpha$}
\psdots[dotsize=3pt 0,dotstyle=*](11.656854249492381,4.65685424949238)
\rput[bl](11.74,4.78){$C$}
\rput[bl](6.46,2.06){$e$}
\rput[bl](4.34,3.36){$f$}
\psdots[dotsize=3pt 0,dotstyle=*](4.828427124746191,1.8284271247461896)
\rput[bl](5.58,-0.76){$\beta$}
\end{pspicture*}}

Gegeben sind die Seitenlänge $a$ und der Winkel $\beta$. Gib eine Formel an, mit der $f$ mithilfe von $a$ und $\beta$ berechnet werden kann.

$f=$ \antwort[\rule{3cm}{0.3pt}]{$2\cdot a\cdot\cos(\frac{\beta}{2})$}
\end{beispiel}