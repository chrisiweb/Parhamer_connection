\section{AG 4.1 - 21 - MAT - Drehkegel - OA - Matura 1.NT 2018/19}

\begin{beispiel}[AG 4.1]{1}
Gegeben ist ein Drehkegel mit einer Höhe von 6\,cm. Der Winkel zwischen der Kegelachse und der Erzeugenden (Mantellinie) beträgt $32^\circ$.

Berechne den Radius $r$ der Grundfläche des Drehkegels.\leer

$r\approx\antwort[\rule{3cm}{0.3pt}]{3,7}$\,cm

\antwort{Toleranzintervall: $[3,7;4,0]$}
\end{beispiel}