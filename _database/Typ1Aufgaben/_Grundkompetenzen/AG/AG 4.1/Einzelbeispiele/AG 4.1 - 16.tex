\section{AG 4.1 - 16 Sinkgeschwindigkeit - OA - Matura NT 1 16/17}

\begin{beispiel}[AG 4.1]{1} %PUNKTE DES BEISPIELS
Ein Kleinflugzeug befindet sich im Landeanflug mit einer Neigung von $\alpha$ (in Grad) zur Horizontalen. Es hat eine Eigengeschwindigkeit von $v$ (in m/s).

Gib eine Formel für den Höhenverlust $x$ (in m) an, den das Flugzeug bei dieser Neigung und dieser Eigengeschwindigkeit in einer Sekunde erfährt!

\antwort{$x=v\cdot\sin(\alpha)$}
\end{beispiel}