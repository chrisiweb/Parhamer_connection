\section{AG 4.1 - 15 Steigungswinkel - OA - Matura 2013/14 1. Nebentermin}

\begin{beispiel}[AG 4.1]{1} %PUNKTE DES BEISPIELS
				Gegeben ist eine Straße mit einer $7\,\%$-igen Steigung, d.h. auf einer horizontalen Entfernung von 100\,m gewinnt die Straße um 7\,m an Höhe.
				
				Gib eine Formel zur Berechnung des Gradmaßes des Steigungswinkels $\alpha$ dieser Straße an!
				
				\antwort{$\tan(\alpha)=\frac{7}{100}$
				
				oder
				
				$\alpha=\arctan(\frac{7}{100})$
				
				oder
				
				$\alpha=\tan^{-1}(\frac{7}{100})$}
\end{beispiel}