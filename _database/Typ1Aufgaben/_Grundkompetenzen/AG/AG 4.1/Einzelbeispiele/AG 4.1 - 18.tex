\section{AG 4.1 - 18 - Rechtwinkliges Dreieck - OA - Matura - 1. NT 2017/18}

\begin{beispiel}[AG 4.1]{1}
Die nachstehende Abbildung zeigt ein rechtwinkliges Dreieck.

\begin{center}
	\resizebox{0.5\linewidth}{!}{\psset{xunit=1.0cm,yunit=1.0cm,algebraic=true,dimen=middle,dotstyle=o,dotsize=5pt 0,linewidth=1.6pt,arrowsize=3pt 2,arrowinset=0.25}
\begin{pspicture*}(1.14,-0.56)(10.84,5.82)
\psline[linewidth=2.pt](2.,5.)(2.,0.)
\psline[linewidth=2.pt](2.,0.)(10.,0.)
\psline[linewidth=2.pt](10.,0.)(2.,5.)
\parametricplot{-1.5707963267948966}{-0.5585993153435624}{1.*cos(t)+2.|1.*sin(t)+5.}
\parametricplot{2.5829933382462307}{3.141592653589793}{1.4*cos(t)+10.|1.4*sin(t)+0.}
\parametricplot{0.0}{1.5707963267948966}{0.6*cos(t)+2.|0.6*sin(t)+0.}
\psellipse*[linewidth=1.pt,fillcolor=black,fillstyle=solid,opacity=1](2.2495670992423107,0.24956709924231057)(0.05,0.05)
\rput[tl](1.86,5.45){C}
\rput[tl](1.84,-0.15){A}
\rput[tl](10,-0.15){B}
\rput[tl](5.9,3.12){w}
\rput[tl](5.42,-0.15){x}
\rput[tl](1.58,2.74){y}
\rput[bl](2.15,4.3){$\gamma$}
\rput[bl](8.86,0.15){$\beta$}
\end{pspicture*}}
\end{center}

Gib einen Term zur Bestimmung der L�nge der Seite $w$ mithilfe von $x$ und $\beta$ an!\leer

$w=$\,\antwort[\rule{3cm}{0.3pt}]{$\frac{x}{\cos(\beta)}$}
\end{beispiel}