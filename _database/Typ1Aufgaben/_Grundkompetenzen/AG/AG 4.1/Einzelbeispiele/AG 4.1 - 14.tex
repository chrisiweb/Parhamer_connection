\section{AG 4.1 - 14 Definition der Winkelfunktion - MC - Matura 2013/14 Haupttermin}

\begin{beispiel}[AG 4.1]{1} %PUNKTE DES BEISPIELS
				Die nachstehende Abbildung zeigt ein rechtwinkeliges Dreieck $PQR$.
				
				\begin{center}
				\resizebox{0.6\linewidth}{!}{\psset{xunit=1.0cm,yunit=1.0cm,algebraic=true,dimen=middle,dotstyle=o,dotsize=5pt 0,linewidth=0.8pt,arrowsize=3pt 2,arrowinset=0.25}
\begin{pspicture*}(-10.72,6.5)(-4.22,11.08)
\psline(-10.,8.)(-8.,10.)
\psline(-10.,8.)(-5.,7.)
\psline(-8.,10.)(-5.,7.)
\parametricplot{-2.356194490192345}{-0.7853981633974483}{0.6*cos(t)+-8.|0.6*sin(t)+10.}
\psellipse*[fillcolor=black,fillstyle=solid,opacity=1](-8.,9.647058823529411)(0.04,0.04)
\parametricplot{-0.19739555984988078}{0.7853981633974483}{0.8*cos(t)+-10.|0.8*sin(t)+8.}
\parametricplot{2.356194490192345}{2.9441970937399122}{0.8*cos(t)+-5.|0.8*sin(t)+7.}

\begin{scriptsize}
\rput[tl](-8.1,10.3){R}
\rput[tl](-10.3,8.08){P}
\rput[tl](-4.8,6.92){Q}
\rput[tl](-6.5,9){p}
\rput[tl](-7.84,7.4){r}
\rput[tl](-9.3,9.2){q}
\rput[bl](-9.58,8.06){$\alpha$}
\rput[bl](-5.6,7.1){$\beta$}
\end{scriptsize}
\end{pspicture*}}
\end{center}

Kreuze jene beiden Gleichungen an, die f�r das dargestellte Dreieck gelten!\leer

\multiplechoice[5]{  %Anzahl der Antwortmoeglichkeiten, Standard: 5
				L1={$\sin\alpha=\frac{p}{r}$},   %1. Antwortmoeglichkeit 
				L2={$\sin\alpha=\frac{q}{r}$},   %2. Antwortmoeglichkeit
				L3={$\tan\beta=\frac{p}{q}$},   %3. Antwortmoeglichkeit
				L4={$\tan\alpha=\frac{r}{p}$},   %4. Antwortmoeglichkeit
				L5={$\cos\beta=\frac{p}{r}$},	 %5. Antwortmoeglichkeit
				L6={},	 %6. Antwortmoeglichkeit
				L7={},	 %7. Antwortmoeglichkeit
				L8={},	 %8. Antwortmoeglichkeit
				L9={},	 %9. Antwortmoeglichkeit
				%% LOESUNG: %%
				A1=1,  % 1. Antwort
				A2=5,	 % 2. Antwort
				A3=0,  % 3. Antwort
				A4=0,  % 4. Antwort
				A5=0,  % 5. Antwort
				}

\end{beispiel}