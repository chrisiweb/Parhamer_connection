\section{AG 4.1 - 19 - MAT - Viereck - OA - Matura 2. NT 2017/18}

\begin{beispiel}[AG 4.1]{1}
Gegeben ist das nachstehende Viereck $ABCD$ mit den Seitenlängen $a,b,c$ und $d$.

\begin{center}
\newrgbcolor{zzttqq}{0.6 0.2 0.}
\newrgbcolor{qqwuqq}{0. 0.39215686274509803 0.}
\psset{xunit=1.0cm,yunit=1.0cm,algebraic=true,dimen=middle,dotstyle=o,dotsize=5pt 0,linewidth=1.6pt,arrowsize=3pt 2,arrowinset=0.25}
\begin{pspicture*}(0.25,0.25)(6.5,6.5)
\multips(0,0)(0,1.0){7}{\psline[linestyle=dashed,linecap=1,dash=1.5pt 1.5pt,linewidth=0.4pt,linecolor=gray]{c-c}(0.16,0)(6.88,0)}
\multips(0,0)(1.0,0){7}{\psline[linestyle=dashed,linecap=1,dash=1.5pt 1.5pt,linewidth=0.4pt,linecolor=gray]{c-c}(0,0.24)(0,6.84)}
\pspolygon[linewidth=1.pt,linecolor=zzttqq,fillcolor=zzttqq,fillstyle=solid,opacity=0.1](1.,1.)(6.,1.)(6.,4.)(1.,6.)
\psline[linewidth=1.pt,linecolor=zzttqq](1.,1.)(6.,1.)
\psline[linewidth=1.pt,linecolor=zzttqq](6.,1.)(6.,4.)
\psline[linewidth=1.pt,linecolor=zzttqq](6.,4.)(1.,6.)
\psline[linewidth=1.pt,linecolor=zzttqq](1.,6.)(1.,1.)
\pscustom[linewidth=1.pt,linecolor=qqwuqq,fillcolor=qqwuqq,fillstyle=solid,opacity=0.1]{
\parametricplot{0.0}{1.5707963267948966}{0.6*cos(t)+1.|0.6*sin(t)+1.}
\lineto(1.,1.)\closepath}
\psellipse*[linewidth=1.pt,linecolor=qqwuqq,fillcolor=qqwuqq,fillstyle=solid,opacity=1](1.2495670992423111,1.249567099242311)(0.05,0.05)
\pscustom[linewidth=1.pt,linecolor=qqwuqq,fillcolor=qqwuqq,fillstyle=solid,opacity=0.1]{
\parametricplot{1.5707963267948966}{3.141592653589793}{0.6*cos(t)+6.|0.6*sin(t)+1.}
\lineto(6.,1.)\closepath}
\psellipse*[linewidth=1.pt,linecolor=qqwuqq,fillcolor=qqwuqq,fillstyle=solid,opacity=1](5.750432900757689,1.2495670992423111)(0.05,0.05)
\antwort{\psline[linewidth=1.pt,linecolor=red](1.,4.)(6.,4.)
\pscustom[linewidth=1.pt,linecolor=red,fillcolor=red,fillstyle=solid,opacity=0.1]{
\parametricplot{0.0}{1.5707963267948966}{0.6*cos(t)+1.|0.6*sin(t)+4.}
\lineto(1.,4.)\closepath}
\psellipse*[linewidth=1.pt,linecolor=red,fillcolor=red,fillstyle=solid,opacity=1](1.2495670992423111,4.249567099242311)(0.05,0.05)
\pscustom[linewidth=1.pt,linecolor=red,fillcolor=red,fillstyle=solid,opacity=0.28]{
\parametricplot{2.761086276477428}{3.141592653589793}{1*cos(t)+6.|1*sin(t)+4.}
\lineto(6.,4.)\closepath}}
\begin{scriptsize}
\psdots[dotsize=3pt 0,dotstyle=*](1.,1.)
\rput[bl](0.64,0.66){$A$}
\psdots[dotsize=3pt 0,dotstyle=*](6.,1.)
\rput[bl](6.08,1.12){$B$}
\psdots[dotsize=3pt 0,dotstyle=*](6.,4.)
\rput[bl](6.08,4.12){$C$}
\psdots[dotsize=3pt 0,dotstyle=*](1.,6.)
\rput[bl](1.08,6.12){$D$}
\rput[bl](3.3,0.66){\zzttqq{$a$}}
\rput[bl](6.3,2.5){\zzttqq{$b$}}
\rput[bl](3.6,5.3){\zzttqq{$c$}}
\rput[bl](0.66,3.5){\zzttqq{$d$}}
\antwort{\rput[bl](5.1,4.05){\red{$\varphi$}}}
\end{scriptsize}
\end{pspicture*}
\end{center}

Zeichne in der obigen Abbildung einen Winkel $\varphi$ ein, für den $\sin(\varphi)=\frac{d-b}{c}$ gilt!

\antwort{Ein Punkt für das Einzeichnen eines richtigen Winkels $\varphi$.}
\end{beispiel}