\section{AG 2.2 - 9 - MAT - Fahrtzeit - OA - Matura 2016/17 2. NT}

\begin{beispiel}{1} %PUNKTE DES BEISPIELS
Um 8:00 Uhr fährt ein Güterzug von Salzburg in Richtung Linz ab. Vom 124\,km entfernten Bahnhof Linz fährt eine halbe Stunde später ein Schnellzug Richtung Salzburg ab. Der Güterzug bewegt sich mit einer mittleren Geschwindigkeit von 100\,km/h, die mittlere Geschwindigkeit des Schnellzugs ist 150\,km/h. \leer

Mit $t$ wird die Fahrzeit des Güterzugs in Stunden bezeichnet, die bis zur Begegnung der beiden Züge vergeht. Gib eine Gleichung für die Berechnung der Fahrzeit $t$ des Güterzugs an und berechne diese Fahrzeit!

\antwort{Mögliche Gleichung:

$100\cdot t +150 \cdot (t-0,5)=124$ \\
$t=0,796 \Rightarrow t \approx 0,8 h$\leer

Toleranzintervall: [0,7 h; 0,8 h]}				
\end{beispiel}