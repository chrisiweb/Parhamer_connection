\section{AG 2.2 - 10 - MAT - Löwenrudel - MC - Matura 2018/19 2. NT}

\begin{beispiel}[AG 2.2]{1}
Ein Rudel von Löwen besteht aus Männchen und Weibchen. Die Anzahl der Männchen in diesem Rudel wird mit $m$ bezeichnet, jene der Weibchen mit $w$.

Die beiden nachstehenden Gleichungen enthalten Informationen über dieses Rudel.

$m+w=21$

$4\cdot m+1=w$

Kreuze die beiden Aussagen an, die auf dieses Rudel zutreffen.

\multiplechoice[5]{  %Anzahl der Antwortmoeglichkeiten, Standard: 5
				L1={In diesem Rudel sind mehr Männchen als Weibchen.},   %1. Antwortmoeglichkeit 
				L2={Die Anzahl der Weibchen ist mehr als viermal so groß wie die Anzahl der Männchen.},   %2. Antwortmoeglichkeit
				L3={Die Anzahl der Männchen ist um 1 kleiner als die Anzahl der Weibchen.},   %3. Antwortmoeglichkeit
				L4={Insgesamt sind mehr als 20 Löwen (Männchen und Weibchen) in diesem Rudel.},   %4. Antwortmoeglichkeit
				L5={Das Vierfache der Anzahl der Männchen ist um 1 größer als die Anzahl der Weibchen.},	 %5. Antwortmoeglichkeit
				L6={},	 %6. Antwortmoeglichkeit
				L7={},	 %7. Antwortmoeglichkeit
				L8={},	 %8. Antwortmoeglichkeit
				L9={},	 %9. Antwortmoeglichkeit
				%% LOESUNG: %%
				A1=2,  % 1. Antwort
				A2=4,	 % 2. Antwort
				A3=0,  % 3. Antwort
				A4=0,  % 4. Antwort
				A5=0,  % 5. Antwort
				}
\end{beispiel}