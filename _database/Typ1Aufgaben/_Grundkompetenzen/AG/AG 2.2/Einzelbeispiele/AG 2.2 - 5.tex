\section{AG 2.2 - 5 - Abgeschlossene Zahlenmengen - OA - MatKon}

\begin{beispiel}[AG 2.2]{1} %PUNKTE DES BEISPIELS
				Der seit 01.12.2012 gültige Taxitarif in Wien für eine Fahrt zwischen 6:00 und 23:00 Uhr kann bei einer Strecke bis zu 4 km mit einer linearen Funktion $f(x)$ dargestellt werden.
\begin{center}
$f(x)=1,05\cdot x+3,80$
\end{center}

Erkläre die Bedeutung der Faktoren 1,05 und 3,8 und berechne die Kosten für eine 4 km lange Fahrt.\\

\antwort{1,05 - Kosten pro gefahrenen Kilometer\\
3,8 - Grundgebühr, Startgeld, Grundtaxe\\
Kosten für eine 4 km lange Fahrt: $1,05\cdot 4+3,8=8$\euro}
\end{beispiel}