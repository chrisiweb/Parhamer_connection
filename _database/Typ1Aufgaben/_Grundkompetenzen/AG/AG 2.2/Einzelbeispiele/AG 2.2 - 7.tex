\section{AG 2.2 - 7 - MAT - Kapitalsparbuch - MC - Matura 2014/15 Haupttermin}

\begin{beispiel}[AG 2.2]{1} %PUNKTE DES BEISPIELS
Frau Fröhlich hat ein Kapitalsparbuch, auf welches sie jährlich am ersten Bank\-öffnungstag des Jahres den gleichen Geldbetrag in Euro einzahlt. An diesem Tag werden in dieser Bank auch die
Zinserträge des Vorjahres gutgeschrieben. Danach wird der neue Gesamtkontostand ausgedruckt. \leer

Zwischen dem Kontostand $K_{i-1}$ des Vorjahres und dem Kontostand $K_i$ des aktuellen Jahres besteht folgender Zusammenhang: 

$$K_i=1,03\cdot K_{i-1} + 5\,000$$

Welche der folgenden Aussagen sind in diesem Zusammenhang korrekt?

Kreuze die beiden zutreffenden Aussagen an.

\multiplechoice[5]{  %Anzahl der Antwortmoeglichkeiten, Standard: 5
				L1={Frau Fröhlich zahlt jährlich \euro\,5.000 auf ihr Kapitalsparbuch ein.},   %1. Antwortmoeglichkeit 
				L2={Das Kapital auf dem Kapitalsparbuch wächst jährlich um \euro\,5.000.},   %2. Antwortmoeglichkeit
				L3={Der relative jährliche Zuwachs des am Ausdruck ausgewiesenen Kapitals ist größer als 3\,\%.},   %3. Antwortmoeglichkeit
				L4={Die Differenz des Kapitals zweier aufeinanderfolgender Jahre ist immer dieselbe.},   %4. Antwortmoeglichkeit
				L5={Das Kapital auf dem Kapitalsparbuch wächst linear an.},	 %5. Antwortmoeglichkeit
				L6={},	 %6. Antwortmoeglichkeit
				L7={},	 %7. Antwortmoeglichkeit
				L8={},	 %8. Antwortmoeglichkeit
				L9={},	 %9. Antwortmoeglichkeit
				%% LOESUNG: %%
				A1=1,  % 1. Antwort
				A2=3,	 % 2. Antwort
				A3=0,  % 3. Antwort
				A4=0,  % 4. Antwort
				A5=0,  % 5. Antwort
				}
\end{beispiel}