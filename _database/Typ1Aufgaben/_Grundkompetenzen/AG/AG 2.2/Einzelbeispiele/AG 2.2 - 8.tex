\section{AG 2.2 - 8 Futtermittel - OA - Matura 2016/17 - Haupttermin}

\begin{beispiel}[AG 2.2]{1} %PUNKTE DES BEISPIELS
Ein Bauer hat zwei Sorten von Fertigfutter für die Rindermast gekauft. Fertigfutter $A$ hat einen Proteinanteil von 14\,\%, während Fertigfutter $B$ einen Proteinanteil von 35\,\% hat.
Der Bauer möchte für seine Jungstiere 100\,kg einer Mischung dieser beiden Fertigfutter-Sorten
mit einem Proteinanteil von 18\,\% herstellen. Es sollen $a$ kg der Sorte $A$ mit $b$ kg der Sorte $B$ gemischt werden.\leer

Gib zwei Gleichungen in den Variablen $a$ und $b$ an, mithilfe derer die für diese Mischung benötigten Mengen berechnet werden können!\leer


1. Gleichung: \antwort[\rule{6cm}{0.3pt}]{$a+b=100$} \leer

2. Gleichung: \antwort[\rule{6cm}{0.3pt}]{$0,14\cdot a+0,35\cdot b = 0,18\cdot (a+b)$}
\end{beispiel}