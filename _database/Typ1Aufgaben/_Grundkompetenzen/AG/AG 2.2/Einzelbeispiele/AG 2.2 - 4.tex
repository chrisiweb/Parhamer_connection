\section{AG 2.2 - 4 - MAT - Fahrenheit und Celsius - OA - Matura 2014/15 - Haupttermin}

\begin{beispiel}[AG 2.2]{1} %PUNKTE DES BEISPIELS
Während man in Europa die Temperatur in Grad Celsius (
$^\circ$\,C) angibt, verwendet man in den USA
die Einheit Grad Fahrenheit ($^\circ$\,F). Zwischen der Temperatur $T_F$ in $^\circ$\,F und der Temperatur $T_C$ in $^\circ$\,C besteht ein linearer Zusammenhang. \leer

Für die Umrechnung von $^\circ$\,F in $^\circ$\,C gelten folgende Regeln:
\begin{itemize}
	\item 32\,$^\circ$\,F entsprechen 0\,$^\circ$\,C.
	\item Eine Temperaturzunahme um 1\,$^\circ$\,F entspricht einer Zunahme der Temperatur um $\frac{5}{9}$\,$^\circ$\,C. 
\end{itemize}

Gib eine Gleichung an, die den Zusammenhang zwischen der Temperatur $T_F$ ($^\circ$\,F, Grad Fahrenheit) und der Temperatur $T_C$ ($^\circ$\,C, Grad Celsius) beschreibt.

\antwort{\leer

$T_C=(T_F-32)\cdot \frac{5}{9}$ 

oder:

$T_F=\frac{9}{5}\cdot T_C + 32$}

\end{beispiel}