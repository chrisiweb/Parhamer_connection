\section{AG 1.2 - 4 �quivalenzumformung - OA - Matura 2015/16 - Haupttermin}

\begin{beispiel}[AG 1.2]{1} %PUNKTE DES BEISPIELS
Nicht jede Umformung einer Gleichung ist eine �quivalenzumformung. \leer

Erkl�re konkret auf das unten angegebene Beispiel bezogen, warum es sich bei der durchgef�hrten
Umformung um keine �quivalenzumformung handelt! Die Grundmenge ist die Menge
der reellen Zahlen.

\begin{align*}
x^2 - 5x &= 0 \qquad |:x\\
x-5 &= 0
\end{align*}

\antwort{M�gliche Erkl�rung: \\
Die Gleichung $x^2 - 5x = 0$ hat die L�sungen $x_1 = 5$ und $x_2 = 0$ (die L�sungsmenge $L = \{0; ~5\}$). Die Gleichung $x - 5 = 0$ hat aber nur mehr die L�sung $x = 5$ (die L�sungsmenge $L = \{5\}$). Durch die durchgef�hrte Umformung wurde die L�sungsmenge ver�ndert, daher ist dies keine �quivalenzumformung. \leer

ODER: \leer

Bei der Division durch $x$ w�rde im Fall $x = 0$ durch null dividiert werden, was keine zul�ssige Rechenoperation ist.}

\end{beispiel}