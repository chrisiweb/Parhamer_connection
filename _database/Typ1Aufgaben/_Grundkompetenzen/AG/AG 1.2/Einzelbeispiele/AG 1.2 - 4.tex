\section{AG 1.2 - 4 - MAT - Äquivalenzumformung - OA - Matura 2015/16-Haupttermin}

\begin{beispiel}[AG 1.2]{1} %PUNKTE DES BEISPIELS
Nicht jede Umformung einer Gleichung ist eine Äquivalenzumformung. \leer

Erkläre konkret auf das unten angegebene Beispiel bezogen, warum es sich bei der durchgeführten
Umformung um keine Äquivalenzumformung handelt! Die Grundmenge ist die Menge
der reellen Zahlen.

\begin{align*}
x^2 - 5x &= 0 \qquad |:x\\
x-5 &= 0
\end{align*}

\antwort{Mögliche Erklärung: \\
Die Gleichung $x^2 - 5x = 0$ hat die Lösungen $x_1 = 5$ und $x_2 = 0$ (die Lösungsmenge $L = \{0; ~5\}$). Die Gleichung $x - 5 = 0$ hat aber nur mehr die Lösung $x = 5$ (die Lösungsmenge $L = \{5\}$). Durch die durchgeführte Umformung wurde die Lösungsmenge verändert, daher ist dies keine Äquivalenzumformung. \leer

ODER: \leer

Bei der Division durch $x$ würde im Fall $x = 0$ durch null dividiert werden, was keine zulässige Rechenoperation ist.}

\end{beispiel}