\section{AG 1.2 - 1 Oberfläche eines Zylinders - MC - BIFIE}

\begin{beispiel}[AG 1.2]{1} %PUNKTE DES BEISPIELS
				Für die Oberfläche $O$ eines Zylinders mit dem Radius $r$ und der Höhe $h$ gilt $O=2r^2\pi+2r\pi h$.
				
				Welche der folgenden Aussagen sind im Zusammenhang mit der gegebenen Formel zutreffend? Kreuze die zutreffende(n) Aussage(n) an!
				\multiplechoice[5]{  %Anzahl der Antwortmoeglichkeiten, Standard: 5
								L1={$O>2r^2\pi + r\pi h$ ist eine Formel.},   %1. Antwortmoeglichkeit 
								L2={$2r^2\pi + 2r\pi h$ ist ein Term.},   %2. Antwortmoeglichkeit
								L3={Jede Variable ist ein Term.},   %3. Antwortmoeglichkeit
								L4={$O=2r\pi \cdot \left(r+h\right)$ entsteht durch Umformung aus $O=2r^2\pi + 2r\pi h$.},   %4. Antwortmoeglichkeit
								L5={$\pi$ ist eine Variable.},	 %5. Antwortmoeglichkeit
								L6={},	 %6. Antwortmoeglichkeit
								L7={},	 %7. Antwortmoeglichkeit
								L8={},	 %8. Antwortmoeglichkeit
								L9={},	 %9. Antwortmoeglichkeit
								%% LOESUNG: %%
								A1=2,  % 1. Antwort
								A2=3,	 % 2. Antwort
								A3=4,  % 3. Antwort
								A4=0,  % 4. Antwort
								A5=0,  % 5. Antwort
								}
\end{beispiel}