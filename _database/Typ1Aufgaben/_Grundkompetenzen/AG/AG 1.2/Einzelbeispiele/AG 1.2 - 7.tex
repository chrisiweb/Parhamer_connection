\section{AG 1.2 - 7 - MAT - Äquivalente Gleichungen - MC - Matura 2018/19 2. NT}

\begin{beispiel}[AG 1.2]{1}
Gegeben ist die Gleichung $\frac{x}{2}-4=3$ in $x\in\mathbb{R}$.

Kreuze die beiden nachstehenden Gleichungen $x\in\mathbb{R}$ an, die zur gegebenen Gleichung äquivalent sind.

\multiplechoice[5]{  %Anzahl der Antwortmoeglichkeiten, Standard: 5
				L1={$x-4=6$},   %1. Antwortmoeglichkeit 
				L2={$\frac{x}{2}=-1$},   %2. Antwortmoeglichkeit
				L3={$\frac{x}{2}-3=4$},   %3. Antwortmoeglichkeit
				L4={$\frac{x-8}{2}=3$},   %4. Antwortmoeglichkeit
				L5={$\left(\frac{x}{2}-4\right)^2=9$},	 %5. Antwortmoeglichkeit
				L6={},	 %6. Antwortmoeglichkeit
				L7={},	 %7. Antwortmoeglichkeit
				L8={},	 %8. Antwortmoeglichkeit
				L9={},	 %9. Antwortmoeglichkeit
				%% LOESUNG: %%
				A1=3,  % 1. Antwort
				A2=4,	 % 2. Antwort
				A3=0,  % 3. Antwort
				A4=0,  % 4. Antwort
				A5=0,  % 5. Antwort
				}
\end{beispiel}