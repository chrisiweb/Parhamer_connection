\section{AG 1.2 - 2 �quivalenz - MC - BIFIE}

\begin{beispiel}[AG 1.2]{1} %PUNKTE DES BEISPIELS
Gegeben ist der Term $\frac{x}{2b}-\frac{y}{b}$ mit $b\neq 0$.
			
			Kreuze den/die zum gegebenen Term �quivalenten Term(e) an!
			\multiplechoice[5]{  %Anzahl der Antwortmoeglichkeiten, Standard: 5
							L1={$\frac{2x-y}{2b}$},   %1. Antwortmoeglichkeit 
							L2={$\frac{x-2y}{b}$},   %2. Antwortmoeglichkeit
							L3={$\frac{x-2y}{2b}$},   %3. Antwortmoeglichkeit
							L4={$\frac{x-y}{b}$},   %4. Antwortmoeglichkeit
							L5={$x-2y:2b$},	 %5. Antwortmoeglichkeit
							L6={},	 %6. Antwortmoeglichkeit
							L7={},	 %7. Antwortmoeglichkeit
							L8={},	 %8. Antwortmoeglichkeit
							L9={},	 %9. Antwortmoeglichkeit
							%% LOESUNG: %%
							A1=3,  % 1. Antwort
							A2=0,	 % 2. Antwort
							A3=0,  % 3. Antwort
							A4=0,  % 4. Antwort
							A5=0,  % 5. Antwort
							}
\end{beispiel}