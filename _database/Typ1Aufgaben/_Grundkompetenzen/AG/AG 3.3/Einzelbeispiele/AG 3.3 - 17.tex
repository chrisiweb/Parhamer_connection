\section{AG 3.3 - 17 Vektoraddition - OA- Matura 2013/14 1. Nebentermin}

\begin{beispiel}[AG 3.3]{1} %PUNKTE DES BEISPIELS
				Gegeben sind die beiden Vektoren $\vec{a}$ und $\vec{b}$.
				
				Stelle im untenstehenden Koordinatensystem den Vektor $\vec{s}$ mit $\vec{s}=2\cdot \vec{a}+\vec{b}$ als Pfeil dar!\leer
				
				\resizebox{0.8\linewidth}{!}{\psset{xunit=1.0cm,yunit=1.0cm,algebraic=true,dimen=middle,dotstyle=o,dotsize=5pt 0,linewidth=0.8pt,arrowsize=3pt 2,arrowinset=0.25}
\begin{pspicture*}(-5.68,-4.62)(5.92,5.84)
\multips(0,-4)(0,1.0){11}{\psline[linestyle=dashed,linecap=1,dash=1.5pt 1.5pt,linewidth=0.4pt,linecolor=lightgray]{c-c}(-5.68,0)(5.92,0)}
\multips(-5,0)(1.0,0){12}{\psline[linestyle=dashed,linecap=1,dash=1.5pt 1.5pt,linewidth=0.4pt,linecolor=lightgray]{c-c}(0,-4.62)(0,5.84)}
\psaxes[labelFontSize=\scriptstyle,xAxis=true,yAxis=true,Dx=1.,Dy=1.,ticksize=-2pt 0,subticks=2]{->}(0,0)(-5.68,-4.62)(5.92,5.84)[x,140] [y,-40]
\psline{->}(0.,0.)(1.,2.)
\psline{->}(0.,0.)(3.,-2.)
\rput[tl](0.2,1.38){$\vec{a}$}
\rput[tl](1.02,-0.88){$\vec{b}$}
\antwort{\psline{->}(1.,2.)(2.,4.)
\psline{->}(2.,4.)(5.,2.)
\psline{->}(0.,0.)(5.,2.)
\rput[tl](1.16,3.42){$\vec{a}$}
\rput[tl](3.28,3.74){$\vec{b}$}
\rput[tl](2.3,1){$\vec{s}=2\cdot\vec{a}+\vec{b}$}}
\end{pspicture*}}

\antwort{Ein Punkt f�r die richtige L�sung. Die L�sung ist dann als richtig zu werten, wenn der Vektor $\vec{s}=\binom{5}{2}$ richtig dargestellt ist. Die Spitze des Vektors $\vec{s}$ muss korrekt und klar erkennbar eingezeichnet sein. Als Ausgangspunkt kann ein beliebiger Punkt gew�hlt werden. Die Summanden m�ssen nicht dargestellt werden.
}
\end{beispiel}