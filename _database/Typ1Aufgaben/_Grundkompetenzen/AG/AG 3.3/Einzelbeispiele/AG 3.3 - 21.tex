\section{AG 3.3 - 21 - MAT - Kräfte - OA - Matura HT 2017/18}

\begin{beispiel}[AG 3.3]{1} %PUNKTE DES BEISPIELS
An einem Massenpunkt $M$ greifen drei Kräfte an. Diese sind durch die Vektoren $\vec{a}, \vec{b}$ und $\vec{c}$ gegeben.

Zeichne in der nachstehenden Abbildung einen Kraftvektor $\vec{d}$ so ein, dass die Summe aller vier Kräfte (in jeder Komponente) gleich null ist!\leer

\begin{center}
	\resizebox{1\linewidth}{!}{\psset{xunit=1.0cm,yunit=1.0cm,algebraic=true,dimen=middle,dotstyle=o,dotsize=5pt 0,linewidth=1.6pt,arrowsize=3pt 2,arrowinset=0.25}
\begin{pspicture*}(-8.9,-7.56)(7.74,6.58)
\multips(0,-7)(0,1.0){15}{\psline[linestyle=dashed,linecap=1,dash=1.5pt 1.5pt,linewidth=0.4pt,linecolor=darkgray]{c-c}(-8.9,0)(7.74,0)}
\multips(-8,0)(1.0,0){17}{\psline[linestyle=dashed,linecap=1,dash=1.5pt 1.5pt,linewidth=0.4pt,linecolor=darkgray]{c-c}(0,-7.56)(0,6.58)}
\psline[linewidth=2.pt]{->}(0.,0.)(-4.,-4.)
\rput[tl](-2.3,-1.54){$\vec{c}$}
\psline[linewidth=2.pt]{->}(0.,0.)(-2.,5.)
\rput[tl](-0.8,2.8){$\vec{b}$}
\psline[linewidth=2.pt]{->}(0.,0.)(3.,1.)
\rput[tl](1.24,0.92){$\vec{a}$}
\antwort{\psline[linewidth=2.pt]{->}(0.,0.)(3.,-2.)
\rput[tl](1.3,-1.16){$\vec{d}$}}
\begin{scriptsize}
\psdots[dotsize=5pt 0,dotstyle=*](0.,0.)
\rput[bl](-0.2,-0.5){$M$}
\end{scriptsize}
\end{pspicture*}}
\end{center}
\end{beispiel}