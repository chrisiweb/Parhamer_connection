\section{AG 3.3 - 2 - Vektoren in einem Quader - MC - BIFIE}

\begin{beispiel}[AG 3.3]{1} %PUNKTE DES BEISPIELS
			Die Grundfläche $ABCD$ des dargestellten Quaders liegt in der xy-Ebene. Festgelegt werden die Vektoren $\vek{a}$=$\vek{AB}$, $\vek{b}$=$\vek{AD}$, und $\vek{c}$=$\vek{AE}$.
			
\psset{xunit=1.0cm,yunit=1.0cm,algebraic=true,dimen=middle,dotstyle=o,dotsize=5pt 0,linewidth=0.8pt,arrowsize=3pt 2,arrowinset=0.25}
\begin{pspicture*}(-4.19796977603117,-5.2057235914983595)(10.749002494779102,5.824133255724308)
\psline(0.,5.)(0.,-1.)
\psline(0.,-1.)(-3.,-4.)
\psline[linestyle=dashed,dash=2pt 2pt](2.3737298875004713,-2.6262701124995287)(4.,-1.)
\psline(2.3737298875004713,-2.6262701124995287)(5.373729887500471,-2.6262701124995287)
\psline(5.373729887500471,-2.6262701124995287)(7.,-1.)
\psline[linewidth=0.4pt,linestyle=dashed,dash=2pt 2pt](0.,-1.)(4.,-1.)
\psline[linestyle=dashed,dash=2pt 2pt](4.,-1.)(7.,-1.)
\psline(5.373729887500471,0.37372988750047137)(7.,2.013470907465088)
\psline(2.3737298875004713,0.3737298875004713)(4.,2.013470907465088)
\psline(4.,2.013470907465088)(7.,2.013470907465088)
\psline(2.3737298875004713,0.3737298875004713)(5.373729887500471,0.37372988750047137)
\psline[linestyle=dashed,dash=2pt 2pt](4.,2.013470907465088)(4.,-1.)
\psline(2.3737298875004713,0.3737298875004713)(2.3737298875004713,-2.6262701124995287)
\psline(7.,2.013470907465088)(7.,-1.)
\psline(5.373729887500471,0.37372988750047137)(5.373729887500471,-2.6262701124995287)
\psline(7.,-1.)(10.,-1.)
\begin{scriptsize}
\psdots[dotsize=3pt 0,dotstyle=triangle*,linecolor=darkgray](0.,5.)
\psdots[dotsize=3pt 0,dotstyle=*,linecolor=darkgray](0.,-1.)
\rput[bl](-0.2936582587420367,3.545395469579347){z}
\psdots[dotsize=3pt 0,dotstyle=triangle*,dotangle=270,linecolor=darkgray](-3.,-4.)
\rput[bl](-2.7830356721777085,-3.3865631739876765){x}
\psdots[dotsize=3pt 0,dotstyle=*,linecolor=darkgray](4.,-1.)
\rput[bl](3.7084946597814668,-0.8205895324462921){\darkgray{D}}
\psdots[dotsize=3pt 0,dotstyle=*,linecolor=darkgray](2.3737298875004713,-2.6262701124995287)
\rput[bl](2.2148682117200633,-2.946134862379827){\darkgray{A}}
\psdots[dotsize=3pt 0,dotstyle=*,linecolor=darkgray](7.,-1.)
\rput[bl](7.0787286964328375,-0.8780367035255768){\darkgray{C}}
\psdots[dotsize=3pt 0,dotstyle=*,linecolor=darkgray](5.373729887500471,-2.6262701124995287)
\rput[bl](5.470207906212865,-2.9078367483269703){\darkgray{B}}
\psdots[dotsize=3pt 0,dotstyle=*,linecolor=darkgray](2.3737298875004713,0.3737298875004713)
\rput[bl](2.0808248125350657,0.1560123759015483){\darkgray{E}}
\psdots[dotsize=3pt 0,dotstyle=*,linecolor=darkgray](4.,2.013470907465088)
\rput[bl](4.072326743283603,2.1283652496236574){\darkgray{H}}
\psdots[dotsize=3pt 0,dotstyle=*,linecolor=darkgray](7.,2.013470907465088)
\rput[bl](7.0787286964328375,2.1283652496236574){\darkgray{G}}
\psdots[dotsize=3pt 0,dotstyle=*,linecolor=darkgray](5.373729887500471,0.37372988750047137)
\rput[bl](5.52765507729215,0.1177142618486918){\darkgray{F}}
\psdots[dotsize=3pt 0,dotstyle=triangle*,dotangle=270,linecolor=darkgray](10.,-1.)
\rput[bl](9.759596680132793,-1.2993159581069982){y}
\psdots[dotsize=3pt 0,dotstyle=*,linecolor=darkgray](7.,0.8836765429058215)
\rput[bl](7.174473981564979,0.768782200747252){\darkgray{T}}
\psdots[dotsize=3pt 0,dotstyle=*,linecolor=darkgray](6.108421749826094,-1.891578250173906)
\rput[bl](6.217021130243567,-2.237619752401982){\darkgray{R}}
\end{scriptsize}
\end{pspicture*}
Welche der folgenden Darstellungen ist/sind möglich, wenn $s,t \in \mathbb{R}$ gilt?
Kreuze die zutreffende(n) Aussage(n) an!
\multiplechoice[5]{  %Anzahl der Antwortmoeglichkeiten, Standard: 5
				L1={$\vek{TC}=t\cdot \vek{c}$},   %1. Antwortmoeglichkeit 
				L2={$\vek{AR}=t\cdot \vek{a}$},   %2. Antwortmoeglichkeit
				L3={$\vek{EG}=s\cdot \vek{a}+t\cdot \vek{b}$},   %3. Antwortmoeglichkeit
				L4={$\vek{BT}=s\cdot \vek{a}+t\cdot \vek{b}$},   %4. Antwortmoeglichkeit
				L5={$\vek{TR}=s\cdot \vek{b}+t\cdot \vek{c}$},	 %5. Antwortmoeglichkeit
				L6={},	 %6. Antwortmoeglichkeit
				L7={},	 %7. Antwortmoeglichkeit
				L8={},	 %8. Antwortmoeglichkeit
				L9={},	 %9. Antwortmoeglichkeit
				%% LOESUNG: %%
				A1=1,  % 1. Antwort
				A2=3,	 % 2. Antwort
				A3=5,  % 3. Antwort
				A4=0,  % 4. Antwort
				A5=0,  % 5. Antwort
				}
\end{beispiel}