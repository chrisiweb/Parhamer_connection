\section{AG 3.3 - 5 Vektoren (als Pfeile) - OA - BIFIE}

\begin{beispiel}[AG 3.3]{1} %PUNKTE DES BEISPIELS
Gegeben sind die Vektoren $\vek{a}$ und $\vek{b}$, die in der untenstehenden Abbildung als Pfeile dargestellt sind.

Stelle $\frac{1}{2}\cdot\vek{b}-\vek{a}$ ausgehend vom Punkt $C$ durch einen Pfeil dar!

\begin{center}

\newrgbcolor{cqcqcq}{0.7529411764705882 0.7529411764705882 0.7529411764705882}
\newrgbcolor{sqsqsq}{0.12549019607843137 0.12549019607843137 0.12549019607843137}
\newrgbcolor{vvvvvv}{0.3333333333333333 0.3333333333333333 0.3333333333333333}
\psset{xunit=1.0cm,yunit=1.0cm,algebraic=true,dimen=middle,dotstyle=o,dotsize=5pt 0,linewidth=0.8pt,arrowsize=3pt 2,arrowinset=0.25}
\begin{pspicture*}(-3.32,-2.18)(9.18,4.14)
\multips(0,-2)(0,1.0){7}{\psline[linestyle=dashed,linecap=1,dash=1.5pt 1.5pt,linewidth=0.4pt,linecolor=darkgray]{c-c}(-3.32,0)(9.18,0)}
\multips(-3,0)(1.0,0){13}{\psline[linestyle=dashed,linecap=1,dash=1.5pt 1.5pt,linewidth=0.4pt,linecolor=darkgray]{c-c}(0,-2.18)(0,4.14)}
\psline[linecolor=sqsqsq](-1.,0.)(0.,2.)
\psline[linecolor=sqsqsq](-1.,0.)(3.,0.)
\rput[tl](-1.14,1.7){$\overrightarrow{a}$}
\rput[tl](1.16,-0.2){$\overrightarrow{b}$}
\rput[bl](-1.32,-0.4){A}
\psdots[dotsize=3pt 0,dotstyle=triangle*,dotangle=90](0.,2.)
\rput[bl](-0.12,2.26){D}
\psdots[dotsize=3pt 0,dotstyle=triangle*,dotangle=270](3.,0.)
\rput[bl](3.06,-0.42){B}
\psdots[dotsize=3pt 0,dotstyle=*](4.,2.)
\rput[bl](3.98,2.22){C}
\antwort{\psline[linecolor=sqsqsq](4.,2.)(6.,2.)
\psline[linecolor=sqsqsq](6.,2.)(5.,0.)
\psline[linecolor=vvvvvv](4.,2.)(5.,0.)
\psdots[dotsize=3pt 0,dotstyle=triangle*,dotangle=206](5,0.)
\rput[tl](4.58,2.69){$\frac{1}{2}\cdot\overrightarrow{b}$}
\rput[tl](5.66,1.26){$-\overrightarrow{a}$}
\rput[tl](2.42,1.42){$\frac{1}{2}\cdot\overrightarrow{b}-\overrightarrow{a}$}
\begin{scriptsize}
\psdots[dotsize=3pt 0,dotstyle=*](-1.,0.)
\end{scriptsize}}
\end{pspicture*}
\end{center}
\end{beispiel}