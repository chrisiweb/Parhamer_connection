\section{AG 3.3 - 11 Gehälter - OA - Matura 2014/15 - Haupttermin}

\begin{beispiel}[AG 3.3]{1} %PUNKTE DES BEISPIELS
Die Gehälter der 8 Mitarbeiter/innen eines Kleinunternehmens sind im Vektor $G=\begin{pmatrix}
	G_1 \\
	G_2 \\
	\vdots \\
	G_3
\end{pmatrix}$ dargestellt. \leer

Gib an,

was der Ausdruck (das Skalarprodukt) \footnotesize $G\cdot \begin{pmatrix}
	1\\
	1\\
	1\\
	1\\
	1\\
	1\\
	1\\
	1\\
	\end{pmatrix}$ \normalsize in diesem Kontext \mbox{bedeutet}. 


\antwort{Der Ausdruck gibt die Summe der Gehälter der 8 Mitarbeiter/innen des Kleinunternehmens an.}
\end{beispiel}