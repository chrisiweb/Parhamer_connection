\section{AG 3.3 - 22 - Verkaufszahlen - ZO - Matura - 1. NT 2017/18}

\begin{beispiel}[AG 3.3]{1}
Ein Sportfachgesch�ft bietet $n$ erschiedene Sportartikel an. Die $n$ Sportartikel sind in einer Datenbank nach ihrer Artikelnummer geordnet, sodass die Liste mit den entsprechenden St�ckzahlen als Vektor (mit $n$ Komponenten) aufgefasst werden kann.

Die Vektoren $B, C$ und $P$ (mit $B, C, P\in\mathbb{R}^n$) haben die folgende Bedeutung:

Vektor $B$: Die Komponente $b_i\in\mathbb{N}$ (mit $1\leq i\leq n)$ gibt den Lagerbestand des $i$-ten Artikels am Montagmorgen einer bestimmten Woche an.\\
Vektor $C$: Die Komponente $c_i\in\mathbb{N}$ (mit $1\leq i\leq n)$ gibt den Lagerbestand des $i$-ten Artikels am Samstagabend dieser Woche an.\\
Vektor $P$: Die Komponente $p_i\in\mathbb{N}$ (mit $1\leq i\leq n)$ gibt den St�ckpreis (in Euro) des $i$-ten Artikels in dieser Woche an.

Das Fachgesch�ft ist in der betrachteten Woche von Montag bis Samstag ge�ffnet und im Laufe dieser Woche werden weder Sportartikel nachgeliefert noch St�ckpreise ver�ndert.

Am Ende der Woche werden Daten f�r die betrachtete Woche (Montag bis Samstag) ausgewertet, wobei die erforderlichen Berechnungen mithilfe von Termen angeschrieben werden k�nnen. Ordne den vier gesuchten Gr��en jeweils den f�r die Berechnung zutreffenden Term (aus A bis F) zu!

\zuordnen[0.15]{
				R1={durchschnittliche Verkaufszahlen (pro Sportartikel) pro Tag in der betrachteten Woche},				% Response 1
				R2={Gesamteinnahmen durch den Verkauf von Sportartikeln in der betrachteten Woche},				% Response 2
				R3={Verkaufszahlen (pro Sportartikel) in der betrachteten Woche},				% Response 3
				R4={Verkaufswert des Lagerbestands an Sportartikeln am Ende der betrachteten Woche},				% Response 4
				%% Moegliche Zuordnungen: %%
				A={$6\cdot(B-C)$}, 				%Moeglichkeit A  
				B={$B-C$}, 				%Moeglichkeit B  
				C={$\frac{1}{6}\cdot(B-C)$}, 				%Moeglichkeit C  
				D={$P\cdot C$}, 				%Moeglichkeit D  
				E={$P\cdot(B-C)$}, 				%Moeglichkeit E  
				F={$6\cdot P\cdot(B-C)$}, 				%Moeglichkeit F  
				%% LOESUNG: %%
				A1={C},				% 1. richtige Zuordnung
				A2={E},				% 2. richtige Zuordnung
				A3={B},				% 3. richtige Zuordnung
				A4={D},				% 4. richtige Zuordnung
				}
\end{beispiel}