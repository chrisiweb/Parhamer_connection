\section{AG 3.3 - 19 - MAT - Vektoren in der Ebene - OA - Matura NT 1 16/17}

\begin{beispiel}[AG 3.3]{1} %PUNKTE DES BEISPIELS
Die unten stehende Abbildung zeigt zwei Vektoren $\vec{a}$ und $\vec{b}$.

Zeichne in die Abbildung einen Vektor $\vec{c}$ so ein, dass die Summe der drei Vektoren den Nullvektor ergibt, also $\vec{a}+\vec{b}+\vec{c}=\Vek{0}{0}{}$ gilt!

\begin{center}
	\resizebox{1\linewidth}{!}{\psset{xunit=1.0cm,yunit=1.0cm,algebraic=true,dimen=middle,dotstyle=o,dotsize=5pt 0,linewidth=1.6pt,arrowsize=3pt 2,arrowinset=0.25}
\begin{pspicture*}(4.82,-6.56)(16.8,2.76)
\multips(0,-6)(0,1.0){10}{\psline[linestyle=dashed,linecap=1,dash=1.5pt 1.5pt,linewidth=0.4pt,linecolor=darkgray]{c-c}(4.82,0)(16.8,0)}
\multips(4,0)(1.0,0){12}{\psline[linestyle=dashed,linecap=1,dash=1.5pt 1.5pt,linewidth=0.4pt,linecolor=darkgray]{c-c}(0,-6.56)(0,2.76)}
\psline[linewidth=2.pt]{->}(11.,-1.)(11.,-4.)
\psline[linewidth=2.pt]{->}(11.,-1.)(8.,1.)
\begin{scriptsize}
\rput[bl](11.2,-2.5){$\vec{a}$}
\rput[bl](9.66,0.12){$\vec{b}$}
\antwort{\psline[linewidth=2.pt]{->}(11.,-1.)(14.,0.)
\rput[bl](12.46,-0.4){$\vec{c}$}}
\end{scriptsize}
\end{pspicture*}}
\end{center}
\end{beispiel}