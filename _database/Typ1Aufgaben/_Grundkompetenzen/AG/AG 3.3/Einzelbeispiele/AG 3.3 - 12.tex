\section{AG 3.3 - 12 - MAT - Vektoren - OA - Matura 2014/15 Nebentermin 1}

\begin{beispiel}[AG 3.3]{1} %PUNKTE DES BEISPIELS
In der unten stehenden Abbildung sind die Vektoren $\vek{a}$, $\vek{b}$ und $\vek{c}$ als Pfeile dargestellt.

Stelle den Vektor $\vek{d}=\vek{a}+\vek{b}-2\cdot \vek{c}$ als Pfeil dar.\leer

\begin{center}
\psset{xunit=0.9cm,yunit=0.9cm,algebraic=true,dimen=middle,dotstyle=o,dotsize=5pt 0,linewidth=0.8pt,arrowsize=3pt 2,arrowinset=0.25}
\begin{pspicture*}(-2.48815737701865,-4.351853465789232)(7.431850458270732,5.765989283594234)
\multips(0,-4)(0,1.0){11}{\psline[linestyle=dashed,linecap=1,dash=1.5pt 1.5pt,linewidth=0.4pt,linecolor=gray]{c-c}(-2.48815737701865,0)(7.431850458270732,0)}
\multips(-2,0)(1.0,0){10}{\psline[linestyle=dashed,linecap=1,dash=1.5pt 1.5pt,linewidth=0.4pt,linecolor=gray]{c-c}(0,-4.351853465789232)(0,5.765989283594234)}
\psline{->}(1.,-1.)(0.,2.)
\psline{->}(1.,-1.)(4.,-2.)
\psline{->}(1.,-1.)(0.,-3.)
\antwort{\psline[linecolor=red]{->}(0.,2.)(3.,1.)
\psline[linecolor=red]{->}(3.,1.)(4.,3.)
\psline[linecolor=red]{->}(4.,3.)(5.,5.)
\psline[linewidth=2.5pt,linecolor=red]{->}(1.,-1.)(5.,5.)}
\begin{scriptsize}
\rput[bl](0.6489391179019239,0.6222815171478905){$\vek{a}$}
\rput[bl](2.344666952994126,-1.2147569708686607){$\vek{b}$}
\rput[bl](0.1402207673742633,-1.8365238437358011){$\vek{c}$}
\antwort{\rput[bl](1.3554923825236749,1.781028871127561){\red{$\vek{b}$}}
\rput[bl](3.672987090483018,1.667980348788081){\red{$-\vek{c}$}}
\rput[bl](4.71868592212321,3.505018836804632){\red{$-\vek{c}$}}
\rput[bl](2.6,2.2614850910703512){\red{$\vek{d}$}}}
\end{scriptsize}
\end{pspicture*}
\end{center}
\end{beispiel}