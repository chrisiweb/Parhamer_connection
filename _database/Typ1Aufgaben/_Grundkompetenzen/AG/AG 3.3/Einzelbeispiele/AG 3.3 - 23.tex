\section{AG 3.3 - 23 - MAT - Darstellung im Koordinatensystem - OA - Matura 1.NT 2018/19}

\begin{beispiel}[AG 3.3]{1}
Im nachstehenden Koordinatensystem sind der Vektor $\vec{v}$ sowie die Punkte $A$ und $B$ dargestellt. Die Komponenten des dargestellten Vektors $\vec{v}$ und die Koordinaten der beiden Punkte $A$ und $B$ sind ganzzahlig.

\begin{center}
\psset{xunit=1.0cm,yunit=1.0cm,algebraic=true,dimen=middle,dotstyle=o,dotsize=5pt 0,linewidth=1.6pt,arrowsize=3pt 2,arrowinset=0.25}
\begin{pspicture*}(-3.5,-1.5)(9.5,9.5)
\multips(0,-1)(0,1.0){11}{\psline[linestyle=dashed,linecap=1,dash=1.5pt 1.5pt,linewidth=0.4pt,linecolor=gray]{c-c}(-3.5,0)(9.5,0)}
\multips(-3,0)(1.0,0){14}{\psline[linestyle=dashed,linecap=1,dash=1.5pt 1.5pt,linewidth=0.4pt,linecolor=gray]{c-c}(0,-1.5)(0,9.5)}
\psaxes[showorigin=false,xAxis=true,yAxis=true,Dx=1.,Dy=1.,ticksize=-2pt 0,subticks=2]{->}(0,0)(-3.5,-1.5)(9.5,9.5)[$x$,140] [$y$,-40]
\psline[linewidth=2.pt]{->}(1.,3.)(3.,2.)
\psdots[dotstyle=*](8.,3.)
\rput[bl](8.08,3.2){$A$}
\psdots[dotstyle=*](-2.,8.)
\rput[bl](-1.92,8.2){$B$}
\rput[bl](2.02,2.7){$\vec{v}$}
\end{pspicture*}
\end{center}

Bestimmen Sie den Wert des Parameters $t$ so, dass die Gleichung $B=A+t\cdot \vec{v}$ erfüllt ist.\leer

$t=\antwort[\rule{5cm}{0.3pt}]{-5}$
\end{beispiel}