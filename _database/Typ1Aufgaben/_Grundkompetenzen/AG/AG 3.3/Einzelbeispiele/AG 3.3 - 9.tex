\section{AG 3.3 - 9 Vegetarische Menüs - OA - BIFIE}

\begin{beispiel}[AG 3.3]{1} %PUNKTE DES BEISPIELS
			In einem Restaurant wird täglich ein vegetarisches Menü angeboten. Der Vektor
				
				
	\footnotesize			\begin{center}
				$\vek{a}=\left(\begin{array}{c}a_1\\ a_2\\ a_3 \\ a_4 \\ a_5 \\ a_6 \\ a_6 \\ a_7\end{array}\right)$
				\end{center}
\normalsize
				
gibt die Anzahl der verkauften vegetarischen Menüs an den Wochentagen Montag bis Sonntag einer bestimmten Woche an, der Vektor
\footnotesize	
\begin{center}
$\vek{p}=\left(\begin{array}{c}p_1\\ p_2\\ \vdots \\ p_7\end{array}\right)$
\end{center}
 \normalsize
die jeweiligen Menüpreise in Euro.

\leer

Interpretiere das Skalarprodukt $\vek{a}\cdot \vek{p}$ in diesem Zusammenhang!

\antwort{Das Skalarprodukt gibt den Erlös aus dem Verkauf des vegetarischen Menüs für die Tage Montag bis Sonntag in dieser Woche an.} 				
\end{beispiel}