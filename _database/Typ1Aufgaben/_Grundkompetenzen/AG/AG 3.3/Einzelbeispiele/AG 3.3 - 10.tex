\section{AG 3.3 - 10 - MAT - Vektoraddition - OA - Matura 2015/16 Haupttermin}

\begin{beispiel}[AG 3.3]{1} %PUNKTE DES BEISPIELS
Die unten stehende Abbildung zeigt zwei Vektoren $\vek{v_1}$ und $\vek{v}$. \leer

Ergänze in der Abbildung einen Vektor $\vek{v_2}$ so, dass $\vek{v_1} + \vek{v_2} = \vek{v}$ ist.


\begin{center}
\resizebox{0.8\linewidth}{!}{
\psset{xunit=1.0cm,yunit=1.0cm,algebraic=true,dimen=middle,dotstyle=o,dotsize=5pt 0,linewidth=0.8pt,arrowsize=3pt 2,arrowinset=0.25}
\begin{pspicture*}(-1.857009398887046,-2.943858604734674)(5.628576492809803,5.2644735109880845)
\multips(0,-2)(0,1.0){9}{\psline[linestyle=dashed,linecap=1,dash=1.5pt 1.5pt,linewidth=0.4pt,linecolor=gray]{c-c}(-2.1,0)(5.628576492809803,0)}
\multips(-1,0)(1.0,0){8}{\psline[linestyle=dashed,linecap=1,dash=1.5pt 1.5pt,linewidth=0.4pt,linecolor=gray]{c-c}(0,-2.943858604734674)(0,5.2644735109880845)}
\psline{->}(-1.,-2.)(4.,-2.)
\psline{->}(4.,-2.)(4.,4.)
\antwort{\psline[linecolor=red]{->}(4.,-2.)(-1.,4.)}
\begin{scriptsize}
\rput[bl](1.4469733395170803,-2.5){$\vek{v_1}$}
\rput[bl](4.234708775045562,0.6354560285364658){$\vek{v}$}
\antwort{\rput[bl](1.2232661749376343,0.5){\red{$\vek{v_2}$}}}
\end{scriptsize}
\end{pspicture*}}
\end{center}

\antwort{Lösungsschlüssel: Ein Punkt für eine korrekte Darstellung von $\vek{v_2}$, wobei der gesuchte Vektor auch von anderen Ausgangspunkten aus gezeichnet werden kann.}
\end{beispiel}