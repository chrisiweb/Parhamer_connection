\section{AG 3.3 - 7 Rechteck - MC - BIFIE}

\begin{beispiel}[AG 3.3]{1} %PUNKTE DES BEISPIELS
Abgebildet ist das Rechteck $RSTU$.
\begin{center}
\newrgbcolor{cqcqcq}{0.7529411764705882 0.7529411764705882 0.7529411764705882}
\newrgbcolor{sqsqsq}{0.12549019607843137 0.12549019607843137 0.12549019607843137}
\psset{xunit=1.0cm,yunit=1.0cm,algebraic=true,dimen=middle,dotstyle=o,dotsize=5pt 0,linewidth=0.8pt,arrowsize=3pt 2,arrowinset=0.25}
\begin{pspicture*}(-4.3,-0.1)(3.28,6.3)
\multips(0,0)(0,1.0){7}{\psline[linestyle=dashed,linecap=1,dash=1.5pt 1.5pt,linewidth=0.4pt,linecolor=lightgray]{c-c}(-4.3,0)(3.28,0)}
\multips(-4,0)(1.0,0){8}{\psline[linestyle=dashed,linecap=1,dash=1.5pt 1.5pt,linewidth=0.4pt,linecolor=lightgray]{c-c}(0,-0.1)(0,6.3)}
\psline[linecolor=sqsqsq](-2.,5.)(-3.,3.)
\psline[linecolor=sqsqsq](-3.,3.)(1.,1.)
\psline[linecolor=sqsqsq](-2.,5.)(2.,3.)
\psline[linecolor=sqsqsq](2.,3.)(1.,1.)
\begin{scriptsize}
\psdots[dotsize=3pt 0,dotstyle=*](-2.,5.)
\rput[bl](-1.92,5.12){U}
\psdots[dotsize=3pt 0,dotstyle=*](-3.,3.)
\rput[bl](-3.28,2.58){R}
\psdots[dotsize=3pt 0,dotstyle=*](1.,1.)
\rput[bl](0.68,0.54){S}
\psdots[dotsize=3pt 0,dotstyle=*](2.,3.)
\rput[bl](2.08,3.12){T}
\end{scriptsize}
\end{pspicture*}
\end{center}
Kreuze die beiden zutreffenden Aussagen an!
\multiplechoice[5]{  %Anzahl der Antwortmoeglichkeiten, Standard: 5
				L1={$\vek{ST}=-\vek{RU}$},   %1. Antwortmoeglichkeit 
				L2={$\vek{SR}\parallel\vek{UT}$},   %2. Antwortmoeglichkeit
				L3={$\vek{RS}+\vek{ST}=\vek{TR}$},   %3. Antwortmoeglichkeit
				L4={$U=T+\vek{SR}$},   %4. Antwortmoeglichkeit
				L5={$\vek{RT}\cdot\vek{SU}=0$},	 %5. Antwortmoeglichkeit
				L6={},	 %6. Antwortmoeglichkeit
				L7={},	 %7. Antwortmoeglichkeit
				L8={},	 %8. Antwortmoeglichkeit
				L9={},	 %9. Antwortmoeglichkeit
				%% LOESUNG: %%
				A1=2,  % 1. Antwort
				A2=4,	 % 2. Antwort
				A3=0,  % 3. Antwort
				A4=0,  % 4. Antwort
				A5=0,  % 5. Antwort
				}
\end{beispiel}