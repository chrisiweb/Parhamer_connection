\section{AG 3.3 - 14 Geometrisches Rechnen mit Vektoren - OA - Matura 2014/15 - Kompensationspr�fung}

\begin{beispiel}[AG 3.3]{1} %PUNKTE DES BEISPIELS
				Gegeben sind die Pfeildarstellungen der vier Vektoren $\vek{a},\vek{b},\vek{c},\vek{d}\in\mathbb{R}^{2}$ und ein Punkt $P$.
				\begin{center}
					\resizebox{0.8\linewidth}{!}{\psset{xunit=1.0cm,yunit=1.0cm,algebraic=true,dimen=middle,dotstyle=o,dotsize=5pt 0,linewidth=0.8pt,arrowsize=3pt 2,arrowinset=0.25}
\begin{pspicture*}(-4.88,-3.42)(10.72,6.94)
\multips(0,-3)(0,1.0){11}{\psline[linestyle=dashed,linecap=1,dash=1.5pt 1.5pt,linewidth=0.4pt,linecolor=lightgray]{c-c}(-4.88,0)(10.72,0)}
\multips(-4,0)(1.0,0){16}{\psline[linestyle=dashed,linecap=1,dash=1.5pt 1.5pt,linewidth=0.4pt,linecolor=lightgray]{c-c}(0,-3.42)(0,6.94)}
\psline{->}(-4.,4.)(0.,2.)
\psline{->}(-3.,0.)(-1.,1.)
\psline{->}(1.,-1.)(1.,1.)
\psline{->}(3.,1.)(2.,4.)
\antwort{\psline{->}(7.,1.)(5.,6.)
\psline[linecolor=red]{->}(7.,1.)(5.,6.)}
\begin{large}
\rput[bl](-1.98,3.08){$\vek{d}$}
\rput[bl](-2.04,0.58){$\vek{a}$}
\rput[bl](1.1,-0.04){$\vek{b}$}
\rput[bl](2.64,2.5){$\vek{c}$}
\psdots[dotsize=3pt 0,dotstyle=*](7.,1.)
\rput[bl](6.84,0.52){$P$}
\end{large}
\end{pspicture*}}
				\end{center}
				
				Ermittle in der gegebenen Abbildung ausgehend vom Punkt $P$ grafisch die Pfeildarstellung des Vektors $2\cdot\vek{b}-\frac{1}{2}\cdot\vek{d}$.
\end{beispiel}