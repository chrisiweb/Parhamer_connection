\section{AG 3.3 - 3 - Rechnen mit Vektoren - MC - BIFIE}

\begin{beispiel}[AG 3.3]{1} %PUNKTE DES BEISPIELS
Gegeben sind die Vektoren $\vek{r}, \vek{s}$, und $\vek{t}$.

\newrgbcolor{sqsqsq}{0.12549019607843137 0.12549019607843137 0.12549019607843137}
\psset{xunit=1.0cm,yunit=1.0cm,algebraic=true,dimen=middle,dotstyle=o,dotsize=5pt 0,linewidth=0.8pt,arrowsize=3pt 2,arrowinset=0.25}
\begin{pspicture*}(1.04,-1.36)(12.14,5.96)
\psaxes[labelFontSize=\scriptstyle,xAxis=false,yAxis=false,Dx=1.,Dy=1.,ticksize=-2pt 0,subticks=2]{->}(0,0)(1.04,-5.36)(12.14,5.96)
\psline[linecolor=sqsqsq](4.,5.)(2.,0.)
\psline[linecolor=sqsqsq](2.,0.)(11.,0.)
\psline[linecolor=sqsqsq](11.,0.)(4.,5.)
\rput[tl](2.36,3.22){$\overrightarrow{r}$}
\rput[tl](5.62,-0.32){$\overrightarrow{t}$}
\rput[tl](7.46,3.4){$\overrightarrow{s}$}
\begin{scriptsize}
\psdots[dotsize=3pt 0,dotstyle=triangle*,dotangle=270](4.,5.)
\psdots[dotsize=3pt 0,dotstyle=triangle*,dotangle=270](2.,0.)
\psdots[dotsize=3pt 0,dotstyle=triangle*,dotangle=270](11.,0.)
\end{scriptsize}
\end{pspicture*}

Kreuze die beiden für diese Vektoren zutreffenden Aussagen an!
\multiplechoice[5]{  %Anzahl der Antwortmoeglichkeiten, Standard: 5
				L1={$\vek{t}+\vek{s}+\vek{r}=\vek{0}$},   %1. Antwortmoeglichkeit 
				L2={$\vek{t}+\vek{s}=-\vek{r}$},   %2. Antwortmoeglichkeit
				L3={$\vek{t}-\vek{s}=\vek{r}$},   %3. Antwortmoeglichkeit
				L4={$\vek{t}-\vek{r}=\vek{s}$},   %4. Antwortmoeglichkeit
				L5={$\vek{t}=\vek{s}+\vek{r}$},	 %5. Antwortmoeglichkeit
				L6={},	 %6. Antwortmoeglichkeit
				L7={},	 %7. Antwortmoeglichkeit
				L8={},	 %8. Antwortmoeglichkeit
				L9={},	 %9. Antwortmoeglichkeit
				%% LOESUNG: %%
				A1=1,  % 1. Antwort
				A2=2,	 % 2. Antwort
				A3=0,  % 3. Antwort
				A4=0,  % 4. Antwort
				A5=0,  % 5. Antwort
				}
 -\end{beispiel}