\section{AG 3.3 - 20 - MAT - Orthogonale Vektoren - OA - Matura 2016/17 2. NT}

\begin{beispiel}{1} %PUNKTE DES BEISPIELS
Gegeben sind die nachstehend angeführten Vektoren:

$\vec{a}=\Vek{2}{3}{}$ 

$\vec{b}=\Vek{x}{0}{}~,x\in \mathbb R$

$\vec{c}=\Vek{1}{-2}{}$

$\vec{d}=\vec{a}-\vec{b}$\leer

Berechne $x$ so, dass die Vektoren $\vec{c}$ und $\vec{d}$ aufeinander normal stehen!

\antwort{$\vec{d}\cdot \vec{c}=0 \Rightarrow (2-x)-6=0 \Rightarrow x=-4$}				
\end{beispiel}