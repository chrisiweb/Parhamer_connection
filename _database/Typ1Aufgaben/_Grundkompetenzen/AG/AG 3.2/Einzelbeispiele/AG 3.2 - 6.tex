\section{AG 3.2 - 6 - MAT - Eckpunkte eines Quaders - OA - Matura-HT-18/19}

\begin{beispiel}[AG 3.2]{1}
In der nachstehenden Abbildung ist ein Quader dargestellt. Die Eckpunkte $A,B,C$ und $E$ sind beschriftet.

\begin{center}
\psset{xunit=1.0cm,yunit=1.0cm,algebraic=true,dimen=middle,dotstyle=o,dotsize=5pt 0,linewidth=1.6pt,arrowsize=3pt 2,arrowinset=0.25}
\begin{pspicture*}(0.5515074171275933,-0.4758415250000178)(8.178649936781305,6.435710964205638)
\psline[linewidth=0.8pt](1.,1.)(6.,0.)
\psline[linewidth=0.8pt,linestyle=dashed,dash=4pt 4pt](1.,1.)(2.5,2.)
\psline[linewidth=0.8pt](6.,0.)(7.5,1.)
\psline[linewidth=0.8pt,linestyle=dashed,dash=4pt 4pt](2.5,2.)(7.5,1.)
\psline[linewidth=0.8pt](1.,1.)(1.,5.)
\psline[linewidth=0.8pt](1.,5.)(2.5,6.)
\psline[linewidth=0.8pt,linestyle=dashed,dash=4pt 4pt](2.5,6.)(2.5,2.)
\psline[linewidth=0.8pt](7.5,1.)(7.5,5.)
\psline[linewidth=0.8pt](6.,0.)(6.,4.)
\psline[linewidth=0.8pt](6.,4.)(7.5,5.)
\psline[linewidth=0.8pt](2.5,6.)(7.5,5.)
\psline[linewidth=0.8pt](1.,5.)(6.,4.)
\begin{scriptsize}
\psdots[dotsize=1pt 0,dotstyle=*](1.,1.)
\rput[bl](0.7434949862721948,0.7808043821282833){$A$}
\psdots[dotsize=1pt 0,dotstyle=*](6.,0.)
\rput[bl](5.647904707147923,-0.23149370972507038){$B$}
\psdots[dotsize=1pt 0,dotstyle=*](7.5,1.)
\rput[bl](7.567780398593937,1.0426056127800127){$C$}
\psdots[dotsize=1pt 0,dotstyle=*](1.,5.)
\rput[bl](0.7260415708954128,5.144158226323773){$E$}
\psdots[dotsize=2pt 0,dotstyle=x,linecolor=red](2.5,6.)
\antwort{
\rput[bl](2.4364762778200437,6.156456318177127){\red{$S$}}
\psdots[dotsize=2pt 0,dotstyle=x,linecolor=red](6.,4.)
\rput[bl](5.787532030162178,4.166766965223983){\red{$R$}}
\psdots[dotsize=2pt 0,dotstyle=x,linecolor=red](2.5,2.)
\rput[bl](2.5761036008342995,2.0723571200101483){\red{$T$}}}
\end{scriptsize}
\end{pspicture*}
\end{center}

Für weitere Eckpunkte $R, S$ und $T$ des Quaders gilt:

$R= E+ \vec{AB}$ \leer

$S=A+\vec{AE}+ \vec{BC}$ \leer

$T=E+\vec{BC} - \vec{AE}$ \leer

Beschrifte in der oben stehenden Abbildung klar erkennbar die Eckpunkte $R, S$ und $T$!
\end{beispiel}