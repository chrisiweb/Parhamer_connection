\section{AG 3.2 - 4 - MAT - Vektoren - OA - Matura 2015/16 Nebentermin 1}

\begin{beispiel}[AG 3.2]{1} %PUNKTE DES BEISPIELS
In der Ebene werden auf einer Geraden in gleichen Abständen nacheinander die Punkte $A,B,C$ und $D$ markiert.

Es gilt also:

$\vek{AB}=\vek{BC}=\vek{CD}$

Die Koordinaten der Punkte $A$ und $C$ sind bekannt.

$A=(3|1)$ \\
$C=(7|8)$ 

Berechne die Koordinaten von $D$.\leer

$D=(\rule{1cm}{0.3pt}|\rule{1cm}{0.3pt})$

\antwort{$\vek{AC}=\Vek{4}{7}{}$

$D=C+\frac{1}{2}\cdot \vek{AC} \Rightarrow D=(9|11,5)$

Lösungsschlüssel:

Ein Punkt für die korrekte Angabe beider Koordinaten des gesuchten Punktes $D$.}
\end{beispiel}