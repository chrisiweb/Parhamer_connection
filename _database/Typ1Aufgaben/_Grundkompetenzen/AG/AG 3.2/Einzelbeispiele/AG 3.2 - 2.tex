\section{AG 3.2 - 2 Parallelogramm - OA - BIFIE}

\begin{beispiel}[AG 3.2]{1} %PUNKTE DES BEISPIELS
Im dargestellten Parallelogramm $ABCD$ teilt der Punkt $F$ die Seite $BC$ im Verhältnis $1:2$.
\begin{center}
\newrgbcolor{uuuuuu}{0.26666666666666666 0.26666666666666666 0.26666666666666666}
\psset{xunit=1.0cm,yunit=1.0cm,algebraic=true,dimen=middle,dotstyle=o,dotsize=5pt 0,linewidth=0.8pt,arrowsize=3pt 2,arrowinset=0.25}
\begin{pspicture*}(-3.5,0.1)(5.7,4.84)
\psline(-3.,1.)(3.,1.)
\psline(3.,1.)(5.,4.)
\psline(5.,4.)(-1.,4.)
\psline(-1.,4.)(-3.,1.)
\begin{scriptsize}
\psdots[dotsize=3pt 0,dotstyle=*,linecolor=darkgray](-3.,1.)
\rput[bl](-3.24,0.64){\darkgray{$A$}}
\psdots[dotsize=3pt 0,dotstyle=*,linecolor=darkgray](3.,1.)
\rput[bl](3.1,0.58){\darkgray{$B$}}
\psdots[dotsize=3pt 0,dotstyle=*,linecolor=darkgray](5.,4.)
\rput[bl](5.08,4.12){\darkgray{$C$}}
\psdots[dotsize=3pt 0,dotstyle=*,linecolor=darkgray](-1.,4.)
\rput[bl](-0.92,4.12){\darkgray{$D$}}
\psdots[dotsize=3pt 0,dotstyle=*,linecolor=darkgray](3.665640235470275,1.9984603532054124)
\rput[bl](3.8,1.6){\darkgray{F}}
\end{scriptsize}
\end{pspicture*}
\end{center}
Drücke den Vektor $\vek{FD}$ durch die Vektoren $\vek{a}=\vek{AB}$ und $\vek{b}=\vek{BC}$ aus.
\leer

$\vek{FD}=$ \antwort[\rule{3cm}{0.3pt}]{$\frac{2}{3}\vek{b}-\vek{a}$}
\end{beispiel}