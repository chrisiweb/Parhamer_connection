\section{AG 3.2 - 3 Resultierende Kraft - OA - BIFIE}

\begin{beispiel}[AG 3.2]{1} %PUNKTE DES BEISPIELS
Drei an einem Punkt P eines K�rpers angreifende Kr�fte $\vek{F_{1}}, \vek{F_{2}}$ und $\vek{F_{3}}$ lassen sich durch eine einzige, am selben Punkt angreifende resultierende Kraft $\vek{F}$ ersetzen, die alleine dieselbe Wirkung aus�bt, wie es $\vek{F_{1}}, \vek{F_{2}}$ und $\vek{F_{3}}$ zusammen tun.

Gegeben sind drei an einem Punkt $P$ angreifende Kr�fte $\vek{F_{1}}, \vek{F_{2}}$ und $\vek{F_{3}}$.
Ermittle grafisch die resultierende Kraft $\vek{F}$ als Summe der Kr�fte $\vek{F_{1}}, \vek{F_{2}}$ und $\vek{F_{3}}$!

\newrgbcolor{cqcqcq}{0.7529411764705882 0.7529411764705882 0.7529411764705882}
\newrgbcolor{uuuuuu}{0.26666666666666666 0.26666666666666666 0.26666666666666666}
\psset{xunit=1.0cm,yunit=1.0cm,algebraic=true,dimen=middle,dotstyle=o,dotsize=5pt 0,linewidth=0.8pt,arrowsize=3pt 2,arrowinset=0.25}
\begin{pspicture*}(-4.94,-1.32)(10.28,8.18)
\multips(0,-1)(0,1.0){10}{\psline[linestyle=dashed,linecap=1,dash=1.5pt 1.5pt,linewidth=0.4pt,linecolor=lightgray]{c-c}(-4.94,0)(10.28,0)}
\multips(-4,0)(1.0,0){16}{\psline[linestyle=dashed,linecap=1,dash=1.5pt 1.5pt,linewidth=0.4pt,linecolor=lightgray]{c-c}(0,-1.32)(0,8.18)}
\psline(0.,1.)(-3.,4.)
\psline(0.,1.)(3.,0.)
\psline(0.,1.)(4.,3.)
\psdots[dotsize=3pt 0,dotstyle=*](0.,1.)
\psdots[dotsize=3pt 0,dotstyle=triangle*,dotangle=270](-3.,4.)
\psdots[dotsize=3pt 0,dotstyle=triangle*](3.,0.)
\psdots[dotsize=3pt 0,dotstyle=triangle*,dotangle=180](4.,3.)
\rput[tl](-2.36,2.72){$\overrightarrow{F_{2}}$}
\rput[tl](1.22,0.24){$\overrightarrow{F_{1}}$}
\rput[tl](2.68,2.14){$\overrightarrow{F_{3}}$}
\rput[bl](-0.18,0.56){P}
\antwort{
\psline[linestyle=dashed,dash=1pt 3pt 5pt 3pt ](0.,1.)(3.9999567915386613,4.997177524115774)
\psline(3.,0.)(6.998089543769623,1.9990447718848117)
\psline(6.998089543769623,1.9990447718848117)(3.9999567915386613,4.997177524115774)
\rput[tl](5.26,1.00){$\overrightarrow{F_{3}}$}
\rput[tl](5.56,4.34){$\overrightarrow{F_{2}}$}
\rput[tl](1.34,3.88){$\overrightarrow{F}$}
\begin{scriptsize}
\psdots[dotsize=3pt 0,dotstyle=triangle*,dotangle=180](6.998089543769623,1.9990447718848117)
\psdots[dotsize=3pt 0,dotstyle=*,linecolor=darkgray](9.996222296000584,-0.9990879803461485)
\psdots[dotsize=3pt 0,dotstyle=triangle*,dotangle=270](3.9999567915386613,4.997177524115774)
\end{scriptsize}}
\end{pspicture*}
\end{beispiel}