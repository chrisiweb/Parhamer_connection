\section{AG 2.3 - 12 Quadratische Gleichung - LT - Matura 2013/14 Haupttermin}

\begin{beispiel}[AG 2.3]{1} %PUNKTE DES BEISPIELS
				Die Anzahl der Lösungen der quadratischen Gleichung $rx²+sx+t=0$ in der Menge der reellen Zahlen hängt von den Koeffizienten $r,s$ und $t$ ab.
				
				\lueckentext{
								text={Die quadratische Gleichung $rx^2+sx+t=0$ hat genau dann \textbf{für alle} \mbox{$r\neq 0;r,s,t\in\mathbb{R}$} \gap , wenn \gap gilt.}, 	%Lueckentext Luecke=\gap
								L1={zwei reelle Lösungen}, 		%1.Moeglichkeit links  
								L2={keine relle Lösung}, 		%2.Moeglichkeit links
								L3={genau eine relle Lösung}, 		%3.Moeglichkeit links
								R1={$r^2-4st>0$}, 		%1.Moeglichkeit rechts 
								R2={$t^2=4rs$}, 		%2.Moeglichkeit rechts
								R3={$s^2-4rt>0$}, 		%3.Moeglichkeit rechts
								%% LOESUNG: %%
								A1=1,   % Antwort links
								A2=3		% Antwort rechts 
								}

\end{beispiel}