\section{AG 2.3 - 5 - Quadratische Gleichungen  - ZO - BIFIE}

\begin{beispiel}[AG 2.3]{1} %PUNKTE DES BEISPIELS
	Quadratische Gleichungen können in der Menge der reellen Zahlen keine, genau eine oder zwei verschiedene Lösungen haben.
				
\leer

Ordnen Sie jeder Lösungsmenge $L$ die entsprechende quadratische Gleichung in der Menge der reellen Zahlen zu!

\zuordnen{
				title1={Lösungsmenge}, 		%Titel Antwortmoeglichkeiten
				A={$(x+4)^2=0$}, 				%Moeglichkeit A  
				B={$(x-4)^2=25$}, 				%Moeglichkeit B  
				C={$x(x-4)=0$}, 				%Moeglichkeit C  
				D={$-x^2=16$}, 				%Moeglichkeit D  
				E={$x^2-16=0$}, 				%Moeglichkeit E  
				F={$x^2-8x+16=0$}, 				%Moeglichkeit F  
				title2={Quadratische Gleichung},		%Titel Zuordnung
				R1={$L=\{\}$},				%1. Antwort rechts
				R2={$L=\{-4;4\}$},				%2. Antwort rechts
				R3={$L=\{0;4\}$},				%3. Antwort rechts
				R4={$L=\{4\}$},				%4. Antwort rechts
				%% LOESUNG: %%
				A1={D},				% 1. richtige Zuordnung
				A2={E},				% 2. richtige Zuordnung
				A3={C},				% 3. richtige Zuordnung
				A4={F},				% 4. richtige Zuordnung
				}
\end{beispiel}