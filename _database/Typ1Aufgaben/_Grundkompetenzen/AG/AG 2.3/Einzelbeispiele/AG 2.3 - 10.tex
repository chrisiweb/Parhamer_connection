\section{AG 2.3 - 10 - Benzinverbrauch - OA - BIFIE}

\begin{beispiel}[AG 2.3]{1} %PUNKTE DES BEISPIELS
Der Zusammenhang zwischen dem Benzinverbrauch y (in L/100\,km) und der Geschwindigkeit $x$ (in km/h) kann für einen bestimmten Autotyp durch die Funktionsgleichung
$y = 0,0005 \cdot x^2- 0,09 \cdot x + 10$ beschrieben werden. 
\leer

Ermittle rechnerisch, bei welcher Geschwindigkeit bzw. welchen Geschwindigkeiten der Verbrauch 6\,L/100\,km beträgt!

\antwort{$6=0,0005 \cdot x^2 - 0,09 \cdot x +10$
$0=x^2-180\cdot x +8\,000$ \leer

$x_{1,2}=90 \pm \sqrt{8\,100 -8\,000}=90\pm 10$ \\
$x_1=80, x_2=100$ \leer

Bei 80\,km/h und bei 100\,km/h beträgt der Benzinverbrauch 6\,L/100\,km.} 				
\end{beispiel}