\section{AG 2.3 - 19 - MAT - Quadratische Gleichung - ZO - Matura 2018/19 2. NT}

\begin{beispiel}[AG 2.3]{1}
Gegeben ist die quadratische Gleichung $x^2+r\cdot x+2=0$ in $x\in\mathbb{R}$ mit $r,s\in\mathbb{R}$.

Ordne den vier Lösungsfällen jeweils diejenige Aussage über die Parameter $r$ und $s$ (aus A bis F) zu, bei der stets der jeweiligen Lösungsfall vorliegt.

\zuordnen{
				R1={Die quadratische Gleichung hat keine reelle Lösung.},				% Response 1
				R2={Die quadratische Gleichung hat nur eine reelle Lösung $x=-\frac{r}{2}$.},				% Response 2
				R3={Die quadratische Gleichung hat die reellen Lösungen $x_1=0$ und\\ 
				$x_2=-r$.},				% Response 3
				R4={Die quadratische Gleichung hat die reellen Lösungen $x_1=-\sqrt{-s}$ und $x_2=\sqrt{-s}$.},				% Response 4
				%% Moegliche Zuordnungen: %%
				A={$\frac{r^2}{4}=s$}, 				%Moeglichkeit A  
				B={$\frac{r^2}{4}-s>0$ mit $r,s\neq 0$}, 				%Moeglichkeit B  
				C={$r\in\mathbb{R}$, $s>0$}, 				%Moeglichkeit C  
				D={$r=0$, $s<0$}, 				%Moeglichkeit D  
				E={$r\neq 0$, $s=0$}, 				%Moeglichkeit E  
				F={$r=0$, $s>0$}, 				%Moeglichkeit F  
				%% LOESUNG: %%
				A1={F},				% 1. richtige Zuordnung
				A2={A},				% 2. richtige Zuordnung
				A3={E},				% 3. richtige Zuordnung
				A4={D},				% 4. richtige Zuordnung
				}
				
				\antwort{Lösungsschlüssel:
				
				Ein Punkt ist genau dann zu geben, wenn jedem der vier Lösungsfälle ausschließlich der laut
Lösungserwartung richtige Buchstabe zugeordnet ist. Bei zwei oder drei richtigen Zuordnungen ist ein halber Punkt zu geben.}
\end{beispiel}