\section{AG 2.3 - 2 - Quadratische Gleichung - LT - BIFIE}

\begin{beispiel}[AG 2.3]{1} %PUNKTE DES BEISPIELS
		Gegeben ist eine quadratische Gleichung der Form
				
				\[x^2+px+q=0 \quad \text{mit } p,q \in \mathbb{R}\]
				
\lueckentext{
				text={Die quadratische Gleichung hat jedenfalls für x \gap in $\mathbb{R}$, wenn \gap gilt.}, 	%Lueckentext Luecke=\gap
				L1={keine Lösung}, 		%1.Moeglichkeit links  
				L2={genau eine Lösung}, 		%2.Moeglichkeit links
				L3={zwei Lösungen}, 		%3.Moeglichkeit links
				R1={$p \neq 0$ und $q<0$}, 		%1.Moeglichkeit rechts 
				R2={$p=q$}, 		%2.Moeglichkeit rechts
				R3={$p<0$ und $q>0$}, 		%3.Moeglichkeit rechts
				%% LOESUNG: %%
				A1=3,   % Antwort links
				A2=1		% Antwort rechts 
				}				
\end{beispiel}