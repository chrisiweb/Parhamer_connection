\section{AG 2.3 - 9 L�sungsf�lle - MC - Matura 2014/15 - Nebentermin 1}

\begin{beispiel}[AG 2.3]{1} %PUNKTE DES BEISPIELS
				Gegeben sind f�nf Gleichungen in der Unbekannten $x$.\\
				
				Welche dieser Gleichungen besitzt/besitzen zumindest eine reelle L�sung?
				
				Kreuze die zutreffende(n) Gleichung(en) an!
				
				\multiplechoice[5]{  %Anzahl der Antwortmoeglichkeiten, Standard: 5
								L1={$2x=2x+1$},   %1. Antwortmoeglichkeit 
								L2={$x=2x$},   %2. Antwortmoeglichkeit
								L3={$x^2+1=0$},   %3. Antwortmoeglichkeit
								L4={$x^2=-x$},   %4. Antwortmoeglichkeit
								L5={$x^3=-1$},	 %5. Antwortmoeglichkeit
								L6={},	 %6. Antwortmoeglichkeit
								L7={},	 %7. Antwortmoeglichkeit
								L8={},	 %8. Antwortmoeglichkeit
								L9={},	 %9. Antwortmoeglichkeit
								%% LOESUNG: %%
								A1=2,  % 1. Antwort
								A2=4,	 % 2. Antwort
								A3=5,  % 3. Antwort
								A4=0,  % 4. Antwort
								A5=0,  % 5. Antwort
								}
\end{beispiel}