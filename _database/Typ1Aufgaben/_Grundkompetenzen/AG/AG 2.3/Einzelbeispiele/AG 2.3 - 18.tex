\section{AG 2.3 - 18 - MAT - Anhalteweg - OA - Matura-HT-18/19}

\begin{beispiel}[AG 2.3]{1}
Schülerinnen und Schüler einer Fahrschule lernen die nachstehende Formel für die näherungsweise Berechnung des Anhaltewegs s. Dabei ist v die Geschwindigkeit des Fahrzeugs (s in m, v in km/h).

$s=\frac{v}{10}\cdot 3 + \left(\frac{v}{10}\right)^2$

Bei "`Fahren auf Sicht"' muss man jederzeit die Geschwindigkeit so wählen, dass man innerhalb
der Sichtweite anhalten kann. "`Sichtweite"' bezeichnet dabei die Länge des Streckenabschnitts,
den man sehen kann.\leer

Berechne die maximal zulässige Geschwindigkeit bei einer Sichtweite von 25\,m!


Die maximal zulässige Geschwindigkeit beträgt $\approx \antwort[\rule{4cm}{0.3pt}]{37,2}$\, km/h.


\antwort{\subsubsection{Lösungserwartung:}
mögliche Vorgehensweise:

$25=\frac{v}{10}\cdot 3 +\left(\frac{v}{10}\right)^2$

$v^2+30\cdot v -2500 =0$

$v_1=-15+\sqrt{2725} \approx 37,2 \quad \left(v_2=-15-\sqrt{2725}\right)$\leer

Toleranzintervall: $[37; 38]$

}
\end{beispiel}