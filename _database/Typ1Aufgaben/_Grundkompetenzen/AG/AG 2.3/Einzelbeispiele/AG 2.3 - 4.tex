\section{AG 2.3 - 4 Graphische Lösung einer quadratischen Gleichung  - LT - BIFIE}

\begin{beispiel}[AG 2.3]{1} %PUNKTE DES BEISPIELS
			Der Graph der Polynomfunktion $f$ mit $f(x)=x^2+px+q$ berührt die x-Achse. Welcher Zusammenhang besteht dann zwischen den Parametern $p$ und $q$?
				
\lueckentext{
				text={Es gibt in diesem Fall \gap mit der x-Achse, deshalb gilt \mbox{\gap.}}, 	%Lueckentext Luecke=\gap
				L1={keinen Schnittpunkt}, 		%1.Moeglichkeit links  
				L2={einen Schnittpunkt}, 		%2.Moeglichkeit links
				L3={zwei Schnittpunkte}, 		%3.Moeglichkeit links
				R1={$\dfrac{p^2}{4}=q$}, 		%1.Moeglichkeit rechts 
				R2={$\dfrac{p^2}{4}<q$}, 		%2.Moeglichkeit rechts
				R3={$\dfrac{p^2}{4}>q$}, 		%3.Moeglichkeit rechts
				%% LOESUNG: %%
				A1=2,   % Antwort links
				A2=1		% Antwort rechts 
				}
\end{beispiel}