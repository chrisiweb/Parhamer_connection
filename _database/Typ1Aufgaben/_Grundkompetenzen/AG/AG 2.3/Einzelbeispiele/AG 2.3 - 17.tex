\section{AG 2.3 - 17 - MAT - Lösungsmenge einer quadratischen Gleichung - OA - Matura 1. NT 2017/18}

\begin{beispiel}[AG 2.3]{1}
Gegeben ist eine quadratische Gleichung der Form $x^2+a\cdot x=0$ in $x$ mit $a\in\mathbb{R}$.

Bestimme denjenigen Wert für $a$, für den die gegebene Gleichung die Lösungsmenge $L=\left\{0;\frac{6}{7}\right\}$ hat!\leer

$a=$\,\antwort[\rule{3cm}{0.3pt}]{$-\frac{6}{7}$}
\end{beispiel}