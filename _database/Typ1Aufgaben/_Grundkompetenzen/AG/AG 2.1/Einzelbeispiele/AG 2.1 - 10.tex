\section{AG 2.1 - 10 - Kegelstumpf - OA - BIFIE}

\begin{beispiel}[AG 2.1]{1} %PUNKTE DES BEISPIELS
Ein 15\,cm hohes Gefäß hat die Form eines geraden Kegelstumpfes. Der Radius am Boden hat eine Länge von 20\,cm, der Radius mit der kleinsten Länge beträgt 11\,cm. 

\begin{center}
\resizebox{0.5\linewidth}{!}{
				\psset{xunit=1.0cm,yunit=1.0cm,algebraic=true,dimen=middle,dotstyle=o,dotsize=3pt 0,linewidth=0.8pt,arrowsize=3pt 2,arrowinset=0.25}
\begin{pspicture*}(-2.24616227935,-0.453625327739)(2.13764118638,2.4216340042)
\psline(1.,2.)(2.,0.)
\psline(0.,2.)(0.,0.)
\psline(0.,2.)(1.,2.)
\psline(0.,0.)(2.,0.)
\psline(-1.,2.)(-2.,0.)
\rput{0.}(0.,2.){\psellipse(0,0)(1.,0.2)}
\psplot[linestyle=dashed,dash=3pt 3pt,plotpoints=200]{-1.9999985633372181}{1.9999984624700529}{sqrt(10.0000000000)*1.00000000000/20.0000000000*sqrt(-x^(2.00000000000)+4.00000000000)}
\psplot[plotpoints=200]{-1.9999985633372181}{1.9999984624700529}{-sqrt(10.0000000000)*1.00000000000/20.0000000000*sqrt(-x^(2.00000000000)+4.00000000000)}
\rput[tl](0.796713067454,-0.05){\tiny{20}}
\rput[tl](0.0617813099638,1.01623818724){\tiny{15}}
\rput[tl](0.2,1.975){\tiny{11}}
\end{pspicture*}}
\end{center}


Gib eine Formel für die Länge $r(h)$ in Abhängigkeit von der Höhe $h$ an! 


\antwort{$r(h)=-0,6\cdot h +20$}
\end{beispiel}