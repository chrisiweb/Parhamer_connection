\section{AG 2.1 - 14 - MAT - Anzahl der Heizungstage - OA - Matura 2014/15 Nebentermin 2}

\begin{beispiel}[AG 2.1]{1} %PUNKTE DES BEISPIELS
				Die Anzahl der Heizungstage, für die ein Vorrat an Heizöl in einem Tank reicht, ist indirekt proportional zum durchschnittlichen Tagesverbrauch $x$ (in Litern).
				
				In einem Tank befinden sich $1500$ Liter Heizöl. Gib einen Term an, der die Anzahl $d(x)$ der Heizungstage in Abhängigkeit vom durchschnittlichen Tagesverbrauch $x$ bestimmt.\\
				
				$d(x)=\antwort[\rule{5cm}{0.3pt}]{\frac{1500}{x}}$
\end{beispiel}