\section{AG 2.1 - 4 Angestellte Frauen und Männer - MC - BIFIE}

\begin{beispiel}[AG 2.1]{1} %PUNKTE DES BEISPIELS
Für die Anzahl $x$ der in einem Betrieb angestellten Frauen und die Anzahl y der im selben Betrieb angestellten Männer kann man folgende Aussagen machen:
\begin{itemize}
	\item Die Anzahl der in diesem Betrieb angestellten Männer ist um 94 größer als jene der Frauen.
	\item Es sind dreimal so viele Männer wie Frauen im Betrieb angestellt.
\end{itemize}
	
	\leer
Kreuzen Sie diejenigen beiden Gleichungen an, die die oben angeführten Aussagen über die Anzahl der Angestellten mathematisch korrekt wiedergeben! 	
	
	\multiplechoice[5]{  %Anzahl der Antwortmoeglichkeiten, Standard: 5
					L1={$x-y=94$},   %1. Antwortmoeglichkeit 
					L2={$3x=94$},   %2. Antwortmoeglichkeit
					L3={$3x=y$},   %3. Antwortmoeglichkeit
					L4={$3y=x$},   %4. Antwortmoeglichkeit
					L5={$y-x=94$},	 %5. Antwortmoeglichkeit
					L6={},	 %6. Antwortmoeglichkeit
					L7={},	 %7. Antwortmoeglichkeit
					L8={},	 %8. Antwortmoeglichkeit
					L9={},	 %9. Antwortmoeglichkeit
					%% LOESUNG: %%
					A1=3,  % 1. Antwort
					A2=5,	 % 2. Antwort
					A3=0,  % 3. Antwort
					A4=0,  % 4. Antwort
					A5=0,  % 5. Antwort
					}
\end{beispiel}