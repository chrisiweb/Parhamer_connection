\section{AG 2.1 - 21 - Umformen - MC - MK}

\begin{beispiel}[AG 2.1]{1}
Gegeben ist folgender Term:
				$$s=\frac{t+u}{t+v}\cdot w^2$$
				
				Welche dieser sechs Umformungen ist richtig? Kreuze die richtige Aussage an!
				\langmultiplechoice[6]{  %Anzahl der Antwortmoeglichkeiten, Standard: 5
								L1={$w=s\cdot\dfrac{t+v}{t+u}$},   %1. Antwortmoeglichkeit 
								L2={$u=\dfrac{s\cdot (t+v)}{w^2}+t$},   %2. Antwortmoeglichkeit
								L3={$v=\dfrac{t+u\cdot w^2}{s}-t$},   %3. Antwortmoeglichkeit
								L4={$s=\dfrac{t+u\cdot w^2}{t+v}$},   %4. Antwortmoeglichkeit
								L5={$u=\dfrac{s\cdot (t+v)}{w^2}-t$},	 %5. Antwortmoeglichkeit
								L6={$t=(s-w^2)\cdot(w^2u-sv)$},	 %6. Antwortmoeglichkeit
								L7={},	 %7. Antwortmoeglichkeit
								L8={},	 %8. Antwortmoeglichkeit
								L9={},	 %9. Antwortmoeglichkeit
								%% LOESUNG: %%
								A1=5,  % 1. Antwort
								A2=0,	 % 2. Antwort
								A3=0,  % 3. Antwort
								A4=0,  % 4. Antwort
								A5=0,  % 5. Antwort
								}
\end{beispiel}