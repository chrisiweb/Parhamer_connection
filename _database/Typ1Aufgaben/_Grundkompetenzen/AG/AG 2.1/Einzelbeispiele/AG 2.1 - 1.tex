\section{AG 2.1 - 1 �quivalenz von Formeln - MC - BIFIE}

\begin{beispiel}[AG 2.1]{1} %PUNKTE DES BEISPIELS
Die nachstehende Abbildung zeigt ein Trapez. 

\begin{center}
\newrgbcolor{qqwuqq}{0. 0.39215686274509803 0.}
\psset{xunit=1.0cm,yunit=1.0cm,algebraic=true,dimen=middle,dotstyle=o,dotsize=5pt 0,linewidth=1.6pt,arrowsize=3pt 2,arrowinset=0.25}
\begin{pspicture*}(0.26,0.4)(7.44,5.28)
\pspolygon[linewidth=1.2pt,fillcolor=black,fillstyle=solid,opacity=0.1](1.,1.)(7.,1.)(3.5,4.5)(1.,4.5)
\psline[linewidth=1.2pt](1.,1.)(7.,1.)
\psline[linewidth=1.2pt](7.,1.)(3.5,4.5)
\psline[linewidth=1.2pt](3.5,4.5)(1.,4.5)
\psline[linewidth=1.2pt](1.,4.5)(1.,1.)
\pscustom[linewidth=0.4pt,linecolor=qqwuqq,fillcolor=qqwuqq,fillstyle=solid,opacity=0.1]{
\parametricplot{0.0}{1.5707963267948966}{0.6*cos(t)+1.|0.6*sin(t)+1.}
\lineto(1.,1.)\closepath}
\psellipse*[linewidth=0.4pt,linecolor=qqwuqq,fillcolor=qqwuqq,fillstyle=solid,opacity=1](1.2495670992423111,1.2495670992423111)(0.02,0.02)
\begin{scriptsize}
\rput[bl](3.8,0.74){$a$}
\rput[bl](2.26,4.72){$c$}
\rput[bl](0.66,2.76){$b$}
\end{scriptsize}
\end{pspicture*}
\end{center}

Mit welchen der nachstehenden Formeln kann man die Fl�che dieses Trapezes berechnen?

Kreuze die zutreffende(n) Formel(n) an!

\multiplechoice[5]{  %Anzahl der Antwortmoeglichkeiten, Standard: 5
				L1={$A_1=\frac{1}{2}\cdot (a+c)\cdot b$},   %1. Antwortmoeglichkeit 
				L2={$A_2=b\cdot c + \frac{(a-c)\cdot b}{2}$},   %2. Antwortmoeglichkeit
				L3={$A_3=a\cdot b - 0,5 \cdot (a-c)\cdot b$},   %3. Antwortmoeglichkeit
				L4={$A_4=0,5\cdot a \cdot b - (a+c)\cdot b$},   %4. Antwortmoeglichkeit
				L5={$A_5=\frac{1}{2}\cdot a \cdot b + b \cdot c$},	 %5. Antwortmoeglichkeit
				L6={},	 %6. Antwortmoeglichkeit
				L7={},	 %7. Antwortmoeglichkeit
				L8={},	 %8. Antwortmoeglichkeit
				L9={},	 %9. Antwortmoeglichkeit
				%% LOESUNG: %%
				A1=1,  % 1. Antwort
				A2=2,	 % 2. Antwort
				A3=3,  % 3. Antwort
				A4=0,  % 4. Antwort
				A5=0,  % 5. Antwort
				}


\end{beispiel}