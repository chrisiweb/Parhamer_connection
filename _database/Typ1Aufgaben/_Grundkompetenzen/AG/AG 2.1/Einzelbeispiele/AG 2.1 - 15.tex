\section{AG 2.1 - 15 Archäologie - OA - Matura 2014/15 - Kompensationsprüfung}

\begin{beispiel}[AG 2.1]{1} %PUNKTE DES BEISPIELS
				In der Archäologie gibt es eine empirische Formel, um von der Länge eines entdeckten Oberschenkelknochens auf die Körpergröße der zugehörigen Person schließen zu können.
				Für Männer gilt näherungsweise: $h=48,8+2,63\cdot l$
				Dabei beschreibt $l$ die Länge des Oberschenkelknochens und $h$ die Körpergröße. Beides wird in Zentimetern (cm) angegeben.\\
				
				Berechne die Körpergröße eines Mannes, dessen Oberschenkelknochen eine Länge von $50\,cm$ aufweist.\\
				
				\antwort{$h=180,3\,cm$}
\end{beispiel}