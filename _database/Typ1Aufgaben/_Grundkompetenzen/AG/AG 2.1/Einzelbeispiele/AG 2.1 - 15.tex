\section{AG 2.1 - 15 Arch�ologie - OA - Matura 2014/15 - Kompensationspr�fung}

\begin{beispiel}[AG 2.1]{1} %PUNKTE DES BEISPIELS
				In der Arch�ologie gibt es eine empirische Formel, um von der L�nge eines entdeckten Oberschenkelknochens auf die K�rpergr��e der zugeh�rigen Person schlie�en zu k�nnen.
				F�r M�nner gilt n�herungsweise: $h=48,8+2,63\cdot l$
				Dabei beschreibt $l$ die L�nge des Oberschenkelknochens und $h$ die K�rpergr��e. Beides wird in Zentimetern (cm) angegeben.\\
				
				Berechne die K�rpergr��e eines Mannes, dessen Oberschenkelknochen eine L�nge von $50\,cm$ aufweist.\\
				
				\antwort{$h=180,3\,cm$}
\end{beispiel}