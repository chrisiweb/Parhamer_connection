\section{AG 2.1 - 19 - MAT - Anzahl der Personen in einem Autobus - MC - Matura 2016/17 2. NT}

\begin{beispiel}[AG 2.1]{1} %PUNKTE DES BEISPIELS
Die Variable $F$ bezeichnet die Anzahl der weiblichen Passagiere in einem Autobus, $M$ bezeichnet die Anzahl der männlichen Passagiere in diesem Autobus. Zusammen mit dem Lenker (männlich) sind doppelt so viele Männer wie Frauen in diesem Autobus. (Der Lenker wird nicht bei den Passagieren mitgezählt.)

Kreuze diejenige Gleichung an, die den Zusammenhang zwischen der Anzahl der Frauen und der Anzahl der Männer in diesem Autobus richtig beschreibt!\leer

\multiplechoice[6]{  %Anzahl der Antwortmoeglichkeiten, Standard: 5
				L1={$2\cdot(M+1)=F$},   %1. Antwortmoeglichkeit 
				L2={$M+1=2\cdot F$},   %2. Antwortmoeglichkeit
				L3={$F=2\cdot M+1$},   %3. Antwortmoeglichkeit
				L4={$F+1=2\cdot M$},   %4. Antwortmoeglichkeit
				L5={$M-1=2\cdot F$},	 %5. Antwortmoeglichkeit
				L6={$2\cdot F=M$},	 %6. Antwortmoeglichkeit
				L7={},	 %7. Antwortmoeglichkeit
				L8={},	 %8. Antwortmoeglichkeit
				L9={},	 %9. Antwortmoeglichkeit
				%% LOESUNG: %%
				A1=2,  % 1. Antwort
				A2=0,	 % 2. Antwort
				A3=0,  % 3. Antwort
				A4=0,  % 4. Antwort
				A5=0,  % 5. Antwort
				}
\end{beispiel}