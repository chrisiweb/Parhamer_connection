\section{AG 2.1 - 22 - MAT - Darstellung von Zusammenhängen durch Gleichungen - ZO - Matura 2. NT 2017/18}

\begin{beispiel}[AG 2.1]{1}
Viele Zusammenhänge können in der Mathematik durch Gleichungen ausgedrückt werden.

Ordne den Beschreibungen eines möglichen Zusammenhangs zweier Zahlen $a$ und $b$ mit $a, b \in \mathbb{R}^+$ jeweils die entsprechende Gleichung (aus A bis F) zu!

\zuordnen{
				R1={$a$ ist halb so groß wie $b$.},				% Response 1
				R2={$b$ ist 2\,\% von $a$.},				% Response 2
				R3={$a$ ist um 2\,\% größer als $b$},				% Response 3
				R4={$b$ ist um 2\,\% kleiner als $a$.},				% Response 4
				%% Moegliche Zuordnungen: %%
				A={$2\cdot a=b$}, 				%Moeglichkeit A  
				B={$2\cdot b=a$}, 				%Moeglichkeit B  
				C={$a=1,02\cdot b$}, 				%Moeglichkeit C  
				D={$b=0,02\cdot a$}, 				%Moeglichkeit D  
				E={$1,2 \cdot b=a$}, 				%Moeglichkeit E  
				F={$b=0,98\cdot a$}, 				%Moeglichkeit F  
				%% LOESUNG: %%
				A1={A},				% 1. richtige Zuordnung
				A2={D},				% 2. richtige Zuordnung
				A3={C},				% 3. richtige Zuordnung
				A4={F},				% 4. richtige Zuordnung
				}
\end{beispiel}