\section{AG 2.1 - 20 - MAT - Solaranlagen - OA - Matura HT 2017/18}

\begin{beispiel}[AG 2.1]{1} %PUNKTE DES BEISPIELS
Eine Gemeinde unterstützt den Neubau von Solaranlagen in $h$ Haushalten mit jeweils $p\,\%$ der Anschaffungskosten, wobei das arithmetische Mittel der Anschaffungskosten für eine Solaranlage für einen Haushalt in dieser Gemeinde $e$ Euro beträgt.

Interpretiere den Term $h\cdot e\cdot\frac{p}{100}$ im angegebenen Kontext!

\antwort{Der Term gibt die Gesamtausgaben der Gemeinde zur Unterstützung der Haushalte bei den Anschaffungskosten für neue Solaranlagen an.}
\end{beispiel}