\section{AG 2.1 - 7 Sparbuch - OA - BIFIE}

\begin{beispiel}[AG 2.1]{1} %PUNKTE DES BEISPIELS
Ein Geldbetrag K wird auf ein Sparbuch gelegt. Er w�chst in $n$ Jahren bei einem effektiven Jahreszinssatz von $p\,\%$ auf $K(n)=K\cdot \left(1+\frac{p}{100}\right)^n$.

\leer

Gib eine Formel an, die es erm�glicht, aus dem aktuellen Kontostand $K(n)$ jenen des n�chsten Jahres $K(n+1)$ zu errechnen!	


\antwort{$K(n+1)=K(n)\cdot\left(1+\frac{p}{100}\right)$}				
\end{beispiel}
