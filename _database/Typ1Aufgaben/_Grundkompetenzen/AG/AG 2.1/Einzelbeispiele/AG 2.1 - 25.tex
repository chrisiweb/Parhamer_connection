\section{AG 2.1 - 25 - MAT - Verkehrsunfallstatistik - OA - Matura 2018/19 2.NT}

\begin{beispiel}[AG 2.1]{1}
Die nachstehenden Angaben beziehen sich auf Straßenverkehrsunfälle im Zeitraum von 2014 bis 2016.

$A\,\ldots$ Anzahl der Straßenverkehrsunfälle im Jahr 2014, davon $a\,\%$ mit Personenschaden.\\
$B\,\ldots$ Anzahl der Straßenverkehrsunfälle im Jahr 2015, davon $b\,\%$ mit Personenschaden.\\
$C\,\ldots$ Anzahl der Straßenverkehrsunfälle im Jahr 2016, davon $c\,\%$ mit Personenschaden.

Gib einen Term für die Gesamtanzahl $N$ der Straßenverkehrsunfälle mit Personenschaden im Zeitraum von 2014 bis 2016 an.\leer

$N=\,\antwort[\rule{3cm}{0.3pt}]{\dfrac{A\cdot a}{100}+\dfrac{B\cdot b}{100}+\dfrac{C\cdot c}{100}}$
\end{beispiel}