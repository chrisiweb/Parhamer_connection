\section{AG 2.1 - 11 - MAT - Treibstoffkosten - OA - BIFIE Matura Juni 2016}

\begin{beispiel}[AG 2.1]{1} %PUNKTE DES BEISPIELS
Der durchschnittliche Treibstoffverbrauch eines PKW beträgt $y$ Liter pro 100\,km Fahrtstrecke.
Die Kosten für den Treibstoff betragen $a$ Euro pro Liter.

Gib einen Term an, der die durchschnittlichen Treibstoffkosten K (in Euro) für eine Fahrtstrecke
von x km beschreibt. \leer

$K=\antwort[\rule{5cm}{0.3pt}]{x\cdot \dfrac{y}{100} \cdot a}$
\end{beispiel}