\section{AG 2.1 - 12 Heizungstage - OA - BIFIE - Kompetenzcheck 2016}

\begin{beispiel}[AG 2.1]{1} %PUNKTE DES BEISPIELS
				Die Anzahl der Heizungstage, für die ein Vorrat an Heizöl in einem Tank reicht, ist indirekt proportional zum durschnittlichen Tagesverbrauch $x$ (in Litern).\\

In einem Tank befinden sich 1500 Liter Heizöl. Gib einen Term an der die Anzahl $d(x)$ der Heizungstage in Abhängigkeit vom durschnittlichen Tagesverbrauch $x$ bestimmt.\\

$d(x)=$\rule{5cm}{0.3pt}\\

\antwort{$d(x)=\frac{1500}{x}$}
\end{beispiel}	