\section{AG 3.5 - 9 - K5 - Normalvektoren - ZO - MarStr UNIVIE}

\begin{beispiel}[AG 3.5]{1}
Die folgende Abbildung definiert vier Vektoren $\vec{a}$, $\vec{b}$, $\vec{c}$ und $\vec{d}$ aus $\mathbb{R}^2$. 

\begin{center}
\psset{xunit=0.7cm,yunit=0.7cm,algebraic=true,dimen=middle,dotstyle=o,dotsize=5pt 0,linewidth=1pt,arrowsize=3pt 2,arrowinset=0.25}
\begin{pspicture*}(-3.9842392011876386,-3.8737955044436427)(3.936599960489496,3.880289148566624)
\multips(0,-3)(0,1.0){8}{\psline[linestyle=dashed,linecap=1,dash=1.5pt 1.5pt,linewidth=0.4pt,linecolor=gray]{c-c}(-3.9842392011876386,0)(3.936599960489496,0)}
\multips(-3,0)(1.0,0){8}{\psline[linestyle=dashed,linecap=1,dash=1.5pt 1.5pt,linewidth=0.4pt,linecolor=gray]{c-c}(0,-3.8737955044436427)(0,3.880289148566624)}
\psaxes[labelFontSize=\scriptstyle, showorigin=false, xAxis=true,yAxis=true,Dx=1.,Dy=1.,ticksize=-2pt 0,subticks=0]{->}(0,0)(-3.9842392011876386,-3.8737955044436427)(3.936599960489496,3.880289148566624)
\psline[linewidth=1.pt]{->}(0,0)(1.,3.)
\psline[linewidth=1.pt]{->}(-3.,-1.)(3.,3.)
\psline[linewidth=1.pt]{->}(-2.,3.)(1.,-3.)
\psline[linewidth=1.pt]{->}(3.,1.)(1.,0.)
\begin{scriptsize}
\rput[bl](0.35,2.2){$\vec{a}$}
\rput[bl](-1.9,1.4){$\vec{b}$}
\rput[bl](1.5,1.6){$\vec{c}$}
\rput[bl](2.3,0.2){$\vec{d}$}
\end{scriptsize}
\end{pspicture*}
\end{center}

Ordne jedem der Vektoren $\vec{a}$, $\vec{b}$, $\vec{c}$ und $\vec{d}$ einen möglichen Normalvektor (aus A bis F) zu!

\zuordnen{
				R1={$\vec{a}$},				% Response 1
				R2={$\vec{b}$},				% Response 2
				R3={$\vec{c}$},				% Response 3
				R4={$\vec{d}$},				% Response 4
				%% Moegliche Zuordnungen: %%
				A={\small{$\vec{n_1}= \Vek{2}{1}{}$}}, 				%Moeglichkeit A  
				B={\small{$\vec{n_2}= \Vek{1}{-2}{}$}}, 			%Moeglichkeit B  
				C={\small{$\vec{n_3}= \Vek{2}{3}{}$}}, 				%Moeglichkeit C  
				D={\small{$\vec{n_4}= \Vek{3}{-1}{}$}}, 			%Moeglichkeit D  
				E={\small{$\vec{n_5}= \Vek{-3}{-1}{}$}}, 			%Moeglichkeit E  
				F={\small{$\vec{n_6}= \Vek{-2}{3}{}$}}, 			%Moeglichkeit F  
				%% LOESUNG: %%
				A1={D},				% 1. richtige Zuordnung
				A2={A},				% 2. richtige Zuordnung
				A3={F},				% 3. richtige Zuordnung
				A4={B},				% 4. richtige Zuordnung
				}
\end{beispiel}