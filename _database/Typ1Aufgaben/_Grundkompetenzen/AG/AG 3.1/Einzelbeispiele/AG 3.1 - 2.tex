\section{AG 3.1 - 2 Perlensterne - OA - BIFIE}

\begin{beispiel}[AG 3.1]{1} %PUNKTE DES BEISPIELS
F�r einen Adventmarkt sollen Perlensterne hergestellt werden. Den Materialbedarf f�r die verschiedenen Modelle kann man der nachstehenden Tabelle entnehmen.
Den Spalten der Tabelle entsprechen Vektoren im $\mathbb{R}^{4}$
\begin{itemize}
	\item Materialbedarfsvektor $S_{1}$ f�r den Stern 1
	\item Materialbedarfsvektor $S_{2}$ f�r den Stern 2
	\item Kostenvektor K pro Packung zu 10 St�ck
	\item Lagerbestand L
\end{itemize}
\begin{longtable}{l|C{2,5cm}|C{2,5cm}|C{3cm}|C{3cm}|}
&Material Stern 1&Material Stern 2&Kosten pro Packung Perlen & Lagerbestand der Perlen-Packungen\\ \hline
Wachsperlen 6mm & 1 & 0 & \euro\,$0,20$ & 8 \\ \hline
Wachsperlen 3mm & 72 & 84 & \euro\,$0,04$ & 100 \\ \hline
Glasperlen 6mm & 0 & 6 & \euro\,$0,90$ & 12 \\ \hline
Glasperlen oval & 8 & 0 & \euro\,$1,50$ & 9 \\ \hline
\end{longtable}

Gib die Bedeutung des Ausdrucks $10\cdot L-(5\cdot S_{1}+8\cdot S_{2})$ in diesem Zusammenhang an!

\antwort{$10\cdot L-(5\cdot S_{1}+8\cdot S_{2})$ gibt die verschiedenen noch vorhanden Perlen nach der Fertigung von 5 Sternen nach Modell 1 und 8 Sternen nach Modell 2 an.}
\end{beispiel}