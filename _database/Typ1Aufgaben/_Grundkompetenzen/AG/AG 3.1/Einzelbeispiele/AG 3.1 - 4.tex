\section{AG 3.1 - 4 Vektoren als Zahlentupel - MC - BIFIE}

\begin{beispiel}[AG 3.1]{1} %PUNKTE DES BEISPIELS
Gegeben sind zwei Vektoren: $\vek{a},\vek{b}\in\mathbb{R}^{2}$.

Welche der nachstehenden Aussagen �ber Vektoren sind korrekt? Kreuze die beiden zutreffenden Aussagen an!
\multiplechoice[5]{  %Anzahl der Antwortmoeglichkeiten, Standard: 5
				L1={Der Vektor $3\cdot\vek{a}$ ist dreimal so lang wie der Vektor $\vek{a}$.},   %1. Antwortmoeglichkeit 
				L2={Das Produkt $\vek{a}\cdot\vek{b}$ ergibt einen Vektor.},   %2. Antwortmoeglichkeit
				L3={Die Vektoren $\vek{a}$ und $-0,5\cdot\vek{a}$ besitzen die gleiche Richtung und sind gleich orientiert.},   %3. Antwortmoeglichkeit
				L4={Die Vektoren $\vek{a}$ und $-2\cdot\vek{a}$ sind parallel.},   %4. Antwortmoeglichkeit
				L5={Wenn $\vek{a}$ und $\vek{b}$ einen rechten Winkel einschlie�en, so ist deren Skalarprodukt gr��er als null.},	 %5. Antwortmoeglichkeit
				L6={},	 %6. Antwortmoeglichkeit
				L7={},	 %7. Antwortmoeglichkeit
				L8={},	 %8. Antwortmoeglichkeit
				L9={},	 %9. Antwortmoeglichkeit
				%% LOESUNG: %%
				A1=1,  % 1. Antwort
				A2=4,	 % 2. Antwort
				A3=0,  % 3. Antwort
				A4=0,  % 4. Antwort
				A5=0,  % 5. Antwort
				}
\end{beispiel}