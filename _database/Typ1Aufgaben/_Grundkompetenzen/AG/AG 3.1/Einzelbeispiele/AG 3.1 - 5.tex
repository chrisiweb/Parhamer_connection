\section{AG 3.1 - 5 Betriebsgewinn - OA - BIFIE}

\begin{beispiel}[AG 3.1]{1} %PUNKTE DES BEISPIELS
Ein Betrieb produziert und verkauft die Produkte $P_{1}, ..., P_{5}$. In der vorangegangenen Woche wurden $x_{i}$ Stück des Produktes $P_{i}$ produziert und auch verkauft. Das Produkt $P_{i}$ wird zu einem Stückpreis $v_{i}$ verkauft, $k_{i}$ sind die Herstellungskosten pro Stück $P_{i}$.

Die Vektoren $X,V$ und $K$ sind folgendermaßen festgelegt:
\leer

\begin{center}
$X=\left(\begin{array}{r}x_1\\x_2\\x_3\\x_4\\x_5\end{array}\right), V=\left(\begin{array}{r}v_1\\v_2\\v_3\\v_4\\v_5\end{array}\right), K=\left(\begin{array}{r}k_1\\k_2\\k_3\\k_4\\k_5\end{array}\right)$
\end{center}

Gib mithilfe der gegebenen Vektoren einen Term an, der für diesen Betrieb den Gewinn $G$ der letzten Woche beschreibt!
\leer

$G=$ \antwort[\rule{3cm}{0.3pt}]{$X\cdot V-X\cdot K$}
\end{beispiel}