\section{AG 3.1 - 3 Torten - OA - BIFIE}

\begin{beispiel}[AG 3.1]{1} %PUNKTE DES BEISPIELS
Eine Konditorei stellt 3 verschiedene Torten her: Malakofftorte $M$, Sachertorte $S$ und Obsttorte $O$. Die Konditorei beliefert damit 5 Wiederverk�ufer.

Die Liefermengen pro Tortenst�ck an die Wiederverk�ufer $W$ werden durch die Vektoren $L_{M}$ f�r die Malakofftorte, $L_{S}$ f�r die Sachertorte und $L_{O}$ f�r die Obsttorte ausgedr�ckt.
\[W=\left(\begin{array}{r}W_1\\W_2\\W_3\\W_4\\W_5\end{array}\right), L_{M}=\left(\begin{array}{r}20\\45\\60\\30\\10\end{array}\right), L_{S}=\left(\begin{array}{r}15\\20\\30\\0\\20\end{array}\right), L_{O}=\left(\begin{array}{r}10\\35\\40\\10\\25\end{array}\right)\]

Ein St�ck Malakofftorte kostet beim Konditor \euro\,$1,80$, ein St�ck Sachertorte \euro\,$2,10$ und ein St�ck Obsttorte \euro\,$1,50$.

Gib an, wie viele Tortenst�cke der Konditor insgesamt an den Wiederverk�ufer $W_{3}$ liefert! Berechne, wie viele St�ck Sachertorte der Konditor insgesamt ausgeliefert hat!

\antwort{An den dritten Wiederverk�ufer hat der Konditor $60+30+40=130$ Tortenst�cke geliefert. Der Konditor hat insgesamt $15+20+30+0+20=85$ St�ck Sachertorte ausgeliefert.}
\end{beispiel}