\section{AG 2.5 - 14 - MAT - Gleichungssysteme aufstellen und lösen - OA - JanRos}

\begin{beispiel}[AG 2.5]{1} %PUNKTE DES BEISPIELS
Als Leonie vom Spazierengehen zurückkommt, stellt sie ihrer Schwester ein Rätsel: \glqq Ich habe ein Gehege mit Hühnern und Ziegen entdeckt. Insgesamt habe ich $46$ Beine gezählt. Außerdem waren um $5$ mehr Hühner als Ziegen im Gehege.\grqq

Es sei $h$ die Anzahl der Hühner und $z$ die Anzahl der Ziegen.

Stelle ein lineares Gleichungssystem auf und berechne die Anzahl der Hühner und die Anzahl der Ziegen.

$I:\antwort[\rule{5cm}{0.3pt}]{2h+4z=46}$

$II:\antwort[\rule{5cm}{0.3pt}]{h=z+5}$

$z=\antwort[\rule{5cm}{0.3pt}]{6}$

$h=\antwort[\rule{5cm}{0.3pt}]{11}$

\end{beispiel}
