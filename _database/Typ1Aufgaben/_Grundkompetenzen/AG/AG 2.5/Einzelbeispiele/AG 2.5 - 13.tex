\section{AG 2.5 - 13 - Drei Gleichungssysteme - LT - MarNeu UNIVIE}

\begin{beispiel}[AG 2.5]{1}
Gegeben sind folgende drei Gleichungen:
\begin{align*}
\text{I}: \hspace{0,5cm} & 3x+5y=10 \\
\text{II}: \hspace{0,5cm} & 3x+5y=4 \\
\text{III}: \hspace{0,5cm} & 3x-5y=-16
\end{align*}
\lueckentext{
				text={Das Gleichungssystem \gap hat \gap.}, 	%Lueckentext Luecke=\gap
				L1={\begin{tabular}{cccc}
				\text{I}:&$3x+5y$&$=$&$10$ \\
				\text{II}:&$3x+5y$&$=$&$4$\\
				\end{tabular}}, 		%1.Moeglichkeit links  
				L2={\begin{tabular}{cccc}
				\text{I}:&$3x+5y$&$=$&$10$ \\
				\text{III}:&$3x-5y$&$=$&$-16$\\
				\end{tabular}}, 		%2.Moeglichkeit links
				L3={\begin{tabular}{cccc}
				\text{II}:&$3x+5y$&$=$&$4$ \\
				\text{III}:&$3x-5y$&$=$&$-16$\\
				\end{tabular}}, 		%3.Moeglichkeit links
				R1={unendlich viele Lösungen}, 		%1.Moeglichkeit rechts 
				R2={die Lösung $(-2 \mid 2)$.}, 		%2.Moeglichkeit rechts
				R3={die Lösung $(0 \mid 2)$.}, 		%3.Moeglichkeit rechts
				%% LOESUNG: %%
				A1=3,   % Antwort links
				A2=2		% Antwort rechts 
				}
\end{beispiel}