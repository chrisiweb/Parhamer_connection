\section{AG 2.5 - 7 - MAT - Gleichungssystem - OA - Matura 2015/16
- Nebentermin 1}

\begin{beispiel}[AG 2.5]{1} %PUNKTE DES BEISPIELS
Gegeben ist ein Gleichungssystem aus zwei linearen Gleichungen in den Variablen $x,y \in \mathbb{R}$:

\begin{align*}
I: \quad x+4\cdot y &= -8 \\
II: a\cdot x +6 \cdot y &= c \qquad \text{mit }a,c \in \mathbb{R} \\
\end{align*}


Ermittle diejenigen Werte für $a$ und $c$, für die das Gleichungssystem unendlich viele Lösungen hat. \leer

$a=$ \antwort[\rule{3cm}{0.3pt}]{$1,5$}\leer

$c=$ \antwort[\rule{3cm}{0.3pt}]{$-12$}		
\end{beispiel}