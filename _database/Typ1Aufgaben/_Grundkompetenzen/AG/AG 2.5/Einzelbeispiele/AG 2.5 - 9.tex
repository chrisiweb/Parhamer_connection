\section{AG 2.5 - 9 - MAT - Projektwoche - MC - Matura NT 1 16/17}

\begin{beispiel}[AG 2.5]{1} %PUNKTE DES BEISPIELS
An einer Projektwoche nehmen insgesamt 25 Schüler/innen teil. Die Anzahl der Mädchen wird mit $x$ bezeichnet, die Anzahl der Burschen mit $y$. Die Mädchen werden in 3-Bett-Zimmern untergebracht, die Burschen in 4-Bett-Zimmern, insgesamt stehen 7 Zimmer zur Verfügung. Die Betten alle 7 Zimmer werden belegt, es bleiben keine leeren Betten übrig.

Mithilfe eines Gleichungssystems aus zwei der nachstehenden Gleichungen kann die Anzahl der Mädchen und die Anzahl der Burschen berechnet werden. Kreuze die beiden zutreffenden Gleichungen an!

\multiplechoice[5]{  %Anzahl der Antwortmoeglichkeiten, Standard: 5
				L1={$x+y=7$},   %1. Antwortmoeglichkeit 
				L2={$x+y=25$},   %2. Antwortmoeglichkeit
				L3={$3\cdot x+4\cdot y=7$},   %3. Antwortmoeglichkeit
				L4={$\frac{x}{3}+\frac{y}{4}=7$},   %4. Antwortmoeglichkeit
				L5={$\frac{x}{3}+\frac{y}{4}=25$},	 %5. Antwortmoeglichkeit
				L6={},	 %6. Antwortmoeglichkeit
				L7={},	 %7. Antwortmoeglichkeit
				L8={},	 %8. Antwortmoeglichkeit
				L9={},	 %9. Antwortmoeglichkeit
				%% LOESUNG: %%
				A1=2,  % 1. Antwort
				A2=4,	 % 2. Antwort
				A3=0,  % 3. Antwort
				A4=0,  % 4. Antwort
				A5=0,  % 5. Antwort
				}
\end{beispiel}