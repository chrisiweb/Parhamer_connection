\section{AG 2.5 - 8 Zusammensetzung einer Schulklasse - MC - Puehringer}

\begin{beispiel}[AG 2.5]{1} %PUNKTE DES BEISPIELS
			In einer Schulklasse sind $m$ Mädchen und $b$ Burschen. Es gilt folgendes Gleichungssystem mit $c\in\mathbb{N}$:
\begin{align*}
2b=m\\
m-c=b
\end{align*}\\
Kreuze die \textbf{beiden} zutreffenden Aussagen an!\\

\multiplechoice[5]{  %Anzahl der Antwortmoeglichkeiten, Standard: 5
				L1={In der Klasse sind insgesamt $c$ Kinder.},   %1. Antwortmoeglichkeit 
				L2={In der Klasse sind um $c$ Mädchen weniger als Burschen.},   %2. Antwortmoeglichkeit
				L3={In der Klasse sind mehr Mädchen als Burschen.},   %3. Antwortmoeglichkeit
				L4={Die Anzahl der Mädchen ist sicher ungerade.},   %4. Antwortmoeglichkeit
				L5={Die Anzahl der Mädchen ist sicher gerade.},	 %5. Antwortmoeglichkeit
				L6={},	 %6. Antwortmoeglichkeit
				L7={},	 %7. Antwortmoeglichkeit
				L8={},	 %8. Antwortmoeglichkeit
				L9={},	 %9. Antwortmoeglichkeit
				%% LOESUNG: %%
				A1=3,  % 1. Antwort
				A2=5,	 % 2. Antwort
				A3=0,  % 3. Antwort
				A4=0,  % 4. Antwort
				A5=0,  % 5. Antwort
				}	
\end{beispiel}

