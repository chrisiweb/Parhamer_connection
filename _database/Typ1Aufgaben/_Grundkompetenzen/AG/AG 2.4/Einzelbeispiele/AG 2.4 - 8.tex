\section{AG 2.4 - 8 - Ungleichungen im Kontext von social-media-marketing - OA - FraKol}

\begin{beispiel}[AG 2.4]{1}
Ein Influencer bietet für Werbe-Partner zwei verschiedene Modelle an, um ihre Produkte auf einer social-media-Plattform bewerben zu können:

\begin{tabular}{ll}
Tarif A: & einmalige Gebühr von 100\euro{}, \\
& 0,10\,\euro pro Klick auf den Werbe-Link \\
Tarif B: & keine einmalige Gebühr, \\
& 0,30\,\euro pro Klick auf den Werbe-Link
\end{tabular}

Gib an, ab welcher Anzahl an Klicks Tarif A billiger als Tarif B für den Werbe-Partner ist.\\
Stelle dafür eine Ungleichung auf, die diesen Sachverhalt passend beschreibt, und löse sie.

\antwort{
mögliche Vorgehensweise:\\
$100 + 0,1x < 0,3x$\\
$100 < 0,2x$\\
$x > 500$\\
Ab dem 501. Klick ist für den Werbe-Partner Tarif A billiger als Tarif B.
}
\end{beispiel}