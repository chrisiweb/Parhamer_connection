\section{AG 2.4 - 2 - Handytarife  - OA - BIFIE}

\begin{beispiel}[AG 2.4]{1} %PUNKTE DES BEISPIELS
Vom Handy-Netzbetreiber TELMAXFON werden zwei Tarifmodelle angeboten: \\
\begin{itemize}
	\item[Tarif A:] keine monatliche Grundgebühr,
 Verbindungsentgelt 6,8 Cent pro Minute in alle Netze
\item[Tarif B:] monatliche Grundgebühr \euro\,15,
 Verbindungsentgelt 2,9 Cent pro Minute in alle Netze
\end{itemize}

Interpretiere in diesem Zusammenhang den Ansatz und das Ergebnis der folgenden Rechnung:

\begin{align*}
15 + 0.029\cdot t &< 0,068\cdot t \\
15 &< 0,039\cdot t \\
t&>384,6
\end{align*}

\antwort{Mit dem Ansatz $(15+0,029\cdot t<0,068\cdot t)$ kann man überprüfen, ob Tarif B bei $t$ telefonierten Minuten günstiger ist als Tarif A. \\
Durch Umformen der Ungleichung sieht man, dass Tarif B günstiger ist als Tarif A, wenn man mehr als 384 Minuten telefoniert.}
\end{beispiel}