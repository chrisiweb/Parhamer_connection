\section{AG 2.4 - 4 Halbebenen - ZO - BIFIE}

\begin{beispiel}[AG 2.4]{1} %PUNKTE DES BEISPIELS
	Lineare Ungleichungen mit zwei Variablen besitzen unendlich viele Lösungspaare, die geometrisch interpretiert Punkte einer offenen oder geschlossenen Halbebene sind.
	
In den nachstehenden Grafiken ist jeweils ein Bereich (eine Halbebene) farblich markiert. 

\leer

Ordne den einzelnen Bereichen die jeweilige Lineare Ungleichung zu, die die Halbebene im Koordinatensystem richtig beschreibt!

\zuordnen{
				title1={Ungleichungen}, 		%Titel Antwortmoeglichkeiten
				A={$y>2$}, 				%Moeglichkeit A  
				B={$2y-3x<0$}, 				%Moeglichkeit B  
				C={$3x+2y\geq 4$}, 				%Moeglichkeit C  
				D={$y\leq \frac{2}{3}x+2$}, 				%Moeglichkeit D  
				E={$x>2$}, 				%Moeglichkeit E  
				F={$3y-2x<6$}, 				%Moeglichkeit F  
				title2={Halbebenen},		%Titel Zuordnung
				R1={\resizebox{1\linewidth}{!}{\newrgbcolor{qqttcc}{0. 0.2 0.8}
\begin{pspicture*}(-5.21325220208,-3.41450087283)(6.41014108686,5.31435089669)
\pspolygon[linewidth=2.4pt,dash=4pt 4pt,linecolor=qqttcc,fillcolor=qqttcc!20,fillstyle=solid](-5.36943143662,10.0541471549)(10.,10.)(9.91565364493,-12.8734804674)
\psset{xunit=1.0cm,yunit=1.0cm,algebraic=true,dimen=middle,dotstyle=o,dotsize=3pt 0,linewidth=0.8pt,arrowsize=3pt 2,arrowinset=0.25}
\multips(0,-3)(0,1.0){9}{\psline[linestyle=dashed,linecap=1,dash=1.5pt 1.5pt,linewidth=0.4pt,linecolor=darkgray]{c-c}(-5.21325220208,0)(6.41014108686,0)}
\multips(-5,0)(1.0,0){12}{\psline[linestyle=dashed,linecap=1,dash=1.5pt 1.5pt,linewidth=0.4pt,linecolor=darkgray]{c-c}(0,-3.41450087283)(0,5.31435089669)}
\psaxes[labelFontSize=\scriptstyle,xAxis=true,yAxis=true,Dx=1.,Dy=1.,ticksize=-2pt 0,subticks=2]{->}(0,0)(-5.21325220208,-3.41450087283)(6.41014108686,5.31435089669)[x,140] [y,-40]
\psline[linewidth=2.4pt,linestyle=dashed,dash=4pt 4pt,linecolor=qqttcc](-5.36943143662,10.0541471549)(10.,10.)
\psline[linewidth=2.4pt,linestyle=dashed,dash=4pt 4pt,linecolor=qqttcc](10.,10.)(9.91565364493,-12.8734804674)
\psline[linewidth=2.4pt,linestyle=dashed,dash=4pt 4pt,linecolor=qqttcc](9.91565364493,-12.8734804674)(-5.36943143662,10.0541471549)
\end{pspicture*}}},				%1. Antwort rechts
				R2={\resizebox{1\linewidth}{!}{\newrgbcolor{qqttcc}{0. 0.2 0.8}
\newrgbcolor{xdxdff}{0.490196078431 0.490196078431 1.}
\begin{pspicture*}(-4.70837391989,-4.57714796729)(5.63070774413,4.60433713005)
\pspolygon[linewidth=0.pt,linestyle=dashed,dash=4pt 4pt,linecolor=qqttcc,fillcolor=qqttcc!20,fillstyle=solid](2.,14.9637536362)(14.20054885,15.3525519007)(14.8536417196,-40.1603420126)(2.,-40.)
\psset{xunit=1.0cm,yunit=1.0cm,algebraic=true,dimen=middle,dotstyle=o,dotsize=3pt 0,linewidth=0.8pt,arrowsize=3pt 2,arrowinset=0.25}
\multips(0,-4)(0,1.0){10}{\psline[linestyle=dashed,linecap=1,dash=1.5pt 1.5pt,linewidth=0.4pt,linecolor=darkgray]{c-c}(-4.70837391989,0)(5.63070774413,0)}
\multips(-4,0)(1.0,0){11}{\psline[linestyle=dashed,linecap=1,dash=1.5pt 1.5pt,linewidth=0.4pt,linecolor=darkgray]{c-c}(0,-4.57714796729)(0,4.60433713005)}
\psaxes[labelFontSize=\scriptstyle,xAxis=true,yAxis=true,Dx=1.,Dy=1.,ticksize=-2pt 0,subticks=2]{->}(0,0)(-4.70837391989,-4.57714796729)(5.63070774413,4.60433713005)[x,140] [y,-40]
\psline[linewidth=2.4pt,linestyle=dashed,dash=4pt 4pt,linecolor=qqttcc](2.,-4.57714796729)(2.,4.60433713005)
\begin{scriptsize}
\rput[bl](1.74562337652,3.7956052821){\qqttcc{$g$}}
\psdots[dotstyle=*,linecolor=blue](14.20054885,15.3525519007)
\rput[bl](8.9290650848,4.41404728348){\blue{$E$}}
\psdots[dotstyle=*,linecolor=blue](14.8536417196,-40.1603420126)
\rput[bl](8.15204821128,-5.14801750702){\blue{$F$}}
\psdots[dotstyle=*,linecolor=xdxdff](2.,-40.)
\rput[bl](6.15400482222,-5.27487740474){\xdxdff{$G$}}
\psdots[dotstyle=*,linecolor=xdxdff](2.,14.9637536362)
\rput[bl](6.39186713044,4.41404728348){\xdxdff{$H$}}
\end{scriptsize}
\end{pspicture*}}},				%2. Antwort rechts
				R3={\resizebox{1\linewidth}{!}{\newrgbcolor{qqttcc}{0. 0.2 0.8}
\newrgbcolor{xdxdff}{0.490196078431 0.490196078431 1.}
\begin{pspicture*}(-5.21865801166,-3.40672333818)(4.88337028588,3.84442580791)
\pspolygon[linewidth=0.pt,linestyle=dashed,dash=4pt 4pt,linecolor=qqttcc,fillcolor=qqttcc!20,fillstyle=solid](-15.8868305593,-8.59122037285)(-34.9869392039,-21.4788897434)(14.6832087268,-40.3152810969)(39.0769230769,28.0512820513)
\psset{xunit=1.0cm,yunit=1.0cm,algebraic=true,dimen=middle,dotstyle=o,dotsize=3pt 0,linewidth=0.8pt,arrowsize=3pt 2,arrowinset=0.25}
\multips(0,-3)(0,1.0){8}{\psline[linestyle=dashed,linecap=1,dash=1.5pt 1.5pt,linewidth=0.4pt,linecolor=darkgray]{c-c}(-5.21865801166,0)(4.88337028588,0)}
\multips(-5,0)(1.0,0){11}{\psline[linestyle=dashed,linecap=1,dash=1.5pt 1.5pt,linewidth=0.4pt,linecolor=darkgray]{c-c}(0,-3.40672333818)(0,3.84442580791)}
\psaxes[labelFontSize=\scriptstyle,xAxis=true,yAxis=true,Dx=1.,Dy=1.,ticksize=-2pt 0,subticks=2]{->}(0,0)(-5.21865801166,-3.40672333818)(4.88337028588,3.84442580791)[x,140] [y,-40]
\psplot[linewidth=2.4pt,linestyle=dashed,dash=4pt 4pt,linecolor=qqttcc]{-5.21865801166}{4.88337028588}{(--2.--0.666666666667*x)/1.}
\begin{scriptsize}
\rput[bl](1.73810687422,3.47257200554){\qqttcc{$g$}}
\psdots[dotstyle=*,linecolor=blue](-34.9869392039,-21.4788897434)
\rput[bl](-5.15668237794,-1.4699847842){\blue{$E$}}
\psdots[dotstyle=*,linecolor=blue](14.6832087268,-40.3152810969)
\rput[bl](2.69872919699,-5.79278523668){\blue{$F$}}
\psdots[dotstyle=*,linecolor=xdxdff](39.0769230769,28.0512820513)
\rput[bl](5.39466926413,-5.80827914511){\xdxdff{$G$}}
\end{scriptsize}
\end{pspicture*}}},				%3. Antwort rechts
				R4={\resizebox{1\linewidth}{!}{\newrgbcolor{qqttcc}{0. 0.2 0.8}
\newrgbcolor{xdxdff}{0.490196078431 0.490196078431 1.}
\begin{pspicture*}(-4.52385029895,-4.60736035517)(4.83775654374,4.37057407593)
\pspolygon[linewidth=0.pt,linestyle=dashed,dash=5pt 5pt,linecolor=qqttcc,fillcolor=qqttcc!20,fillstyle=solid](-14.9637536362,-22.4456304543)(-34.9546307373,-22.3629541014)(14.7155171934,-41.1993454549)(40.,60.)
\psset{xunit=1.0cm,yunit=1.0cm,algebraic=true,dimen=middle,dotstyle=o,dotsize=3pt 0,linewidth=0.8pt,arrowsize=3pt 2,arrowinset=0.25}
\multips(0,-4)(0,1.0){9}{\psline[linestyle=dashed,linecap=1,dash=1.5pt 1.5pt,linewidth=0.4pt,linecolor=darkgray]{c-c}(-4.52385029895,0)(4.83775654374,0)}
\multips(-4,0)(1.0,0){10}{\psline[linestyle=dashed,linecap=1,dash=1.5pt 1.5pt,linewidth=0.4pt,linecolor=darkgray]{c-c}(0,-4.60736035517)(0,4.37057407593)}
\psaxes[labelFontSize=\scriptstyle,xAxis=true,yAxis=true,Dx=1.,Dy=1.,ticksize=-2pt 0,subticks=2]{->}(0,0)(-4.52385029895,-4.60736035517)(4.83775654374,4.37057407593)[x,140] [y,-40]
\psplot[linewidth=2.4pt,linestyle=dashed,dash=5pt 5pt,linecolor=qqttcc]{-4.52385029895}{4.83775654374}{(-0.--1.5*x)/1.}
\begin{scriptsize}
\rput[bl](2.03694793917,3.71833097624){\qqttcc{$g$}}
\psdots[dotstyle=*,linecolor=blue](-34.9546307373,-22.3629541014)
\rput[bl](5.27897981707,-7.56163792438){\blue{$F$}}
\psdots[dotstyle=*,linecolor=xdxdff](40.,60.)
\rput[bl](8.61692979786,-7.58082154496){\xdxdff{$G$}}
\end{scriptsize}
\end{pspicture*}}},				%4. Antwort rechts
				%% LOESUNG: %%
				A1={C},				% 1. richtige Zuordnung
				A2={E},				% 2. richtige Zuordnung
				A3={F},				% 3. richtige Zuordnung
				A4={B},				% 4. richtige Zuordnung
				}			
\end{beispiel}