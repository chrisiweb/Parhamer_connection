\section{AG 2.4 - 6 - MAT - Erdgasanbieter - OA - Matura 1. NT 2017/18}

\begin{beispiel}[AG 2.4]{1}
Ein Haushalt möchte seinen Erdgaslieferanten wechseln und schwankt noch bei der Wahl zwischen dem Anbieter $A$ und dem Anbieter $B$.

Der Energiegehalt des verbrauchten Erdgases wird in Kilowattstunden (kWh) gemessen.

Anbieter $A$ verrechnet jährlich eine fixe Gebühr von 340 Euro und 2,9 Cent pro kWh.\\
Anbieter $B$ verrechnet jährlich eine fixe Gebühr von 400 Euro und 2,5 Cent pro kWh.

Die Ungleichung $0,025\cdot x+400<0,029\cdot x+340$ dient dem Vergleich der zu erwartenden Kosten bei den beiden Anbietern.

Löse die oben angeführte Ungleichung und interpretiere das Ergebnis im gegebenen Kontext!

\antwort{$x>15.000$

Bei einem Jahresverbrauch von mehr als $15.000$\,kWh ist Anbieter $B$ günstiger als Anbieter $A$.}
\end{beispiel}