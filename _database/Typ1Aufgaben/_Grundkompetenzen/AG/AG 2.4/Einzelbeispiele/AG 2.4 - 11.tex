\section{AG 2.4 - 11 - MAT - Delegation - MC - Matura 2019/20 1. HT}

\begin{beispiel}[AG 2.4]{1}
Aus einer großen Gruppe von Jugendlichen und Erwachsenen soll eine Delegation gebildet werden.

Dabei gelten die folgenden drei Vorschriften:
\begin{itemize}
\item Die Delegation soll mindestens 8 Mitglieder umfassen.
\item Die Delegation soll höchstens 12 Mitglieder umfassen.
\item In der Delegation sollen mindestens doppelt so viele Jugendliche wie Erwachsene sein.
\end{itemize}

Zwei der drei Vorschriften sind unten stehend jeweils durch eine Ungleichung beschrieben. Dabei wird die Anzahl der Jugendlichen in dieser Delegation mit $J$ und die Anzahl der Erwachsenen in dieser Delegation mit $E$ bezeichnet.

Kreuze die beiden zutreffenden Ungleichungen an.

\multiplechoice[5]{  %Anzahl der Antwortmoeglichkeiten, Standard: 5
				L1={$J+E\leq 12$},   %1. Antwortmoeglichkeit 
				L2={$J\geq 2\cdot E$},   %2. Antwortmoeglichkeit
				L3={$J+E\leq 8$},   %3. Antwortmoeglichkeit
				L4={$J-2\cdot E<0$},   %4. Antwortmoeglichkeit
				L5={$E\geq 2\cdot J$},	 %5. Antwortmoeglichkeit
				L6={},	 %6. Antwortmoeglichkeit
				L7={},	 %7. Antwortmoeglichkeit
				L8={},	 %8. Antwortmoeglichkeit
				L9={},	 %9. Antwortmoeglichkeit
				%% LOESUNG: %%
				A1=1,  % 1. Antwort
				A2=2,	 % 2. Antwort
				A3=0,  % 3. Antwort
				A4=0,  % 4. Antwort
				A5=0,  % 5. Antwort
				}
\end{beispiel}