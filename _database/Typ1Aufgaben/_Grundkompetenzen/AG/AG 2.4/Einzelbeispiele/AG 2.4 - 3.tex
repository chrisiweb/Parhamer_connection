\section{AG 2.4 - 3 - Biobauer - OA - BIFIE}

\begin{beispiel}[AG 2.4]{1} %PUNKTE DES BEISPIELS
	Bei einem Biobauern kauft man 1\,kg Kartoffeln um \euro\,0,38. Für die Fahrtkosten hin und zurück müssen allerdings noch \euro\,7,40 veranschlagt werden. Kauft man 1\,kg derselben Kartoffelsorte
im Geschäft, so bezahlt man pro Kilogramm \euro\,0,46. 

\leer

Bei welcher Menge Kartoffeln ist der Preisunterschied zwischen Geschäft und Biobauern größer als \euro\,25? Gib eine Ungleichung an, mit der du diese Fragestellung bearbeiten kannst,
und formuliere eine Antwort für den gegebenen Kontext! 

\antwort{$0,46x-0,38x-7,4 > 25 \rightarrow x>405 $\\
Der Preisunterschied ist größer als \euro\,25, wenn man mehr als 405\,kg Kartoffeln kauft.}		
\end{beispiel}