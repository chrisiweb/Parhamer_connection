\section{AG 2.4 - 7 - MAT - Ungleichungen lösen - OA - Matura-HT-18/19}

\begin{beispiel}[AG 2.4]{1}
Gegeben sind zwei lineare Ungleichungen.
\begin{align*}
I&: 7\cdot x+67 > -17\\
II&: -25-4\cdot x >7 \\
\end{align*} 
Gesucht sind alle reellen Zahlen $x$, die beide Ungleichungen erfüllen. Gib die Menge dieser Zahlen als Intervall an!

\antwort{\subsubsection{Lösungserwartung:}

mögliche Vorgehensweise:
\begin{align*}
I&: 7\cdot x+67 > -17 \Rightarrow x>-12\\
II&: -25-4\cdot x >7 \Rightarrow x<-8\\
\end{align*}

Menge aller reellen Zahlen $x$, die beide Ungleichungen erfüllen: $(-12; -8)$

Lösungsschlüssel:

Ein Punkt für das richtige Intervall. Andere Schreibweisen der Lösungsmenge sind ebenfalls als
richtig zu werten. Bei Angabe eines halboffenen oder geschlossenen Intervalls ist der Punkt nicht zu geben.
}
\end{beispiel}