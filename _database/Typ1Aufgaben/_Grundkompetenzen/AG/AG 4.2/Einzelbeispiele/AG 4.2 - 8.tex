\section{AG 4.2 - 8 Winkelfunktionswert - OA - BIFIE}

\begin{beispiel}[AG 4.2]{1} %PUNKTE DES BEISPIELS
In der nachstehenden Abbildung ist ein Winkelfunktionswert eines Winkels $\gamma$ am Einheitskreis farbig dargestellt.

\begin{center}
\psset{xunit=1.0cm,yunit=1.0cm,algebraic=true,dimen=middle,dotstyle=o,dotsize=5pt 0,linewidth=0.8pt,arrowsize=3pt 2,arrowinset=0.25}
\begin{pspicture*}(-4.9550043525219145,-4.8366692582529405)(5.631154325492476,5.290647812201217)
\psaxes[labelFontSize=\scriptstyle,xAxis=true,yAxis=true,labels=none,Dx=2.,Dy=2.,ticksize=0pt 0,subticks=2]{}(0,0)(-4.9550043525219145,-4.8366692582529405)(5.631154325492476,5.290647812201217)[x,140] [y,-40]
\pscircle(0.,0.){4.}
\rput[tl](0.6166581095909227,4.6351580341782945){1}
\rput[tl](4.352949407713648,0.7349938549419035){1}
\psline[linewidth=1.6pt,linecolor=magenta](0.,0.)(2.5503527287947896,0.)
\antwort{\psline[linewidth=1.6pt](0.,0.)(2.5503527287947896,3.0815095259838103)
\psline[linewidth=1.6pt,linestyle=dashed,dash=4pt 4pt](0.,0.)(2.5503527287947896,-3.0815095259838112)
\psline[linewidth=1.6pt](2.5503527287947896,3.0815095259838103)(2.5503527287947896,0.)
\psline[linewidth=1.6pt,linestyle=dashed,dash=4pt 4pt](2.5503527287947896,0.)(2.5503527287947896,-3.0815095259838112)
\parametricplot{0.0}{0.8794328293073885}{1.*2.*cos(t)+0.*2.*sin(t)+0.|0.*2.*cos(t)+1.*2.*sin(t)+0.}
\parametricplot[linestyle=dashed,dash=4pt 4pt]{0.0}{5.403752477872198}{1.*2.5503527287947896*cos(t)+0.*2.5503527287947896*sin(t)+0.|0.*2.5503527287947896*cos(t)+1.*2.5503527287947896*sin(t)+0.}
\begin{scriptsize}
\psdots[dotsize=3pt 0,dotstyle=*,linecolor=darkgray](0.,0.)
\psdots[dotsize=3pt 0,dotstyle=*,linecolor=darkgray](2.5503527287947896,-3.0815095259838112)
\psdots[dotsize=3pt 0,dotstyle=*,linecolor=darkgray](2.5503527287947896,3.0815095259838103)
\rput[bl](1.1410498707309544,0.37447447702929587){$\gamma_{1}$}
\rput[bl](-1.710330330467968,1.2921601662613877){$\gamma_{2}$}
\end{scriptsize}}
\end{pspicture*}
\end{center}

Gib an, um welche Winkelfunktion es sich dabei handelt, und zeichne alle Winkel im Einheitskreis ein, die diesen Winkelfunktionswert besitzen! Kennzeichne diese durch Winkelb�gen!

\antwort{$cos(\gamma)$}
\end{beispiel}