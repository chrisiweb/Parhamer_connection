\section{AG 4.2 - 7 Winkelfunktionen im Einheitskreis - OA - BIFIE}

\begin{beispiel}[AG 4.2]{1} %PUNKTE DES BEISPIELS
In der nachstehenden Abbildung ist ein Winkelfunktionswert eines Winkels $\beta$  am Einheitskreis farbig dargestellt.

\begin{center}
\psset{xunit=1.0cm,yunit=1.0cm,algebraic=true,dimen=middle,dotstyle=o,dotsize=5pt 0,linewidth=0.8pt,arrowsize=3pt 2,arrowinset=0.25}
\begin{pspicture*}(-4.922229867450662,-5.2299631250666945)(5.663928810563728,4.897353945387463)
\psaxes[labelFontSize=\scriptstyle,xAxis=true,yAxis=true,labels=none,Dx=2.,Dy=2.,ticksize=0pt 0,]{}(0,0)(-4.922229867450662,-5.2299631250666945)(5.663928810563728,4.897353945387463)[x,140] [y,-40]
\pscircle(0.,0.){4.}
\psline[linewidth=1.6pt,linecolor=magenta](-2.398594516964261,0.)(-2.398594516964261,-3.201053630164447)
\rput[tl](0.6166581095909227,4.6351580341782945){1}
\rput[tl](4.352949407713648,0.7349938549419035){1}
\antwort{\psline[linewidth=1.6pt,linestyle=dashed,dash=4pt 4pt,linecolor=magenta](2.398594516964261,0.)(2.3985945169642604,-3.2010536301644477)
\psline[linewidth=1.6pt](-2.398594516964261,-3.201053630164447)(0.,0.)
\psline[linewidth=1.6pt,linestyle=dashed,dash=4pt 4pt](0.,0.)(2.3985945169642604,-3.2010536301644477)
\parametricplot{0.0}{4.069327012737334}{1.*1.2065988408734556*cos(t)+0.*1.2065988408734556*sin(t)+0.|0.*1.2065988408734556*cos(t)+1.*1.2065988408734556*sin(t)+0.}
\parametricplot[linestyle=dashed,dash=4pt 4pt]{0.0}{5.355450948032044}{1.*2.9764210347210627*cos(t)+0.*2.9764210347210627*sin(t)+0.|0.*2.9764210347210627*cos(t)+1.*2.9764210347210627*sin(t)+0.}
\begin{scriptsize}
\rput[bl](-0.6615468081879047,0.30892549922700363){$\beta_1$}
\rput[bl](-1.972526211037984,1.587130566371703){$\beta_2$}
\end{scriptsize}}
\end{pspicture*}
\end{center}

Gib an, um welche Winkelfunktion es sich dabei handelt, und zeichne alle Winkel im Einheitskreis ein, die diesen Winkelfunktionswert besitzen! Kennzeichne diese durch Winkelbögen!

\antwort{$sin(\beta)$}
\end{beispiel}