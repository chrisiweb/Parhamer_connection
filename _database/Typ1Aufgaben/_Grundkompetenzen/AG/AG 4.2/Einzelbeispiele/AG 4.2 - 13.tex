\section{AG 4.2 - 13 - MAT - Winkel mit gleichem Sinuswert - MC - Matura 1.NT 2018/19}

\begin{beispiel}[AG 4.2]{1}
Gegeben sei eine reelle Zahl $c$ mit $0<c<1$. Für die zwei unterschiedlichen Winkel $\alpha$ und $\beta$ soll gelten: $\sin(\alpha)=\sin(\beta)=c$.\\
Dabei soll $\alpha$ ein spitzer Winkel und $\beta$ ein Winkel aus dem Intervall $(0^\circ; 360^\circ)$ sein.

Welche Beziehung besteht zwischen den Winkeln $\alpha$ und $\beta$?\\
Kreuze die zutreffende Beziehung an.

\multiplechoice[6]{  %Anzahl der Antwortmoeglichkeiten, Standard: 5
				L1={$\alpha+\beta=90^\circ$},   %1. Antwortmoeglichkeit 
				L2={$\alpha+\beta=180^\circ$},   %2. Antwortmoeglichkeit
				L3={$\alpha+\beta=270^\circ$},   %3. Antwortmoeglichkeit
				L4={$\alpha+\beta=360^\circ$},   %4. Antwortmoeglichkeit
				L5={$\beta-\alpha=270^\circ$},	 %5. Antwortmoeglichkeit
				L6={$\beta-\alpha=180^\circ$},	 %6. Antwortmoeglichkeit
				L7={},	 %7. Antwortmoeglichkeit
				L8={},	 %8. Antwortmoeglichkeit
				L9={},	 %9. Antwortmoeglichkeit
				%% LOESUNG: %%
				A1=2,  % 1. Antwort
				A2=0,	 % 2. Antwort
				A3=0,  % 3. Antwort
				A4=0,  % 4. Antwort
				A5=0,  % 5. Antwort
				}
\end{beispiel}