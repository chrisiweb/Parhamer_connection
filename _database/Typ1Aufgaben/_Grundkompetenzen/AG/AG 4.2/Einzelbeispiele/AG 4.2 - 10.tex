\section{AG 4.2 - 10 - MAT - Koordinaten eines Punktes - OA - Matura 2016/17 - Haupttermin}

\begin{beispiel}[AG 4.2]{1} %PUNKTE DES BEISPIELS
In der unten stehenden Abbildung ist der Punkt $P = (-3|-2)$ dargestellt. \leer

Die Lage des Punktes $P$ kann auch durch die Angabe des Abstands $r = \overline{OP}$ und die Größe des Winkels $\varphi$ eindeutig festgelegt werden. \leer

\begin{center}
\psset{xunit=1.0cm,yunit=1.0cm,algebraic=true,dimen=middle,dotstyle=o,dotsize=5pt 0,linewidth=0.8pt,arrowsize=3pt 2,arrowinset=0.25}
\begin{pspicture*}(-3.5,-3.5)(3.5,3.5)
\psaxes[labelFontSize=\scriptstyle,xAxis=true,yAxis=true,labels=none,Dx=1.,Dy=1.,ticksize=0pt 0]{->}(0,0)(-3.5,-3.5)(3.5,3.5)[$x$,140] [$y$,-40]
\psline(0.,0.)(-3.,-2.)
\pscustom[linecolor=black,fillcolor=black,fillstyle=solid,opacity=0.1]{
\parametricplot{0.0}{3.7295952571373605}{0.6*cos(t)+0.|0.6*sin(t)+0.}
\lineto(0.,0.)\closepath}
\begin{footnotesize}
\psdots[dotsize=4pt 0,dotstyle=x](-3.,-2.)
\rput[bl](-3.34,-1.82){$P$}
\psdots[dotsize=1pt 0,dotstyle=*,linecolor=darkgray](0.,0.)
\rput[bl](0.08,-0.34){\darkgray{$O$}}
\rput[bl](-1.2,-1.24){$r$}
\rput[bl](0.15,0.16){$\varphi$}
\end{footnotesize}
\end{pspicture*}
\end{center}

Berechne die Größe des Winkels $\varphi$!

\antwort{Mögliche Berechnung:\leer

$\tan(\varphi-180^\circ)=\frac{2}{3} \Rightarrow \varphi \approx 213,69^\circ$ \leer

Lösungsschlüssel:

Ein Punkt für die richtige Lösung, wobei die Einheit "`Grad"' nicht angeführt sein muss.

Toleranzintervall: $[213^\circ; 214^\circ]$
}
\end{beispiel}