\section{AG 4.2 - 14 - Winkel im Einheitskreis - ZO - BarTri UNIVIE}

\begin{beispiel}[AG 4.2]{1}
Am Einheitskreis lassen sich die Winkelfunktionen abhängig vom Winkel $\varphi$ als vorzeichenbehaftete Achsenabschnitte interpretieren. Im Folgenden sind vier Werte der Winkelfunktionen und sechs Skizzen des Einheitskreises mit fett eingezeichneten Achsenabschnitten gegeben.\\ 
Ordne den Werten der Winkelfunktionen die richtigen Skizzen (von A bis F) zu.

\zuordnen{
				R1={$\sin\left(\frac{3}{2} \pi\right)$},				% Response 1
				R2={$\cos\left(\frac{6}{8} \pi\right)$},				% Response 2
				R3={$\sin\left(\frac{5}{4} \pi\right)$},				% Response 3
				R4={$\cos\left(\frac{5}{4} \pi\right)$},				% Response 4
				%% Moegliche Zuordnungen: %%
				A={\newrgbcolor{qqwuqq}{0. 0.39215686274509803 0.}
\psset{xunit=0.6cm,yunit=0.6cm,algebraic=true,dimen=middle,dotstyle=o,dotsize=5pt 0,linewidth=0.5pt,arrowsize=3pt 2,arrowinset=0.25}
\begin{pspicture*}(-2.1,-2.1)(2.1,2.1)
\psaxes[labelFontSize=\scriptstyle,xAxis=true,yAxis=true,labels=none,Dx=2.,Dy=2.,ticksize=-2pt 0,subticks=0]{}(0,0)(-2.1,-2.1)(2.1,2.1)
\pscircle[linewidth=2pt](0.,0.){1.224489}
\parametricplot{0.0}{1.5707963267948966}{0.8069951200501573*cos(t)+0.|0.8069951200501573*sin(t)+0.}
\psline[linewidth=3.6pt,linecolor=qqwuqq](0.,2.)(0.,0.)
\begin{scriptsize}
\rput[bl](0.2371738371468692,0.20480279571726812){$\varphi$}
\end{scriptsize}
\end{pspicture*}}, 				%Moeglichkeit A  
				B={\psset{xunit=0.6cm,yunit=0.6cm,algebraic=true,dimen=middle,dotstyle=o,dotsize=5pt 0,linewidth=0.5pt,arrowsize=3pt 2,arrowinset=0.25}
\begin{pspicture*}(-2.1,-2.1)(2.1,2.1)
\psaxes[labelFontSize=\scriptstyle,xAxis=true,yAxis=true,labels=none,Dx=2.,Dy=2.,ticksize=-2pt 0,subticks=0]{}(0,0)(-2.1,-2.1)(2.1,2.1)
\pscircle[linewidth=2pt](0.,0.){1.224489}
\parametricplot{0.0}{2.356194490192345}{0.8069951200501573*cos(t)+0.|0.8069951200501573*sin(t)+0.}
\psline[linewidth=0.4pt](0.,0.)(-1.414213562373095,1.4142135623730951)
\psline[linewidth=0.4pt](-1.414213562373095,1.4142135623730951)(-1.414213562373095,0.)
\psline[linewidth=3.6pt,linecolor=red](0.,0.)(-1.414213562373095,0.)
\begin{scriptsize}
\rput[bl](0.07577481313683776,0.311041591461183){$\varphi$}
\end{scriptsize}
\end{pspicture*}}, 				%Moeglichkeit B  
				C={\newrgbcolor{qqwuqq}{0. 0.39215686274509803 0.}
\psset{xunit=0.6cm,yunit=0.6cm,algebraic=true,dimen=middle,dotstyle=o,dotsize=5pt 0,linewidth=0.5pt,arrowsize=3pt 2,arrowinset=0.25}
\begin{pspicture*}(-2.1,-2.1)(2.1,2.1)
\psaxes[labelFontSize=\scriptstyle,xAxis=true,yAxis=true,labels=none,Dx=2.,Dy=2.,ticksize=-2pt 0,subticks=0]{}(0,0)(-2.1,-2.1)(2.1,2.1)
\pscircle[linewidth=2pt](0.,0.){1.224489}
\parametricplot{0.0}{3.9269908169872414}{0.8069951200501573*cos(t)+0.|0.8069951200501573*sin(t)+0.}
\psline[linewidth=0.5pt](0.,0.)(-1.4142135623730954,-1.414213562373095)
\psline[linewidth=3.6pt,linecolor=qqwuqq](-1.4142135623730954,0.)(-1.4142135623730954,-1.414213562373095)
\psline[linewidth=0.5pt](-1.4142135623730954,0.)(0.,0.)
\begin{scriptsize}
\rput[bl](-0.27392307221823037,0.311041591461183){$\varphi$}
\end{scriptsize}
\end{pspicture*}}, 				%Moeglichkeit C  
				D={\psset{xunit=0.6cm,yunit=0.6cm,algebraic=true,dimen=middle,dotstyle=o,dotsize=5pt 0,linewidth=0.5pt,arrowsize=3pt 2,arrowinset=0.25}
\begin{pspicture*}(-2.1,-2.1)(2.1,2.1)
\psaxes[labelFontSize=\scriptstyle,xAxis=true,yAxis=true,labels=none,Dx=2.,Dy=2.,ticksize=-2pt 0,subticks=0]{}(0,0)(-2.1,-2.1)(2.1,2.1)
\pscircle[linewidth=2pt](0.,0.){1.224489}
\parametricplot{0.0}{3.9269908169872414}{0.8069951200501573*cos(t)+0.|0.8069951200501573*sin(t)+0.}
\psline[linewidth=0.4pt](0.,0.)(-1.4142135623730954,-1.414213562373095)
\psline[linewidth=0.4pt](-1.4142135623730954,0.)(-1.4142135623730954,-1.414213562373095)
\psline[linewidth=3.6pt,linecolor=red](-1.4142135623730954,0.)(0.,0.)
\begin{scriptsize}
\rput[bl](-0.27392307221823037,0.311041591461183){$\varphi$}
\end{scriptsize}
\end{pspicture*}}, 				%Moeglichkeit D  
				E={\newrgbcolor{qqwuqq}{0. 0.39215686274509803 0.}
\psset{xunit=0.6cm,yunit=0.6cm,algebraic=true,dimen=middle,dotstyle=o,dotsize=5pt 0,linewidth=0.5pt,arrowsize=3pt 2,arrowinset=0.25}
\begin{pspicture*}(-2.1,-2.1)(2.1,2.1)
\psaxes[labelFontSize=\scriptstyle,xAxis=true,yAxis=true,labels=none,Dx=2.,Dy=2.,ticksize=-2pt 0,subticks=0]{}(0,0)(-2.1,-2.1)(2.1,2.1)
\pscircle[linewidth=2pt](0.,0.){1.224489}
\psline[linewidth=3.6pt,linecolor=qqwuqq](0.,0.)(0.,-2.)
\parametricplot{0.0}{4.71238898038469}{0.8069951200501573*cos(t)+0.|0.8069951200501573*sin(t)+0.}
\begin{scriptsize}
\rput[bl](-0.43532209622826185,0.20480279571726812){$\varphi$}
\end{scriptsize}
\end{pspicture*}}, 				%Moeglichkeit E  
				F={\newrgbcolor{qqwwtt}{0. 0.4 0.2}
\psset{xunit=0.6cm,yunit=0.6cm,algebraic=true,dimen=middle,dotstyle=o,dotsize=5pt 0,linewidth=0.5pt,arrowsize=3pt 2,arrowinset=0.25}
\begin{pspicture*}(-2.1,-2.1)(2.1,2.1)
\psaxes[labelFontSize=\scriptstyle,xAxis=true,yAxis=true,labels=none,Dx=2.,Dy=2.,ticksize=-2pt 0,subticks=0]{}(0,0)(-2.1,-2.1)(2.1,2.1)
\pscircle[linewidth=2pt](0.,0.){1.224489}
\parametricplot{0.0}{2.356194490192345}{0.8069951200501573*cos(t)+0.|0.8069951200501573*sin(t)+0.}
\psline[linewidth=0.4pt](0.,0.)(-1.414213562373095,1.4142135623730951)
\psline[linewidth=3.6pt,linecolor=qqwwtt](-1.414213562373095,1.4142135623730951)(-1.414213562373095,0.)
\psline[linewidth=0.4pt](0.,0.)(-1.414213562373095,0.)
\begin{scriptsize}
\rput[bl](0.07577481313683776,0.311041591461183){$\varphi$}
\end{scriptsize}
\end{pspicture*}}, 				%Moeglichkeit F  
				%% LOESUNG: %%
				A1={E},				% 1. richtige Zuordnung
				A2={B},				% 2. richtige Zuordnung
				A3={C},				% 3. richtige Zuordnung
				A4={D},				% 4. richtige Zuordnung
				}
\end{beispiel}