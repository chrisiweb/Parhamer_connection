\section{AG 4.2 - 9 - MAT - Winkel bestimmen - OA - Matura 2015/16 Nebentermin 1}

\begin{beispiel}[AG 4.2]{1} %PUNKTE DES BEISPIELS
Für einen Winkel $\alpha \in [0^\circ,~ 360^\circ)$ gilt: 

$\sin(\alpha)=0,4$ und $\cos(\alpha)<0$ \leer

Berechne den Winkel $\alpha$.


\antwort{$\sin(\alpha)=0,4 \Rightarrow \alpha_1\approx 23,6^\circ;~ \alpha_2 \approx 156,4^\circ$ 

$\cos(\alpha_1)>0;~ \cos(\alpha_2)<0 \Rightarrow \alpha = \alpha_2 \approx 156,4^\circ$ \leer

Lösungsschlüssel:

Ein Punkt für die richtige Lösung, wobei die Einheit "`Grad"' nicht angeführt sein muss. Eine korrekte Angabe der Lösung in einer anderen Einheit ist ebenfalls als richtig zu werten.
Toleranzintervall: $[156^\circ;~ 157^\circ]$}


\end{beispiel}