\section{AG 3.4 - 6 - Lagebeziehung zweier Geraden - LT - BIFIE}

\begin{beispiel}[AG 3.4]{1} %PUNKTE DES BEISPIELS
Gegeben sind die Geraden $g:X=\Vek{1}{1}{}+s\cdot\Vek{-1}{2}{}$ und $h:x-2\cdot y=-1$.

\lueckentext{
				text={Die Geraden $g$ und $h$ \gap, weil \gap.}, 	%Lueckentext Luecke=\gap
				L1={sind parallel}, 		%1.Moeglichkeit links  
				L2={sind ident}, 		%2.Moeglichkeit links
				L3={stehen normal aufeinander}, 		%3.Moeglichkeit links
				R1={der Richtungsvektor von $g$ zum Normalvektor von $h$ parallel ist}, 		%1.Moeglichkeit rechts 
				R2={die Richtungsvektoren der beiden Geraden $g$ und $h$ parallel sind}, 		%2.Moeglichkeit rechts
				R3={der Punkt $P=(1/1)$ auf beiden Geraden $g$ und $h$ liegt}, 		%3.Moeglichkeit rechts
				%% LOESUNG: %%
				A1=3,   % Antwort links
				A2=1		% Antwort rechts 
				}
\end{beispiel}