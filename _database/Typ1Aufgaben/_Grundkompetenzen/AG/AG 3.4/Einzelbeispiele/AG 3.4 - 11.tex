\section{AG 3.4 - 11 Gerade aufstellen - OA - BIFIE - Kompetenzcheck 2016}

\begin{beispiel}[AG 3.4]{1} %PUNKTE DES BEISPIELS
				In der nachstehenden Abbildung sind eine Gerade $g$ durch die Punkte $P$ und $Q$ sowie der Punkt $A$ dargestellt.

\begin{center}
\resizebox{0.7\linewidth}{!}{\newrgbcolor{cqcqcq}{0.7529411764705882 0.7529411764705882 0.7529411764705882}
\psset{xunit=1.0cm,yunit=1.0cm,algebraic=true,dimen=middle,dotstyle=o,dotsize=5pt 0,linewidth=0.8pt,arrowsize=3pt 2,arrowinset=0.25}
\begin{pspicture*}(-1.8946,-0.8563303092783504)(6.85676,5.927669690721655)
\multips(0,0)(0,1.0){7}{\psline[linestyle=dashed,linecap=1,dash=1.5pt 1.5pt,linewidth=0.4pt,linecolor=lightgray]{c-c}(-1.8946,0)(6.85676,0)}
\multips(-1,0)(1.0,0){9}{\psline[linestyle=dashed,linecap=1,dash=1.5pt 1.5pt,linewidth=0.4pt,linecolor=lightgray]{c-c}(0,-0.8563303092783504)(0,5.927669690721655)}
\psaxes[labelFontSize=\scriptstyle,xAxis=true,yAxis=true,Dx=1.,Dy=1.,ticksize=-2pt 0,subticks=2]{->}(0,0)(-1.8946,-0.8563303092783504)(6.85676,5.927669690721655)[x,140] [y,-40]
\psplot{-1.8946}{6.85676}{(--6.--1.*x)/3.}
\begin{scriptsize}
\psdots[dotsize=3pt 0,dotstyle=*](1.,5.)
\rput[bl](1.09036,5.147509690721654){$A = (1, 5)$}
\psdots[dotsize=3pt 0,dotstyle=*](0.,2.)
\rput[bl](0.1406,1.7046296907216516){$P = (0, 2)$}
\psdots[dotsize=3pt 0,dotstyle=*](3.,3.)
\rput[bl](3.12556,2.6543896907216524){$Q = (3, 3)$}
\rput[bl](-1.62324,1.5689496907216516){$g$}
\end{scriptsize}
\end{pspicture*}}
\end{center}

Ermittle eine Gleichung der Geraden $h$, die durch $A$ verläuft und normal zu $g$ ist.\\

\antwort{$h:3x+y=8$\\
oder: $h:X=\Vek{1}{5}{}+t\cdot\Vek{1}{-3}{}$ mit $t\in\mathbb{R}$\\

Ein Punkt für eine korrekte Gleichung bzw. eine korrekte Parameterdarstellung der Geraden $h$, wobei $t\in\mathbb{R}$ nicht angegeben sein muss.}
\end{beispiel}	