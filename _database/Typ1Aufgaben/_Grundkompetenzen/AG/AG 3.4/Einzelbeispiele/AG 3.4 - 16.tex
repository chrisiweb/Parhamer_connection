\section{AG 3.4 - 16 - MAT - Parallele Gerade - OA - Matura NT 2 15/16}

\begin{beispiel}[AG 3.4]{1} %PUNKTE DES BEISPIELS
Gegeben ist die Gerade $g:X=\Vek{1}{-2}{}+s\cdot\Vek{2}{3}{}$.

Die Gerade $h$ verläuft parallel zu $g$ durch den Koordinatenursprung.

Gib die Gleichung der Geraden $h$ in der Form $a\cdot x+b\cdot y=c$ mit $a,b,c\in\mathbb{R}$ an.

$h:$ \antwort[\rule{3cm}{0.3pt}]{$3\cdot x-2\cdot y=0$}
\end{beispiel}