\section{AG 3.4 - 14 - MAT - Archäologie - OA - Matura 2014/15 Kompensationsprüfung}

\begin{beispiel}[AG 3.4]{1} %PUNKTE DES BEISPIELS
				Gegeben sind zwei Geraden $g$ und $h$ in $\mathbb{R}^{3}$.
				Die Gerade $g$ ist durch eine Parameterdarstellung $X=\Vek{3}{-4}{-7}+t\cdot\Vek{1}{-1}{-2}$ mit $t\in\mathbb{R}$ festgelegt.
				Die Gerade $h$ verläuft durch die Punkte $A=(0|8|0)$ und $B=(-2|28|6)$.
				
				Ermittle die Koordinaten des Schnittpunktes dieser beiden Geraden.
				
				\antwort{Mögliche Berechnung:\\
				
				$h:X=\Vek{0}{8}{0}+s\cdot\Vek{-2}{20}{6}$\\
				
				$I:3+t=-2s$\\
				$II:-4-t=8+20s$\\
				$III:-7-2t=6s$\\
				
				$\Rightarrow t=-2$ bzw. $s=-0,5\Rightarrow S=(1|-2|-3)$}
\end{beispiel}