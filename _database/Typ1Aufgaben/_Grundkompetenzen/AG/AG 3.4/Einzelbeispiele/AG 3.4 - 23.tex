\section{AG 3.4 - 23 - MAT - Gleichung einer Geraden aufstellen - OA - Matura 1.NT 2018/19}

\begin{beispiel}[AG 3.4]{1}
Die Punkte $A=(7\mid 6)$, $M=(-1\mid 7)$ und $N=(8\mid 1)$ sind gegeben.\\
Eine Gerade $g$ verläuft durch den Punkt $A$ und steht normal auf die Verbindungsgerade durch die Punkte $M$ und $N$.

Gib eine Gleichung der Geraden $g$ an.

\antwort{$g$: $3\cdot x-2\cdot y=9$

oder:

$g$: $X=\Vek{7}{6}{}+t\cdot\Vek{6}{9}{}$ mit $t\in\mathbb{R}$\leer

Ein Punkt für die richtige Gleichung bzw. eine korrekte Parameterdarstellung der Geraden $g$, wobei "`$t\in\mathbb{R}$"' nicht angegeben sein muss.}
\end{beispiel}