\section{AG 3.4 - 2 - Idente Geraden - OA - BIFIE}

\begin{beispiel}[AG 3.4]{1} %PUNKTE DES BEISPIELS
			Gegeben sind die beiden Geraden
\[g:X=P+t\cdot \Vek{g_{1}}{g_{2}}{g_{3}}\]
und
\[h:X=Q+s\cdot \Vek{h_{1}}{h_{2}}{h_{3}}\]
mit $t,s \in \mathbb{R}$.
Gib an, welche Schritte notwendig sind, um die Identität der Geraden nachzuweisen.

\antwort{Wenn der Richtungsvektor der Geraden g ein Vielfaches des Richtungsvektors der Geraden ha ist (bzw. umgekehrt h ein Vielfaches von g ist), so sind die beiden Geraden parallel oder ident. Liegt außerdem noch der Punkt P auf der Geraden h (seine Koordinaten müssen die Gleichung \[P=Q+s\cdot \Vek{h_{1}}{h_{2}}{h_{3}}\] erfüllen) bzw. liegt der Punkt Q auf der Geraden g (seine Koordinaten müssen die Gleichung \[Q=P+t\cdot \Vek{g_{1}}{g_{2}}{g_{3}}\] erfüllen), so sind die Geraden ident.}
\end{beispiel}