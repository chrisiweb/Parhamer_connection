\section{AG 3.4 - 21 - MAT - Parallele Geraden - MC - Matura 2. NT 2017/18}

\begin{beispiel}[AG 3.4]{1}
Gegeben sind die Parameterdarstellungen zweier Geraden $g\!:~ X=P+t\cdot \vec{u}$ und $h\!:~X=Q+s\cdot \vec{v}$ mit $s, t \in \mathbb{R}$ und $\vec{u}, \vec{v}\neq \Vek{0}{0}{}$.

Welche der nachstehend angeführten Aussagen sind unter der Voraussetzung, dass die beiden Geraden zueinander parallel, aber nicht identisch sind, stets zutreffend?

Kreuze die beiden zutreffenden Aussagen an!

\multiplechoice[5]{  %Anzahl der Antwortmoeglichkeiten, Standard: 5
				L1={$P=Q$},   %1. Antwortmoeglichkeit 
				L2={$P\in h$},   %2. Antwortmoeglichkeit
				L3={$Q\notin g$},   %3. Antwortmoeglichkeit
				L4={$\vec{u}\cdot \vec{v}=0$},   %4. Antwortmoeglichkeit
				L5={$\vec{u}=a\cdot \vec{v}$ für ein $a\in \mathbb{R}\backslash\{0\}$},	 %5. Antwortmoeglichkeit
				L6={},	 %6. Antwortmoeglichkeit
				L7={},	 %7. Antwortmoeglichkeit
				L8={},	 %8. Antwortmoeglichkeit
				L9={},	 %9. Antwortmoeglichkeit
				%% LOESUNG: %%
				A1=3,  % 1. Antwort
				A2=5,	 % 2. Antwort
				A3=0,  % 3. Antwort
				A4=0,  % 4. Antwort
				A5=0,  % 5. Antwort
				}
\end{beispiel}