\section{AG 3.4 - 22 - MAT - Parameterdarstellung einer Geraden - OA - Matura-HT-18/19}

\begin{beispiel}[AG 3.4]{1}
In der nachstehenden Abbildung ist eine Gerade $g$ dargestellt. Die gekennzeichneten Punkte der Geraden $g$ haben ganzzahlige Koordinaten.

\begin{center}
\psset{xunit=1.0cm,yunit=1.0cm,algebraic=true,dimen=middle,dotstyle=o,dotsize=7pt 0,linewidth=0.8pt,arrowsize=3pt 2,arrowinset=0.25}
\begin{pspicture*}(-5.5,-5.5)(5.5,5.5)
\multips(0,-5)(0,1.0){12}{\psline[linestyle=dashed,linecap=1,dash=1.5pt 1.5pt,linewidth=0.4pt,linecolor=gray]{c-c}(-5.5,0)(5.5,0)}
\multips(-5,0)(1.0,0){12}{\psline[linestyle=dashed,linecap=1,dash=1.5pt 1.5pt,linewidth=0.4pt,linecolor=gray]{c-c}(0,-5.5)(0,5.5)}
\psaxes[labelFontSize=\scriptstyle,xAxis=true,yAxis=true,showorigin=false,Dx=1.,Dy=1.,ticksize=-2pt 0,subticks=2]{->}(0,0)(-5.5,-5.5)(5.5,5.5)[$x$,140] [$y$,-40]
\psplot[linewidth=0.8pt]{-5.5}{5.5}{(--2.-4.*x)/6.}
\begin{scriptsize}
\psdots[dotsize=5pt 0,dotstyle=*](-1.,1.)
\psdots[dotsize=5pt 0,dotstyle=*](5.,-3.)
\rput[bl](-4.48,3.6){$g$}
\end{scriptsize}
\end{pspicture*}
\end{center}

Vervollständige folgende Parameterdarstellung der Geraden $g$ durch Angaben der Werte für $a$ und $b$ mit $a,b \in \mathbb{R}$!

$g:~ X=\Vek{a}{3}{}+t\cdot \Vek{3}{b}{}$ mit $t\in \mathbb{R}$ \leer

$a=\antwort[\rule{4cm}{0.3pt}]{-4}$\leer

$b=\antwort[\rule{4cm}{0.3pt}]{-2}$
\end{beispiel}