\section{AG 3.4 - 19 - MAT - Parallelität von Geraden - OA - Matura 2016/17 Haupttermin}

\begin{beispiel}[AG 3.4]{1} %PUNKTE DES BEISPIELS
Gegeben sind folgende Parameterdarstellungen der Geraden $g$ und $h$:\leer

$g:~ X=\Vek{1}{1}{1}+t\cdot \Vek{-3}{1}{2} \qquad$ mit $t\in \mathbb{R}$

$h:~ X=\Vek{3}{1}{1}+s\cdot \Vek{6}{h_y}{h_z} \qquad$ mit $s\in \mathbb{R}$ \leer

Bestimme die Koordinaten $h_y$ und $h_z$ des Richtungsvektors der Geraden $h$ so, dass die
Gerade $h$ zur Geraden $g$ parallel ist!

\antwort{\leer

$h_y=-2$

$h_z=-4$}

\end{beispiel}