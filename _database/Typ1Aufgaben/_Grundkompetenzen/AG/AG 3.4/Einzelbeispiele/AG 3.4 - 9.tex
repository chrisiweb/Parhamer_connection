\section{AG 3.4 - 9 - Punkt und Gerade - OA - BIFIE}

\begin{beispiel}[AG 3.4]{1} %PUNKTE DES BEISPIELS
	Gegeben sind der Punkt $P = (-1|5|6)$ und die Gerade $g$, die durch die Punkte $A = (2|-3|2)$
und $B = (5|1|0)$ verläuft. 	

\leer

Geben Sie an, ob der gegebene Punkt $P$ auf der Geraden $g$ liegt, und überprüfen Sie diese Aussage anhand einer Rechnung! 


\antwort{Der Punkt $P$ liegt nicht auf der Geraden $g$, denn: \\

$g:X=\Vek{2}{-3}{2} + s \cdot \Vek{3}{4}{-2}\qquad \vek{AP}=\Vek{-3}{8}{4}, \vek{AB}=\Vek{3}{4}{-2}$
\leer

Die Überprüfung, ob $\vek{AP}\parallel\vek{AB}$ gilt,ergibt, dass $\vek{AP}$ kein Vielfaches von $\vek{AB}\Rightarrow P \notin g$ ist. Alternativ kann man auch rechnerisch zeigen, dass es keinen Wert für $s$ gibt, sodass die Gleichung $\Vek{-1}{5}{6}=\Vek{2}{-3}{2} + s\cdot \Vek{3}{4}{-2}$ erfüllt ist.}		
\end{beispiel}