\section{AG 3.4 - 20 - Zur x-Achse parallel Gerade - OA - Matura - 1. NT 2017/18}

\begin{beispiel}[AG 3.4]{1}
Gegeben ist eine Gerade $g$ mit der Parameterdarstellung $g\!:X=\Vek{2}{1}{}+t\cdot\vec{a}$ mit $t\in\mathbb{R}$.

Gib einen Vektor $\vec{a}\in\mathbb{R}^2$ mit $\vec{a}\neq\Vek{0}{0}{}$ so an, dass die Gerade $g$ parallel zur $x$-Achse verläuft!

$\vec{a}=$\,\antwort[\rule{3cm}{0.3pt}]{$\Vek{1}{0}{}$}
\end{beispiel}