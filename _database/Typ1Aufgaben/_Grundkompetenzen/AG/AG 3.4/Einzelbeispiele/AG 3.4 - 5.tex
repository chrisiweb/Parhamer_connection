\section{AG 3.4 - 5 Geraden im R3 - MC - BIFIE}

\begin{beispiel}[AG 3.4]{1} %PUNKTE DES BEISPIELS
Gegeben ist die Gerade $g$ mit der Gleichung $X=\Vek{4}{2}{4}+t\cdot\Vek{1}{-1}{2}$ mit $t\in\mathbb{R}$.

Zwei der folgenden Gleichungen sind ebenfalls Parameterdarstellungen der Geraden $g$. Kreuze die beiden Gleichungen an!
\multiplechoice[5]{  %Anzahl der Antwortmoeglichkeiten, Standard: 5
				L1={$X=\Vek{4}{2}{4}+t\cdot\Vek{2}{-1}{3}$ mit $t\in\mathbb{R}$},   %1. Antwortmoeglichkeit 
				L2={$X=\Vek{5}{7}{9}+t\cdot\Vek{2}{-2}{4}$ mit $t\in\mathbb{R}$},   %2. Antwortmoeglichkeit
				L3={$X=\Vek{6}{0}{8}+t\cdot\Vek{1}{-1}{2}$ mit $t\in\mathbb{R}$},   %3. Antwortmoeglichkeit
				L4={$X=\Vek{4}{2}{4}+t\cdot\Vek{-1}{1}{-2}$ mit $t\in\mathbb{R}$},   %4. Antwortmoeglichkeit
				L5={$X=\Vek{3}{3}{2}+t\cdot\Vek{1}{0}{1}$ mit $t\in\mathbb{R}$},	 %5. Antwortmoeglichkeit
				L6={},	 %6. Antwortmoeglichkeit
				L7={},	 %7. Antwortmoeglichkeit
				L8={},	 %8. Antwortmoeglichkeit
				L9={},	 %9. Antwortmoeglichkeit
				%% LOESUNG: %%
				A1=3,  % 1. Antwort
				A2=4,	 % 2. Antwort
				A3=0,  % 3. Antwort
				A4=0,  % 4. Antwort
				A5=0,  % 5. Antwort
				}
\end{beispiel}