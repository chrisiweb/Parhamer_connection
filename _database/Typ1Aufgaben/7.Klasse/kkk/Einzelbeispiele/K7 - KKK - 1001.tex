\section{K6 - KKK - 1001 - Eigenschaften einer Ellipse - MC - Dimensionen 7 Schulbuch}

\begin{beispiel}[K6 - KKK]{1} %PUNKTE DES BEISPIELS
				Kreuze die beiden Aussagen an, die für Ellipsen mit der Gleichung\\ ell: $b^2x^2+a^2y^2=a^2b^2$ zutreffen.
				
				\multiplechoice[5]{  %Anzahl der Antwortmoeglichkeiten, Standard: 5
								L1={Eine Ellipse, für die $a=b$ gilt, ist ein Kreis.},   %1. Antwortmoeglichkeit 
								L2={Die Summe der Abstände von einem Punkt $X$ der Ellipse zu den beiden Brennpunkten beträgt $a$.},   %2. Antwortmoeglichkeit
								L3={Die beiden Brennpunkte liegen symmetrisch zum Mittelpunkt einer Ellipse.},   %3. Antwortmoeglichkeit
								L4={Die Summe der Abstände von einem Nebenscheitel der Ellipse zu den beiden Hauptscheiteln beträgt 2a.},   %4. Antwortmoeglichkeit
								L5={Für jede Ellipse gilt die Gleichung: $e^2=a^2+b^2$},	 %5. Antwortmoeglichkeit
								L6={},	 %6. Antwortmoeglichkeit
								L7={},	 %7. Antwortmoeglichkeit
								L8={},	 %8. Antwortmoeglichkeit
								L9={},	 %9. Antwortmoeglichkeit
								%% LOESUNG: %%
								A1=1,  % 1. Antwort
								A2=3,	 % 2. Antwort
								A3=0,  % 3. Antwort
								A4=0,  % 4. Antwort
								A5=0,  % 5. Antwort
								}
\end{beispiel}