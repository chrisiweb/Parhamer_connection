\section{K7 - KKK -  - 1047 - Parameterdarstellung - OA - Dimensionen Mathematik, Schularbeiten-Trainer 7. Klasse}

\begin{beispiel}[K7 - KKK]{1} %PUNKTE DES BEISPIELS
Eine Kurve $k$ wird durch nachfolgende Parameterdarstellung beschrieben:

$$k=\{X\in\mathbb{R}^2|X=(t^2\,|\,2t-1),t\in\mathbb{R}^+\}$$

Stelle die Kurve im abgebildeten Koordinatensystem dar. Begründe, dass die Kurve als Graph einer Funktion $f$ aufgefasst werden kann, und gib die Funktionsgleichung von $f$ an.

\begin{center}
	\resizebox{0.5\linewidth}{!}{\psset{xunit=1.0cm,yunit=1.0cm,algebraic=true,dimen=middle,dotstyle=o,dotsize=5pt 0,linewidth=1.6pt,arrowsize=3pt 2,arrowinset=0.25}
\begin{pspicture*}(-2.42,-2.31)(4.94,5.79)
\multips(0,-2)(0,0.5){17}{\psline[linestyle=dashed,linecap=1,dash=1.5pt 1.5pt,linewidth=0.4pt,linecolor=gray]{c-c}(-2.42,0)(4.94,0)}
\multips(-2,0)(0.5,0){15}{\psline[linestyle=dashed,linecap=1,dash=1.5pt 1.5pt,linewidth=0.4pt,linecolor=gray]{c-c}(0,-2.31)(0,5.79)}
\psaxes[labelFontSize=\scriptstyle,xAxis=true,yAxis=true,Dx=1.,Dy=1.,ticksize=-2pt 0,subticks=2]{->}(0,0)(-2.42,-2.31)(4.94,5.79)[x,140] [y,-40]
\antwort{\psplot[linewidth=1.2pt,plotpoints=200]{0}{4.94}{2.0*sqrt(x)-1.0}}
\end{pspicture*}}
\end{center}

\antwort{Die Gleichung der zugehörigen Funktion lautet: $y=2\cdot\sqrt{x}-1$.

(Wurzelfunktion, die jedem $x\in\mathbb{R}^+$ genau einen $y$-Wert zuordnet)}
				\end{beispiel}