\section{K7 - KKK - 1051 - Halbachsenlänge b - OA - Mathematik verstehen}

\begin{beispiel}[K7 - KKK]{1}
Das 1. Keplersche Gesetz lautet: "`Die Bahnen der Planeten sind Ellipsen, wobei die Sonne in einem Brennpunkt steht."'

Für den Planeten Neptun ist die Halbachsenlänge $a=29,80$\,AE und die sogenannte numerische Exzentrizität $\varepsilon=\frac{e}{a}=0,0086$ gegeben. 

Entfernung werden in astronomischen Eineiheiten (AE) gemessen:\\
1\,AE = Länge der großen Halbachse der Erdumlaufbahn $\approx 149597871$\,km

Berechne die Halbachsenlänge $b$ (in Kilometern).\leer

\antwort{$e=a\cdot 0,0086=29,80\cdot 0,0086=0.25628$

$e^2=a^2-b^2 \Rightarrow b^2=888.1056794384 \Rightarrow b=29.80110198362..$}
\end{beispiel}