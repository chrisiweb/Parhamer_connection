\section{K7 - KKK - 1019 Kugel - Thema Mathematik Schularbeiten 7. Klasse}

\begin{beispiel}[K7 - KKK]{1} %PUNKTE DES BEISPIELS
			Gegeben sind zwei Kugeln $K_1\,[M_1(x_1|y_1|z_1);r_1]$ und $K_2\,[M_2(x_2|y_2|z_2);r_2]$ mit Zentralabstand $z=|\overrightarrow{M_1M_2}|$.
			
			\lueckentext{
							text={Die zwei Kugeln \gap, wenn \gap ist.}, 	%Lueckentext Luecke=\gap
							L1={schneiden einander in einer Kreislinie}, 		%1.Moeglichkeit links  
							L2={berühren einander von außen}, 		%2.Moeglichkeit links
							L3={haben keine gemeinsamen Punkte}, 		%3.Moeglichkeit links
							R1={$r_1+r_2<z$}, 		%1.Moeglichkeit rechts 
							R2={$|r_1-r_2|=z$}, 		%2.Moeglichkeit rechts
							R3={$r_1=r_2$}, 		%3.Moeglichkeit rechts
							%% LOESUNG: %%
							A1=3,   % Antwort links
							A2=1		% Antwort rechts 
							}
			\end{beispiel}