\section{K7 - KKK - 1 - Eigenschaften Kegelschnitte - MC - MatKon}

\begin{beispiel}[K7 - KKK]{1}
Kreuze die zutreffende Aussage an!
			
			\multiplechoice[6]{  %Anzahl der Antwortmoeglichkeiten, Standard: 5
							L1={Schneidet man eine Parabel in 1. Hauptlage mit einer Ellipse in 1. Hauptlage erhält man als Schnittmenge die leere Menge.},   %1. Antwortmoeglichkeit 
							L2={Eine Parallele zu einer Passante eines Kreises ist stets eine Kreistangente.},   %2. Antwortmoeglichkeit
							L3={Jede zur Leitlinie einer Parabel parallele Gerade ist eine Sekante.},   %3. Antwortmoeglichkeit
							L4={Eine Normale auf die Leitlinie einer Parabel kann eine Tangente sein.},   %4. Antwortmoeglichkeit
							L5={Die Asymptote einer Hyperbel ist stets eine Passante.},	 %5. Antwortmoeglichkeit
							L6={Eine Sekante, die einen Hyperbelast in 2 Punkten schneidet, verläuft immer durch einen der beiden Brennpunkte},	 %6. Antwortmoeglichkeit
							L7={},	 %7. Antwortmoeglichkeit
							L8={},	 %8. Antwortmoeglichkeit
							L9={},	 %9. Antwortmoeglichkeit
							%% LOESUNG: %%
							A1=5,  % 1. Antwort
							A2=0,	 % 2. Antwort
							A3=0,  % 3. Antwort
							A4=0,  % 4. Antwort
							A5=0,  % 5. Antwort
							}
\end{beispiel}