\section{K7 - KKK - 1011 Parabel - Thema Mathematik Schularbeiten 7. Klasse}

\begin{beispiel}[K7 - KKK]{1} %PUNKTE DES BEISPIELS
			Kreuze die zutreffe(n) Aussage(n) an!
			
			\multiplechoice[5]{  %Anzahl der Antwortmoeglichkeiten, Standard: 5
							L1={Eine Parabel besteht aus zwei Parabelästen.},   %1. Antwortmoeglichkeit 
							L2={Die Symmetrieachse einer Parabel verläuft normal zur Leitlinie.},   %2. Antwortmoeglichkeit
							L3={Eine Parabel schneidet die Leitlinie im Brennpunkt.},   %3. Antwortmoeglichkeit
							L4={Durch die Lage des Brennpunktes ist eine Parabel in der 1. Hauptlage eindeutig bestimmt.},   %4. Antwortmoeglichkeit
							L5={Eine Parabel in 2. Hauptlage stellt den Graphen einer Funktion dar.},	 %5. Antwortmoeglichkeit
							L6={},	 %6. Antwortmoeglichkeit
							L7={},	 %7. Antwortmoeglichkeit
							L8={},	 %8. Antwortmoeglichkeit
							L9={},	 %9. Antwortmoeglichkeit
							%% LOESUNG: %%
							A1=2,  % 1. Antwort
							A2=4,	 % 2. Antwort
							A3=5,  % 3. Antwort
							A4=0,  % 4. Antwort
							A5=0,  % 5. Antwort
							}
			\end{beispiel}