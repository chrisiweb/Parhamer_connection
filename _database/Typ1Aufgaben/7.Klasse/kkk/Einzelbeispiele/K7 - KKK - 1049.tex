\section{K7 - KKK -  - 1049 - Parameterdarstellung - OA - Dimensionen Mathematik, Schularbeiten-Trainer 7. Klasse}

\begin{beispiel}[K7 - KKK]{1} %PUNKTE DES BEISPIELS
Die unten abgebildete Gerade $g$ wird durch nachfolgende Parameterdarstellung beschrieben:

$$g=\{X\in\mathbb{R}^2\,|\,X=(t+1\,|\,f(t)),t\in\mathbb{R}\}$$

\begin{center}
	\resizebox{0.5\linewidth}{!}{\psset{xunit=1.0cm,yunit=1.0cm,algebraic=true,dimen=middle,dotstyle=o,dotsize=5pt 0,linewidth=1.6pt,arrowsize=3pt 2,arrowinset=0.25}
\begin{pspicture*}(-3.6,-2.4)(3.48,4.44)
\multips(0,-2)(0,0.5){14}{\psline[linestyle=dashed,linecap=1,dash=1.5pt 1.5pt,linewidth=0.4pt,linecolor=gray]{c-c}(-3.6,0)(3.48,0)}
\multips(-3,0)(0.5,0){15}{\psline[linestyle=dashed,linecap=1,dash=1.5pt 1.5pt,linewidth=0.4pt,linecolor=gray]{c-c}(0,-2.4)(0,4.44)}
\psaxes[labelFontSize=\scriptstyle,xAxis=true,yAxis=true,Dx=1.,Dy=1.,ticksize=-2pt 0,subticks=2]{->}(0,0)(-3.6,-2.4)(3.48,4.44)[x,140] [y,-40]
\psplot[linewidth=0.8pt]{-3.6}{3.48}{(--0.5--1.*x)/0.5}
\rput[tl](1.34,3.46){g}
\end{pspicture*}}
\end{center}

Gib die Gleichung f�r $f(t)$ an.\leer

\antwort{$f(t)=2t+3$}
				\end{beispiel}