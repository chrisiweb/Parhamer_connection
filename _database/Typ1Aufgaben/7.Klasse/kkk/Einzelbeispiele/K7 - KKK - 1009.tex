\section{K7 - KKK - 1009 Lage zweier Kreise - Thema Mathematik Schularbeiten 7. Klasse}

\begin{beispiel}[K7 - KKK]{1} %PUNKTE DES BEISPIELS
			Gegeben sind zwei Kreise mit Mittelpunkt und Radius $k_1 [M_1;r_1]$ und $k_2 [M_2;r_2]$.
			
			\lueckentext{
							text={Die zu $k_1$ und $k_2$ gehördenden Kreisflächen haben \gap gemeinsam, wenn \gap.}, 	%Lueckentext Luecke=\gap
							L1={keinen Punkt}, 		%1.Moeglichkeit links  
							L2={genau einen Punkt}, 		%2.Moeglichkeit links
							L3={mehrere Punkte}, 		%3.Moeglichkeit links
							R1={$r_1<r_2$}, 		%1.Moeglichkeit rechts 
							R2={$|\vec{M_1M_2}|\geq r_1+r_2$}, 		%2.Moeglichkeit rechts
							R3={$|\vec{M_1M_2}|= r_1+r_2$}, 		%3.Moeglichkeit rechts
							%% LOESUNG: %%
							A1=2,   % Antwort links
							A2=3		% Antwort rechts 
							}
			\end{beispiel}