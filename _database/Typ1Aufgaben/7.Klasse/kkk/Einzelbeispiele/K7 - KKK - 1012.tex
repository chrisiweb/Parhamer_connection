\section{K7 - KKK - 1012 Ellipse - Thema Mathematik Schularbeiten 7. Klasse}

\begin{beispiel}[K7 - KKK]{1} %PUNKTE DES BEISPIELS
			Kreuze die beiden Aussagen an, die für eine Ellipse mit der Gleichung \mbox{$b²x²+a²y²=a²b²$} zutreffen!
			
			\multiplechoice[5]{  %Anzahl der Antwortmoeglichkeiten, Standard: 5
							L1={Die Differenz der Brennstrecken ist konstant.},   %1. Antwortmoeglichkeit 
							L2={Eine Ellipse, für die $a=b$ gilt, ist ein Kreis.},   %2. Antwortmoeglichkeit
							L3={Die Summe der Abstände von einem Punkt $X$ der Ellipse zu den beiden Brennpunkten beträgt $a$.},   %3. Antwortmoeglichkeit
							L4={Die beiden Brennpunkte liegen symmetrisch zum Mittelpunkt der Ellipse.},   %4. Antwortmoeglichkeit
							L5={Die Summe der Abstände von einem Nebenscheitel der Ellipse zu den beiden Hauptscheiteln beträgt $2a$.},	 %5. Antwortmoeglichkeit
							L6={},	 %6. Antwortmoeglichkeit
							L7={},	 %7. Antwortmoeglichkeit
							L8={},	 %8. Antwortmoeglichkeit
							L9={},	 %9. Antwortmoeglichkeit
							%% LOESUNG: %%
							A1=2,  % 1. Antwort
							A2=4,	 % 2. Antwort
							A3=0,  % 3. Antwort
							A4=0,  % 4. Antwort
							A5=0,  % 5. Antwort
							}
			\end{beispiel}