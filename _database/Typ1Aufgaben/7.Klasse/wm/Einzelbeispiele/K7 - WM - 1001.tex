\section{K7 - WM - 1001 - Gewinn berechnen - OA - BHS-Aufgabenpool}

\begin{beispiel}[K7 - WM]{1}
Ein Unternehmen stellt B�geleisen her. Die Produktionskosten lassen sich n�herungsweise durch die folgende Funktion $K$ beschreiben:\\
				$K(x)=0,001\cdot x^3-0,03\cdot x^2+0,8\cdot x+69$ mit $x\geq 0$
				
				$x$ ... Prodkutionsmenge in Mengeneinheiten (ME)\\
				$K(x)$ ... Kosten der Produktionsmenge $x$ in Geldeinheiten (GE)
				
				Der Graph der Erl�sfunktion $E$ mit $E(x)=a\cdot x^2+b\cdot x$ f�r den Absatz von B�geleisen ist in der nachstehenden Grafik dargestellt:
				\begin{center}
					\resizebox{0.5\linewidth}{!}{\psset{xunit=0.1cm,yunit=0.02cm,algebraic=true,dimen=middle,dotstyle=o,dotsize=5pt 0,linewidth=1.6pt,arrowsize=3pt 2,arrowinset=0.25}
\begin{pspicture*}(-7.315862068965503,-21.394285714284695)(124.93241379310332,288.8228571428449)
\multips(0,0)(0,50.0){7}{\psline[linestyle=dashed,linecap=1,dash=1.5pt 1.5pt,linewidth=0.4pt,linecolor=darkgray]{c-c}(0,0)(124.93241379310332,0)}
\multips(0,0)(10.0,0){14}{\psline[linestyle=dashed,linecap=1,dash=1.5pt 1.5pt,linewidth=0.4pt,linecolor=darkgray]{c-c}(0,0)(0,288.8228571428449)}
\psaxes[labelFontSize=\scriptstyle,xAxis=true,yAxis=true,Dx=10.,Dy=50.,ticksize=-2pt 0,subticks=2]{->}(0,0)(0.,0.)(124.93241379310332,288.8228571428449)[Menge in ME,140] [Erl�s in GE,-40]
\psplot[linewidth=2.pt,plotpoints=200]{0}{100}{-0.1*(x-50.0)^(2.0)+250.0}
\rput[tl](81.41241379310338,178.28571428570675){E}
\end{pspicture*}}
				\end{center}
				
				Berechne die H�he des Gewinns bei einer Produktion von 50\,ME, wenn die oben definierte Kostenfunktion $K$ zugrunde gelegt wird.
				
				\antwort{$K(50)=159, E(50)=250, G(50)=E(50)-K(50)=91$}
\end{beispiel}