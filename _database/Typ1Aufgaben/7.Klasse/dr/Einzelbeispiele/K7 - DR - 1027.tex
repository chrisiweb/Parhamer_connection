\section{K7 - DR -  - 1027 - Stetig - OA - Dimensionen Mathematik, Schularbeiten-Trainer 7. Klasse}

\begin{beispiel}[K7 - DR]{1} %PUNKTE DES BEISPIELS
Gegeben ist die Funktion $f$ mit $f(x)=\begin{cases}
x^2-1,&x\leq 1\\
-2x+d,&x>1\\
\end{cases}$.

Ermittle den Wert des Parameters $d$ so, dass die Funktion $f$ im Definitionsbereich $D=\mathbb{R}$ stetig ist.\leer

\antwort{$f(1)=0$

Damit an der Stelle 1 keine Unstetigkeitsstelle vorliegt, muss auch der Term $-2x+d$ an dieser Stelle den Wert 0 annehmen: $0=-2\cdot 1+d \Rightarrow d=2$}
				
				\end{beispiel}