\section{K7 - DR - 1004 Abschuss eines Körpers - ZO - Thema Mathematik Schularbeiten 7. Klasse}

\begin{beispiel}[K7 - DR]{1} %PUNKTE DES BEISPIELS
			 Ein Körper wird zum Zeitpunkt $t=0$ senkrecht nach oben abgeschossen. Er befindet sich nach $t$ Sekunden $h(t)$ Meter über dem Erdboden.
			
			Ordne den angeführten verbalen Beschreibungen die richtigen Terme zu!
			
			\zuordnen[0.2]{
							R1={Durchschnittsgeschwindigkeit in den ersten 3 Sekunden},				% Response 1
							R2={Absolute Zunahme der Geschwindigkeit in den ersten 3 Sekunden},				% Response 2
							R3={Momentangeschwindigkeit nach 3 Sekunden},				% Response 3
							R4={Entfernung vom Erdboden nach 3 Sekunden},				% Response 4
							%% Moegliche Zuordnungen: %%
							A={$h'(3)$}, 				%Moeglichkeit A  
							B={$h''(3)-h''(0)$}, 				%Moeglichkeit B  
							C={$h(3)$}, 				%Moeglichkeit C  
							D={$h''(3)$}, 				%Moeglichkeit D  
							E={$h'(3)-h'(0)$}, 				%Moeglichkeit E  
							F={$\frac{h(3)-h(0)}{3}$}, 				%Moeglichkeit F  
							%% LOESUNG: %%
							A1={F},				% 1. richtige Zuordnung
							A2={E},				% 2. richtige Zuordnung
							A3={A},				% 3. richtige Zuordnung
							A4={C},				% 4. richtige Zuordnung
							}
			\end{beispiel}