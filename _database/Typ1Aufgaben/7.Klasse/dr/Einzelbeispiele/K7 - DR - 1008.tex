\section{K7 - DR - 1008 2. Ableitung - OA - Thema Mathematik Schularbeiten 7. Klasse}

\begin{beispiel}[K7 - DR]{1} %PUNKTE DES BEISPIELS
			Eine Polynomfunktion 3. Grades $f$ ist durch ihren Graphen gegeben.
			
			\begin{center}
				\resizebox{0.7\linewidth}{!}{\psset{xunit=1.0cm,yunit=1.0cm,algebraic=true,dimen=middle,dotstyle=o,dotsize=4pt 0,linewidth=0.8pt,arrowsize=3pt 2,arrowinset=0.25}
\begin{pspicture*}(-2.74,-5.58)(8.76,3.58)
\multips(0,-5)(0,1.0){10}{\psline[linestyle=dashed,linecap=1,dash=1.5pt 1.5pt,linewidth=0.4pt,linecolor=lightgray]{c-c}(-2.74,0)(8.76,0)}
\multips(-2,0)(1.0,0){12}{\psline[linestyle=dashed,linecap=1,dash=1.5pt 1.5pt,linewidth=0.4pt,linecolor=lightgray]{c-c}(0,-5.58)(0,3.58)}
\psaxes[labelFontSize=\scriptstyle,xAxis=true,yAxis=true,Dx=1.,Dy=1.,ticksize=-2pt 0,subticks=2]{->}(0,0)(-2.74,-5.58)(8.76,3.58)[x,140] [y,-40]
\psplot[linewidth=1.2pt,plotpoints=200]{-2.7399999999999975}{8.760000000000009}{0.24991550191860348*x^(3.0)-1.499155019186035*x^(2.0)-0.002027953953516223*x+3.0}
\rput[tl](1.4,1.66){f}
\antwort{\psplot[linewidth=1.2pt,plotpoints=200]{-1}{5}{1.499493011511621*x-2.99831003837207}}
\end{pspicture*}}
			\end{center}
			
			Erg�nze im Diagramm einen m�glichen Graphen der 2. Ableitung im Intervall $[-1;5]$
			\end{beispiel}