\section{K7 - DR - 1011 Eigenschaften - MC - Thema Mathematik Schularbeiten 7. Klasse}

\begin{beispiel}[K7 - DR]{1} %PUNKTE DES BEISPIELS
			Beurteile folgende Schlussfolgerungen für eine reelle Funktion $f$.
			
			Kreuze die richtige Schlussfolgerung an!\leer
			
			\multiplechoice[6]{  %Anzahl der Antwortmoeglichkeiten, Standard: 5
							L1={$f$ ist im Intervall $[a;b]$ monoton steigend $\Rightarrow f$ ist in $[a;b]$ differenzierbar},   %1. Antwortmoeglichkeit 
							L2={$f'(3)=0 \Rightarrow f$ hat an der Stelle $x=3$ keine Tangente},   %2. Antwortmoeglichkeit
							L3={$f''(3)=0 \Rightarrow f$ hat an der Stelle $x=3$ sicher eine waagrechte Tangente},   %3. Antwortmoeglichkeit
							L4={$f''$ ist im Intervall $[a;b]$ positiv $\Rightarrow f$ ist im Intervall $[a;b]$ monoton fallend},   %4. Antwortmoeglichkeit
							L5={$f'(4)$ ist nicht definiert $\Rightarrow f$ ist an der Stelle $x=4$ nicht differenzierbar},	 %5. Antwortmoeglichkeit
							L6={$f$ ist im Intervall $[a;b]$ stetig $\Rightarrow f$ ist in $[a;b]$ differenzierbar},	 %6. Antwortmoeglichkeit
							L7={},	 %7. Antwortmoeglichkeit
							L8={},	 %8. Antwortmoeglichkeit
							L9={},	 %9. Antwortmoeglichkeit
							%% LOESUNG: %%
							A1=5,  % 1. Antwort
							A2=0,	 % 2. Antwort
							A3=0,  % 3. Antwort
							A4=0,  % 4. Antwort
							A5=0,  % 5. Antwort
							}
			\end{beispiel}