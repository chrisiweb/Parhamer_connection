\section{AN 3.3 - K7 - DR - 1010 Eigenschaften - LT - Thema Mathematik Schularbeiten 7. Klasse}

\begin{beispiel}[K7 - DR]{1} %PUNKTE DES BEISPIELS
			Gegen ist eine Aussage über die Polynomfunktion $f$.
			
			\lueckentext{
							text={Eine Polynomfunktion $f$ ist in einem Intervall $[a;b]$ genau dann \gap, wenn \gap gilt.}, 	%Lueckentext Luecke=\gap
							L1={streng monoton wachsend}, 		%1.Moeglichkeit links  
							L2={streng monoton fallend}, 		%2.Moeglichkeit links
							L3={stetig}, 		%3.Moeglichkeit links
							R1={$f(x)>0$ für alle $x\in[a;b]$}, 		%1.Moeglichkeit rechts 
							R2={$f'(x)>0$ für alle $x\in[a;b]$}, 		%2.Moeglichkeit rechts
							R3={$f''(x)>0$ für alle $x\in[a;b]$}, 		%3.Moeglichkeit rechts
							%% LOESUNG: %%
							A1=1,   % Antwort links
							A2=2		% Antwort rechts 
							}
			\end{beispiel}