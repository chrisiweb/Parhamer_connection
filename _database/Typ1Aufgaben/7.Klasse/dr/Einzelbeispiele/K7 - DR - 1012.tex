\section{K7 - DR - 1012 Eigenschaften - LT - Thema Mathematik Schularbeiten 7. Klasse}

\begin{beispiel}[K7 - DR]{1} %PUNKTE DES BEISPIELS
			Gegeben ist eine Aussage über reelle Funktionen.
			
			\lueckentext{
							text={Eine reelle Funktion $f$: $A\rightarrow\mathbb{R}$ heißt stetig an der Stelle $p$, wenn \gap und diese(r) \gap.}, 	%Lueckentext Luecke=\gap
							L1={der Funktionswert $f(p)$ an der Stelle $p$ existiert}, 		%1.Moeglichkeit links  
							L2={die Ableitung $f'(p)$ an der Stelle $p$ existiert}, 		%2.Moeglichkeit links
							L3={der Grenzwert von $f$ an der Stelle $p$ existiert}, 		%3.Moeglichkeit links
							R1={den Funktionswert $f(p)$ annimmt}, 		%1.Moeglichkeit rechts 
							R2={nicht den Funktionswert $f(p)$ annimmt}, 		%2.Moeglichkeit rechts
							R3={null ist}, 		%3.Moeglichkeit rechts
							%% LOESUNG: %%
							A1=3,   % Antwort links
							A2=1		% Antwort rechts 
							}
			\end{beispiel}