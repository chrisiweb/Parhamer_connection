\section{FA 1.6 - K7 - DR - 1023 - Näherungsweises Gleichungslösen - OA - Dimensionen Mathematik, Schularbeiten-Trainer 7. Klasse}

\begin{beispiel}[K7 - DR]{1} %PUNKTE DES BEISPIELS
Gegeben ist die Funktion $f$ mit $f(x)=-0,1x^4+2x^2-3x+1$.

Ermittle die Nullstellen der Funktion, runde die Ergebnisse auf drei Dezimalen.\leer

\antwort{$x_1=-5,122; x_2=0,494; x_3=1,129; x_4=3,499$}
				
				\end{beispiel}