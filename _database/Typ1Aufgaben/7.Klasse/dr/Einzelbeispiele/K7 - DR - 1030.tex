\section{K7 - DR -  - 1030 - Stetig - LT - Dimensionen Mathematik, Schularbeiten-Trainer 7. Klasse}

\begin{beispiel}[K7 - DR]{1} %PUNKTE DES BEISPIELS
Eine Funktion $f$ mit dem Definitionsbereich $D=\mathbb{R}^*=\mathbb{R}\backslash \{0\}$ ist wie folgt abschnittsweise definiert:

$f(x)=\begin{cases}-1&x\leq -1\\
g(x),&-1<x<1, x\neq 0\\
1,&x\geq 1\\
\end{cases}$

\lueckentext[-0.15]{
				text={Für \gap weist die Funktion $f$ \gap auf.}, 	%Lueckentext Luecke=\gap
				L1={$g(x)=\frac{1}{x}$}, 		%1.Moeglichkeit links  
				L2={$g(x)=\frac{1}{x^2}$}, 		%2.Moeglichkeit links
				L3={$g(x)=\frac{4}{x}$}, 		%3.Moeglichkeit links
				R1={genau die drei Unstetigkeitsstellen $x_1=-1, x_2=0$ und $x_3=1$}, 		%1.Moeglichkeit rechts 
				R2={genau die zwei Unstetigkeitsstellen $x_1=-1$ und $x_2=0$}, 		%2.Moeglichkeit rechts
				R3={keine Unstetigkeitsstelle}, 		%3.Moeglichkeit rechts
				%% LOESUNG: %%
				A1=1,   % Antwort links
				A2=3		% Antwort rechts 
				}
				
				\end{beispiel}