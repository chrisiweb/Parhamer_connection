\section{FA 1.6 - K7 - DR - 1017 - Schnittpunkte - OA - Dimensionen Mathematik, Schularbeiten-Trainer 7. Klasse}

\begin{beispiel}[K7 - DR]{1} %PUNKTE DES BEISPIELS
Gegeben ist die Funktion $f$ mit der Funktionsgleichung $f(x)=-x^3+2x^2-5x$.

Berechne die Koordinaten der Schnittpunkte zwischen dem Graphen der Funktion $f$ und der Geraden g:$y=6$ ohne Technologieeinsatz.\leer

\antwort{$P_1(1/6), P_2(3/6), P_3(-2/6)$}
				
				\end{beispiel}