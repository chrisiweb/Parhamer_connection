\section{K7 - DR - 1006 Eigenschaften von Funktionen - MC - Thema Mathematik Schularbeiten 7. Klasse}

\begin{beispiel}[K7 - DR]{1} %PUNKTE DES BEISPIELS
			Eine reelle Funktion $f$ ist durch ihren Graphen gegeben.
			
			\begin{center}
				\resizebox{0.5\linewidth}{!}{\psset{xunit=1.0cm,yunit=0.7cm,algebraic=true,dimen=middle,dotstyle=o,dotsize=4pt 0,linewidth=0.8pt,arrowsize=3pt 2,arrowinset=0.25}
\begin{pspicture*}(-3.66,-2.6020512820512876)(3.68,3.774404057480984)
\multips(0,-2)(0,1.0){7}{\psline[linestyle=dashed,linecap=1,dash=1.5pt 1.5pt,linewidth=0.4pt,linecolor=lightgray]{c-c}(-3.66,0)(3.68,0)}
\multips(-3,0)(1.0,0){8}{\psline[linestyle=dashed,linecap=1,dash=1.5pt 1.5pt,linewidth=0.4pt,linecolor=lightgray]{c-c}(0,-2.6020512820512876)(0,3.774404057480984)}
\psaxes[labelFontSize=\scriptstyle,xAxis=true,yAxis=true,Dx=1.,Dy=1.,ticksize=-2pt 0,subticks=2]{->}(0,0)(-3.66,-2.6020512820512876)(3.68,3.774404057480984)[x,140] [y,-40]
\psplot[linewidth=1.2pt,plotpoints=200]{-3}{3}{-0.2008443733879232*x^(3.0)-0.30337749355169275*x^(2.0)+1.2991556266120767*x+2.205066240327539}
\rput[tl](1.9,2.8593970132431696){f}
\end{pspicture*}}
			\end{center}
			
			Kreuze die beiden zutreffenden Aussagen an!\leer
			
			\multiplechoice[5]{  %Anzahl der Antwortmoeglichkeiten, Standard: 5
							L1={$f$ ist im Intervall $]-2;2[$ negativ gekrümmt.},   %1. Antwortmoeglichkeit 
							L2={$f$ ist für $1<x<3$ monoton steigend.},   %2. Antwortmoeglichkeit
							L3={Die Funktion hat bei $x=-2$ eine waagrechte Tangente.},   %3. Antwortmoeglichkeit
							L4={Ein globales Minimum liegt bei $x=-2$.},   %4. Antwortmoeglichkeit
							L5={Ein Wendepunkt liegt bei $x\approx -0,5$},	 %5. Antwortmoeglichkeit
							L6={},	 %6. Antwortmoeglichkeit
							L7={},	 %7. Antwortmoeglichkeit
							L8={},	 %8. Antwortmoeglichkeit
							L9={},	 %9. Antwortmoeglichkeit
							%% LOESUNG: %%
							A1=3,  % 1. Antwort
							A2=5,	 % 2. Antwort
							A3=0,  % 3. Antwort
							A4=0,  % 4. Antwort
							A5=0,  % 5. Antwort
							}
			\end{beispiel}