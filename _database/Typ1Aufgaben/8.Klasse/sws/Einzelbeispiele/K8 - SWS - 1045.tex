\section{K8 - SWS - 1045 - Stimmenanteil der Partei - OA - Dimensionen Mathematik 8 - Schularbeiten-Trainer}

\begin{beispiel}[K8 - SWS]{1}
Bei der letzten Wahl hat eine politische Partei $28\,\%$ der Stimmen erhalten. Es wird vermutet, dass sich in der Zwischenzeit der Stimmenanteil der Partei ver�ndert hat. Die entsprechende Hypothese ist bei einem Signifikanzniveau von $\alpha=0,05$ zu testen. Zur Verf�gung steht eine Zufallsstichprobe der Gr��e $n=450$. Als Teststatistik $X$ wird die Anzahl jener Personen in der Stichprobe verwendet, die sich als W�hler der betreffenden Partei deklarieren.

Ermittle die kritischen Werte f�r die Teststatistik $X$. Interpretiere das Ergebnis.

\antwort{Nullhypothese: $H_0$: $p=0,28$\\
Arbeitshypothese $H_A$: $p\neq 0,28$\\
$n=450; p=0,28, \mu=450\cdot 0,28=126$ $\sigma=\sqrt{450\cdot 0,28\cdot 0,72}=9,91$\\
$\Phi(z)=0,975$ bzw. $z=1,96$\\
$x_{1,2}=126\pm 1,96\cdot 9,91$, $x_1\approx 106,6; x_2\approx 145,4$

Befinden sich in der Stichprobe h�chstens 106 oder mindestens 146 Personen, die sich als W�hler der betreffenden Partei deklarieren, so kann die Nullhypothese verworfen werden bzw. von einer Ver�nderung des Stimmenanteils der Partei ausgegangen werden.}
\end{beispiel}