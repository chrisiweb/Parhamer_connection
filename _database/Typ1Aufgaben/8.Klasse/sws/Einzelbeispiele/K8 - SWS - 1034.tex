\section{K8 - SWS - 1034 - Begriffe des Hypothesentests - ZO - Dimensionen Mathematik 8 - Schularbeiten-Trainer}

\begin{beispiel}[K8 - SWS]{1}
Im Folgen den geht es um Begriffe, die im Rahmen des Hypothesentests eine Rolle spielen.

\zuordnen{
				R1={Signifikanz},				% Response 1
				R2={Signifikanzniveau},				% Response 2
				R3={Alternativhypothese},				% Response 3
				R4={Annahmebereich der Nullhypothese},				% Response 4
				%% Moegliche Zuordnungen: %%
				A={Arbeitshypothese}, 				%Moeglichkeit A  
				B={Nullhypothese}, 				%Moeglichkeit B  
				C={Maximal zul�ssige Irrtumswahrscheinlichkeit}, 				%Moeglichkeit C  
				D={Irrtumswahrscheinlichkeit}, 				%Moeglichkeit D  
				E={Ablehnungsbereich der Arbeitshypothese}, 				%Moeglichkeit E  
				F={Annahmebereich der Alternativhypothese}, 				%Moeglichkeit F  
				%% LOESUNG: %%
				A1={D},				% 1. richtige Zuordnung
				A2={C},				% 2. richtige Zuordnung
				A3={A},				% 3. richtige Zuordnung
				A4={E},				% 4. richtige Zuordnung
				}
\end{beispiel}