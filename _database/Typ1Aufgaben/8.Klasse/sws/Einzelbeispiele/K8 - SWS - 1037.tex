\section{K8 - SWS - 1037 - Alternativhypothese - LT - Dimensionen Mathematik 8 - Schularbeiten-Trainer}

\begin{beispiel}[K8 - SWS]{1}
Gegeben ist die Alternativhypothese $p<0,4$. Die Teststatistik $X$ wird anhand einer Zufallsstichprobe der Gr��e $n=100$ ermittelt.

\lueckentext{
				text={Wenn der Wert \gap zu einer Entscheidung zugunsten der Alternativhypothese f�hrt, dann w�rde der Wert \gap mit Sicherheit zur gleichen Entscheidung f�hren.}, 	%Lueckentext Luecke=\gap
				L1={$X=42$}, 		%1.Moeglichkeit links  
				L2={$X=35$}, 		%2.Moeglichkeit links
				L3={$X=50$}, 		%3.Moeglichkeit links
				R1={$X=33$}, 		%1.Moeglichkeit rechts 
				R2={$X=36$}, 		%2.Moeglichkeit rechts
				R3={$X=40$}, 		%3.Moeglichkeit rechts
				%% LOESUNG: %%
				A1=2,   % Antwort links
				A2=1		% Antwort rechts 
				}
\end{beispiel}