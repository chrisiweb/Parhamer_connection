\section{K8 - SWS - 1027 - Benzinverbrauch - OA - Dimensionen Mathematik 8 - Schularbeiten-Trainer}

\begin{beispiel}[K8 - SWS]{1}
Für ein Auto einer bestimmten Marke haben Tests ergeben, dass der Benzinverbrauch für Stadtfahren im Mittel 7,6\,l auf 100\,km beträgt. Als Standardabweichung wird 0,4\,l angegeben.

Berechne mithilfe zweckmäßiger Technologie die Wahrscheinlichkeit, dass der Benzinverbrauch zwischen 7,5 und 8,1 Liter auf 100\,km liegt.

\antwort{$\mu=7,6; \sigma=0,4$

Die Wahrscheinlichkeit, dass der Benzinverbrauch zwischen 7,5 und 8,1 Liter auf 100\,km liegt, beträgt 49,31\,\%.}
\end{beispiel}