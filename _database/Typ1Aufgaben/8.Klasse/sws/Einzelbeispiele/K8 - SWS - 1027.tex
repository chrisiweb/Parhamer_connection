\section{K8 - SWS - 1027 - Benzinverbrauch - OA - Dimensionen Mathematik 8 - Schularbeiten-Trainer}

\begin{beispiel}[K8 - SWS]{1}
F�r ein Auto einer bestimmten Marke haben Tests ergeben, dass der Benzinverbrauch f�r Stadtfahren im Mittel 7,6\,l auf 100\,km betr�gt. Als Standardabweichung wird 0,4\,l angegeben.

Berechne mithilfe zweckm��iger Technologie die Wahrscheinlichkeit, dass der Benzinverbrauch zwischen 7,5 und 8,1 Liter auf 100\,km liegt.

\antwort{$\mu=7,6; \sigma=0,4$

Die Wahrscheinlichkeit, dass der Benzinverbrauch zwischen 7,5 und 8,1 Liter auf 100\,km liegt, betr�gt 49,31\,\%.}
\end{beispiel}