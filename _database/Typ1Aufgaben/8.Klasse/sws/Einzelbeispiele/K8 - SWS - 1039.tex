\section{K8 - SWS - 1039 - Alternativhypothese eines einseiten Anteilstests - MC - Dimensionen Mathematik 8 - Schularbeiten-Trainer}

\begin{beispiel}[K8 - SWS]{1}
Die Alternativhypothese eines einseitigen Anteilstests lautet: $H_A\!:p>0,25$. Das Signifikanzniveau wird auf einen bestimmten Wert $\alpha$ festgelegt. Die Entscheidung erfolgt anhand einer Zufallsstichprobe bestimmter Größe. Als Teststatistik wird die Anzahl jener Personen in der Stichprobe verwendet, die das relevante Merkmal aufweisen. Diese Anzahl wird mit $x_1$ bezeichnet. Berechnungen ergeben, dass im Falle $\alpha=0,05$ und $x_1=115$ die Entscheidung zugunsten der Alternativhypothese $H_1$ ausfällt.

Kreuze die zutreffenden Aussagen an!

\multiplechoice[5]{  %Anzahl der Antwortmoeglichkeiten, Standard: 5
				L1={Im Falle von $\alpha=0,01$ und $x_1=115$ würde die Entscheidung auch zugunsten der Alternativhypothese ausfallen.},   %1. Antwortmoeglichkeit 
				L2={Es ist möglich, dass im Falle von $\alpha=0,05$ und $x_1=116$ die Entscheidung zugunsten der Nullhypothese ausfallen würde.},   %2. Antwortmoeglichkeit
				L3={Im Falle von $\alpha=0,1$ und $x_1=125$ würde die Entscheidung sicher zugunsten der Alternativhypothese ausfallen.},   %3. Antwortmoeglichkeit
				L4={Es ist möglich, dass im Falle von $\alpha=0,01$ und $x_1=116$ die Entscheidung zugunsten der Alternativhypothese ausfallen würde.},   %4. Antwortmoeglichkeit
				L5={Im Falle von $\alpha=0,01$ und $x_1=113$ würde die Entscheidung sicher zugunsten der Nullhypothese ausfallen.},	 %5. Antwortmoeglichkeit
				L6={},	 %6. Antwortmoeglichkeit
				L7={},	 %7. Antwortmoeglichkeit
				L8={},	 %8. Antwortmoeglichkeit
				L9={},	 %9. Antwortmoeglichkeit
				%% LOESUNG: %%
				A1=3,  % 1. Antwort
				A2=4,	 % 2. Antwort
				A3=0,  % 3. Antwort
				A4=0,  % 4. Antwort
				A5=0,  % 5. Antwort
				} 
\end{beispiel}