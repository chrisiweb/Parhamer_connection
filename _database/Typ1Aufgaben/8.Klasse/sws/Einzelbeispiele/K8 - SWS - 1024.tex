\section{K8 - SWS - 1024 - Berechne den Wert der Standardabweichung - OA - Dimensionen Mathematik 8 - Schularbeiten-Trainer}

\begin{beispiel}[K8 - SWS]{1}
Eine Maschine füllt kleine Flaschen ab. Sie ist auf eine Soll-Füllmenge von 150\,ml eingestellt. Infolge unvermeidlicher Ungenauigkeiten variieren die tatsächlichen Füllmengen aber. In guter Näherung können die Füllmengen normalverteilt mit den Parametern $\mu=150$\,ml und $\sigma$ angenommen werden.

Berechne, welchen Wert die Standardabweichung $\sigma$ (maximal) annehmen kann, damit mindestens 95\,\% aller Flaschen eine Füllmenge zwischen 149\,ml und 151\,ml aufweisen.

\antwort{$\Phi(z)=0,975$ bzw. $z=1,96$

$151=150+1,96\cdot\sigma \Rightarrow \sigma=0,51$

Die Standardabweichung darf maximal den Wert $\sigma=0,51$ annehmen.}
\end{beispiel}