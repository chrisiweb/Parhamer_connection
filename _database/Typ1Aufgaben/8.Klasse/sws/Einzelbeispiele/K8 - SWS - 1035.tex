\section{K8 - SWS - 1035 - Werbekampagne - ZO - Dimensionen Mathematik 8 - Schularbeiten-Trainer}

\begin{beispiel}[K8 - SWS]{1}
Von einer Werbekampagne beträgt der Anteil $p$ jener Personen, die eine bestimmte Webseite kennen, $60\,\%$. Nach der Kampagne wird die Arbeitshypothese $p>0,6$ einem Test unterworfen (Signifikanzniveau $\alpha=0,05$). Als Teststatistik wird die Anzahl $X$ jener Personen in einer Zufallsstichprobe verwendet, die angeben, die Webseite zu kennen. Im konkreten Fall nimmt diese Anzahl den Wert $X_1$ an.

Ordne den angeführten "`Wenn-Satzteilen"' jeweils jenen "`Dann-Satzteil"' zu, der aus dem betreffenden "`Wenn-Satzteil"' logisch korrekt gefolgert werden kann.

\zuordnen{
				R1={Wenn $P(X\geq X_1)=0,03$ gilt,},				% Response 1
				R2={Wenn man das Signifikanzniveau verkleinert,},				% Response 2
				R3={Wenn $P(X\leq X_1)=0,9$ gilt,},				% Response 3
				R4={Wenn sich unter Annahme der Nullhypothese für $X$ der Erwartungswert $E(X)=240$ ergibt,},				% Response 4
				%% Moegliche Zuordnungen: %%
				A={dann gilt für die Größe der Stichprobe $n<200$.}, 				%Moeglichkeit A  
				B={dann gilt für die Größe der Stichprobe $n=400$.}, 				%Moeglichkeit B  
				C={dann wird der Ablehnungsbereich für die Nullhypothese kleiner.}, 				%Moeglichkeit C  
				D={dann entscheidet man sich für die Nullhypothese.}, 				%Moeglichkeit D  
				E={dann wird der Annahmebereich für die Arbeitshypothese größer.}, 				%Moeglichkeit E  
				F={dann entscheidet man sich für die Arbeitshypothese.}, 				%Moeglichkeit F  
				%% LOESUNG: %%
				A1={F},				% 1. richtige Zuordnung
				A2={C},				% 2. richtige Zuordnung
				A3={D},				% 3. richtige Zuordnung
				A4={B},				% 4. richtige Zuordnung
				}
\end{beispiel}