\section{K8 - SWS - 1040 - Rechtsseitige Anteilstest - MC - Dimensionen Mathematik 8 - Schularbeiten-Trainer}

\begin{beispiel}[K8 - SWS]{1}
Gegeben ist ein rechtsseitiger Anteilstest. Das Signifikanzniveau wird auf den Wert $\alpha=0,05$ festgelegt. Die Entscheidung erfolgt anhand einer Zufallsstichprobe. Als Teststatistik wird die Anzahl $X$ jener Personen in der Stichprobe verwendet, die das relevante Merkmal aufweisen. Im konkreten Fall nimmt $X$ den Wert $x_1$ an.

Kreuze die zutreffenden Aussagen an.

\multiplechoice[5]{  %Anzahl der Antwortmoeglichkeiten, Standard: 5
				L1={Die Richtigkeit jener Hypothese, zu deren Gunsten die Entscheidung letztlich ausfällt, kann als bewiesen angesehen werden.},   %1. Antwortmoeglichkeit 
				L2={Der Berechnung der Wahrscheinlichkeit $P(X\geq x_1)$ liegt die Annahme zugrunde, dass die Nullhypothese zutrifft.},   %2. Antwortmoeglichkeit
				L3={Wenn $P(X\geq x_1)\leq 0,05$ gilt, dann fällt die Entscheidung zugunsten der Alternativhypothese aus.},   %3. Antwortmoeglichkeit
				L4={Wenn $x_1$ mindestens so groß ist wie jener kritische Wert $x_k$ , für den $P(X\geq x_k)=0,05$ gilt, dann entscheidet man sich für die Nullhypothese.},   %4. Antwortmoeglichkeit
				L5={Es gibt zwei kritische Werte $x_{k_1}$ und $x_{k_2}$, die man so bestimmt, dass $P(X\geq x_{k_1})=0,05$ bzw. $P(X\leq x_{k_2})=0,05$ gilt.},	 %5. Antwortmoeglichkeit
				L6={},	 %6. Antwortmoeglichkeit
				L7={},	 %7. Antwortmoeglichkeit
				L8={},	 %8. Antwortmoeglichkeit
				L9={},	 %9. Antwortmoeglichkeit
				%% LOESUNG: %%
				A1=2,  % 1. Antwort
				A2=3,	 % 2. Antwort
				A3=0,  % 3. Antwort
				A4=0,  % 4. Antwort
				A5=0,  % 5. Antwort
				}
\end{beispiel}