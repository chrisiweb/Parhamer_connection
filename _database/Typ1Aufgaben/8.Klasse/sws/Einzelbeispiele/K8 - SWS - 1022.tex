\section{K8 - SWS - 1022 - Soll-Füllmenge - OA - Dimensionen Mathematik 8 - Schularbeiten-Trainer}

\begin{beispiel}[K8 - SWS]{1}
Eine Maschine füllt kleine Flaschen ab. Sie ist auf eine Soll-Füllmenge von 150\,ml eingestellt. Infolge unvermeidlicher Ungenauigkeiten variieren die tatsächlichen Füllmengen. In guter Näherung können die Füllmengen normalverteilt mit den Parameterwerten $\mu=150$\,ml und $\sigma=1$\,ml angenommen werden.

Berechne, in welchem zur Soll-Füllmenge von 150\,ml symmetrischen Intervall $95\,\%$ aller Füllmengen liegen.

\antwort{$\Phi(z)=0,975$ bzw. $z=1,96$

$x_{1,2}=150\pm 1,96\cdot 1$

$95\,\%$ aller Füllmengen liegen im symmetrischen Intervall $[148,04; 151,96]$.}
\end{beispiel}