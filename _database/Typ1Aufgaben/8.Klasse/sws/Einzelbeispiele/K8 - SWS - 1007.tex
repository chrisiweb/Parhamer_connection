\section{K8 - SWS - 1007 - standardnormalverteilt - LT - Dimensionen Mathematik 8 - Schularbeiten-Trainer}

\begin{beispiel}[K8 - SWS]{1}
Die Variable $Z$ ist standardnormalverteilt. Die Verteilungsfunktion von $Z$ wird mit $\Phi$ bezeichnet.

\lueckentext{
				text={Der Wert \gap entspricht der Wahrscheinlichkeit \gap.}, 	%Lueckentext Luecke=\gap
				L1={$\Phi(1,5)$}, 		%1.Moeglichkeit links  
				L2={$\Phi(1,5)-\Phi(0,5)$}, 		%2.Moeglichkeit links
				L3={$\Phi(0,5)$}, 		%3.Moeglichkeit links
				R1={$P(0,5<Z<1,5)$}, 		%1.Moeglichkeit rechts 
				R2={$P(Z>1,5)$}, 		%2.Moeglichkeit rechts
				R3={$P(0\leq Z<0,5)$}, 		%3.Moeglichkeit rechts
				%% LOESUNG: %%
				A1=2,   % Antwort links
				A2=1		% Antwort rechts 
				}
\end{beispiel}