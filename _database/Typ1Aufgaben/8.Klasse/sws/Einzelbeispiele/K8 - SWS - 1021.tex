\section{K8 - SWS - 1021 - Weitsprungleistungen - OA - Dimensionen Mathematik 8 - Schularbeiten-Trainer}

\begin{beispiel}[K8 - SWS]{1}
Die Weitsprungleistungen in einer bestimmten Sch�lerpopulation sind normalverteilt mit $\mu=4,4$\,m und $\sigma=0,4$\,m.

Berechne, wie weit ein Sch�ler/eine Sch�lerin mindestens springen muss, damit er/sie zu den besten 10\,\% der Population z�hlt.

\antwort{$1-\Phi(z)=0,1 \Rightarrow \Phi(z)=0,9$ bzw. $z\approx 1,28$

$x=4,4+1,28\cdot 0,4\approx 4,91$

Ein Sch�ler/eine Sch�lerin muss mindestens 4,91\,m springen, damit er/sie zu den besten 10\,\% der Population z�hlt.}
\end{beispiel}