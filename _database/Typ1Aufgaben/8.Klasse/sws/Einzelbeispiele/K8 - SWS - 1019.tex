\section{K8 - SWS - 1019 - Weitsprungleistungen - OA - Dimensionen Mathematik 8 - Schularbeiten-Trainer}

\begin{beispiel}[K8 - SWS]{1}
Die Weitsprungleistungen in einer bestimmten Sch�lerpopulation sind normalverteilt mit $\mu=4,4$\,m und $\sigma=0,4$\,m.

Ein Sch�ler/eine Sch�lerin wird zuf�llig ausgew�hlt. Berechne die Wahrscheinlichkeit, dass seine/ihre Weite zwischen 4,2\,m und 4,8\,m liegt.

\antwort{$z_1=\frac{4,8-4,4}{0,4}=1; z_2=\frac{4,2-4,4}{0,4}=-0,5$

$P(4,2<X<4,8)=\Phi(1)-\Phi(-0,5)=0,5328$

Die Wahrscheinlichkeit, dass seine/ihre Weite zwischen 4,2\,m und 4,8\,m liegt, betr�gt 53,28\,\%.}
\end{beispiel}