\section{AG-L 3.5 - K8 - SWS - 1001 Gau�'sche Glockenkurve - ZO -  Mathematik Schularbeiten 8. Klasse}

\begin{beispiel}[K8 - SWS]{1} %PUNKTE DES BEISPIELS
Gegeben sind vier Gau�'sche Glockenkurven und sechs Wertepaare f�r den Erwartungswert $\mu$ und die Standardabweichung $\sigma$ einer normalverteilten Zufallsvariablen.

\zuordnen[0.2]{
				R1={\psset{xunit=0.5cm,yunit=4.0cm,algebraic=true,dimen=middle,dotstyle=o,dotsize=5pt 0,linewidth=0.8pt,arrowsize=3pt 2,arrowinset=0.25}
\begin{pspicture*}(-1.5,-0.13)(10.60958339757793,0.45424808802349087)
\multips(0,0)(0,0.1){5}{\psline[linestyle=dashed,linecap=1,dash=1.5pt 1.5pt,linewidth=0.2pt,linecolor=black!60]{c-c}(-0.5871040257313656,0)(10.60958339757793,0)}
\multips(0,0)(1.0,0){12}{\psline[linestyle=dashed,linecap=1,dash=1.5pt 1.5pt,linewidth=0.2pt,linecolor=black!60]{c-c}(0,-0.037995743650574676)(0,0.45424808802349087)}
\psaxes[labelFontSize=\scriptstyle,linewidth=0.2pt,xAxis=true,yAxis=true,Dx=1.,Dy=0.1,ticksize=-2pt 0,subticks=2]{->}(0,0)(-0.5871040257313656,-0.037995743650574676)(10.60958339757793,0.45424808802349087)[$x$,120][,0]
\pscustom[fillcolor=black,fillstyle=solid,opacity=0.5]{\psplot{3.}{5.}{EXP((-(x-4.0)^(2.0))/(1.0^(2.0)*2.0))/(abs(1.0)*sqrt(3.141592653589793*2.0))}\lineto(5.,0)\lineto(3.,0)\closepath}
\psplot[linewidth=1pt,plotpoints=200]{-0.5871040257313656}{10.60958339757793}{EXP((-(x-4.0)^(2.0))/(1.0^(2.0)*2.0))/(abs(1.0)*sqrt(3.141592653589793*2.0))}
\begin{scriptsize}
\rput[bl](-4.330729286349283,-0.025905544276123942){$f$}
\end{scriptsize}
\end{pspicture*}},				% Response 1
				R2={\psset{xunit=0.5cm,yunit=4.0cm,algebraic=true,dimen=middle,dotstyle=o,dotsize=5pt 0,linewidth=0.8pt,arrowsize=3pt 2,arrowinset=0.25}
\begin{pspicture*}(-1.5,-0.13)(10.60958339757793,0.45424808802349087)
\multips(0,0)(0,0.1){5}{\psline[linestyle=dashed,linecap=1,dash=1.5pt 1.5pt,linewidth=0.2pt,linecolor=black!60]{c-c}(-0.5871040257313656,0)(10.60958339757793,0)}
\multips(0,0)(1.0,0){12}{\psline[linestyle=dashed,linecap=1,dash=1.5pt 1.5pt,linewidth=0.2pt,linecolor=black!60]{c-c}(0,-0.037995743650574676)(0,0.45424808802349087)}
\psaxes[labelFontSize=\scriptstyle,linewidth=0.2pt,xAxis=true,yAxis=true,Dx=1.,Dy=0.1,ticksize=-2pt 0,subticks=2]{->}(0,0)(-0.5871040257313656,-0.037995743650574676)(10.60958339757793,0.45424808802349087)[$x$,120][,0]
\pscustom[fillcolor=black,fillstyle=solid,opacity=0.5]{\psplot{3.}{5.}{EXP((-(x-4.0)^(2.0))/(1.0^(2.0)*2.0))/(abs(1.0)*sqrt(3.141592653589793*2.0))}\lineto(5.,0)\lineto(3.,0)\closepath}
\psplot[linewidth=1pt,plotpoints=200]{-0.5}{19.707165686219295}{EXP((-(x-5.0)^(2.0))/(2.0^(2.0)*2.0))/(abs(2.0)*sqrt(3.141592653589793*2.0))}\begin{scriptsize}
\rput[bl](-4.330729286349283,-0.025905544276123942){$f$}
\end{scriptsize}
\end{pspicture*}},				% Response 2
				R3={\psset{xunit=0.5cm,yunit=4.0cm,algebraic=true,dimen=middle,dotstyle=o,dotsize=5pt 0,linewidth=0.8pt,arrowsize=3pt 2,arrowinset=0.25}
\begin{pspicture*}(-1.5,-0.13)(10.60958339757793,0.45424808802349087)
\multips(0,0)(0,0.1){5}{\psline[linestyle=dashed,linecap=1,dash=1.5pt 1.5pt,linewidth=0.2pt,linecolor=black!60]{c-c}(-0.5871040257313656,0)(10.60958339757793,0)}
\multips(0,0)(1.0,0){12}{\psline[linestyle=dashed,linecap=1,dash=1.5pt 1.5pt,linewidth=0.2pt,linecolor=black!60]{c-c}(0,-0.037995743650574676)(0,0.45424808802349087)}
\psaxes[labelFontSize=\scriptstyle,linewidth=0.2pt,xAxis=true,yAxis=true,Dx=1.,Dy=0.1,ticksize=-2pt 0,subticks=2]{->}(0,0)(-0.5871040257313656,-0.037995743650574676)(10.60958339757793,0.45424808802349087)[$x$,120][,0]
\pscustom[fillcolor=black,fillstyle=solid,opacity=0.5]{\psplot{3.}{5.}{EXP((-(x-4.0)^(2.0))/(1.0^(2.0)*2.0))/(abs(1.0)*sqrt(3.141592653589793*2.0))}\lineto(5.,0)\lineto(3.,0)\closepath}
\psplot[linewidth=1pt,plotpoints=200]{-0.5544272182775634}{13.373921893919075}{EXP((-(x-5.0)^(2.0))/(1.0^(2.0)*2.0))/(abs(1.0)*sqrt(3.141592653589793*2.0))}\begin{scriptsize}
\rput[bl](-4.330729286349283,-0.025905544276123942){$f$}
\end{scriptsize}
\end{pspicture*}},				% Response 3
				R4={\psset{xunit=0.5cm,yunit=4.0cm,algebraic=true,dimen=middle,dotstyle=o,dotsize=5pt 0,linewidth=0.8pt,arrowsize=3pt 2,arrowinset=0.25}
\begin{pspicture*}(-1.5,-0.13)(10.60958339757793,0.45424808802349087)
\multips(0,0)(0,0.1){5}{\psline[linestyle=dashed,linecap=1,dash=1.5pt 1.5pt,linewidth=0.2pt,linecolor=black!60]{c-c}(-0.5871040257313656,0)(10.60958339757793,0)}
\multips(0,0)(1.0,0){12}{\psline[linestyle=dashed,linecap=1,dash=1.5pt 1.5pt,linewidth=0.2pt,linecolor=black!60]{c-c}(0,-0.037995743650574676)(0,0.45424808802349087)}
\psaxes[labelFontSize=\scriptstyle,linewidth=0.2pt,xAxis=true,yAxis=true,Dx=1.,Dy=0.1,ticksize=-2pt 0,subticks=2]{->}(0,0)(-0.5871040257313656,-0.037995743650574676)(10.60958339757793,0.45424808802349087)[$x$,120][,0]
\pscustom[fillcolor=black,fillstyle=solid,opacity=0.5]{\psplot{3.}{5.}{EXP((-(x-4.0)^(2.0))/(1.0^(2.0)*2.0))/(abs(1.0)*sqrt(3.141592653589793*2.0))}\lineto(5.,0)\lineto(3.,0)\closepath}
\psplot[linewidth=1pt,plotpoints=200]{-0.5}{14.697764989182081}{EXP((-(x-4.0)^(2.0))/(1.5^(2.0)*2.0))/(abs(1.5)*sqrt(3.141592653589793*2.0))}\begin{scriptsize}
\rput[bl](-4.330729286349283,-0.025905544276123942){$f$}
\end{scriptsize}
\end{pspicture*}},				% Response 4
				%% Moegliche Zuordnungen: %%
				A={$\mu=5$, $\sigma=2$}, 				%Moeglichkeit A  
				B={$\mu=4$, $\sigma=1,5$}, 				%Moeglichkeit B  
				C={$\mu=4$, $\sigma=1$}, 				%Moeglichkeit C  
				D={$\mu=6$, $\sigma=2$}, 				%Moeglichkeit D  
				E={$\mu=3$, $\sigma=1$}, 				%Moeglichkeit E  
				F={$\mu=5$, $\sigma=1$}, 				%Moeglichkeit F  
				%% LOESUNG: %%
				A1={},				% 1. richtige Zuordnung
				A2={},				% 2. richtige Zuordnung
				A3={},				% 3. richtige Zuordnung
				A4={},				% 4. richtige Zuordnung
				}
				
	
				
\end{beispiel}