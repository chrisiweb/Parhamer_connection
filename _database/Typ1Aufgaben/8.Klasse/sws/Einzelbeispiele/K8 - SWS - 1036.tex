\section{K8 - SWS - 1036 - Politische Wahl - LT - Dimensionen Mathematik 8 - Schularbeiten-Trainer}

\begin{beispiel}[K8 - SWS]{1}
Bei der letzten politische Wahl betrug der Anteil der für die Partei A abgegebenen Stimmen $20\,\%$. Im Rahmen einer aktuellen Untersuchung wird von der Nullhypothese $H_0\!:p=0,2$ ausgegangen. Als Entscheidungsgrundlage dient die Anzahl $X$ jener Personen, die in einer Zufallsstichprobe der Größe $n=400$ eine Präferenz für die Partei A angeben.

\lueckentext{
				text={Im Falle einer \gap Formulierung der Alternativhypothese würde im Fall, dass der Wert $x=71$ zu einer Entscheidung für die Alternativhypothese führt, der Wert \gap mit Sicherheit zur gleichen Entscheidung führen.}, 	%Lueckentext Luecke=\gap
				L1={linksseitigen}, 		%1.Moeglichkeit links  
				L2={zweiseitigen}, 		%2.Moeglichkeit links
				L3={rechtsseitigen}, 		%3.Moeglichkeit links
				R1={$X=73$}, 		%1.Moeglichkeit rechts 
				R2={$X=91$}, 		%2.Moeglichkeit rechts
				R3={$X=81$}, 		%3.Moeglichkeit rechts
				%% LOESUNG: %%
				A1=2,   % Antwort links
				A2=2		% Antwort rechts 
				}
\end{beispiel}