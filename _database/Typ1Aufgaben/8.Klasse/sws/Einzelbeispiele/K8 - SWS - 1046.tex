\section{K8 - SWS - 1046 - Sportsch�tze - OA - Dimensionen Mathematik 8 - Schularbeiten-Trainer}

\begin{beispiel}[K8 - SWS]{1}
Ein Sportsch�tze hat eine Trefferquote von 92\,\%. Durch ein Spezialtraining auf einem Trainingslager soll die Trefferquote erh�ht werden. Am Ende des Trainingslagers wird die Effizienz des Training einem Signifikanztest unterzogen (Signifikanz $\alpha=0,05$). Dazu wird die Anzahl $X$ der Treffer in einer Serie von 200 Sch�ssen ermittelt.

Ermittle den kritischen Wert f�r die Teststatistik $X$. Interpretiere das Ergebnis.

\antwort{Nullhypothese $H_0$: $p=0,92$\\
Arbeitshypothese $H_A$: $p>0,92$\\
$n=200; p=0,92; \mu=200\cdot 0,92=184; \sigma=\sqrt{200\cdot 0,92\cdot 0,08}\approx 3,84$\\
$\Phi(z)=0,95$ bzw. $z=1,645$\\
$x=184+1,645\cdot 3,84\approx 190,3$

Betr�gt in einer Serie von 200 Sch�ssen die Anzahl der Treffen mindestens 191, kann von einer Erh�hung der Trefferquote ausgegangen bzw. die Nullhypothese verworfen werden.}
\end{beispiel}