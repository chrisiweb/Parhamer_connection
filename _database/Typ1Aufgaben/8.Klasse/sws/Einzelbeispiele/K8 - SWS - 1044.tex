\section{K8 - SWS - 1044 - Vorgeschriebene Geschwindigkeitsbeschränkung - OA - Dimensionen Mathematik 8 - Schularbeiten-Trainer}

\begin{beispiel}[K8 - SWS]{1}
Erfahrungsgemäß halten sich 30\,\% der Lenker/-innen bei einem bestimmten Streckenabschnitt nicht an die vorgeschriebene Geschwindigkeitsbeschränkung. Durch Anbringung zusätzlicher Hinweisschilder soll der Anteil der "`Raser"' verringert werden. Vier Wochen nach Anbringung der Schilder wird die Arbeitshypothese $H_A$: $p<0,3$ einem Signifikanztest unterzogen $(\alpha=0,05)$. Dazu wird in einer Zufallsstichprobe der Größe $n=300$ die Anzahl $X$ der zu schnell fahrenden Fahrzeuge ermittelt.

Ermittle die Signifikanz des Wertes $X=82$ und interpretiere das Ergebnis.

\antwort{Nullhypothese $H_0$: $p=0,3$\\
Arbeitshypothese $H_A$: $p<0,3$\\
$n=300; p=0,3; \mu=300\cdot 0,3=90; \sigma=\sqrt{300\cdot 0,3\cdot 0,7}\approx 7,937$\\
$z=\frac{82-90}{7,937}\approx -1$ $\Phi(-1)=0,157$

Die Signifikanz des Wertes $X=82$ beträgt ca. 15,7\,\%.

Dieser Wert liegt über dem vereinbarten Signifikanzniveau (maximal zulässige Irrtumswahrscheinlichkeit). Die Nullhypothese kann also nicht verworfen werden.}
\end{beispiel}