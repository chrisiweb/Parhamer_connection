\section{K8 - SWS - 1020 - Weg zur Arbeit - OA - Dimensionen Mathematik 8 - Schularbeiten-Trainer}

\begin{beispiel}[K8 - SWS]{1}
Die Zeit, die eine bestimmte Person für den Weg zur Arbeit benötigt, kann als normalverteilt mit den Parameterwerten $\mu=26$ Minuten und $\sigma=3$ Minuten angesehen werden.

Berechne, wie viele Minuten vor Arbeitsbeginn die Person (mindestens) von zuhause aus starten müsste, damit sie mit einer Wahrscheinlichkeit von höchstens 0,01 zu spät am Arbeitsplatz erscheint.

\antwort{$1-\Phi(z)=0,01 \Rightarrow \Phi(z)=0,99$ bzw. $z=2,33$

$z=\frac{x-\mu}{\sigma}$ bzw. $x=\mu+z\cdot\sigma$

$x=26+2,33\cdot 3\approx 33$

Startet die Person ca. 33 Minuten vor Arbeitsbeginn von zuhause, so erscheint sie mit einer Wahrscheinlichkeit von höchstens 0,01 zu spät am Arbeitsplatz.}
\end{beispiel}