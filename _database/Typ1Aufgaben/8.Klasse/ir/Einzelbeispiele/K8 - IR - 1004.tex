\section{K8 - IR - 1004 - Integral und Stammfunktion - OA - Dimensionen Mathematik 8 - Schularbeiten-Trainer}

\begin{beispiel}[K8 - IR]{1}
Der Hauptsatz der Differential- und Integralrechnung macht eine Aussage über die Integral- und Stammfunktion.

Erkläre den Begriff Integral- und Stammfunktion und beschreibe ihren Zusammenhang.

\antwort{Unter einer Stammfunktion $F$ einer reellen Funktion $f$ versteht man eine differenzierbare Funktion, deren Ableitungsfunktion $F'$ mit $f$ übereinstimmt, d.h. $F'(x)=f(x)$.

Eine Integralfunktion $A$ ist eine Funktion, die den "`orientierten"' Flächeninhalt zwischen einer Funktion $f$ und der ersten Achse von einer gegebenen Stelle $a$ bis zur Stelle $x$ angibt, d.h. $A(x)=\displaystyle\int^x_a{f(t)}\,\text{d}t$.

Stammfunktion $F$ und Integralfunktion $A$ unterscheiden sich nur durch eine Konstante. $A(x)=F(x)+c$.}
\end{beispiel}