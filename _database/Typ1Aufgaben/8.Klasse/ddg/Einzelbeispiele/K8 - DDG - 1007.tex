\section{K8 - DDG - 1007 - Zehnkampf - LT - Dimensionen Mathematik 8 - Schularbeiten-Trainer}

\begin{beispiel}[K8 - DDG]{1}
Gemäß der Einschätzung von Experten kann ein Leichtathlet aufgrund seines Talents und seiner körperlichen Voraussetzung im Zehnkampf zur absoluten Weltspitze vorstoßen und bis zu 8\,600 Punkte in einem Wettkampf erzielen. Die Funktion $P$ beschreibt die aktuelle Bestleistung des Sportlers in Abhängigkeit von der Zeit $t$. Die momentane Bestleistung (Zeitpunkt $t=0$) beträgt $P(0)=7000$ Punkte. Erfahrungsgemäß ist eine weitere Leistungssteigerung umso schwieriger und nur in immer kleineren Ausmaßen möglich, je näher sich die Leistung des Athleten der maximalen Leistungsgrenze nähert.

\lueckentext{
				text={Die jeweils noch realisierbare weitere Leistungssteigerung kann modellhaft als direkt proportional zu der durch den Term \gap beschriebenen "`Verbesserungskapazität"' $K(t)$ angesehen werden, sodass die Funktion $P$ eine Lösung der Differentialgleichung \gap darstellt.}, 	%Lueckentext Luecke=\gap
				L1={$K(t)=P(t)-7000$}, 		%1.Moeglichkeit links  
				L2={$K(t)=8600-7000=1600$}, 		%2.Moeglichkeit links
				L3={$K(t)=8600-P(t)$}, 		%3.Moeglichkeit links
				R1={$P'(t)=c\cdot K(t)$}, 		%1.Moeglichkeit rechts 
				R2={$P(t)=c\cdot K(t)$}, 		%2.Moeglichkeit rechts
				R3={$P'(t)=c\cdot(8600-P(t))$}, 		%3.Moeglichkeit rechts
				%% LOESUNG: %%
				A1=3,   % Antwort links
				A2=3		% Antwort rechts 
				}
\end{beispiel}