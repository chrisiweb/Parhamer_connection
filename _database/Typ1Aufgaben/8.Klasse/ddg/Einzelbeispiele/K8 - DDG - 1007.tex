\section{K8 - DDG - 1007 - Zehnkampf - LT - Dimensionen Mathematik 8 - Schularbeiten-Trainer}

\begin{beispiel}[K8 - DDG]{1}
Gem�� der Einsch�tzung von Experten kann ein Leichtathlet aufgrund seines Talents und seiner k�rperlichen Voraussetzung im Zehnkampf zur absoluten Weltspitze vorsto�en und bis zu 8\,600 Punkte in einem Wettkampf erzielen. Die Funktion $P$ beschreibt die aktuelle Bestleistung des Sportlers in Abh�ngigkeit von der Zeit $t$. Die momentane Bestleistung (Zeitpunkt $t=0$) betr�gt $P(0)=7000$ Punkte. Erfahrungsgem�� ist eine weitere Leistungssteigerung umso schwieriger und nur in immer kleineren Ausma�en m�glich, je n�her sich die Leistung des Athleten der maximalen Leistungsgrenze n�hert.

\lueckentext{
				text={Die jeweils noch realisierbare weitere Leistungssteigerung kann modellhaft als direkt proportional zu der durch den Term \gap beschriebenen "`Verbesserungskapazit�t"' $K(t)$ angesehen werden, sodass die Funktion $P$ eine L�sung der Differentialgleichung \gap darstellt.}, 	%Lueckentext Luecke=\gap
				L1={$K(t)=P(t)-7000$}, 		%1.Moeglichkeit links  
				L2={$K(t)=8600-7000=1600$}, 		%2.Moeglichkeit links
				L3={$K(t)=8600-P(t)$}, 		%3.Moeglichkeit links
				R1={$P'(t)=c\cdot K(t)$}, 		%1.Moeglichkeit rechts 
				R2={$P(t)=c\cdot K(t)$}, 		%2.Moeglichkeit rechts
				R3={$P'(t)=c\cdot(8600-P(t))$}, 		%3.Moeglichkeit rechts
				%% LOESUNG: %%
				A1=3,   % Antwort links
				A2=3		% Antwort rechts 
				}
\end{beispiel}