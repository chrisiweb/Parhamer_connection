\section{K8 - DDG - 1015 - Radioaktive Substanz - OA - Dimensionen Mathematik 8 - Schularbeiten-Trainer}

\begin{beispiel}[K8 - DDG]{1}
Die Funktion $n$ beschreibt die von einer radioaktiven Substanz vorhandenen Menge (in mg) in Abh�ngigkeit von der Zeit $t$ (in Stunden). Zum Zeitpunkt $t=0$ sind von der Substanz 30\,mg vorhanden. Die momentane Zerfallsrate der Substanz zu einem Zeitpunkt betr�gt 3,05\,\% der zu diesem Zeitpunkt vorhandenen Menge.

Bestimme eine Gleichung der Funktion $n$.

\antwort{$n'(t)=-0,0305\cdot n(t)$\\
$\frac{n'(t)}{n(t)}=-0,0305$\\
$\ln(n(t))=-0,0305\cdot t+c$\\
$n(t)=e^{-0,0305t+c}$\\
$n(t)=e^{-0,0305\cdot t}\cdot e^c$\\
Wegen $n(0)=30 \Rightarrow e^c=30$

$n(t)=30\cdot e^{-0,0305\cdot t}$ bzw. $n(t)=30\cdot 0,9695^t$ bzw. $n(t)\approx 30\cdot 0,97^t$}
\end{beispiel}