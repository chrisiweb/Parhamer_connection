\section{K8 - DDG - 1005 - Momentane �nderungsrate - OA - Dimensionen Mathematik 8 - Schularbeiten-Trainer}

\begin{beispiel}[K8 - DDG]{1}
Die momentane �nderungsrate einer Funktion $f$ an einer Stelle $x$ $(x\in\mathbb{R})$ ist direkt proportional zur Summe von $x$ und dem Wert $f(x)$. An der Stelle 2 nimmt die momentane �nderungsrate den Wert 0,5 an und der Funktionswert betr�gt 4.

Beschreibe den Sachverhalt mithilfe einer Differentialgleichung.

\antwort{$f'(x)=c\cdot[x+f(x)]$

$f'(2)=0,5$, $f(2)=4 \Rightarrow 0,5=c\cdot(2+4) \Rightarrow c=\frac{1}{12}$

Differentialgleichung $f'(x)=\frac{1}{12}\cdot[x+f(x)]$}
\end{beispiel}