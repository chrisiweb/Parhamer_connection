\section{K8 - DDG - 1011 - Momentane Änderungsrate - OA - Dimensionen Mathematik 8 - Schularbeiten-Trainer}

\begin{beispiel}[K8 - DDG]{1}
Wie aus der nachfolgenden Tabelle hervorgeht, besteht zwischen der momentanen Änderungsrate einer Funktion $f$ (mit $f(x)=y$) an einer Stelle $x$ einerseits und dem jeweiligen Argument $x$ und/oder dem zugehörigen Funktionswert $y$ andererseits ein bestimmter Zusammenhang.

\begin{tabular}{|l|c|c|c|c|}\hline
\cellcolor[gray]{0.9}$x$&$0,125$&$0,5$&$2$&$8$\\ \hline
\cellcolor[gray]{0.9}$y$&$0,5$&$1$&$2$&$4$\\ \hline
\cellcolor[gray]{0.9}$y'$&$2$&$1$&$0,5$&$0,25$\\ \hline
\end{tabular}

Gib eine Differentialgleichung an, die den gemäß der Tabelle vorliegenden Zusammenhang angemessen beschreibt, und bestimme eine Gleichung der Funktion $f$.

\antwort{Das Produkt aus momentaner Änderungsrate und Funktionswert beträgt 1.\\
Differentialgleichung: $f'(x)\cdot f(x)=1$ bzw. $f'(x)=\frac{1}{f(x)}$\\
Eine Vervierfachung des $x$-Wertes hat eine Verdoppelung des $y$-Wertes zu Folge.

$f(x)=a\cdot \sqrt{x}$\\
$f(0,125)=a\cdot\sqrt{0,125}=0,5$\\
$a^2\cdot 0,125=0,25$\\
$a^2; a=\pm\sqrt{2} \Rightarrow f(x)=\sqrt{2}\cdot\sqrt{x}=\sqrt{2x}=(2x)^{\frac{1}{2}}$\\
(Da die angeführten $y$-Werte größer als 0 sind, kann $a=\sqrt{2}$ angenommen werden.)

$f'(x)=\frac{1}{2}\cdot(2x)^{-\frac{1}{2}}\cdot 2=\frac{1}{\sqrt{2x}}$}
\end{beispiel}