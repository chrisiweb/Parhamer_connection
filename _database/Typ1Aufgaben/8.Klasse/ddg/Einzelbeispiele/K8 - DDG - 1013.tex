\section{K8 - DDG - 1013 - Größe einer Population - OA - Dimensionen Mathematik 8 - Schularbeiten-Trainer}

\begin{beispiel}[K8 - DDG]{1}
Die Funktion $N$ beschreibt die Größe einer "`Population"' in Abhängigkeit von der Zeit $t$. Die momentane Wachstumsrate der Population ist zum Anfangsbestand $N(0)=500$ direkt proportional. Der Proportionalitätsfaktor nimmt den Wert 0,04 an.

Bestimme eine Gleichung der Funktion $N$.

\antwort{$N'(t)=0,04\cdot 500=20$\\
$N(t)=20\cdot t+c$\\
$N(0)=500=c$

$N(t)=20\cdot t+500$}
\end{beispiel}