\section{K8 - DDG - 1014 - Größe einer Population - OA - Dimensionen Mathematik 8 - Schularbeiten-Trainer}

\begin{beispiel}[K8 - DDG]{1}
Die Funktion $M$ beschreibt die Größe einer "`Population"' in Abhängigkeit von der Zeit $t$. Die momentane Wachstumsrate der Population ist zum Zeitpunkt $t$ ist zur Größe der Population zu diesem Zeitpunkt direkt proportional. Der Proportionalitätsfaktor nimmt den Wert $\ln(0,98)$ an.

Bestimme eine Gleichung der Funktion $M$.

\antwort{$M'(t)=\ln(0,98)\cdot M(t)$\\
$\frac{M'(t)}{M(t)}=\ln(0,98)$\\
$\ln(M(t))=\ln(0,98)\cdot t+c$\\
$M(t)=e^{\ln(0,98)\cdot t+c}$\\
$M(t)=0,98^t\cdot e^c$\\
Wegen $M(0)=M_0=0,98^0\cdot e^c \Rightarrow M_0=e^c$

$M(t)=M_0\cdot 0,98^t$}
\end{beispiel}