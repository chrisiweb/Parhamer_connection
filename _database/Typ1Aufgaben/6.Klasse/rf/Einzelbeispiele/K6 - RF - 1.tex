\section{K6 - RF - 1 Eigenschaften einer Logarithmusfunktion - LT}

\begin{beispiel}[K6 - RF]{1} %PUNKTE DES BEISPIELS
				Gegeben ist eine Logarithmusfunktion $f:\mathbb{R}^+\mapsto\mathbb{R}$ mit $f(x)=c\cdot^a\log(x)$ mit ($c\in\mathbb{R}, a\in\mathbb{R}^+, a\neq 1$).

				\lueckentext{
								text={Für alle $a\in\mathbb{R}^+\backslash\{1\}$ gilt: Für \gap gilt: \gap.}, 	%Lueckentext Luecke=\gap
								L1={$a>1$}, 		%1.Moeglichkeit links  
								L2={$a>0$}, 		%2.Moeglichkeit links
								L3={$a=1$}, 		%3.Moeglichkeit links
								R1={$f(x)<0$ für $0<x<1$}, 		%1.Moeglichkeit rechts 
								R2={$f(x)<0$ für $x>1$}, 		%2.Moeglichkeit rechts
								R3={$f(x)=0$ für $x=0$}, 		%3.Moeglichkeit rechts
								%% LOESUNG: %%
								A1=1,   % Antwort links
								A2=1		% Antwort rechts 
								}
\end{beispiel}