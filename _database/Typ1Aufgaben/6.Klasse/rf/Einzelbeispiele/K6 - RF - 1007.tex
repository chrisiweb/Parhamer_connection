\section{FA 1.6 - K6 - RF - 1007 Quadratische Funktion - LT - Thema Mathematik Schularbeiten 6. Klasse}

\begin{beispiel}[K6 - RF]{1} %PUNKTE DES BEISPIELS
			Gegeben ist die Funktion $f$ mit $f(x)=x²-4x+4$.
			
			\lueckentext{
							text={Der Graph von $f$ besitzt an der Stelle \gap einen Schnittpunkt mit \gap.}, 	%Lueckentext Luecke=\gap
							L1={$x=-2$}, 		%1.Moeglichkeit links  
							L2={$x=4$}, 		%2.Moeglichkeit links
							L3={$x=-4$}, 		%3.Moeglichkeit links
							R1={der x-Achse}, 		%1.Moeglichkeit rechts 
							R2={der y-Achse}, 		%2.Moeglichkeit rechts
							R3={dem 1. Median}, 		%3.Moeglichkeit rechts
							%% LOESUNG: %%
							A1=2,   % Antwort links
							A2=3		% Antwort rechts 
							}
							\end{beispiel}