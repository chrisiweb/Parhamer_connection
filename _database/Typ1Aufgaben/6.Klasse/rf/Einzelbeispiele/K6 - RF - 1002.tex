\section{K6 - RF - 1002 Frau Huber - MC - Thema Mathematik Schularbeiten 6. Klasse}

\begin{beispiel}[K6 - RF]{1} %PUNKTE DES BEISPIELS
			Frau Huber f�hrt t�glich dieselbe Strecke zur Arbeit.
			
			Das Diagramm zeigt, wie die Fahrzeit $t$ von der mittleren Geschwindigkeit $v$ abh�ngt:
			
			\begin{center}
				\resizebox{0.7\linewidth}{!}{\psset{xunit=1.0cm,yunit=1.0cm,algebraic=true,dimen=middle,dotstyle=o,dotsize=4pt 0,linewidth=1.6pt,arrowsize=3pt 2,arrowinset=0.25}
\begin{pspicture*}(-0.84,-0.64)(14.88,8.3)
\psaxes[labelFontSize=\scriptstyle,xAxis=true,yAxis=true,labels=none,Dx=1.,Dy=1.,ticks=none]{->}(0,0)(0.,0.)(14.88,8.3)
\psplot[linewidth=1.2pt,plotpoints=200]{0.083993}{14.8800}{5.0/x}
\rput[tl](5.82,-0.16){mittlere Geschwindigkeit $v$}
\rput[tl](-0.64,5.02){\rotatebox{90}{\text{Fahrtzeit} $t$}}
\end{pspicture*}}
			\end{center}
			
			Kreuze die zutreffende Aussage an!
			
			\multiplechoice[6]{  %Anzahl der Antwortmoeglichkeiten, Standard: 5
							L1={Frau Huber f�hrt immer gleich schnell.},   %1. Antwortmoeglichkeit 
							L2={Frau Huber ben�tigt immer die gleiche Fahrzeit.},   %2. Antwortmoeglichkeit
							L3={Eine l�ngere Fahrzeit ben�tigt eine h�here Geschwindigkeit.},   %3. Antwortmoeglichkeit
							L4={Die Fahrzeit ist direkt proportional zur mittleren Geschwindigkeit.},   %4. Antwortmoeglichkeit
							L5={Eine Verdopplung der mittleren Geschwindigkeit bewirkt eine Verdopplung der Fahrzeit.},	 %5. Antwortmoeglichkeit
							L6={Die mittlere Geschwindigkeit ist indirekt proportional zur Fahrzeit.},	 %6. Antwortmoeglichkeit
							L7={},	 %7. Antwortmoeglichkeit
							L8={},	 %8. Antwortmoeglichkeit
							L9={},	 %9. Antwortmoeglichkeit
							%% LOESUNG: %%
							A1=6,  % 1. Antwort
							A2=0,	 % 2. Antwort
							A3=0,  % 3. Antwort
							A4=0,  % 4. Antwort
							A5=0,  % 5. Antwort
							}
							\end{beispiel}