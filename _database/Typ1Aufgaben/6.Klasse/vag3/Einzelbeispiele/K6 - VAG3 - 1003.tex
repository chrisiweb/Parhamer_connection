\section{K6 - VAG3 - 1003 - Parallele Ebenen - OA - Dimensionen 6 Schulbuch}

\begin{beispiel}[K6 - VAG3]{1}
Gegeben sind die Ebenen $\epsilon_1$ und $\epsilon_2$ mit $\epsilon_1:X=\Vek{1
}{0}{0}+r_1\Vek{3}{-2}{1}+s_1\Vek{4}{8}{1}$ und $\epsilon_2:X=\Vek{2}{1}{3}+r_
2\Vek{6}{y}{2}+s_2\Vek{x}{4}{\frac{1}{2}}$.

Welchen Wert m�ssen $y$ und $x$ in $\epsilon_2$ haben, damit $\epsilon_1$ und $\epsilon_2$ parallel sind? Begr�nde.\leer

\antwort{$x=2$ da der zweite Richtungsvektor von $\epsilon_2$ die H�lfte des zweiten Richtungsvektors von $\epsilon_1$ sein soll.

$y=-4$ da der erste Richtungsvektor von $\epsilon_2$ das Doppelte des ersten Richtungsvektors von $\epsilon_1$ sein soll.}
\end{beispiel}