\section{AG 1.2 - K6 - PWLU - 1001 Logarithmen - ZO - Thema Mathematik Schularbeiten 6. Klasse}

\begin{beispiel}[K6 - PWLU ]{1} %PUNKTE DES BEISPIELS
			Gegeben sei eine Aussage �ber Logarithmen.
			
			\lueckentext{
							text={Die Hochzahl $x$, mit der man $a$ potenzieren muss, um $b$ zu erhalten hei�t \gap von $b$ zur Basis $a$ wird mit \gap bezeichnet.}, 	%Lueckentext Luecke=\gap
							L1={Potenz}, 		%1.Moeglichkeit links  
							L2={Logarithmus}, 		%2.Moeglichkeit links
							L3={Basis}, 		%3.Moeglichkeit links
							R1={$^a\log b$}, 		%1.Moeglichkeit rechts 
							R2={$^x\log b$}, 		%2.Moeglichkeit rechts
							R3={$^b\log a$}, 		%3.Moeglichkeit rechts
							%% LOESUNG: %%
							A1=2,   % Antwort links
							A2=1		% Antwort rechts 
							}
			\end{beispiel}