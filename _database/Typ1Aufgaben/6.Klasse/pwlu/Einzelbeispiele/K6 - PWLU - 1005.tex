\section{K6 - PWLU - 1005 Logarithmus - ZO - Thema Mathematik Schularbeiten 6. Klasse}

\begin{beispiel}[K6 - PWLU]{1} %PUNKTE DES BEISPIELS
			Ordne jedem Logarithmus den entsprechenden Wert (aus A bis F) zu!
			
			\zuordnen{
							R1={$\log_2\sqrt[3]{2}$},				% Response 1
							R2={$\log_{10} 100\,000$},				% Response 2
							R3={$\log_2\frac{1}{64}$},				% Response 3
							R4={$\log_{10}\sqrt[3]{0,1}$},				% Response 4
							%% Moegliche Zuordnungen: %%
							A={$-\frac{1}{3}$}, 				%Moeglichkeit A  
							B={$-5$}, 				%Moeglichkeit B  
							C={$\frac{1}{3}$}, 				%Moeglichkeit C  
							D={$6$}, 				%Moeglichkeit D  
							E={$-6$}, 				%Moeglichkeit E  
							F={$5$}, 				%Moeglichkeit F  
							%% LOESUNG: %%
							A1={C},				% 1. richtige Zuordnung
							A2={F},				% 2. richtige Zuordnung
							A3={E},				% 3. richtige Zuordnung
							A4={A},				% 4. richtige Zuordnung
							}
							\end{beispiel}