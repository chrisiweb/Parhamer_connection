\section{AG 2.4 - K6 - PWLU - 1004 L�sungsintervalle - ZO - Thema Mathematik Schularbeiten 6. Klasse}

\begin{beispiel}[K6 - PWLU]{1} %PUNKTE DES BEISPIELS
			Ordne jeder Ungleichung die entsprechende L�sungsmenge (aus A bis F) zu!\leer
			
			\zuordnen{
							R1={$-2x>5$},				% Response 1
							R2={$-2\geq -\frac{x}{5}$},				% Response 2
							R3={$\frac{2x}{3}<6$},				% Response 3
							R4={$-6\leq 3x$},				% Response 4
							%% Moegliche Zuordnungen: %%
							A={$L=\left\{x\in\mathbb{R}\,|\,x<9\right\}$}, 				%Moeglichkeit A  
							B={$L=(-\infty;-2,5)$}, 				%Moeglichkeit B  
							C={$L=\left\{x\in\mathbb{R}\,|\,x\geq -2\right\}$}, 				%Moeglichkeit C  
							D={$L=[10;\infty)$}, 				%Moeglichkeit D  
							E={$L=(-\infty;-2,5]$}, 				%Moeglichkeit E  
							F={$L=\left\{x\in\mathbb{R}\,|\,x\leq -2\right\}$}, 				%Moeglichkeit F  
							%% LOESUNG: %%
							A1={B},				% 1. richtige Zuordnung
							A2={D},				% 2. richtige Zuordnung
							A3={A},				% 3. richtige Zuordnung
							A4={C},				% 4. richtige Zuordnung
							}
			\end{beispiel}