\section{AG 2.1 - K6 - PWLU - 1 Potenzrechenregel - LT - MK}

\begin{beispiel}[K6 - PWLU]{1} %PUNKTE DES BEISPIELS
				\lueckentext{
								text={Für alle $a,b\in\mathbb{R}_0^+$. $b\neq 0$ und alle $m,n\in\mathbb{N}^*$ gilt: \gap=\gap}, 	%Lueckentext Luecke=\gap
								L1={$\sqrt[m]{\sqrt[n]{a}}$}, 		%1.Moeglichkeit links  
								L2={$\left(\sqrt[n]{a}\right)^n$}, 		%2.Moeglichkeit links
								L3={$\sqrt[n]{a\cdot b}$}, 		%3.Moeglichkeit links
								R1={$\frac{\sqrt[n]{a}}{\sqrt[n]{b}}$}, 		%1.Moeglichkeit rechts 
								R2={$\sqrt[m\cdot n]{a}$}, 		%2.Moeglichkeit rechts
								R3={$a^n$}, 		%3.Moeglichkeit rechts
								%% LOESUNG: %%
								A1=1,   % Antwort links
								A2=2		% Antwort rechts 
								}
\end{beispiel}