\section{K4 - RE.rqw - 1 - Rechenregel Wurzel - MC - MatKon}

\begin{langesbeispiel} \item[4] %PUNKTE DES BEISPIELS
Kreuze die beiden zutreffenden Aussagen an!\leer
				
				\multiplechoice[5]{  %Anzahl der Antwortmoeglichkeiten, Standard: 5
								L1={$\sqrt{8^2+13^2}=\sqrt{233}$},   %1. Antwortmoeglichkeit 
								L2={$\sqrt{10^2-5^2}\neq\sqrt{10^2}-\sqrt{5^2}$},   %2. Antwortmoeglichkeit
								L3={$\sqrt[3]{8}+\sqrt[3]{27}=\sqrt[3]{8+27}$},   %3. Antwortmoeglichkeit
								L4={$\sqrt{x^2-y^2}=x-y$},   %4. Antwortmoeglichkeit
								L5={$(\sqrt{12-18})^2=12+18$},	 %5. Antwortmoeglichkeit
								L6={},	 %6. Antwortmoeglichkeit
								L7={},	 %7. Antwortmoeglichkeit
								L8={},	 %8. Antwortmoeglichkeit
								L9={},	 %9. Antwortmoeglichkeit
								%% LOESUNG: %%
								A1=1,  % 1. Antwort
								A2=2,	 % 2. Antwort
								A3=0,  % 3. Antwort
								A4=0,  % 4. Antwort
								A5=0,  % 5. Antwort
								}
\end{langesbeispiel}