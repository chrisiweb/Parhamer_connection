\section{05 - MAT - WS 2.2, WS 3.1, WS 3.3 - Mathematikschularbeiten - BIFIE Aufgabensammlung}

\begin{langesbeispiel} \item[0] %PUNKTE DES BEISPIELS
				Wenn in der Oberstufe in einem Semester h�chstens zwei Mathematikschularbeiten vorgesehen sind, muss jede vers�umte Schularbeit nachgeholt werden.
				Ein Mathematiklehrer hat auf Basis seiner langj�hrigen Erfahrung die untenstehende Tabelle erstellt. Dabei beschreibt $h(n)$ die relative H�ufigkeit, dass bei einer Schularbeit insgesamt $n$ Sch�ler/innen fehlen. 
				\leer
				
\begin{center}
				\begin{tabular}{|c|c|c|c|c|c|c|c|c|c|} \hline
				$n$&0&1&2&3&4&5&6&7&>7\\ \hline
				$h(n)$&0,15&0,15&0,2&0,3&0,1&0,05&0,03&0,02&0\\ \hline
				\end{tabular}	
\end{center}
			

\subsection{Aufgabenstellung:}
\begin{enumerate}
	\item Gib an, mit wie vielen Fehlenden der Mathematiklehrer im Durchschnitt bei jeder Schularbeit rechnen muss!
	
	L�sst sich aus dem errechneten Durchschnittswert mit Sicherheit behaupten, dass bei jeder Mathematikschularbeit mindestens eine Sch�lerin/ein Sch�ler fehlt? Begr�nde deine Antwort!
	\item Kreuze die beiden zutreffenden Aussagen an!
	
	\multiplechoice[5]{  %Anzahl der Antwortmoeglichkeiten, Standard: 5
					L1={Es kann nie passieren, dass acht Sch�ler/innen bei einer Schularbeit fehlen.},   %1. Antwortmoeglichkeit 
					L2={Die Wahrscheinlichkeit, dass bei einer Mathematikschularbeit niemand fehlt, ist gleich gro� wie die Wahrscheinlichkeit, dass eine Sch�lerin/ein Sch�ler fehlt.},   %2. Antwortmoeglichkeit
					L3={Die Wahrscheinlichkeit, dass drei Sch�ler/innen bei einer Mathematikschularbeit fehlen, ist gr��er als die Wahrscheinlichkeit, dass h�chstens zwei oder mindestens vier Sch�ler/innen fehlen.},   %3. Antwortmoeglichkeit
					L4={Die Wahrscheinlichkeit, dass eine Schularbeit nachgeholt werden muss, weil mindestens eine Sch�lerin/ein Sch�ler fehlt, betr�gt $85\,\%$.},   %4. Antwortmoeglichkeit
					L5={Im Durchschnitt muss eine von vier Schularbeiten pro Jahr nicht nachgeholt werden.},	 %5. Antwortmoeglichkeit
					L6={},	 %6. Antwortmoeglichkeit
					L7={},	 %7. Antwortmoeglichkeit
					L8={},	 %8. Antwortmoeglichkeit
					L9={},	 %9. Antwortmoeglichkeit
					%% LOESUNG: %%
					A1=2,  % 1. Antwort
					A2=4,	 % 2. Antwort
					A3=0,  % 3. Antwort
					A4=0,  % 4. Antwort
					A5=0,  % 5. Antwort
					}
					
					In einer bestimmten Klasse werden im kommenden Schuljahr vier Schularbeiten (zwei pro Semester) geschrieben.
					
					Gib einen Term an, mit dem die Wahrscheinlichkeitsverteilung f�r die Anzahl der Mathematikschularbeiten dieser Klasse, die aufgrund fehlender Sch�lerInnen nachgeholt werden m�ssen, berechnet werden kann!
					
					$P(X=k)=$ \antwort[\rule{3cm}{0.3pt}]{$\Vek{4}{k}{}\cdot 0,85^k\cdot 0,15^{4-k}$} mit $k=$ \antwort[\rule{3cm}{0.3pt}]{0,1,2,...,4}
\end{enumerate}

\antwort{\subsection{L�sungserwartung:}
\begin{enumerate}
	\item Im Durchschnitt muss der Mathematiklehrer mit 2,42 Fehlenden rechnen.
	
	Eine auf ganze Zahlen gerundete Antwort ist nicht korrekt, da der Erwartungswert nur statistische Aussagekraft hat und somit die Rundung die Aussage ver�ndert.
	
	Daraus l�sst sich aber nicht mit Sicherheit behaupten, dass bei jeder Mathematikschularbeit jemand fehlt, da es sich dabei um eine statistische Kenngr��e handelt, die keine konkrete Aussage �ber die einzelne Schularbeit erlaubt.
	
	Eine Sch�lerantwort, die darauf abzielt, dass es entsprechend der empirischen H�ufigkeitsverteilung mit $15\,\%$iger H�ufigkeit zu keinem Fehlen kommt, ist als nicht korrekt zu bewerten, da in der Aufgabenstellung verlangt wird, den Erwartungswert zu interpretieren.
	\item siehe oben.
\end{enumerate}}
\end{langesbeispiel}