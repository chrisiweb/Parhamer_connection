\section{05 - MAT - WS 2.2, WS 3.1, WS 3.3 - Mathematikschularbeiten - BIFIE Aufgabensammlung}

\begin{langesbeispiel} \item[0] %PUNKTE DES BEISPIELS
				Wenn in der Oberstufe in einem Semester höchstens zwei Mathematikschularbeiten vorgesehen sind, muss jede versäumte Schularbeit nachgeholt werden.
				Ein Mathematiklehrer hat auf Basis seiner langjährigen Erfahrung die untenstehende Tabelle erstellt. Dabei beschreibt $h(n)$ die relative Häufigkeit, dass bei einer Schularbeit insgesamt $n$ Schüler/innen fehlen. 
				\leer
				
\begin{center}
				\begin{tabular}{|c|c|c|c|c|c|c|c|c|c|} \hline
				$n$&0&1&2&3&4&5&6&7&>7\\ \hline
				$h(n)$&0,15&0,15&0,2&0,3&0,1&0,05&0,03&0,02&0\\ \hline
				\end{tabular}	
\end{center}
			

\subsection{Aufgabenstellung:}
\begin{enumerate}
	\item Gib an, mit wie vielen Fehlenden der Mathematiklehrer im Durchschnitt bei jeder Schularbeit rechnen muss!
	
	Lässt sich aus dem errechneten Durchschnittswert mit Sicherheit behaupten, dass bei jeder Mathematikschularbeit mindestens eine Schülerin/ein Schüler fehlt? Begründe deine Antwort!
	\item Kreuze die beiden zutreffenden Aussagen an!
	
	\multiplechoice[5]{  %Anzahl der Antwortmoeglichkeiten, Standard: 5
					L1={Es kann nie passieren, dass acht Schüler/innen bei einer Schularbeit fehlen.},   %1. Antwortmoeglichkeit 
					L2={Die Wahrscheinlichkeit, dass bei einer Mathematikschularbeit niemand fehlt, ist gleich groß wie die Wahrscheinlichkeit, dass eine Schülerin/ein Schüler fehlt.},   %2. Antwortmoeglichkeit
					L3={Die Wahrscheinlichkeit, dass drei Schüler/innen bei einer Mathematikschularbeit fehlen, ist größer als die Wahrscheinlichkeit, dass höchstens zwei oder mindestens vier Schüler/innen fehlen.},   %3. Antwortmoeglichkeit
					L4={Die Wahrscheinlichkeit, dass eine Schularbeit nachgeholt werden muss, weil mindestens eine Schülerin/ein Schüler fehlt, beträgt $85\,\%$.},   %4. Antwortmoeglichkeit
					L5={Im Durchschnitt muss eine von vier Schularbeiten pro Jahr nicht nachgeholt werden.},	 %5. Antwortmoeglichkeit
					L6={},	 %6. Antwortmoeglichkeit
					L7={},	 %7. Antwortmoeglichkeit
					L8={},	 %8. Antwortmoeglichkeit
					L9={},	 %9. Antwortmoeglichkeit
					%% LOESUNG: %%
					A1=2,  % 1. Antwort
					A2=4,	 % 2. Antwort
					A3=0,  % 3. Antwort
					A4=0,  % 4. Antwort
					A5=0,  % 5. Antwort
					}
					
					In einer bestimmten Klasse werden im kommenden Schuljahr vier Schularbeiten (zwei pro Semester) geschrieben.
					
					Gib einen Term an, mit dem die Wahrscheinlichkeitsverteilung für die Anzahl der Mathematikschularbeiten dieser Klasse, die aufgrund fehlender SchülerInnen nachgeholt werden müssen, berechnet werden kann!
					
					$P(X=k)=$ \antwort[\rule{3cm}{0.3pt}]{$\Vek{4}{k}{}\cdot 0,85^k\cdot 0,15^{4-k}$} mit $k=$ \antwort[\rule{3cm}{0.3pt}]{0,1,2,...,4}
\end{enumerate}

\antwort{\subsection{Lösungserwartung:}
\begin{enumerate}
	\item Im Durchschnitt muss der Mathematiklehrer mit 2,42 Fehlenden rechnen.
	
	Eine auf ganze Zahlen gerundete Antwort ist nicht korrekt, da der Erwartungswert nur statistische Aussagekraft hat und somit die Rundung die Aussage verändert.
	
	Daraus lässt sich aber nicht mit Sicherheit behaupten, dass bei jeder Mathematikschularbeit jemand fehlt, da es sich dabei um eine statistische Kenngröße handelt, die keine konkrete Aussage über die einzelne Schularbeit erlaubt.
	
	Eine Schülerantwort, die darauf abzielt, dass es entsprechend der empirischen Häufigkeitsverteilung mit $15\,\%$iger Häufigkeit zu keinem Fehlen kommt, ist als nicht korrekt zu bewerten, da in der Aufgabenstellung verlangt wird, den Erwartungswert zu interpretieren.
	\item siehe oben.
\end{enumerate}}
\end{langesbeispiel}