\section{62 - MAT - AN 1,3, FA 4.3, AN 2.1, FA 1.5 - Schilauf-Trainingsstrecke - Matura 2015/16 1. Nebentermin}

\begin{langesbeispiel} \item[0] %PUNKTE DES BEISPIELS
	
Schirennläuferinnen absolvieren Trainingsfahrten auf einer eigens dafür präparierten Strecke. Die Trainerin legt den Schwerpunkt ihrer Analyse auf einen 240\,m langen Streckenabschnitt vom Starthaus $A$ bis zu einem Geländepunkt $B$. Mithilfe von Videoanalysen wird die von den Rennläuferinnen zurückgelegte Weglänge in Abhängigkeit von der Zeit ermittelt.

Für eine bestimmte Trainingsfahrt einer Läuferin kann die Abhängigkeit des zurückgelegten  Weges von der Zeit während der Fahrt von $A$ nach $B$ modellhaft durch die Funktion $s$ mit $s(t)=-\frac{1}{144}\cdot t^4+\frac{8}{3}\cdot t^2$ beschrieben werden. Die Läuferin verlässt zum Zeitpunkt $t=0$ das Starthaus. Die Zeit $t$ wird in Sekunden gemessen. $s(t)$ gibt die bis zum Zeitpunkt $t$ zurückgelegte Weglänge in Metern an.

Die folgenden Fragestellungen beziehen sich auf die gegebene Zeit-Weg-Funktion $s$.


\subsection{Aufgabenstellung:}
\begin{enumerate}
	\item Um die Effektivität des Starts zu überprüfen, wird die mittlere Geschwindigkeit $\bar{v}$ der Läuferin im Zeitintervall $[0\,\text{s};3\,\text{s}]$ ermittelt.
	
	Berechne die mittlere Geschwindigkeit $\bar{v}$ der Läuferin in m/s!
	
	Berechne die für die Fahrt von $A$ nach $B$ benötigte Zeit!
	
	\item \fbox{A} Berechne denjenigen Zeitpunkt $t_1$, für den $s''(t_1)=0$ gilt!
			
			Interpretiere $t_1$ im Hinblick auf die Fahrt der Rennläuferin von $A$ nach $B$!
			
	\item Berechne die Momentangeschwindigkeit der Läuferin zum Zeitpunkt $t_2=6$!
	
	Angenommen, die Geschwindigkeit der Rennläuferin bleibe ab dem Zeitpunkt $t_2$ unverändert. Gib an, nach wie vielen Sekunden ab dem Zeitpunkt $t_2$ die Läuferin den Geländepunkt $B$ erreichen würde!
	
	\item Bei einem mathematischen Modell für die zeitliche Abhängigkeit des von der Läuferin zurückgelegten Weges sollen folgende Sachverhalte gelten:
	
	\begin{enumerate}
		\item Zum Zeitpunkt $t=0$ beträgt die momentane Geschwindigkeit der Läuferin 0 m/s.
		\item Während der Fahrt von $A$ nach $B$ nimmt die zurückgelegte Weglänge streng monoton zu.
	\end{enumerate}
	
	Gib an, welche mathematischen Eigenschaften einer differenzierbaren Zeit-Weg-Funktion $s_1$ diese Sachverhalte garantieren.
						\end{enumerate}\leer
				
\antwort{
\begin{enumerate}
	\item \subsection{Lösungserwartung:} 
	
$\bar{v}=\frac{s(3)-s(0)}{3-0}=\frac{23,4375-0}{3}\approx 7,8$

Die mittlere Geschwindigkeit innerhalb der ersten drei Fahrsekunden beträgt ca. 7,8\,m/s\leer

Mögliche Berechnung:

$-\frac{1}{144}\cdot t^4+\frac{8}{3}\cdot t^2=240 \Rightarrow t_{1,2}=\pm\,12; t_{3,4}\approx\pm\,15,5$

also: $t=12$\,s

Die Fahrt von $A$ nach $B$ dauert 12 Sekunden.

	\subsection{Lösungsschlüssel:}
	\begin{itemize}
		\item  Ein Punkt für die richtige Lösung, wobei die Einheit "`m/s"' nicht angegeben sein muss. 
		
		Toleranzintervall: $[7,5\,\text{m/s}; 8\,\text{m/s}]$ 
		\item   Ein Punkt für die richtige Lösung, wobei die Einheit "`s"' nicht angegeben sein muss.  
		
		Die Aufgabe ist auch dann als richtig gelöst zu werten, wenn bei korrektem Ansatz das Ergebnis aufgrund eines Rechenfehlers nicht richtig ist.
	\end{itemize}
	
	\item \subsection{Lösungserwartung:}
			
	Mögliche Berechnung:
	
	$s'(t)=-\frac{1}{36}\cdot t^3+\frac{16}{3}\cdot t$
	
	$s''(t)=-\frac{1}{12}\cdot t^2+\frac{16}{3}$
	
	$s''(t)=0 \Leftrightarrow -\frac{1}{12}\cdot t^2+\frac{16}{3}=0 \Rightarrow t_{1,2}=\pm\,8$
	
	also: $t_1=8$\,s\leer
	
	Mögliche Interpretation:
	
	Die Läuferin erreicht nach 8 Sekunden die maximale Geschwindigkeit.

	\subsection{Lösungsschlüssel:}
	
\begin{itemize}
	\item Ein Ausgleichspunkt für die richtige Lösung, wobei die Einheit "`s"' nicht angegeben sein muss.  
	
	Die Aufgabe ist auch dann als richtig gelöst zu werten, wenn bei korrektem Ansatz das Ergebnis aufgrund eines Rechenfehlers nicht richtig ist. 
	\item Ein Punkt für eine (sinngemäß) korrekte Interpretation.
\end{itemize}

\item \subsection{Lösungserwartung:}
			
$s'(t)=-\frac{1}{36}\cdot t^3+\frac{16}{3}\cdot t \Rightarrow s'(6)=26$

Zum Zeitpuinkt $t_2=6$ hat die Läuferin eine Geschwindigkeit von 26\,m/s.\leer

Mögliche Berechnung:

$s(6)=87 \Rightarrow \frac{240-87}{26}\approx 5,9$

Die Läuferin würde den Geländepunkt $B$ ca. 5,9\,s nach dem Zeitpunkt $t_2$ erreichen.

	\subsection{Lösungsschlüssel:}
	
\begin{itemize}
	\item Ein Punkt für die richtige Lösung, wobei die Einheit "`m/s"' nicht angegeben sein muss. 
	
	Die Aufgabe ist auch dann als richtig gelöst zu werten, wenn bei korrektem Ansatz das Ergebnis aufgrund eines Rechenfehlers nicht richtig ist.
	\item Ein Punkt für die richtige Lösung, wobei die Einheit "`s"' nicht angegeben sein muss.  
	
	Toleranzintervall: $[5,8\,\text{s}; 5,9\,\text{s}]$
\end{itemize}

\item \subsection{Lösungserwartung:}
			
Die Funktion $s_1$ muss die Eigenschaften
\begin{itemize}
	\item $s_1'(0)=0$ und
	\item $s_1'(t)>0$
\end{itemize}
während der Fahrt von $A$ nach $B$ erfüllen.

	\subsection{Lösungsschlüssel:}
	
\begin{itemize}
	\item Ein Punkt für die Nennung der Eigenschaft $s_1'(0)=0$.
	\item Ein Punkt für die Nennung der Eigenschaft $s_1'(t)>0$.
\end{itemize}
\end{enumerate}}
		\end{langesbeispiel}