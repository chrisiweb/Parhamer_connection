\section{56 - MAT - WS 1.3, WS 1.4, WS 1.1, WS 3.3, WS 3.2 - Reaktionstest - Matura 2014/15 2. Nebentermin}

\begin{langesbeispiel} \item[0] %PUNKTE DES BEISPIELS
				
	Bei einem Reaktionstest am Computer werden der getesteten Person am Bildschirm nacheinander 20 Muster gezeigt, die klassifiziert werden müssen. Protokolliert werden die für die 20 Reaktionen insgesamt benötigte Reaktionszeit $t$ sowie die Anzahl $f$ der dabei auftretenden fehlerhaften Klassifikationen. 
	
	In der nachstehenden Tabelle sind die Ergebnisse eines Reaktionstests am Computer einer Testperson in einer Serie von zehn Testdurchgängen angegeben.
	
	\begin{center}
		\begin{tabular}{|c|c|c|}\hline
		Nummer der Testdurchführung&$t$ (in Sekunden)&$f$\\ \hline
		$1^*$&$t_1=22,3$&$f_1=3$\\ \hline
		$2$&$t_2=24,6$&$f_2=2$\\ \hline
		$3$&$t_3=21,8$&$f_3=3$\\ \hline
		$4$&$t_4=23,5$&$f_4=1$\\ \hline
		$5$&$t_5=32,8$&$f_5=5$\\ \hline
		$6$&$t_6=21,7$&$f_6=4$\\ \hline
		$7$&$t_7=22,6$&$f_7=3$\\ \hline
		$8$&$t_8=22,8$&$f_8=2$\\ \hline
		$9$&$t_9=35,4$&$f_9=3$\\ \hline
		$10$&$t_{10}=22,5$&$f_10=1$\\ \hline
		\end{tabular}
		
		$^*$ Erläuterung: Die Person benötigt bei der ersten Testdurchführung 22,3 Sekunden, drei ihrer Klassifikationen waren falsch.
	\end{center}



\subsection{Aufgabenstellung:}
\begin{enumerate}
	\item \fbox{A} Berec hne das arithmetische Mittel $\bar{t}$ der zehn Reaktionszeiten $t_1, t_2, ..., t_{10}$ sowie die Standardabweichung $s_t$ dieser zehn Werte!\leer
	
	Die getestete Person absolviert zwei weitere Testdurchgänge und erreicht dabei die Zeiten $t_{11}$ und $t_{12}$. Das arithmetische Mittel der neuen Datenreihe $t_1, t_2, ..., t_{10}, t_{11}, t_{12}$ wird mit $\bar{t}_\text{neu}$ bezeichnet, die entsprechende Standardabweichung mit $s_\text{neu}$. Gib Werte für $t_{11}$ und $t_{12}$ so an, dass $t_{11}\neq t_{12}, \bar{t}_\text{neu}=\bar{t}$ und $s_\text{neu}<s_t$ gilt!\leer
	
\item Im Laufe einer Diskussion vertritt eine Person die Meinung, dass das arithmetische Mittel der 10 Reaktionszeiten die gegebene Datenliste nicht optimal beschreibt. Gib ein mögliches Argument an, das diese Meinung stützt, und nenne ein alternatives statistisches Zentralmaß!\leer

Die Datenreihe der 500 Reaktionszeiten von insgesamt 50 Testpersonen wird durch das nachstehende Kastenschaubild dargestellt.

\begin{center}
	\resizebox{1\linewidth}{!}{\psset{xunit=1.0cm,yunit=1.0cm,algebraic=true,dimen=middle,dotstyle=o,dotsize=5pt 0,linewidth=0.8pt,arrowsize=3pt 2,arrowinset=0.25}
\begin{pspicture*}(19.42,-1.52)(37.08,3.32)
\psaxes[labelFontSize=\scriptstyle,xAxis=true,yAxis=true,Dx=1.,Dy=1.,ticksize=-2pt 0,subticks=2]{->}(0,0)(19.42,-1.52)(37.08,3.32)
\psframe[linewidth=0.8pt,fillcolor=black,fillstyle=solid,opacity=0.1](22.4,0.5)(27.9,1.5)
\psline[linewidth=0.8pt,fillcolor=black,fillstyle=solid,opacity=0.1](21.4,0.5)(21.4,1.5)
\psline[linewidth=0.8pt,fillcolor=black,fillstyle=solid,opacity=0.1](35.8,0.5)(35.8,1.5)
\psline[linewidth=0.8pt,fillcolor=black,fillstyle=solid,opacity=0.1](23.,0.5)(23.,1.5)
\psline[linewidth=0.8pt,fillcolor=black,fillstyle=solid,opacity=0.1](21.4,1.)(22.4,1.)
\psline[linewidth=0.8pt,fillcolor=black,fillstyle=solid,opacity=0.12](27.9,1.)(35.8,1.)
\begin{scriptsize}
\rput[tl](21.15,1.8){21,4}
\rput[tl](22.10,1.8){22,4}
\rput[tl](22.8,1.8){23,0}
\rput[tl](27.65,1.8){27,9}
\rput[tl](35.55,1.8){35,8}
\end{scriptsize}
\end{pspicture*}}
\end{center}

Entscheide, ob die folgende Aussage jedenfalls korrekt ist: "`Höchstens 125 der 500 Reaktionszeiten betragen höchstens 22,4\,s."' Begründe deine Entscheidung!\leer

\item Die Zufallsvariable $H$ ordnet jedem Testdurchgang, bei dem einer bestimmten Person 20 Bilder vorgelegt werden, die Anzahl der dabei auftretenden fehlerhaften Reaktionen zu.\leer
 
Nenne unter Bezugnahme auf den dargelegten Sachverhalt die Voraussetzungen, die für den Reaktionstest als erfüllt angesehen werden müssen, damit die Zufallsvariable $H$ durch eine Binomialverteilung beschrieben werden kann!\leer

Berechne $P(H>2)$, wenn die getestete Person mit einer Wahrscheinlichkeit von $p=0,15$ fehlerhaft reagiert!
						\end{enumerate}\leer
				
\antwort{
\begin{enumerate}
	\item \subsection{Lösungserwartung:} 
	
	$\bar{t}=25$\,s
	
	$s_1\approx 4,9$\,2\leer
	
	Mögliche Angaben der Werte für $t_{11}$ und $t_{12}$:\leer
	
	$t_{11}=23$\,s, $t_{12}=27$\,s\leer
	
	oder:\leer
	
	$t_{11}=24$\,s, $t_{12}=26$\,s
		
	\subsection{Lösungsschlüssel:}
	\begin{itemize}
		\item Ein Ausgleichspunkt für die korrekten Werte von $\bar{t}$ und $s_t$, wobei die Einheit nicht angeführt sein muss.
		
		Toleranzintervall für $s_t:[4,6\,s; 5,0\,s]$
		\item  Ein Punkt für eine geeignete Angabe von je einem Wert für $t_{11}$ und $t_{12}$, wobei die Eineit "`s"' nicht angheführt sein muss und wobei allgemein gilt: Eine der beiden Zeiten muss den Wert $25+x$, die andere den Wert $25-x$ annehmen, wobei $x\in[0;s_t)$. Der Punkt ist auch dann zu geben, wenn die Werte für $t_{11}$ und $t_{12}$ aus der Basis falsch berechneter Werte von $\bar{t}$ bzw. $s_t$ ermittelt werden, der Lösungsweg aber prinzipiell korrekt ist.
	\end{itemize}
	
	\item \subsection{Lösungserwartung:}
			
		Mögliches Argument:
		
		Durch die beiden "`Ausreißer"' 32,8 und 35,4 wird das arithmetische Mittel stark nach oben verzerrt. Das alternative Zentralmaß \textit{Median} ist gegenüber Ausreißern robust.\leer
		
		Die Aussage ist so nicht korrekt, da aus dem Kastenschaubild nicht hervorgeht, ob es mehrere Reaktionszeiten gibt, die genau $22,4$\,s betragen und die im zweiten Viertel der geordneten Datenreihe liege.

	\subsection{Lösungsschlüssel:}
	
\begin{itemize}
	\item   Ein Punkt für ein richtiges Argument und die Angabe des Medians als alternatives Zentralmaß.
	\item  Ein Punkt für eine korrekte Entscheidung und eine (sinngemäß) korrekte Begründung.
\end{itemize}

	\item \subsection{Lösungserwartung:}
			
		Damit die Zufallsvariable $H$ als binomialverteilt angesehen werden kann, müssen folgende Bedingungen erfüllt sein:
		
		
		\begin{enumerate}
			\item  Die Wahrscheinlichkeit für eine fehlerhafte Reaktion muss für alle 20 Reaktionen gleich hoch sein (darf also bei bestimmten Bildern nicht höher oder niedriger sein als bei anderen Bildern).
			\item  Die Reaktionen müssen voneinander unabhängig sein. (Ob eine vorangegangene Reaktion richtig oder falsch war, darf keinen Einfluss auf die Richtigkeit einer nachfolgenden Reaktion haben.)
			\item  Die Reaktionen können nur entweder fehlerhaft oder korrekt sein.
		\end{enumerate}
		
		oder:\leer
		
		Jeder einzelne Versuch wird unter denselben Bedingungen durchgeführt.\leer
		
		$P(H>2)=1-[P(H=0)+P(H=1)+P(H=2)]\approx 0,595$

	\subsection{Lösungsschlüssel:}
	
\begin{itemize}
	\item    Ein Punkt für die (sinngemäß) richtige Angabe erforderlicher kontextbezogener Voraussetzungen für die Verwendung der Binomialverteilung, wobei der 3. Punkt nicht angeführt sein muss.
	\item   Ein Punkt für die richtige Lösung. Andere Schreibweisen des Ergebnisses (z.B. in Prozent) sind ebenfalls als richtig zu werten. 
	
	Toleranzintervall: $[0,59; 0,61]$ 
	
	Die Aufgabe ist auch dann als richtig gelöst zu werten, wenn bei korrektem Ansatz das Ergebnis aufgrund eines Rechenfehlers nicht richtig ist.
\end{itemize}

\end{enumerate}}
		\end{langesbeispiel}