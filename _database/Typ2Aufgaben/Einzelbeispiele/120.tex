\section{120 - K6 - FA 1.5, FA 4.3, FA 4.4 - Vulkaninseln - MatKon}

\begin{langesbeispiel} \item[6] %PUNKTE DES BEISPIELS
Durch vulkanische Aktivitäten haben sich im Laufe der Jahrtausende in der Karibik zwei neue Inseln gebildet. Das Relief ("`Höhenprofil"') der Inseln wird unter und über dem Meeresspiegel annähernd beschrieben durch folgende Funktion:
						
						$$f(x)=-\frac{2}{55}x^4+\frac{25}{88}x^2-\frac{11}{160}$$
						
						Die Funktion bzw. Koordinatenachsen wurden bereits vorteilhaft gewählt, sodass sich der Koordinatenursprung in der Mitte der beiden Inseln befindet (siehe Skizze).
						
						\begin{center}
						\newrgbcolor{srsrsr}{0.12941176470588237 0.12941176470588237 0.12941176470588237}
\newrgbcolor{sasasa}{0.16470588235294117 0.16470588235294117 0.16470588235294117}
\newrgbcolor{sasasa}{0.16470588235294117 0.16470588235294117 0.16470588235294117}
\newrgbcolor{sasasa}{0.16470588235294117 0.16470588235294117 0.16470588235294117}
\newrgbcolor{sasasa}{0.16470588235294117 0.16470588235294117 0.16470588235294117}
\newrgbcolor{vvvvvv}{0.3333333333333333 0.3333333333333333 0.3333333333333333}
\newrgbcolor{vvvvvv}{0.3333333333333333 0.3333333333333333 0.3333333333333333}
\newrgbcolor{vvvvvv}{0.3333333333333333 0.3333333333333333 0.3333333333333333}
\newrgbcolor{vvvvvv}{0.3333333333333333 0.3333333333333333 0.3333333333333333}
\psset{xunit=1.0cm,yunit=4.0cm,algebraic=true,dimen=middle,dotstyle=o,dotsize=5pt 0,linewidth=1.6pt,arrowsize=3pt 2,arrowinset=0.25}
\begin{pspicture*}(-3.56,-0.32711538461538664)(4.06,0.7567006487488384)
\psaxes[labelFontSize=\scriptstyle,xAxis=true,yAxis=true,labels=none,Dx=1.,Dy=0.2,ticks=none]{->}(0,0)(-3.56,-0.32711538461538664)(4.06,0.7567006487488384)
\pscustom[linewidth=0.8pt,linecolor=sasasa,fillcolor=sasasa,fillstyle=solid,opacity=0.1]{\psplot{-2.7}{-0.5}{-2.0/55.0*x^(4.0)+25.0/88.0*x^(2.0)-11.0/160.0}\lineto(-0.5,0)\lineto(-2.7,0)\closepath}
\pscustom[linewidth=0.8pt,linecolor=sasasa,fillcolor=sasasa,fillstyle=solid,opacity=0.1]{\psplot{0.5}{2.75}{-2.0/55.0*x^(4.0)+25.0/88.0*x^(2.0)-11.0/160.0}\lineto(2.75,0)\lineto(0.5,0)\closepath}
\psplot[linewidth=2.pt,linecolor=srsrsr,plotpoints=200]{-3.560000000000002}{4.0600000000000005}{-2.0/55.0*x^(4.0)+25.0/88.0*x^(2.0)-11.0/160.0}
\pspolygon[linewidth=2.pt,linecolor=vvvvvv,fillcolor=vvvvvv,fillstyle=solid,opacity=0.25](-3.76,0.)(-3.76,-0.3665268767377222)(4.26,-0.3665268767377222)(4.26,0.)
\rput[tl](0.74,-0.15){Wasser}
\end{pspicture*}
\end{center}

So bedeutet zum Beispiel $f(1)=0,1789$, dass die Höhe der Insel bei einem Abstand von 100\,m zur Mitte bereits 17,89\,m über dem Meeresspiegel ist.
						
						Ein Überlebender einer Flugzeugkatastrophe ist auf einer der beiden Inseln gestrandet, auf der es kein Süßwasser gibt.
						
						Beurteile die Überlebenschance des Gestrandeten für den Fall, dass es auf der anderen Insel Süßwasser gibt. Beantworte dafür folgende Fragestellungen:%Aufgabentext

\begin{aufgabenstellung}
\item %Aufgabentext

\Subitem{Wie weit sind die beiden Inseln voneinander entfernt? (Tipp: Nullstellen)} %Unterpunkt1
\Subitem{Wie lange würde man brauchen, um von einer zu anderen Insel zu schwimmen? Gib dafür eine geeignete durchschnittliche Schwimmgeschwindigkeit an.} %Unterpunkt2

\item %Aufgabentext

\ASubitem{Könnte ein Nichtschwimmer "`zu Fuß"' die Entfernung zwischen beiden Inseln überwinden? Argumentiere mit Hilfe der Funktion.} %Unterpunkt1
\Subitem{Wie man der Funktion entnehmen kann, sind die beiden Inseln gleich hoch. Gib an, wie hoch die Inseln über dem Meeresspiegel liegen.} %Unterpunkt2

\item %Aufgabentext

\Subitem{Für welche $x$-Werte hat die Funktion den $y$-Wert 0,2.} %Unterpunkt1
\Subitem{Könnte man das gegebene Höhenprofil auch mit einem Polynomfunktion dritten Grades beschrieben? Begründe deine Meinung.} %Unterpunkt2

\end{aufgabenstellung}

\begin{loesung}
\item \subsection{Lösungserwartung:} 

\Subitem{Nullstellen: $x_1=-\frac{11}{4}, x_2=-\frac{1}{2}, x_3=\frac{1}{2}, x_4=\frac{11}{4}$
	
	Abstand zwischen den beiden Inseln: $(0,5-(-0,5))\cdot 100=100$\,m} %Lösung von Unterpunkt1
\Subitem{Annahme einer Schwimmgeschwindigkeit eines normalen Menschen (z.B. 2\,km/h) $\Rightarrow$ $100$\,m in 3 min.} %%Lösung von Unterpunkt2

\setcounter{subitemcounter}{0}
\subsection{Lösungsschlüssel:}
 
\Subitem{Ein Punkt für den richtigen Inselabstand.} %Lösungschlüssel von Unterpunkt1
\Subitem{Ein Punkt für eine (halbwegs) vernünftige Schwimmgeschwindigkeit und die daraus richtig berechnete Zeitspanne.} %Lösungschlüssel von Unterpunkt2

\item \subsection{Lösungserwartung:} 

\Subitem{$f(0)=-0,07 \Rightarrow$ tiefster Punkt zwischen den beiden Inseln liegt bei $0,07\cdot 100=7$\,m. Ein Nichtschwimmer kann also NICHT auf die andere Insel gelangen.} %Lösung von Unterpunkt1
\Subitem{Da der Graph der Funktion symmetrisch zur y-Achse ist, sind beide Inseln gleich hoch. Der höchste Punkt liegt jeweils bei 49\,m.} %%Lösung von Unterpunkt2

\setcounter{subitemcounter}{0}
\subsection{Lösungsschlüssel:}
 
\Subitem{Ein Punk für eine richtige Begründung.} %Lösungschlüssel von Unterpunkt1
\Subitem{Ein Punkt für die richtige Höhe.} %Lösungschlüssel von Unterpunkt2

\item \subsection{Lösungserwartung:} 

\Subitem{$x$-Werte: $-2,59; -1,05; 1,05; 2,59$} %Lösung von Unterpunkt1
\Subitem{Nein, da die Funktion drei Extremstellen hat. Eine Funktion dritten Grades hat jedoch höchsten zwei Nullstellen.} %%Lösung von Unterpunkt2

\setcounter{subitemcounter}{0}
\subsection{Lösungsschlüssel:}
 
\Subitem{Ein Punkt für die korrekten $x$-Werte.} %Lösungschlüssel von Unterpunkt1
\Subitem{Ein Punkt für eine richtige Begründung.} %Lösungschlüssel von Unterpunkt2

\end{loesung}

\end{langesbeispiel}