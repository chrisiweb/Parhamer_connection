\section{90 - MAT - AN 1.1, AN 3.3, AN 3.2, FA 2.1, FA 2.2, FA 5.3 - Hopfen - Matura 2017/18}

\begin{langesbeispiel} \item[8] %PUNKTE DES BEISPIELS
			Hopfen ist eine schnell wachsende Kletterpflanze. Die Modellfunktion $h:\mathbb{R}^+_0\rightarrow\mathbb{R}^+$ mit $h(t)=\dfrac{a}{1+b\cdot e^{k\cdot t}}$ mit $a,b\in\mathbb{R}^+, k\in\mathbb{R}^-$ gibt näherungsweise die Pflanzenhöhe einer bestimmten Hopfensorte zum Zeitpunkt $t$ an, wobei $h(t)$ in Metern und $t$ in Wochen angegeben wird.
			
			In der nachstehenden Tabelle sind die gemessenen Höhen einer Hopfenpflanze ab Anfang April $(t=0)$ zusammengefasst.
			
			\begin{center}
				\begin{tabular}{|l|c|c|c|c|c|c|c|}\hline
				\cellcolor[gray]{0.9}{Zeit (in Wochen)}&0&2&4&6&8&10&12\\ \hline
				\cellcolor[gray]{0.9}{Höhe (in m)}&0,6&1,2&2,3&4,2&5,9&7,0&7,6\\ \hline
				\end{tabular}
			\end{center}
			
Anhand dieser Messwerte wurden für die Modellfunktion $h$ die Parameterwerte $a=8, b=15$ und $k=-0,46$ ermittelt.

\subsection{Aufgabenstellung:}
\begin{enumerate}
	\item \fbox{A} Gib unter Verwendung der Modellfunktion $h$ einen Ausdruck an, mit dem berechnet werden kann, um wie viele Meter die Hopfenpflanze im Zeitintervall $[0;t_1]$ gewachsen ist!
	
	Berechne unter Verwendung der Modellfunktion $h$ mithilfe deines Ausdrucks, wie viele Meter die Pflanze in den ersten 10 Wochen gewachsen ist und gib die prozentuelle Abweichung vom tatsächlichen gemessenen Wert an!
	
	\item Wird das Wachstum der Pflanze mithilfe der Funktion $h$ modelliert, gibt es einen Zeitpunkt $t_2$, zu dem sie am schnellsten wächst. Gib eine Gleichung an, mit der dieser Zeitpunkt berechnet werden kann, und ermittle diesen Zeitpunkt!
	
	Berechne die zugehörige maximale Wachstumsgeschwindigkeit und skizziere im nachstehenden Koordinatensystem unter Berücksichtigung des von dir ermittelten Maximums den Verlauf des Graphen derjenigen Funktion $g$, die basierend auf der Modellfunktion $h$ die Wachstumsgeschwindigkeit der Hopfenplfanze in Abhängigkeit von $t$ beschreibt!	
	
		\begin{center}
		\resizebox{0.8\linewidth}{!}{\psset{xunit=1.0cm,yunit=1.8cm,algebraic=true,dimen=middle,dotstyle=o,dotsize=5pt 0,linewidth=1.6pt,arrowsize=3pt 2,arrowinset=0.25}
\begin{pspicture*}(-0.6,-0.5215426113581649)(12.94,3.544361011883063)
\multips(0,0)(0,1.0){5}{\psline[linestyle=dashed,linecap=1,dash=1.5pt 1.5pt,linewidth=0.4pt,linecolor=darkgray]{c-c}(0,0)(12.94,0)}
\multips(0,0)(1.0,0){14}{\psline[linestyle=dashed,linecap=1,dash=1.5pt 1.5pt,linewidth=0.4pt,linecolor=darkgray]{c-c}(0,0)(0,3.544361011883063)}
\psaxes[labelFontSize=\scriptstyle,xAxis=true,yAxis=true,Dx=1.,Dy=1.,ticksize=-2pt 0,subticks=2]{->}(0,0)(0.,0.)(12.94,3.544361011883063)
\antwort{\psplot[linewidth=2.pt,plotpoints=200]{0}{12.940000000000005}{276.0*2.718281828459045^(-23.0/50.0*x)/(1125.0*(2.718281828459045^(-23.0/50.0*x))^(2.0)+150.0*2.718281828459045^(-23.0/50.0*x)+5.0)}}
\rput[tl](10.64,-0.33){$t$ in Wochen}
\rput[tl](0.26,3.2995552963475965){$g(t)$ in Metern pro Woche}
\end{pspicture*}}
	\end{center}
	
\item Ermittle eine lineare Funktion $h_1$, deren Werte bei $t=0$ und $t=12$ mit den gemessenen Höhen aus der angegebenen Tabelle übereinstimmen, und interpretiere die Steigung dieser linearen Funktion im gegebenen Kontext!

$h_1(t)=$\,\antwort[\rule{3cm}{0.3pt}]{$0,58\dot{3}\cdot t+0,6$}
	
Begründe anhand des Verlaufs der Graphen von $h$ und $h_1$, warum es mindestens zwei Zeitpunkte gibt, in denen die Wachstumsgeschwindigkeit der Pflanze denselben Wert hat wie die Steigung von $h_1$!

\item Für größer werdende $t$ nähert sich $h(t)$ einem Wert an, der als $h_{\text{max}}$ bezeichnet wird. Weise anhand der gegebenen Funktionsgleichung der Modellfunktion $h$ rechnerisch nach, dass der Parameter $k$ (mit $k<0$) keinen Einfluss auf $h_{\text{max}}$ hat, und gib $h_{\text{max}}$ an!

Günstige Witterungsverhältnisse können dazu führen, dass die Hopfenpflanze schneller und höher wächst, d.h., dass sie sich früher einem größeren Wert von $h_{\text{max}}$ annähert. Gib für ein derartiges Pflanzenwachstum an, wie $a$ und $k$ verändert werden müssen!

	\end{enumerate}
	
	\antwort{
\begin{enumerate}
	\item \subsection{Lösungserwartung:} 

Mögliche Ausdruck: $h(t_1)-h(0)$

$h(10)-h(0)=6,45$

Die Pflanze ist in den ersten 10 Wochen um ca. 6,45\,m gewachsen.\\
Die mit der Modellfunktion $h$ berechnete Zunahme der Höhe der Pflanze im Zeitintervall $[0;10]$ ist um ca. 0,8\,\%größer als die in diesem Zeitintervall tatsächlich beobachtete Zunahme (6,4\,m)

Toleranzintervall: $[6,4\,m;6,5\,m]$

\item \subsection{Lösungserwartung:} 

Mögliche Gleichung:
$h''(t)=0\Rightarrow t_2$

$t_2\approx 5,9$ Wochen Toleranzintervall: [5,4 Wochen; 6,3 Wochen]

$h'(t)\approx 0,92$

Die maximale Wachstumsgeschwindigkeit beträgt ca. 0,92 Meter pro Woche. Toleranzintervall: [0,90 Meter pro Woche; 1 Meter pro Woche]

Graph: siehe oben!

\item \subsection{Lösungserwartung:}

$h_1$: siehe oben

Mögliche Intepretation:

Die Pflanze wächst in den ersten 12 Wochen durchschnittlich um ca. 58\,cm pro Woche. Toleranzintervall: $[0,58; 0,59]$

Mögliche Begründung:

Die Steigung von $h$ ist anfangs kleiner als jene von $h_1$, dann größer und dann wieder kleiner. Es gibt daher mindestens zwei Zeitpunkte, in denen sie gleich ist.

\item \subsection{Lösungserwartung:}

Möglicher Nachweis:

Für alle $k<0$ gilt: $\lim\limits_{t \rightarrow \infty}{h(t)=\frac{a}{1+b\cdot 0}=a}$, also ist $h_{\text{max}}$ unabhängig von $k$.

$h_{\text{max}}=a$\leer

Für das beschriebene Pflanzenwachstum muss $a$ vergrößert werden und $k$ verkleinert werden.
\end{enumerate}}
	
	\end{langesbeispiel}