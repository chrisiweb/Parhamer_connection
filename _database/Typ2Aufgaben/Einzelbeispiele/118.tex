\section{118 - K5 - MZR - AG 2.1, AG-L 1.3 - Bewegungs-Sport-Woche - MatKon}

\begin{langesbeispiel} \item[6] %PUNKTE DES BEISPIELS
Die 5. Klassen fahren dieses Jahr im April gemeinsam auf eine Bewegung- und Sport-Woche nach Zell am See. Die Wintersportfans aus der Klasse können sich dabei für drei unterschiedliche Diszipline anmelden: Snowboard, Skifahren oder Langlaufen.

$B\ldots$ Anzahl der Schülerinnen und Schüler die Snowboard gewählt haben.\\
$S\ldots$ Anzahl der Schülerinnen und Schüler die Skifahren gewählt haben.\\
$L\ldots$ Anzahl der Schülerinnen und Schüler die Langlaufen gewählt haben.%Aufgabentext

\begin{aufgabenstellung}
\item 30\,\% der Kinder die Snowboard gewählt haben sind weiblich, beim Skifahren sind es 60\,\% Mädchen und beim Langlaufen sogar 63\,\%. Insgesamt sind 35 Mädchen mit auf der Skiwoche.%Aufgabentext

\Subitem{Stelle eine Gleichung auf die diese Aussage widerspiegelt.} %Unterpunkt1

Die Unterkunft im Hotel kostet für SchülerInnen für die 5 Tage insgesamt 470\,\euro. Sollte die Gruppe aber größer als 72 sein gewährt das Hotel einen Rabatt von 15\,\% auf die Gesamtkosten. Zusätzlich muss jedes 6er Zimmer noch eine Reinigungspauschale von 22\,\euro bezahlen.

\Subitem{Berechne die Gesamtkosten für die 84 SchülerInnen die auf 14 Zimmer verteilt die Woche in diesem Hotel verbracht haben.} %Unterpunkt2

\item Da nicht jede Schülerin/jeder Schüler eine eigene Ausrüstung hat müssen sich einige SchülerInnen sich ihre Ausrüstung ausborgen.

$H\ldots$ Menge aller SchülerInnen die sich einen Helm ausborgen.\\
$S\ldots$ Menge aller SchülerInnen die sich Stöcke ausborgen.\\
$K\ldots$ Menge aller SchülerInnen die sich Ski ausborgen.

\begin{center}
\resizebox{0.4\linewidth}{!}{
\psset{xunit=1.0cm,yunit=1.0cm,algebraic=true,dimen=middle,dotstyle=o,dotsize=5pt 0,linewidth=1.6pt,arrowsize=3pt 2,arrowinset=0.25}
\begin{pspicture*}(0.7154761723875425,-0.21516994702280584)(8.313331564205987,6.239434476747702)
\antwort{\parametricplot[linewidth=0.8pt,linecolor=red,hatchcolor=red,fillstyle=hlines,hatchangle=45.0,hatchsep=0.1]{2.4188584057763776}{3.8643269014032087}{1.*2.*cos(t)+0.*2.*sin(t)+6.|0.*2.*cos(t)+1.*2.*sin(t)+4.}
\parametricplot[linewidth=0.8pt,linecolor=red,hatchcolor=red,fillstyle=hlines,hatchangle=45.0,hatchsep=0.1]{-0.7227342478134151}{0.7227342478134158}{1.*2.*cos(t)+0.*2.*sin(t)+3.|0.*2.*cos(t)+1.*2.*sin(t)+4.}
\pscustom[linewidth=0.8pt,linecolor=white,fillcolor=white,fillstyle=solid,opacity=1]{\parametricplot{1.0955602282695138}{2.0814733472897293}{1.*2.002498439450078*cos(t)+0.*2.002498439450078*sin(t)+4.5|0.*2.002498439450078*cos(t)+1.*2.002498439450078*sin(t)+2.}\lineto(4.5,2.)\closepath}}
\pscircle[linewidth=0.8pt](3.,4.){2.}
\pscircle[linewidth=0.8pt](6.,4.){2.}
\pscircle[linewidth=0.8pt](4.5,2.){2.0024984394500787}
\rput[tl](7.551164252174042,5.7){$K$}
\rput[tl](6.15,0.8804455640231118){$H$}
\rput[tl](1.1203775569045127,5.88216854923273){$S$}
\end{pspicture*}}
\end{center}%Aufgabentext

\Subitem{Schraffiere (Kennzeichne) im obigen Diagramm folgende Menge: $(S\cap K)\backslash H$} %Unterpunkt1
\Subitem{Beschreibe die Menge $(S\cap K)\backslash H$ in eigenen Worten.} %Unterpunkt2

\item Um den SchülerInnen die größte Sicherheit zu gewährleisten, wird jeden Tag in der Früh die aktuelle "`Schneeart"' klassizifiert. Dafür greift man auf folgende Tabelle zurück:
\begin{center}
\resizebox{1\linewidth}{!}{
\begin{tabular}{|c|c|c|}\hline
\cellcolor[gray]{0.9}Dichte&\cellcolor[gray]{0.9}Bezeichnung&\cellcolor[gray]{0.9}Beschreibung\\ \hline
$30$ bis $50$ kg\,$\cdot$\,m$^{-3}$&trockener, lockerer Neuschnee&100\,kg sind auf einem m$^2$ ca. 2-3\,m hoch\\ \hline
$50$ bis $100$ kg\,$\cdot$\,m$^{-3}$&gebundener Neuschnee&100\,kg sind auf einem m$^2$ ca. 1-2\,m hoch\\ \hline
$100$ bis $200$ kg\,$\cdot$\,m$^{-3}$& stark gebundener Neuschnee&100\,kg sind auf einem m$^2$ ca. 0,5-1\,m hoch\\ \hline
$200$ bis $400$ kg\,$\cdot$\,m$^{-3}$&trockener Altschnee&100\,kg sind auf einem m$^2$ ca. 25-50\,cm hoch\\ \hline
$300$ bis $500$ kg\,$\cdot$\,m$^{-3}$&feuchtnasse Altschnee&100\,kg sind auf einem m$^2$ ca. 20-35\,cm hoch\\ \hline
$150$ bis $300$ kg\,$\cdot$\,m$^{-3}$&Schwimmschnee&100\,kg sind auf einem m$^2$ ca. 30-70\,cm hoch\\ \hline
$500$ bis $800$ kg\,$\cdot$\,m$^{-3}$&mehrjähriger Film&100\,kg sind auf einem m$^2$ ca. 12-20\,cm hoch\\ \hline
$800$ bis $900$ kg\,$\cdot$\,m$^{-3}$&Eis&100\,kg sind auf einem m$^2$ ca. 11-12\,cm hoch\\ \hline
\end{tabular}}


\tiny{Quelle: https://wiki.bildungsserver.de/klimawandel/index.php/Schnee}
\end{center}

Bei einer der morgendlichen Schneemessungen kam man bei einer Menge von 100\,kg auf einem m$^2$ auf eine Höhe von $150\cdot 10^{-3}$\,m heraus.%Aufgabentext

\Subitem{Gib an, um welche Art von Schnee es sich dabei handelt.} %Unterpunkt1

Am darauffolgenden Tag wurde in der Nachbargemeinde Schnee mit einer Dichte von $3,5\cdot 10^4$\,g\,$\cdot$\,m$^{-3}$ gemessen.

\Subitem{Gib an, um welche Art von Schnee es sich dabei handelt.} %Unterpunkt2

\end{aufgabenstellung}

\begin{loesung}
\item \subsection{Lösungserwartung:} 

\Subitem{$0,3\cdot B+0,6\cdot S+0,63\cdot L=35$} %Lösung von Unterpunkt1
\Subitem{$470\cdot 84\cdot 0,85+14\cdot 22=33\,866$\,\euro} %%Lösung von Unterpunkt2

\setcounter{subitemcounter}{0}
\subsection{Lösungsschlüssel:}
 
\Subitem{Ein Punkt für die richtige Gleichung.} %Lösungschlüssel von Unterpunkt1
\Subitem{Ein Punkt für die richtige Berechnung des Gesamtpreises.} %Lösungschlüssel von Unterpunkt2

\item \subsection{Lösungserwartung:} 

\Subitem{Lösung: siehe Grafik oben.} %Lösung von Unterpunkt1
\Subitem{Menge aller SchülerInnen die sich Stöcke und Ski ausborgen aber keinen Helm.} %%Lösung von Unterpunkt2

\setcounter{subitemcounter}{0}
\subsection{Lösungsschlüssel:}
 
\Subitem{Ein Punkt für die richtige grafische Darstellung.} %Lösungschlüssel von Unterpunkt1
\Subitem{Ein Punkt für die richtige Interpretation.} %Lösungschlüssel von Unterpunkt2

\item \subsection{Lösungserwartung:} 

\Subitem{$150\cdot 10^{-3}\,\text{m}=0,15\,\text{m}=15\,\text{cm} \Rightarrow$ mehrjähriger Film} %Lösung von Unterpunkt1
\Subitem{$3,5\cdot 10^4=35000\,\text{g}\cdot\text{m}^{-3}=35\,\text{kg}\cdot\text{m}^{-3} \Rightarrow$ trockener, lockerer Neuschnee} %%Lösung von Unterpunkt2

\setcounter{subitemcounter}{0}
\subsection{Lösungsschlüssel:}
 
\Subitem{Ein Punkt für die richtige Schneeart.} %Lösungschlüssel von Unterpunkt1
\Subitem{Ein Punkt für die richtige Schneeart.} %Lösungschlüssel von Unterpunkt2

\end{loesung}

\end{langesbeispiel}