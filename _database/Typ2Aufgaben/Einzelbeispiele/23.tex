\section{23 - MAT - AN 1.3, AN 3.3, FA 1.2, FA 1.5 - Kugelstoßen - BIFIE Aufgabensammlung}

\begin{langesbeispiel} \item[0] %PUNKTE DES BEISPIELS
				Für die Beschreibung der Flugbahn der gestoßenen Kugel beim Kugelstoßen kann mit guter Näherung die Gleichung der Wurfparabel verwendet werden.
				
				Diese Gleichung lautet: $y=\tan\beta\cdot x-\dfrac{g}{2v_0^2\cdot \cos²\beta}\cdot x² (g=9,81\,m/s²)$.
				
				Dabei ist $v_0$ die Abwurfgeschwindigkeit der Kugel und $\beta$ der Winkel, unter dem die Parabel die x-Achse schneidet. Die größte Wurfweite wird für $\beta=45^\circ$ erzielt.
				
				\begin{center}\resizebox{0.6\linewidth}{!}{
\psset{xunit=1.0cm,yunit=1.0cm,algebraic=true,dimen=middle,dotstyle=o,dotsize=5pt 0,linewidth=0.5pt,arrowsize=3pt 2,arrowinset=0.25}
\begin{pspicture*}(-0.46813656224127825,-0.6563896332019663)(4.866580767978309,3.567991793359467)
\psaxes[labelFontSize=\scriptstyle,xAxis=true,yAxis=true,labels=none,Dx=1.,Dy=1.,ticksize=0pt 0]{->}(0,0)(-0.46813656224127825,-0.6563896332019663)(4.866580767978309,3.567991793359467)
\psplot[linewidth=0.5pt,plotpoints=200]{0.81}{3.5}{0.84*x*4.0-0.06*(x*4.0)^(2.0)}
\psplot{0}{4.866580767978309}{(-0.--3.36*x)/1.}
\parametricplot{-0.0}{1.281525321185755}{0.38287444953729094*cos(t)+0.|0.38287444953729094*sin(t)+0.}
\psline(0.8145856533065148,2.1)(2.16595570624108,2.1)
\psplot{0.81}{4.866580767978309}{(--0.6370077951098894--1.7959955456514916*x)/1.}
\parametricplot{-0.0}{1.0627517676068867}{0.5104992660497213*cos(t)+0.8145856533065146|0.5104992660497213*sin(t)+2.1}
\psplot[linewidth=0.5pt,linestyle=dashed,dash=2pt 2pt,plotpoints=200]{0}{0.81}{0.84*x*4.0-0.06*(x*4.0)^(2.0)}
\psline[linestyle=dashed,dash=2pt 2pt](0.8145856533065148,2.1)(0.8145856533065148,0.)
\psline(3.5,-0.2)(0.81,-0.2)
\begin{scriptsize}
\rput[tl](0.1444625570183873,3.516941866754495){y}
\rput[tl](4.305031575323616,0.2390839355949878){x}
\rput[tl](3.249945304690333,1.2835075777869698){Flugbahn}
\rput[tl](1.637672910213822,-0.3373275919208913){Wurfweite}
\rput[tl](1.400986052420316,2.319368444747597){Abstoßwinkel}
\rput[tl](0.9378362726053996,1.9799191053028489){Abstoßpunkt}
\rput[bl](0.08,0.03907182091888473){$\beta$}
\psdots[dotsize=3pt 0,dotstyle=*,linecolor=darkgray](0.8145856533065148,2.1)
\rput[bl](1.0250737909541567,2.189643775025223){$\alpha$}
\psdots[dotsize=3pt 0,dotstyle=*,linecolor=darkgray](3.5,0.)
\psdots[dotsize=3pt 0,dotstyle=*](1.2395770320384778,2.6898896580262575)
\psdots[dotsize=3pt 0,dotstyle=*](2.1473623906205583,2.7884190052995015)
\psdots[dotsize=3pt 0,dotstyle=*](2.8366455041888265,1.8064334862971618)
\psdots[dotsize=3pt 0,dotstyle=*](3.3987647610482936,0.3303117722163478)
\psdots[dotsize=3pt 0,dotstyle=|](0.81,-0.2)
\psdots[dotsize=3pt 0,dotstyle=|](3.5,-0.2)
\end{scriptsize}\end{pspicture*}} \end{center}

Die Computersimulation der Flugbahn der gestoßenen Kugel eines Athleten ergab für eine Gleichung der Bahnkurve $y=0,84\cdot x-0,06\cdot x²$. Der Abstoßpunkt der Kugel befand sich in einer Höhe von 2,1\,m.

\subsection{Aufgabenstellung:}
\begin{enumerate}
	\item Berechne die Größe des Abstoßwinkels $\alpha$ und die maximale Höhe, die von der Kugel des Athleten erreicht wurde! Runde auf cm!
	
	\item Welche Wurfweite hat der Athlet erzielt? Welchen Einfluss hat die Größe der Fallbeschleunigung $g$ bei sonst gleichen Bedingungen auf die Wurfweite? Begründe deine Antwort!
	
	\item Berechne für die Bahnkurve $y=0,84\cdot x-0,06\cdot x²$ die Größe des Winkel $\beta$ und überprüfe, ob dieser Athlet die größte Wurfweite erreicht hat!
	
	Erläutere, ob anhand der Parameter $a$ und $b$ in der allgemeinen Bahnkurve $y=ax-bx²$ bereits feststellbar ist, ob eine Athletin/ein Athlet die größte Wurfweite erzielt hat!	
						\end{enumerate}\leer
				
\antwort{\subsection{Lösungserwartung:}
\begin{enumerate}
	\item $f:y=0,84\cdot x-0,06\cdot x²$
	
	Abstoßpunkt: Höhe 2,1\,m
	
	$2,1=0,84\cdot x-0,06\cdot x²$, $x_1\approx 3,26$, $x_2\approx 10,74$
	
	$f'(x)=0,84-0,12\cdot x$, $f'(3,26)=0,4488$, $\tan\alpha=0,4488$, $\alpha\approx 24,18^\circ$ bzw. $\alpha\approx 0,42$ rad
	
	$f'(x)=0,84-0,12\cdot x$, $0=0,81-0,12\cdot x$, $x=7$, $f(7)=2,94$
	
	Die maximale Höhe der Kugel betrug 2,94\,m.
	
	\item $f:y=0,84\cdot x-0,06\cdot x²$
	
	Abstoßpunkt: Höhe 2,1\,m
	
	$2,1=0,84\cdot x-0,06\cdot x²$, $(x_1\approx 3,26)$, $x_2\approx 10,74$
	
	Nullstellen: $0=0,84\cdot x-0,06\cdot x²$, $(x_1=0)$ und $x_2=14$
	
	$14-3,26=10,74$
	
	Die Wurfweite der Kugel war 10,74\,m.
	
	Die Wurfweite wird bestimmt durch die rechte Nullstelle der Parabel:
	
	$0=\tan\beta\cdot x-\dfrac{g}{2v_0^2\cdot \cos²\beta}\cdot x²$
	
	$0=x\cdot\left(\tan\beta-\dfrac{g}{2v_0^2\cdot \cos²\beta}\cdot x\right)$
	
	$x=\dfrac{\tan\beta\cdot 2v_0^2\cdot\cos^2\beta}{g}$
			
			Bei größerem $g$ wird die Wurfweite kleiner. Es liegt eine indirekte Proportionalität vor.
			
			\item $f'(x)=0,84-0,12\cdot x$, $f'(0)=k$, $\tan\beta=k$, $\beta\approx 40,04^\circ<45^\circ$
			
			Da der Winkel $\beta$ ungleich $45^\circ$ ist, könnte der Athlet durch Veränderung des Abstoßwinkels eine größere Wurfweite erzielen.
			
			Die Berechnung von Beta kann auch graphisch mit Hilfe von Technologie erfolgen.
			\end{enumerate}}
		\end{langesbeispiel}