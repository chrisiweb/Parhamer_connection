\section{15 - MAT - AG 2.1, FA 1.5, FA 2.3, FA 2.5 - Treibstoffverbrauch - BIFIE Aufgabensammlung}

\begin{langesbeispiel} \item[0] %PUNKTE DES BEISPIELS
Fast vier Fünftel aller Güter werden zumindest auf einem Teil ihres Weges vom Erzeuger zum Konsumenten mit dem Schiff transportiert.
				
				In der Schifffahrt werden Entfernungen in Seemeilen (1\,sm = 1,852\,km) und Geschwindigkeiten in Knoten (1\,K = 1\,sm/h) angegeben.
				
				Der stündliche Treibstoffverbrauch $y$ des Schiffs \textit{Ozeanexpress} kann in Abhängigkeit von der Geschwindigkeit $x$ (in Knoten) durch die Gleichung\\ 
				$y=0,00002*x^4+0,6$ beschrieben werden. Dieses Schiff hat noch einen Treibstoffvorrat von 600 Tonnen.%Aufgabentext

\begin{aufgabenstellung}
\item %Aufgabentext

\Subitem{Gib eine Formel für die Zeit $t$ (in Stunden) an, die das Schiff mit einer konstanten Geschwindigkeit $x$ unterwegs sein kann, bis dieser Treibstoffvorrat aufgebraucht ist.
	
	Die Funktion $f$ soll den Weg $f(x)$ beschreiben, den das Schiff mit diesem Treibstoffvorrat bei einer konstanten Geschwindigkeit $x$ zurücklegen kann. Gib den Term der Funktion $f$ an.} %Unterpunkt1
\Subitem{Die Funktion $f$ hat in $H=(10\mid 7\,500)$ ein Maximum, Interpretiere die Koordinaten dieses Punktes im vorliegenden Kontext.} %Unterpunkt2

\item Der Chef eines Schifffahrtsunternehmens stellte fest, dass sich der Treibstoffverbrauch um rund $50\,\%$ verringert, wenn Schiffe statt mit 25 nur noch mit 20 Knoten unterwegs sind.
	
	In der nachstehenden Grafik wird der Treibstoffverbrauch in Abhängigkeit vom zurückgelegten Weg bei einer Geschwindigkeit von 25 Knoten dargestellt.
	\begin{center}
	\psset{xunit=0.006cm,yunit=0.014cm,algebraic=true,dimen=middle,dotstyle=o,dotsize=5pt 0,linewidth=0.8pt,arrowsize=3pt 2,arrowinset=0.25}
\begin{pspicture*}(-121.42274509803957,-56.53721854305261)(1389.3564102564137,551.2378807947575)
\multips(0,0)(0,50.0){11}{\psline[linestyle=dashed,linecap=1,dash=1.5pt 1.5pt,linewidth=0.4pt,linecolor=gray]{c-c}(0,0)(1389.3564102564137,0)}
\multips(-200,0)(200.0,0){8}{\psline[linestyle=dashed,linecap=1,dash=1.5pt 1.5pt,linewidth=0.4pt,linecolor=gray]{c-c}(0,0)(0,551.2378807947575)}
\psaxes[labelFontSize=\scriptstyle,showorigin=false,xAxis=true,yAxis=true,Dx=200.,Dy=100.,ticksize=-2pt 0,subticks=0]{->}(0,0)(-121.42274509803957,-56.53721854305261)(1389.3564102564137,551.2378807947575)
\psline(200.,100.)(1400.,450.)
\antwort{\psplot{200.}{1389.3564102564137}{(--20000.--150.*x)/1000.}}
\begin{scriptsize}
\rput[tl](16.345369532428347,517.6689072848203){Treibstoffverbrauch in $t$}
\rput[tl](810.9953242835618,30.70327814569971){zurückgelegter Weg in sm}
\rput[tl](674.8302564102581,300.85579470201947){$25\,K$}
\antwort{\rput[tl](866.304585218705,189.6824503311487){$20\,K$}}
\end{scriptsize}
\end{pspicture*}\end{center}%Aufgabentext

\Subitem{Überlege, wie sich diese Grafik ändert, wenn die Geschwindigkeit nur 20 Knoten beträgt, und zeichne den entsprechenden Graphen ein.} %Unterpunkt1
\Subitem{Interpretiere, was die $50\,\%$ige Treibstoffreduktion für die Steigung der Geraden bedeutet.} %Unterpunkt2

\item Eine Reederei hat den Auftrag erhalten, in einem vorgegebenen Zeitraum eine bestimmte Warenmenge zu transportieren. Ursprünglich plante sie, dafür acht Schiffe einzusetzen.%Aufgabentext

\Subitem{Gib an, wie viele zusätzliche Schiffe gleichen Typs bei einer Drosselung der Geschwindigkeit von 25 auf 20 Knoten erforderlich sind, damit der Auftrag zeitgerecht ausgeführt werden kann (Die Stehzeiten der Schiffe sind dabei zu vernachlässigen).} %Unterpunkt1
\Subitem{Gib eine Formel an, mit der die erforderliche Anzahl der Schiffe in Abhängigkeit von der Geschwindigkeit $x$ ermittelt werden kann.} %Unterpunkt2

\end{aufgabenstellung}

\begin{loesung}
\item \subsection{Lösungserwartung:} 

\Subitem{$t=\dfrac{600}{0,00002x^4+0,6}$

$f(x)=\dfrac{600}{0,00002x^4+0,6}\cdot x$} %Lösung von Unterpunkt1
\Subitem{Bei einer Geschwindigkeit von 10 Knoten kann mit dem vorhandenen Treibstoff die längste Strecke, nämlich 7\,500 Seemeilen, zurückgelegt werden.} %%Lösung von Unterpunkt2

\setcounter{subitemcounter}{0}
\subsection{Lösungsschlüssel:}
 
\Subitem{Ein Punkt für die richtigen Gleichungen für $t$ und $f(x)$.} %Lösungschlüssel von Unterpunkt1
\Subitem{Ein Punkt für eine korrekte Interpretation.} %Lösungschlüssel von Unterpunkt2

\item \subsection{Lösungserwartung:} 

\Subitem{Grafik: siehe oben.} %Lösung von Unterpunkt1
\Subitem{Die Steigung der Geraden wird halbiert, wenn der Treibstoffverbrauch um 50\,\% reduziert wird.} %%Lösung von Unterpunkt2

\setcounter{subitemcounter}{0}
\subsection{Lösungsschlüssel:}
 
\Subitem{Ein Punkt für den richtig eingezeichneten Graphen.} %Lösungschlüssel von Unterpunkt1
\Subitem{Ein Punkt für die richtige Interpretation.} %Lösungschlüssel von Unterpunkt2

\item \subsection{Lösungserwartung:} 

\Subitem{Es müssen zwei weitere Schiffe eingesetzt werden.} %Lösung von Unterpunkt1
\Subitem{Anzahl der Schiffe = $\frac{200}{x}$} %%Lösung von Unterpunkt2

\setcounter{subitemcounter}{0}
\subsection{Lösungsschlüssel:}
 
\Subitem{Ein Punkt für die richtige Anzahl der Schiffe.} %Lösungschlüssel von Unterpunkt1
\Subitem{Ein Punkt für die richtige Formel.} %Lösungschlüssel von Unterpunkt2

\end{loesung}

\end{langesbeispiel}