\section{15 - MAT - AG 2.1, FA 1.5, FA 2.3, FA 2.5 - Treibstoffverbrauch - BIFIE Aufgabensammlung}

\begin{langesbeispiel} \item[0] %PUNKTE DES BEISPIELS
				Fast vier F�nftel aller G�ter werden zumindest auf einem Teil ihres Weges vom Erzeuger zum Konsumenten mit dem Schiff transportiert.
				
				In der Schifffahrt werden Entfernungen in Seemeilen (1\,sm = 1,852\,km) und Geschwindigkeiten in Knoten (1\,K = 1\,sm/h) angegeben.
				
				Der st�ndliche Treibstoffverbrauch $y$ des Schiffs \textit{Ozeanexpress} kann in Abh�ngigkeit von der Geschwindigkeit $x$ (in Knoten) durch die Gleichung $y=0,00002*x^4+0,6$ beschrieben werden. Dieses Schiff hat noch einen Treibstoffvorrat von 600 Tonnen.
				
\subsection{Aufgabenstellung:}
\begin{enumerate}
	\item Gib eine Formel f�r die Zeit $t$ (in Stunden) an, die das Schiff mit einer konstanten Geschwindigkeit $x$ unterwegs sein kann, bis dieser Treibstoffvorrat aufgebraucht ist.
	
	Die Funktion $f$ soll den Weg $f(x)$ beschreiben, den das Schiff mit diesem Treibstoffvorrat bei einer konstanten Geschwindigkeit $x$ zur�cklegen kann. Gib den Term der Funktion $f$ an!
	
	Die Funktion $f$ hat in $H\,(10|7\,500)$ ein Maximum, Interpretiere die Koordinaten dieses Punktes im vorliegenden Kontext!
	
	\item Der Chef eines Schifffahrtsunternehmens stellte fest, dass sich der Treibstoffverbrauch um rund $50\,\%$ verringert, wenn Schiffe statt mit 25 nur noch mit 20 Knoten unterwegs sind.
	
	In der nachstehenden Grafik wird der Treibstoffverbrauch in Abh�ngigkeit vom zur�ckgelegten Weg bei einer Geschwindigkeit von 25 Knoten dargestellt.
	
	�berlege, wie sich diese Grafik �ndert, wenn die Geschwindigkeit nur 20 Knoten betr�gt, und zeichne den entsprechenden Graphen ein!
	
	Interpretiere, was die $50\,\%$ige Treibstoffreduktion f�r die Steigung der Geraden bedeutet!
	
	\psset{xunit=0.006cm,yunit=0.014cm,algebraic=true,dimen=middle,dotstyle=o,dotsize=5pt 0,linewidth=0.8pt,arrowsize=3pt 2,arrowinset=0.25}
\begin{pspicture*}(-121.42274509803957,-56.53721854305261)(1389.3564102564137,551.2378807947575)
\multips(0,0)(0,50.0){13}{\psline[linestyle=dashed,linecap=1,dash=1.5pt 1.5pt,linewidth=0.4pt,linecolor=lightgray]{c-c}(0,0)(1389.3564102564137,0)}
\multips(-200,0)(200.0,0){8}{\psline[linestyle=dashed,linecap=1,dash=1.5pt 1.5pt,linewidth=0.4pt,linecolor=lightgray]{c-c}(0,0)(0,551.2378807947575)}
\psaxes[labelFontSize=\scriptstyle,xAxis=true,yAxis=true,Dx=200.,Dy=100.,ticksize=-2pt 0,subticks=2]{->}(0,0)(-121.42274509803957,-56.53721854305261)(1389.3564102564137,551.2378807947575)
\psline(200.,100.)(1400.,450.)
\antwort{\psplot{200.}{1389.3564102564137}{(--20000.--150.*x)/1000.}}
\begin{scriptsize}
\rput[tl](16.345369532428347,517.6689072848203){Treibstoffverbrauch in $t$}
\rput[tl](810.9953242835618,30.70327814569971){zur�ckgelegter Weg in sm}
\rput[tl](674.8302564102581,300.85579470201947){$25\,K$}
\antwort{\rput[tl](866.304585218705,189.6824503311487){$20\,K$}}
\end{scriptsize}
\end{pspicture*}\leer

\item Eine Reederei hat den Auftrag erhalten, in einem vorgegebenen Zeitraum eine bestimmte Warenmenge zu transporten. Urspr�nglich plante sie, daf�r acht Schiffe einzusetzen.

Gib an, wie viele zus�tzliche Schiffe gleichen Typs bei einer Drosselung der Geschwindigkeit von 25 auf 20 Knoten erforderlich sind, damit der Auftrag zeitgerecht ausgef�hrt werden kann (Die Stehzeiten der Schiffe sind dabei zu vernachl�ssigen)

Gib eine Formel an, mit der die erforderliche Anzahl der Schiffe in Abh�ngigkeit von der Geschwindigkeit $x$ ermittelt werden kann!
					\end{enumerate}\leer
				
\antwort{\subsection{L�sungserwartung:}
\begin{enumerate}
	\item $t=\dfrac{600}{0,00002x^4+0,6}; f(x)=\dfrac{600}{0,00002x^4+0,6}\cdot x$\leer
	
	Bei einer Geschwindigkeit von 10 Knoten kann mit dem vorhandenen Treibstoff die l�ngste Strecke, n�mlich 7\,500 Seemeilen, zur�ckgelegt werden.
	
	\item Grafik siehe oben
	
	Die Steigung der Geraden wird halbiert, wenn der Treibstoffverbrauch um 50\,\% reduziert wird.
	
	\item Es m�ssen zwei weitere Schiffe eingesetzt werden.
	
	Anzahl der Schiffe = $\frac{200}{x}$	
		\end{enumerate}}
\end{langesbeispiel}