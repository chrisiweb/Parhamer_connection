\section{13 - MAT - FA 1.3, FA 1.4, FA 1.7, AG 2.1 - Photovoltaikanlagen - BIFIE Aufgabensammlung}

\begin{langesbeispiel} \item[0] %PUNKTE DES BEISPIELS
Die benachbarten Familien Lux und Hell haben auf den Dächern ihrer Einfamilienhäuser zwei baugleiche Photovoltaikanlagen installiert, deren Maximalleistung jeweils 5 kW beträgt. Nicht selbst verbrauchte elektrische Energie wird zu einem Einspeisetarif ins Netz geliefert. Energie, die nicht durch die Photovoltaikanlage bereitgestellt werden kann, muss von einem Energieunternehmen zum regulären Stromtarif zugekauft werden (netzgekoppelter Betrieb). Beide Familien wollen die Wirtschaftlichkeit ihrer Anlagen durch Messungen überprüfen. %Aufgabentext

\begin{aufgabenstellung}
\item Familie Lux hat dazu an einem durchschnittlichen Frühlingstag folgende Leistungsdaten für $P_B$ (im Haus der Familie Lux benötigte Leistung) und $P_E$ (durch die Photovoltaikanlage erzeugte elektrische Leistung) in Abhängigkeit vom Zeitpunkt $t$ über den Tagesverlauf ermittelt. Die Leistungsdaten wurden um Mitternacht beginnend alle zwei Stunden aufgezeichnet.
	
	Leistungsdaten:
	
\resizebox{1\linewidth}{!}{\begin{tabular}{|l|c|c|c|c|c|c|c|c|c|c|c|c|c|} \hline
	$t$ in h&0&2&4&6&8&10&12&14&16&18&20&22&24\\ \hline
	$P_B$ in kW&0,5&0,5&0,5&0,5&3&1,5&3,5&2&2,9&3,7&2,5&0,5&0,5\\ \hline
	$P_E$ in kW&0&0&0&0&1,5&3,75&4,2&4,2&3&0,75&0&0&0\\ \hline	
	\end{tabular}}%Aufgabentext

\Subitem{Zeichne in die nachstehende Abbildung aus den vorgegebenen Graphen für $P_B$ und $P_E$ den Tagesverlauf der elektrischen Gesamtleistungsbilanz $P_{ges}$ für die Photovoltaikanlage der Familie Lux ein. Die Gesamtleistungsbilanz $P_{ges}$ ist die Differenz der beiden Leistungsteile $P_E-P_B$.

\newrgbcolor{zzttqq}{0.6 0.2 0.}
\psset{xunit=0.5cm,yunit=1.0cm,algebraic=true,dimen=middle,dotstyle=o,dotsize=5pt 0,linewidth=0.8pt,arrowsize=3pt 2,arrowinset=0.25}
\begin{pspicture*}(-1.565217391304349,-4.1)(25.808695652173924,5.22)
\multips(0,-5)(0,1.0){12}{\psline[linestyle=dashed,linecap=1,dash=1.5pt 1.5pt,linewidth=0.4pt,linecolor=gray]{c-c}(0,0)(25.808695652173924,0)}
\multips(0,0)(2.0,0){14}{\psline[linestyle=dashed,linecap=1,dash=1.5pt 1.5pt,linewidth=0.4pt,linecolor=gray]{c-c}(0,-4.06)(0,5.22)}
\psaxes[labelFontSize=\scriptstyle,showorigin=false,xAxis=true,yAxis=true,Dx=2.,Dy=1.,ticksize=-2pt 0,subticks=0]{->}(0,0)(0,-4.06)(25.808695652173924,5.22)[\text{$t[h]$},140] [\text{$P[kW]$},-40]
\antwort{\pspolygon[linecolor=zzttqq,fillcolor=zzttqq,fillstyle=solid,opacity=0.10](6.,0.)(8.,1.5)(10.,3.75)(12.,4.2)(14.,4.2)(16.,3.)(18.,0.75)(20.,0.)}
\psline[linewidth=2.4pt,linecolor=blue](0.,0.5)(6.,0.5)
\psline[linewidth=2.4pt,linecolor=blue](6.,0.5)(8.,3.)
\psline[linewidth=2.4pt,linecolor=blue](8.,3.)(10.,1.5)
\psline[linewidth=2.4pt,linecolor=blue](10.,1.5)(12.,3.5)
\psline[linewidth=2.4pt,linecolor=blue](12.,3.5)(14.,2.)
\psline[linewidth=2.4pt,linecolor=blue](14.,2.)(16.,2.9)
\psline[linewidth=2.4pt,linecolor=blue](16.,2.9)(18.,3.7)
\psline[linewidth=2.4pt,linecolor=blue](18.,3.7)(20.,2.5)
\psline[linewidth=2.4pt,linecolor=blue](20.,2.5)(22.,0.5)
\psline[linewidth=2.4pt,linecolor=blue](22.,0.5)(24.,0.5)
\rput[tl](2.9565217391304355,1.16){$P_B$}
\psline[linewidth=2.4pt,linestyle=dashed,dash=2pt 2pt,linecolor=green](0.,0.)(6.,0.)
\psline[linewidth=2.4pt,linestyle=dashed,dash=2pt 2pt,linecolor=green](6.,0.)(8.,1.5)
\psline[linewidth=2.4pt,linestyle=dashed,dash=2pt 2pt,linecolor=green](8.,1.5)(10.,3.75)
\psline[linewidth=2.4pt,linestyle=dashed,dash=2pt 2pt,linecolor=green](10.,3.75)(12.,4.2)
\psline[linewidth=2.4pt,linestyle=dashed,dash=2pt 2pt,linecolor=green](12.,4.2)(14.,4.2)
\psline[linewidth=2.4pt,linestyle=dashed,dash=2pt 2pt,linecolor=green](14.,4.2)(16.,3.)
\psline[linewidth=2.4pt,linestyle=dashed,dash=2pt 2pt,linecolor=green](16.,3.)(18.,0.75)
\psline[linewidth=2.4pt,linestyle=dashed,dash=2pt 2pt,linecolor=green](18.,0.75)(20.,0.)
\psline[linewidth=2.4pt,linestyle=dashed,dash=2pt 2pt,linecolor=green](20.,0.)(24.,0.)
\rput[tl](11.61739130434783,4.82){$P_E$}
\antwort{\psline[linewidth=2.4pt,linecolor=red](0.,-0.5)(6.,-0.5)
\psline[linewidth=2.4pt,linecolor=red](6.,-0.5)(8.,-1.5)
\psline[linewidth=2.4pt,linecolor=red](8.,-1.5)(10.,2.25)
\psline[linewidth=2.4pt,linecolor=red](10.,2.25)(12.,0.7)
\psline[linewidth=2.4pt,linecolor=red](12.,0.7)(14.,2.2)
\psline[linewidth=2.4pt,linecolor=red](14.,2.2)(16.,0.)
\psline[linewidth=2.4pt,linecolor=red](16.,0.)(18.,-3.)
\psline[linewidth=2.4pt,linecolor=red](18.,-3.)(20.,-2.5)
\psline[linewidth=2.4pt,linecolor=red](20.,-2.5)(22.,-0.5)
\psline[linewidth=2.4pt,linecolor=red](22.,-0.5)(24.,-0.5)}
\antwort{\rput[tl](21.356521739130443,-1.74){$P_{ges}$}}
\end{pspicture*}} %Unterpunkt1
\Subitem{Markiere zusätzlich in der Abbildung die gesamte über den Tagesverlauf erzeugte elektrische Energie dieser Photovoltaikanlage.} %Unterpunkt2

\item Familie Hell hat aus einer an einem durchschnittlichen Frühlingstag erfolgten stündlichen Messung folgenden Tagesverlauf für die Gesamtleistungsbilanz $P_{ges}$ ihrer elektrischen Leistung ermittelt und durch den gegebenen Graphen modelliert:\leer

\begin{center}
\resizebox{0.8\linewidth}{!}{\psset{xunit=0.5cm,yunit=1.0cm,algebraic=true,dimen=middle,dotstyle=o,dotsize=5pt 0,linewidth=0.8pt,arrowsize=3pt 2,arrowinset=0.25}
\begin{pspicture*}(-1.683348622767671,-5.603257843543216)(25.49141747376692,4.568685301396118)
\multips(0,-5)(0,1.0){13}{\psline[linestyle=dashed,linecap=1,dash=1.5pt 1.5pt,linewidth=0.4pt,linecolor=gray]{c-c}(0,0)(25.49141747376692,0)}
\multips(0,0)(1.0,0){28}{\psline[linestyle=dashed,linecap=1,dash=1.5pt 1.5pt,linewidth=0.4pt,linecolor=gray]{c-c}(0,-5.603257843543216)(0,6.568685301396118)}
\begin{scriptsize}
\psaxes[xAxis=true,yAxis=true,showorigin=false,Dx=2.,Dy=1.,ticksize=-2pt 0,subticks=0]{->}(0,0)(-1.683348622767671,-5.603257843543216)(25.49141747376692,4.568685301396118)[\text{$t[h]$},140] [\text{$P[kW]$},-40]
\psline[linewidth=2.4pt](0.,-0.5)(6.,-0.5)
\psline[linewidth=2.4pt](10.,2.25)(12.,0.7)
\psline[linewidth=2.4pt](14.,2.2)(16.,0.)
\psline[linewidth=2.4pt](16.,0.)(18.,-3.)
\psline[linewidth=2.4pt](22.,-0.5)(24.,-0.5)
\psline[linewidth=2.4pt](6.,-0.5)(7.,-2.)
\psline[linewidth=2.4pt](7.,-2.)(8.,-1.5)
\psline[linewidth=2.4pt](8.,-1.5)(9.,0.)
\psline[linewidth=2.4pt](9.,0.)(10.,2.25)
\psline[linewidth=2.4pt](12.,0.7)(13.01020571726129,1.2945018317046904)
\psline[linewidth=2.4pt](13.01020571726129,1.2945018317046904)(14.,2.2)
\psline[linewidth=2.4pt](18.,-3.)(19.,-4.)
\psline[linewidth=2.4pt](19.,-4.)(20.,-2.5)
\psline[linewidth=2.4pt](20.,-2.5)(21.,-2.)
\psline[linewidth=2.4pt](21.,-2.)(22.,-0.5)

\rput[tl](14.553134479838377,2.004206622043868){$P_{ges}$}
\end{scriptsize}
\end{pspicture*}}
\end{center}

\Subitem{Gib an, was in dieser Grafik positive bzw. negative Funktionswerte für $P_{ges}$ bedeuten!\vspace{0,3cm}

Positive Funktionswerte \rule{8cm}{0.3pt}\vspace{0,3cm}

Negative Funktionswerte \rule{8cm}{0.3pt}}

Familie Hell möchte den Amortisationszeitpunkt für die Photovoltaikanlage ermitteln. Das ist derjenige Zeitpunkt, ab dem die Errichtungskosten gleich hoch wie die Einsparungen durch den Betrieb der Anlage sind. Ab diesem Zeitpunkt arbeitet die Anlage rentabel.

\Subitem{Kann sich die Anlage für die Familie Hell auch amortisieren, wenn die finanzielle Tagesbilanz der Photovoltaikanlage für alle Tage im Jahr negativ ist? Begründe deine Antwort.}



\end{aufgabenstellung}

\begin{loesung}
\item \subsection{Lösungserwartung:} 

\Subitem{Graph von $P_{ges}$: siehe oben.} %Lösung von Unterpunkt1
\Subitem{Die eingefärbte Fläche oben stellt die gesamte über den Tagesverlauf erzeugte elektrische Energie dieser Photovoltaikanlage dar.} %%Lösung von Unterpunkt2

\setcounter{subitemcounter}{0}
\subsection{Lösungsschlüssel:}
 
\Subitem{Ein Punkt für den richtigen Graphen von $P_{ges}$.} %Lösungschlüssel von Unterpunkt1
\Subitem{Ein Punkt für die Markierung der gesamten erzeugten Energie.} %Lösungschlüssel von Unterpunkt2

\item \subsection{Lösungserwartung:} 

\Subitem{Positive Funktionswerte bedeuten, dass elektrische Energie ans Stromnetz geliefert wird.
	
	Negative Funktionswerte bedeuten, dass elektrische Energie aus dem Netz entnommen wird.} %Lösung von Unterpunkt1
\Subitem{Die Anlage kann sich amortisieren, wenn der Betrag, den man aufgrund der Anlage an Stromkosten eingespart hat, größer ist als der Anschaffungsbetrag der Anlage.} %%Lösung von Unterpunkt2

\setcounter{subitemcounter}{0}
\subsection{Lösungsschlüssel:}
 
\Subitem{Ein Punkt für die richtige Interpretation der Funktionswerte.} %Lösungschlüssel von Unterpunkt1
\Subitem{Ein Punkt für eine korrekte Begründung. Formulierungen, die sinngemäß dieser Aussage entsprechen, sind als richtig zu werten.} %Lösungschlüssel von Unterpunkt2

\end{loesung}

\end{langesbeispiel}