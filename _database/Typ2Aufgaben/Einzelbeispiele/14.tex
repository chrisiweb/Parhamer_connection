\section{14 - MAT - AG 2.2, FA 1.7, FA 5.2, WS 2.3, WS 3.1, WS 3.3 - Schwarzfahren als Volkssport - BIFIE Aufgabensammlung}

\begin{langesbeispiel} \item[0] %PUNKTE DES BEISPIELS
				Im Jahr 2010 wurden in den Graz-Linien exakt 36\,449 Schwarzfahrer und Schwarzfahrerinnen auf frischer Tat ertappt.  
				
\textit{"`Ihren Fahrschein, bitte!"' - diese freundliche, aber bestimmte Aufforderung treibt Schwarzfahrern regelmäßig den Angstschweiß ins Gesicht. Zu Recht, heißt es dann doch 65 Euro Strafe zahlen. Mehr als 800\,000 Fahrschein-
kontrollen wurden im Vorjahr in den Grazer Bus- und Straßenbahnlinien durchgeführt. 36\,449 Personen waren Schwarzfahrer. Gegenüber 2009 ist das ein leichtes Minus von 300 Beanstandungen. Für die Graz-Linien ist das ein Beweis für den Erfolg der strengen Kontrollen. Für den Vorstand der Graz-Linien steht darum eines fest: "`Wir werden im Interesse unserer zahlenden Fahrgäste 
auch 2011 die Kontrollen im gleichen Ausmaß fortsetzen."' Denn den Graz-Linien entgehen durch den Volkssport Schwarzfahren jedes Jahr Millionen. Rechnet m
an die Quote der bei den Kontrollen ertappten Schwarzfahrer (ca. 5\,\%) 
auf die Gesamtzahl der beförderten Personen hoch (ca. 100 Mio. pro Jahr), dann werden aus 36\,449 schnell fünf Millionen, die aufs Ticket pfeifen...} 

\begin{footnotesize}(Quelle: Meine Woche Graz, April 2011, adaptiert)\end{footnotesize}

In diesem Zeitungsartikel wird der Begriff Schwarzfahrer für Personen, die ohne gültigen Fahrschein angetroffen werden, verwendet. Fahrgäste, die ihre Zeitkarte (z. B. Wochenkarte, Schülerfreifahrtsausweis) nicht bei sich haben,
 gelten nicht als Schwarzfahrer/innen. 

Nach Angaben der Graz-Linien beträgt der Anteil der Schwarzfahrer/innen etwa 5\,\%. 

Zwei Kontrolleure steigen an der Haltestelle Jakominiplatz in einen Wagen der Straßenbahnlinie 5 und kontrollieren alle 25 Fahrgäste. An der Haltestelle 
Hauptplatz steigen sie in einen Wagen der Linie 3 um, in dem sie alle 18 Fahrgäste kontrollieren.
				
\subsection{Aufgabenstellung:}
\begin{enumerate}
	\item Es soll die Wahrscheinlichkeit $p_1$ berechnet werden, dass die Kontrolleure mindestens eine Schwarzfahrerin/einen Schwarzfahrer ermitteln, aber erst in der Linie 3 auf diese Person treffen. Gib einen geeigneten Term an, mit dem diese Wahrscheinlichkeit $p_1$ ermittelt werden kann, und berechne diese! 
	
Es sei $p_2$ die Wahrscheinlichkeit, bereits im Wagen der Linie 5 auf mindestens eine Schwarzfahrerin/einen Schwarzfahrer zu treffen. Begründe, warum $p_2$ größer als $p_1$ sein muss, ohne $p_2$ zu berechnen!

\item  Es wird angenommen, dass bei den durchgeführten Kontrollen nur 1\,\% aller fünf Millionen Personen, die keinen Fahrschein mithaben, entdeckt werden. Man weiß, dass 10\,\% dieser fünf Millionen Personen eine Zeitkarte besitzen, die sie aber nicht bei sich haben, und daher nicht als Schwarzfahrer/innen gelten. Wird eine Schwarzfahrerin/ein Schwarzfahrer erwischt, muss sie/er zusätzlich zum Fahrpreis von \EUR{2} noch \EUR{65} Strafe zahlen. Gehe 
davon aus, dass im Durchschnitt die nicht erwischten Schwarzfahrer/innen jeweils entgangene Einnahmen eines Einzelfahrscheins von \EUR{2} verursachen.
  
Berechne den in einem Jahr durch die Schwarzfahrer/innen entstandenen finanziellen Verlust für die Grazer Linien! 

Das Bußgeld müsste wesentlich erhöht werden, um eine Kostendeckung zu erreichen. Ermittle den neuen Betrag für ein kostendeckendes Bußgeld!

\item Die Anzahl der entdeckten Schwarzfahrer/innen nahm gegenüber 2009 um 300 ab und betrug 2010 nur mehr 36\,449. Man geht davon aus, dass durch verstärkte Kontrollen eine weitere Abnahme der Anzahl an Schwarzfahrerinnen/Schwarzfahrern erreicht werden kann. 

Beschreibe diese Abnahme beginnend mit dem Jahr 2009 sowohl als lineares als auch als exponentielles Modell!  

Gib jeweils einen Funktionsterm an, der die Anzahl $S$ der Schwarzfahrer/innen nach $t$ Jahren, ausgehend von dem Jahr 2009, beschreibt! 

Berechne die Anzahl der Schwarzfahrer/innen nach 10 Jahren, also im Jahr 2019, mit beiden Modellen! Welche Schlussfolgerungen über die beiden Modelle ziehst du aus dem Ergebnis?
	
				\end{enumerate}\leer
				
				
\antwort{\subsection{Lösungserwartung:}
\begin{enumerate}
	\item $p_1=0,95^{25}\cdot (1-0,95^{18})\approx 0,1672$
	
	Mögliche Argumentationen:
	
	\begin{itemize}
		\item Die Wahrscheinlichkeit $p_1$ ist höchstens die Wahrscheinlichkeit, unter 18 Personen mindestens 1 Schwarzfahrer/in zu finden. Die Wahrscheinlichkeit $p_2$, bereits im Wagen der Linie 5 auf mindestens 1 Schwarzfahrer/in zu treffen, ist größer als die Wahrscheinlichkeit $p_1$, da die Wahrscheinlichkeit, unter 25 Kontrollierten eher 1 Schwarzfahrer/in anzu-
treffen, größer ist als unter 18 Kontrollierten.

\item Die Wahrscheinlichkeit $p_1$ ist höchstens die Wahrscheinlichkeit, unter 25 Personen keine Schwarzfahrerin/keinen Schwarzfahrer zu finden. Diese 
ist kleiner als 0,5. Die Wahrscheinlichkeit $p_2$ ist die Wahrscheinlichkeit, unter 25 Personen mindestens 1 Schwarzfahrer/in zu treffen. $p_2$ ist größer als 0,5, also $p_2>p_1$.
	\end{itemize}

		\item Der zu erwartende Verlust wird wie folgt berechnet:
		
		10\,\% der Fahrgäste ohne Fahrschein besitzen eine Zeitkarte, daraus folgt, dass 90\,\% von den 99\,\% Schwarzfahrer/innen sind.
		
		$V=(-0,99\cdot 0,9\cdot 2+0,01\cdot 0,9\cdot 65)\cdot 5\,000\,000=$
		
		$=(-0,891\cdot 2+0,009\cdot 65)\cdot 5\,000\,000\approx -1,197\cdot 5\,000\,000\approx$ \EUR{-5.985-000}
		
		Soll der Verlust $V=0$ sein, dann gilt: $0=-0,891\cdot 2+0,009\cdot B$
		
		$\rightarrow B=$ \EUR{198}.
		
		Das Bußgeld $B$ müsste auf \EUR{198} erhöht werden.
		
		\item lineare Abnahme: $S(t)=36\,749-300\cdot t$
		
		exponentielle Abnahme: $S(t)=36\,749\cdot (36\,449/36\,749)^t$
		
		Bei linearer Abnahme sind es nach 10 Jahren noch 33\,749, bei exponentieller Abnahme 33\,857 Personen. Der Unterschied ist gering und beide Modelle sind für diesen Zeitraum gleich gut.
		\end{enumerate}}
\end{langesbeispiel}