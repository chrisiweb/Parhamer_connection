\section{79 - MAT - AN 1.3, AN 2.1, AN 4.3, FA 1.5, FA 1.6, FA 1.7 - Abkühlungsprozesse - BIFIE Aufgabensammlung}

\begin{langesbeispiel} \item[0] %PUNKTE DES BEISPIELS
	
Wird eine Tasse mit heißem Kaffe am Frühstückstisch abgestellt, kühlt der Kaffee anfangs rasch ab, bleibt aber relativ lange warm.\leer

Die Temperatur einer Flüssigkeit während des Abkühlens kann nach dem Newton'schen Abkühlungsgesetz durch eine Funktion der Form $t\mapsto T_U+(T_0-T_U)\cdot e^{-k\cdot t}$ beschrieben werden. Dabei gibt $T_0$ die Anfangstemperatur der Flüssigkeit (in $^\circ\text{C}$) zum Zeitpunkt $t=0$ an, $T_U$ ist die konstante Umgebungstemperatur (in $^\circ\text{C}$) und $k\in\mathbb{R}^+$ (in $s^{-1}$) ist eine von den Eigenschaften der Flüssigkeit und des Gefäßes abhängige Konstante.\leer

Ein zu untersuchender Abkühlungsprozess wird durch eine Funktion $T$ der obigen Form beschrieben. Dabei beträgt die Anfangstemperatur $T_0=90\,^\circ\text{C}$ und die Umgebungstemperatur $T_U=20\,^\circ\text{C}$. Die Abkühlungskonstante hat den Wert $k=0,002$. Die Zeit $t$ wird in Sekunden gemessen, die Temperatur $T(t)$ in $^\circ\text{C}$.

\subsection{Aufgabenstellung:}
\begin{enumerate}
	\item Berechne den Wert des Differenzenquotienten der Funktion $T$ im Intervall $[0\,\text{s};300\,\text{s}]$ und interpretiere den berechneten Wert im Hinblick auf den beschriebenen Abkühlungsprozess!\leer
	
	Beschreibe den Verlauf des Graphen von $T$ für große Werte von $t$ und interpretiere den Verlauf im gegebenen Kontext!\leer
	
	\item Der Wert $T'(t)$ kann als "`Abkühlungsgeschwindigkeit"' der Flüssigkeit zum Zeitpunkt $t$ gedeutet werden.\leer
	
	Gib für den zu untersuchenden Abkühlungsprozess eine Funktionsgleichung für $T'$ an!
	
	Gib weiters denjenigen Zeitpunkt an, zu dem der Betrag der Abkühlungsgeschwindigkeit am größten ist!\leer
	
	Der Graph von $T'$ und die $t-$Achse schließen im Intervall $[0\,\text{s}; 600\,\text{s}]$ eine Fläche von ca. 49 Flächeneinheiten ein.
	
	Interpretiere diesen Wert unter Verwendung der entsprechenden Einheit im gegebenen Kontext!\leer
	
	\item Eine zweite Flüssigkeit in einem anderen Gefäß hat zum Zeitpunkt $t=0$ eine Temperatur von $95\,^\circ\text{C}$. Nach einer Minute ist die Temperatur auf $83,4^\circ\text{C}$ gesunken, die Umgebungstemperatur beträgt $T_U=20\,^\circ\text{C}$. Die Funktion $T_2$ beschreibt den Abkühlungsprozess dieser Flüssigkeit.\leer
	
	Gib eine Gleichung an, mit der die Abkühlungskonstante $k_2$ für diesen Abkühlungsprozess berechnet werden kann, und ermittle diesen Wert!\leer
	
	Ermittle den Schnittpunkt der Graphen der Funktionen $T$ und $T_2$ und interpretiere die Koordinaten des Schnittpunkts im gegebenen Kontext!
	
\end{enumerate}

\antwort{
\begin{enumerate}
	\item \subsection{Lösungserwartung:} 

$\frac{T(300-T(0)}{300}\approx -0,1053$

In den ersten fünf Minuten kühlt die Flüssigkeit durchschnittlich um ca. $0,1\,^\circ\text{C}$ pro Sekunde ab.\leer

Der Graph von $T$ nähert sich im Laufe der Zeit der Umgebungstemperatur ($20\,^\circ\text{C}$ an.

\begin{center}
	\resizebox{0.9\linewidth}{!}{\newrgbcolor{qqwuqq}{0. 0.39215686274509803 0.}
\newrgbcolor{ffxfqq}{1. 0.4980392156862745 0.}
\psset{xunit=0.0083cm,yunit=0.1cm,algebraic=true,dimen=middle,dotstyle=o,dotsize=4pt 0,linewidth=0.8pt,arrowsize=3pt 2,arrowinset=0.25}
\begin{pspicture*}(-79.25907731814803,-6.597025544297572)(1909.2670297804618,96.91104311759474)
\psaxes[labelFontSize=\scriptstyle,xAxis=true,yAxis=true,Dx=100.,Dy=10.,ticksize=-2pt 0,subticks=2]{->}(0,0)(0.,0.)(1909.2670297804618,96.91104311759474)
\psplot[linewidth=1.2pt,linecolor=qqwuqq,plotpoints=200]{0}{1909.2670297804618}{20.0+70.0*2.718281828459045^(-0.002*x)}
\psplot[linewidth=1.2pt,linecolor=ffxfqq,plotpoints=200]{0}{1909.2670297804618}{20.0}
\rput[tl](519.8043432625328,52.35840502935908){T}
\rput[tl](1550,4.5){$t$ in Sekunden}
\rput[tl](50,93.0702984548158){$T(t)\,\text{in}\,^\circ C$}
\end{pspicture*}}
\end{center}

	\item \subsection{Lösungserwartung:}
	
	$T'(t)=-0,14\cdot e^{-0,002\cdot t}$\leer
	
	Der Betrag der Abkühlungsgeschwindigkeit ist zum Zeitpunkt $t=0$ am größten.\leer
	
	Die Flüssigkeit kühlt in den ersten zehn Minuten insgesamt um ca. $49\,^\circ\text{C}$ ab.

\item \subsection{Lösungserwartung:}
	
$T_2(t)=20+75\cdot e^{-k_2\cdot t} \Rightarrow$

$T_2(60)=20+75\cdot e^{-k_2\cdot 60}=83,4$

$k_2\approx 0,0028\,\text{s}^{-1}$

\begin{center}
	\resizebox{0.9\linewidth}{!}{\newrgbcolor{qqwuqq}{0. 0.39215686274509803 0.}
\newrgbcolor{ffxfqq}{1. 0.4980392156862745 0.}
\psset{xunit=0.0083cm,yunit=0.1cm,algebraic=true,dimen=middle,dotstyle=o,dotsize=4pt 0,linewidth=0.8pt,arrowsize=3pt 2,arrowinset=0.25}
\begin{pspicture*}(-79.25907731814803,-6.597025544297572)(1909.2670297804618,106.91104311759474)
\psaxes[labelFontSize=\scriptstyle,xAxis=true,yAxis=true,Dx=100.,Dy=10.,ticksize=-2pt 0,subticks=2]{->}(0,0)(0.,0.)(1909.2670297804618,106.91104311759474)
\psplot[linewidth=1.2pt,linecolor=qqwuqq,plotpoints=200]{0}{1909.2670297804618}{20.0+70.0*2.718281828459045^(-0.002*x)}
\rput[tl](519.8043432625328,52.35840502935908){T}
\rput[tl](1550,4.5){$t$ in Sekunden}
\rput[tl](50,93.0702984548158){$T(t)\,\text{in}\,^\circ C$}
\psplot[linewidth=1.2pt,linecolor=blue,plotpoints=200]{0}{1909.267029780462}{20.0+75.0*2.718281828459045^(-0.0028*x)}
\psline[linewidth=1.2pt](0.,78.91)(86.24108935868952,78.91013006154637)
\psline[linewidth=1.2pt](86.24108935868952,78.91013006154637)(86.24,0.)
\rput[tl](121.18799876968438,83.08436233159057){S}
\rput[tl](444.63668972959573,35){$T_2$}
\psdots[dotsize=3pt 0,dotstyle=*,linecolor=darkgray](86.24108935868952,78.91013006154637)
\end{pspicture*}}
\end{center}

Schnittpunkt: $S\approx (86,2|78,9)$

Nach ca. 86,2 Sekunden haben beide Flüssigkeiten eine Temperatur von ca. $78,9\,^\circ\text{C}$.

\end{enumerate}}
		\end{langesbeispiel}