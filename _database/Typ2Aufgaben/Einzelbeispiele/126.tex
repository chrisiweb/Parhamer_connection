\section{126 - K7 - WS 3.1 - Glücksrad - VerSie}

\begin{langesbeispiel} \item[8] %PUNKTE DES BEISPIELS
Auf einem Jahrmarkt werden nach dem Drehen eines Glücksrades 0\,\euro , 1\,\euro , 2\,\euro oder 4\,\euro ausbezahlt. Jedes Mal, bevor das Rad gedreht wird, ist eine Spielgebühr $e$ (in \euro) zu entrichten. Der Spielbetreiber hat für mathematisch interessierte den Graphen der Verteilungsfunktion $F$ mit $F(x)=P(X\leq x)$ angegeben. Die Zufallsvariable $X$ gibt dabei die Größe des auszahlenden Betrags an. Aus der Abbildung lassen sich die Wahrscheinlichkeiten für die einzelnen Auszahlungsbeträge ermitteln.

\begin{center}
\psset{xunit=2.2cm,yunit=8.0cm,algebraic=true,dimen=middle,dotstyle=o,dotsize=5pt 0,linewidth=1pt,arrowsize=3pt 2,arrowinset=0.25}
\begin{pspicture*}(-0.7969811320754708,-0.053512046148625196)(5.329811320754713,1.0895052595860168)
\psaxes[comma,labelFontSize=\scriptstyle,showorigin=false,xAxis=true,yAxis=true,Dx=1.,Dy=0.1,ticksize=-2pt 0,subticks=0]{->}(0,0)(0.,0.)(5.329811320754713,1.0895052595860168)[x,140] [y,-40]
\psline[linewidth=2.5pt](-3.,0.)(0.,0.)
\psline[linewidth=2.5pt](0.,0.2)(1.,0.2)
\psline[linewidth=2.5pt](1.,0.5)(2.,0.5)
\psline[linewidth=2.5pt](2.,0.9)(4.,0.9)
\psline[linewidth=2.5pt](4.,1.)(5.,1.)
\begin{scriptsize}
\psdots[dotsize=6pt 0,linecolor=darkgray](0.,0.)
\psdots[dotsize=6pt 0,dotstyle=*,linecolor=darkgray](0.,0.2)
\psdots[dotsize=6pt 0,linecolor=darkgray](1.,0.2)
\psdots[dotsize=6pt 0,dotstyle=*,linecolor=darkgray](1.,0.5)
\psdots[dotsize=6pt 0,linecolor=darkgray](2.,0.5)
\psdots[dotsize=6pt 0,dotstyle=*,linecolor=darkgray](2.,0.9)
\psdots[dotsize=6pt 0,linecolor=darkgray](4.,0.9)
\psdots[dotsize=6pt 0,dotstyle=*,linecolor=darkgray](4.,1.)
\end{scriptsize}
\end{pspicture*}
\end{center} %Aufgabentext

\begin{aufgabenstellung}
\item %Aufgabentext

\ASubitem{Ermittle aus der gegebenen Graphik die einzelnen Wahrscheinlichkeiten der Zufallsvariable $X$ und trage sie in der nachstehenden Tabelle an.\vspace{0,3cm}
	
	\begin{tabular}{|p{3.5cm}|C{2cm}|C{2cm}|C{2cm}|C{2cm}|}\hline
	Auszahlungsbetrag $x$ in \euro &0&1&2&4\\ \hline
	$P(X=x)$&\antwort{0,2}&\antwort{0,3}&\antwort{0,4}&\antwort{0,1}\\ \hline
	\end{tabular}} %Unterpunkt1
\Subitem{Fertige den Graphen der Wahrscheinlichkeitsverteilung an.} %Unterpunkt2

\item %Aufgabentext

\Subitem{Berechne den Erwartungswert der Zufallsvariable $X$.} %Unterpunkt1
\Subitem{Begründe, warum die Funktion F monoton steigend ist und warum das Maximum von F immer 1 sein muss.} %Unterpunkt2

\item %Aufgabentext

\Subitem{Der Erwartungswert von $X$ beträgt 1,50\,\euro. Gib die Bedeutung des Werts im gegebenen Kontext an.} %Unterpunkt1
\Subitem{Versetze dich in die Lage des Spielbetreibers. Wie groß wählst du den Betrag der Spielgebühr $e$ pro Drehung mindestens, wenn du die Größe des Erwartungswerts von $X$ kennst? Begründe deine Wahl} %Unterpunkt2

\item Die Zufallsvariable $Y$ gibt die Höhe des (tatsächlichen) Gewinns aus der Sicht des Spielers/der Spielerin an. Welche Werte $y$ wird der Gewinn in Abhängigkeit von $e$ bei den bekannten Auszahlungsbeträgen 0\,\euro , 1\,\euro , 2\,\euro und 4\,\euro annehmen?\leer
	
	\begin{tabular}{|p{3cm}|p{2cm}|p{2cm}|p{2cm}|p{2cm}|}\hline
	Gewinn $y$ in \euro &\antwort{$0-e$}&\antwort{$1-e$}&\antwort{$2-e$}&\antwort{$4-e$}\\ \hline
	$P(Y=y)$&\antwort{$0,2$}&\antwort{$0,3$}&\antwort{$0,4$}&\antwort{$0,1$}\\ \hline
	\end{tabular}%Aufgabentext

\Subitem{Vervollständige die Gewinne.} %Unterpunkt1
\Subitem{Vervollständige die Wahrscheinlichkeiten.} %Unterpunkt2

\end{aufgabenstellung}

\begin{loesung}
\item \subsection{Lösungserwartung:} 

\Subitem{Tabelle: siehe oben.} %Lösung von Unterpunkt1
\Subitem{\begin{center}
	\psset{xunit=1.7cm,yunit=8.0cm,algebraic=true,dimen=middle,dotstyle=o,dotsize=5pt 0,linewidth=1.6pt,arrowsize=3pt 2,arrowinset=0.25}
\begin{pspicture*}(-0.43584905660377304,-0.05565252799457031)(5.690943396226411,0.6699708177807918)
\psaxes[labelFontSize=\scriptstyle,labels=y,xAxis=true,yAxis=true,Ox=-1,Dx=1, Dy=0.1,ticksize=-2pt 0,subticks=0]{->}(0,0)(0.,0.)(5.690943396226411,0.6699708177807918)[x,140] [y,-40]
\begin{scriptsize}
\psline[linewidth=4.8pt](1.,0.)(1.,0.2)
\psline[linewidth=4.8pt](2.,0.)(2.,0.3)
\psline[linewidth=4.8pt](3.,0.)(3.,0.4)
\psline[linewidth=4.8pt](5.,0.)(5.,0.1)
\rput[tl](0.95,-0.02){0}
\rput[tl](1.95,-0.02){1}
\rput[tl](2.95,-0.02){2}
\rput[tl](3.95,-0.02){3}
\rput[tl](4.95,-0.02){4}
\end{scriptsize}
\end{pspicture*}
	\end{center}} %%Lösung von Unterpunkt2

\setcounter{subitemcounter}{0}
\subsection{Lösungsschlüssel:}
 
\Subitem{Ein Punkt für die richtige Berechnung der Einzelwahrscheinlichkeiten.} %Lösungschlüssel von Unterpunkt1
\Subitem{Ein Punkt für die den Graphen der Wahrscheinlichkeitsverteilung. Der Punkt ist auch zu geben, wenn die Wahrscheinlichkeiten oben falsch berechnet worden ist, mit den falschen Werten dann aber ein richtiger Graph konstruiert wurde.} %Lösungschlüssel von Unterpunkt2

\item \subsection{Lösungserwartung:} 

\Subitem{$E(X)=0\cdot 0,2+1\cdot 0,3+2\cdot 0,4+4\cdot 0,1=1,5$} %Lösung von Unterpunkt1
\Subitem{Die Verteilungsfunktion ist monoton steigend, da die Wahrscheinlichkeiten aufsummiert werden und es keine negativen Wahrscheinlichkeiten gibt.\\
	Das Maximum der Verteilungsfunktion ist 1, da die Gesamtwahrscheinlichkeit immer 100\,\% beträgt.} %%Lösung von Unterpunkt2

\setcounter{subitemcounter}{0}
\subsection{Lösungsschlüssel:}
 
\Subitem{Ein Punkt für die richtige Berechnung des Erwartungswerts.} %Lösungschlüssel von Unterpunkt1
\Subitem{Ein Punkt für die richtige Begründung.} %Lösungschlüssel von Unterpunkt2

\item \subsection{Lösungserwartung:} 

\Subitem{Im langfristigen Mittel macht der Spieler einen Gewinn von 1,5\,\euro pro Spiel.} %Lösung von Unterpunkt1
\Subitem{Aus Sicht des Spielbetreibers wäre alles über dem Erwartungswert sinnvoll.} %%Lösung von Unterpunkt2

\setcounter{subitemcounter}{0}
\subsection{Lösungsschlüssel:}
 
\Subitem{Ein Punkt für die richtige Interpretation.} %Lösungschlüssel von Unterpunkt1
\Subitem{Ein Punkt für eine richtige Angabe der Spielgebühr.} %Lösungschlüssel von Unterpunkt2

\item \subsection{Lösungserwartung:} 

\Subitem{Siehe Tabelle oben} %Lösung von Unterpunkt1
\Subitem{Siehe Tabelle oben} %%Lösung von Unterpunkt2

\setcounter{subitemcounter}{0}
\subsection{Lösungsschlüssel:}
 
\Subitem{Ein Punkt für eine richtige Angabe der Gewinne.} %Lösungschlüssel von Unterpunkt1
\Subitem{Ein Punkt für eine richtige Angabe der Wahrscheinlichkeiten.} %Lösungschlüssel von Unterpunkt2

\end{loesung}

\end{langesbeispiel}