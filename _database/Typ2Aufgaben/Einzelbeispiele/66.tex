\section{66 - MAT - FA 4.4, AN 3.3, AN 4.3 - Graphen von Polynomfunktionen dritten Grades - Matura 2015/16 2. Nebentermin}

\begin{langesbeispiel} \item[0] %PUNKTE DES BEISPIELS
	
Der Verlauf des Graphen einer Polynomfunktion $f$ dritten Grades mit der Funktionsgleichung $f(x)=a\cdot x^3+b\cdot x^2+c\cdot x+d$ mit $a,b,c,d\in\mathbb{R}, a\neq 0$ hängt von den Werten der Koeffizienten $a,b,c,d$ ab.

Je nach Wahl der Werte für die Koeffizienten ergibt sich unter anderem die Anzahl und Lage der Nullstellen, Extremstellen und Wendestellen von $f$. 

\subsection{Aufgabenstellung:}
\begin{enumerate}
	\item Wie viele lokale Extremstellen kann $f$ höchstens haben? Gib die Anzahl an und begründe deine Antwort mithilfe der Ableitungsfunktion von $f$! \leer
	
	\fbox{A} Zeige für den Spezialfall $a=1, b=-3, c=3, d=0$, dass die Funktion $f$ keine lokale Extremstelle hat.\leer
	
	\item Wenn für alle $x\in\mathbb{R}$ die Beziehung $f(-x)=-f(x)$ gilt, dann ist der Graph von $f$ symmetrisch bezüglich des Ursprungs.\leer
	
	Gib diejenigen Werte an, die die Koeffizienten $b$ und $d$ annehmen müssen, damit der Graph von $f$ mit der Gleichung $f(x)=a\cdot x^3+b\cdot x^2+c\cdot x+d$ symmetrisch bezüglich des Ursprungs ist.\leer
	
	Ermittle für eine solche Funktion $f$ den Wert des Integrals $\int^{x_1}_{-x_1}{f(x)}$d$x$ für ein beliebiges $x_1>0$ und gib eine Begründung für deine Lösung an!\leer
	
	\item Der Graph von $f$ hat auf jeden Fall einen Wendepunkt. Welcher der Koeffizienten $a, b, c, d$ ist ausschlaggebend dafür, dass der Wendepunkt der Funktion $f$ auf der senkrechten Koordinatenachse liegt? 
	
	Gib diesen Koeffizienten und die zugehörige Bedingung an!\leer
	
	Geben Sie eine zusätzliche Bedingung dafür an, dass der Graph von $f$ im Wendepunkt  $W=(0|f(0))$ eine zur $x$-Achse parallele Tangente hat!	
\end{enumerate}
\antwort{
\begin{enumerate}
	\item \subsection{Lösungserwartung:} 

Mögliche Begründung:

Nur an denjenigen Stellen, an denen $f'(x)=0$ ist, können lokale Extremstellen von $f$ liegen. Die Ableitungsfunktion $f'$ ist eine Polynomfunktion zweiten Grades. Da die quadratische Gleichung $f'(x)=0$ maximal zwei Lösungen hat, kann die Funktion $f$ höchstens zwei Extremstellen haben.\leer

Mögliche Vorgehensweise:

Die 1. Ableitung $f'(x)=3x^2-6x+3$ hat genau eine Nullstelle bei $x=1$ und hat sowohl links als auch rechts von der Nullstelle positive Werte. Damit ist die Funktion $f$ auf ihrem gesamten Definitionsbereich streng monoton wachsend und hat keine Extremstelle.
	\subsection{Lösungsschlüssel:}
	\begin{itemize}
		\item Ein Punkt für eine (sinngemäß) richtige Begründung.
		\item Ein Ausgleichspunkt für einen korrekten Nachweis. Andere korrekte Nachweise sind ebenfalls als richtig zu werten.
	\end{itemize}
	
	\item \subsection{Lösungserwartung:}
			
	$b=0$
	
	$d=0$
	
	$$\int^{x_1}_{-x_1}{f(x)}dx=0$$\leer
	
	Mögliche Begründung:
	
	Wegen der Symmetrie des Graphen von $f$ bezüglich des Ursprungs begrenzt der Graph von $f$ mit der $x$-Achse in den Intervallen $[-x_1; 0]$ und $[0; x_1]$ zwei gleich große Flächenstücke, von denen eines oberhalb und eines unterhalb der $x$-Achse liegt.
	
	\subsection{Lösungsschlüssel:}
	
\begin{itemize}
	\item Ein Punkt für die Angabe der beiden korrekten Werte.
	\item Ein Punkt für die richtige Lösung und eine korrekte Begründung.  
\end{itemize}

	\item \subsection{Lösungserwartung:}
			
	$f''(x)=0$
	
	$6a\cdot x+2b=0$
	
	$x=-\frac{b}{3a}$
	
	$x=0 \Rightarrow b=0$\leer
	
	$f'(x)=3a\cdot x^2+c$
	
	$f'(0)=0 \Rightarrow c=0$
	
	\subsection{Lösungsschlüssel:}
	
\begin{itemize}
	\item Ein Punkt für die richtige Lösung.
	\item Ein Punkt für die richtige Lösung, wobei die Aufgabe auch bei korrektem Ansatz als richtig gelöst zu werten ist.  
\end{itemize}

\end{enumerate}}
		\end{langesbeispiel}