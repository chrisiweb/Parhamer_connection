\section{125 - K7 - AN 1.1, AN 1.2, AN 1.3, AN 3.3, FA 2.2, FA 4.1 - Druckbehälter - VerSie}

\begin{langesbeispiel} \item[8] %PUNKTE DES BEISPIELS
Im Zuge eines 15 Minuten dauernden Experiments wird der Druck $P$ in einem Behälter gezielt verändert. Die untenstehende Abbildung zeigt die Entwicklung des Drucks während der Versuchszeit. Die für die Lösung der anschließenden Aufgabenstellungen abzulesenden Werte sind ganzzahlig. Der Druck wird im Weiteren in Bar angegeben, die Zeit $t$ in Minuten. Die Zeitmessung startet am Beginn des Experiments. 

\begin{center}
\psset{xunit=0.7cm,yunit=0.5cm,algebraic=true,dimen=middle,dotstyle=o,dotsize=5pt 0,linewidth=1.6pt,arrowsize=3pt 2,arrowinset=0.25}
\begin{pspicture*}(-1.0211764705882354,-0.8421052631578938)(15.682352941176479,20.982456140350855)
\multips(0,0)(0,1.0){21}{\psline[linestyle=dashed,linecap=1,dash=1.5pt 1.5pt,linewidth=0.4pt,linecolor=gray]{c-c}(0,0)(15.682352941176479,0)}
\multips(0,0)(1.0,0){19}{\psline[linestyle=dashed,linecap=1,dash=1.5pt 1.5pt,linewidth=0.4pt,linecolor=gray]{c-c}(0,0)(0,20.982456140350855)}
\psaxes[labelFontSize=\scriptstyle,xAxis=true,yAxis=true,Dx=2.,Dy=2.,ticksize=-2pt 0,subticks=2]{->}(0,0)(0.,0.)(15.682352941176479,20.982456140350855)[Zeit in Minuten,140] [Druck in Bar,-40]
\psplot[linewidth=2.pt,plotpoints=200]{0}{10}{-0.25*x^(2.0)+4.0*x+4.0}
\psplot[linewidth=2.pt]{10.}{15.682352941176479}{(--116.-4.*x)/4.}
\antwort{\psplot[linewidth=2.pt]{0}{15.682352941176479}{(--20.-0.*x)/1.}}
\end{pspicture*}
\end{center}%Aufgabentext

\begin{aufgabenstellung}
\item %Aufgabentext

\Subitem{Skizziere in der nebenstehenden Abbildung die momentane Druckänderung zum Zeitpunkt $t=8$ Minuten.} %Unterpunkt1
\Subitem{Interpretiere das Ergebnis im gegebenen Kontext.} %Unterpunkt2

\item %Aufgabentext

\Subitem{Welche der folgenden Informationen über die Entwicklung des Drucks während des Experiments sind korrekt? Kreuze die zutreffende(n) Aussage(n) an.
	
	\multiplechoice[5]{  %Anzahl der Antwortmoeglichkeiten, Standard: 5
				L1={Am Beginn beträgt der Druck 4 bar.},   %1. Antwortmoeglichkeit 
				L2={Innerhalb der ersten 10 Minuten beschreibt eine Funktion $f$ zweiten Grades mit $P=f(t)$ die Höhe des Drucks $P$ zur Zeit $t$.},   %2. Antwortmoeglichkeit
				L3={Ab dem Zeitpunkt $t=8$ verändert sich der Druck nicht mehr.},   %3. Antwortmoeglichkeit
				L4={Die mittlere Änderungsrate des Drucks innerhalb der ersten 4 Minuten beträgt 3 bar.},   %4. Antwortmoeglichkeit
				L5={Die absolute Änderung des Drucks im Zeitintervall $[4;6]$ beträgt 3\,bar.},	 %5. Antwortmoeglichkeit
				L6={},	 %6. Antwortmoeglichkeit
				L7={},	 %7. Antwortmoeglichkeit
				L8={},	 %8. Antwortmoeglichkeit
				L9={},	 %9. Antwortmoeglichkeit
				%% LOESUNG: %%
				A1=1,  % 1. Antwort
				A2=2,	 % 2. Antwort
				A3=4,  % 3. Antwort
				A4=5,  % 4. Antwort
				A5=0,  % 5. Antwort
				}} %Unterpunkt1
\Subitem{Verändere eine der falschen Antwortmöglichkeiten so, dass es eine wahre Aussage wird.} %Unterpunkt2

\item Für das Zeitintervall $[0;10]$ kann der Druckverlauf mittels der Funktion $P=f(t)=-\frac{1}{4}t^2+4t+4$ modelliert werden. Ab dem Zeitpunkt $t=10$ verändert sich der Druck mit der zu diesem Zeitpunkt gerade gegebenen momentanen Änderungsrate gleichmäßig, linear weiter und es gilt:\\ 
	$P=g(t)$.%Aufgabentext

\ASubitem{Ermittle $g(t)$.} %Unterpunkt1
\Subitem{Berechne, nach wie vielen Minuten der Druck auf 0 bar gesunken wäre, wenn sich der Druck ab dem Zeitpunkt $t=10$ gemäß der Funktion $g$ weiter entwickeln würde.} %Unterpunkt2

\item Nach Beendigung des Experiments stellt eine Person fest, dass innerhalb der ersten 10 Minuten die Zunahmegeschwindigkeit des Drucks zu einem bestimmten Zeitpunkt gleich der Abnahmegeschwindigkeit des Drucks zu einem anderen Zeitpunkt war.%Aufgabentext

\Subitem{Erläutere, um welche Zeitpunkte es sich handeln kann.} %Unterpunkt1
\Subitem{Begründe deine Antwort mittels Rechnung, wenn für das Zeitintervall $[0;10]$ gilt:\\ 
	$P=f(t)=-\frac{1}{4}t^2+4t+4$} %Unterpunkt2

\end{aufgabenstellung}

\begin{loesung}
\item \subsection{Lösungserwartung:} 

\Subitem{Momentane Druckänderung siehe Abbildung} %Lösung von Unterpunkt1
\Subitem{Zum Zeitpunkt 8 beträgt die momentane Druckänderung 0, da der Druck an dieser Stelle maximal ist.} %%Lösung von Unterpunkt2

\setcounter{subitemcounter}{0}
\subsection{Lösungsschlüssel:}
 
\Subitem{Ein Punkt für das richtige Einzeichnen der momentanen Druckänderung.} %Lösungschlüssel von Unterpunkt1
\Subitem{Ein Punkt für die richtige Interpretation.} %Lösungschlüssel von Unterpunkt2

\item \subsection{Lösungserwartung:} 

\Subitem{Multiple Choice siehe oben} %Lösung von Unterpunkt1
\Subitem{mögliche Antworten:\\	
	Ab dem Zeitpunkt $t=8$ sinkt der Druck wieder ab.
	Zum Zeitpunkt $t=8$ ist die momentane Änderungsrate des Drucks gleich Null.} %%Lösung von Unterpunkt2

\setcounter{subitemcounter}{0}
\subsection{Lösungsschlüssel:}
 
\Subitem{Ein Punkt für die richtigen MC-Antworten.} %Lösungschlüssel von Unterpunkt1
\Subitem{Ein Punkt für eine korrekte Richtigstellung.} %Lösungschlüssel von Unterpunkt2

\item \subsection{Lösungserwartung:} 

\Subitem{$f'(t)=-\frac{1}{2}t+4$
	
	$f'(10)=-\frac{1}{2}\cdot 10+4=-1$
	
	$f(10)=-\frac{1}{4}\cdot 10^2+4\cdot 10+4=19 \Rightarrow (10\mid 19)$
	
	$y=-x+d \Rightarrow 19=-10+d \Rightarrow d=29$
	
	$g(t)=-x+29$} %Lösung von Unterpunkt1
\Subitem{$g(t)=0 \Rightarrow 0=-x+29 \Rightarrow x=29$
	
	Nach 29 Minuten wäre der Druck auf 0\,bar gesunken.} %%Lösung von Unterpunkt2

\setcounter{subitemcounter}{0}
\subsection{Lösungsschlüssel:}
 
\Subitem{Ein Punkt für die korrekte Funktionsgleichung von $g(t)$.} %Lösungschlüssel von Unterpunkt1
\Subitem{Ein Punkt für die korrekte Berechnung der Minuten.} %Lösungschlüssel von Unterpunkt2

\item \subsection{Lösungserwartung:} 

\Subitem{Für alle Werte zwischen $[6;8)$ findet man Werte im Intervall $(8;10]$ die die gleiche Steigung mit umgekehrten Vorzeichen haben.} %Lösung von Unterpunkt1
\Subitem{zum Beispiel:\\
	$f'(t)=-\frac{1}{2}t+4$
	
	$f(6)=-\frac{1}{2}\cdot 6+4=-3+4=1$ bzw. $f(10)=-\frac{1}{2}\cdot 10+4=-5+4=-1$} %%Lösung von Unterpunkt2

\setcounter{subitemcounter}{0}
\subsection{Lösungsschlüssel:}
 
\Subitem{Ein Punkt für die richtige Angabe der möglichen Werte.} %Lösungschlüssel von Unterpunkt1
\Subitem{Ein Punkt für eine (äquivalente) Rechnung die die Behauptung bestätigt.} %Lösungschlüssel von Unterpunkt2

\end{loesung}

\end{langesbeispiel}