\section{06 - MAT - WS 2.2, WS 3.1, WS 3.2 - Ärztliche Untersuchung an einer Schule - BIFIE Aufgabensammlung}

\begin{langesbeispiel} \item[0] %PUNKTE DES BEISPIELS
Eine Schulärztin hat die Untersuchung der Oberstufenschüler/innen einer Schule abgeschlossen. Die Auswertung der Ergebnisse ist in den nachstehenden Abbildungen grafisch dargestellt.
				\leer
				\begin{center}
				\begin{tabular}{|c|c|} \hline
				Mädchen&Burschen\\ \hline
				\multicolumn{2}{|c|}{Körpergröße}\\
				\multicolumn{1}{|c}{\resizebox{0.3\linewidth}{!}{\kreisdiagramm\begin{tikzpicture}
				\tikzstyle{every node}=[font=\footnotesize]
				\pie[color={black!10 ,black!20 , black!30, black!40}, %Farbe
				text=inside %Format: inside,pin, legend
				]
				{22/unter 150\,cm , 14/über 175\,cm , 64/150-175\,cm} %Werte
				\end{tikzpicture}}}&\multicolumn{1}{c|}{\resizebox{0.3\linewidth}{!}{\kreisdiagramm\begin{tikzpicture}
				\tikzstyle{every node}=[font=\footnotesize]
				\pie[color={black!10 ,black!20 , black!30, black!40}, %Farbe
				text=inside %Format: inside,pin, legend
				]
				{7/unter 150\,cm , 34/über 175\,cm , 59/150-175\,cm} %Werte
				\end{tikzpicture}}}\\ \hline
				\multicolumn{2}{|c|}{Gewicht}\\
				\multicolumn{1}{|c}{\resizebox{0.3\linewidth}{!}{\kreisdiagramm\begin{tikzpicture}
				\tikzstyle{every node}=[font=\footnotesize]
				\pie[color={black!10 ,black!20 , black!30, black!40}, %Farbe
				text=inside %Format: inside,pin, legend
				]
				{28/Untergewicht , 11/Übergewicht, 61/Normalgewicht} %Werte
				\end{tikzpicture}}}&\multicolumn{1}{c|}{\resizebox{0.3\linewidth}{!}{\kreisdiagramm\begin{tikzpicture}
				\tikzstyle{every node}=[font=\footnotesize]
				\pie[color={black!10 ,black!20 , black!30, black!40}, %Farbe
				text=inside %Format: inside,pin, legend
				]
				{13/Untergewicht, 21/Übergewicht, 66/Normalgewicht} %Werte
				\end{tikzpicture}}}\\ \hline
				\multicolumn{2}{|c|}{Rauchverhalten}\\
				\multicolumn{1}{|c}{\resizebox{0.3\linewidth}{!}{\kreisdiagramm\begin{tikzpicture}
				\tikzstyle{every node}=[font=\footnotesize]
				\pie[color={black!10 ,black!20 , black!30, black!40}, %Farbe
				text=inside %Format: inside,pin, legend
				]
				{42/Raucherinnen, 58/Nichtraucherinnen} %Werte
				\end{tikzpicture}}}&\multicolumn{1}{c|}{\resizebox{0.3\linewidth}{!}{\kreisdiagramm\begin{tikzpicture}
				\tikzstyle{every node}=[font=\footnotesize]
				\pie[color={black!10 ,black!20 , black!30, black!40}, %Farbe
				text=inside %Format: inside,pin, legend
				]
				{37/Raucher, 63/Nichtraucher} %Werte
				\end{tikzpicture}}}\\ \hline
				\end{tabular}
				\end{center}
				
Von den einzelnen Klassen wurde dabei folgende Schüleranzahl erfasst:
				
				\begin{center}
				\begin{tabular}{|c||c|c|c|c|c|c|c|c|}\hline
				Klasse&5A&5B&6A&6B&7A&7B&8A&8B\\ \hline \hline
				weiblich&18&22&20&16&16&15&12&18\\ \hline
				männlich&14&9&7&9&10&11&13&8\\ \hline \hline
				gesamt&32&31&27&25&26&26&25&26\\ \hline
				\end{tabular}
				\end{center}%Aufgabentext

\begin{aufgabenstellung}
\item %Aufgabentext

\Subitem{Gib auf Basis der erhobenen Daten einen Term zur Berechnung der Wahrscheinlichkeit für folgendes Ereignis an, wobei angenommen wird, dass die angegebenen Prozentsätze unabhängig von der Schulstufe sind: 
\begin{center}
"`In den 5. Klassen gibt es höchstens 2 Raucherinnen."'
\end{center}} %Unterpunkt1

In dieser Schule wurden die Schüler/innen nicht nach Klassen geordnet untersucht, sondern jede/r entschied selbst, wann sie/er zur Schulärztin ging (zufällige Reihenfolge angenommen).

\Subitem{Wie viele Schüler/innen musste die Schulärztin untersuchen, um mit absoluter Sicherheit mindestens eine Raucherin/einen Raucher aus den 5. Klassen zu finden, wenn sie weiß, dass es in den 5. Klassen mindestens eine/n davon gibt?} %Unterpunkt2

\item Auf Basis der oben angeführten Daten wurde für die Burschen eines Jahrgangs der folgende statistische Kennwert ermittelt:
\begin{center}
$\mu=n\cdot p=23\cdot 0,34=7,82$
\end{center} %Aufgabentext

\Subitem{Was drückt dieser Kennwert aus? Interpretiere diesen Kennwert im gegebenen 
Zusammenhang und nutzen Sie dabei sowohl die grafische Abbildung der Untersuchungsergebnisse als auch die tabellarische Übersicht über die Klassenschülerzahlen.} %Unterpunkt1
\Subitem{Ist der so errechnete Kennwert aussagekräftig? Begründe deine Antwort.} %Unterpunkt2

\end{aufgabenstellung}

\begin{loesung}
\item \subsection{Lösungserwartung:} 

\Subitem{$X$ ... Anzahl der Raucherinnen aus allen 5. Klassen
	
	$X$ ... Binomialverteilung mit $n=40,p=0,42$
	
	$P(X\leq2)=P(X=0)+P(X=1)+P(X=2)=$
	
	$=\binom{40}{0}\cdot 0,42^0\cdot 0,58^{40}+\binom{40}{1}\cdot 0,42^1\cdot 0,58^{39}+\binom{40}{2}\cdot 0,42^2\cdot 0,58^{38}$} %Lösung von Unterpunkt1
\Subitem{Da in der Angabe nur statistische Aussagen gemacht werden, aufgrund derer nicht mit Sicherheit behauptet werden kann, ob es mehr als eine Raucherin/einen Raucher in den 5. Klassen gibt, müssen alle Schüler/innen untersucht werden, um mit Sicherheit eine Raucherin/einen Raucher in der 5. Klasse zu finden.} %%Lösung von Unterpunkt2

\setcounter{subitemcounter}{0}
\subsection{Lösungsschlüssel:}
 
\Subitem{Ein Punkt für einen korrekten Term.} %Lösungschlüssel von Unterpunkt1
\Subitem{Ein Punkt für die richtige Angabe der zu untersuchenden Schüler/innen.} %Lösungschlüssel von Unterpunkt2

\item \subsection{Lösungserwartung:} 

\Subitem{In allen 5. Klassen zusammen gibt es 23 Burschen. Der Prozentsatz der Burschen mit einer Körpergröße von über 175 cm beträgt 34\,\%.
	
Somit drückt der berechnete Wert $\mu=n\cdot p=23\cdot 0,34=7,82$ die Anzahl der zu erwartenden Burschen in den 5. Klassen mit einer Körpergröße von über 175 cm aus.} %Lösung von Unterpunkt1
\Subitem{Da die Daten für die gesamte Oberstufe ausgewertet sind und Schüler/innen während der Oberstufe noch wachsen, werden voraussichtlich weniger so große Schüler in den 5. Klassen zu finden sein. Somit ist der errechnete Erwartungswert nicht aussagekräftig bzw. sinnvoll.} %%Lösung von Unterpunkt2

\setcounter{subitemcounter}{0}
\subsection{Lösungsschlüssel:}
 
\Subitem{Ein Punkt für die korrekte Deutung des Kennwerts.

Wesentlich für die Richtigkeit der Antwort sind:
\subitem $\circ$ sinngemäße Formulierung für "`Erwartungswert"'
\subitem $\circ$ Burschen aus 5. Klassen mit über 175 cm Körpergröße 

Eine Rundung auf 8 ist in diesem Zusammenhang als falsch zu werten, da
 es sich bei $\mu$ nur um einen statistischen Kennwert und nicht um einen realen Ausgang eines Zufallsexperiments handelt.} %Lösungschlüssel von Unterpunkt1
\Subitem{Ein Punkt für eine korrekte Begründung.

Wesentlich für die Richtigkeit der Antwort ist die sinngemäße Darstellung einer der folgenden Interpretationen:
\subitem $\circ$ Die unabhängige Wiederholung (im Sinne des Bernoulli-Experiments) ist nicht gegeben.
\subitem $\circ$ Die verwendete Wahrscheinlichkeit (über 175 cm Körpergröße) bzw. relative Häufigkeit ist auf die beobachtete Eigenschaft (Bursch aus 5. Klassen) nicht anzuwenden.

Die Burschen/Jugendlichen wachsen im Laufe der Oberstufe, daher ist die relative Häufigkeit auf diese Gruppe nur eingeschränkt übertragbar.} %Lösungschlüssel von Unterpunkt2

\end{loesung}

\end{langesbeispiel}