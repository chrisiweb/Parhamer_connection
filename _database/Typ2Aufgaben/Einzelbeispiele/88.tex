\section{88 - MAT - AG 2.1, AG 4.1, AN 1.3, AN 2.1, AN 3.3 - Ansteigende Straße - Matura 2016/17 2. NT}

\begin{langesbeispiel} \item[0] %PUNKTE DES BEISPIELS
Ein Auto legt auf einem ansteigenden, kurvenfreien Straßenabschnitt in einem bestimmten Zeitintervall den Weg zwischen den Punkten $A$ und $B$ zurück. Der Höhenverlauf dieses Straßenabschnitts zwischen $A$ und $B$ bezogen auf das Niveau des Punktes $A$ wird durch den Graphen einer
Polynomfunktion $h$ in Abhängigkeit von $x$ modelliert. Dabei ist $x$ die waagrechte Entfernung des (punktförmig modellierten) Autos vom Ausgangspunkt $A$ und $h(x)$ die jeweilige Höhe der Position des Autos über dem Niveau des Punktes $A$ ($h(x)$ in m, $x$ in m). In diesem Modell haben die Punkte
$A$ und $B$ die Koordinaten $A = (0|0)$ und $B = (60|10)$.

\begin{center}
\resizebox{0.9\linewidth}{!}{
\newrgbcolor{qqwuqq}{0. 0.39215686274509803 0.}
\psset{xunit=0.2cm,yunit=0.3cm,algebraic=true,dimen=middle,dotstyle=o,dotsize=5pt 0,linewidth=0.8pt,arrowsize=3pt 2,arrowinset=0.25}
\begin{pspicture*}(-7.9696919352131665,-2.474219379779093)(69.08039190510043,12.795706326755795)
\multips(0,0)(0,10.0){2}{\psline[linestyle=dashed,linecap=1,dash=1.5pt 1.5pt,linewidth=0.4pt,linecolor=lightgray]{c-c}(-8,0)(69.08039190510043,0)}
\multips(-5,0)(5.0,0){16}{\psline[linestyle=dashed,linecap=1,dash=1.5pt 1.5pt,linewidth=0.4pt,linecolor=lightgray]{c-c}(0,0)(0,12.795706326755795)}
\psaxes[labelFontSize=\scriptstyle,xAxis=true,yAxis=true,Dx=5.,Dy=10.,ticksize=-2pt 0,subticks=2]{->}(0,0)(-7.9696919352131665,-2.474219379779093)(69.08039190510043,12.795706326755795)[x in m,140] [h(x) in m,-40]
\psplot[linewidth=.8pt,plotpoints=200]{0}{60}{1.0/64800.0*(-x^(3.0)+120.0*x^(2.0)+7200.0*x)}
\rput[tl](22.149886293273056,5.3){$h$}
\begin{scriptsize}
\psdots[dotsize=4pt 0,dotstyle=*,linecolor=black](0.,0.)
\rput[bl](0.57586281798525,0.5377384430695321){$A$}
\psdots[dotsize=4pt 0,dotstyle=*,linecolor=black](60.,10.)
\rput[bl](60.53483715190201,10.554249342310307){$B$}
\end{scriptsize}
\end{pspicture*}}	
\end{center}

Eine Gleichung der Funktion $h$ lautet:

\[h(x)=\frac{1}{64\,800}\cdot(-x^3+120\cdot x^2+7\,200\cdot x) \qquad \text{für } x\in [0; 60] \]

\subsection{Aufgabenstellung:}

\begin{enumerate}
	\item Gib den Wert des Differenzenquotienten der Funktion $h$ im Intervall $[0; 60]$ an und interpretiere diesen Wert im gegebenen Kontext! \leer
	
Eine Person behauptet: "`Wenn ein (beliebiger) ansteigender Straßenabschnitt durch eine Polynomfunktion dritten Grades modelliert werden kann, deren Wendestelle im betreffenden
Abschnitt liegt, so handelt es sich bei dieser Wendestelle um diejenige Stelle, an der die Straße am steilsten verläuft."'\leer

Gib an, ob diese Behauptung mit Sicherheit zutrifft, und begründe deine Entscheidung!

\item Ein Neubau der Straße ist geplant, wobei die Straße zwischen $A$ und $B$ nach dem Neubau
eine konstante Steigung aufweisen soll.\leer

 \fbox{A} Bestimme eine Gleichung der Funktion $h_1$, die den Verlauf der neuen Straße zwischen $A$ und $B$ beschreibt, wobei $h_1(x)$ wieder die Höhe (in m) der Position des Autos über
dem Niveau des Punktes $A$ ist!\leer

Berechne denjenigen Winkel $\alpha$, unter dem die neu gebaute Straße (bezüglich der Horizontalen) ansteigt!

\item Bei einer Fahrt ins Gebirge entsteht ein unangenehmer Druck auf das Trommelfell, den viele Menschen als ein "`Verschlagen"' der Ohren beschreiben. Man kann annehmen, dass bei einer Person im Auto dieses unangenehme Druckgefühl auftritt, wenn die momentane Änderungsrate der Höhe einen Wert von 4\,m/s überschreitet. \leer

Die Funktion $g$ mit $g(t)=\frac{1}{5}\cdot t^2+ t$ modelliert die Position des Autos über dem Niveau von $A$ während der Fahrt von $A = (0|0)$ nach $B = (60|10)$ in Abhängigkeit von der Zeit. Dabei beschreibt $g(t)$, in welcher Höhe sich das Auto zum Zeitpunkt $t$ befindet ($g(t)$ in Metern; $t$ in Sekunden gemessen ab dem Zeitpunkt, in dem sich das Auto im Punkt $A$ befindet).\leer

Berechne, wie viele Sekunden die Fahrt von $A$ nach $B$ dauert!
Gib an, ob die momentane Änderungsrate der Höhe während dieser Zeitspanne einen Wert von 4\,m/s überschreitet, und begründe deine Entscheidung!
\end{enumerate}

\antwort{\begin{enumerate}
	\item \subsubsection{Lösungserwartung:}
	
	$\frac{h(60)-h(0)}{60-0}=\frac{10-0}{60}=\frac{1}{6}=0,1\dot{6}\approx 0,17$
	
	\subsubsection{Mögliche Interpretation:}
	
	Die Straße steigt von $A$ nach $B$ pro Meter in waagrechter Richtung im Mittel um ca. 17\,cm in senkrechter Richtung an.\leer
	
Die Behauptung trifft nicht zu.

\subsubsection{Mögliche Begründung:}
Die Wendestelle einer Funktion 3. Grades kann auch derjenigen Stelle entsprechen, an der der Anstieg minimal ist.

\begin{center}
\resizebox{0.7\linewidth}{!}{
\psset{xunit=1.0cm,yunit=1.0cm,algebraic=true,dimen=middle,dotstyle=o,dotsize=5pt 0,linewidth=0.8pt,arrowsize=3pt 2,arrowinset=0.25}
\begin{pspicture*}(-0.5597505478996176,-0.34457176544976925)(6.389289121119028,3.3)
\psaxes[labelFontSize=\scriptstyle,xAxis=true,yAxis=true,labels=none,Dx=1.,Dy=1.,ticksize=0pt 0,subticks=2]{->}(0,0)(-0.5597505478996176,-0.34457176544976925)(6.389289121119028,3.0036018932592126)[x,140] [\text{f(x),g(x)},-40]
\psplot[linewidth=1.2pt,plotpoints=200, linecolor=red]{0.3}{1.8}{COS(1.5*x-PI)+1.0}
\psplot[linewidth=1.2pt,linestyle=dashed,dash=3pt 3pt,plotpoints=200, linecolor=red]{-0.5597505478996176}{2.5}{COS(1.5*x-PI)+1.0}
\psplot[linewidth=1.2pt,plotpoints=200, linecolor=red]{3.1}{4.5}{(x-4.0)^(3.0)+0.2*x+0.2}
\psplot[linewidth=1.2pt,linestyle=dashed,dash=3pt 3pt,plotpoints=200, linecolor=red]{-0.5597505478996176}{6.389289121119028}{(x-4.0)^(3.0)+0.2*x+0.2}
\begin{scriptsize}
\rput[tl](1.4996921176459084,2.232890220877145){$f$}
\rput[tl](4.380384925893638,1.7401401352558232){$g$}
\psdots[dotsize=4pt 0,dotstyle=*](1.05,1.0042036608246887)
\rput[bl](1.1332882078249251,0.8304476694933828){$W_f$}
\psdots[dotsize=4pt 0,dotstyle=*](3.9759794581635988,0.9951820321060361)
\rput[bl](3.988711780912587,0.6){$W_g$}
\end{scriptsize}
\end{pspicture*}}
\end{center}
\color[rgb]{1,0,0}
$W_f$\ldots Wendepunkt der Funktion $f$, in dem die Steigung der Tangente Maximal ist

$W_g$\ldots Wendepunkt der Funktion $g$, in dem die Steigung der Tangente minimal ist.

oder:

$f(x)=x^3$\ldots Die Funktion $f$ hat eine Wendestelle, an der die Steigung der Tangente minimal ist.

\subsubsection{Lösungsschlüssel:}

\begin{itemize}
	\item[-]  Ein Punkt für die richtige Lösung und eine korrekte Interpretation. Andere Schreibweisen der
Lösung sind ebenfalls als richtig zu werten.

Toleranzintervall: $[0,16; 0,17]$ bzw. $[16\,\%; 17\,\%]$.

\item[-] Ein Punkt für die richtige Entscheidung und eine korrekte Begründung. Andere korrekte
Begründungen sind ebenfalls als richtig zu werten.
\end{itemize}

\item \subsubsection{Lösungserwartung:}

$h_1(x)=\frac{1}{6}\cdot x$

$\tan(\alpha)=\frac{1}{6} \Rightarrow \alpha \approx 9,5^\circ$

\subsubsection{Lösungsschlüssel:}

\begin{itemize}
	\item[-] Ein Ausgleichspunkt für eine korrekte Gleichung. Äquivalente Gleichungen sind als richtig zu
werten.
Toleranzintervall für den Wert der Steigung der linearen Funktion $h_1: [0,16; 0,17]$
\item[-] Ein Punkt für die richtige Lösung, wobei die Einheit "`Grad"' nicht angeführt sein muss.
Toleranzintervall: $[9^\circ; 10^\circ]$

Eine korrekte Angabe der Lösung in einer anderen Einheit ist ebenfalls als richtig zu werten.
\end{itemize}

\item \subsubsection{Lösungserwartung:}

$g(t)=10 \Leftrightarrow t^2+5\cdot t-50=0 \Rightarrow t_1=5, (t_2=-10)$

Die Fahrt von $A$ nach $B$ dauert 5 Sekunden.\leer

Die maximale momentane Änderungsrate der Höhe im Zeitintervall $[0\,\text{s}; 5\,\text{s}]$ beträgt 3\,m/s,
also überschreitet die momentane Änderungsrate der Höhe während dieser Zeitspanne einen
Wert von 4\,m/s nicht.

Mögliche Begründung:

$g'(t)=0,4\cdot t +1$

Die momentane Änderungsrate der Höhe ist im Intervall $[0\,\text{s};5\,\text{s}]$ streng monoton steigend, also liegt ihr maximaler Wert an der Stelle $t=5$.

$g'(5)=3$

\subsubsection{Lösungsschlüssel:}

\begin{itemize}
	\item[-] Ein Punkt für die richtige Lösung, wobei die Einheit "`s"' nicht angeführt sein muss.
	\item[-] Ein Punkt für die richtige Entscheidung und eine korrekte Begründung. Andere korrekte Begründungen sind ebenfalls als richtig zu werten.
\end{itemize}

\end{enumerate}}				
\end{langesbeispiel}