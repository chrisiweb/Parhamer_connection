\section{74 - MAT - AG 2.1, AN 2.1, AN 3.3, FA 1.2, FA 1.5, FA 1.7 - Aufnahme einer Substanz ins Blut - BIFIE Aufgabensammlung}

\begin{langesbeispiel} \item[0] %PUNKTE DES BEISPIELS
	
Wenn bei einer medizinischen Behandlung eine Substanz verabreicht wird, kann die Konzentration der Substanz im Blut (kurz: Blutkonzentration) in Abhängigkeit von der Zeit $t$ in manchen Fällen durch eine sogenannte Bateman-Funktion $c(t)=d\cdot(e^{-a\cdot t}-e^{-b\cdot t})$ mit den personenbezogenen Parametern $a,b,d>0, a<b$ modelliert werden. Die Zeit $t$ wird in Stunden gemessen, $t=0$ entspricht dem Zeitpunkt der Verabreichung der Substanz.\leer

Die Bioverfügbarkeit $f$ gibt den Anteil der verabreichten Substanz an, der unverändert in den Blutkreislauf gelangt. Bei einer intravenösen Verabreichung (d.h. einer direkten Verabreichung in eine Vene) beträgt der Wert der Bioverfügbarkeit 1.\leer

Das Verteilungsvolumen $V$ beschreibt, in welchem Ausmaß sich die Substanz aus dem Blut in das Gewebe verteilt.\leer

Der Parameter $d$ ist direkt proportional zur verabreichten Dosis $D$ und zur Bioverfügbarkeit $f$, außerdem ist $d$ indirekt proportional zum Verteilungsvolumen $V$.\leer

Die Nachstehende Abbildung zeigt exemplarisch den zeitlichen Verlauf der Blutkonzentration in Nanogramm pro Milliliter (ng/ml) für den Fall der Einnahme einer bestimmten Dosis der Substanz Lysergsäurediethylamid und kann mit der Bateman-Funktion $c_1$ mit den Parametern $d=19,5, a=0,4$ und $b=1,3$ beschrieben werden.\leer

Der Graph der Bateman-Funktion weist für große Zeiten $t$ einen asymptotischen Verlauf gegen die Zeitachse auf.

\begin{center}
\resizebox{0.5\linewidth}{!}{\psset{xunit=1.0cm,yunit=0.5cm,algebraic=true,dimen=middle,dotstyle=o,dotsize=4pt 0,linewidth=0.8pt,arrowsize=3pt 2,arrowinset=0.25}
\begin{pspicture*}(-0.5097864597270574,-1.5)(8.993877976986889,9.416)
\multips(0,0)(0,2.0){6}{\psline[linestyle=dashed,linecap=1,dash=1.5pt 1.5pt,linewidth=0.4pt,linecolor=lightgray]{c-c}(0,0)(8.993877976986889,0)}
\multips(0,0)(1.0,0){10}{\psline[linestyle=dashed,linecap=1,dash=1.5pt 1.5pt,linewidth=0.4pt,linecolor=lightgray]{c-c}(0,0)(0,9.416)}
\psaxes[labelFontSize=\scriptstyle,xAxis=true,yAxis=true,Dx=1.,Dy=2.,ticksize=-2pt 0,subticks=2]{->}(0,0)(0.,0.)(8.993877976986889,9.416)
\psplot[linewidth=1.2pt,plotpoints=200]{0}{8.993877976986889}{19.5*(2.718281828459045^(-0.4*x)-2.718281828459045^(-1.3*x))}
\begin{scriptsize}
\rput[tl](3.7937597002943524,4.928){$c_1$}
\rput[tl](0.1585672999732374,8.888){Blutkonzentration $c_1(t)$ (in ng/ml)}
\rput[tl](7,-1){Zeit $t$ (in h)}
\end{scriptsize}
\end{pspicture*}}
\end{center}

\subsection{Aufgabenstellung:}
\begin{enumerate}
	\item Gib eine Gleichung an, mit der der Zeitpunkt der maximalen Blutkonzentration für die in der Einleitung beschriebene Bateman-Funktion $c_1$ berechnet werden kann, und ermittle diesen Zeitpunkt!\leer
	
	Begründe allgemein, warum der Wert des Parameters $d$ in der Bateman-Funktion $c$ nur die Größe der maximalen Blutkonzentration beeinflusst, aber nicht den Zeitpunkt, zu dem diese erreicht wird!\leer
	
	\item Die Werte der Parameter $a,b$ und $d$ der Bateman-Funktion variieren von Patient zu Patient. Es wird im Folgenden angenommen, dass der Wert des Parameters $d$ für drei untersuchte Patienten $P_1, P_2, P_3$ identisch ist.\leer
	
	Für den Patienten $P_1$ gelten die Parameter aus der Einleitung. Bei Patient $P_2$ ist der Wert des Parameters $a$ etwas größer als bei Patient $P_1$.
	
	Beschreibe, wie sich der Graph der Bateman-Funktion verändert, wenn der Wert des Parameters $a$ erhöht wird, der Parameter $b$ unverändert bleibt und $a<b$ gilt!
	
	Interpretiere diese Veränderung im gegebenen Kontext.\leer
	
	Patient $P_3$ erreicht (bei gleicher verabreichter Dosis) die maximale Blutkonzentration zeitgleich mit Patient $P_1$, die maximale Blutkonzentration von Patient $P_3$ ist aber größer.
	
	Ermittle, wie sich die Werte von $a$ und $b$ bei der Bateman-Funktion für Patient $P_3$ von jenen von Patient $P_1$ unterscheiden!\leer
	
	\item Kreuze diejenige Formel an, die den Zusammenhang zwischen dem Parameter $d$ der Bateman-Funktion und den in der Einleitung beschriebenen Größen $V,D$ und $f$ korrekt beschreibt! Der Parameter $\lambda$ ist dabei ein allgemeiner Proportionalitätsfaktor.
	
	\multiplechoice[6]{  %Anzahl der Antwortmoeglichkeiten, Standard: 5
					L1={$d=\lambda\cdot\frac{D}{V\cdot f}$},   %1. Antwortmoeglichkeit 
					L2={$d=\lambda\cdot\frac{D\cdot V}{f}$},   %2. Antwortmoeglichkeit
					L3={$d=\lambda\cdot\frac{V\cdot f}{D}$},   %3. Antwortmoeglichkeit
					L4={$d=\lambda\cdot\frac{D\cdot f}{V}$},   %4. Antwortmoeglichkeit
					L5={$d=\lambda\cdot\frac{V}{D\cdot f}$},	 %5. Antwortmoeglichkeit
					L6={$d=\lambda\cdot\frac{f}{V\cdot D}$},	 %6. Antwortmoeglichkeit
					L7={},	 %7. Antwortmoeglichkeit
					L8={},	 %8. Antwortmoeglichkeit
					L9={},	 %9. Antwortmoeglichkeit
					%% LOESUNG: %%
					A1=4,  % 1. Antwort
					A2=0,	 % 2. Antwort
					A3=0,  % 3. Antwort
					A4=0,  % 4. Antwort
					A5=0,  % 5. Antwort
					}
					
Bei einem konstanten Wert des Parameters $d$ und der Bioverfügbarkeit $f$ kann man die verabreichte Dosis $D(V)$ als Funktion $D$ in Abhängigkeit vom Verteilungsvolumen $V$ auffassen. Beziehe dich auf die von dir angekreuzte Formel und gib für die Parameterwerte der in der Einleitung dargestellten Bateman-Funktion und für den Fall einer intrevenösen Verabreichung die Funktionsgleichung $D(V)$ an! Gib weiters an, um welchen Funktionstyp es sich bei $D$ handelt!
\end{enumerate}

\antwort{
\begin{enumerate}
	\item \subsection{Lösungserwartung:} 

$c_1(t)=19,5\cdot(e^{-0,4\cdot t}-e^{-1,3\cdot t})$

$c_1'(t)=19,5\cdot (-0,4\cdot e^{-0,4\cdot t}+1,3\cdot e^{-1,3\cdot t})=0$

$t\approx 1,31$\, Stunden

$c_1''(1,31)\approx -4,15<0$\leer

Mögliche Begründungen:

Für die Berechnung des Zeitpunkts der (lokalen) maximalen Blutkonzentration muss die Gleichung $c'(t)=0$ nach $t$ gelöst werden. Der Parameter $d$ fällt bei dieser Berechnung weg und beeinflusst somit nur die Höhe der maximalen Blutkonzentration zum ermittelten Zeitpunkt.

oder:

$c'(t)=d\cdot (-a\cdot e^{-a\cdot t}+b\cdot e^{-b\cdot t})=0 \Rightarrow t=\frac{\ln(a)-\ln(b)}{a-b} \Rightarrow$ Der Parameter $d$ tritt in dieser Formel nicht auf. Der Zeitpunkt der maximalen Blutkonzentration $t$ ist somit von $d$ unabhängig.
	
	\item \subsection{Lösungserwartung:}
	
	Bei einer Erhöhung des Wertes von $a$ verschiebt sich das lokale Maximum der Funktion bei einem niedrigeren Funktionswert "`nach link"'. Das bedeutet, dass die maximale Blutkonzentration früher erreicht wird und geringer ist.\leer
	
	Der Patient $P_3$ ist (bei der Bateman-Funktion) der Wert von $a$ kleiner und der Wert von $b$ größer als bei (der Bateman-Funktion von) Patient $P_1$.

\item \subsection{Lösungserwartung:}
	
	Multiple Choice - siehe oben.\leer
	
	Die Funktionsgleichung lautet $D(V)=\frac{19,5}{\lambda}\cdot V$.
	
	Es handelt sich um eine lineare Funktion.
\end{enumerate}}
		\end{langesbeispiel}