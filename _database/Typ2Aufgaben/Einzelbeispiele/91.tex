\section{91 - MAT - WS 1.1, WS 1.3, WS 3.3, WS 3.2, FA 2.1, FA 2.2 - Abstandsmessung - Matura 2017/18}

\begin{langesbeispiel} \item[6] %PUNKTE DES BEISPIELS
			Im Rahmen der polizeilichen Kontrollmaßnahmen des öffentlichen Verkehrs werden Abstandsmessungen vorgenommen. Im Folgenden beschreibt der Begriff Abstand eine Streckenlänge und der Begriff Tiefenabstand eine Zeitspanne.
			
Beträgt der Abstand zwischen dem hinteren Ende des voranfahrenden Fahrzeugs und dem vorderen Ende des nachfahrenden Fahrzeugs $\Delta\,s$ Meter, so versteht man unter dem Tiefenabstand diejenige Zeit $t$ in Sekunden, in der das nachfahrende Fahrzeug die Strecke der Länge $\Delta\,s$ zurücklegt.

Nachstehend sind Tiefenabstände, die im Rahmen einer Schwerpunktkontrolle von 1 000 Fahrzeugen ermittelt wurden, in einem Kastenschaubild (Boxplot) dargestellt. Alle kontrollierten Fahrzeuge waren mit einer Geschwindigkeit von ca. 130 km/h unterwegs.

\begin{center}
	\resizebox{1\linewidth}{!}{\psset{xunit=3cm,yunit=3cm,algebraic=true,dimen=middle,dotstyle=o,dotsize=5pt 0,linewidth=1.6pt,arrowsize=3pt 2,arrowinset=0.25}
\begin{pspicture*}(-0.04234286164968088,-0.23)(3.1328993760761024,0.5244278240922793)
\multips(0,0)(0.1,0){32}{\psline[linestyle=dashed,linecap=1,dash=1.5pt 1.5pt,linewidth=0.4pt,linecolor=darkgray]{c-c}(0,0)(0,0.5244278240922793)}
\psaxes[labelFontSize=\scriptstyle,xAxis=true,yAxis=false,Dx=1.,Dy=0.2,ticksize=-2pt 0,subticks=2]{->}(0,0)(0.,0.)(3.1328993760761024,0.5244278240922793)
\psframe[linewidth=0.8pt,fillcolor=black,fillstyle=solid,opacity=0.2](0.9,0.19999999999999998)(2.2,0.4)
\psline[linewidth=0.8pt,fillcolor=black,fillstyle=solid,opacity=0.2](0.3,0.2)(0.3,0.4)
\psline[linewidth=0.8pt,fillcolor=black,fillstyle=solid,opacity=0.2](2.7,0.2)(2.7,0.4)
\psline[linewidth=0.8pt,fillcolor=black,fillstyle=solid,opacity=0.2](1.2,0.2)(1.2,0.4)
\psline[linewidth=0.8pt,fillcolor=black,fillstyle=solid,opacity=0.2](0.3,0.3)(0.9,0.3)
\psline[linewidth=0.8pt,fillcolor=black,fillstyle=solid,opacity=0.2](2.2,0.3)(2.7,0.3)
\begin{tiny}
\rput[tl](2,-0.18){Tiefenabstand in Sekunden}
\end{tiny}
\end{pspicture*}}
\end{center}

\subsection{Aufgabenstellung:}
\begin{enumerate}
	\item \fbox{A} Gib das erste Quartil $q_1$ und das dritte Quartil $q_3$ der Tiefenabstände an und deute den Bereich von $q_1$ bis $q_3$ im gegebenen Kontext!
	
	Nach den Erfahrungswerten eines österreichischen Autofahrerclubs halten ungefähr drei Viertel der Kraftfahrer/innen bei einer mittleren Fahrgeschwindigkeit von ca. 130\,km/h einen Abstand von mindestens 30 Metern zum voranfahrenden Fahrzeug ein. Gib an, ob die im Kastenschaubild dargestellten Daten in etwa diese Erfahrungswerte bestätigen oder nicht und begründe deine Entscheidung!
	
	\item Einer üblichen Faustregel zufolge wird auf Autobahnen generell ein Tiefenabstand von mindestens zwei Sekunden empfohlen. Jemand behauptet, dass aus dem dargestellten Kastenschaubild ablesbar ist, dass mindestens 20\,\% der Kraftfahrer/innen diesen Tiefenabstand eingehalten haben. Gib einen größeren Prozentsatz an, der aus dem Kastenschaubild mit Sicherheit abgelesen werden kann, und begründe deine Wahl!
	
Nimm den von dir ermittelten Prozentsatz als Wahrscheinlichkeit an, dass der empfohlene Tiefenabstand eingehalten wird. Gib an, wie hoch die Wahrscheinlichkeit ist, dass bei zehn zufällig und unabhängig voneinander ausgewählten Messungen dieser Schwerpunktkontrolle zumindest sechs Mal der empfohlene Tiefenabstand von mindestens zwei Sekunden eingehalten wurde!

\item Bei einer anderen Abstandsmessung wird ein kontrolliertes Fahrzeug auf den letzten 300 Metern vor der Messung zusätzlich gefilmt, damit die Messung nicht verfälscht wird, wenn sich ein anderes Fahrzeug vor das kontrollierte Fahrzeug drängt.

Fahrzeug A fährt während des Messvorgangs mit konstanter Geschwindigkeit und benötigt für die gefilmten 300 Meter eine Zeit von neun Sekunden. Stelle den zurückgelegten Weg $s_A(t)$ in Abhängigkeit von der Zeit $t$ im unten stehenden Zeit-Weg-Diagramm dar ($s_A(t)$ in
Metern, $t$ in Sekunden) und gib an, mit welcher Geschwindigkeit in km/h das Fahrzeug unterwegs ist!

Ein Fahrzeug B legt die 300 Meter ebenfalls in neun Sekunden zurück, verringert dabei aber kontinuierlich seine Geschwindigkeit. Skizziere ausgehend vom Ursprung einen möglichen Graphen der entsprechenden Zeit-Weg-Funktion $s_B$ in das unten stehende Zeit-Weg-Diagramm!\leer

\begin{center}
	\resizebox{0.8\linewidth}{!}{\psset{xunit=1.0cm,yunit=0.02cm,algebraic=true,dimen=middle,dotstyle=o,dotsize=5pt 0,linewidth=1.6pt,arrowsize=3pt 2,arrowinset=0.25}
\begin{pspicture*}(-0.84,-20.851999999999645)(9.5,324.3644444444353)
\multips(0,0)(0,50.0){7}{\psline[linestyle=dashed,linecap=1,dash=1.5pt 1.5pt,linewidth=0.4pt,linecolor=darkgray]{c-c}(0,0)(10.02,0)}
\multips(0,0)(1,0){11}{\psline[linestyle=dashed,linecap=1,dash=1.5pt 1.5pt,linewidth=0.4pt,linecolor=darkgray]{c-c}(0,0)(0,324.3644444444353)}
\psaxes[labelFontSize=\scriptstyle,xAxis=true,yAxis=true,Dx=1.,Dy=50.,ticksize=-2pt 0,subticks=2]{->}(0,0)(0.,0.)(9.5,324.3644444444353)
\rput[tl](0.26,311.2354074073986){$s_A(t),s_B(t)$ in $m$}
\rput[tl](8.3,20){$t$ in $s$}
\antwort{\psplot[linewidth=2.pt]{0.}{9}{(-0.--300.*x)/9.}
\psplot[linewidth=2.pt,plotpoints=200]{0}{9}{-3.0555555555555554*x^(2.0)+60.833333333333336*x}
\rput[tl](6.3,285.74962962962155){$s_B$}
\rput[tl](7.38,238.63955555554878){$s_A$}}
\end{pspicture*}}
\end{center}

	\end{enumerate}
	
	\antwort{
\begin{enumerate}
	\item \subsection{Lösungserwartung:} 

$q_1=0,9$\\
$q_3=2,1$\\
Etwa die Hälfte der kontrollierten Fahrzeuge halten einen Tiefenabstand von mindestens 0,9 Sekunden und höchsten 2,1 Sekunden ein.\leer

Die im Kastenschaubild dargestellten Daten bestätigen etwa diese Erfahrungswerte.

Mögliche Begründung:\\
$130$\,km/h$=36,\dot{1}$\,m/s\\
$36,\dot{1}$\,m/s$\cdot 0,9$\,s$=32,5$\,m $\Rightarrow$ Mindestens drei Viertel der Kraftfahrer/innen halten einen Abstand von 30\,m und mehr ein.

\item \subsection{Lösungserwartung:} 

ein möglicher Prozentsatz: 25\,\%\\
Toleranzintervall: $(20\,\%;25\,\%]$

Mögliche Begründung:\\
Der Tiefenabstand von zwei Sekunden liegt zwischen dem Median und dem dritten Quartil.\leer

Mögliche Vorgehensweise:\\
Zufallsvariable $X=$ Anzahl der Kraftfahrlenker/innen, die den empfohlenen Mindestabstand eingehalten haben

$p=0,25$ ... Wahrscheinlichkeit, dass der empfohlene Mindestabstand eingehalten wurden\\
$n=10$ ... Anzahl der ausgewählten Messungen\\
$P(X\geq 6)\approx 0,0197$

\item \subsection{Lösungserwartung:}

Fahrzeug A fährt mit einer Geschwindigkeit von 120\,km/h.

Grafik: siehe oben!
\end{enumerate}}
	
	\end{langesbeispiel}