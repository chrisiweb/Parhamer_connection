\section{117 - MAT - AN 1.1, AN 1.3, AN 2.1, AN 3.3, FA 5.2 - Einsatz von Antibiotika - Matura 2018/19 2. NT}

\begin{langesbeispiel} \item[6] %PUNKTE DES BEISPIELS
Die Entwicklung einer Bakterienpopulation kann durch die Zufuhr von Antibiotika beeinflusst werden, was letztlich durch die Giftwirkung von Antibiotika zum Aussterben der Bakterienpopulation führen soll.

In bestimmten Fällen kann diese Entwicklung näherungsweise durch eine Funktion $B$: $\mathbb{R}^+_0\rightarrow\mathbb{R}$ beschrieben werden:

$B(t)=b\cdot e^{k\cdot t-\frac{c}{2}\cdot t^2}$ mit $b,c,k\in\mathbb{R}^+$\vspace{0,3cm}

$t\,\ldots$ Zeit in Stunden\\
$B(t)\,\ldots$ Anzahl der Bakterien in Millionen zum Zeitpunkt $t$\\
$b\,\ldots$ Anzahl der Bakterien in Millionen zum Zeitpunkt $t=0$\\
$k\,\ldots$ Konstante\\
$c\,\ldots$ Parameter für die Giftwirkung%Aufgabentext

\begin{aufgabenstellung}
\item Die Funktion $B$ hat genau eine positive Extremstelle $t_1$.%Aufgabentext

\Subitem{Bestimme $t_1$ in Abhängigkeit von $k$ und $c$.} %Unterpunkt1
\Subitem{Gib an, welche Auswirkungen eine Vergrößerung von $c$ bei gegebenem $k$ auf die Lage der Extremstelle $t_1$ der Funktion $B$ hat.} %Unterpunkt2

\item Die Funktion $B_1$: $\mathbb{R}^+_0\Rightarrow\mathbb{R}$ mit $B_1(t)=20\cdot e^{2\cdot t-0,45\cdot t^2}$ beschreibt die Anzahl der Bakterien einer bestimmten Bakterienpopulation in Millionen in Abhängigkeit von der Zeit $t$.
	
	Zum Zeitpunkt $t_2\neq 0$ erreicht die Bakterienpopulation ihre ursprüngliche Anzahl von 20 Millionen.%Aufgabentext

\Subitem{Gib $t_2$ an.} %Unterpunkt1
\Subitem{Deute $B_1'(t_2)$ im vorliegenden Kontext unter Verwendung der entsprechenden Einheit.} %Unterpunkt2

\item Die Funktion $B_2$: $\mathbb{R}^+_0\rightarrow\mathbb{R}$ mit $B_2(t)=5\cdot e^{4\cdot t-\frac{t^2}{2}}$ beschreibt die Anzahl der Bakterien einer anderen Bakterienpopulation in Millionen, die zum Zeitpunkt $t=4$ ihr Maximum aufweist.%Aufgabentext

\Subitem{Bestimme denjenigen Zeitpunkt $t_3$, zu dem die stärkste Abnahme der Bakterienpopulation stattfindet.} %Unterpunkt1
\Subitem{Gib an, wie viel Prozent der maximalen Anzahl an Bakterien zum Zeitpunkt $t_3$ noch vorhanden sind.} %Unterpunkt2

\end{aufgabenstellung}

\begin{loesung}
\item \subsection{Lösungserwartung:} 

\Subitem{mögliche Vorgehensweise:
	
	$B'(t)=b\cdot(k-c\cdot t)\cdot e^{k\cdot t-\frac{c}{2}\cdot t^2}$\\
	$B'(t_1)=0$\\
	$k-c\cdot t_1=0$\\
	$t_1=\frac{k}{c}$} %Lösung von Unterpunkt1
\Subitem{mögliche Beschreibung:
	
	Die Extremstelle $t_1$ wird zu einem früheren Zeitpunkt erreicht.} %%Lösung von Unterpunkt2

\setcounter{subitemcounter}{0}
\subsection{Lösungsschlüssel:}
 
\Subitem{Ein Punkt für die richtige Lösung. Andere Schreibweisen der Lösung sind ebenfalls als richtig zu werten.} %Lösungschlüssel von Unterpunkt1
\Subitem{Ein Punkt für eine richtige Beschreibung.} %Lösungschlüssel von Unterpunkt2

\item \subsection{Lösungserwartung:} 

\Subitem{mögliche Vorgehensweise:
	
	$20=20\cdot e^{2\cdot t-0,45\cdot t^2}$\\
	$1=e^{2\cdot t-0,45\cdot t^2}$\\
	$0=2\cdot t-0,45\cdot t^2 \Rightarrow t_2=4,\dot{4}$\,h} %Lösung von Unterpunkt1
\Subitem{mögliche Deutung:
	
	$B'_1(t_2)$ gibt die (momentane) Abnahmegeschwindigkeit in Bakterien pro Stunde zum Zeitpunkt $t_2$ an.} %%Lösung von Unterpunkt2

\setcounter{subitemcounter}{0}
\subsection{Lösungsschlüssel:}
 
\Subitem{Ein Ausgleichspunkt für die richtige Lösung, wobei die Einheit "`h"' nicht angeführt sein muss.
		
		Toleranzintervall: $[4,4\,\text{h}; 4,5\,\text{h}]$\\		
		Die Aufgabe ist auch dann als richtig gelöst zu werten, wenn bei korrektem Ansatz das Ergebnis aufgrund eines Rechenfehlers nicht richtig ist.} %Lösungschlüssel von Unterpunkt1
\Subitem{Ein Punkt für eine richtige Deutung.} %Lösungschlüssel von Unterpunkt2

\item \subsection{Lösungserwartung:} 

\Subitem{mögliche Vorgehensweise:
	
	$B''_2(t)=5\cdot(t^2-8\cdot t+15)\cdot e^{4\cdot t-\frac{t^2}{2}}$
	
	$t^2-8\cdot t+15=0 \Rightarrow t_1=3; t_2=5$
	
	Es gilt:
	
	$B'_2(3)>0$ und $B'_2(5)<0$ (und $B'''_2(5)\neq 0$)
	
	Zum Zeitpunkt $t_3=5$ findet die stärkste Abnahme der Bakterienpopulation statt.} %Lösung von Unterpunkt1
\Subitem{$\dfrac{B_2(5)}{B_2(4)}=0,60653\ldots\approx 0,6065$
	
	Zum Zeitpunkt $t_3=5$ sind noch ca. $60,65\,\%$ der maximalen Anzahl an Bakterien vorhanden.} %%Lösung von Unterpunkt2

\setcounter{subitemcounter}{0}
\subsection{Lösungsschlüssel:}
 
\Subitem{Ein Punkt für die richtige Lösung, wobei die Einheit "`h"' nicht angeführt sein muss. Die Aufgabe ist auch dann als richtig gelöst zu werten, wenn bei korrektem Ansatz das Ergebnis aufgrund eines Rechenfehlers nicht richtig ist.} %Lösungschlüssel von Unterpunkt1
\Subitem{Ein Punkt für die richtige Lösung. Andere Schreibweisen der Lösung sind ebenfalls als
richtig zu werten.

Toleranzintervall: $[0,60;0,61]$} %Lösungschlüssel von Unterpunkt2

\end{loesung}

\end{langesbeispiel}