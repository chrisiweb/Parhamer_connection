\section{129 - K6 - FA 5.1, FA 5.2, FA 5.5 - Altersbestimmung - VerSie}

\begin{langesbeispiel} \item[4] %PUNKTE DES BEISPIELS
Mit der $^{14}$C-Methode, auch Radiokohlenstoffdatierung genannt, können Archäologen das Alter von Funden bestimmen. Sie beruht auf dem Zerfall eines bestimmten Kohlenstoffisotops, das in den oberen Schichten der Atmosphäre entsteht und später von allen Lebewesen auf der Erde aufgenommen wird.

Das Kohlenstoffisotop $^{14}$C zerfällt mit einer Halbwertszeit von ca. 5\,730 Jahren. Mit seiner Hilfe lässt sich das Alter von Fossilien bestimmen.%Aufgabentext

\begin{aufgabenstellung}
\item %Aufgabentext

\Subitem{Was versteht man unter dem Begriff Halbwertszeit? Erkläre in eigenen Worten!} %Unterpunkt1
\ASubitem{Stelle das Zerfallsgesetz in der Form $N(t)=N_0\cdot e^{\lambda\cdot t}$ für das Kohlenstoffisotop $^{14}$C auf.} %Unterpunkt2

\item %Aufgabentext

\Subitem{In einem Fossil wurde ein $^{14}$C-Gehalt von 7,5\,\% des ursprünglichen Menge gemessen. Berechne das Alter des Fossils (runde auf 1\,000 Jahre).} %Unterpunkt1
\Subitem{Bis zu welchem Alter lässt sich die $^{14}$C-Methode anwenden, wenn man noch 0,1\,\% des ursprünglichen $^{14}$C-Gehalts mit hinreichender Genauigkeit messen kann?} %Unterpunkt2

\end{aufgabenstellung}

\begin{loesung}
\item \subsection{Lösungserwartung:} 

\Subitem{Halbwertszeit ist die Zeitspanne, nach der eine mit der Zeit abnehmende Größe die Hälfte des anfänglichen Werts (oder, in Medizin und Pharmakologie, die Hälfte des Höchstwertes) erreicht.} %Lösung von Unterpunkt1
\Subitem{$0,5\cdot N_0=N_0\cdot e^{\lambda\cdot 5\,370}$
	
	$\ln(0,5)=\lambda\cdot 5\,730\cdot\ln(e) \Rightarrow \lambda=\frac{\ln(0,5)}{5\,730}\approx (-0.0001209680943386)$
	
	$N(t)=N_0\cdot e^{-0.0001209680943386\cdot t}$} %%Lösung von Unterpunkt2

\setcounter{subitemcounter}{0}
\subsection{Lösungsschlüssel:}
 
\Subitem{Ein Punkt für die richtige Erklärung des Begriffs "`Halbwertszeit"'.} %Lösungschlüssel von Unterpunkt1
\Subitem{Ein Punkt für das richtige Zerfallsgesetz.} %Lösungschlüssel von Unterpunkt2

\item \subsection{Lösungserwartung:} 

\Subitem{$0,075\cdot N_0=N_0\cdot e^{-0,0001209680943386\cdot t}$
	
	$\ln(0,075)=-0,0001209680943386\cdot t \Rightarrow t\approx 21412.8$
	
	Nach 21\,413 Jahren ist der gemessene $^{14}$C-Gehalt auf 7,5\,\% gesunken.} %Lösung von Unterpunkt1
\Subitem{$0,001\cdot N_0=N_0\cdot e^{-0,0001209680943386\cdot t}$
	
	$\ln(0,001)=-0,0001209680943386\cdot t \Rightarrow t\approx 57103,9$
	
	Man kann mit der Methode auf ungefähr 57\,104 Jahre genau messen.} %%Lösung von Unterpunkt2

\setcounter{subitemcounter}{0}
\subsection{Lösungsschlüssel:}
 
\Subitem{Ein Punkt für die richtige Berechnung des Alters.} %Lösungschlüssel von Unterpunkt1
\Subitem{Ein Punkt für die richtige Berechnung der Jahren.} %Lösungschlüssel von Unterpunkt2


\end{loesung}

\end{langesbeispiel}