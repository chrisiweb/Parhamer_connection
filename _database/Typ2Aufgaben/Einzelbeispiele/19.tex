\section{19 - MAT - AN 3.3, AN 1.5, AN 1.3 - Grippeepidemie - BIFIE Aufgabensammlung}

\begin{langesbeispiel} \item[0] %PUNKTE DES BEISPIELS
				Betrachtet man den Verlauf einer Grippewelle in einer Stadt mit 5\,000 Einwohnern, so lässt sich die Anzahl an Erkrankten $E$ in Abhängigkeit von der Zeit $t$ (in Tagen) annähernd durch eine Polynomfunktion 3. Grades mit der Gleichung $E(t)=at³+bt²+ct+d$ beschreiben.
				
				Folgende Informationen liegen vor:
				
				
				\begin{itemize}
					\item Zu Beginn der Beobachtung sind 10 Personen mit dem Grippevirus infiziert.
					\item Nach einem Tag sind bereits 100 Personen an Grippe erkrankt.
					\item Am 3. Tag nimmt die Anzahl an Erkrankten am stärksten zu.
					\item Am 8. Tag sind bereits 730 Personen erkrankt.
					\item Am 10. Tag erreicht die Grippewelle (d.h. die Anzahl an Erkrankten) ihr Maximum.
				\end{itemize}
				
				\begin{center}
				\resizebox{0.8\linewidth}{!}{\psset{xunit=1.0cm,yunit=0.01cm,algebraic=true,dimen=middle,dotstyle=o,dotsize=5pt 0,linewidth=0.8pt,arrowsize=3pt 2,arrowinset=0.25}
\begin{pspicture*}(-0.98,-106.44735758690118)(17.56,913.840522680063)
\multips(0,-100)(0,100.0){11}{\psline[linestyle=dashed,linecap=1,dash=1.5pt 1.5pt,linewidth=0.4pt,linecolor=lightgray]{c-c}(-0.98,0)(17.56,0)}
\multips(0,0)(1.0,0){19}{\psline[linestyle=dashed,linecap=1,dash=1.5pt 1.5pt,linewidth=0.4pt,linecolor=lightgray]{c-c}(0,-100)(0,913.840522680063)}
\psaxes[labelFontSize=\scriptstyle,xAxis=true,yAxis=true,Dx=1.,Dy=100.,ticksize=-2pt 0,subticks=2]{->}(0,0)(-0.98,-106.44735758690118)(17.56,913.840522680063)
\psplot[linewidth=1.2pt,plotpoints=200]{0}{16}{-(25.0/32.0)*x^(3.0)+225.0/32.0*x^(2.0)+375.0/4.0*x}
\rput[tl](16.42,54.27010269398862){Tage}
\rput[tl](0.3,871.6632677871374){Anzahl der Erkrankten}
\end{pspicture*}}
\end{center}

\subsection{Aufgabenstellung:}
\begin{enumerate}
	\item Berechne den Wert des Ausdruck $\frac{E(8)-E(0)}{8}$!
	
	Kreuze diejenige(n) Aussage(n) an, die eine korrekte Interpretation des Ausdrucks $\frac{E(8)-E(0)}{8}$ ist/sind!\leer
	
	\multiplechoice[5]{  %Anzahl der Antwortmoeglichkeiten, Standard: 5
					L1={Der Ausdruck gibt die prozentuelle Änderung der Anzahl an 
Erkrankten innerhalb der ersten 8 Tage an.},   %1. Antwortmoeglichkeit 
					L2={Der Ausdruck gibt die Zunahme der Anzahl an Erkrankten in den ersten 8 Tagen an.},   %2. Antwortmoeglichkeit
					L3={Der Ausdruck gibt die Ausbreitungsgeschwindigkeit der Grippewelle am 8. Tag an.},   %3. Antwortmoeglichkeit
					L4={Der Ausdruck beschreibt, wie viele Neuerkrankte es am 8. Tag gibt. },   %4. Antwortmoeglichkeit
					L5={Der Ausdruck beschreibt die mittlere Änderungsrate der Anzahl an Erkrankten innerhalb der ersten 8 Tage.},	 %5. Antwortmoeglichkeit
					L6={},	 %6. Antwortmoeglichkeit
					L7={},	 %7. Antwortmoeglichkeit
					L8={},	 %8. Antwortmoeglichkeit
					L9={},	 %9. Antwortmoeglichkeit
					%% LOESUNG: %%
					A1=5,  % 1. Antwort
					A2=0,	 % 2. Antwort
					A3=0,  % 3. Antwort
					A4=0,  % 4. Antwort
					A5=0,  % 5. Antwort
					}
						
		\item Zur Bestimmung der Koeffizienten $a,b,c$ und $d$ werden folgende Gleichungen aufgestellt:
		
		\begin{itemize}
		\item $d=10$
		\item $a+b+c+d=100$
		\item $18a+2b=0$
		\item $300a+20b+c=0$
		\end{itemize}
		
		Gib an, welche der angegebenen Informationen durch die vierte Gleichung modelliert werden kann, und erkläre den Zusammenhang zwischen Information und Gleichung!
		
		\item Gib an, an welchem Tag die progressive Zunahme der Anzahl an Erkrankten (das heißt der Zuwachs an Erkrankten wird von Tag zu Tag größer) in eine degressive Zunahme (das heißt der Zuwachs an Erkrankten nimmt pro Tag wieder ab) übergeht!
		
		Kreuze diejenige(n) Aussage(n) an, mit der/denen man eine progressive Zunahme bestimmen kann!\leer
		
		\multiplechoice[5]{  %Anzahl der Antwortmoeglichkeiten, Standard: 5
						L1={$E'(t)>0$},   %1. Antwortmoeglichkeit 
						L2={$E(t)\geq 0$},   %2. Antwortmoeglichkeit
						L3={$E(t_1)<E(t_2)$ für alle $t_1>t_2$},   %3. Antwortmoeglichkeit
						L4={$E''(t)>0$},   %4. Antwortmoeglichkeit
						L5={$E'(t)=E''(t)=0$},	 %5. Antwortmoeglichkeit
						L6={},	 %6. Antwortmoeglichkeit
						L7={},	 %7. Antwortmoeglichkeit
						L8={},	 %8. Antwortmoeglichkeit
						L9={},	 %9. Antwortmoeglichkeit
						%% LOESUNG: %%
						A1=4,  % 1. Antwort
						A2=0,	 % 2. Antwort
						A3=0,  % 3. Antwort
						A4=0,  % 4. Antwort
						A5=0,  % 5. Antwort
						}
						\end{enumerate}\leer
				
\antwort{\subsection{Lösungserwartung:}
\begin{enumerate}
	\item $\frac{E(8)-E(0)}{8}=\frac{730-10}{8}=90$
	
	(Innterhalb der ersten 8 Tage nimmt die Anzahl der Erkrankten um durchschnittlich 90 Personen pro Tag zu.)
	
	Lösung Multiple Choice: siehe oben
	
	\item "`Am 10. Tag erreicht die Grippewelle (d.h. die Anzahl an Erkrankten) ihr Maximum"' bzw. die 5. Information
	
	Diese Textstelle beschreibt das lokale Maximum (den Hochpunkt), d.h., an dieser Stelle gilt: $E'(10)=0$.
	
	Durch das Aufstellen der ersten Ableitungsfunktion und das Einsetzen des Wertes $t=10$ erhält man die nachstehende Gleichung:
	
	$E'(t)=3at²+2bt+c \Rightarrow E'(10)=300a+20b+c=0$
	
	\item An 3. Tag. - Lösung Multiple Choice siehe oben
	\end{enumerate}
		
		\subsection{Lösungsschlüssel:}
\begin{enumerate}
	\item - 1 Grundkompetenzpunkt (für die Berechnung des Ausdrucks)
	
	- 1 Reflexionspunkt (für das richtige Ankreuzen der zutreffenden Aussage)
	
	\item 2 Reflexionspunkte, davon:
	
	\hspace*{0.5cm} - 1 Punkt für das Erkennen der zugehörigen Information
	
	\hspace*{0.5cm} - 1 Punkt für die Erklärung (dieser Punkt ist auch zu geben, wenn \hspace*{0,5cm}die Erklärung nu in verbaler Form vorliegt oder nur die Rechenschritte \hspace*{0.5cm}durchgeführt wurden)
	
	\item - 1 Reflexionspunkt für die kontextbezogene Frage
	
	- 1 Grundkompetenzpunkt für das alleinige Ankreuzen der richtigen Aussage
			\end{enumerate}
		}
		\end{langesbeispiel}