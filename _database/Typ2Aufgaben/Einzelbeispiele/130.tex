\section{130 - K6 - AG 3.4 - Flugbahnen - VerSie}

\begin{langesbeispiel} \item[6] %PUNKTE DES BEISPIELS
Die Flugbahnen zweier Flugzeuge A und B werden durch folgende Geradengleichungen beschrieben:

A: $X=\Vek{-300}{-100}{200}+a\cdot\Vek{70}{60}{0}$ und B: $X=\Vek{100}{-100}{-100}+b\cdot\Vek{100}{200}{100}$

Alle Koordinaten sind in Metern und die Parameter $a$ und $b$ werden in Sekunden seit Beginn der Beobachtung angegeben.%Aufgabentext

\begin{aufgabenstellung}
\item %Aufgabentext

\ASubitem{Zeige, dass die Flugbahnen weder ident noch parallel sind.} %Unterpunkt1
\Subitem{Zeige, dass sie sich schneiden, indem du den Schnittpunkt berechnest.} %Unterpunkt2

\item %Aufgabentext

\ASubitem{Wo war das Flugzeug A zu Beginn der Beobachtung?} %Unterpunkt1
\Subitem{Warum kommt es unter gleichbleibender Flugbedingungen zu keinem Zusammenstoß?} %Unterpunkt2

\item Ein weiteres Flugzeug C fliegt gleichförmig (Geschwindigkeit und Richtung bleibt gleich) und befindet sich zum Zeitpunkt $t=-2$\,s im Punkt\\ 
	$P=(100\mid -200\mid 300)$ und zum Zeitpunkt $t=2$\,s in $Q=(100\mid 0\mid 300)$.%Aufgabentext

\Subitem{Gib eine Geradengleichung an, die die Flugbahn des Flugzeuges C beschreibt.} %Unterpunkt1
\Subitem{Berechne den Abstand von A und C zum Zeitpunkt $t=a=0$.} %Unterpunkt2

\end{aufgabenstellung}

\begin{loesung}
\item \subsection{Lösungserwartung:} 

\Subitem{$\vec{a}=\Vek{70}{60}{0}\neq r\cdot\Vek{100}{200}{100}$
	
	Da der Richtungsvektor von A kein Vielfaches vom Richtungvektor von B ist, handelt es sich bei den Geraden weder um idente noch um parallele Geraden.} %Lösung von Unterpunkt1
\Subitem{\begin{tabular}{crcl}
I:&$-300+70\cdot a$&$=$&$100+100\cdot b$\\
II:&$-100+60\cdot a$&$=$&$-100+200\cdot b$\\ 
III:&$200+0\cdot a$&$=$&$-100+100\cdot b$\\ \hline
III:&$b=3$&&\\
II:&$a=10$&&\\
I:&$-300+70\cdot 10$&$=$&$100+100\cdot 3$
\end{tabular}

	Der Schnittpunkt liegt bei $S=\Vek{-300}{-100}{200}+10\cdot\Vek{70}{60}{0}=\Vek{400}{500}{200}$} %%Lösung von Unterpunkt2

\setcounter{subitemcounter}{0}
\subsection{Lösungsschlüssel:}
 
\Subitem{Ein Punkt für eine korrekte Begründung.} %Lösungschlüssel von Unterpunkt1
\Subitem{Ein Punkt für die Berechnung des Schnittpunkts.} %Lösungschlüssel von Unterpunkt2

\item \subsection{Lösungserwartung:} 

\Subitem{Zu Beginn der Beobachtung war das Flugzeug A am Punkt $\Vek{-300}{-100}{200}$.} %Lösung von Unterpunkt1
\Subitem{Unter gleichbleibenden Flugbedingungen kann es zu keinem Zusammenstoß kommen, da die Flugzeuge zu unterschiedlichen Zeiten am Schnittpunkt sind.} %%Lösung von Unterpunkt2

\setcounter{subitemcounter}{0}
\subsection{Lösungsschlüssel:}
 
\Subitem{Ein Punkt für die richtige Angabe des "`Startpunkts"'.} %Lösungschlüssel von Unterpunkt1
\Subitem{Ein Punkt für eine korrekte Begründung.} %Lösungschlüssel von Unterpunkt2

\item \subsection{Lösungserwartung:} 

\Subitem{$\vec{PQ}=\Vek{100}{0}{300}-\Vek{100}{-200}{300}=\Vek{0}{200}{0}$
	
	$\frac{1}{4}\cdot\vec{PQ}=\Vek{0}{50}{0}$
	
	C: $X=\Vek{100}{-100}{300}+t\cdot\Vek{0}{50}{0}$} %Lösung von Unterpunkt1
\Subitem{$\bigg|\Vek{-300}{-100}{200}-\Vek{100}{-100}{300}\bigg|=\bigg|\Vek{-400}{0}{-100}\bigg|=\sqrt{(-400)^2+(-100)^2}\approx 412,31$\,m} %%Lösung von Unterpunkt2

\setcounter{subitemcounter}{0}
\subsection{Lösungsschlüssel:}
 
\Subitem{Ein Punkt für die richtige Geradengleichung.} %Lösungschlüssel von Unterpunkt1
\Subitem{Ein Punkt für den richtigen Abstand.} %Lösungschlüssel von Unterpunkt2

\end{loesung}

\end{langesbeispiel}