\section{04 - MAT - AG 2.3, AN 1.3, AN 2.1, AN 3.3 - Wachstum einer Pflanze - BIFIE Aufgabensammlung}

\begin{langesbeispiel} \item[0] %PUNKTE DES BEISPIELS
Manche einjährige Nutz- und Zierpflanzen wachsen in den ersten Wochen nach der Pflanzung sehr rasch. Im Folgenden wird nun eine spezielle Sorte betrachtet. Die endgültige Größe einer
Pflanze der betrachteten Sorte hängt auch von ihrem Standort ab und kann im Allgemeinen zwischen 1,0\,m und 3,5\,m liegen. Pflanzen dieser Sorte, die im Innenbereich gezüchtet werden,
erreichen Größen von 1,0\,m bis 1,8\,m. 

In einem Experiment wurde der Wachstumsverlauf dieser Pflanze im Innenbereich über einen Zeitraum von 17 Wochen beobachtet und ihre Höhe dokumentiert. Im Anschluss wurde die
Höhe $h$ dieser Pflanze in Abhängigkeit von der Zeit t durch eine Funktion $h$ mit $h(t) = \frac{1}{24}\cdot \left(-t^3+27t^2+120\right)$ modelliert. Dabei bezeichnet $t$ die Anzahl der Wochen seit der Pflanzung und $h(t)$ die Höhe zum Zeitpunkt $t$ in Zentimetern. Die nachstehende Abbildung zeigt den Graphen der Funktion h im Beobachtungszeitraum $[0; 17]$.

\begin{center}
\resizebox{0.8\linewidth}{!}{\psset{xunit=0.5cm,yunit=0.05cm,algebraic=true,dimen=middle,dotstyle=o,dotsize=5pt 0,linewidth=0.8pt,arrowsize=3pt 2,arrowinset=0.25}
\begin{pspicture*}(-1.8,-16.388636363636373)(17.764967532467544,130.55535714285685)
\multips(0,0)(0,10.0){13}{\psline[linestyle=dashed,linecap=1,dash=1.5pt 1.5pt,linewidth=0.4pt,linecolor=gray]{c-c}(0,0)(17.764967532467544,0)}
\multips(0,0)(1.0,0){19}{\psline[linestyle=dashed,linecap=1,dash=1.5pt 1.5pt,linewidth=0.4pt,linecolor=gray]{c-c}(0,0)(0,130.55535714285685)}
\begin{scriptsize}
\psaxes[xAxis=true,yAxis=true,Dx=1.,Dy=20.,ticksize=-2pt 0,ysubticks=2, subticksize=1, subtickcolor=black]{->}(0,0)(0,0)(17.764967532467544,130.55535714285685)[$t$,140] [$h(t)$,-40]
\psplot[linewidth=1.2pt,plotpoints=200]{0}{17}{1.0/24.0*(-x^(3.0)+27.0*x^(2.0)+120.0)}

\rput[bl](8.5,70){$h$}
\end{scriptsize}
\end{pspicture*}}
\end{center}%Aufgabentext

\begin{aufgabenstellung}
\item %Aufgabentext

\Subitem{Berechne den Wert des Quotienten $\frac{h(13)-h(9)}{4}$ und den Wert von $h'(9)$.} %Unterpunkt1
\Subitem{Gib an, welche Bedeutung die beiden berechneten Ergebnisse im gegebenen Kontext haben.} %Unterpunkt2

\item %Aufgabentext

\Subitem{Zeige durch Rechnung, dass die Funktion $h$ im gegebenen Intervall keinen lokalen Hochpunkt hat.}
\Subitem{Begründe deine Rechenschritte.}

\item Für das Wachstum der beobachteten Pflanze ist auch die entsprechende Düngung von Bedeutung. Im gegebenen Fall wurde die Pflanze zwei Wochen vor dem Zeitpunkt des stärksten Wachstums gedüngt.%Aufgabentext

\Subitem{Ermittle diesen Zeitpunkt durch Rechnung.} %Unterpunkt1
\Subitem{Begründe deine Überlegungen.} %Unterpunkt2

\item Im selben Zeitraum wurde das Höhenwachstum von zwei weiteren Pflanzen der gleichen Sorte beobachtet und modelliert. Die nachstehenden Abbildungen zeigen die Graphen der
entsprechenden Funktionen $h_1$ und $h_2$.

\meinlr{
\begin{center}
\psset{xunit=0.37cm,yunit=0.03cm,algebraic=true,dimen=middle,dotstyle=o,dotsize=5pt 0,linewidth=0.8pt,arrowsize=3pt 2,arrowinset=0.25}
\begin{pspicture*}(-1.8,-16.388636363636373)(17.764967532467544,130.55535714285685)
\multips(0,0)(0,10.0){15}{\psline[linestyle=dashed,linecap=1,dash=1.5pt 1.5pt,linewidth=0.4pt,linecolor=black!60]{c-c}(0,0)(17.764967532467544,0)}
\multips(0,0)(1.0,0){19}{\psline[linestyle=dashed,linecap=1,dash=1.5pt 1.5pt,linewidth=0.4pt,linecolor=black!60]{c-c}(0,0)(0,130.55535714285685)}
\psaxes[labelFontSize=\scriptstyle,xAxis=true,yAxis=true,Dx=1.,Dy=20.,ticksize=-2pt 0,ysubticks=2, subticksize=1, subtickcolor=black]{->}(0,0)(0,0)(17.764967532467544,130.55535714285685)[t,140] [$h_1(t)$,-40]
\psplot[linewidth=1.2pt,plotpoints=200]{0}{17}{-7.42*2.718281828459045^(-0.2*(x-14.1))+130.0}
\begin{scriptsize}
\rput[bl](3.5,85){$h_1$}
\end{scriptsize}
\end{pspicture*}
\end{center}}
{\begin{center}
\psset{xunit=0.37cm,yunit=0.03cm,algebraic=true,dimen=middle,dotstyle=o,dotsize=5pt 0,linewidth=0.8pt,arrowsize=3pt 2,arrowinset=0.25}
\begin{pspicture*}(-1.8,-16.388636363636373)(17.764967532467544,130.55535714285685)
\multips(0,0)(0,10.0){15}{\psline[linestyle=dashed,linecap=1,dash=1.5pt 1.5pt,linewidth=0.4pt,linecolor=black!60]{c-c}(0,0)(17.764967532467544,0)}
\multips(0,0)(1.0,0){19}{\psline[linestyle=dashed,linecap=1,dash=1.5pt 1.5pt,linewidth=0.4pt,linecolor=black!60]{c-c}(0,0)(0,130.55535714285685)}
\psaxes[labelFontSize=\scriptstyle,xAxis=true,yAxis=true,Dx=1.,Dy=20.,ticksize=-2pt 0,ysubticks=2, subticksize=1, subtickcolor=black]{->}(0,0)(0,0)(17.764967532467544,130.55535714285685)[t,140] [$h_2(t)$,-40]
\psplot[linewidth=1.2pt,plotpoints=200]{0}{17}{5.55*1.21^(x-0.6)}
\begin{scriptsize}
\rput[bl](5.3,20){$h_2$}
\end{scriptsize}
\end{pspicture*}
\end{center}
}%Aufgabentext

\Subitem{Vergleiche das Krümmungsverhalten der Funktionen $h$, $h_1$ und $h_2$ im Intervall $[0; 17]$.} %Unterpunkt1
\Subitem{Interpretiere das Krümmungsverhalten im Hinblick auf das Wachstum der drei Pflanzen.} %Unterpunkt2

\end{aufgabenstellung}

\begin{loesung}
\item \subsection{Lösungserwartung:} 

\Subitem{$\frac{h(13)-h(9)}{4}\approx 9,47$

$h'(t)=\frac{1}{24}\cdot(-3t^2+54t)=\frac{1}{8}\cdot(-t^2+18t) \Rightarrow h'(9)\approx 10,13$} %Lösung von Unterpunkt1
\Subitem{Die mittlere Wachstumsgeschwindigkeit im Zeitintervall $[9;13]$ beträgt rund $9,5\,cm$ pro Woche. Die momentane Wachstumsgeschwindigkeit zum Zeitpunkt $t=9$, d.h. nach 9 Wochen beträgt rund $10,1\,cm$ pro Woche.} %%Lösung von Unterpunkt2

\setcounter{subitemcounter}{0}
\subsection{Lösungsschlüssel:}
 
\Subitem{Ein Punkt für die korrekte Berechnung des Quotienten sowie des Werts $h'(9)$.} %Lösungschlüssel von Unterpunkt1
\Subitem{Ein Punkt für die korrekte Interpretation im Kontext.} %Lösungschlüssel von Unterpunkt2

\item \subsection{Lösungserwartung:} 

\Subitem{$h'(t)=\frac{1}{24}\cdot(-3t^2+54t)=\frac{1}{8}\cdot(-t^2+18t)$
	
	$t\cdot(-t+18)=0$
	
	$t_1=0, t_2=18$} %Lösung von Unterpunkt1
\Subitem{In einem lokalen Hochpunkt muss die Tangente an den Graphen horizontal sein, d.h., die 1. Ableitung muss den Wert 0 haben.
	
	Die Funktion hat an der Stelle $t=0$ ein lokales Minimum und an der Stelle $t=18$ ein lokales Maximum. Der Wert $t=18$ liegt nicht im Beobachtungsintervall, d.h., die Funktion hat im gegebenen Intervall keinen lokalen Hochpunkt.} %%Lösung von Unterpunkt2

\setcounter{subitemcounter}{0}
\subsection{Lösungsschlüssel:}
 
\Subitem{Ein Punkt für die korrekte Berechnung.} %Lösungschlüssel von Unterpunkt1
\Subitem{Ein Punkt für eine korrekte Begründung.} %Lösungschlüssel von Unterpunkt2

\item \subsection{Lösungserwartung:} 

\Subitem{$h''(t)=\frac{1}{4}\cdot(-t+9)$} %Lösung von Unterpunkt1
\Subitem{Die Kurve ist für $t<9$ linksgekrümmt, d.h., die Wachstumsgeschwindigkeit nimmt zu. Die Kurve ist für $t>9$ rechtsgekrümmt, d.h., die Wachstumsgeschwindigkeit nimmt ab. Daher ist die Wachstumsgeschwindigkeit nach 9 Wochen am größten. Die Pflanz wurde also am Beginn der 8. Woche gedüngt.
	
	Ein weiterer Lösungsansatz wäre, das Maximum der Wachstumsfunktion (also von $h'$) zu bestimmen.} %%Lösung von Unterpunkt2

\setcounter{subitemcounter}{0}
\subsection{Lösungsschlüssel:}
 
\Subitem{Ein Punkt für die korrekte Berechnung.} %Lösungschlüssel von Unterpunkt1
\Subitem{Ein Punkt für die korrekte Begründung.} %Lösungschlüssel von Unterpunkt2

\item \subsection{Lösungserwartung:} 

\Subitem{Die Funktion $h_1$ ist rechtsgekrümmt, die Funktion $h_2$ ist linksgekrümmt, das Krümmungsverhalten der Funktion $h$ ändert sich.} %Lösung von Unterpunkt1
\Subitem{Das bedeutet, die Wachstumsgeschwindigkeit derjenigen Pflanze, die durch $h_1$ beschrieben wird, wird immer kleiner (sie wächst immer langsamer) und die Wachstumsgeschwindigkeit derjenigen Pflanze, die durch $h_2$ beschrieben wird, wird immer größer (sie wächst immer schneller).
	
	Im Vergleich dazu ändert sich das Monotonieverhalten der Wachstumsgeschwindigkeit bei derjenigen Pflanze, die durch $h$ beschrieben wird, an der Stelle $t=9$ [vgl. c)].} %%Lösung von Unterpunkt2

\setcounter{subitemcounter}{0}
\subsection{Lösungsschlüssel:}
 
\Subitem{Ein Punkt für das korrekte Beschreiben des Krümmungsverhalten.} %Lösungschlüssel von Unterpunkt1
\Subitem{Ein Punkt für die richtige Interpretation.} %Lösungschlüssel von Unterpunkt2

\end{loesung}
\end{langesbeispiel}