\section{49 - MAT - FA 1.4, AN 3.3, AN 1.3 - Die Bedeutung der Parameter in der Funktionsgleichung einer Polynomfunktion - Matura 2014/15 1. Nebentermin}

\begin{langesbeispiel} \item[0] %PUNKTE DES BEISPIELS
				
				Betrachtet werden Polynomfunktion $f$ mit $f(x)=x^2+b\cdot x+16$ $(b\in\mathbb{R})$

\subsection{Aufgabenstellung:}
\begin{enumerate}
	\item Der Graph einer solchen Funktion $f$ verläuft durch den Punkt $P=(-1|7)$.
	
	\fbox{A} Bestimme den Parameter $b$ dieser Funktion $f$!
	
	Gib die Steigung dieser Funktion $f$ an der Stelle $x=-1$ an!
	
\item Gib an, welcher allgemeine Zusammenhang zwischen der Extremstelle $x_E$ einer solchen Funktion $f$ und dem Parameter $b$ besteht!

Berechne alle Werte des Parameters $b$, für die $f(x_E)=-9$ gilt!

\item Für bestimmte Werte von $b$ liegt der Tiefpunkt des Graphen von $f$ auf einer der Koordinatenachsen. Bestimme diese Tiefpunkte!

Der Graph der Polynomfunktion $g$ zweiten Grades geht durch diese Punkte. Bestimme eine Funktionsgleichung für $g$!

\item Auf dem Graphen einer solchen Funktion $f$ liegt der Punkt $Q=(2|f(2))$.

Drücke den Funktionswert $f(2)$ in Abhängigkeit vom Parameter $b$ aus!

Die Tangente im Punkt $Q$ an den Graphen der Funktion $f$ schneidet die senkrechte Achse in einem Punkt $R$. Zeige mittels einer Rechnung, dass die Lage dieses Punktes $R$ von der Wahl von $b$ unabhängig ist!
						\end{enumerate}\leer
				
\antwort{
\begin{enumerate}
	\item \subsection{Lösungserwartung:} 
	
	$7=(-1)^2+b\cdot (-1)+16 \Rightarrow -10=-b$
	
	$b=10$\leer
	
	$f'(-1)=2\cdot (-1)+10=8$
	
	\subsection{Lösungsschlüssel:}
	\begin{itemize}
		\item Ein Ausgleichspunkt für die korrekte Angabe des Parameters $b$.
		\item Ein Punkt für die korrekte Angabe der Steigung der Funktion $f$ an der Stelle $x = -1$.
	\end{itemize}
	
	\item \subsection{Lösungserwartung:}
			
		$f'(x)=2\cdot x+b \Rightarrow 2\cdot x_E+b=0$
		
		oder:
		
		$x_E=-\frac{b}{2}$
		
		$f\left(-\frac{b}{2}\right)=-9 \Rightarrow\frac{b^2}{4}-\frac{b^2}{2}+16=-9$
		
		$\Rightarrow b=\pm 10$

	\subsection{Lösungsschlüssel:}
	
\begin{itemize}
	\item Ein Punkt für einen korrekten Zusammenhang zwischen $x_E$ und $b$.
	\item Ein Punkt für die Angabe der beiden korrekten Werte für $b$.
\end{itemize}

\item \subsection{Lösungserwartung:}

	Mögliche Bestimmung der Tiefpunkte:
	
	\begin{itemize}
		\item Tiefpunkt des Graphen von $f$ liegt auf der $x$-Achse $\Rightarrow$ Die Funktion $f$ besitzt genau eine reelle Nullstelle.
		
		$f(x)=0 \Rightarrow x_{1,2}=-\frac{b}{2}\pm\sqrt{\left(-\frac{b}{2}\right)^2-16}$
		
		$\Rightarrow\left(-\frac{b}{2}\right)^2-16=0 \Rightarrow b_1=-8, \hspace{0,5cm}b_2=8$
		
		$\Rightarrow x_1=4, \hspace{0,5cm} x_2=-4 \Rightarrow T_1=(4|0), T_2=(-4|0)$
		\item Tiefpunkt des Graphen von $f$ liegt auf der senkrechten Achse
		
		$\Rightarrow b=0 \Rightarrow T_3=(0|16)$\leer
		
		$g(x)=a\cdot x^2+c$
		
		$g(0)=16 \hspace{1cm} c=16$
		
		$g(4)=0 \Rightarrow 16\cdot a+16=0 \Rightarrow a=-1 \Rightarrow g(x)=-x^2+16$
	\end{itemize}

	\subsection{Lösungsschlüssel:}
	
\begin{itemize}
	\item Ein Punkt für die Angabe aller drei Tiefpunkte
	\item Ein Punkt für die Angabe einer korrekten Funktionsgleichung der Funktion $g$. Äquvalente Funktionsgleichungen sind ebenfalls als richtig zu werten.
\end{itemize}

\item \subsection{Lösungserwartung:}

	$f(2)=2^2+2\cdot b+16=2\cdot b+20$
	
	Die Lage der Tangente ergibt sich aus $f(2)=f'(2)\cdot 2+d$.
	
	Daraus folgt: $2\cdot b+20=(4+b)\cdot 2+d$ und daraus $d=12$, daher ist die Lage des Punktes $R$ unabhängig von $b$.

	\subsection{Lösungsschlüssel:}
	
\begin{itemize}
	\item  Ein Punkt für die korrekte Angabe des Funktionswertes $f(2)$ in Abhängigkeit von $b$.
	\item Ein Punkt für einen korrekten rechnerischen Nachweis.
\end{itemize}
\end{enumerate}}
		\end{langesbeispiel}