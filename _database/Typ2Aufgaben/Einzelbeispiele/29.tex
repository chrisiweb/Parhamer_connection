\section{29 - MAT - AG 2.1, FA 1.2, AN 2.1, AN 3.3 - Aufnahme einer Substanz ins Blut - Saturn-V-Rakete - BIFIE Aufgabensammlung}

\begin{langesbeispiel} \item[0] %PUNKTE DES BEISPIELS
				Wenn bei einer medizinischen Behandlung eine Substanz verabreicht wird, kann die Konzentration der Substanz im Blut (kurz: Blutkonzentration) in Abhängigkeit von der Zeit $t$ in manchen Fällen durch eine sogenannte Bateman-Funktion c mit der Funktionsgleichung $c(t)=d\cdot (\textbf{\textit{e}}^{-a\cdot t}-\textbf{\textit{e}}^{-b\cdot t})$ mit den Parametern $a,b,d\in\mathbb{R}$ und $a,b,d>0$ modelliert werden. Die Zeit $t$ wird in Stunden gemessen, $t=0$ entspricht dem Zeitpunkt der Verabreichung der Substanz.\\
Die Bioverfügbarkeit $f$ gibt den Anteil der verabreichten Substanz an, der unverändert in den
Blutkreislauf gelangt. Bei einer intravenösen Verabreichung beträgt der Wert der Bioverfügbarkeit 1.\\
Das Verteilungsvolumen $V$ beschreibt, in welchem Ausmaß sich die Substanz aus dem Blut ins Gewebe verteilt.\\
Der Parameter $d$ ist direkt proportional zur verabreichten Dosis $D$ und zur Bioverfügbarkeit $f$, außerdem ist d indirekt proportional zum Verteilungsvolumen $V$.\\
Die nachstehende Abbildung zeigt exemplarisch den zeitlichen Verlauf der Blutkonzentration in
Nanogramm pro Milliliter (ng/ml) für den Fall der Einnahme einer bestimmten Dosis der Substanz Lysergsäurediethylamid.

\begin{center}
\resizebox{0.8\linewidth}{!}{\psset{xunit=1.0cm,yunit=0.5cm,algebraic=true,dimen=middle,dotstyle=o,dotsize=5pt 0,linewidth=0.8pt,arrowsize=3pt 2,arrowinset=0.25}
\begin{pspicture*}(-0.5083674461248454,-1.6869948801997758)(9.20426795498609,9.811469816975478)
\multips(0,0)(0,2.0){6}{\psline[linestyle=dashed,linecap=1,dash=1.5pt 1.5pt,linewidth=0.4pt,linecolor=lightgray]{c-c}(0,0)(9.20426795498609,0)}
\multips(0,0)(1.0,0){10}{\psline[linestyle=dashed,linecap=1,dash=1.5pt 1.5pt,linewidth=0.4pt,linecolor=lightgray]{c-c}(0,0)(0,9.811469816975478)}
\psaxes[labelFontSize=\scriptstyle,xAxis=true,yAxis=true,Dx=1.,Dy=2.,ticksize=-2pt 0,subticks=2]{->}(0,0)(0.,0.)(9.20426795498609,9.811469816975478)
\psplot[linewidth=1.2pt,plotpoints=200]{0}{9.20426795498609}{19.5*(EXP(-0.4*x)-EXP(-1.3*x))}
\begin{scriptsize}
\rput[tl](0.25,9.355737984465483){Blutkonzentration $c(t)$ (in ng/ml)}
\rput[tl](6.5,-0.9508126892220918){Zeit $t$ (in Stunden)}
\rput[tl](4,4.4){c}
\end{scriptsize}
\end{pspicture*}}
\end{center}

Dieser zeitliche Verlauf wird durch die Bateman-Funktion $c$ mit den Parametern $d=19,5,
 a=0,4$ und $b=1,3$ beschrieben.\\
Der Graph der Bateman-Funktion nähert sich für große Zeiten $t$ wie eine Exponentialfunktion
asymptotisch der Zeitachse an.

\subsection{Aufgabenstellung:}
\begin{enumerate}
	\item Berechne für die in der Einleitung angegebene Bateman-Funktion denjenigen Zeitpunkt, zu dem die maximale Blutkonzentration erreicht wird! Gib dazu die Gleichung der entsprechenden Ableitungsfunktion und den Ansatz in Form einer Gleichung an!
	
Begründe allgemein, warum der Wert des Parameters $d$ in der Bateman-Funktion nur
die Größe der maximalen Blutkonzentration beeinflusst, aber nicht den Zeitpunkt, zu dem
diese erreicht wird!

\item Kreuze diejenige Formel an, die den Zusammenhang zwischen dem Parameter $d$ der
Bateman-Funktion und den in der Einleitung beschriebenen Größen $V$, $D$ und $f$ korrekt beschreibt! Der Parameter $\lambda$ ist dabei ein allgemeiner Proportionalitätsfaktor.\leer

\multiplechoice[6]{  %Anzahl der Antwortmoeglichkeiten, Standard: 5
				L1={$d=\lambda\cdot\frac{D}{V\cdot f}$},   %1. Antwortmoeglichkeit 
				L2={$d=\lambda\cdot\frac{D\cdot V}{f}$},   %2. Antwortmoeglichkeit
				L3={$d=\lambda\cdot\frac{V\cdot f}{D}$},   %3. Antwortmoeglichkeit
				L4={$d=\lambda\cdot\frac{D\cdot f}{V}$},   %4. Antwortmoeglichkeit
				L5={$d=\lambda\cdot\frac{V}{D\cdot f}$},	 %5. Antwortmoeglichkeit
				L6={$d=\lambda\cdot\frac{f}{V\cdot D}$},	 %6. Antwortmoeglichkeit
				L7={},	 %7. Antwortmoeglichkeit
				L8={},	 %8. Antwortmoeglichkeit
				L9={},	 %9. Antwortmoeglichkeit
				%% LOESUNG: %%
				A1=4,  % 1. Antwort
				A2=0,	 % 2. Antwort
				A3=0,  % 3. Antwort
				A4=0,  % 4. Antwort
				A5=0,  % 5. Antwort
				}\leer
				
				Bei einem konstanten Wert des Parameters $d$ und der Bioverfügbarkeit $f$ kann man die
verabreichte Dosis $D$ als Funktion des Verteilungsvolumens $V$ auffassen. Beziehe dich auf die von dir angekreuzte Formel und gib für die in der Einleitung dargestellte Bateman-Funktion und für den Fall einer intravenösen Verabreichung die Funktionsgleichung $D(V)$ der Funktion $D$ an! Gib an, um welchen Funktionstyp es sich bei $D$ handelt!	
						\end{enumerate}\leer
				
\antwort{\subsection{Lösungserwartung:}
\begin{enumerate}
	\item $c(t)=19,5\cdot(\textbf{\textit{e}}^{-0,4\cdot t}-\textbf{\textit{e}}^{-1,3\cdot t})$\\
	$c'(t)=19,5\cdot(-0,4\cdot\textbf{\textit{e}}^{-0,4\cdot t}+1,3\cdot\textbf{\textit{e}}^{-1,3\cdot t})=0\Rightarrow t\approx 1,31$ Stunde
	
	Mögliche Begründung:
	
	Für die Berechnung des Zeitpunktes der maximalen Blutkonzentration muss die Gleichung
$c'(t)=0$ nach $t$ gelöst werden. Da der Parameter $d$ in der Funktion $c$ eine multiplikative
Konstante ist und bei der Berechnung der ersten Ableitung von $c$ unverändert bleibt, beeinflusst er nicht die Lösungsmenge dieser Gleichung. Für die Berechnung der Größe der
maximalen Blutkonzentration muss in die Funktion $c$ eingesetzt werden, und deshalb beeinflusst der Wert von $d$ dieses Ergebnis.

oder:

$c'(t)=d\cdot(-a\cdot\textbf{\textit{e}}^{-a\cdot t}+b\cdot\textbf{\textit{e}}^{-b\cdot t})=0\Rightarrow t=\frac{\ln(a)-\ln(b)}{a-b}$
	
Der Parameter $d$ tritt in dieser Formel nicht auf, $t$ ist also von $d$ unabhängig.

\item Lösung Multiple Choice siehe oben

Die Funktionsgleichung lautet: $D(V)=\frac{19,5}{\lambda}\cdot V$; dabei handelt es sich um eine lineare Funktion.
\end{enumerate}}
		\end{langesbeispiel}