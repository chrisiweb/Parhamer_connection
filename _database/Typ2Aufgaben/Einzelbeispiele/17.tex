\section{17 - MAT - AN 1.1, WS 1.1, AN 1.3, FA 1.9 - Emissionen - BIFIE Aufgabensammlung}

\begin{langesbeispiel} \item[0] %PUNKTE DES BEISPIELS
				Laut Immissionsschutzgesetz - Luft (IG-L) gilt auf manchen Autobahnabschnitten in Österreich für PKW eine Tempo-100-Beschränkung, wenn die Grenzwerte für bestimmte Luftschadstoffe überschritten werden. Für LKW gilt ein generelles Tempolimit von 80\,km/h.
				
				Abbildung 1 zeigt vier Messwerte für die freigesetzte Menge von Stickoxiden ($\text{NO}_x$) bei unterschiedlichen Fahrgeschwindigkeiten (in km/h) für einen durchschnittlichen PKW. Die freigesetzte $\text{NO}_x$-Menge wird in Gramm pro gefahrenem Kilometer angegeben. Die Abhängigkeit von Fahrgeschwindigkeit und $\text{NO}_x$-Ausstoß wurde durch eine Funktion $A$ modelliert, deren Graph ebenfalls in Abbildung 1 dargestellt ist.

Abbildung 2 zeigt den Anteil der Verkehrsmittel (PKW, LKW, sonstige) am Verkehrsaufkommen und am Ausstoß (=\, Emission) von Stickoxiden und Feinstaub (PM 10) im Unterinntal in Tirol.

\meinlr{\resizebox{0.9\linewidth}{!}{\psset{xunit=0.1cm,yunit=10.0cm,algebraic=true,dimen=middle,dotstyle=o,dotsize=5pt 0,linewidth=0.8pt,arrowsize=3pt 2,arrowinset=0.25}
\begin{pspicture*}(41.30981178490294,-0.06843236608355657)(173.93575390877118,1.2559528422129693)
\multips(0,0)(0,0.1){14}{\psline[linestyle=dashed,linecap=1,dash=1.5pt 1.5pt,linewidth=0.4pt,linecolor=lightgray]{c-c}(41.30981178490294,0)(173.93575390877118,0)}
\multips(40,0)(10.0,0){14}{\psline[linestyle=dashed,linecap=1,dash=1.5pt 1.5pt,linewidth=0.4pt,linecolor=lightgray]{c-c}(0,-0.06843236608355657)(0,1.2559528422129693)}
\psaxes[labelFontSize=\scriptstyle,xAxis=true,yAxis=true,Ox=50,Dx=10.,Dy=0.1,ticksize=-2pt 0,subticks=0]{->}(50,0)(41.30981178490294,-0.06843236608355657)(173.93575390877118,1.2559528422129693)
\psplot[linewidth=1.2pt,plotpoints=200]{50}{173.93575390877118}{2.0833333333333553E-7*x^(3.0)+2.0833333333325842E-6*x^(2.0)-4.583333333332503E-4*x+0.26666666666666367}
\rput[tl](119.0744204197815,0.5426711944212202){A}
\rput[tl](153.16301598575566,0.05536031659104452){v in km/h}
\rput[tl](52.85022174213378,1.1734243870965686){A in g/km}
\begin{scriptsize}
\psdots[dotstyle=*](80.,0.35)
\psdots[dotstyle=*](100.,0.45)
\psdots[dotstyle=*](130.,0.7)
\psdots[dotstyle=*](160.,1.1)
\end{scriptsize}
\end{pspicture*}}

\scriptsize{Abbildung 1}}{\resizebox{1\linewidth}{!}{\psset{xunit=1.0cm,yunit=0.1cm,algebraic=true,dimen=middle,dotstyle=o,dotsize=5pt 0,linewidth=0.8pt,arrowsize=3pt 2,arrowinset=0.25}
\begin{pspicture*}(-1.0428663167474181,-8.170837668794338)(11.388098799340856,107.06742802791399)
\multips(0,0)(0,10.0){12}{\psline[linestyle=dashed,linecap=1,dash=1.5pt 1.5pt,linewidth=0.4pt,linecolor=black!60]{c-c}(0,0)(11.388098799340856,0)}
\multips(0,0)(100.0,0){1}{\psline[linestyle=dashed,linecap=1,dash=1.5pt 1.5pt,linewidth=0.4pt,linecolor=black!60]{c-c}(0,0)(0,107.06742802791399)}
\psaxes[labelFontSize=\scriptstyle,xAxis=true,yAxis=true,labels=y,ylabelFactor={\%},Dx=1.,Dy=10
,ticksize=-2pt 0,subticks=2](0,0)(0.,0.)(11.388098799340856,107.06742802791399)
\pspolygon[fillcolor=black!75,fillstyle=solid,opacity=1](1.,100.)(3.,100.)(3.,92.)(1.,92.)
\pspolygon[fillcolor=black!50,fillstyle=solid,opacity=1](1.,76.)(1.,92.)(3.,92.)(3.,76.)
\pspolygon[fillcolor=black!15,fillstyle=solid,opacity=1](1.,76.)(1.,0.)(3.,0.)(3.,76.)
\pspolygon[fillcolor=black!75,fillstyle=solid,opacity=1](4.,100.)(4.,89.)(6.,89.)(6.,100.)
\pspolygon[fillcolor=black!50,fillstyle=solid,opacity=1](4.,89.)(4.,35.)(6.,35.)(6.,89.)
\pspolygon[fillcolor=black!15,fillstyle=solid,opacity=1](4.,35.)(4.,0.)(6.,0.)(6.,35.)
\pspolygon[fillcolor=black!15,fillstyle=solid,opacity=1](7.,0.)(9.,0.)(9.,60.)(7.,60.)
\pspolygon[fillcolor=black!50,fillstyle=solid,opacity=1](7.,60.)(7.,83.)(9.,83.)(9.,60.)
\pspolygon[fillcolor=black!75,fillstyle=solid,opacity=1](7.,83.)(7.,100.)(9.,100.)(9.,83.)
\pspolygon[fillcolor=black!50,fillstyle=solid,opacity=1](9.75,72.5)(10.,72.5)(10.,70.)(9.75,70.)
\pspolygon[fillcolor=black!15,fillstyle=solid,opacity=1](9.75,67.5)(9.75,65.)(10.,65.)(10.,67.5)
\pspolygon[fillcolor=black!75,fillstyle=solid,opacity=1](9.75,77.5)(9.75,75.)(10.,75.)(10.,77.5)
\psline(1.,100.)(3.,100.)
\psline(3.,100.)(3.,92.)
\psline(3.,92.)(1.,92.)
\psline(1.,92.)(1.,100.)
\psline(1.,76.)(1.,92.)
\psline(1.,92.)(3.,92.)
\psline(3.,92.)(3.,76.)
\psline(3.,76.)(1.,76.)
\psline(1.,76.)(1.,0.)
\psline(1.,0.)(3.,0.)
\psline(3.,0.)(3.,76.)
\psline(3.,76.)(1.,76.)
\psline(4.,100.)(4.,89.)
\psline(4.,89.)(6.,89.)
\psline(6.,89.)(6.,100.)
\psline(6.,100.)(4.,100.)
\psline(4.,89.)(4.,35.)
\psline(4.,35.)(6.,35.)
\psline(6.,35.)(6.,89.)
\psline(6.,89.)(4.,89.)
\psline(4.,35.)(4.,0.)
\psline(4.,0.)(6.,0.)
\psline(6.,0.)(6.,35.)
\psline(6.,35.)(4.,35.)
\psline(7.,0.)(9.,0.)
\psline(9.,0.)(9.,60.)
\psline(9.,60.)(7.,60.)
\psline(7.,60.)(7.,0.)
\psline(7.,60.)(7.,83.)
\psline(7.,83.)(9.,83.)
\psline(9.,83.)(9.,60.)
\psline(9.,60.)(7.,60.)
\psline(7.,83.)(7.,100.)
\psline(7.,100.)(9.,100.)
\psline(9.,100.)(9.,83.)
\psline(9.,83.)(7.,83.)
\rput[tl](1.8,84){16}
\rput[tl](4.8,62){54}
\rput[tl](7.8,71.5){23}
\rput[tl](7.8,30){60}
\rput[tl](4.8,17.5){35}
\rput[tl](1.8,38){76}
\psline(9.75,72.5)(10.,72.5)
\psline(10.,72.5)(10.,70.)
\psline(10.,70.)(9.75,70.)
\psline(9.75,70.)(9.75,72.5)
\psline(9.75,67.5)(9.75,65.)
\psline(9.75,65.)(10.,65.)
\psline(10.,65.)(10.,67.5)
\psline(10.,67.5)(9.75,67.5)
\psline(9.75,77.5)(9.75,75.)
\psline(9.75,75.)(10.,75.)
\psline(10.,75.)(10.,77.5)
\psline(10.,77.5)(9.75,77.5)
\begin{scriptsize}
\rput[tl](10.1,77.2){sonstige}
\rput[tl](10.1,72.2){LKW}
\rput[tl](10.1,67.2){PKW}
\rput[tl](1.6074770023635618,-2.3423126696833667){Verkehr}
\rput[tl](4.787888985296738,-3.0084298124389064){$\text{NO}_x$}
\rput[tl](7.797311076674367,-2.009254098305597){PM10}
\end{scriptsize}
\end{pspicture*}}

\scriptsize{Abbildung 2}

\tiny{Quelle: http://www.tirol.gv.at/themen/verkehr/verkehrsplanung/ 
verkehrsprojekte/tempo100}}
				
\subsection{Aufgabenstellung:}
\begin{enumerate}
	\item Ermittle anhand der Messwerte in Abbildung 1, um wie viele Prozent der Sickoxid-Ausstoß eines PWK abnimmt, wenn statt der sonst erlaubten 130\,km/h nur mit einer Geschwindigkeit von 100\,km/h gefahren werden darf!
	
Ist der Stickoxid-Ausstoß eines PKW direkt proportional zur Fahrgeschwindigkeit? Begründe deine Antwort anhand des Graphen der Modellfunktion $A$ in Abbildung 1.

\item Verursachen im Tiroler Unterinntal die Verkehrsmittel mit dem größten Anteil am Verkehrsaufkommen auch die meisten Stickoxid- bzw. Feinstaub-Emissionen? Begründe deine Antwort!

Geschwindigkeitsmessungen auf der Autobahn A12 im Tiroler Unterinntal haben gezeigt, dass die Geschwindigkeitslimits von mehr als 90\,\% der Verkehrsteilnehmer/innen eingehalten werden und weniger als 1\,\% der Verkehrsteilnehmer/innen die Geschwindigkeitslimits um mehr als 10\,\% überschreiten. Die Geschwindigkeitsüberschreitungen können daher für die
 folgende Fragestellung vernachlässigt werden.

Begründe, welche der beiden Maßnahmen (A oder B) wirkungsvoller ist, wenn entlang der A12 die Stickoxid-Emissionen weiter reduziert werden sollten! Durch Maßnahme A eventuell anfallende zusätzliche Emissionen durch die Bahn werden vernachlässigt.

\hspace*{1cm}A\hspace*{1cm} eine Verlagerung der Hälfte des Gütertransports durch LKW \hspace*{2.3cm} auf die Schiene (d. h. Transport der LKW mit der Bahn)

\hspace*{1cm}B\hspace*{1cm} ein Tempolimit von 80 km/h für PKW und LKW

Entnehme die für die Begründung benötigten Werte den Abbildungen 1 und 2 und 
führe diese an!

\item Ermittle rechnerisch anhand von Abbildung 1 das Ergebnis des Ausdrucks $\frac{A(160)-A(100)}{60}$ auf vier Dezimalstellen genau!

Interpretiere das Ergebnis dieses Ausdrucks im Hinblick auf die $\text{NO}_x$Emissionen!

\item Zur Modellierung der in Abbildung 1 dargestellten Abhängigkeit des $\text{NO}_x$-Ausstoßes $A$ von der Fahrgeschwindigkeit $v$ kommen unterschiedliche Funktionstypen in Frage.

Welche Funktionstypen können zur Modellierung der Funktion $A$ verwendet worden sein? Kreuze die beiden geeigneten Funktionsgleichungen an!

\multiplechoice[5]{  %Anzahl der Antwortmoeglichkeiten, Standard: 5
				L1={$A(v)=a\cdot v+b$ mit $a>0,b>0$},   %1. Antwortmoeglichkeit 
				L2={$A(v)=a\cdot v²+b$ mit $a<0,b>0$},   %2. Antwortmoeglichkeit
				L3={$A(v)=a\cdot v²+b\cdot v+c$ mit $a>0,c>0,b\in\mathbb{R}$},   %3. Antwortmoeglichkeit
				L4={$A(v)=a\cdot b^v$ mit $a>0,b>1$},   %4. Antwortmoeglichkeit
				L5={$A(v)=a\cdot b^v$ mit $a>0,b<1$},	 %5. Antwortmoeglichkeit
				L6={},	 %6. Antwortmoeglichkeit
				L7={},	 %7. Antwortmoeglichkeit
				L8={},	 %8. Antwortmoeglichkeit
				L9={},	 %9. Antwortmoeglichkeit
				%% LOESUNG: %%
				A1=3,  % 1. Antwort
				A2=4,	 % 2. Antwort
				A3=0,  % 3. Antwort
				A4=0,  % 4. Antwort
				A5=0,  % 5. Antwort
				}
				
				Begründe, warum die drei restlichen Funktionsgleichungen für die Modellierung von $A$ in Abbildung 1 nicht geeignet sind!
						
						\end{enumerate}\leer
				
\antwort{\subsection{Lösungserwartung:}
\begin{enumerate}
	\item Richtige Berechnung der Abnahme der $\text{NO}_x$-Emissionen: $\frac{0,5}{0,75}\approx 0,67$.
	
	\textit{Der Stickoxid-Ausstoß nimmt um ungefähr 33\,\% ab.}
	
	Alle Ergebnisse im Intervall [30\,\%;35\,\%] sind als richtig zu werten.
	
	Auch die Antwort, dass die Emissionen bei einer Reduktion der Geschwindigkeit auf 100\,km/h nur mehr 67\,\% des Wertes bei 130\,km/h betragen, ist als richtig zu werden (Lösungsintervall, falls die noch vorhandenen Emissionen angegeben werden: [65\,\%; 70\,\%]).
	
	Zudem muss eine Begründung angegeben sein, dass $A$ nicht direkt proportional zu $v$ ist, z.B.: \textit{Der Stickoxid-Ausstoß ist nicht direkt proportional zur Fahrgeschwindigkeit, weil der Graph von $A$ nicht linear verläuft.}
	
	Auch andere, aus der Abbildung ableitbare Formulierungen wie z.B. \textit{Nicht direkt proportional, weil sich die Emissionen mehr als verdoppeln, wenn die Geschwindigkeit verdoppelt wird}, aufgrund derer eine direkte Proportionalität ausgeschlossen werden kann, sind als richtig zu werten.
	
	\item Die Antwort ist als richtig zu werten, wenn sinngemäß begründet ist, dass PKW zwar den größten Anteil an den Feinstaub-Emissionen besitzen, bei den Stickoxid-Emissionen aber die LKW die Hauptverursacher sind, z. B.: Im Tiroler Unterinntal haben PKW mit 76\,\% den größten Anteil am Verkehrsaufkommen. Sie verursachen mit 60\,\% zwar den größten Anteil der Feinstaub-Emissionen, aber nur 35\,\% der Stickoxid-Emissionen. Anmerkung: Die Zahlenwerte müssen nicht angeführt sein.
	
	Zudem muss eine schlüssige Begründung angegeben werden, dass Maßnahme 
$A$ wirkungsvoller für eine Stickoxid.Reduktion ist, z. B.: LKW verursachen 54\,\% der $\text{NO}_x$-Emissionen im Straßenverkehr, obwohl ihr Anteil am Verkehrsaufkommen nur 16\,\% beträgt. Eine Reduktion des LKW-Verkehrs auf die Hälfte würde die $\text{NO}_x$-Emissionen um ca. 27\,\% reduzieren.

PKW haben zwar einen Anteil von 76\,\% am Verkehrsaufkommen, sind aber nur für 35\,\% der $\text{NO}_x$-Emissionen verantwortlich. Durch eine Reduktion des Tempolimits von 130\,km/h auf 80\,km/h könnten laut Abbildung 1 maximal die Hälfte dieser Emissionen, also etwa 17\,\%, vermieden werden. Eine Verlagerung der Hälfte des LKW-Verkehrs auf die Schiene wäre daher die wirkungsvollere Maßnahme zur Reduktion der $\text{NO}_x$-Emissionen.

Anmerkung: Auch eine Begründung mit gerundeten relativen Anteilen (drei Viertel
 etc.) ist als richtig zu werten.
	
	\item Richtige Berechnung des Differenzenquotienten: $\frac{A(160)-A(100)}{60}=\frac{1,05-0,5}{60}\approx 0,0092$, wobei Ergebnisse aus dem Intervall [0,0088;0,0095] als richtig zu werten sind. Die Angabe der Einheit ist nicht erforderlich.
	
	Zudem muss der Differenzenquotient richtig interpretiert werden, z.B.: \textit{Wenn die Geschwindigkeit von 100\,km/h auf 160\,km/h erhöht wird, beträgt die mittlere Zunahme der $\text{NO}_x$-Emissionen 0,0092\,g/km pro km/h}.
	
	Auch analoge Formulierungen wie z.B. \textit{mittlere Änderungsrate des Stickoxid-Ausstoßes} sind als richtig zu werten. Das Geschwindigkeisintervall [100\,km/h; 160\,km/h] muss in der Interpretation in irgendeiner Form vorkommen.
	
	\item Lösung Multiple Choice: siehe oben.
	
	Zudem müssen drei sinngemäß richtige Begründungen angegeben sein, warum die restlichen Funktionsgleichungen für die Modellierung nicht geeignet sind, z.B.:
	
	\textit{Der Graph von $A(v)=a\cdot v+b$ ist linear und daher nicht geeignet.}
	
	\textit{Der Graph von $A(v)=a\cdot v²+b$ mit $a<0,b>0$ ist eine nach unten geöffnete Parabel und daher nicht geeignet.}
	
	\textit{Der Graph von $A(v)=a\cdot b^v$ mit $a>0,b<1$ ist fallend und daher nicht geeignet.}
	
		\end{enumerate}}
\end{langesbeispiel}