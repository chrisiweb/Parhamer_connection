\section{138 - K8 - AN 1.4, FA 2.1, FA 5.1, FA 5.2, FA 5.3, FA 5.5 - Koffein - VerSie}

\begin{langesbeispiel} \item[6] %PUNKTE DES BEISPIELS
Koffein ist ein Purin-Alkaloid und ein anregend wirkender Bestandteil von Genussmitteln wie Kaffee, Tee, Cola, Mate, Guaraná, Energy-Drinks und Kakao. Es ist eines der ältesten von Menschen genutzten Stimulanzien. 

Die biologische Halbwertszeit von Koffein im Plasma beträgt zwischen 2,5 und 4,5 Stunden  bei gesunden Erwachsenen. Bei Rauchern reduziert sich die Koffein-Halbwertszeit um 30-50\,\%, während sie sich bei Frauen, die orale Verhütungsmittel einnehmen, verdoppelt.

In der folgenden Tabelle ist der Koffeingehalt in einigen Lebensmitteln angegeben:
\begin{center}
\begin{tabular}{|l|l|}\hline
Koffeingehalt von&\\
ausgewählten Lebensmitteln&\\ \hline
Tasse Filterkaffee&100\,mg\\ \hline
Tasse Espresso&40\,mg\\ \hline
Tasse Schwarztee&20\,mg\\ \hline
1 Tafel Bitterschokolade&90\,mg\\ \hline
Dose Red Bull&96\,mg\\ \hline
0,5\,l Cola&50\,mg\\ \hline
Koffeintablette&200\,mg\\ \hline
\end{tabular}
\end{center}%Aufgabentext

\begin{aufgabenstellung}
\item Für Tom liegt die Halbwertszeit des Koffeinabbaus bei 3,5 Stunden.%Aufgabentext

\Subitem{Modelliere den Abbau mittels Formel $N(t)=N(0)\cdot e^{-\lambda\cdot t}$ und zeige, dass\\
	 $\lambda=0,198$ ist.} %Unterpunkt1
\Subitem{Tom konsumiert zwei Dosen Red Bull und isst eine Tafel Bitterschokolade. Nach wie vielen Stunden sind nur noch 10\,mg Koffein in seinem Körper nachweisbar?} %Unterpunkt2

\item %Aufgabentext

\ASubitem{ Zeichne in die untenstehende Grafik den Verlauf des Abbaus von Koffein bei Sarah ein, die Koffein mit der Halbwertszeit zwei Stunden abbaut, nachdem sie eine Tasse Filterkaffee konsumiert hat.
	
	\begin{center}
	\psset{xunit=1.3cm,yunit=0.06cm,algebraic=true,dimen=middle,dotstyle=o,dotsize=5pt 0,linewidth=1.6pt,arrowsize=3pt 2,arrowinset=0.25}
\begin{pspicture*}(-0.76,-10)(10.88,116.15853658538416)
\multips(0,0)(0,10.0){13}{\psline[linestyle=dashed,linecap=1,dash=1.5pt 1.5pt,linewidth=0.4pt,linecolor=gray]{c-c}(0,0)(10.88,0)}
\multips(0,0)(1.0,0){12}{\psline[linestyle=dashed,linecap=1,dash=1.5pt 1.5pt,linewidth=0.4pt,linecolor=gray]{c-c}(0,0)(0,116.15853658538416)}
\psaxes[labelFontSize=\scriptstyle,xAxis=true,yAxis=true,Dx=1.,Dy=10.,ticksize=-2pt 0,subticks=0]{->}(0,0)(-0.76,-4.756097560976728)(10.88,116.15853658538416)[$t$ in h,140] [$N(t)$ in mg,-40]
\antwort{\psplot[linewidth=2.pt,plotpoints=200]{0}{10.880000000000004}{100.0*2.718281828459045^((-8.664339757E9)/2.5E10*x)}}
\end{pspicture*}
	\end{center}} %Unterpunkt1
	\setcounter{subitemcounter}{1}
	Angenommen Sarah würde Koffein linear abbauen. Zwei Stunden nach dem Konsum der Tasse Filterkaffee hat sie noch 70 mg Koffein im Körper. 
\Subitem{Stelle die Funktionsgleichung für diesen linearen Abbau auf und gib an, zu welchem Zeitpunkt in Sarahs Körper kein Koffein mehr nachweisbar ist.} %Unterpunkt2

\item In Sarahs Körper wird das aufgenommene Koffein nach folgender Formel abgebaut:
	$$N(t)=N_0\cdot 0,40^t$$
	
	$N(t)\ldots$ Koffeinmenge in mg zum Zeitpunkt $t$\\
	$N_0\ldots$ Koffeinmenge in mg zum Zeitpunkt $t=0$\\
	$t\ldots$ Zeit in Stunden%Aufgabentext

\Subitem{Gib wie viel Prozent des ursprünglich aufgenommenen Koffeins nach einer Stunde im Körper abgebaut sind.} %Unterpunkt1
\Subitem{Stelle den Abbau $A(t+1,t)$ mit Hilfe einer Differenzengleichung dar.} %Unterpunkt2

\end{aufgabenstellung}

\begin{loesung}
\item \subsection{Lösungserwartung:} 

\Subitem{$0,5\cdot N(0)=N(0)\cdot e^{-\lambda\cdot 3,5} \Rightarrow 0,5=e^{-\lambda\cdot 3,5} \Rightarrow \ln(0,5)=-\lambda\cdot 3,5 \Rightarrow$\\
	 $\lambda=0,1980420515886$} %Lösung von Unterpunkt1
\Subitem{$10=(2\cdot 96+90)\cdot e^{-0,1980420515886\cdot t} \Rightarrow 10=282\cdot e^{-0,1980420515886\cdot t} \Rightarrow$\\
	$t=16.86168140129$
	
	Nach 16,9 Stunden sind nur noch 10\,mg vorhanden.} %%Lösung von Unterpunkt2

\setcounter{subitemcounter}{0}
\subsection{Lösungsschlüssel:}
 
\Subitem{Ein Punkt für die richtige Berechnung von $\lambda$.} %Lösungschlüssel von Unterpunkt1
\Subitem{Ein Punkt für den richtige Zeitdauer.} %Lösungschlüssel von Unterpunkt2

\item \subsection{Lösungserwartung:} 

\Subitem{Graph des Abbaus von Sarah: siehe oben.} %Lösung von Unterpunkt1
\Subitem{Bei linearem Abbau: $N(t)=-15\cdot t+100$
	
	Nach $6,\dot{6}$ Stunden wäre kein Koffein mehr vorhanden.} %%Lösung von Unterpunkt2

\setcounter{subitemcounter}{0}
\subsection{Lösungsschlüssel:}
 
\Subitem{Ein Punkt für den Graphen.} %Lösungschlüssel von Unterpunkt1
\Subitem{Ein Punkt für den Zeitpunkt.} %Lösungschlüssel von Unterpunkt2

\item \subsection{Lösungserwartung:} 

\Subitem{Nach einer Stunde wurden bereits 60\,\% des ursprünglich aufgenommenen Koffeeins abgebaut.} %Lösung von Unterpunkt1
\Subitem{•} %%Lösung von Unterpunkt2

\setcounter{subitemcounter}{0}
\subsection{Lösungsschlüssel:}
 
\Subitem{Ein Punkt für die korrekte Angabe des Abbaus in Prozent.} %Lösungschlüssel von Unterpunkt1
\Subitem{Ein Punkt für die Differenzengleichung.} %Lösungschlüssel von Unterpunkt2

\end{loesung}

\end{langesbeispiel}