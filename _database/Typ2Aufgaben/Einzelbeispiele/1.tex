\section{01 - MAT - AN 1.1, AN 1.3, AN 3.3 - Wasserstand eines Bergsees - BIFIE Aufgabensammlung}

\begin{langesbeispiel} \item[0] %PUNKTE DES BEISPIELS
				Die Funktion $h$ beschreibt die Wasserh�he eines Bergsees in Abh�ngigkeit von der Zeit $t$. Die nachstehende Abbildung zeigt den Graphen der Funktion $h$.
				
				\begin{center}\resizebox{1\linewidth}{!}{\newrgbcolor{ttqqqq}{0.2 0. 0.}
\psset{xunit=1.0cm,yunit=1.0cm,algebraic=true,dimen=middle,dotstyle=o,dotsize=5pt 0,linewidth=0.8pt,arrowsize=3pt 2,arrowinset=0.25}
\begin{pspicture*}(-1.08,-2.4)(9.15,5.98)
\multips(0,-2)(0,1.0){9}{\psline[linestyle=dashed,linecap=1,dash=1.5pt 1.5pt,linewidth=0.4pt,linecolor=lightgray]{c-c}(0,0)(9.15,0)}
\multips(0,0)(1.0,0){11}{\psline[linestyle=dashed,linecap=1,dash=1.5pt 1.5pt,linewidth=0.4pt,linecolor=lightgray]{c-c}(0,-2.4)(0,5.98)}
\psaxes[labelFontSize=\scriptstyle,xAxis=true,yAxis=true,Dx=1.,Dy=1.,ticksize=-2pt 0,subticks=2]{->}(0,0)(0.,-2.2)(9.15,5.98)
\psplot[linewidth=1.2pt,plotpoints=200]{0}{9.15}{2.2686248826559873E-4*x^(7.0)-0.00614949610902518*x^(6.0)+0.058139356739772344*x^(5.0)-0.21026452712641674*x^(4.0)+0.1312581165522796*x^(3.0)+0.4502153007836318*x^(2.0)-0.19131188236159785*x+2.6}
\rput[tl](0.28,5.5){Wasserh�he h (in mm)}
\rput[tl](5.82,0.62){Zeit t (in Wochen)}
\begin{scriptsize}
\psdots[dotstyle=*,linecolor=ttqqqq](0.,2.6)
\rput[bl](0.2,2.14){\ttqqqq{$A = (0/2,6)$}}
\psdots[dotstyle=*,linecolor=ttqqqq](2.,3.2)
\rput[bl](2.08,3.4){\ttqqqq{$B = (2/3,2)$}}
\psdots[dotstyle=*,linecolor=ttqqqq](3.6,2.1)
\rput[bl](3.68,2.3){\ttqqqq{$C = (3,6/2,1)$}}
\psdots[dotstyle=*,linecolor=ttqqqq](4.9,1.2)
\rput[bl](4.98,0.72){\ttqqqq{$D = (4,9/1,2)$}}
\psdots[dotstyle=*,linecolor=ttqqqq](6.5,3.1)
\rput[bl](6.5,2.6){\ttqqqq{$E = (6,5/3,1)$}}
\psdots[dotstyle=*,linecolor=ttqqqq](7.9,4.7)
\rput[bl](7.32,4.78){\ttqqqq{$F = (7,9/4,7)$}}
\rput[bl](6.48,3.92){$h$}
\end{scriptsize}
\end{pspicture*}}\end{center}

\subsection{Aufgabenstellung:}
\begin{enumerate}
	\item Bestimme den Wert des Differenzenquotienten des Wasserstand
$s$ im Intervall $[0; 2]$ und beschreibe in Worten, was dieser Wert angibt! Um wie viel Prozent ist die Wasserh�he w�hrend der ersten zwei Wochen gestiegen.
	\item Was beschreibt die erste Ableitungsfunktion $h'$ der Funktion $h$? Betimme n�herungsweise den Wert des Differenzialquotienten der Wasserh�he zum Zeitpunkt $t=6$ und beschreibe in Worten, was dieser Wert angibt!
	\item Was beschreibt die zweite Ableitungsfunktion $h''$ der Funktion $h$? Wann etwa nimmt die Wasserh�he am st�rksten zu?
\end{enumerate}

\antwort{\subsection{L�sungserwartung:}
\begin{enumerate}
	\item $\dfrac{3,2-2,6}{2-0}=0,3 \rightarrow$ Bis zum Ende der zweiten Woche nimmt die Wasserh�he im Mittel pro Woche um 0,3 $m$ zu.
	
	$3,2:2,6\approx 1,23 \rightarrow$ Der Wasserstand nahm um ca. $23\,\%$ zu.
	\item Durch die erste Ableitungsfunktion $h'$ ist die �nderungsgeschwindigkeit der Wasserh�he bestimmt.
	
	$h'(6)\approx 1,6 \rightarrow$ Das bedeutet, dass nach sechs Wochen die momentane �nderungsrate $1,6\,mm$ pro Woche betr�gt.
	\item Die zweite Ableitungsfunktion $h''$ beschreibt das Monotonieverhalten der �nderungsrate der Wasserh�he bzw. die momentane �nderungsrate der �nderungsgeschwindigkeit der Wasserh�he. Die Wasserh�he nimmt nach ca. 6,5 Wochen am st�rksten zu.
\end{enumerate}}
\end{langesbeispiel}