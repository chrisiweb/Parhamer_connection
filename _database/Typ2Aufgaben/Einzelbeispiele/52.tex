\section{52 - MAT - WS 2.3, WS 4.1 - FSME-Impfung - Matura 2014/15 1. Nebentermin}

\begin{langesbeispiel} \item[0] %PUNKTE DES BEISPIELS
				
	Die Frühsommer-Meningoenzephalitis (FSME) ist eine durch das FSME-Virus ausgelöste Erkrankung, die mit grippeähnlichen Symptomen und bei einem Teil der Patientinnen und Patienten mit einer Entzündung von Gehirn und Hirnhäuten verläuft. Die FSME wird durch den Biss einer infizierten Zecke übertragen, wobei die Übertragungswahrscheinlichkeit bei einem Biss 30\,\% beträgt. Nur bei 10\,\% bis 30\,\% der mit dem FSME-Virus infizierten Personen treten Krankheitserscheinungen auf. Im Durchschnitt verläuft 1\,\% der Erkrankungen tödlich. In Risikogebieten liegt der Anteil der FSME-infizierten Zecken bei etwa 0,5\,\% bis 5\,\%, während man sonst davon ausgeht, dass nur jede 20000. Zecke das FSME-Virus in sich trägt.


\subsection{Aufgabenstellung:}
\begin{enumerate}
	\item Eine nicht geimpfte Person wird in einem Risikogebiet von einer Zecke gebissen.\leer
	
	Gib die Wahrscheinlichkeit, dass diese Person Krankheitserscheinungen zeigt, in Prozent an; gehe dazu bei den angegebenen Wahrscheinlichkeiten immer von dem Fall aus, der für die gebissene Person am ungünstigsten ist!\leer
	
	Gib an, mit welchem Faktor sich die berechnete Wahrscheinlichkeit für eine FSME-Erkrankung verändert, wenn der Zeckenbiss nicht in einem Risikogebiet erfolgt ist!\leer
	
\item \fbox{A} Im Jahr 2011 gab es in Österreich vier FSME-bedingte Todesfälle. Waren dies weniger oder mehr Todesfälle, als bei 113 Erkrankungen zu erwarten waren? Begründe deine Antwort auf Basis der gegebenen Daten.\leer

In einem österreichischen Risikogebiet nahmen 400 Personen an einer Umfrage teil. Es wird angenommen, dass die Personen, die an dieser Umfrage teilnahmen, eine Zufallsstichprobe darstellen. Von den befragten Personen gaben 64 Personen an, schon einmal von einer  Zecke gebissen worden zu sein. 

Berechne auf Basis dieses Umfrageergebnisses ein symmetrisches 95-\%-Konfidenzintervall für den tatsächlichen (relativen) Anteil $p$ der Personen in diesem Gebiet, die schon einmal von einer Zecke gebissen worden sind!

						\end{enumerate}\leer
				
\antwort{
\begin{enumerate}
	\item \subsection{Lösungserwartung:} 
	
	$0,05\cdot 0,3\cdot 0,3=0,0045$
	
	Die Wahrscheinlichkeit beträgt $0,45\,\%$.\leer
	
	In einem Risikogebiet ist schlimmstenfalls jede zwanzigste Zecke mit FSME infiziert, d.h., der Anteil infizierter Zecken ist bis zu 1\,000-mal höher als in einem Nichtrisikogebiet. Daher ändert sich die berechnete Wahrscheinlichkeit für eine FSME-Erkrankung um den Faktor $\frac{1}{1\,000}$.
		
	\subsection{Lösungsschlüssel:}
	\begin{itemize}
		\item Ein Punkt für die richtige Lösung. Äquivalente Schreibweisen des Ergebnisses (als Bruch oder Dezimalzahl) sind ebenfalls als richtig zu werten.  
		
		Toleranzintervalle: $[0,4\,\%; 0,5\,\%]$ bzw. $[0,004; 0,005]$
		\item  Ein Punkt für die richtige Lösung sowie eine (sinngemäß) korrekte Begründung. Bei einer entsprechenden (sinngemäß) korrekten Begründung ist der Faktor 1\,000 ebenfalls als richtig zu werten.
	\end{itemize}
	
	\item \subsection{Lösungserwartung:}
			
		Da im Durchschnitt $1\,\%$ der Erkrankungen tödlich verlaufen, war nur ein Todesfall (1\,\% von 113) zu erwarten. Vier Todesfälle sind aher mehr, als zu erwarten war.\leer
		
		Mögliche Berechnung:
		
		$n=400, h=0,16$
		
		$2\cdot\Phi(z)-1=0,95 \Rightarrow z=1,96$
		
		$$h\,\pm\,z\cdot\sqrt{\frac{h\cdot(1-h)}{n}}=0,16\,\pm\, 1,96\cdot\sqrt{\frac{0,16\cdot(1-0,16)}{400}}\approx 0,16\,\pm\, 0,036 \Rightarrow [0,124;0,196]$$

	\subsection{Lösungsschlüssel:}
	
\begin{itemize}
	\item Ein Ausgleichspunkt für die richtige Antwort sowie eine (sinngemäß) korrekte Begründung.
	\item Ein Punkt für ein korrektes Intervall. Andere Schreibweisen des Ergebnisses (als Bruch oder Dezimalzahl) sind ebenfalls als richtig zu werten. 
	
	Toleranzintervall für den unteren Wert: $[0,12; 0,13]$ 
	
	Toleranzintervall für den oberen Wert: $[0,19; 0,2]$ 
	
	Die Aufgabe ist auch dann als richtig gelöst zu werten, wenn bei korrektem Ansatz das Ergebnis aufgrund eines Rechenfehlers nicht richtig ist
\end{itemize}

\end{enumerate}}
		\end{langesbeispiel}