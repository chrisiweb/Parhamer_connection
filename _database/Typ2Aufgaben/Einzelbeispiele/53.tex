\section{53 - MAT - AG 2.3, FA 4.3, AN 4.2 - Quadratische Gleichungen und ihre Lösungen - Matura 2014/15 2. Nebentermin}

\begin{langesbeispiel} \item[0] %PUNKTE DES BEISPIELS
				
	Gegeben sind eine (normierte) quadratische Gleichung $x²+p\cdot x+q=0$ mit $p,q\in\mathbb{R}$ und die zugehörige Polynomfunktion $f$ mit $f(x)=x²+p\cdot x+q$.

\subsection{Aufgabenstellung:}
\begin{enumerate}
	\item Lässt sich die Gleichung $x²+p\cdot x+q=0$ in der Form $\left(x-z\right)\cdot\left(x-\frac{1}{z}\right)=0$ mit $z\in\mathbb{R}$ und $z\neq 0$ schreiben, dann spricht man von einer reziproken quadratischen Gleichung.\leer
	
	Gib mithilfe von Gleichungen an, wie die Parameter $p$ und $q$ jeweils von $z$ abhängen!\leer
	
	Bestimme die Werte für $z$, für die die reziproke quadratische Gleichung genau eine Lösung besitzt. Gib für jeden dieser Werte von $z$ jeweils die lokalen Minimumstellen von $f$ an!\leer
	
\item Wählt man in der gegebenen Funktionsgleichung den Wert $q=-1$, dann erhält man eine Polynomfunktion zweiten Grades $f$ mit $f(x)=x²+p\cdot x-1$.\leer

\fbox{A} Begründe rechnerisch, warum die Gleichung $f(x)=0$ genau zwei verschiedene Lösungen in $\mathbb{R}$ haben muss!\leer

Begründe, warum die Funktion $f$ eine positive und eine negative Nullstelle haben muss!\leer

\item Für $q=p-\frac{1}{3}$ erhält man eine Funktion $f$ mit $f(x)=x²+p\cdot x+p-\frac{1}{3}$.\leer

Bestimme für diese Funktion $f$ denjenigen Wert für $p$, für den $\int^{1}_{-1}{f(x)}$d$x=-6$ gilt!\leer

Gib an, ob für dieses $p$ die Gleichung $\int^0_{-1}{f(x)}$d$x=\int^1_0{f(x)}$d$x$ eine wahre Aussage ergibt, und begründe deine Entscheidung!
						\end{enumerate}\leer
				
\antwort{
\begin{enumerate}
	\item \subsection{Lösungserwartung:} 
	
	$p=-\left(\frac{1}{z}+z\right)$
	
	$q=\frac{1}{z}\cdot z=1$, $q$ ist somit unabhängig von $z$\leer
	
	Die reziproke quadratische Gleichung hat genau eine Lösung, wenn $z=1$ oder $z=-1$ ist.  
	
	Minimumstelle für $z=1$: bei $x=1$ 
	
	Minimumstelle für $z=-1$: bei $x=-1$
		
	\subsection{Lösungsschlüssel:}
	\begin{itemize}
		\item Ein Punkt für die korrekte Angabe beider Parameter $p$ und $q$ in Abhängigkeit von $z$.  Äquivalente Gleichungen sind ebenfalls als richtig zu werten.
		\item  Ein Punkt für die Angabe der richtigen Werte von $z$ und der jeweils richtigen Minimumstelle.
	\end{itemize}
	
	\item \subsection{Lösungserwartung:}
			
		$x²+p\cdot x-1=0 \Rightarrow x_{1,2}=-\frac{p}{2}\,\pm\,\sqrt{\frac{p²}{4}+1}$
		
		Da der Ausdruck $\frac{p²}{4}$ für alle $p\in\mathbb{R}$ größer oder gleich null ist, ist der Ausdruck unter der Wurzel (die Diskriminante) positiv. Somit gibt es genau zwei Lösungen in $\mathbb{R}$.\leer
		
		Mögliche Begrüdungen:\leer
		
		Jeder mögliche Funktionsgraph von $f$ verläuft durch den Punkt $(0|-1)$ und ist eine nach oben offene Parabel. Somit hat jede Funktion $f$ genau eine positive und eine negative Nullstelle. Diese Werte entsprechen genau den Lösungen der quadratischen Gleichung $f(x)=0$.\leer
		
		oder:\leer
		
		Es gilt: $\frac{p²}{4}+1>\frac{p²}{4}$ und somit auch $\sqrt{\frac{p²}{4}+1}>\frac{p}{2}$.
		
		Daraus folgt: $-\frac{p}{2}+\sqrt{\frac{p²}{4}+1}>0$ und $-\frac{p}{2}-\sqrt{\frac{p²}{4}+1}<0 \Rightarrow$ Es gibt immer genau eine positive und eine negative Lösung in $\mathbb{R}$.

	\subsection{Lösungsschlüssel:}
	
\begin{itemize}
	\item Ein Ausgleichspunkt für eine korrekte rechnerische Begründung. 
	\item EEin Punkt für eine (sinngemäß) korrekte Begründung dafür, dass die Parabel genau eine positive und eine negative Nullstelle hat. 
\end{itemize}

\item \subsection{Lösungserwartung:}
			
		$$\int^1_{-1}{\left(x²+p\cdot x+p-\frac{1}{3}\right)}dx=\left(\frac{x³}{3}+\frac{px²}{2}+px-\frac{x}{3}\right)\Bigg|^1_{-1}=-6 \Rightarrow p=-3$$
		
		Für $p=3$ ergibt die gegebene Gleichung keine wahre Aussage, weil für eine solche Funktion der Graph der Funktion $f$ achsensymmetrisch liegen müsste, damit die Gleichung eine wahre Aussage ergibt, und das ist nur für $p=0$ der Fall.

	\subsection{Lösungsschlüssel:}
	
\begin{itemize}
	\item Ein Punkt für den korrekten Wert von $p$. Die Aufgabe ist auch dann als richtig gelöst zu werten, wenn bei korrektem Ansatz das Ergebnis aufgrund eines Rechenfehlers nicht richtig ist. 
	\item  Ein Punkt für die korrekte Angabe, ob die Aussage zutrifft, und eine korrekte Begründung. Andere korrekte Begründungen, zum Beispiel durch Rechnung, sind ebenfalls als richtig zu werten. 
\end{itemize}
\end{enumerate}}
		\end{langesbeispiel}