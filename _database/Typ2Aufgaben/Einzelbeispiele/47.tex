\section{47 - MAT - WS 1.1, WS 1.3, WS 3.4, WS 4.1, WS 2.3,  - Blutgruppen - Matura 2014/15 Haupttermin}

\begin{langesbeispiel} \item[0] %PUNKTE DES BEISPIELS
				Die wichtigsten Blutgruppensysteme beim Menschen sind das AB0-System und das Rhesussystem. Es werden dabei die vier Blutgruppen A, B, AB und 0 unterschieden. Je nach Vorliegen eines bestimmten Antikörpers, den man erstmals bei Rhesusaffen entdeckt hat, wird bei jeder Blutgruppe noch zwischen \textit{Rhesus-positiv} (+) und \textit{Rhesus-negativ} (-) unterschieden. A- bedeutet z. B. Blutgruppe A mit Rhesusfaktor negativ. 
				
In den nachstehenden Diagrammen sind die relativen Häufigkeiten der vier Blutgruppen in Österreich und Deutschland und im weltweiten Durchschnitt ohne Berücksichtigung des Rhesusfaktors dargestellt.


\hspace{1,4cm}Österreich\hspace{2,8cm}Deutschland\hspace{3,2cm}weltweit

\begin{scriptsize}
\resizebox{0.3\linewidth}{!}{\kreisdiagramm\begin{tikzpicture}
\pie[color={black!10 ,black!20 , black!30, black!40}, %Farbe
text=inside %Format: inside,pin, legend
]
{41/A , 15/B , 37/C , 7/AB} %Werte
\end{tikzpicture}}\hspace{0,7cm}
\resizebox{0.3\linewidth}{!}{\kreisdiagramm\begin{tikzpicture}
\pie[color={black!10 ,black!20 , black!30, black!40}, %Farbe
text=inside %Format: inside,pin, legend
]
{43/A , 11/B , 41/C , 5/AB} %Werte
\end{tikzpicture}}\hspace{0,7cm}
\resizebox{0.3\linewidth}{!}{\kreisdiagramm\begin{tikzpicture}
\pie[color={black!10 ,black!20 , black!30, black!40}, %Farbe
text=inside %Format: inside,pin, legend
]
{40/A , 11/B , 45/C , 4/AB} %Werte
\end{tikzpicture}}
\end{scriptsize}

Die nachstehende Tabelle enthält die relativen Häufigkeiten der Blutgruppen in Deutschland und Österreich zusätzlich aufgeschlüsselt nach den Rhesusfaktoren.

\begin{center}
\begin{tabular}{|l|c|c|c|c|c|c|c|c|}\hline
Land/Blutgruppe&A+&A-&B+&B-&0+&0-&AB+&AB-\\ \hline
Deutschland&37\,\%&6\,\%&9\,\%&2\,\%&35\,\%&6\,\%&4\,\%&1\,\%\\ \hline
Österreich&33\,\%&8\,\%&12\,\%&3\,\%&30\,\%&7\,\%&6\,\%&1\,\%\\ \hline
\end{tabular}
\end{center}

Aufgrund von Unverträglichkeiten kann für eine Bluttransfusion nicht Blut einer beliebigen Blutgruppe verwendet werden. Jedes Kreuz (X) in der nachstehenden Tabelle bedeutet, dass eine Transfusion vom Spender zum Empfänger möglich ist.

\begin{center}
	\begin{tabular}{l|c|c|c|c|c|c|c|c|}\cline{2-9}
	&\multicolumn{8}{|c|}{\cellcolor[gray]{0.9}Spender}\\ \hline
	\multicolumn{1}{|l|}{\cellcolor[gray]{0.7}Empfänger}&\cellcolor[gray]{0.9}0-&\cellcolor[gray]{0.9}0+&\cellcolor[gray]{0.9}B-&\cellcolor[gray]{0.9}B+&\cellcolor[gray]{0.9}A-&\cellcolor[gray]{0.9}A+&\cellcolor[gray]{0.9}AB-&\cellcolor[gray]{0.9}AB+\\ \hline
	\multicolumn{1}{|l|}{\cellcolor[gray]{0.7}AB+}&X&X&X&X&X&X&X&X\\ \hline
	\multicolumn{1}{|l|}{\cellcolor[gray]{0.7}AB-}&X&&X&&X&&X&\\ \hline
	\multicolumn{1}{|l|}{\cellcolor[gray]{0.7}A+}&X&X&&&X&X&&\\ \hline
	\multicolumn{1}{|l|}{\cellcolor[gray]{0.7}A-}&X&&&&X&&&\\ \hline
	\multicolumn{1}{|l|}{\cellcolor[gray]{0.7}B+}&X&X&X&X&&&&\\ \hline
	\multicolumn{1}{|l|}{\cellcolor[gray]{0.7}B-}&X&&X&&&&&\\ \hline
	\multicolumn{1}{|l|}{\cellcolor[gray]{0.7}0+}&X&X&&&&&&\\ \hline
	\multicolumn{1}{|l|}{\cellcolor[gray]{0.7}0-}&X&&&&&&&\\ \hline
	\end{tabular}
\end{center}
\begin{scriptsize}Datenquelle: https://de.wikipedia.org/wiki/Blutgruppe [26.11.2014]\end{scriptsize}

\subsection{Aufgabenstellung:}
\begin{enumerate}
	\item \fbox{A}  Gib diejenigen Blutgruppen an, die laut der abgebildeten Diagramme sowohl in Österreich als auch in Deutschland häufiger anzutreffen sind als im weltweiten Durchschnitt! 
	
	 Jemand argumentiert anhand der gegebenen Diagramme, dass die Blutgruppe B in Deutschland und Österreich zusammen eine relative Häufigkeit von 13\,\% hat.  Entscheide, ob diese Aussage richtig ist, und begründe deine Entscheidung!


\item  Eine in Österreich lebende Person $X$ hat Blutgruppe A-.

 Gib anhand der in der Einleitung angeführten Daten und Informationen die Wahrscheinlichkeit an, mit der diese Person $X$ als Blutspender/in für eine zufällig ausgewählte, in Österreich lebende Person $Y$ geeignet ist!

  Wie viele von 100 zufällig ausgewählten Österreicherinnen/Österreichern kommen als Blutspender/in für die Person $X$ in Frage? Gib für die Anzahl der potenziellen Blutspender/innen näherungsweise ein um den Erwartungswert symmetrisches Intervall mit 90\,\% Wahrscheinlichkeit an!

\item In einer österreichischen Gemeinde, in der 1\,800 Einwohner/innen Blut spenden könnten, nahmen 150 Personen an einer freiwilligen Blutspendeaktion teil. Es wird angenommen, dass die Blutspender/innen eine Zufallsstichprobe darstellen. 72 Blutspender/innen hatten Blutgruppe A.

 Berechne aufgrund dieses Stichprobenergebnisses ein symmetrisches 95-\%-Konfi denzintervall für den tatsächlichen (relativen) Anteil $p$ der Einwohner/innen dieser Gemeinde mit Blutgruppe A, die Blut spenden könnten.

Die Breite des Konfidenzintervalls wird vom Konfidenzniveau (Sicherheitsniveau) und vom Umfang der Stichprobe bestimmt. Gib an, wie jeweils einer der beiden Parameter geändert werden müsste, um eine Verringerung der Breite des Konfidenzintervalls zu erreichen! Gehe dabei von einem unveränderten (gleichbleibenden) Stichprobenergebnis aus.

\item  Blutgruppenmerkmale werden von den Eltern an ihre Kinder weitervererbt. Dabei sind die  Wahrscheinlichkeiten in der nachstehenden Tabelle angeführt.\leer

\begin{tabular}{|p{3cm}|>{\centering\arraybackslash}p{1,5cm}|>{\centering\arraybackslash}p{1,5cm}|>{\centering\arraybackslash}p{1,5cm}|>{\centering\arraybackslash}p{1,5cm}|}\hline
\multirow{2}{3cm}{Blutgruppe der Eltern}&\multicolumn{4}{|c|}{mögliche Blutgruppe des Kindes}\\ \cline{2-5}
&A&B&AB&0\\ \hline
A und A&$93,75\,\%$&$-$&$-$&$6,25\,\%$\\ \hline
A und B&$18,75\,\%$&$18,75\,\%$&$56,25\,\%$&$6,25\,\%$\\ \hline
A und AB&$50\,\%$&$12,5\,\%$&$37,5\,\%$&$-$\\ \hline
A und 0&$75\,\%$&$-$&$-$&$25\,\%$\\ \hline
B und B&$-$&$93,75\,\%$&$-$&$6,25\,\%$\\ \hline
B und AB&$12,5\,\%$&$50\,\%$&$37,5\,\%$&$-$\\ \hline
B und 0&$-$&$75\,\%$&$-$&$25\,\%$\\ \hline
AB und AB&$25\,\%$&$25\,\%$&$50\,\%$&$-$\\ \hline
AB und 0&$50\,\%$&$50\,\%$&$-$&$-$\\ \hline
0 und 0&$-$&$-$&$-$&$100\,\%$\\ \hline
\end{tabular}

\begin{scriptsize}Datenquelle: https://de.wikipedia.org/wiki/AB0-System [26.11.2014]\end{scriptsize}

Eine Frau mit Blutgruppe A und ein Mann mit Blutgruppe 0 haben zwei (gemeinsame) leibliche Kinder.  Berechne die Wahrscheinlichkeit, dass beide Kinder die gleiche Blutgruppe haben!

Ein Kind aus der Nachbarschaft dieser Familie hat Blutgruppe 0. Gibt es eine Blutgruppe bzw. Blutgruppen, die der leibliche Vater dieses Kindes sicher nicht haben kann? Begründe deine Antwort anhand der gegebenen Daten!
						\end{enumerate}\leer
				
\antwort{
\begin{enumerate}
	\item \subsection{Lösungserwartung:} 
	
		Blutgruppen: A und AB
		
		Die Aussage ist nicht richtig, weil die Anzahl der Einwohner/innen in den beiden genannten Ländern nicht gleich groß ist.
	 	
	\subsection{Lösungsschlüssel:}
	\begin{itemize}
		\item  Ein Ausgleichspunkt für die ausschließliche Angabe der beiden Blutgruppen A und AB
		\item  Ein Punkt für die Angabe, dass die Aussage nicht richtig ist, und eine (sinngemäß) korrekte Begründung dafür.
	\end{itemize}
	
	\item \subsection{Lösungserwartung:}
			
		Die Wahrscheinlichkeit beträgt $48\,\%$.
		
		Mögliche Berechnung:
		
		$n=100, p=0,15 \Rightarrow \mu=15$
		
		$2\cdot\Phi(z)-1=0,9 \Rightarrow z=1,645$
		
		$\mu\pm z\cdot\sigma=15\pm 1,645\cdot\sqrt{100\cdot 0,15\cdot 0,85}\approx 15\pm 6 \Rightarrow [9;21]$

	\subsection{Lösungsschlüssel:}
	
\begin{itemize}
	\item Ein Punkt für die richtige Lösung. Äquivalente Schreibweisen des Ergebnisses (als Bruch oder Dezimalzahl) sind ebenfalls als richtig zu werten. 
	\item Ein Punkt für ein korrektes Intervall.  
	
	Toleranzintervall für den unteren Wert: $[9; 10]$  
	
	Toleranzintervall für den oberen Wert: $[20; 21]$  
	
	Die Aufgabe ist auch dann als richtig gelöst zu werten, wenn bei korrektem Ansatz das Ergebnis aufgrund eines Rechenfehlers nicht richtig ist.
\end{itemize}

\item \subsection{Lösungserwartung:}
	Mögliche Berechnung:
	
	$n=150, h=0,48$
	
	$2\cdot\Phi(z)-1=0,95 \Rightarrow z=1,96$
	
	$h\pm z\cdot\sqrt{\frac{h\cdot (1-h)}{n}}=0,48\pm 1,96\cdot\sqrt{\frac{0,48\cdot (1-0,48)}{150}}\approx 0,48\pm 0,08 \Rightarrow [40\,\%;56\,\%]$
	
	Bei gleichem Stichprobenergebnis führen eine größere Stichprobe und/oder ein geringeres Konfidenzniveau zu einer Verringerung der Breite des Konfidenzintervalls.

	\subsection{Lösungsschlüssel:}
	
\begin{itemize}
	\item Ein Punkt für ein korrektes Intervall. Äquivalente Schreibweisen des Ergebnisses (als Bruch oder Dezimalzahl) sind ebenfalls als richtig zu werten. 
	
	Toleranzintervall für den unteren Wert: $[39\,\%; 43\,\%]$  
	
	Toleranzintervall für den oberen Wert: $[53\,\%; 57\,\%]$  
	
	Die Aufgabe ist auch dann als richtig gelöst zu werten, wenn bei korrektem Ansatz das Ergebnis aufgrund eines Rechenfehlers nicht richtig ist.
	\item Ein Punkt für eine (sinngemäß) korrekte Angabe der entsprechenden Änderungen beider  Parameter.
\end{itemize}

\item \subsection{Lösungserwartung:}
	$0,75²+0,25²=0,625$
	
	Die Wahrscheinlichkeit, dass beide Kinder die gleiche Blutgruppe haben, beträgt 62,5\,\%.
	
	Der Vater kann nicht Blutgruppe AB haben, denn sobald ein Elternteil Blutgruppe AB hat, hat das Kind laut Tabelle nie Blutgruppe 0.
	\subsection{Lösungsschlüssel:}
	
\begin{itemize}
	\item  Ein Punkt für die richtige Lösung. Äquivalente Schreibweisen des Ergebnisses (als Bruch oder in Prozenten) sind ebenfalls als richtig zu werten. Toleranzintervall: $[0,62; 0,63]$. 
	\item Ein Punkt für die richtige Antwort und eine (sinngemäß) korrekte Begründung, warum (nur) Blutgruppe AB auszuschließen ist.
\end{itemize}
\end{enumerate}}
		\end{langesbeispiel}