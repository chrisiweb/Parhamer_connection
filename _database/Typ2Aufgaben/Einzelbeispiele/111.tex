\section{111 - MAT - AG 2.1, AN 1.1, AN 3.2, AN 4.3, FA 1.7, FA 2.6, FA 3.4 - Bremsvorgang - Matura 1. NT 2018/19}

\begin{langesbeispiel} \item[8] %PUNKTE DES BEISPIELS
Der Bremsweg $s_B$ ist die Länge derjenigen Strecke, die ein Fahrzeug ab dem Wirksamwerden der Bremsen bis zum Stillstand zurücklegt. Entscheidend für den Bremsweg sind die Fahrgeschwindigkeit $v_0$ des Fahrzeugs zu Beginn des Bremsvorgangs und die Bremsverzögerung $b$. Der Bremsweg $s_B$ kann mit der Formel $s_B=\dfrac{v_0^2}{2\cdot b}$ berechnet werden ($v_0$ in m/s, $b$ in m/s$^2$, $s_B$ in m).
		
		Der Anhalteweg $s_A$ berücksichtigt zusätzlich zum Bremsweg den während der Reaktionszeit $t_R$ zurückgelegten Weg. Dieser sogenannte \textit{Reaktionsweg} $s_R$ kann mit der Formel $s_R=v_0\cdot t_R$ berechnet werden ($v_0$ in m/s, $t_R$ in s, $s_R$ in m).
		
		Der Anhalteweg $s_A$ ist gleich der Summe aus Reaktionsweg $s_R$ und Bremsweg $s_B$.%Aufgabentext

\begin{aufgabenstellung}
\item %Aufgabentext

\ASubitem{Stelle eine Formel zur Berechnung der Fahrgeschwindigkeit $v_0$ in Abhängigkeit vom Bremsweg $s_B$ und von der Bremsverzögerung $b$ auf.\leer
	
	$v_0=\antwort[\rule{5cm}{0.3pt}]{\sqrt{2\cdot b\cdot s_B}}$} %Unterpunkt1
\Subitem{Kreuze die beiden zutreffenden Aussagen an.\vspace{0,2cm}
	
	\multiplechoice[5]{  %Anzahl der Antwortmoeglichkeiten, Standard: 5
				L1={Der Reaktionsweg $s_R$ ist direkt proportional zur Fahrgeschwindigkeit $v_0$.},   %1. Antwortmoeglichkeit 
				L2={Der Bremsweg $s_B$ ist direkt proportional zur Fahrgeschwindigkeit $v_0$.},   %2. Antwortmoeglichkeit
				L3={Der Bremsweg $s_B$ ist indirekt proportional zur Bremsverzögerung $b$.},   %3. Antwortmoeglichkeit
				L4={Der Anhalteweg $s_A$ ist direkt proportional zur Fahrgeschwindigkeit $v_0$.},   %4. Antwortmoeglichkeit
				L5={Der Anhalteweg $s_A$ ist direkt proportional zur Reaktionszeit $t_R$.},	 %5. Antwortmoeglichkeit
				L6={},	 %6. Antwortmoeglichkeit
				L7={},	 %7. Antwortmoeglichkeit
				L8={},	 %8. Antwortmoeglichkeit
				L9={},	 %9. Antwortmoeglichkeit
				%% LOESUNG: %%
				A1=1,  % 1. Antwort
				A2=3,	 % 2. Antwort
				A3=0,  % 3. Antwort
				A4=0,  % 4. Antwort
				A5=0,  % 5. Antwort
				}} %Unterpunkt2

\item Die oft in Fahrschulen verwendeten Formeln für die näherungsweise Berechnung des Reaktions- und des Bremswegs (jeweils in m) lauten:

	$s_R=\dfrac{v_0}{10}\cdot 3$ und $s_B=\left(\dfrac{v_0}{10}\right)^2$ mit $v_0$ in km/h und $s_R$ bzw. $s_B$ in m%Aufgabentext

\Subitem{Zeige anhand geeigneter Umformungen, dass die für die näherungsweise Berechnung des Reaktionswegs verwendete Formel für eine Reaktionszeit von etwa einer Sekunde annähernd die gleichen Ergebnisse wie die Formel für $s_R$ aus der Einleitung liefert.} %Unterpunkt1
\Subitem{Berechne, welcher Wert für die Bremsverzögerung bei der Näherungsformel für den Bremsweg angenommen wird.} %Unterpunkt2

\item Es kann eine Bremsverzögerung $b$ von 8\,m/s$^2$ bei trockener Fahrbahn, von 6\,m/s$^2$ bei nasser Fahrbahn und von höchstens 4\,m/s$^2$ bei Schneefahrbahn angenommen werden.%Aufgabentext

\Subitem{Gib denjenigen Bruchteil an, um den bei gleicher Fahrgeschwindigkeit der Bremsweg bei nasser Fahrbahn länger als bei trockener Fahrbahn ist.} %Unterpunkt1

Ein Fahrzeug fährt mit einer Geschwindigkeit von $v_0=20$\,m/s. Der Anhalteweg ist bei Schneefahrbahn länger als bei trockener Fahrbahn.

\Subitem{Ermittle unter der Annahme $t_R=1$\,s für diese beiden Fahrbahnzustände den Mindestwert für die absolute Zunahme des Anhaltewegs.} %Unterpunkt2

\item Das Wirksamwerden der Bremsen eines Fahrzeugs beginnt zum Zeitpunkt $t=0$. Die Geschwindigkeit $v(t)$ des Fahrzeugs kann für das Zeitintervall $[0;3]$ durch die Funktion $v$ modelliert werden, die Beschleunigung $a(t)$ durch die Funktion $a$ und der in diesem Zeitintervall zurückgelegte Weg $s(t)$ durch die Funktion $s$ ($v(t)$ in m/s, $a(t)$ in m/s$^2$, $s(t)$ in m, $t$ in s).%Aufgabentext

\Subitem{Interpretiere die Bedeutung des bestimmten Integrals $\displaystyle\int^3_0 v(t)\,\text{d}t$ im gegebenen Kontext.} %Unterpunkt1

Jede der sechs nachstehenden Abbildungen zeigt - jeweils im Zeitintervall $[0;3]$ - den Graphen einer Beschleunigungsfunktion $a$, den Graphen einer Geschwindigkeitsfunktion $v$ und den Graphen einer Wegfunktion $s$.

\Subitem{Kreuze diejenige Abbildung an, die drei zusammengehörige Graphen eines drei Sekunden dauernden Bremsvorgangs zeigt.\vspace{0,2cm}
	
	\langmultiplechoice[6]{  %Anzahl der Antwortmoeglichkeiten, Standard: 5
				L1={\psset{xunit=0.8cm,yunit=0.6cm,algebraic=true,dimen=middle,dotstyle=o,dotsize=5pt 0,linewidth=0.6pt,arrowsize=3pt 2,arrowinset=0.25}
\begin{pspicture*}(-0.58,-1.3)(3.98,4.94)
\begin{scriptsize}
\psaxes[xAxis=true,yAxis=true,showorigin=false,labels=x,Dx=1.,Dy=1.,ticksize=-2pt 0,subticks=0]{->}(0,0)(0.,-2)(3.98,4.94)[$t$,140] [\text{$s(t),v(t),a(t)$},-40]
\rput[tl](-0.3,0.16){0}
\psplot[linewidth=1.pt]{0}{3}{(--1.-0.*x)/1.}
\psplot[linewidth=1.pt,plotpoints=200]{0}{3}{x}
\psplot[linewidth=1.pt,plotpoints=200]{0}{3}{1.0/2.0*x^(2.0)}
\rput[tl](2.62,4.46){$s$}
\rput[tl](2.8,2.65){$v$}
\rput[tl](2.22,1.46){$a$}
\end{scriptsize}
\end{pspicture*}},   %1. Antwortmoeglichkeit 
				L2={\psset{xunit=0.8cm,yunit=0.6cm,algebraic=true,dimen=middle,dotstyle=o,dotsize=5pt 0,linewidth=0.6pt,arrowsize=3pt 2,arrowinset=0.25}
\begin{pspicture*}(-0.58,-1.3)(3.98,4.94)
\begin{scriptsize}
\psaxes[xAxis=true,yAxis=true,showorigin=false,labels=x,Dx=1.,Dy=1.,ticksize=-2pt 0,subticks=0]{->}(0,0)(0.,-2)(3.98,4.94)[$t$,140] [\text{$s(t),v(t),a(t)$},-40]
\rput[tl](-0.3,0.16){0}
\psplot[linewidth=1.pt]{0}{3}{(--1.-0.*x)/1.}
\psplot[linewidth=1.pt,plotpoints=200]{0}{3}{x+1}
\psplot[linewidth=1.pt,plotpoints=200]{0}{3}{1.0/2.0*x^(2.0)}
\rput[tl](2.62,4.46){$s$}
\rput[tl](1.3,2.75){$v$}
\rput[tl](2.22,1.46){$a$}
\end{scriptsize}
\end{pspicture*}},   %2. Antwortmoeglichkeit
				L3={\psset{xunit=0.8cm,yunit=0.6cm,algebraic=true,dimen=middle,dotstyle=o,dotsize=5pt 0,linewidth=0.6pt,arrowsize=3pt 2,arrowinset=0.25}
\begin{pspicture*}(-0.58,-1.3)(3.98,4.94)
\begin{scriptsize}
\psaxes[xAxis=true,yAxis=true,showorigin=false,labels=x,Dx=1.,Dy=1.,ticksize=-2pt 0,subticks=0]{->}(0,0)(0.,-2)(3.98,4.94)[$t$,140] [\text{$s(t),v(t),a(t)$},-40]
\rput[tl](-0.3,0.16){0}
\psplot[linewidth=1.pt]{0}{3}{(-1.-0.*x)/1.}
\psplot[linewidth=1.pt,plotpoints=200]{0}{3}{x}
\psplot[linewidth=1.pt,plotpoints=200]{0}{3}{1.0/2.0*x^(2.0)}
\rput[tl](2.62,4.46){$s$}
\rput[tl](1.3,1.75){$v$}
\rput[tl](2.33,-0.65){$a$}
\end{scriptsize}
\end{pspicture*}},   %3. Antwortmoeglichkeit
				L4={\psset{xunit=0.8cm,yunit=0.6cm,algebraic=true,dimen=middle,dotstyle=o,dotsize=5pt 0,linewidth=0.6pt,arrowsize=3pt 2,arrowinset=0.25}
\begin{pspicture*}(-0.58,-1.3)(3.98,4.94)
\begin{scriptsize}
\psaxes[xAxis=true,yAxis=true,showorigin=false,labels=x,Dx=1.,Dy=1.,ticksize=-2pt 0,subticks=0]{->}(0,0)(0.,-2)(3.98,4.94)[$t$,140] [\text{$s(t),v(t),a(t)$},-40]
\rput[tl](-0.3,0.16){0}
\psplot[linewidth=1.pt]{0}{3}{(-1.-0.*x)/1.}
\psline[linewidth=1.pt](0.,3.)(3.,0.)
\psplot[linewidth=1.pt,plotpoints=200]{0}{3}{1.0/2.0*x^(2.0)}
\rput[tl](2.62,4.46){$s$}
\rput[tl](2.5,1){$v$}
\rput[tl](2.33,-0.65){$a$}
\end{scriptsize}
\end{pspicture*}},   %4. Antwortmoeglichkeit
				L5={\psset{xunit=0.8cm,yunit=0.6cm,algebraic=true,dimen=middle,dotstyle=o,dotsize=5pt 0,linewidth=0.6pt,arrowsize=3pt 2,arrowinset=0.25}
\begin{pspicture*}(-0.58,-1.3)(3.98,4.94)
\begin{scriptsize}
\psaxes[xAxis=true,yAxis=true,showorigin=false,labels=x,Dx=1.,Dy=1.,ticksize=-2pt 0,subticks=0]{->}(0,0)(0.,-2)(3.98,4.94)[$t$,140] [\text{$s(t),v(t),a(t)$},-40]
\rput[tl](-0.3,0.16){0}
\psplot[linewidth=1.pt]{0}{3}{(--1.-0.*x)/1.}
\psline[linewidth=1.pt](0.,3.)(3.,0.)
\psplot[linewidth=1.pt,plotpoints=200]{0}{3}{-1.0/2.0*x^(2.0)+3.0*x}
\rput[tl](2.62,4.35){$s$}
\rput[tl](1.4,2.1){$v$}
\rput[tl](2.9,1.46){$a$}
\end{scriptsize}
\end{pspicture*}},	 %5. Antwortmoeglichkeit
				L6={\psset{xunit=0.8cm,yunit=0.6cm,algebraic=true,dimen=middle,dotstyle=o,dotsize=5pt 0,linewidth=0.6pt,arrowsize=3pt 2,arrowinset=0.25}
\begin{pspicture*}(-0.58,-1.3)(3.98,4.94)
\begin{scriptsize}
\psaxes[xAxis=true,yAxis=true,showorigin=false,labels=x,Dx=1.,Dy=1.,ticksize=-2pt 0,subticks=0]{->}(0,0)(0.,-2)(3.98,4.94)[$t$,140] [\text{$s(t),v(t),a(t)$},-40]
\rput[tl](-0.3,0.16){0}
\psplot[linewidth=1.pt]{0}{3}{(-1.-0.*x)/1.}
\psline[linewidth=1.pt](0.,3.)(3.,0.)
\psplot[linewidth=1.pt,plotpoints=200]{0}{3}{-1.0/2.0*x^(2.0)+3.0*x}
\rput[tl](2.62,4.35){$s$}
\rput[tl](1.4,2.1){$v$}
\rput[tl](2.33,-0.65){$a$}
\end{scriptsize}
\end{pspicture*}},	 %6. Antwortmoeglichkeit
				L7={},	 %7. Antwortmoeglichkeit
				L8={},	 %8. Antwortmoeglichkeit
				L9={},	 %9. Antwortmoeglichkeit
				%% LOESUNG: %%
				A1=6,  % 1. Antwort
				A2=0,	 % 2. Antwort
				A3=0,  % 3. Antwort
				A4=0,  % 4. Antwort
				A5=0,  % 5. Antwort
				}} %Unterpunkt2

\end{aufgabenstellung}

\begin{loesung}
\item \subsection{Lösungserwartung:} 

\Subitem{siehe oben} %Lösung von Unterpunkt1
\Subitem{siehe oben} %%Lösung von Unterpunkt2

\setcounter{subitemcounter}{0}
\subsection{Lösungsschlüssel:}
 
\Subitem{Ein Ausgleichspunkt für eine richtige Formel. Äquivalente Formeln sind als richtig zu werten.} %Lösungschlüssel von Unterpunkt1
\Subitem{Ein Punkt ist genau dann zu geben, wenn ausschließlich die beiden laut Lösungserwartung richtigen Aussagen angekreuzt sind.} %Lösungschlüssel von Unterpunkt2

\item \subsection{Lösungserwartung:} 

\Subitem{mögliche Umformungen:\\
	$s_R=v_0\cdot t_R$\\
	Für $v_0$ in m/s und $t_R=1$ Sekunde gilt: $s_R=v_0$\\
	Für $v_0$ in km/h und $t_R=1$ Sekunde gilt:\\ 
	$s_R=\dfrac{v_0}{3,6}=v_0\cdot 0,278..\approx v_0\cdot 0,3=\dfrac{v_0}{10}\cdot 3$\\
	Daher liefern diese beiden Formeln annähernd die gleichen Ergebnisse.} %Lösung von Unterpunkt1
\Subitem{mögliche Vorgehensweise:\\
	$s_B=\dfrac{v_0^2}{2\cdot b}$ mit $v_0$ in m/s $\Rightarrow$ $s_B=\dfrac{v_0^2}{2\cdot b}\cdot\dfrac{1}{3,6^2}=\dfrac{v_0^2}{25,92\cdot b}$ mit $v_0$ in km/h,\\
	$\dfrac{v_0^2}{25,92\cdot b}=\dfrac{v_0^2}{100} \Rightarrow b\approx 3,9$\,m/s$^2$\\
	Bei der Näherungsformel wird eine Bremsverzögerung von ca. $3,9$\,m/s$^2$ angenommen.} %%Lösung von Unterpunkt2

\setcounter{subitemcounter}{0}
\subsection{Lösungsschlüssel:}
 
\Subitem{Ein Punkt für die Angabe geeigneter Umformungen.} %Lösungschlüssel von Unterpunkt1
\Subitem{Ein Punkt für die richtige Lösung, wobei die Einheit "`m/s$^2$"' nicht angeführt sein muss.

	Toleranzintervall: $[3,8\,\text{m/s}^2; 4\,\text{m/s}^2]$\\
	Die Aufgabe ist auch dann als richtig gelöst zu werten, wenn bei korrektem Ansatz das Ergebnis aufgrund eines Rechenfehlers nicht richtig ist.} %Lösungschlüssel von Unterpunkt2

\item \subsection{Lösungserwartung:} 

\Subitem{$\dfrac{\frac{v_0^2}{2\cdot 6}}{\frac{v_0^2}{2\cdot 8}}=\frac{8}{6}=\frac{4}{3} \Rightarrow$ Bei nasser Fahrbahn ist der Bremsweg um $\frac{1}{3}$ länger als der Bremsweg bei trockener Fahrbahn.} %Lösung von Unterpunkt1
\Subitem{mögliche Vorgehensweise:\\
	Anhalteweg bei trockener Fahrbahn: $s_A=20\cdot 1+\dfrac{20^2}{2\cdot 8}=45$\,m
	
	Mindestwert für den Anhalteweg bei Schneefahrbahn: $s_A=20\cdot 1+\dfrac{20^2}{2\cdot 4}=70$\,m
	
	Der Anhalteweg nimmt (bei $v_0=20$\,m/s und $t_R=1$\,s) bei Schneefahrbahn um mindestens 25\,m zu.} %%Lösung von Unterpunkt2

\setcounter{subitemcounter}{0}
\subsection{Lösungsschlüssel:}
 
\Subitem{Ein Punkt für die richtige Lösung. Andere Schreibweisen des Ergebnisses sind ebenfalls als richtig zu werten.} %Lösungschlüssel von Unterpunkt1
\Subitem{Ein Punkt für die richtige Lösung.} %Lösungschlüssel von Unterpunkt2

\item \subsection{Lösungserwartung:} 

\Subitem{Das bestimmte Integral $\displaystyle\int^3_0 v(t)\,\text{d}t$ beschreibt den zurückgelegten Weg (in Metern) im Zeitintervall $[0;3]$.} %Lösung von Unterpunkt1
\Subitem{Siehe oben} %%Lösung von Unterpunkt2

\setcounter{subitemcounter}{0}
\subsection{Lösungsschlüssel:}
 
\Subitem{Ein Punkt für eine (sinngemäß) richtige Interpretation.} %Lösungschlüssel von Unterpunkt1
\Subitem{Ein Punkt ist genau dann zu geben, wenn ausschließlich die laut Lösungserwartung richtige Abbildung angekreuzt ist.} %Lösungschlüssel von Unterpunkt2

\end{loesung}

\end{langesbeispiel}