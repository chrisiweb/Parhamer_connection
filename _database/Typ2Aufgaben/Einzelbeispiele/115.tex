\section{115 - K5 - VAG2 - AG 3.2, AG 3.3, AG 3.4, AG 3.5, AG-L 3.6, AG-L 3.7 - Mathematischer Orientierungslauf - MatKon}

\begin{langesbeispiel} \item[6] %PUNKTE DES BEISPIELS
Frau Prof. Gonmall und Herr Prof. Konzett bieten im Sommersemester 2020 zum ersten Mal die unverbindliche Übung "`Mathematischer Orientierungslauf"' an. Im Unterschied zum bisherigen Orientierungslauf wird diesmal der Weg zu den einzelnen Stationen jeweils mit Hilfe der Vektorrechnung beschrieben. 

Da die Anmeldezahlen bereits im Vorfeld explodiert sind, waren die beiden ProfessorInnen gezwungen einen Aufnahmetest für die BewerberInnen zu kreieren, der aus drei Teilen besteht und mit dem "`goldenen Zirkel"' belohnt wird. Zu Beginn musst du deine Partnerin sicher zum Ausgang des Labyrinths führen, damit sie sich anschließend auf die Suche nach den Zirkel-Koordinaten begeben kann. Der Aufnahmetest endet, sobald deine Partnerin den goldenen Zirkel in der Hand hält. Natürlich wollen du und deine Teampartnerin dabei sein!%Aufgabentext

\begin{aufgabenstellung}
\item Zu Beginn des Aufnahmetests wird deine Partnerin in die Mitte eines Labyrinths gestellt. Nun muss sie den Weg zum Ausgang $A$ finden. Du bist mit ihr über Telefon verbunden und hältst außerdem folgende Karte in den Händen:

\begin{center}
\psset{xunit=0.6cm,yunit=0.2cm,algebraic=true,dimen=middle,dotstyle=o,dotsize=5pt 0,linewidth=0.8pt,arrowsize=3pt 2,arrowinset=0.25}
\begin{pspicture*}(-6.949797086796506,-13.744147420534643)(6.553817483534391,20.61929870065968)
\multips(0,-15)(0,2.5){15}{\psline[linestyle=dashed,linecap=1,dash=1.5pt 1.5pt,linewidth=0.4pt,linecolor=gray]{c-c}(-6.949797086796506,0)(6.553817483534391,0)}
\multips(-6,0)(2,0){7}{\psline[linestyle=dashed,linecap=1,dash=1.5pt 1.5pt,linewidth=0.4pt,linecolor=gray]{c-c}(0,-13.744147420534643)(0,20.61929870065968)}
\psaxes[labelFontSize=\scriptstyle,showorigin=false,xAxis=true,yAxis=true,Dx=2,Dy=5,ticksize=-2pt 0,subticks=0]{->}(0,0)(-6.949797086796506,-13.744147420534643)(6.553817483534391,20.61929870065968)[x,140] [y,-40]
\psline[linewidth=2pt](-1,0)(-1,8)
\psline[linewidth=2pt](-1,8)(3,8)
\psline[linewidth=2pt](1,0)(1,4)
\psline[linewidth=2pt](3,8)(3,-6)
\psline[linewidth=2pt](-1,0)(-1,-2)
\psline[linewidth=2pt](1,-2)(1,0)
\psline[linewidth=2pt](1,-2)(1,-10)
\psline[linewidth=2pt](1,-10)(5,-10)
\psline[linewidth=2pt](5,-10)(5,12)
\psline[linewidth=2pt](5,12)(-1,12)
\psline[linewidth=2pt](-1,8)(-3,8)
\psline[linewidth=2pt](-3,8)(-3,15)
\psline[linewidth=2pt](-1,12)(-1,15)
\psline[linewidth=2pt](1,-6)(-3,-6)
\psline[linewidth=2pt](-3,-6)(-3,4)
\psline[linewidth=2pt](1,15)(1,12)
\psline[linewidth=2pt](1,15)(5,15)
\psline[linewidth=2pt](-1,-6)(-1,-10)
\psline[linewidth=2pt](-1,-10)(-5,-10)
\psline[linewidth=2pt](-5,-6)(-5,15)
\psline[linewidth=2pt](-5,15)(-3,15)
\begin{scriptsize}
\psdots[dotsize=8pt 0,dotstyle=*](0,0)
\rput[bl](0.12142685623158189,-1.3369296338652374){$S$}
\psdots[dotsize=8pt 0,dotstyle=*](-2,15)
\rput[bl](-1.9052167660693031,16.007191565300896){$A$}
\end{scriptsize}
\end{pspicture*}
\end{center}

(Vorsicht: Deiner Partnerin ist es nur erlaubt sich parallel zur $x$- bzw. zur $y$-Achse zu bewegen. Die Einheiten des Koordinatensystems sind jeweils in Metern angegeben.)
%Aufgabentext

\ASubitem{Schreibe die sieben Vektoren auf, die deine Partnerin nacheinander zum Ausgang führen:\vspace{0,1cm}

$\Vek{0}{5}{}\rightarrow\Vek{\antwort[\rule{0.7cm}{0.3pt}]{2}}{\antwort[\rule{0.7cm}{0.3pt}]{0}}{}\rightarrow\Vek{\antwort[\rule{0.7cm}{0.3pt}]{0}}{\antwort[\rule{0.7cm}{0.3pt}]{-12,5}}{}\rightarrow\Vek{\antwort[\rule{0.7cm}{0.3pt}]{2}}{\antwort[\rule{0.7cm}{0.3pt}]{0}}{}\rightarrow\Vek{\antwort[\rule{0.7cm}{0.3pt}]{0}}{\antwort[\rule{0.7cm}{0.3pt}]{17,5}}{}\rightarrow\Vek{\antwort[\rule{0.7cm}{0.3pt}]{-6}}{\antwort[\rule{0.7cm}{0.3pt}]{0}}{}\rightarrow\Vek{0}{5}{}$\vspace{0,1cm}} %Unterpunkt1
\Subitem{Berechne die kürzeste Entfernung (Luftlinie) zwischen ihrer Startposition $S$ und dem Ausgang $A$.} %Unterpunkt2

\item Vom Ausgang des Labyrinths $A$ sieht man eine kleine Holzhütte, die die Koordinaten $H=(18\mid 30)$ hat. Wenn man vom Ausgang des Labyrinths gerade auf die Holzhütte zugeht, wurden fünf Meter vor der Holzhütte die Koordinaten des goldenen Zirkels vergraben.%Aufgabentext

\Subitem{Berechne jene Distanz, die zwischen Holzhütte und dem Ausgang des Labyrinths liegt.} %Unterpunkt1
\Subitem{Berechne die Koordinaten jenes Punkts $P$ an dem deine Teampartnerin nach der letzten Aufgabe graben muss.} %Unterpunkt2

\item Die letzte Aufgabe ist es, den goldenen Zirkel zu erlangen! Auf dem erdbeschmierten Zettel kann deine Teampartnerin gerade noch die Zirkel-Koordinaten erkennen: $(-23\mid 24)$.

Du behauptest, deine Teampartnerin soll sich entlang der folgenden Gerade bewegen: $X=\Vek{18}{30}{}+t\cdot\Vek{-3}{-1}{}$.

Deine Teampartnerin ist mit dir nicht einer Meinung und meint sie sollte entlang der folgenden Gerade laufen: $41x-6y=558$.%Aufgabentext

\Subitem{Wer von euch beiden hat Recht? Begründe deine Antwort mit entsprechenden Rechnungen.} %Unterpunkt1
\Subitem{Berechne den Winkel um den sich eure beiden Geraden unterscheiden.} %Unterpunkt2

\end{aufgabenstellung}

\begin{loesung}
\item \subsection{Lösungserwartung:} 

\Subitem{$\Vek{0}{5}{}+\Vek{2}{0}{}+\Vek{0}{-12,5}{}+\Vek{2}{0}{}+\Vek{0}{17,5}{}+\Vek{-6}{0}{}+\Vek{0}{5}{}=\Vek{-2}{15}{}$\leer} %Lösung von Unterpunkt1
\Subitem{$\sqrt{(-2)^2+15^2}\approx 15,13$

Die Zielposition befindet sich 15,13\,m von der Startposition entfernt.} %%Lösung von Unterpunkt2

\setcounter{subitemcounter}{0}
\subsection{Lösungsschlüssel:}
 
\Subitem{Ein Punkt für das Aufstellen der sieben Vektoren.} %Lösungschlüssel von Unterpunkt1
\Subitem{Ein Punkt für die richtige Berechnung der Luftlinie.} %Lösungschlüssel von Unterpunkt2

\item \subsection{Lösungserwartung:} 

\Subitem{$\vec{AH}=\Vek{20}{15}{} \Rightarrow |\vec{AH}|=\sqrt{20^2+15^2}\approx 25$} %Lösung von Unterpunkt1
\Subitem{$\vec{AH}_0=\Vek{0,8}{0,6}{}$

$P=\Vek{-2}{15}{}+20\cdot\Vek{0,8}{0,6}{}=\Vek{14}{27}{}$} %%Lösung von Unterpunkt2

\setcounter{subitemcounter}{0}
\subsection{Lösungsschlüssel:}
 
\Subitem{Ein Punkt für die richtige Länge des Vektors.} %Lösungschlüssel von Unterpunkt1
\Subitem{Ein Punkt für die Koordinaten von $P$.} %Lösungschlüssel von Unterpunkt2

\item \subsection{Lösungserwartung:} 

\Subitem{$\Vek{-23}{24}{}=\Vek{18}{30}{}+t\cdot\Vek{-3}{-1}{}$\vspace{0,3cm}

$\Rightarrow$ I:$-23=18-3t \Rightarrow t=13,\dot{6}$, II:$24=30-t \Rightarrow t=6 \Rightarrow (-23\mid 24)\notin g$ 

$41\cdot(-23)-6\cdot 24=-1087\neq 558 \Rightarrow (-23\mid 24)\notin g$

Keine der beiden hat Recht mit der Aussage.} %Lösung von Unterpunkt1
\Subitem{$\cos(\varphi)=\dfrac{\Vek{-3}{-1}{}\cdot\Vek{6}{41}{}}{\sqrt{9+1}\cdot\sqrt{36+1681}}=\dfrac{-18-41}{\sqrt{17170}}=-0,4502636$\leer

$\varphi=116,76^\circ$} %%Lösung von Unterpunkt2

\setcounter{subitemcounter}{0}
\subsection{Lösungsschlüssel:}
 
\Subitem{Ein Punkt für die begründete Feststellung, dass keine Recht hat.} %Lösungschlüssel von Unterpunkt1
\Subitem{Ein Punkt für die richtige Berechnung des Winkels.} %Lösungschlüssel von Unterpunkt2

\end{loesung}

\end{langesbeispiel}