\section{128 - K6 - FA 2.1, FA 2.2, FA 5.1, FA 5.2, WS 1.2 - Bevölkerungsentwicklung - VerSie}

\begin{langesbeispiel} \item[8] %PUNKTE DES BEISPIELS
Die nachstehende Grafik zeigt dir die Bevölkerungsentwicklung von 1960 an.

\begin{tiny}


\begin{tabular}{lcccccccccccc}\hline
\cellcolor[gray]{0.5}&\cellcolor[gray]{0.5}1960&\cellcolor[gray]{0.5}1965&\cellcolor[gray]{0.5}1970&\cellcolor[gray]{0.5}1975&\cellcolor[gray]{0.5}1980&\cellcolor[gray]{0.5}1985&\cellcolor[gray]{0.5}1990&\cellcolor[gray]{0.5}1995&\cellcolor[gray]{0.5}2000&\cellcolor[gray]{0.5}2005&\cellcolor[gray]{0.5}2010&\cellcolor[gray]{0.5}2012\\
\multicolumn{13}{c}{\cellcolor[gray]{0.5}(in Mio)}\\ \hline
\cellcolor[gray]{0.9}Welt&\cellcolor[gray]{0.9}3\,038&\cellcolor[gray]{0.9}3\,333&\cellcolor[gray]{0.9}3\,696&\cellcolor[gray]{0.9}4\,076&\cellcolor[gray]{0.9}4\,453&\cellcolor[gray]{0.9}4\,863&\cellcolor[gray]{0.9}5\,306&\cellcolor[gray]{0.9}5\,726&\cellcolor[gray]{0.9}6\,123&\cellcolor[gray]{0.9}6\,507&\cellcolor[gray]{0.9}6\,896&\cellcolor[gray]{0.9}7\,052\\ \hline
\cellcolor[gray]{0.9}Europa&\cellcolor[gray]{0.9}604&\cellcolor[gray]{0.9}634&\cellcolor[gray]{0.9}656&\cellcolor[gray]{0.9}676&\cellcolor[gray]{0.9}693&\cellcolor[gray]{0.9}707&\cellcolor[gray]{0.9}720&\cellcolor[gray]{0.9}727&\cellcolor[gray]{0.9}727&\cellcolor[gray]{0.9}731&\cellcolor[gray]{0.9}738&\cellcolor[gray]{0.9}740\\ \hline
Afrika&287&324&368&420&483&555&635&721&811&911&1\,022&1\,070\\ \hline
Asien&1\,708&1\,886&2\,135&2\,393&2\,638&2\,907&3\,199&3\,470&3\,719&3\,945&4\,164&4\,250\\ \hline
Lateinamerika u. Karibik&220&253&286&323&362&402&443&483&521&557&590&603\\ \hline
Nordamerika&204&219&231&242&254&267&281&296&313&329&345&351\\ \hline
Ozeanien&16&17&20&21&23&25&27&29&31&34&37&38\\ \hline
\cellcolor[gray]{0.5}&\cellcolor[gray]{0.5}1960&\cellcolor[gray]{0.5}1965&\cellcolor[gray]{0.5}1970&\cellcolor[gray]{0.5}1975&\cellcolor[gray]{0.5}1980&\cellcolor[gray]{0.5}1985&\cellcolor[gray]{0.5}1990&\cellcolor[gray]{0.5}1995&\cellcolor[gray]{0.5}2000&\cellcolor[gray]{0.5}2005&\cellcolor[gray]{0.5}2010&\cellcolor[gray]{0.5}2012\\
\multicolumn{13}{c}{\cellcolor[gray]{0.5}(in \% der Weltbevölkerung)}\\ \hline
\cellcolor[gray]{0.9}Europa&\cellcolor[gray]{0.9}19,9&\cellcolor[gray]{0.9}19,0&\cellcolor[gray]{0.9}17,7&\cellcolor[gray]{0.9}16,6&\cellcolor[gray]{0.9}15,6&\cellcolor[gray]{0.9}14,5&\cellcolor[gray]{0.9}13,6&\cellcolor[gray]{0.9}12,7&\cellcolor[gray]{0.9}11,9&\cellcolor[gray]{0.9}11,2&\cellcolor[gray]{0.9}10,7&\cellcolor[gray]{0.9}10,5\\ \hline
Afrika&9,4&9,7&10,0&10,3&10,8&11,4&12,0&12,6&13,2&14,0&14,8&15,2\\ \hline
Asien&56,2&56,6&57,8&58,7&59,2&59,8&60,3&60,6&60,7&60,6&60,4&60,3\\ \hline
Lateinamerika u. Karibik&7,2&7,6&7,7&7,9&8,1&8,3&8,3&8,4&8,5&8,6&8,6&8,6\\ \hline
Nordamerika&6,7&6,6&6,3&5,9&5,7&5,5&5,3&5,2&5,1&5,1&5,0&5,0\\ \hline
Ozeanien&0,5&0,5&0,5&0,5&0,5&0,5&0,5&0,5&0,5&0,5&0,5&0,5\\ \hline
\end{tabular}
\end{tiny}%Aufgabentext

\begin{aufgabenstellung}
\item Stelle die prozentuelle Häufigkeit der Weltbevölkerung auf die Kontinente im Jahr 2012 in einem Kreisdiagramm dar.
	
	\begin{center}
	
	
	\psset{xunit=1.0cm,yunit=1.0cm,algebraic=true,dimen=middle,dotstyle=o,dotsize=5pt 0,linewidth=1.6pt,arrowsize=3pt 2,arrowinset=0.25}
\begin{pspicture*}(-4.3,-4.2)(4.22,4.2)
\pscircle[linewidth=2.pt](0.,0.){4.}
\begin{scriptsize}
\psdots[dotsize=7pt 0,dotstyle=*](0.,0.)
\rput[bl](0.08,0.28){$M$}
\end{scriptsize}
\end{pspicture*}
\end{center}%Aufgabentext

\item %Aufgabentext

\Subitem{Stelle für Asien aufgrund der Daten von 1960 $(t=0)$ und 1965 eine Formel zur Wachstumsprognose für lineares Wachstum auf.} %Unterpunkt1
\Subitem{Berechne mit dieser Formel die Bevölkerungszahl für den asiatischen Raum im Jahr 2012 $(t=52)$.\\
(Sollte das Aufstellen der Formel nicht klappen verwende für weitere Berechnungen: $y=36,1x+1700$)} %Unterpunkt2

\item %Aufgabentext

\ASubitem{Stelle für Asien aufgrund der Daten von 1960 $(t=0)$ und 1965 eine Formel zur Wachstumsprognose für exponentielles Wachstum auf.} %Unterpunkt1
\Subitem{Berechne damit die Bevölkerungszahl für den asiatischen Raum im Jahr 2012 $(t=52)$.\\
(Sollte das Aufstellen der Formel nicht klappen verwende für weitere Berechnung: $N(t)=1700-1,023^t$)} %Unterpunkt2

\item Diskutiere die prognostizierten Werte für 2012 aus (b) und (c) im Vergleich mit der tatsächlich erreichten Bevölkerungszahl laut Tabelle.%Aufgabentext


\end{aufgabenstellung}

\begin{loesung}
\item \subsection{Lösungserwartung:} 

\kreisdiagramm\begin{tikzpicture}
\pie[color={black!10 ,black!20 ,black!30 ,black!40 ,black!50 ,black!60}, %Farbe
text=pin %Format: inside,pin, legend
]
{10.5/Europa , 15.2/Afrika , 60.3/Asien , 8.6/Lateinamerika , 5.0/Nordamerika , 0.5/Ozeanien} %Werte
\end{tikzpicture}

\setcounter{subitemcounter}{0}
\subsection{Lösungsschlüssel:}
 
\Subitem{Ein Punkt für ein teilweise richtige Kreisdiagramm.} %Lösungschlüssel von Unterpunkt1
\Subitem{Ein (weiterer) Punkt für ein vollständig richtiges Kreisdiagramm.} %Lösungschlüssel von Unterpunkt2

\item \subsection{Lösungserwartung:} 

\Subitem{Asien: 1960: $1\,708$; 1965: $1\,886$; $t=5$
	
	$k=\frac{1\,886-1\,708}{5}=\frac{178}{5}=35,6$
	
	$d=1\,708 \Rightarrow y=35,6\cdot x+1\,708$} %Lösung von Unterpunkt1
\Subitem{$y=35,6\cdot 52+1\,708=3559,2$
	
	Im Jahr 2012 leben (laut dieser Prognose) ungefähr 3560 Menschen in Asien.} %%Lösung von Unterpunkt2

\setcounter{subitemcounter}{0}
\subsection{Lösungsschlüssel:}
 
\Subitem{Ein Punkt für das richtige Aufstellen der linearen Funktion.} %Lösungschlüssel von Unterpunkt1
\Subitem{Ein Punkt für die richtige Berechnung der prognostizierten Bevölkerungszahl.} %Lösungschlüssel von Unterpunkt2

\item \subsection{Lösungserwartung:} 

\Subitem{$y=a\cdot b^x$
	
	$a=1\,708 \Rightarrow 1\,886=1\,708\cdot b^5 \Rightarrow b=1.020024878586$
	
	$N(t)=1\,708\cdot 1,020024879^t$} %Lösung von Unterpunkt1
\Subitem{$N(52)=1\,708\cdot 1,020024879^{52}\approx 4789$
	
	Laut der Prognose sollten im Jahr 2012 in Asien 4\,789 Menschen leben.} %%Lösung von Unterpunkt2

\setcounter{subitemcounter}{0}
\subsection{Lösungsschlüssel:}
 
\Subitem{Ein Punkt für das richtige Aufstellen der exponentielle Funktion.} %Lösungschlüssel von Unterpunkt1
\Subitem{Ein Punkt für die richtige Berechnung der prognostizierten Bevölkerungszahl.} %Lösungschlüssel von Unterpunkt2

\item \subsection{Lösungserwartung:} 

Beide Modelle sind ungefähr gleich ungenau, da der eigentliche Wert 4\,250 beträgt. Die lineare Approximation ist zu niedrig, die exponentielles Approximation ist zu hoch.
	
	Der Mittelwert der beiden Schätzungen beträgt 4\,174,5 und ist damit eine sehr gute Schätzung für den eigentlichen Wert (4\,250).

\setcounter{subitemcounter}{0}
\subsection{Lösungsschlüssel:}
 
\Subitem{Ein Punkt für eine der beiden Tatsachen.} %Lösungschlüssel von Unterpunkt1
\Subitem{Ein (weiterer) Punkt für beide Tatsachen.} %Lösungschlüssel von Unterpunkt2

\end{loesung}

\end{langesbeispiel}