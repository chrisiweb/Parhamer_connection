\section{37 - MAT - AN 1.3, FA 1.5, FA 2.2, FA 2.1, FA 1.6 - Kosten und Erlös - Matura 2013/14 1. Nebentermin}

\begin{langesbeispiel} \item[0] %PUNKTE DES BEISPIELS
				Die für einen Betrieb anfallenden Gesamtkosten bei der Produktion einer Ware können annähernd durch eine Polynomfunktion $K$ beschrieben werden. Die lineare Funktion $E$ gibt den Erlös (Umsatz) in Abhängigkeit von der Stückzahl $x$ an.
				
				Die Stückzahl $x$ wird in Mengeneinheiten $[ME]$ angegeben, die Produktionskosten $K(x)$ und der Erlös $E(x)$ werden in Geldeinheiten $[GE]$ angegeben.
				
				\begin{center}
				\begin{scriptsize}
		\psset{xunit=0.4cm,yunit=0.004cm,algebraic=true,dimen=middle,dotstyle=o,dotsize=5pt 0,linewidth=0.8pt,arrowsize=3pt 2,arrowinset=0.25}
\begin{pspicture*}(-1.7989386315630714,-96.39412811239365)(21.622997403440056,1562.810546565777)
\multips(0,0)(0,100.0){17}{\psline[linestyle=dashed,linecap=1,dash=1.5pt 1.5pt,linewidth=0.4pt,linecolor=lightgray]{c-c}(0,0)(21.622997403440056,0)}
\multips(0,0)(1.0,0){23}{\psline[linestyle=dashed,linecap=1,dash=1.5pt 1.5pt,linewidth=0.4pt,linecolor=lightgray]{c-c}(0,0)(0,1562.810546565777)}
\psaxes[labelFontSize=\scriptstyle,xAxis=true,yAxis=true,Dx=2.,Dy=200.,ticksize=-2pt 0,subticks=2]{->}(0,0)(0.,0.)(21.622997403440056,1562.810546565777)[x,140] [\text{$E(x)$, $K(x)$},-40]
\psplot[linewidth=0.8pt]{0}{21.622997403440056}{(-0.--400.*x)/5.}
\psplot[plotpoints=200]{0}{21.622997403440056}{0.4230662989291667*x^(3.0)-5.1151933807250005*x^(2.0)+38.69171274003334*x+100.0}
\rput[tl](9,839.286618656422){E}
\rput[tl](12.823950792094209,611.2951558425417){K}
\rput[tl](4.1,165){Kostenkehre}
\psdots[dotsize=3pt 0,dotstyle=*](4.,200.)
\end{pspicture*}
\end{scriptsize}
\end{center}

Man spricht von einer Kostendegression, wenn der Produktionskostenzuwachs bei einer Erhöhung der Anzahl der erzeugten Mengeneinheiten immer kleiner wird. Man spricht von einer Kostenprogression, wenn der Produktionskostenzuwachs bei einer Erhöhung der Anzahl der erzeugten Mengeneinheiten immer größer wird.
				
\subsection{Aufgabenstellung:}
\begin{enumerate}
	\item \fbox{A} Berechne den durchschnittlichen Kostenanstieg pro zusätzlich produzierter Mengeneinheit im Intervall $[10;14]$!
	
	Gib dasjeniger Intervall an, in dem ein degressiver Kostenverlauf vorliegt!

\item Gib den Verkaufspreis pro Mengeneinheit an!

Stelle eine Gleichung der Erlösfunktion $E$ auf!

\item Interpretiere die x-Koordinate der Schnittpunkte des Graphen der Kostenfunktion $K$ mit dem Graphen der Erlösfunktion $E$ und gib die Bedetung des Bereichs zwischen den beiden Schnittpunkten für das Unternehmen an!

Gib den Gewinn an, wenn 10 Mengeneinheiten produziert und verkauft werden!
						\end{enumerate}\leer
				
\antwort{
\begin{enumerate}
	\item \subsection{Lösungserwartung:} 
	
	$K(10)=400, K(14)=800, \dfrac{K(14)-K(10)}{14-10}=100$
	
	Der durchschnittliche Kostenanstieg beträgt im Intervall $[10 ME; 14 ME]$ 100 GE/ME. Kostendegression im Intervall: $[0;4)$.
 
	 	
	\subsection{Lösungsschlüssel:}
	\begin{itemize}
		\item Ein Ausgleichspunkt für eine korrekte Berechnung des Differenzenquotienten.
		\item  Ein Punkt für die Angabe des korrekten Intervalls (es sind sowohl offene, geschlossene als auch halboffene Intervalle zulässig).
	\end{itemize}
	
	\item \subsection{Lösungserwartung:}
		Der Verkaufspreis beträgt $80$ GE pro ME.
		
		$E(x)=80\cdot x$
		
		
	\subsection{Lösungsschlüssel:}
	
\begin{itemize}
	\item Ein Punkt für die korrekte Angabe des Verkaufspreises.
	\item Ein Punkt, wenn $E(x)$ richtig angegeben ist.
\end{itemize}

\item \subsection{Lösungserwartung:} 
	
	Die Kosten und der Erlös sind gleich hoch, daher wird kein Gewinn erzielt. Die x-Koordinaten der Schnittpunkte geben die Gewinnschwellen an. Bei einer Menge $x$, die sich zwischen den beiden Gewinnschwellen befindet, macht das Unternehmen Gewinn.
	
	$K(10)=400, E(10)=800;$ das Unternehmen macht einen Gewinn von 400 GE.
	 	
	\subsection{Lösungsschlüssel:}
	\begin{itemize}
		\item Ein Punkt für eine richtige Interpretation der Schnittpunkte und des Bereiches zwischen den Stellen der Schnittpunkte. Sinngemäß gleichwertige Aussagen sind als richtig zu werten.
		\item  Ein Punkt für die korrekte Berechnung des Gewinns.
	\end{itemize}

\end{enumerate}}
		\end{langesbeispiel}