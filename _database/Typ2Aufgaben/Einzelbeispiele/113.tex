\section{113 - MAT - AG 2.3, AN 1.4, FA 1.4, FA 5.2 - Fibonacci-Zahlen und der Goldene Schnitt - Matura 1.NT 2018/19}

\begin{langesbeispiel} \item[4] %PUNKTE DES BEISPIELS
Die sogenannten \textit{Fibonacci-Zahlen} werden für $n\in\mathbb{N}$ und $n>2$ durch die Differenzengleichung $f(n)=f(n-1)+f(n-2)$ mit den Startwerten $f(1)=1$ und $f(2)=1$ definiert.

Das Verhältnis $f(n):f(n-1)$ nähert sich für große Werte von $n$ dem \textit{Goldenen Schnitt} $\Phi=\dfrac{1+\sqrt{5}}{2}$ an.%Aufgabentext

\begin{aufgabenstellung}
\item %Aufgabentext

\ASubitem{Gib dasjenige $n$ an, für das das Verhältnis $f(n):f(n-1)$ erstmals auf zwei Nachkommastellen mit dem Goldenen Schnitt $\Phi$ übereinstimmt.} %Unterpunkt1

Für Fibonacci-Zahlen gilt für $k\in\mathbb{N}$ und $k>2$ folgende Gleichung:\\
	$f(n+k)=f(n-1)\cdot f(k)+f(n)\cdot f(k+1)$

\Subitem{Zeige die Gültigkeit dieser Gleichung für $n=3$ und $k=5$.} %Unterpunkt2

\item Eine Möglichkeit zur näherungsweisen Bestimmung von Fibonacci-Zahlen durch einen einfachen expliziten Ausdruck ist die Approximation\\ 
$f(n)\approx g(n)=\dfrac{1}{\sqrt{5}}\left(\dfrac{1+\sqrt{5}}{2}\right)^n$ mit $n\in\mathbb{N}\backslash\{0\}$.
	
	Die Zahl $832\,040$ ist eine Fibonacci-Zahl, das heißt, es gibt ein $n\in\mathbb{N}$ mit\\ 
	$f(n)=832\,040$ bzw. $g(n)\approx 832\,040$.%Aufgabentext

\Subitem{Bestimme dieses $n$.} %Unterpunkt1

Eine exakte explizite Möglichkeit zur Berechnung der Fibonacci-Zahlen $f(n)$ ist die Formel von Moivre/Binet:\\
	$f(n)=\dfrac{1}{\sqrt{5}}\cdot\left(x_1^n-x_2^n\right)$\\
	Dabei sind $x_1=\Phi$ und $x_2$ die Lösungen der Gleichung $x^2+a\cdot x-1=0$ mit $a\in\mathbb{R}$.

\Subitem{Berechne $a$ und $x_2$} %Unterpunkt2

\end{aufgabenstellung}

\begin{loesung}
\item \subsection{Lösungserwartung:} 

\Subitem{mögliche Vorgehensweise:\\
$\Phi=\dfrac{1+\sqrt{5}}{2}\approx 1,618$

\begin{center}
\begin{tabular}{|l|l|l|}\hline
\cellcolor[gray]{0.9}$n$&\cellcolor[gray]{0.9}$f(n)$&\cellcolor[gray]{0.9}$f(n):f(n-1)$\\ \hline
1&1&-\\ \hline
2&1&1\\ \hline
3&2&2\\ \hline
4&3&1,5\\ \hline
5&5&1,666...\\ \hline
6&8&1,6\\ \hline
7&13&1,625\\ \hline
8&21&1,625...\\ \hline
\end{tabular}
\end{center}

Für $n=8$ stimmt das Verhältnis $f(n):f(n-1)$ erstmals auf zwei Nachkommastellen mit $\Phi$ überein.} %Lösung von Unterpunkt1
\Subitem{mögliche Vorgehensweise:\\
$n=3$, $k=5$ $\Rightarrow$ $f(8)=f(2)\cdot f(5)+f(3)\cdot f(6)$\\
$21=1\cdot 5+2\cdot 8$\\
$21=21$ w.A.} %%Lösung von Unterpunkt2

\setcounter{subitemcounter}{0}
\subsection{Lösungsschlüssel:}
 
\Subitem{Ein Ausgleichspunkt für die richtige Lösung.} %Lösungschlüssel von Unterpunkt1
\Subitem{Ein Punkt für einen richtigen Nachweis.} %Lösungschlüssel von Unterpunkt2

\item \subsection{Lösungserwartung:} 

\Subitem{mögliche Vorgehensweise:\\
$g(n)=\dfrac{1}{\sqrt{5}}\cdot\left(\dfrac{1+\sqrt{5}}{2}\right)^n\approx 832\,040 \Rightarrow n=30$} %Lösung von Unterpunkt1
\Subitem{mögliche Vorgehensweise:\\
$\left(\dfrac{1+\sqrt{5}}{2}\right)^2+a\cdot\dfrac{1+\sqrt{5}}{2}-1=0 \Rightarrow a=-1$

Lösen der Gleichung: $x^2-x-1=0: x_1=\dfrac{1+\sqrt{5}}{2}=\Phi$ und $x_2=\dfrac{1-\sqrt{5}}{2}$ $(\approx -0,618)$} %%Lösung von Unterpunkt2

\setcounter{subitemcounter}{0}
\subsection{Lösungsschlüssel:}
 
\Subitem{Ein Punkt für die richtige Lösung.} %Lösungschlüssel von Unterpunkt1
\Subitem{Ein Punkt für die Angabe der beiden richtigen Werte.

Toleranzintervall für $x_2$: $[-0,62; -0,60]$\\
Die Aufgabe ist auch dann als richtig gelöst zu werten, wenn bei korrektem Ansatz das Ergebnis aufgrund eines Rechenfehlers nicht richtig ist.} %Lösungschlüssel von Unterpunkt2

\end{loesung}

\end{langesbeispiel}