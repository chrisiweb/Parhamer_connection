\section{131 - K6 - AG 3.4, AG-L 3.8, AG-L 3.9 - Goldenes Dachl - VerSie}

\begin{langesbeispiel} \item[4] %PUNKTE DES BEISPIELS
Im Auftrag von Kaiser Maximilian I. wurde das Goldene Dachl in Innsbruck von Niclas Türing dem Älteren errichtet und laut Inschrift 1500 fertig gestellt. Es besteht aus 2700 feuervergoldeten Kupferschindeln.

\meinlr[0.2]{Ein Künstler richtet für eine nächtliche Lichtinstallation drei Lichtstrahlen vom Platz auf das Goldene Dachl. Ein Strahl soll vor dem Dach parallel zur Dachfläche verlaufen und den Blick in den Sternenhimmel lenken.

 Ein Strahl soll das Dach in einem Punkt treffen, um eine einzelne Schindel zum Leuchten zu bringen. Ein Strahl soll das Dach streifen, um eine strahlende Spur auf dem Dach zu hinterlassen.}{\begin{center}
\includegraphics[width=0.75\textwidth]{../_database/Bilder/131_dachl.eps}\\
\tiny Quelle: Wikipedia\end{center}}



Das Dach wird zur Vereinfachung als eben betrachtet. Die Trägerebene $e$ der vorderen Dachfläche geht durch die Punkte\\ 
$A=(6\mid 0\mid 12)$, $B=(2\mid 4\mid 12)$ und $C=(3\mid 0\mid 16)$.%Aufgabentext

\begin{aufgabenstellung}
\item %Aufgabentext

\ASubitem{Stelle die Ebenengleichung $e$ in Parameterform auf.} %Unterpunkt1
\Subitem{Wie lautet die parameterfreie Ebenengleichung?} %Unterpunkt2

\item Gegeben ist ein Lichtstrahl mit Hilfe der Geradengleichung\\ 
	$g$: $X=\Vek{9}{0}{8}+s\cdot\Vek{-3}{0}{4}$.%Aufgabentext

\Subitem{Untersuche welche Lagebeziehung die Gerade zu der Trägerebene $e$ des goldenen Dachls hat.} %Unterpunkt1
\Subitem{Interpretiere die Lagebeziehung im vorliegenden Kontext.} %Unterpunkt2

\end{aufgabenstellung}

\begin{loesung}
\item \subsection{Lösungserwartung:} 

\Subitem{$\vec{AB}=\Vek{-4}{4}{0}$ und $\vec{AC}=\Vek{-3}{0}{4}$
	
	$e$: $X=\Vek{6}{0}{12}+t\cdot\Vek{-4}{4}{0}+s\cdot\Vek{-3}{0}{4}$} %Lösung von Unterpunkt1
\Subitem{$\vec{n}=\Vek{-4}{4}{0}\times\Vek{-3}{0}{4}=\Vek{16}{16}{12}\parallel\Vek{4}{4}{3}$
	
	$e$: $\vec{n}\cdot X=\vec{n}\cdot A \Rightarrow 4x+4y+3z=60$} %%Lösung von Unterpunkt2

\setcounter{subitemcounter}{0}
\subsection{Lösungsschlüssel:}
 
\Subitem{Ein Punkt für die Parameterform der Ebene.} %Lösungschlüssel von Unterpunkt1
\Subitem{Ein Punkt für die Normalvektorform der Ebene.} %Lösungschlüssel von Unterpunkt2

\item \subsection{Lösungserwartung:} 

\Subitem{Da die Richtungsvektoren ident sind, liegt die Gerade entweder parallel zur Ebene oder direkt auf der Ebene.
	
	$P\in e$? $4\cdot 9+4\cdot 0+3\cdot 8=60$
	
	Die Gerade liegt auf der Ebene.} %Lösung von Unterpunkt1
\Subitem{Es handelt sich bei diesem Lichtstrahl um jenen, der das Dach streifen soll um so einen strahlende Spur auf dem Dach zu hinterlassen.} %%Lösung von Unterpunkt2

\setcounter{subitemcounter}{0}
\subsection{Lösungsschlüssel:}
 
\Subitem{Ein Punkt für die Lagebeziehung.} %Lösungschlüssel von Unterpunkt1
\Subitem{Ein Punkt für eine richtige Interpretation.} %Lösungschlüssel von Unterpunkt2

\end{loesung}

\end{langesbeispiel}