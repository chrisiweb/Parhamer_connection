\section{75 - MAT - AG 2.1, AN 1.3, FA 2.1, FA 2.2 - Stratosph�rensprung - BIFIE Aufgabensammlung}

\begin{langesbeispiel} \item[0] %PUNKTE DES BEISPIELS
	
Am 14.10.2012 sprang der �sterreichische Extremsportler Felix Baumgartner aus einer H�he von $38\,969$\,m �ber dem Meeresspiegel aus einer Raumkapsel. Er erreichte nach $50$\,s in der nahezu luftleeren Stratosph�re eine H�chstgeschwindigkeit von $1\,357,6$\,km/h $(\approx 377,1$\,m/s) und �berschritt dabei als erster Mensch im freien Fall die Schallgeschwindigkeit, die bei $20^\circ$C ca. $1\,236$\,km/h $(\approx 343,3$\,m/s) betr�gt, in der Stratosph�re wegen der niedrigen Lufttemperaturen aber deutlich geringer ist.\leer

Die Schallgeschwindigkeit in trockener Luft h�ngt bei Windstille nur von der Lufttemperatur $T$ ab. F�r die Berechnung der Schallgeschwindigkeit in Metern pro Sekunde (m/s) werden nachstehend zwei Formeln angegeben, die - bis auf einen (gerundeten) Faktor - �quivalent sind.

Die Lufttemperatur $T$ wird in beiden Formeln in $^\circ$C angegeben.\leer

$v_1=\sqrt{401,87\cdot (T+273,15)}$

$v_2=331,5\cdot\sqrt{1+\frac{T}{273,15}}$

\subsection{Aufgabenstellung:}
\begin{enumerate}
	\item Die Fallbeschleunigung $a$ eines K�rpers im Schwerefeld der Erde ist abh�ngig vom Abstand des K�rpers zum Erdmittelpunkt. Die Fallbeschleunigung an der Erdoberfl�che auf Meeresniveau, d.h. bei einer Entfernung von $r=6\,371\,000$\,m vom Erdmittelpunkt, betr�gt bei vernachl�ssigbarem Luftwiderstand ca. $9,81$\,m/s$�$.
	
	F�r die Fallbeschleunigung $a$ gilt: $a(r)=\frac{G\cdot M}{r�}$, wobei $G$ die Gravitationskonstante, $M$ die Erdmasse und $r$ der Abstand des K�rpers vom Erdmittelpunkt ist. Es gilt:
	
	$G=6,67\cdot 10^{-11}\,\text{N}\frac{\text{m}�}{\text{kg}�}; \, M=5,97\cdot 10^{24}$\,kg
	
	Berechne den Wert der Fallbeschleunigung, die auf Felix Baumgartner beim Absprung aus der Raumkapsel wirkte!\leer
	
	$a=$ \rule{3cm}{0.3pt}\,m/s$�$\leer
	
	Berechne die mittlere Fallbeschleunigung, die auf Felix Baumgartner bis zum Erreichen der H�chstgeschwindigkeit wirkte, wenn von konstanter Lufttemperatur w�hrend dieser Zeit ausgegangen wird!\leer
	
	\item Als Felix Baumgartner seine H�chstgeschwindigkeit erreichte, bewegte er sich um $25\,\%$ schneller als der Schall in dieser H�he.\leer
	
	Gib eine Gleichung an, mit der unter Verwendung einer der beiden in der Einleitung genannten Formeln die Lufttemperatur, die zu diesem Zeitpunkt geherrscht hat, berechnet werden kann, und ermittle diese Lufttemperatur!\leer
	
	Untersuche mithilfe der beiden Formeln den Quotienten der Schallgeschwindigkeiten im Lufttemperaturintervall $[-60\,^\circ\text{C}; 20\,^\circ\text{C}]$ in Schritten von $10\,^\circ\text{C}$ und gib eine Formel an, die in diesem Lufttemperaturintervall den Zusammenhang zwischen $v_1$ und $v_2$ beschreibt!\leer
	
	\item Zeige mithilfe von �quivalenzumformungen, dass die beiden Formeln f�r die Schallgeschwindigkeit in der Einleitung bis auf einen (gerundeten) Faktor �quivalent sind! Geh dabei von der Formel f�r $v_1$ aus!\leer
	
	Die Abh�ngigkeit der Schallgeschwindigkeit $v_1$ von der Lufttemperatur $T$ kann im Lufttemperaturintervall $[-20\,^\circ\text{C}; 40\,^\circ\text{C}]$ in guter N�herung durch eine lineare Funktion $f$ mit $f(T)=k\cdot T+d$ modelliert werden.
	
	Ermittle die Werte der Parameter $k$ und $d$ und interpretiere diese Werte im gegebenen Kontext!
	
\end{enumerate}

\antwort{
\begin{enumerate}
	\item \subsection{L�sungserwartung:} 

$r_1=6\,371\,000+38\,969=6\,409969$\,m

$a(r_1)=\frac{6,67\cdot 10^{-11}\cdot 5,97\cdot 10^{24}}{6\,409\,969�}=9,69$\,m/s$�$

mittlere Fallbeschleunigung: $a=\frac{377,1}{50}=7,54$\,m/s$�$
	
	\item \subsection{L�sungserwartung:}
	
	$\frac{377,1}{1,25}=301,7$\,m/s
	
	$v_1(T)=301,7 \Rightarrow T\approx -46,7\,^\circ\text{C}$
	
	bzw. $v_2(T)=301,7 \Rightarrow T\approx -46,9\,^\circ\text{C}$
	
	\begin{center}
		\begin{tabular}{|c|c|c|c|}\hline
		\cellcolor[gray]{0.9}$T$ in $\,^\circ\text{C}$&\cellcolor[gray]{0.9}$v_1$ in m/s&\cellcolor[gray]{0.9}$v_2$ in m/s&\cellcolor[gray]{0.9}$\frac{v_2}{v_1}$\\ \hline
		$-60$&292,67&292,84&1,00055\\ \hline
		$-50$&299,46&299,63&1,00055\\ \hline
		$-40$&306,10&306,27&1,00055\\ \hline
		$-30$&312,59&312,77&1,00055\\ \hline
		$-20$&318,96&319,13&1,00055\\ \hline
		$-10$&325,20&325,38&1,00055\\ \hline
		$0$&331,32&331,50&1,00055\\ \hline
		$10$&337,33&337,51&1,00055\\ \hline
		$20$&343,23&343,42&1,00055\\ \hline		
		\end{tabular}
	\end{center}
	
	$v_2\approx 1,00055\cdot v_1$ bzw. $v_1\approx 0,99945\cdot v_2$

\item \subsection{L�sungserwartung:}
	
$v_1=\sqrt{401,87\cdot (T+273,15)}=\sqrt{401,87\cdot 273,15\cdot(\frac{T}{273,15}+1)}\approx$

$\approx\sqrt{109\,770,8\cdot(\frac{T}{273,15}+1)}\approx 331,3\cdot\sqrt{\frac{T}{273,15}+1}$

Der Faktor 331,3 unterscheidet sich nur geringf�gig vom Faktor 331,5 in der Formel f�r $v_2$.\leer

$k=\frac{v_1(40)-v_1(-20)}{60}\approx 0,6$ ... pro $1\,^\circ\text{C}$ nimmt die Schallgeschwindigkeit um ca 0,6\,m/s zu

$d=v_1(0)\approx 331,3$ ... Schallgeschwindigkeit bei $0\,^\circ\text{C}$
\end{enumerate}}
		\end{langesbeispiel}