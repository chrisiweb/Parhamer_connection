\section{141 - MAT - AN 1.1, AN 1.3, AN 4.3 - Fallschirmsprung - Matura 2019/20 1. HT}

\begin{langesbeispiel}\item[6] %PUNKTE DER AUFGABE
Bei einem Fallschirmsprung aus einer Höhe von 4\,000\,m über Grund wird 30\,s nach dem Absprung der Fallschirm geöffnet.

Für $t\in[0;30]$ gibt die Funktion $v_1$ mit $v_1(t)=56-56\cdot e^{-\frac{t}{4}}$ (unter Berücksichtigung des Luftwiderstands) die Fallgeschwindigkeit des Fallschirmspringers zum Zeitpunkt $t$ an ($t$ in s nach dem Absprung, $v_1(t)$ in m/s).

Für $t\geq 30$ gibt die Funktion $v_2$ mit $v_2(t)=\frac{51}{(t-29)^2}+5-56\cdot e^{-7,5}$ die Fallgeschwindigkeit des Fallschirmspringers zum Zeitpunkt $t$ bis zum Zeitpunkt der Landung an ($t$ in s nach dem Absprung, $v_2(t)$ in m/s).

Modellhaft wird angenommen, dass der Fallschirmsprung lotrecht ist.%Aufgabentext

\begin{aufgabenstellung}
\item %Aufgabentext

\Subitem{Deute $w=\dfrac{v_1(10)-v_1(5)}{10-5}$ im gegebenen Kontext.} %Unterpunkt1

Für ein $t_1\in[0;30]$ gilt: $v_1'(t_1)=w$.

\Subitem{Deute $t_1$ im gegebenen Kontext.} %Unterpunkt2

\item %Aufgabentext

\Subitem{Berechne mithilfe der Funktion $v_1$, in welcher Höhe der Fallschirm geöffnet wird.} %Unterpunkt1
\Subitem{Berechne die Zeitdauer des gesamten Fallschirmsprungs vom Absprung bis zur Landung.} %Unterpunkt2

\item Ohne Berücksichtigung des Luftwiderstands hätte der Fallschirmspringer eine Anfangsgeschwindigkeit von 0\,m/s und im Zeitintervall $[0;30]$ eine konstante Beschleunigung von $9,81$\,m/s$^2$. Die Fallgeschwindigkeit 9\,s nach dem Absprung beträgt dann $v^*$.%Aufgabentext

\Subitem{Berechne, um wie viel $v_1(9)$ kleiner ist als $v^*$,} %Unterpunkt1
\Subitem{Berechne, um wie viel Prozent 9\,s nach dem Absprung die Beschleunigung des Fallschirmspringers geringer ist als bei einem Sprung ohne Berücksichtigung des Luftwiderstands.} %Unterpunkt2

\end{aufgabenstellung}

\begin{loesung}
\item \subsection{Lösungserwartung:} 

\ASubitem{mögliche Deutung:

Im Zeitintervall $[5;10]$ nimmt die Fallgeschwindigkeit (in m/s) des Fallschirmspringers pro Sekunde durchschnittlich um $w$ zu.

oder:

Die mittlere Beschleunigung des Fallschirmspringers im Zeitintervall $[5;10]$ beträgt $w$ (in m/s$^2$).} %Lösung von Unterpunkt1
\Subitem{mögliche Deutung:

Zum Zeitpunkt $t_1$ ist die Momentanbeschleunigung genauso hoch wie die mittlere Beschleunigung im Zeitintervall $[5;10]$.} %%Lösung von Unterpunkt2

\setcounter{subitemcounter}{0}
\subsection{Lösungsschlüssel:}
 
\Subitem{Ein Ausgleichspunkt für eine richtige Deutung.} %Lösungschlüssel von Unterpunkt1
\Subitem{Ein Punkt für die richtige Deutung.} %Lösungschlüssel von Unterpunkt2

\item \subsection{Lösungserwartung:} 

\Subitem{$4\,000-\displaystyle\int^{30}_0 v_1(t)\dx[t]=2\,543,8\ldots\approx 2\,544$

Der Fallschirm wird in einer Höhe von ca. 2\,544\,m geöffnet.} %Lösung von Unterpunkt1
\Subitem{$\displaystyle\int^x_{30} v_2(t)\dx[t]=2\,543,8\ldots$

$x=531,7\ldots\approx 532$

Die Zeitdauer des gesamten Fallschirmsprungs beträgt ca. 532\,s} %%Lösung von Unterpunkt2

\setcounter{subitemcounter}{0}
\subsection{Lösungsschlüssel:}
 
\Subitem{Ein Punkt für die richtige Lösung, wobei die Einheit "`m"' nicht angegeben sein muss.} %Lösungschlüssel von Unterpunkt1
\Subitem{Ein Punkt für die richtige Lösung, wobei die Einheit "`s"' nicht angegeben sein muss.} %Lösungschlüssel von Unterpunkt2

\item \subsection{Lösungserwartung:} 

\Subitem{mögliche Vorgehensweise:

$9,81\cdot 9 - v_1(9)=38,192\ldots\approx 38,19$

$v_1(9)$ ist um ca. 38,19\,m/s kleiner als $v^*$} %Lösung von Unterpunkt1
\Subitem{mögliche Vorgehensweise:

$\dfrac{9,81-v_1'(9)}{9,81}=0,84958\ldots\approx 0,8496$

Die Beschleunigung unter Berücksichtigung des Luftwiderstands ist um ca. 84,96\,\% geringer als jene ohne Berücksichtigung des Luftwiderstands.} %%Lösung von Unterpunkt2

\setcounter{subitemcounter}{0}
\subsection{Lösungsschlüssel:}
 
\Subitem{Ein Punkt für die richtige Lösung, wobei die Einheit "`m/s"' nicht angegeben sein muss.} %Lösungschlüssel von Unterpunkt1
\Subitem{Ein Punkt für die richtige Lösung. Andere Schreibweisen der Lösung sind ebenfalls als richtig zu werten.} %Lösungschlüssel von Unterpunkt2

\end{loesung}

\antwort{GK/Themen: AN 1.1, AN 1.3, AN 4.3}
\end{langesbeispiel}