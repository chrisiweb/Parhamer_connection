\section{127 - K7 - AN 3.3, FA 4.3 - Pegelstand - VerSie}

\begin{langesbeispiel} \item[6] %PUNKTE DES BEISPIELS
Starke Regenfälle haben Auswirkungen auf den Pegelstand eines Flusses.\\
Die Funktion $P$ beschreibt die Höhe des Pegelstandes im Zeitintervall $[0;t_A]$. Wobei $t_A$ jenen Zeitpunkt beschreibt, zu dem der Pegelstand wieder auf die zum Zeitpunkt $t=0$ gegebene Anfangshöhe gefallen ist.\\
$P(t)$ gibt die Höhe des Pegelstandes in m an, die Zeit $t$ wird in Stunden gemessen. Für $P$ ist eine geeignete Modellfunktion zu ermitteln, dazu soll unter den anschließend angeführten Funktionen die geeignetste ausgewählt werden.

$P_1(t)=-\frac{1}{16}t^2+\frac{1}{2}t+1$

$P_2(t)=\frac{1}{32}(-t^3+6t^2+32)$

$P_3(t)=\frac{1}{108}(t^3-18t^2+81t+108)$\leer

Nachfolgende Bedingungen sind zu berücksichtigen:
\singlespacing
\begin{enumerate}[1)]
\item Zum Zeitpunkt $t=0$ nimmt der Pegelstand den (üblichen) Wert von 1\,m an.
\item Zum Zeitpunkt $t=4$ erreicht die Pegelhöhe ein Maximum.
\item Im Zeitintervall $[0;t_A]$ wechselt der Pegelstand genau einmal das Monotonieverhalten.
\item Das Zeitintervall $[0;4]$ ist länger als das Zeitintervall $[4;t_A]$.
\end{enumerate}
\onehalfspacing%Aufgabentext

\begin{aufgabenstellung}
\item %Aufgabentext

\Subitem{Arbeite dabei mit nebenstehender Tabelle und kreuze an, welche Bedingungen die einzelnen Funktionen erfüllen. Entscheide danach, welche von den drei Funktionen am geeignetsten ist.
	\begin{center}
	\begin{tabular}{|p{1,5cm}|C{1,5cm}|C{1,5cm}|C{1,5cm}|C{1,5cm}|}\hline
	&(1)&(2)&(3)&(4)\\ \hline
	$P_1$&$\times$&$\times$&$\times$&\\ \hline
	$P_2$&\antwort{$\times$}&\antwort{$\times$}&\antwort{$\times$}&\antwort{$\times$}\\ \hline
	$P_3$&\antwort{$\times$}&&\antwort{$\times$}&\\ \hline
	\end{tabular}
	\end{center}} %Unterpunkt1
\Subitem{Begründe rechnerisch, warum für die Funktion $P_1$ die zweite Aussage richtig ist.} %Unterpunkt2

\item Verwende für die Aufgabenstellung die von dir gewählte Modellfunktion.%Aufgabentext

\Subitem{Veranschauliche den Graphen der gewählten Modellfunktion im untenstehenden Koordinatensystem mit geeignet skalierten Koordinatenachsen.

\begin{center}
\psset{xunit=0.8cm,yunit=0.8cm,algebraic=true,dimen=middle,dotstyle=o,dotsize=5pt 0,linewidth=1.6pt,arrowsize=3pt 2,arrowinset=0.25}	
\begin{pspicture*}(-5.76,-5.42)(7.76,5.62)
\psaxes[labelFontSize=\scriptstyle,showorigin=false,xAxis=true,yAxis=true,Dx=1.,Dy=1.,labels=none,ticks=none]{->}(0,0)(-5.76,-5.42)(7.76,5.62)
\end{pspicture*}
	\end{center}} %Unterpunkt1
\Subitem{Ermittle die maximale Pegelhöhe.} %Unterpunkt2

\item Verwende weiterhin für die Aufgabenstellung die von dir gewählte Modellfunktion.%Aufgabentext

\Subitem{Ermittle den Zeitpunkt $t_A$.} %Unterpunkt1
\Subitem{Ermittle, zu welchem Zeitpunkt der Pegelstand mit der größten Geschwindigkeit ansteigt.} %Unterpunkt2

\end{aufgabenstellung}

\begin{loesung}
\item \subsection{Lösungserwartung:} 

\Subitem{Tabelle: siehe oben.} %Lösung von Unterpunkt1
\Subitem{$P_1'(t)=-\frac{1}{8}t+\frac{1}{2}$

$P_1'(t)=0 \Rightarrow 0=-\frac{1}{8}t+\frac{1}{2} \Rightarrow \frac{1}{8}t=\frac{1}{2}\Rightarrow t=4$

An der stelle $t=4$, hat die Funktion $P_1$ eine Extremstelle.

$P_1''(t)=\frac{-1}{8} \Rightarrow$ es muss sich dabei um eine Maximumstelle handeln.} %%Lösung von Unterpunkt2

\setcounter{subitemcounter}{0}
\subsection{Lösungsschlüssel:}
 
\Subitem{Ein Punkt für die richtig ausgefüllte Tabelle.} %Lösungschlüssel von Unterpunkt1
\Subitem{Ein Punkt für den rechnerischen Nachweis.} %Lösungschlüssel von Unterpunkt2

\item \subsection{Lösungserwartung:} 

\Subitem{Graph von $P_2$:

\psset{xunit=0.8cm,yunit=0.8cm,algebraic=true,dimen=middle,dotstyle=o,dotsize=5pt 0,linewidth=1.6pt,arrowsize=3pt 2,arrowinset=0.25}	
\begin{pspicture*}(-5.76,-5.42)(7.76,5.62)
\multips(0,-5)(0,1.0){12}{\psline[linestyle=dashed,linecap=1,dash=1.5pt 1.5pt,linewidth=0.4pt,linecolor=gray]{c-c}(-5.76,0)(7.76,0)}
\multips(-5,0)(1.0,0){14}{\psline[linestyle=dashed,linecap=1,dash=1.5pt 1.5pt,linewidth=0.4pt,linecolor=gray]{c-c}(0,-5.42)(0,5.62)}
\psaxes[labelFontSize=\scriptstyle,showorigin=false,xAxis=true,yAxis=true,Dx=1.,Dy=1.,ticksize=-2pt 0,subticks=0]{->}(0,0)(-5.76,-5.42)(7.76,5.62)
\psplot[linewidth=2.pt,plotpoints=200]{-5.760000000000003}{7.760000000000002}{1.0/32.0*(-x^(3.0)+6.0*x^(2.0)+32.0)}
\begin{scriptsize}
\rput[bl](-3.2,4.3){$P_2$}
\end{scriptsize}
\end{pspicture*}
} %Lösung von Unterpunkt1
\Subitem{Maximale Pegelhöhe: $P_2(4)=2$} %%Lösung von Unterpunkt2

\setcounter{subitemcounter}{0}
\subsection{Lösungsschlüssel:}
 
\Subitem{Ein Punkt für den richtigen Graph mit korrekter Achsenbeschriftung.} %Lösungschlüssel von Unterpunkt1
\Subitem{Ein Punkt für die Berechnung der maximalen Pegelhöhe.} %Lösungschlüssel von Unterpunkt2

\item \subsection{Lösungserwartung:} 

\Subitem{$P_2(t)=1 \Rightarrow t_1=0; t_2=6$.
	
	$t_A=6$} %Lösung von Unterpunkt1
\Subitem{$P_2'(t)=\frac{1}{32}(-3t^2+12t) \Rightarrow P_2''(t)=\frac{1}{32}(-6t+12)$
	
	$P_2''(t)=\frac{1}{32}(-6t+12)=0 \Rightarrow t=2$
	
	Zwei Stunden nach Beginn steigt der Pegelstand mit der größten Geschwindigkeit an.
} %%Lösung von Unterpunkt2

\setcounter{subitemcounter}{0}
\subsection{Lösungsschlüssel:}
 
\Subitem{Ein Punkt für die richtige Berechnung von $t_A$.} %Lösungschlüssel von Unterpunkt1
\Subitem{Ein Punkt für die richtige Berechnung der Zeitpunkts.} %Lösungschlüssel von Unterpunkt2

\end{loesung}

\end{langesbeispiel}