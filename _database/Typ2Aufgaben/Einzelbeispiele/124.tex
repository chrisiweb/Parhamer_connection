\section{124 - K6 - AG 1.2, AG 2.1, FA 5.5 - Rutherford Experiment - VerSie}

\begin{langesbeispiel} \item[8] %PUNKTE DES BEISPIELS
Der englische Physiker E. Rutherford erforschte den Bau der Atome. Er bestrahlte eine dünne Goldfolie mit $\alpha$-Strahlen (He$^{2+}$ -Teilchen, bestehend aus 2 Protonen und 2 Neutronen), um weitere Aufschlüsse über den Bau der Atome zu erhalten. 

Als Strahlungsquelle diente ihm Radium. Das Radium befand sich in einem Bleiblock, aus dem die positiv geladenen $\alpha$-Teilchen  durch eine Öffnung gebündelt austreten konnten. Durch den luftleeren Raum wurden diese Teilchen auf eine $ 0,5\, \mu$m dünne Goldfolie geschossen und rund um diese Folie war ein Fotoschirm aufgebaut, um die Strahlen aufzufangen.

Die extrem seltene Ablenkung der $\alpha$-Teilchen und deren Winkelverteilung lassen sich dadurch verstehen, dass sich in den Atomen nur ein sehr kleines Massezentrum befindet, das positiv geladen ist. Man nennt dieses Massezentrum den Atomkern, er enthält die Protonen und Neutronen. Da die meisten Teilchen die Goldfolie ungehindert passieren, muss zwischen den Kernen ein großer Freiraum bestehen. Dieses Ergebnis nennt man das rutherfordsche Atommodell.%Aufgabentext

\begin{aufgabenstellung}
\item Goldatome sind kugelförmige Teilchen mit einem Radius von  144\,pm.%Aufgabentext

\Subitem{Wie viele dieser Goldatome aneinandergereiht ergeben eine $ 0,5 \, \mu$m dicke Folie?\\
(Hilfestellung: Mikro $\mu = 10^{-6}$ und Piko $\text{p}  = 10^{-12}$)} %Unterpunkt1
\Subitem{Berechne das Volumen eines solchen Goldatoms. Kreuze das richtige Ergebnis an.

\multiplechoice[5]{  %Anzahl der Antwortmoeglichkeiten, Standard: 5
				L1={$V\approx 1,25\cdot 10^{-29}$\,m$^3$},   %1. Antwortmoeglichkeit 
				L2={$V\approx 1,25\cdot 10^{-20}$\,m$^3$},   %2. Antwortmoeglichkeit
				L3={$V\approx 1,25\cdot 10^{-39}$\,m$^3$},   %3. Antwortmoeglichkeit
				L4={$V\approx 1,25\cdot 10^{-12}$\,m$^3$},   %4. Antwortmoeglichkeit
				L5={$V\approx 1,25\cdot 10^{-6}$\,m$^3$},	 %5. Antwortmoeglichkeit
				L6={$V\approx 1,25\cdot 10^{-36}$\,m$^3$},	 %6. Antwortmoeglichkeit
				L7={},	 %7. Antwortmoeglichkeit
				L8={},	 %8. Antwortmoeglichkeit
				L9={},	 %9. Antwortmoeglichkeit
				%% LOESUNG: %%
				A1=6,  % 1. Antwort
				A2=0,	 % 2. Antwort
				A3=0,  % 3. Antwort
				A4=0,  % 4. Antwort
				A5=0,  % 5. Antwort
				}} %Unterpunkt2

\item Die Masse eines Protons im Kern beträgt $1,6724 \cdot  10^{-24} \,$g, ein Neutron im Kern wiegt eben soviel. Ein Elektron in der Hülle ist ca. 2000 Mal leichter.
%Aufgabentext

\Subitem{Berechne das Gewicht eines Elektrons.} %Unterpunkt1
\Subitem{Berechne das Gewicht eines He$^{2+}$ -Teilchen.} %Unterpunkt2

\item Rutherford verwendete für seinen Versuch 1g Radium, das entspricht \mbox{$2,6646\cdot 10^{21}$} Teilchen Radium. Radium zerfällt nach folgendem Zerfallsgesetz \\
\mbox{$N_t = N_0\cdot e^{-0,000432676\cdot t}$} ($t$ in Jahren).%Aufgabentext

\Subitem{Berechne wie viele Teilchen nach zehn Jahren noch vorhanden sind.} %Unterpunkt1
\Subitem{Berechne, wann die Hälfte der zu Beginn vorhandenen Radium Teilchen zerfallen sind.} %Unterpunkt2

\item

\Subitem{Wie viel Prozent der ursprünglichen Teilchen sind nach 3200 Jahren noch übrig?}
\Subitem{Gib an, nach wie vielen ganzen Jahren weniger als 1\,000\,000 Teilchen des vorhanden Radiums übrig sind.}

\end{aufgabenstellung}

\begin{loesung}
\item \subsection{Lösungserwartung:} 

\Subitem{$\frac{0,5\cdot 10^{-6}}{288\cdot 10^{-12}}\approx 1736,1$ Teilchen} %Lösung von Unterpunkt1
\Subitem{$V=\frac{4}{3}\cdot\pi\cdot r^3=\frac{4}{3}\cdot\pi\cdot \left(144\cdot 10^{-12}\right)^3=\frac{4}{3}\cdot\pi\cdot 2985984\cdot 10^{-36}\approx 12507660,53\cdot 10^{-36}\approx 1,25\cdot 10^{-29}$\,m$^3$} %%Lösung von Unterpunkt2

\setcounter{subitemcounter}{0}
\subsection{Lösungsschlüssel:}
 
\Subitem{Ein Punkt für die korrekte Anzahl der Teilchen.} %Lösungschlüssel von Unterpunkt1
\Subitem{Ein Punkt für die richtige MC-Antwort.} %Lösungschlüssel von Unterpunkt2

\item \subsection{Lösungserwartung:} 

\Subitem{$1,6724 \cdot  10^{-24} : 2000= 8,362 \cdot 10^{-28}$} %Lösung von Unterpunkt1
\Subitem{$1,6724 \cdot  10^{-24} \cdot 4= 6,6896 \cdot 10^{-24}$} %%Lösung von Unterpunkt2

\setcounter{subitemcounter}{0}
\subsection{Lösungsschlüssel:}
 
\Subitem{Ein Punkt für das richtige Gewicht eines Elektrons.} %Lösungschlüssel von Unterpunkt1
\Subitem{Ein Punkt für das richtige Gewicht eines HE$^{2+}$-Teilchens.} %Lösungschlüssel von Unterpunkt2

\item \subsection{Lösungserwartung:} 

\Subitem{$N_{10} = N_0\cdot e^{-0,000432676\cdot 10}=2,65\cdot 10^{21}$} %Lösung von Unterpunkt1
\Subitem{$\frac{1}{2} = e^{-0,000432676\cdot t}\Rightarrow t\approx 1602$ Jahre} %%Lösung von Unterpunkt2

\setcounter{subitemcounter}{0}
\subsection{Lösungsschlüssel:}
 
\Subitem{Ein Punkt für die richtige Anzahl an Teilchen.} %Lösungschlüssel von Unterpunkt1
\Subitem{Ein Punkt für die richtige Anzahl an Jahren.} %Lösungschlüssel von Unterpunkt2

\item \subsection{Lösungserwartung:} 

\Subitem{$N_{3200} = N_0\cdot e^{-0,000432676\cdot 3200}=6,67\cdot 10^{20} \Rightarrow \frac{6,67\cdot 10^{20}}{2,6646\cdot 10^{21}}\cdot 100\approx 25\%$} %Lösung von Unterpunkt1
\Subitem{$1\,000\,000 = N_0\cdot e^{-0,000432676\cdot t} \Rightarrow t\approx 82091,06$ Jahre} %%Lösung von Unterpunkt2

\setcounter{subitemcounter}{0}
\subsection{Lösungsschlüssel:}
 
\Subitem{Ein Punkt für die richtige Prozentzahl.} %Lösungschlüssel von Unterpunkt1
\Subitem{Ein Punkt korrekte Jahresanzahl.} %Lösungschlüssel von Unterpunkt2

\end{loesung}

\end{langesbeispiel}