\section{55 - MAT - FA 3.2, AN 3.2, FA 2.2, FA 2.3, AN 1.2 - Schiefer Turm von Pisa - Matura 2014/15 2. Nebentermin}

\begin{langesbeispiel} \item[0] %PUNKTE DES BEISPIELS
				
	Der Schiefe Turm von Pisa zählt zu den bekanntesten Gebäuden der Welt. Historisch nicht verbürgt sind Galileo Galileis (1564 - 1642) Fallversuche aus verschiedenen Höhen des Schiefen Turms von Pisa. Tatsache ist jedoch, dass Galilei die Gesetze des freien Falls erforscht hat. Die Fallzeit eines Körpers aus der Höhe $h_0$ ist bei Vernachlässigung des Luftwiderstandes (im Vakuum) unabhängig von seiner Form und seiner Masse. 
	
	Modellhaft kann die Höhe des fallenden Körpers in Abhängigkeit von der Zeit näherungsweise durch die Funktion $h$ mit der Gleichung $h(t)=h_0-5t^2$ beschrieben werden. 
	
	Die Höhe $h(t)$ wird in Metern und die Zeit $t$ in Sekunden gemessen.


\subsection{Aufgabenstellung:}
\begin{enumerate}
	\item Ein Körper fällt im Vakuum aus einer Höhe $h_0=45$\,m.
	
	\fbox{A} Berechne seine Geschwindigkeit in m/s zum Zeitpunkt $t_1$ des Aufpralls!\leer
	
	Begründe, warum der Betrag der Geschwindigkeit dieses Körpers im Intervall $[0;t_1]$ monoton steigt!
	
\item In der unten stehenden Abbildung ist der Graph der Funktion $h$ für $h_0=45$\,m dargestellt.  Bestimmen Sie die Steigung der Sekante $s$ durch die Punkte $A=(0|45)$ und $B=(3|0)$ und  deuten Sie diesen Wert im Hinblick auf die Bewegung des Körpers!

\begin{center}
	\resizebox{0.8\linewidth}{!}{\psset{xunit=1.0cm,yunit=0.15cm,algebraic=true,dimen=middle,dotstyle=o,dotsize=4pt 0,linewidth=0.8pt,arrowsize=3pt 2,arrowinset=0.25}
\begin{pspicture*}(-1.34,-5.087368421052554)(8.9,59.446842105262796)
\multips(0,0)(0,5.0){13}{\psline[linestyle=dashed,linecap=1,dash=1.5pt 1.5pt,linewidth=0.4pt,linecolor=lightgray]{c-c}(0,0)(8.9,0)}
\multips(0,0)(1.0,0){11}{\psline[linestyle=dashed,linecap=1,dash=1.5pt 1.5pt,linewidth=0.4pt,linecolor=lightgray]{c-c}(0,0)(0,59.446842105262796)}
\psaxes[labelFontSize=\scriptstyle,xAxis=true,yAxis=true,Dx=1.,Dy=5.,ticksize=-2pt 0,subticks=2]{->}(0,0)(0.,0.)(8.9,59.446842105262796)
\rput[tl](0.3,57.18578947368386){$h(t)$ in m}
\rput[tl](7.8,2.732105263157919){$t$ in s}
\psplot[linewidth=1.2pt,plotpoints=200]{0}{3}{-5*x^2+45}
\psline[linewidth=1.2pt](3.,0.)(0.,45.)
\rput[tl](2.92,8.102105263157883){$h$}
\psline[linewidth=1.2pt](0.5907591988821798,47.388612016767304)(2.9837377736376363,11.493933395435455)
\rput[tl](2.6,22.79894736842094){$t$}
\begin{scriptsize}
\psdots[dotstyle=*](0.,45.)
\rput[bl](0.08,45.78631578947341){A}
\psdots[dotstyle=*](3.,0.)
\rput[bl](3.08,0.7536842105263535){B}
\psdots[dotstyle=*,linecolor=darkgray](1.5,33.75)
\rput[bl](1.58,34.481052631578756){P}
\rput[bl](1.12,22.327894736841994){s}
\end{scriptsize}
\end{pspicture*}}
\end{center}

Die Tangente $t$ im Punkt $P=(1,5|h(1,5))$ ist parallel zur Sekante $s$. Interpretiere diese Tatsache im Hinblick auf die Bewegung des Körpers.
						\end{enumerate}\leer
				
\antwort{
\begin{enumerate}
	\item \subsection{Lösungserwartung:} 
	
	Zeit-Weg-Funktion $h(t)=45-5t^2$
	
	$0=45-5t^2$
	
	$t_1=3$
	
	Geschwindigkeitsfunktion $v(t)=h'(t)=-10t$
	
	$v(3)=h'(3)=-30$
	
	Der Betrag der Geschwindigkeit zum Zeitpunkt des Aufpralls beträgt $30$\,m/s.\leer
	
	Die Geschwindigkeitsfunktion ist eine lineare Funktion, die im Intervall $[0\,s; 3\,s]$ von  $v(0)=h'(0)=0$ ausgehend monoton fallend ist - daher wird der Betrag der Geschwindigkeit immer größer.\leer
	
	Die Bewegung ist gleichmäßig beschleunigt - das heißt, der Betrag der Geschwindigkeit ist monoton wachsend.
		
	\subsection{Lösungsschlüssel:}
	\begin{itemize}
		\item Ein Ausgleichspunkt für die richtige Lösung, wobei die Einheit "`m/s"' nicht angeführt sein muss und auch -30 m/s als korrekt zu werten ist.  
		
		Die Aufgabe ist auch dann als richtig gelöst zu werten, wenn bei korrektem Ansatz das Ergebnis aufgrund eines Rechenfehlers nicht richtig ist.
		\item  Ein Punkt für eine (sinngemäß) korrekte Begründung.
	\end{itemize}
	
	\item \subsection{Lösungserwartung:}
			
		Die Steigung der Sekante beträgt -15.
		
		Das bedeutet, dass der Betrag der Durchschnittsgeschwindigkeit bei der Bewegung des Körpers im Zeitraum von 0 Sekunden bis 3 Sekunden 15\,m/s beträgt.\leer
		
		Mögliche Interpretation:
		
		Der Betrag der Momentangeschwindigkeit ist zum Zeitpunkt $t=1,5$ gleich groß wie die Durchschnittsgeschwindigkeit des Körpers im Intervall $[0\,s; 3\,s]$.


	\subsection{Lösungsschlüssel:}
	
\begin{itemize}
	\item  Ein Punkt für eine korrekte Bestimmung der Sekantensteigung und eine (sinngemäß) korrekte Deutung.
	\item Ein Punkt für eine (sinngemäß) korrekte Interpretation.
\end{itemize}

\end{enumerate}}
		\end{langesbeispiel}