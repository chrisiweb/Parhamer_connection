\section{61 - MAT - WS 3.1, WS 3.2, WS 3.4,  - Würfel mit unterschiedlichen Zahlen - Matura 2015/16 Haupttermin}

\begin{langesbeispiel} \item[0] %PUNKTE DES BEISPIELS
	
 Gegeben sind die Netze von drei fairen Würfeln, deren Seitenflächen auf unterschiedliche Weise mit verschiedenen Zahlen beschriftet sind. (Ein Würfel ist "`fair"', wenn die Wahrscheinlichkeit, nach einem Wurf nach oben zu zeigen, für alle Seitenflächen gleich groß ist.)

\begin{center}
	\resizebox{0.9\linewidth}{!}{\psset{xunit=1.0cm,yunit=1.0cm,algebraic=true,dimen=middle,dotstyle=o,dotsize=4pt 0,linewidth=0.8pt,arrowsize=3pt 2,arrowinset=0.25}
\begin{pspicture*}(0.17804133915570747,4.040102465944372)(15.692206630376225,9.148214730906863)
\psline[linewidth=0.8pt](1.,7.)(1.,6.)
\psline[linewidth=0.8pt](1.,6.)(2.,6.)
\psline[linewidth=0.8pt](2.,6.)(2.,7.)
\psline[linewidth=0.8pt](2.,7.)(1.,7.)
\psline[linewidth=0.8pt](2.,6.)(3.,6.)
\psline[linewidth=0.8pt](3.,6.)(3.,5.)
\psline[linewidth=0.8pt](3.,5.)(4.,5.)
\psline[linewidth=0.8pt](4.,5.)(4.,6.)
\psline[linewidth=0.8pt](4.,6.)(5.,6.)
\psline[linewidth=0.8pt](5.,6.)(5.,7.)
\psline[linewidth=0.8pt](5.,7.)(4.,7.)
\psline[linewidth=0.8pt](4.,7.)(4.,8.)
\psline[linewidth=0.8pt](4.,8.)(3.,8.)
\psline[linewidth=0.8pt](3.,8.)(3.,7.)
\psline[linewidth=0.8pt](3.,7.)(2.,7.)
\psline[linewidth=0.8pt](3.,7.)(3.,6.)
\psline[linewidth=0.8pt](3.,7.)(4.,7.)
\psline[linewidth=0.8pt](3.,6.)(4.,6.)
\psline[linewidth=0.8pt](4.,7.)(4.,6.)
\psline[linewidth=0.8pt](5.94,6.97)(5.94,5.97)
\psline[linewidth=0.8pt](5.94,5.97)(6.94,5.97)
\psline[linewidth=0.8pt](6.94,5.97)(6.94,6.97)
\psline[linewidth=0.8pt](6.94,6.97)(5.94,6.97)
\psline[linewidth=0.8pt](6.94,5.97)(7.94,5.97)
\psline[linewidth=0.8pt](7.94,5.97)(7.94,4.97)
\psline[linewidth=0.8pt](7.94,4.97)(8.94,4.97)
\psline[linewidth=0.8pt](8.94,4.97)(8.94,5.97)
\psline[linewidth=0.8pt](8.94,5.97)(9.94,5.97)
\psline[linewidth=0.8pt](9.94,5.97)(9.94,6.97)
\psline[linewidth=0.8pt](9.94,6.97)(8.94,6.97)
\psline[linewidth=0.8pt](8.94,6.97)(8.94,7.97)
\psline[linewidth=0.8pt](8.94,7.97)(7.94,7.97)
\psline[linewidth=0.8pt](7.94,7.97)(7.94,6.97)
\psline[linewidth=0.8pt](7.94,6.97)(6.94,6.97)
\psline[linewidth=0.8pt](7.94,6.97)(7.94,5.97)
\psline[linewidth=0.8pt](7.94,6.97)(8.94,6.97)
\psline[linewidth=0.8pt](7.94,5.97)(8.94,5.97)
\psline[linewidth=0.8pt](8.94,6.97)(8.94,5.97)
\psline[linewidth=0.8pt](10.96,6.95)(10.96,5.95)
\psline[linewidth=0.8pt](10.96,5.95)(11.96,5.95)
\psline[linewidth=0.8pt](11.96,5.95)(11.96,6.95)
\psline[linewidth=0.8pt](11.96,6.95)(10.96,6.95)
\psline[linewidth=0.8pt](11.96,5.95)(12.96,5.95)
\psline[linewidth=0.8pt](12.96,5.95)(12.96,4.95)
\psline[linewidth=0.8pt](12.96,4.95)(13.96,4.95)
\psline[linewidth=0.8pt](13.96,4.95)(13.96,5.95)
\psline[linewidth=0.8pt](13.96,5.95)(14.96,5.95)
\psline[linewidth=0.8pt](14.96,5.95)(14.96,6.95)
\psline[linewidth=0.8pt](14.96,6.95)(13.96,6.95)
\psline[linewidth=0.8pt](13.96,6.95)(13.96,7.95)
\psline[linewidth=0.8pt](13.96,7.95)(12.96,7.95)
\psline[linewidth=0.8pt](12.96,7.95)(12.96,6.95)
\psline[linewidth=0.8pt](12.96,6.95)(11.96,6.95)
\psline[linewidth=0.8pt](12.96,6.95)(12.96,5.95)
\psline[linewidth=0.8pt](12.96,6.95)(13.96,6.95)
\psline[linewidth=0.8pt](12.96,5.95)(13.96,5.95)
\psline[linewidth=0.8pt](13.96,6.95)(13.96,5.95)
\rput[tl](1.4,6.6){1}
\rput[tl](2.4,6.6){2}
\rput[tl](3.4,6.6){1}
\rput[tl](4.4,6.6){2}
\rput[tl](3.4,7.6){3}
\rput[tl](3.4,5.6){3}
\rput[tl](6.3,6.6){-1}
\rput[tl](7.35,6.6){2}
\rput[tl](8.3,6.6){-1}
\rput[tl](9.35,6.6){2}
\rput[tl](8.35,7.6){5}
\rput[tl](8.35,5.6){5}
\rput[tl](11.4,6.6){0}
\rput[tl](12.4,6.6){0}
\rput[tl](13.4,6.6){6}
\rput[tl](13.4,5.6){6}
\rput[tl](13.4,7.6){0}
\rput[tl](14.4,6.6){6}
\rput[tl](2.6,8.665014922059061){Würfel $A$}
\rput[tl](7.6,8.665014922059061){Würfel $B$}
\rput[tl](12.6,8.665014922059061){Würfel $C$}
\end{pspicture*}}
\end{center}



\subsection{Aufgabenstellung:}
\begin{enumerate}
	\item Herr Fischer wirft Würfel $A$ zweimal. Die Zufallsvariable $X$ gibt die Summe der beiden geworfenen Zahlen an. Die Zufallsvariable $X$ kann die Werte 2,3,4,5 und 6 annehmen.
	
	Frau Fischer wirft die Würfel $A$ und $B$. Die Zufallsvariable $Y$ gibt die Summe der beiden geworfenen Zahlen an.\leer
	
	\fbox{A} Gib für die Zufallsvariable $Y$ alle möglichen Werte an!
	
	mögliche Wert für $Y$: \rule{5cm}{0.3pt}
	
	Es gibt Werte der Zufallsvariablen, die bei Herrn Fischer wahrscheinlicher auftreten als bei Frau Fischer. Gib denjenigen Wert an, bei dem der Unterschied der beiden Wahrscheinlichkeiten am größten ist, und berechne diesen Unterschied!
	
	\item Bei einem Spiel wird Würfel $B$ dreimal geworfen. Der Einsatz des Spiels für eine Spielerin/einen Spieler beträgt \EUR{2}. Die jeweilige Auszahlung ist von der Summe der drei geworfenen Zahlen abhängig und wird in der nachstehenden Tabelle teilweise angegeben
	
	\begin{center}
		\begin{tabular}{|c|c|}\hline
		\cellcolor[gray]{0.9}Summe der drei geworfenen Zahlen&\cellcolor[gray]{0.9}Auszahlung an die Spielerin/den Spieler\\ \hline
		positiv&0\\ \hline
		null&2\\ \hline
		negativ&?\\ \hline		
		\end{tabular}
	\end{center}

Eine Person spielt dieses Spiel fünfmal. Berechne die Wahrscheinlichkeit, dass dabei genau zweimal die Summe der drei geworfenen Zahlen genau null ist!

 Berechne, welchen Betrag der Anbieter des Spiels für das Würfeln einer negativen Summe höchstens auszahlen darf, um langfristig mit keinem Verlust rechnen zu müssen! 

\item  Peter wirft den Würfel $C$ 100-mal. Die Zufallsvariable $Z$ beschreibt die Anzahl der gewürfelten Sechser.

 Berechne den Erwartungswert und die Standardabweichung von $Z$!

 Ermittle die Wahrscheinlichkeit, dass die Summe der geworfenen Zahlen größer als 350 ist!    
						\end{enumerate}\leer
				
\antwort{
\begin{enumerate}
	\item \subsection{Lösungserwartung:} 
	
mögliche Werte für $Y$: $0, 1, 2, 3, 4, 5, 6, 7, 8$

Bei $Y$ hat jeder Wert die gleiche Wahrscheinlichkeit $\left(=\frac{1}{9}\right)$, bei $X$ hat 4 die größte Wahrscheinlichkeit $\left(=3\cdot\frac{1}{3}\cdot\frac{1}{3}=\frac{1}{3}\right)$. Der Unterschied ist bei 4 am größten, er beträgt $\frac{2}{9}$.

oder:

Die Wahrscheinlichkeit für 4 ist bei Herrn Fischer dreimal so groß wie bei Frau Fischer.

	\subsection{Lösungsschlüssel:}
	\begin{itemize}
		\item Ein Ausgleichspunkt für die vollständige Angabe der korrekten Werte für $Y$. 
		\item Ein Punkt für die Angabe des gesuchten Wertes und einer korrekten Berechnung des Unterschieds.
	\end{itemize}
	
	\item \subsection{Lösungserwartung:}
			
	Mögliche Berechnung:
	
	Zufallsvariable $X$ = Anzahl der Spiele, bei denen die Summe der drei geworfenen Zahlen genau null ist.
	
	$P(\text{"`Summe der drei geworfenen Zahlen ist null"'})=p=\frac{1}{3}\cdot\frac{1}{3}\cdot\frac{1}{3}\cdot 3=\frac{1}{9}$
	
	Binomialverteilung mit den Parametern $n=5, k=2, p=\frac{1}{9}$
	
	$P(X=2)=\binom{5}{2}\cdot\left(\frac{1}{9}\right)²\cdot\left(\frac{8}{9}\right)³\approx 0,087 \Rightarrow$ Die gesuchte Wahrscheinlichkeit liegt bei ca. 8,7\,\%.
	
	Mögliche Berechnung:
	
	$x$ ... Auszahlung für das Würfeln einer negativen Summe
	
	$2\cdot\frac{1}{9}+x\cdot\frac{1}{27}<2 \Rightarrow x<48$
	
	Die Auszahlung für das Würfeln einer negativen Summe darf höchstens \EUR{48} betragen, damit der Anbieter des Spiels langfristig mit keinem Verlust rechnen muss.

	\subsection{Lösungsschlüssel:}
	
\begin{itemize}
	\item   Ein Punkt für die richtige Lösung. Andere Schreibweisen des Ergebnisses sind ebenfalls als richtig zu werten. 
	
	Toleranzintervall: $[0,08; 0,09]$ bzw. $[8\,\%; 9\,\%]$
	\item  Ein Punkt für die richtige Lösung, wobei die Einheit "`\EUR{ }"' nicht angegeben sein muss. Die Aufgabe ist auch dann als richtig gelöst zu werten, wenn bei korrektem Ansatz das Ergebnis aufgrund eines Rechenfehlers nicht richtig ist.

\end{itemize}

\item \subsection{Lösungserwartung:}
			
$n=100$ und $p=0,5$\leer

Erwartungswert: $E(Z)=50$

Standardabweichung: $\sqrt{V(Z)}=5$

Mögliche Berechnung (z.B. durch Approximation durch die Normalverteilung ohne Stetigkeitskorrektur):  

Die Summe ist größer als 350, wenn die Anzahl der Sechser mindestens 59 ist. Es ist möglich, die (für die Anzahl der Sechser) zugrunde liegende Binomialverteilung mit $n=100$ und $p=0,5$ durch die Normalverteilung mit $\mu=50$ und $\sigma=5$ zu approximieren.

$P(Z\geq 59)\approx 0,036=36\,\%$

	\subsection{Lösungsschlüssel:}
	
\begin{itemize}
	\item Ein Punkt für die Angabe der beiden korrekten Werte für den Erwartungswert und die Standardabweichung.
	\item Ein Punkt für die richtige Lösung, wobei Ergebnisse durch Berechnung mit Stetigkeitskorrektur oder exakt mittels Binomialverteilung ebenfalls als richtig zu werten sind. 
	
	Die Aufgabe ist auch dann als richtig gelöst zu werten, wenn bei korrektem Ansatz das Ergebnis aufgrund eines Rechenfehlers nicht richtig ist. 
	
	Toleranzintervall: $[0,035; 0,045]$ bzw. $[3,5\,\%; 4,5\,\%]$
\end{itemize}

\end{enumerate}}
		\end{langesbeispiel}