\section{59 - MAT - AN 1.1, FA 5.6, FA 2.2, FA 2.5 - ZAMG-Wetterballon - Matura 2015/16 Haupttermin}

\begin{langesbeispiel} \item[0] %PUNKTE DES BEISPIELS
	
 Ein Wetterballon ist ein mit Helium oder Wasserstoff befüllter Ballon, der in der Meteorologie zum Transport von Radiosonden (Messgeräten) verwendet wird. Die Zentralanstalt für Meteorologie und Geodynamik (ZAMG) lässt an 365 Tagen im Jahr zwei Mal am Tag einen Wetterballon von der Wetterstation Hohe Warte aufsteigen. Während des Aufstiegs werden kontinuierlich Messungen von Temperatur, Luftfeuchtigkeit, Luftdruck, Windrichtung und Windgeschwindigkeit durchgeführt.

Die bei einem konkreten Aufstieg eines Wetterballons gemessenen Werte für den Luftdruck und die Temperatur in der Höhe $h$ über dem Meeresspiegel liegen in der nachstehenden Tabelle vor.

\begin{center}
	\begin{tabular}{|c|c|c|}\hline
	\multicolumn{1}{|p{5cm}|}{Die Höhe $h$ des Ballons über dem Meeresspiegel (in m)}&Luftdruck $p$ (in hPa)&Temperatur (in $^\circ$C)\\ \hline
	1\,000&906&1,9\\ \hline
	2\,000&800&-3,3\\ \hline
	3\,000&704&-8,3\\ \hline
	4\,000&618&-14,5\\ \hline
	5\,000&544&-21,9\\ \hline
	6\,000&479&-30,7\\ \hline
	7\,000&421&-39,5\\ \hline
	8\,000&370&-48,3\\ \hline	
	\end{tabular}
\end{center}


\subsection{Aufgabenstellung:}
\begin{enumerate}
	\item \fbox{A} Bestimme die relative (prozentuelle) Änderung des Luftdrucks bei einem Anstieg des Wetterballons von 1 000 m auf 2 000 m!\leer
	
	Die Abhängigkeit des Luftdrucks von der Höhe kann näherungsweise durch eine Exponentialfunktion beschrieben werden. Beschreibe, wie dies anhand obiger Tabelle begründet werden kann!
	
	\item Die Temperatur in Abhängigkeit von der Höhe lässt sich im Höhenintervall $[5\,000\,\text{m}; 8\,000\,\text{m}]$ durch eine lineare Funktion $T$ beschreiben.\leer
	
	Begründe dies anhand der in der obigen Tabelle angegebenen Werte!\leer
	
	Berechne für diese Funktion $T$ mit $T(h)=k\cdot h+d$ die Werte der Parameter $k$ und $d$!
	
	\item Das Volumen des Wetterballons ist näherungsweise indirekt proportional zum Luftdruck $p$. In 1\,000 Metern Höhe hat der Wetterballon ein Volumen von 3\,m$³$.
	
	Beschreibe die funktionale Abhängigkeit des Volumens (in m$³$) vom Luftdruck (in hPa) durch eine Gleichung!
	
	$V(p)=$ \rule{5cm}{0.3pt}
	
	Berechne die absolute Änderung des Ballonvolumens im Höhenintervall $[1\,000\,\text{m}; 2\,000\,\text{m}]$
						\end{enumerate}\leer
				
\antwort{
\begin{enumerate}
	\item \subsection{Lösungserwartung:} 
	
$\frac{800-906}{906}\approx -0,117$

Der Luftdruck nimmt bei diesem Anstieg um ca. $11,7\,\%$ ab.

Eine Exponentialfunktion eignet sich in diesem Fall, da eine gleiche Zunahme der Höhe $h$ stets eine Verminderung des Luftdrucks um den annähernd gleichen Prozentsatz vom jeweiligen Ausgangswert bewirkt (z.B. Höhenzunahme um 1\,000\,m $\Leftrightarrow$ Luftdruckabnahme um ca. 12\,\%).

	\subsection{Lösungsschlüssel:}
	\begin{itemize}
		\item Ein Ausgleichspunkt für die richtige Lösung.  
		
		Toleranzintervall: $[-0,12; -0,115]$ bzw. $[-12\,\%;-11,5\,\%]$ 
		\item Ein Punkt für eine (sinngemäß) korrekte Begründung.
	\end{itemize}
	
	\item \subsection{Lösungserwartung:}
			
	Eine lineare Funktion eignet sich in diesem Fall, da eine gleiche Zunahme der Höhe $h$ stets eine gleiche Verminderung der Temperatur vom jeweiligen Ausgangswert bewirkt (z.B. Höhenzunahme um 1\,000\,m $\Leftrightarrow$ Temperaturverminderung um $8,8\,^\circ$C). 
	
	$k=-0,0088$ 
	
	$d=22,1$

	\subsection{Lösungsschlüssel:}
	
\begin{itemize}
	\item Ein Punkt für eine (sinngemäß) korrekte Begründung.
	\item   Ein Punkt die korrekte Angabe beider Parameterwerte $k$ und $d$. 
	
	Toleranzintervall für $k: [-0,009; -0,0088]$
\end{itemize}

\item \subsection{Lösungserwartung:}
			
	$V(p)=\frac{2\,718}{p}$
	
	$V(800)-V(906)=0,3975$
	
	Die absolute Änderung des Ballonvolumens in diesem Höhenintervall beträgt $0,3975\,$m$³$.
	\subsection{Lösungsschlüssel:}
	
\begin{itemize}
	\item Ein Punkt für eine korrekte Gleichung. Äquivalente Gleichungen sind als richtig zu werten
	\item Ein Punkt für die richtige Lösung, wobei die Einheit "`m$³$"' nicht angeführt sein muss.  
	
	Toleranzintervall: $[0,39\,\text{m}³; 0,4\,\text{m}³]$
\end{itemize}

\end{enumerate}}
		\end{langesbeispiel}