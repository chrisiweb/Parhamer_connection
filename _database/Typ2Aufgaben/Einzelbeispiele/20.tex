\section{20 - MAT - AN 1.1, AN 1.3, FA 1.1, FA 2.2, FA 3.1 - Höhe der Schneedecke - BIFIE Aufgabensammlung}

\begin{langesbeispiel} \item[0] %PUNKTE DES BEISPIELS
				Die Höhe $H(t)$ einer Schneedecke nimmt aufgrund von Witterungseinflüssen mit der Zeit $t$ ab. Zuerst ist die Abnahme gering, mit der Zeit wird sie aber immer stärker. Daher kann die Höhe der Schneedecke durch folgende quadratische Funktion $H(t)$ beschrieben werden:
				\begin{center} $H(t)=H_0-a\cdot t^2$ mit $a>0,t\geq 0$\end{center}
($H$ wird in cm und $t$ in Tagen gemessen, $H_0$ beschreibt die Schneehöhe zu Beginn der Messung)

Das beschriebene Modell gilt in guter Näherung bei einer Witterung mit gleichbleibender Temperatur bis zur vollständigen Schneeschmelze. Dabei wird vorausgesetzt, dass bis zur vollständigen Schneeschmelze kein weiterer Schnee hinzukommt.

\subsection{Aufgabenstellung:}
\begin{enumerate}
	\item Eine 20 cm dicke Schneedecke reduziert sich innerhalb von 12 Stunden auf 18 cm. Nach wie vielen Tagen (von Anfang an) ist der Schnee gänzlich geschmolzen? Gib die Lösung auf zwei Dezimalstellen genau an! 
	
Wie wirkt sich eine Erhöhung des Parameters $a$ auf $H(t)$ aus? Begründe deine Antwort! 

\item In einem Alpendorf gilt für die Schneehöhe $H$ (gemessen in cm) und die Zeit $t$ (gemessen in Tagen) der folgende funktionale Zusammenhang:
\begin{center}$H(t)=40-5t^2$\end{center}
Wie hoch ist die mittlere Änderungsrate der Schneehöhe innerhalb der ersten beiden Tage nach Beginn der Messung? Berechne diese!

Begründe, warum die Berechnung der mittleren Änderungsrate im Zeitintervall $[0; 3]$ mithilfe der angegebenen Funktion nicht sinnvoll ist, um Aussagen über den Verlauf der Höhe der Schneedecke zu machen!     
	
	\item Berechne $H'(0,5)$ für $H(t)=H_0-a\cdot t^2$ und $a=3$!
	
	Deute das Ergebnis hinsichtlich der Entwicklung der Schneehöhe $H$!
	
	\item Der nachstehende Graph beschreibt idealisiert den Verlauf der Schneehöhe in Dezimetern innerhalb einer Woche in einem Alpendorf.
	
	\begin{center}
	\resizebox{0.8\linewidth}{!}{\psset{xunit=1.0cm,yunit=1.0cm,algebraic=true,dimen=middle,dotstyle=o,dotsize=5pt 0,linewidth=0.8pt,arrowsize=3pt 2,arrowinset=0.25}
\begin{pspicture*}(-2.6036363636363635,-1.6636363636363598)(7.9163636363636405,7.976363636363644)
\multips(0,-1)(0,0.5){20}{\psline[linestyle=dashed,linecap=1,dash=1.5pt 1.5pt,linewidth=0.4pt,linecolor=lightgray]{c-c}(-2.6036363636363635,0)(7.9163636363636405,0)}
\multips(-2,0)(0.5,0){22}{\psline[linestyle=dashed,linecap=1,dash=1.5pt 1.5pt,linewidth=0.4pt,linecolor=lightgray]{c-c}(0,-1.6636363636363598)(0,7.976363636363644)}
\psaxes[labelFontSize=\scriptstyle,xAxis=true,yAxis=true,Dx=1.,Dy=1.,ticksize=-2pt 0,subticks=2]{->}(0,0)(-2.6036363636363635,-1.6636363636363598)(7.9163636363636405,7.976363636363644)
\psplot[linewidth=2.4pt,plotpoints=200]{0}{2}{-x^(2.0)+4.0}
\psplot[linewidth=2.4pt,plotpoints=200]{2}{3}{0.0}
\psplot[linewidth=2.4pt]{3}{5}{(-6.--2.*x)/2.}
\psplot[linewidth=2.4pt,plotpoints=200]{5}{7}{2.0}
\begin{scriptsize}
\rput[tl](0.2363636363636375,7.456363636363643){Schneehöhe}
\rput[tl](6.2563636363636395,0.596363636363641){Tage}
\end{scriptsize}
\end{pspicture*}}
\end{center}

Handelt es sich bei diesem Graphen um eine auf $[0;7]$ definierte Funktion? Begründe deine Antwort!

Bestimme die Gleichung $y=k\cdot x+d$ mit $k,d\in\mathbb{R}$ einer Funktion $f$, welche den Graphen im Intervall $[3;5]$ beschreibt!
						\end{enumerate}\leer
				
\antwort{\subsection{Lösungserwartung:}
\begin{enumerate}
	\item Frage 1: $18=20-a\cdot 0,5^2 \Rightarrow a=8; 20-8t^2=0 \Rightarrow t=1,58$ Tage
	
	Frage 2: Je größer $a$, desto schneller nimmt die Schneehöhe ab!
	
	\item Frage 1: $\frac{H(2)-H(0)}{2}=\frac{20-40}{2}=-10$ cm/Tag
	
	Frage 2: Der Anwendungsbereich (Definitionsbereich) der Formel $H(t)$ liegt im Bereich $[0;\sqrt{8}]$, wobei $\sqrt{8}\approx 2,8$ die positive Nullstelle von $H(t)$ ist.
	
	Da $[0;2,8]$ eine Teilmenge des Intervalls $[0;3]$ ist, ist die Berechnung des Differenzenquotienten im Intervall $[0;3]$ nicht sinnvoll.
	
	Oder:
	
	An der Stelle $t=3$ wird der Funktionswert $H(t)$ negativ. Die Schneehöhe H kann allerdings nicht negativ sein, daher ist die Berechnung der mittleren Änderungsrate im Intervall $[0;3]$ nicht sinnvoll.
	\item Frage 1: $H(t)=H_0-3\cdot t^2$
	
	$H'(t)=-6\cdot t$
	
	$H'(0,5)=-3$ cm/Tag
	
	Frage 2: Nach $t=0,5$ Tagen nimmt die Höhe der Schneedecke mit einer Geschwindigkeit von 3 cm/Tag ab.
	\item Frage 1: Der Graph beschreibt eine Funktion, da jedem Zeitpunkt $x$ genau eine Schneehöhe $y$ zugeordnet wird.
	
	Frage 2: 
	
	$f(3)=0\Rightarrow 0=k\cdot 3+d$
	
	$f(5)=2\Rightarrow 2=k\cdot 5+d$
	
	daher: $k=1; d=-3$
	
	$y=x-3$
	\end{enumerate}}
		\end{langesbeispiel}