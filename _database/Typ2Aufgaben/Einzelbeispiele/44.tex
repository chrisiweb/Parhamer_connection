\section{44 - MAT - FA 6.4, AN 4.2, AN 4.3 - Atmung - Matura 2013/14 2. Nebentermin}

\begin{langesbeispiel} \item[0] %PUNKTE DES BEISPIELS
				 Beim Ein- bzw. Ausatmen wird Luft in unsere Lungen gesaugt bzw. wieder aus ihnen herausgepresst. Ein Atemzyklus erstreckt sich �ber den gesamten Vorgang des einmaligen Einatmens und anschlie�enden Ausatmens. W�hrend des Atemvorganges l�sst sich das bewegte Luftvolumen messen. Der Luftstrom wird in Litern pro Sekunde angegeben. 
				
				Der Luftstrom $L(t)$ kann in Abh�ngigkeit von der Zeit $t$ ($t$ in Sekunden) n�herungsweise durch die Sinusfunktion $L$ beschrieben werden. Ein Atemzyklus beginnt zum Zeitpunkt $t = 0$ mit dem Einatmen. Nach einer Messung kann der Atemvorgang einer bestimmten Person modellhaft durch folgende Sinusfunktion beschrieben werden: 
				
				$$L(t)=0,6\cdot\sin\left(\frac{\pi}{2}\cdot t\right)$$
	
\subsection{Aufgabenstellung:}
\begin{enumerate}
	\item Ermittle die Periodenl�nge der gegebenen Funktion $L$!
	
	Die Periodenl�nge betr�gt \antwort[\rule{3cm}{0.3pt}]{4} Skunden.
	
	Erkl�re die Bedeutung der Periodenl�nge in Bezug auf den Atemvorgang!

\item \fbox{A} Berechne $\int^2_0{L(t)}$d$t$ und runde das Ergebnis auf zwei Nachkommastellen!

Beschreibe f�r das in Diskussion stehende Problem, was mit dem oben stehenden mathematischen Ausdruck berechnet wird!
						\end{enumerate}\leer
				
\antwort{
\begin{enumerate}
	\item \subsection{L�sungserwartung:} 
	
		Die Periodenl�nge betr�gt 4 Sekunden.
		
		Die Periodenl�nge gibt die Zeitdauer eines Atemzyklus (= einmal Einatmen und einmal Ausatmen) an.
	 	
	\subsection{L�sungsschl�ssel:}
	\begin{itemize}
		\item Ein Punkt f�r die korrekte Ermittlung der Periodenl�nge. Es gen�gt dabei die Angabe des gesuchten Wertes, eine Rechnung oder Zeichnung ist nicht erforderlich.
		\item Ein Punkt f�r eine (sinngem��) korrekte Deutung der Periodenl�nge. Zul�ssig sind auch andere sinngem�� richtige Antworten, die auf den Atemvorgang konkret Bezug nehmen. Ohne konkreten Bezug zum gegebenen Kontext ist die Antwort nicht als korrekt zu werten.
	\end{itemize}
	
	\item \subsection{L�sungserwartung:}
			
		$$\int^2_0{L(t)}dt=\int^2_0{0,6\sin\left(\frac{\pi}{2}\cdot t\right)}dt=\left(-0,6\cdot \frac{2}{\pi}\cdot \cos\left(\frac{\pi}{2}\cdot t\right)\right)\bigg|^2_0\approx 0,76$$
		
		Durch das bestimmte Integral wird das gesamte Luftvolumen (in Litern) berechnet, das w�hrend des Einatmens (in den ersten beiden Sekunden) in die Lunge str�mt.
		
	\subsection{L�sungsschl�ssel:}
	
\begin{itemize}
	\item  Ein Ausgleichspunkt f�r die richtige L�sung. Toleranzintervall: $[0,76; 0,80]$. Die Einheit muss beim Ergebnis nicht zwingend angef�hrt werden.
	\item Ein Punkt f�r eine (sinngem��) korrekte Interpretation. Zul�ssig sind auch andere sinngem�� richtige Antworten, die auf den Atemvorgang konkret Bezug nehmen.
\end{itemize}

\end{enumerate}}
		\end{langesbeispiel}