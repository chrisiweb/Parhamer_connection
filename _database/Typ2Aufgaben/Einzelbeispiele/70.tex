\section{70 - MAT - FA 1.5, AN 4.3, AG 2.3 - Quadratische Funktion - Matura 2016/17 Haupttermin}

\begin{langesbeispiel} \item[0] %PUNKTE DES BEISPIELS
	
Betrachtet werden quadratische Funktionen der Form $x\mapsto a\cdot x²+b\cdot x+c$ mit $a,b,c\in\mathbb{R}$ und $a\neq 0$. Die Wahl der Koeffizienten $a,b$ und $c$ beeinflusst verschiedene Eigenschaften wie Monotonie, Monotoniewechsel, Achsensymmetrie und Schnittpunkte mit den Achsen.

\subsection{Aufgabenstellung:}
\begin{enumerate}
	\item Der Graph einer quadratischen Funktion $f$ ist symmetrisch zur senkrechten Achse und schneidet die $x$-Achse an den Stellen $x_1$ und $x_2$ mit $x_1<x_2$. Es gilt: $\int^{x_2}_{x_1}{f(x)}$d$x=d$ mit $d\in\mathbb{R}^+$.
	
	Veranschauliche den Wert $d$ mithilfe eines passenden Graphen einer solchen Funktion $f$ im nachstehenden Koordinatensystems!
	
	\begin{center}
		\resizebox{0.4\linewidth}{!}{\psset{xunit=1.0cm,yunit=1.0cm,algebraic=true,dimen=middle,dotstyle=o,dotsize=4pt 0,linewidth=0.8pt,arrowsize=3pt 2,arrowinset=0.25}
\begin{pspicture*}(-4.526147391499742,-4.633854256009012)(4.748685856341189,4.827834606715162)
\multips(0,-4)(0,1.0){10}{\psline[linestyle=dashed,linecap=1,dash=1.5pt 1.5pt,linewidth=0.4pt,linecolor=lightgray]{c-c}(-4.526147391499742,0)(4.748685856341189,0)}
\multips(-4,0)(1.0,0){10}{\psline[linestyle=dashed,linecap=1,dash=1.5pt 1.5pt,linewidth=0.4pt,linecolor=lightgray]{c-c}(0,-4.633854256009012)(0,4.827834606715162)}
\psaxes[labelFontSize=\scriptstyle,xAxis=true,yAxis=true,labels=none,Dx=1.,Dy=1.,ticksize=-2pt 0,subticks=2]{->}(0,0)(-4.526147391499742,-4.633854256009012)(4.748685856341189,4.827834606715162)[x,140] [f(x),-40]
\end{pspicture*}}
	\end{center}
	
	Gib für jeden der Koeffizienten $a,b$ und $c$ dieser Funktion $f$ an, ob er positiv, negativ oder genau null sein muss!\leer
	
	\item Der Graph einer quadratischen Funktion $g$ hat einen Tiefpunkt und an den Stellen $x_1=0$ und $x_2>0$ Schnittpunkte mit der $x$-Achse. Die Nullstelle $x_2$ lässt sich mithilfe der Koeffizienten der Funktion $g$ berechnen. Stelle eine entsprechende Formel auf!\leer
	
	Der Graph der Funktion $g$ begrenzt mit der $x$-Achse eine endliche Fläche. Gib ein bestimmtes Integral an, mit dessen Hilfe der Inhalt dieser endlichen Fläche berechnet werden kann!\leer
	
	\item Für eine Stelle $k$ $(k\in\mathbb{R})$ des Graphen einer quadratischen Funktion $h$ gelten die Bedingungen $h(k)=0$ und $h'(k)=0$.
	
	\fbox{A} Skizziere einen möglichen Verlauf des Graphen von $h$ und kennzeichne die Stelle $k$ im nachstehenden Koordinatensystem!
	
	\begin{center}
		\resizebox{0.4\linewidth}{!}{\psset{xunit=1.0cm,yunit=1.0cm,algebraic=true,dimen=middle,dotstyle=o,dotsize=4pt 0,linewidth=0.8pt,arrowsize=3pt 2,arrowinset=0.25}
\begin{pspicture*}(-4.526147391499742,-4.633854256009012)(4.748685856341189,4.827834606715162)
\multips(0,-4)(0,1.0){10}{\psline[linestyle=dashed,linecap=1,dash=1.5pt 1.5pt,linewidth=0.4pt,linecolor=lightgray]{c-c}(-4.526147391499742,0)(4.748685856341189,0)}
\multips(-4,0)(1.0,0){10}{\psline[linestyle=dashed,linecap=1,dash=1.5pt 1.5pt,linewidth=0.4pt,linecolor=lightgray]{c-c}(0,-4.633854256009012)(0,4.827834606715162)}
\psaxes[labelFontSize=\scriptstyle,xAxis=true,yAxis=true,labels=none,Dx=1.,Dy=1.,ticksize=-2pt 0,subticks=2]{->}(0,0)(-4.526147391499742,-4.633854256009012)(4.748685856341189,4.827834606715162)[x,140] [f(x),-40]
\end{pspicture*}}
	\end{center}
	
	Zeige rechnerisch, dass eine Funktion $h$ mit der Funktionsgleichung \mbox{$h(x)=x²-2\cdot k\cdot x+k²$} die Bedingungen $h(k)=0$ und $h'(k)=0$ erfüllt!
	
	
\end{enumerate}

\antwort{
\begin{enumerate}
	\item \subsection{Lösungserwartung:} 

\begin{center}
	\resizebox{0.5\linewidth}{!}{\newrgbcolor{zzttqq}{0.6 0.2 0.}
\psset{xunit=1.0cm,yunit=1.0cm,algebraic=true,dimen=middle,dotstyle=o,dotsize=4pt 0,linewidth=0.8pt,arrowsize=3pt 2,arrowinset=0.25}
\begin{pspicture*}(-4.526147391499742,-4.633854256009012)(4.748685856341189,4.827834606715162)
\multips(0,-4)(0,1.0){10}{\psline[linestyle=dashed,linecap=1,dash=1.5pt 1.5pt,linewidth=0.4pt,linecolor=lightgray]{c-c}(-4.526147391499742,0)(4.748685856341189,0)}
\multips(-4,0)(1.0,0){10}{\psline[linestyle=dashed,linecap=1,dash=1.5pt 1.5pt,linewidth=0.4pt,linecolor=lightgray]{c-c}(0,-4.633854256009012)(0,4.827834606715162)}
\psaxes[labelFontSize=\scriptstyle,xAxis=true,yAxis=true,labels=none,Dx=1.,Dy=1.,ticksize=-2pt 0,subticks=2]{->}(0,0)(-4.526147391499742,-4.633854256009012)(4.748685856341189,4.827834606715162)[x,140] [f(x),-40]
\rput[tl](0.28113797322367096,1.481420412897097){d}
\pscustom[linewidth=0.8pt,linecolor=zzttqq,hatchcolor=zzttqq,fillstyle=hlines,hatchangle=45.0,hatchsep=0.16]{\psplot{-3.}{3.}{-0.33*x^2+3}\lineto(3.,0)\lineto(-3.,0)\closepath}
\psplot[linewidth=1.2pt,plotpoints=200]{-4.526147391499742}{4.748685856341189}{-0.33*x^2+3}
\rput[tl](-2.9,-0.1){$x_1$}
\rput[tl](2.6,-0.1){$x_2$}
\rput[tl](0.28113797322367096,1.481420412897097){d}
\rput[bl](-4.2,-2.){$f$}
\end{pspicture*}}
\end{center}

$a<0, b=0$ und $c>0$

	\subsection{Lösungsschlüssel:}
	\begin{itemize}
		\item Ein Punkt für eine korrekte Veranschaulichung des Wertes $d$, wobei der Graph von $f$ klar erkennbar die Form einer achsensymmetrischen und nach unten offenen Parabel haben muss. 
		\item Ein Punkt für die Angabe der richtigen Bedingungen für die Koeffizienten $a, b$ und $c$.
	\end{itemize}
	
	\item \subsection{Lösungserwartung:}
			
Mögliche Vorgehensweise:

$g(x)=a\cdot x²+b\cdot x$

$a\cdot x²+b\cdot x=0 \Rightarrow (x_1=0), x_2=-\frac{b}{a}$ mit $a>0$ und $b<0$\leer

Mögliche Berechnung des gesuchten Flächeninhalts:

$$\int^0_{-\frac{b}{a}}{g(x)}dx$$

oder:

$$-\int^{x_2}_0{g(x)}dx$$
	
	\subsection{Lösungsschlüssel:}
	
\begin{itemize}
	\item Ein Punkt für eine korrekte Formel, wobei die Bedingungen  $a>0$  und  $b<0$  nicht angeführt werden müssen. Äquivalente Formeln sind als richtig zu werten.  
	\item Ein Punkt für ein korrektes bestimmtes Integral. Äquivalente Ausdrücke sind als richtig zu werten. 
\end{itemize}

\item\subsection{Lösungserwartung:}
			
			Mögliche Graph von $h$:
			
\begin{center}
	\resizebox{0.5\linewidth}{!}{\newrgbcolor{zzttqq}{0.6 0.2 0.}
\psset{xunit=1.0cm,yunit=1.0cm,algebraic=true,dimen=middle,dotstyle=o,dotsize=4pt 0,linewidth=0.8pt,arrowsize=3pt 2,arrowinset=0.25}
\begin{pspicture*}(-4.526147391499742,-4.633854256009012)(4.748685856341189,4.827834606715162)
\multips(0,-4)(0,1.0){10}{\psline[linestyle=dashed,linecap=1,dash=1.5pt 1.5pt,linewidth=0.4pt,linecolor=lightgray]{c-c}(-4.526147391499742,0)(4.748685856341189,0)}
\multips(-4,0)(1.0,0){10}{\psline[linestyle=dashed,linecap=1,dash=1.5pt 1.5pt,linewidth=0.4pt,linecolor=lightgray]{c-c}(0,-4.633854256009012)(0,4.827834606715162)}
\psaxes[labelFontSize=\scriptstyle,xAxis=true,yAxis=true,labels=none,Dx=1.,Dy=1.,ticksize=-2pt 0,subticks=2]{->}(0,0)(-4.526147391499742,-4.633854256009012)(4.748685856341189,4.827834606715162)[x,140] [h(x),-40]
\psplot[linewidth=1.2pt,plotpoints=200]{-4.526147391499742}{4.748685856341189}{x^2-4*x+4}
\psdots[dotsize=5pt 0,dotstyle=*](2.,0.)
\rput[tl](2,-0.1){$k$}
\rput[bl](3.5,1.7){$h$}
\end{pspicture*}}
\end{center}

$h(k)=k²-2\cdot k²+k²=0 \Rightarrow h(k)=0$

$h'(x)=2\cdot x-2\cdot k$

$h'(k)=2\cdot k-2\cdot k=0 \Rightarrow h'(k)=0$
	
	\subsection{Lösungsschlüssel:}
	
\begin{itemize}
	\item Ein Ausgleichspunkt für eine korrekte Skizze eines entsprechenden Graphen von $h$. Der Graph von $h$ muss die Form einer nach oben oder unten offenen Parabel haben und an der gekennzeichneten Stelle von $k$ müssen die Nullstelle und (somit) die Extremstelle der Funktion $h$ klar erkennbar sein, die Symmetrie bezüglich der Geraden  $x=k$  muss erkennbar sein
	\item Ein Punkt für einen korrekten rechnerischen Nachweis beider Bedingungen.
\end{itemize}

\end{enumerate}}
		\end{langesbeispiel}