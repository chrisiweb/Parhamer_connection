\section{107 - MAT - AG 4.2, AN 4.3, FA 2.1, FA 4.3 - Körper mit rechteckigen Querschnittsflächen - Matura-HT-18/19-OT}

\begin{langesbeispiel} \item[6] %PUNKTE DES BEISPIELS
Die nachstehenden Abbildungen 1 und 2 stellen einen Körper mit ebenen Seitenflächen im Schrägriss bzw. eine Schnittfläche dar.
		\begin{center}
		\resizebox{0.8\linewidth}{!}{\includegraphics{../_database/Bilder/107_koerper.eps}}
		\end{center}

Die vordere Seitenfläche $ABFE$ steht normal auf die horizontale Grundfläche $ABCD$ und auf die horizontale Deckfläche $EFGH$, während die hintere Seitenfläche $DCGH$ zur Grundfläche unter dem Winkel $\alpha$ $(0^\circ<\alpha<90^\circ)$ geneigt ist.

Die beiden Seitenflächen $ADHE$ und $BCGF$ weisen gegenüber der Grundfläche den gleichen Neigungswinkel $\beta$ (mit $\beta\approx 76^\circ)$ auf.

Die horizontalen Querschnittsflächen des Körpers sind in jeder Höhe rechteckig. Die Längen $a(h)$ und die Breiten $b(h)$ dieser Rechtecke ändern sich lineare in Abhängigkeit von der Höhe $h$. Die Grundfläche hat eine Länge von 10\,cm und eine Breite von 6\,cm, die Deckfläche hat eine Länge von 20\,cm und eine Breite von 3\,cm. Die Höhe des Körpers beträgt 20\,cm.%Aufgabentext

\begin{aufgabenstellung}
\item Die Funktion $Q\!:[0;20]\rightarrow\mathbb{R}$ beschreibt die Größe der Querschnittsfläche $Q(h)$ in Abhängigkeit von der Höhe $h$ (mit $Q(h)$ in cm$^2$, $h$ in cm).
	
	Es gilt: $Q(h)=s\cdot h^2+1,5\cdot h+t$ mit $s,t\in\mathbb{R}$.%Aufgabentext

\Subitem{Bestimme die Werte von $s$ und $t$.} %Unterpunkt1
\Subitem{Berechne das Volumen des Körpers und gib das Ergebnis inklusive Einheit an.} %Unterpunkt2

\item Die lokale Änderungsrate der Breite $b(h)$ nimmt für jedes $h\in[0;20]$ einen konstanten Wert $c\in\mathbb{R}$ an.%Aufgabentext

\Subitem{Berechne $c$.} %Unterpunkt1
\Subitem{Beschreibe den Zusammenhang zwischen $c$ und $\alpha$ mithilfe einer Gleichung.} %Unterpunkt2

\item Die Funktion $a$ beschreibt die Länge $a(h)$ in der Höhe $h$ mit $a(h)$ und $h$ in cm.%Aufgabentext

\ASubitem{Gib eine Funktionsgleichung von $a$ an.} %Unterpunkt1

Ändert man den Winkel $\beta$ auf $45^\circ$ und lässt die Länge der Grundlinie $AB$ und die Körperhöhe unverändert, so erhält man eine neue vordere Seitenfläche $ABF_1E_2$, bei der die Funktion $a_1$ mit $a_1(h)=2\cdot h+10$ die Länge $a_1(h)$ in der Höhe $h$ beschreibt ($a_1(h)$ und $h$ in cm).

\Subitem{Gib das Verhältnis $\displaystyle\int^{20}_0 a_1(h)\dx[h]:\displaystyle\int^{20}_0 a(h)\dx[h]$ an und interpretiere das Ergebnis in Bezug auf die neue vordere Seitenfläche $ABF_1E_1$ und die ursprüngliche vordere Seitenfläche $ABFE$.} %Unterpunkt2

\end{aufgabenstellung}

\begin{loesung}
\item \subsection{Lösungserwartung:} 

\Subitem{mögliche Vorgehensweise:\\
	$Q(0)=60 \Rightarrow t=60$\\
	$Q(20)=60 \Rightarrow 60=400\cdot s+90 \Rightarrow s=-\frac{3}{40}$} %Lösung von Unterpunkt1
\Subitem{$V=\displaystyle\int^{20}_0 Q(h)\dx[h]=-\frac{1}{40}\cdot 20^3+0,75\cdot 20^2+60\cdot 20=1\,300$\,cm$^3$} %%Lösung von Unterpunkt2

\setcounter{subitemcounter}{0}
\subsection{Lösungsschlüssel:}
 
\Subitem{Ein Punkt für die Angabe der beiden richtigen Werte.

	Toleranzintervall für $s\!:[-0,08;-0,07]$\\
	Die Aufgabe ist auch dann als richtig gelöst zu werten, wenn bei korrektem Ansatz das Ergebnis aufgrund eines Rechenfehlers nicht richtig ist.} %Lösungschlüssel von Unterpunkt1
\Subitem{Ein Punkt für die richtige Lösung unter Angabe einer richtigen Einheit.

	Toleranzintervall: $[1\,280\,\text{cm}^3;1\,320\,\text{cm}^3]$\\
	Die Aufgabe ist auch dann als richtig gelöst zu werten, wenn bei korrektem Ansatz das Ergebnis aufgrund eines Rechenfehlers nicht richtig ist.} %Lösungschlüssel von Unterpunkt2

\item \subsection{Lösungserwartung:} 

\Subitem{$c=\frac{3-6}{20}=-\frac{3}{20}$} %Lösung von Unterpunkt1
\Subitem{$\tan(\alpha)=-\frac{1}{c}$} %%Lösung von Unterpunkt2

\setcounter{subitemcounter}{0}
\subsection{Lösungsschlüssel:}
 
\Subitem{Ein Punkt für die richtige Lösung.

	Toleranzintervall: $[-0,2;-0,1]$} %Lösungschlüssel von Unterpunkt1
\Subitem{Ein Punkt für die richtige Gleichung. Äquivalente Gleichungen sind als richtig zu werten.} %Lösungschlüssel von Unterpunkt2

\item \subsection{Lösungserwartung:} 

\Subitem{mögliche Vorgehensweise:\\
	$a(h)=\frac{a(20)-a(0)}{20}\cdot h+a(0)$\\
	$a(h)=\frac{1}{2}\cdot h+10$} %Lösung von Unterpunkt1
\Subitem{mögliche Vorgehensweise:\\
	$\displaystyle\int^{20}_0 a_1(h)\dx[h]=\displaystyle\int^{20}_0(2\cdot h+10)\dx[h]=600$
	
	$\displaystyle\int^{20}_0 a(h)\dx[h]=\displaystyle\int^{20}_0\left(\frac{1}{2}\cdot h+10\right)\dx[h]=300$
	
	$\displaystyle\int^{20}_0 a_1(h)\dx[h]:\displaystyle\int^{20}_0 a(h)\dx[h]=600:300=2:1$
	
	Das Verhältnis des Flächeninhalts der neuen Seitenwand $ABF_1E_1$ zum Flächeninhalt der ursprünglichen Seitenwand $ABFE$ ist $2:1$.	} %%Lösung von Unterpunkt2

\setcounter{subitemcounter}{0}
\subsection{Lösungsschlüssel:}
 
\Subitem{Ein Ausgleichspunkt für eine richtige Gleichung. Äquivalente Gleichungen sind als richtig zu werten.} %Lösungschlüssel von Unterpunkt1
\Subitem{Ein Punkt für die richtige Lösung und eine richtige Interpretation.

	Die Aufgabe ist auch dann als richtig gelöst zu werten, wenn bei korrektem Ansatz das Ergebnis aufgrund eines Rechenfehlers nicht richtig ist.} %Lösungschlüssel von Unterpunkt2

\end{loesung}

\end{langesbeispiel}