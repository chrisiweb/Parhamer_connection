\section{121 - MAT - AG 2.1, AN 3.3, WS 1.1, WS 2.1 - Tennis - Matura 2018/19 2. NT}

\begin{langesbeispiel} \item[6] %PUNKTE DES BEISPIELS
Tennis ist ein Rückschlagspiel zwischen zwei oder vier Personen, bei dem ein Tennisball über
ein Netz geschlagen werden muss. Das Spielfeld ist rechteckig und wird durch ein Netz in zwei
Hälften geteilt (siehe Abbildung 1). Für ein Spiel zwischen zwei Personen ist der Platz $23,77$\,m
lang und $8,23$\,m breit. Das Spielfeld wird durch die Grundlinien und die Seitenlinien begrenzt. Das
Netz weist eine maximale Höhe von $1,07$\,m auf.

Abbildung 1:

\begin{center}
\includegraphics[width=0.4\textwidth]{../_database/Bilder/121_tennis.eps}
\end{center}%Aufgabentext

\begin{aufgabenstellung}
\item Die Funktion $f$: $\mathbb{R}^+_0\rightarrow\mathbb{R}$ mit $f(x)=-0,0007\cdot x^3+0,005\cdot x^2+0,2\cdot x+0,4$ beschreibt eine Bahnkurve eines Tennisballs bis zu derjenigen Stelle, an der der Tennisball erstmals den Boden berührt. Dabei gibt $x$ die waagrechte Entfernung des Tennisballs vom Abschlagpunkt und $f(x)$ die Flughöhe des Tennisballs über dem Boden an ($x$ und $f(x)$ in m). Die Flugbahn des Tennisballs startet zwischen den Seitenlinien an der Grundlinie und die Ebene, in der die Flugbahn liegt, verläuft parallel zu Seitenlinie des Tennisfelds.%Aufgabentext

\ASubitem{Gib an, in welcher waagrechten Entfernung vom Abschlagpunkt der Tennisball seine maximale Höhe erreicht.\leer
	
	waagrechte Entfernung vom Abschlagpunkt: $\antwort[\rule{3cm}{0.3pt}]{12,4}$\,m} %Unterpunkt1
\Subitem{Überprüfe rechnerisch, ob der Tennisball im gegnerischen Spielfeld oder hinter der Grundlinie landet.} %Unterpunkt2

\item Fällt ein Tennisball lotrecht (ohne Drehung) auf den Boden, so springt er wieder lotrecht zurück. Der Restitutionskoeffizient $r$ ist ein Maß für die Sprungfähigkeit des Tennisballs.
	
	Es gilt: $r=\dfrac{v_2}{v_1}$, wobei $v_1$ der Betrag der Geschwindigkeit des Tennisballs vor und $v_2$ der Betrag der Geschwindigkeit des Tennisballs nach dem Aufprall ist.
	
	Die Differenz der vertikalen Geschwindigkeiten unmittelbar vor und nach dem Aufprall ist aufgrund der unterschiedlichen Bewegungsrichtungen des Tennisballs definiert durch:\\
	$\Delta v=v_2-(-v_1)$.%Aufgabentext

\Subitem{Gib $\Delta v$ in Abhängigkeit von $v_1$ und $r$ an.\leer
	
	$\Delta v=\,\antwort[\rule{3cm}{0.3pt}]{r\cdot v_1+v_1}$} %Unterpunkt1
	
Ein Tennisball trifft mit der Geschwindigkeit $v_1=4,4$\,m/s lotrecht auf dem Boden auf. Der Restitutionskoeffizient beträgt für diesen Tennisball $r=0,6$. Die Kontaktzeit mit dem Boden beträgt 0,01\,s.	
	
\Subitem{Berechne die durchschnittliche Beschleunigung $a$ (in m/s$^2$) des Tennisballs in vertikaler Richtung beim Aufprall (während der Kontaktzeit).\leer
	
	$a=\,\antwort[\rule{3cm}{0.3pt}]{704}$\,m/s$^2$} %Unterpunkt2

\item Bei einem Fünf-Satz-Tennismatch gewinnt ein Spieler, sobald er drei Sätze gewonnen hat. Für einen Satzgewinn müssen in der Regel sechs Games gewonnen werden, wobei es für jedes gewonnene Game einen Punkt gibt.
	
	Für unterschiedliche Wahrscheinlichkeiten $p$ für ein gewonnenes Game wurden die daraus resultierenden Wahrscheinlichkeiten $m$ für einen Matchgewinn bei einem Fünf-Satz-Match ermittelt. In der nachstehenden Tabelle sind diese Wahrscheinlichkeiten angeführt.
	
	\begin{center}
	\begin{tabular}{|c|c|}\hline
	\cellcolor[gray]{0.9}$p$&\cellcolor[gray]{0.9}$m$\\ \hline
	0,5&0,5\\ \hline
	0,51&0,6302\\ \hline
	0,55&0,9512\\ \hline
	0,6&0,9995\\ \hline
	0,7&1,000\\ \hline
	\end{tabular}
	\end{center}
	
	Die Wahrscheinlichkeit, dass Spieler $A$ ein Game gewinnt, ist um 2 Prozentpunkte höher als die Wahrscheinlichkeit, dass sein Gegenspieler $B$ ein Game gewinnt.%Aufgabentext

\Subitem{Gib an, um wie viel Prozentpunkte die Wahrscheinlichkeit, dass Spieler $A$ ein Fünf-Satz-Match gewinnt, höher ist als jene für seine Gegenspieler $B$.} %Unterpunkt1

Gegenüber einem anderen, schwächeren Gegenspieler $C$ hat Spieler $A$ einen Vorteil von 10 Prozentpunkten, ein Game zu gewinnen.

\Subitem{Zeige, dass die Wahrscheinlichkeit, dass Spieler $A$ ein Fünf-Satz-Match gegen Gegenspieler $C$ gewinnt, um 50,94 Prozent höher ist als bei einem Fünf-Satz-Match gegen $B$.} %Unterpunkt2

\end{aufgabenstellung}

\begin{loesung}
\item \subsection{Lösungserwartung:} 

\Subitem{mögliche Vorgehensweise:
	
	$f'(x)=0$\\
	$-0,0021\cdot x^2+0,01\cdot x+0,2=0 \Rightarrow x_1=12,42\ldots$\quad$(x_2=-7,66\ldots)$\\
	waagrechte Entfernung vom Abschlagpunkt: ca. 12,4\,m} %Lösung von Unterpunkt1
\Subitem{$f(x)=0 \Rightarrow x_1=21,597\ldots$\quad$(x_2=-2,15\ldots, x_3=-12,30\ldots)$\\
	Die einzige positive Nullstelle von $f$ ist $x_1\approx 21,6$.
	
	Da das Spielfeld 23,77\,m lang ist, landet der Tennisball im gegnerischen Spielfeld.} %%Lösung von Unterpunkt2

\setcounter{subitemcounter}{0}
\subsection{Lösungsschlüssel:}
 
\Subitem{Ein Ausgleichspunkt für die richtige Lösung.

		Toleranzintervall: $[12,4\,\text{m}; 12,5\,\text{m}]$\\
		Die Aufgabe  ist auch dann als richtig gelöst zu werten, wenn bei korrektem Ansatz das
Ergebnis aufgrund eines Rechenfehlers nicht richtig ist.} %Lösungschlüssel von Unterpunkt1
\Subitem{Ein Punkt für einen richtigen rechnerischen Nachweis.} %Lösungschlüssel von Unterpunkt2

\item \subsection{Lösungserwartung:} 

\Subitem{$\Delta v=r\cdot v_1+v_1$} %Lösung von Unterpunkt1
\Subitem{mögliche Vorgehensweise:\\
	$\Delta v=v_1\cdot(1+r)=4,4\cdot(1+0,6)=7,04$\\
	$a=7,04:0,01=704$\\
	$a=704$\,m/s$^2$} %%Lösung von Unterpunkt2

\setcounter{subitemcounter}{0}
\subsection{Lösungsschlüssel:}
 
\Subitem{Ein Punkt für die richtige Lösung. Andere Schreibweisen der Lösung sind ebenfalls als
richtig zu werten.} %Lösungschlüssel von Unterpunkt1
\Subitem{Ein Punkt für die richtige Lösung.

		Toleranzintervall für $a$: $[700\,\text{m/s}^2;710\,\text{m/s}^2]$\\
		Die Aufgabe ist auch dann als richtig gelöst zu werten, wenn bei korrektem Ansatz das Ergebnis aufgrund eines Rechenfehlers nicht richtig ist.} %Lösungschlüssel von Unterpunkt2

\item \subsection{Lösungserwartung:} 

\Subitem{$0,6302-0,3698=0,2604$
	
	Diese Wahrscheinlichkeit ist um ca. 26 Prozentpunkte höher.} %Lösung von Unterpunkt1
\Subitem{Wahrscheinlichkeit, dass Spieler $A$ ein Fünf-Satz-Match gegen Spieler $C$ gewinnt: $0,9512$\\
	Wahrscheinlichkeit, dass Spieler $A$ ein Fünf-Satz-Match gegen Spieler $B$ gewinnt: $0,6302$
	
	$\frac{0,9512}{0,6302}=1,50936\ldots\approx 1,5094$
	
	$\Rightarrow 0,9512$ ist um ca. 50,94 Prozent höher als 0,6302. } %%Lösung von Unterpunkt2

\setcounter{subitemcounter}{0}
\subsection{Lösungsschlüssel:}
 
\Subitem{Ein Punkt für die richtige Lösung.

		Toleranzintervall: $[26; 26,1]$} %Lösungschlüssel von Unterpunkt1
\Subitem{Ein Punkt für einen richtigen rechnerischen Nachweis.

		Die Aufgabe ist auch dann als richtig gelöst zu werten, wenn bei korrektem Ansatz des Ergebnis aufgrund eines Rechenfehlers nicht richtig ist.} %Lösungschlüssel von Unterpunkt2

\end{loesung}

\end{langesbeispiel}