\section{82 - MAT - FA 5.1, FA 5.3, FA 5.6, FA 2.2, AN 1.4 - Brasilien - Matura NT 1 16/17}


\begin{langesbeispiel} \item[6] %PUNKTE DES BEISPIELS

Brasilien ist der größte und bevölkerungsreichste Staat Südamerikas.\leer

Im Jahr 2014 hatte Brasilien eine Einwohnerzahl von 202,74 Millionen.\leer

Aufgrund von Volkszählungen sind folgende Einwohnerzahlen bekannt:

\begin{center}
	\begin{tabular}{|c|c|}\hline
	\cellcolor[gray]{0.9}Jahr&\cellcolor[gray]{0.9}Einwohnerzahl\\ \hline
	1970&94\,508\,583\\ \hline
	1980&121\,150\,573\\ \hline
	1991&146\,917\,459\\ \hline
	2000&169\,590\,693\\ \hline
	2010&190\,755\,799\\ \hline
	\end{tabular}
\end{center}

\subsection{Aufgabenstellung:}
\begin{enumerate}
	\item \fbox{A} Gib die Bedeutung der nachstehend angeführten Werte im Kontext der Entwicklung der Einwohnerzahl an!\leer
	
	$\sqrt[10]{\dfrac{121\,150\,573}{94\,508\,583}}\approx 1,02515$\leer
	
	$\sqrt[9]{\dfrac{169\,590\,693}{146\,917\,459}}\approx 1,01607$\leer
	
	Begründe anhand der beiden angeführten Werte, warum man die Entwicklung der Einwohnerzahl im gesamten Zeitraum von 1970 bis 2010 nicht angemessen durch eine Exponentialfunktion beschreiben kann!\leer
	
	\item Gib unter Annahme eines linearen Wachstums anhand der Einwohnerzahlen von 1991 und 2010 eine Gleichung derjenigen Funktion $f$ an, die die Einwohnerzahl beschreibt! Die Zeit $t$ wird dabei in Jahren gemessen, der Zeitpunkt $t=0$ entspricht dem Jahr 1991.
	
	Berechne, um wie viel Prozent die Vorhersage des linearen Modells für das Jahr 2014 von dem in der Einleitung angegebenen tatsächlichen Wert abweicht!\leer
	
	\item Für Brasilien wird für die Jahre 2010 bis 2015 jeweils eine konstante Geburtenrate $b=14,6$ sowie eine konstante Sterberate $d=6,6$ angenommen. Das bedeutet, dass es jährlich 14,6 Geburten pro 1\,000 Einwohner/innen und 6,6 Todesfälle pro 1\,000 Einwohner/innen gibt.
	
	Die Entwicklung der Einwohnerzahl kann in diesem Zeitraum mithilfe der Differenzengleichung $x_{n+1}=x_n+x_n\cdot\frac{1}{1\,000}\cdot(b-d)+m_n$ beschrieben werden, wobei $x_n$ die Anzahl der Einwohner/innen im Jahr $n$ beschreibt und $m_n$ die Differenz aus der Anzahl der zugewanderten und jener der abgewanderten Personen angibt. Diese Differenz wird als Wnaderungsbilanz bezeichnet.
	
	Gib die Bedeutung des Ausdrucks $x_n\cdot\frac{1}{1\,000}\cdot(b-d)$ im Kontext der Entwicklung der Einwohnerzahl an!
	
	Berechne die maximale Größe der Wanderungsbilanz für den Fall, dass die Einwohnerzahl im Jahr 2015 gegenüber der Einwohnerzahl des Vorjahres maximal um 1\,\% größer ist!
	
	\end{enumerate}

\antwort{
\begin{enumerate}
	\item \subsection{Lösungserwartung:} 

Im Zeitintervall $[1970;1980]$ steigt die Einwohnerzahl pro Jahr um ca. 2,515\,\%, im Zeitintervall $[1991;2000]$ steigt die Einwohnerzahl pro Jahr um ca. 1,607\,\%.

Damit eine Beschreibung durch eine Exponentialfunktion angemessen ist, müsste die relative jährliche Zunahme der Einwohnerzahl in den beiden betrachteten Zeitintervallen annähernd gleich sein. Im Zeitintervall $[1970;1980]$ ist die relative jährliche Zunahme der Einwohnerzahl mit ca. 2,5\,\% deutlich größer als im Zeitintervall $[1991;2000]$, wo es nur mehr ca. 1,6\,\% beträgt. Daher wäre eine Beschreibung der Entwicklung der Einwohnerzahl durch eine Exponentialfunktion nicht angemessen.

\subsection{Lösungsschlüssel:}
\begin{itemize}
	\item Ein Ausgleichspunkt für eine (sinngemäß) korrekte Deutung der beiden Werte.
	\item Ein Punkt für eine (sinngemäß) korrekte Begründung.
\end{itemize}

	\item \subsection{Lösungserwartung:}

Mögliche Vorgehensweise:

$f(t)=146\,917\,459+k\cdot t$

$k=\frac{190\,755\,799-146\,917\,459}{19}\approx 2\,307\,281$

$f(t)=146\,917\,459+2\,307\,281\cdot t$\leer

Mögliche Vorgehensweise:

$f(23)=199\,984\,922$

$\frac{199\,984\,922}{202\,740\,000}\approx 0,986$

Die Abweichung zur Vorhersage beträgt ca. 1,4\,\%

\subsection{Lösungsschlüssel:}
\begin{itemize}
	\item Ein Punkt für eine korrekte Funktionsgleichung. Äquivalente Funktionsgleichungen sind als richtig zu werten.
	
	Toleranzintervall für $k:\,[2\,305\,000;2\,310\,000]$
	
	\item Ein Punkt für die richtige Lösung, wobei die Abweichung auch als negativer Wert angegeben sein kann.
	
	Toleranzintervall: $[1\,\%;2\,\%]$ bzw. $[0,01;0,02]$
\end{itemize}

\item \subsection{Lösungserwartung:}

Mögliche Duetung:

Der angeführte Ausdruck gibt die Anzahl derjenigen Personen an, die die Einwohnerzahl $x_n$ im Zeitintervall $[n;n+1]$ aufgrund von Geburten und/oder Todesfällen erhöhen (bzw. verringern).\leer

Mögliche Vorgehensweise:

$x_{2015}\leq 1,01\cdot x_{2014}$

$x_{2014}+x_{2014}\cdot\frac{14,6-6,6}{1\,000}+m_{2014}\leq 1,02\cdot x_{2014}$

daher

$m_{2014}\leq \left(1,01-1-\frac{14,6-6,6}{1\,000}\right)\cdot x_{2014}$

$m_{2014}\leq 0,002\cdot 202\,740\,000=405\,480$\leer

Damit die Einwohnerzahl im Jahr 2015 gegenüber der Einwohnerzahl im Jahr davor maximal um 1\,\% größer wird, dürfen höchstens 405\,480 Personen mehr zuwandern als abwandern.

\subsection{Lösungsschlüssel:}
\begin{itemize}
	\item Ein Punkt für eine korrekte Deutung.
	\item Ein Punkt für die richtige Lösung.
	
	Toleranzintervall: $[405\,000\text{ Personen};406\,000\text{ Personen}]$
	
	Die Aufgabe ist auch dann als richtig gelöst zu werten, wenn bei korrektem Ansatz das Ergebnis aufgrund eines Rechenfehlers nicht richtig ist.
\end{itemize}
\end{enumerate}}
		\end{langesbeispiel}