\section{32 - MAT - AG 2.1, FA 1.2, AN 1.3, AN 2.1  - Zustandsgleichung idealer Gase - Matura 2013/14 Haupttermin}

\begin{langesbeispiel} \item[0] %PUNKTE DES BEISPIELS
				Die Formel $p\cdot V=n\cdot R\cdot T$ beschreibt modellhaft den Zusammenhang zwischen dem Druck $p$, dem Volumen $V$, der Stoffmenge $n$ und der absoluten Temperatur $T$ eines idealen Gases und wird thermische Zustandsgleichung idealer Gase genannt. $R$ ist eine Konstante.\\ 
				Das Gas befindet sich in einem geschlossenen Gefäß, in dem die Zustandsgrößen $p$, $V$ und $T$ verändert werden können. Die Stoffmenge $n$ bleibt konstant.

\subsection{Aufgabenstellung:}
\begin{enumerate}
	\item  Führe alle Möglichkeiten an, die zu einer Verdopplung des Drucks führen, wenn jeweils eine der Zustandsgrößen verändert wird und die anderen Größen konstant bleiben!
	
 Genau zwei der folgenden Graphen stellen die Abhängigkeit zweier Zustandsgrößen gemäß dem oben genannten Zusammenhang richtig dar. Kreuze diese beiden Graphen an!  Beachte: Die im Diagramm nicht angeführten Größen sind jeweils konstant.

\langmultiplechoice[5]{  %Anzahl der Antwortmoeglichkeiten, Standard: 5
				L1={\resizebox{0.5\linewidth}{!}{\psset{xunit=1.0cm,yunit=1.0cm,algebraic=true,dimen=middle,dotstyle=o,dotsize=5pt 0,linewidth=0.8pt,arrowsize=3pt 2,arrowinset=0.25}
\begin{pspicture*}(-0.40545454545454307,-0.4327272727272741)(9.854545454545452,9.387272727272736)
\psaxes[labelFontSize=\scriptstyle,xAxis=true,yAxis=true,labels=none,Dx=1.,Dy=1.,ticksize=0pt 0,subticks=0]{->}(0,0)(0.,0.)(9.854545454545452,9.387272727272736)[V,140] [p(V),-40]
\psplot[linewidth=1.2pt,plotpoints=200]{0.5}{9.854545454545452}{10.0/x}
\end{pspicture*}}},   %1. Antwortmoeglichkeit 
				L2={\resizebox{0.5\linewidth}{!}{\psset{xunit=1.0cm,yunit=1.0cm,algebraic=true,dimen=middle,dotstyle=o,dotsize=5pt 0,linewidth=0.8pt,arrowsize=3pt 2,arrowinset=0.25}
\begin{pspicture*}(-0.40545454545454307,-0.4327272727272741)(9.854545454545452,9.387272727272736)
\psaxes[labelFontSize=\scriptstyle,xAxis=true,yAxis=true,labels=none,Dx=1.,Dy=1.,ticksize=0pt 0,subticks=0]{->}(0,0)(0.,0.)(9.854545454545452,9.387272727272736)[T,140] [V(T),-40]
\psplot{0}{9.854545454545452}{(--12.--2.*x)/4.}
\end{pspicture*}}},   %2. Antwortmoeglichkeit
				L3={\resizebox{0.5\linewidth}{!}{\psset{xunit=1.0cm,yunit=1.0cm,algebraic=true,dimen=middle,dotstyle=o,dotsize=5pt 0,linewidth=0.8pt,arrowsize=3pt 2,arrowinset=0.25}
\begin{pspicture*}(-0.40545454545454307,-0.4327272727272741)(9.854545454545452,9.387272727272736)
\psaxes[labelFontSize=\scriptstyle,xAxis=true,yAxis=true,labels=none,Dx=1.,Dy=1.,ticksize=0pt 0,subticks=0]{->}(0,0)(0.,0.)(9.854545454545452,9.387272727272736)[V,140] [p(V),-40]
\psplot{0}{8}{(--64.-8.*x)/8.}
\end{pspicture*}}},   %3. Antwortmoeglichkeit
				L4={\resizebox{0.5\linewidth}{!}{\psset{xunit=1.0cm,yunit=1.0cm,algebraic=true,dimen=middle,dotstyle=o,dotsize=5pt 0,linewidth=0.8pt,arrowsize=3pt 2,arrowinset=0.25}
\begin{pspicture*}(-0.40545454545454307,-0.4327272727272741)(9.854545454545452,9.387272727272736)
\psaxes[labelFontSize=\scriptstyle,xAxis=true,yAxis=true,labels=none,Dx=1.,Dy=1.,ticksize=0pt 0,subticks=0]{->}(0,0)(0.,0.)(9.854545454545452,9.387272727272736)[p,140] [V(p),-40]
\psplot[linewidth=1.2pt,plotpoints=200]{0}{9.854545454545452}{1.0/16.0*(x-11.0)^(2.0)+1.0}
\end{pspicture*}}},   %4. Antwortmoeglichkeit
				L5={\resizebox{0.5\linewidth}{!}{\psset{xunit=1.0cm,yunit=1.0cm,algebraic=true,dimen=middle,dotstyle=o,dotsize=5pt 0,linewidth=0.8pt,arrowsize=3pt 2,arrowinset=0.25}
\begin{pspicture*}(-0.40545454545454307,-0.4327272727272741)(9.854545454545452,9.387272727272736)
\psaxes[labelFontSize=\scriptstyle,xAxis=true,yAxis=true,labels=none,Dx=1.,Dy=1.,ticksize=0pt 0,subticks=0]{->}(0,0)(0.,0.)(9.854545454545452,9.387272727272736)[T,140] [p(T),-40]
\psplot{0}{9.854545454545452}{(-0.--9.*x)/9.}
\end{pspicture*}}},	 %5. Antwortmoeglichkeit
				L6={},	 %6. Antwortmoeglichkeit
				L7={},	 %7. Antwortmoeglichkeit
				L8={},	 %8. Antwortmoeglichkeit
				L9={},	 %9. Antwortmoeglichkeit
				%% LOESUNG: %%
				A1=1,  % 1. Antwort
				A2=5,	 % 2. Antwort
				A3=0,  % 3. Antwort
				A4=0,  % 4. Antwort
				A5=0,  % 5. Antwort
				}

\item Bei gleichbleibender Stoffmenge und gleichbleibender Temperatur kann das Volumen des Gases durch Änderung des Drucks variiert werden. 
 
 Begründe, warum die mittlere Änderung des Drucks in Abhängigkeit vom Volumen 
\begin{center}$\dfrac{p(V_2)-p(V_1)}{V_2-V_1}$\end{center}

für jedes Intervall [$V1; V2]$ mit $V1 \neq V2$ ein negatives Ergebnis liefert!

 Ermittle jene Funktionsgleichung, die die momentane Änderung des Druckes in Abhängigkeit vom Volumen des Gases beschreibt!
 
						\end{enumerate}\leer
				
\antwort{
\begin{enumerate}
	\item \subsection{Lösungserwartung:} 
	
	Volumen halbieren oder Temperatur verdoppeln. (Lösung Multiple Choice - siehe oben)
	
	\subsection{Lösungsschlüssel:}
	\begin{itemize}
		\item Ein Punkt für die korrekte Angabe beider Möglichkeiten, die zu einer Verdoppelung des Drucks führen, wobei beide Möglichkeiten angegeben werden müssen. 
		\item Ein Punkt für das ausschließliche Ankreuzen der beiden richtigen Graphen.
	\end{itemize}
	
	\item \subsection{Lösungserwartung:}
		\textit{Mögliche Begründungen:}
	
	Der Druck nimmt mit steigendem Volumen ab. Die Funktion ist streng monoton fallend.\leer
	
$\left.\begin{array}{r}p(V_2)<p(V_1)\\V_2>V_1\end{array}\right\}$ Daher ist der Quotient negativ.\leer

$p'(V)=-\frac{n\cdot R\cdot T}{V^2}$ beschreibt die momentane Druckänderung.

	\subsection{Lösungsschlüssel:}
	
\begin{itemize}
	\item Ein Punkt für eine (sinngemäß) korrekte Begründung.
	\item  Ein Punkt für die Ermittlung der Funktionsgleichung für die Druckänderung. Die Schreibweise $p'(V)$ muss nicht verwendet werden. Wichtig ist, dass der Funktionsterm stimmt und eine funktionale Schreibweise verwendet wird.
\end{itemize}
\end{enumerate}}
		\end{langesbeispiel}