\section{87 - MAT - AN 3.2, AN 3.2, AN 4.3, AN 4.2, WS 3.2 - Dichtefunktion und Verteilungsfunktion - Matura 2016/17 2. NT}

\begin{langesbeispiel} \item[3] %PUNKTE DES BEISPIELS
						Es sei $X$ eine Zufallsvariable, für die sich die Wahrscheinlichkeit, dass $X$ in einem Intervall $I$ liegt, mithilfe einer sogenannten Dichtefunktion $f$ folgendermaßen ermitteln lässt:
						
						$P(a\leq X\leq b)=\displaystyle\int^b_a f(x)$d$x$ für alle $a,b\in I$ mit $a\leq b$
						
						In diesem Fall gilt für die Verteilungsfunktion $F:F(x)=P(X\leq x)$ für alle $x\in\mathbb{R}$, d.h. insbesondere $F(b)-F(a)=P(a\leq X\leq b)$ für $a,b\in I$ und $a\leq b$.
						
						Die nachstehende Grafik zeigt den Graphen einer Dichtefunktion $f$ mit $f(x)=k\cdot\sin(x)$ für $x\in[0;c]$, wobei $k\in\mathbb{R}, k>0$ und $f(c)=0$ gilt. Für $x\neq[0;c]$ gilt: $f(x)=0$.
						
						\begin{center}
							\resizebox{0.8\linewidth}{!}{\psset{xunit=4.0cm,yunit=4.0cm,algebraic=true,dimen=middle,dotstyle=o,dotsize=5pt 0,linewidth=1.6pt,arrowsize=3pt 2,arrowinset=0.25}
\begin{pspicture*}(-0.1490384615384615,-0.1384157660521301)(3.2980769230769234,1.2816274634456448)
\psaxes[labelFontSize=\scriptstyle,xAxis=true,yAxis=true,labels=y,Dx=0.2,Dy=1.,yticksize=-2pt 0,xticksize=0]{->}(0,0)(0.,0.)(3.2980769230769234,1.2816274634456448)[x,140] [f(x),-40]
\psplot[linewidth=2.pt,plotpoints=200]{0}{3.14}{0.5*SIN(x)}
\antwort{\psplot[linewidth=2.pt,plotpoints=200]{0}{3.14}{-1.0/2.0*COS(x)+0.5}}
\rput[tl](3.1153846153846154,-0.03075905912269592){c}
\rput[tl](2.331730769230769,0.5075244755244751){$f$}
\antwort{\rput[tl](2.394230769230769,1.0406815003178636){$F$}}
\end{pspicture*}}
						\end{center}

				\subsection{Aufgabenstellung:}
\begin{enumerate}
	\item Gib für die gegebenen Dichtefunktion $f$ den Funktionswert $F(0)$ der zugehörigen Verteilungsfunktion $F$ an und begründe, warum $F(c)=1$ ist!\leer
	
	$F(0)=$\,\antwort[\rule{3cm}{0.3pt}]{0}
	
	Skizziere in der oben stehenden Grafik den Graphen der zugehörigen Verteilungsfunktion $F$ und beschreibe das Krümmungsverhalten von $F$ im Intervall $[0;c]$!
	
	\item Gib an, durch welche Eigenschaft von $f$ der Wert des Parameters $k$ festgelegt ist, und berechne den Wert von $k$!
	
	Gib einen Term der zugehörigen Verteilungsfunktion $F$ im Intervall $[0;c]$ an!\leer
	
	$F(x)=$\,\antwort[\rule{3cm}{0.3pt}]{$-0,5\cdot\cos(x)+0,5$}
	
	\item Für ein Ereignis $E$ gilt: $P(E)=1-P(X\leq c-a)$ für ein beliebiges $a\in[0;c]$.
	
	\fbox{A} Beschreibe dieses Ereignis $E$ verbal!
	
	Stelle für $a\leq\frac{c}{2}$ die Wahrscheinlichkeit $P(a\leq X\leq c-a)$ in nachstehender Grafik als Fläche dar und begründe den Zusammenhang \\ \mbox{$P(a\leq X\leq c-a)=1-2\cdot P(X\leq a)$} anhand dieser Darstellung!
	
	\begin{center}
		\resizebox{0.8\linewidth}{!}{\psset{xunit=4.0cm,yunit=4.0cm,algebraic=true,dimen=middle,dotstyle=o,dotsize=5pt 0,linewidth=1.6pt,arrowsize=3pt 2,arrowinset=0.25}
\begin{pspicture*}(-0.1490384615384615,-0.1384157660521301)(3.2980769230769234,1.2816274634456448)
\psaxes[labelFontSize=\scriptstyle,xAxis=true,yAxis=true,labels=y,Dx=0.2,Dy=1.,yticksize=-2pt 0,xticksize=0]{->}(0,0)(0.,0.)(3.2980769230769234,1.2816274634456448)[x,140] [f(x),-40]
\psplot[linewidth=2.pt,plotpoints=200]{0}{3.14}{0.5*SIN(x)}
\rput[tl](3.1153846153846154,-0.0307){c}
\rput[tl](2.331730769230769,0.5075244755244751){$f$}
\antwort{\pscustom[linewidth=0.8pt,fillcolor=black,fillstyle=solid,opacity=0.1]{\psplot{1.0471975511965976}{2.0943951023931953}{0.5*SIN(x)}\lineto(2.0943951023931953,0)\lineto(1.0471975511965976,0)\closepath}
\rput[tl](1.0144230769230769,-0.0307){$a$}
\rput[tl](2.0,-0.0307){$c-a$}}
\end{pspicture*}}
	\end{center}
	
						\end{enumerate}\leer
				
\antwort{
\begin{enumerate}
	\item \subsection{Lösungserwartung:}
	$F(c)=1$ bzw. $P(X\leq c)=1$, d.h., die Wahrscheinlichkeit, dass $X$ einen Wert kleiner gleich $c$ annimmt, beträgt 100\,\%, da $f(x)=0$ für $x>c$.
	
	Skizze: siehe oben.
	
	Bis zum lokalen Maximum von $f$ ist die Funktion $F$ linksgekrümmt, danach ist die Funktion $F$ rechtsgekrümmt.
	
	\subsection{Lösungsschlüssel:}
	
	- Ein Punkt für die richtige Lösung und eine (sinngemäß) korrekte Begründung.
	
	- Ein Punkt für eine korrekte Skizze und eine (sinngemäß) korrekte Beschreibung des Krümmungsverhaltens der Funktion $F$. Die Skizze ist als korrekt zu betrachten, wenn das korrekte Krümmungsverhalten des Graphen von $F$ in der Skizze klar erkennbar ist und die Wendestelle von $F$ dabei bei $x=\frac{c}{2}$ liegt. Für $x>c$ muss der Graph von $F$, sofern er in diesem Bereich skizziert ist, waagrecht verlaufen.
	
	\item \subsection{Lösungserwartung:}
	
	Der Wert von $k$ ist durch die Eigenschaft $F(c)=1$ festgelegt, d.h., der Inhalt der vom Graphen von $f$ und der $x$-Achse im Intervall $[0;c]$ eingeschlossenen Fläche muss 1 sein. Da die rechte Nullstelle bei $x=\pi$ liegt und somit $c=\pi$ ist, muss gelten:
	
	$\displaystyle\int^\pi_0\,k\cdot\sin(x)$d$x=1 \Rightarrow k=0,5$
	
	Mögliche Vorgehensweise:
	
	$F(x)=-0,5\cdot\cos(x)+C$
	
	$F(0)=0 \Rightarrow C=0,5$
	
	$F(x)=-0,5\cdot\cos(x)+0,5$
	
	\subsection{Lösungsschlüssel:}
	
	- Ein Punkt für eine (sinngemäß) korrekte Angabe, welche Eigenschaft von $f$ den Wert von $k$ festlegt, und für die richtige Lösung.
	
	- Ein Punkt für einen korrekten Term. Äquivalente Terme sind als richtig zu werten.
	
	\item \subsection{Lösungserwartung:}
	
	Das Ereignis $E$ beschreibt, dass die Zufallsvariable $X$ einen Wert annimmt, der größer (oder gleich) $c-a$ ist.
	
	Grafik siehe oben
	
	Mögliche Begründung:
	
	Wegen der Symmetrie der Dichtefunktion gilt: $P(X\leq a)=P(x\geq c-a)$.
	
	Aus $F(c)=1$ folgt: $P(a\leq X\leq c-a)=1-P(X\leq a)-P(X\geq c-a)=1-2\cdot P(X\leq a)$.
	
	\subsection{Lösungsschlüssel:}
	
	- Ein Ausgleichspunkt für eine (sinngemäß) korrekte Beschreibung.
	
	- Ein Punkt für eine korrekte Darstellung der Wahrscheinlichkeit als Fläche, wobei die beiden Grenzen symmetrisch zur Stelle des Maximums der Funktion $f$ liegen müssen, und eine korrekte Begründung. Andere korrekte Begründungen sind ebenfalls als richtig zu werten.
		\end{enumerate}}		
	
		\end{langesbeispiel}