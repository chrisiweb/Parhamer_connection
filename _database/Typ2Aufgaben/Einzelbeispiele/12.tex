\section{12 - MAT - AN 1.3, AN 3.3, AG 2.3 - Kostenfunktion - BIFIE Aufgabensammlung}

\begin{langesbeispiel} \item[0] %PUNKTE DES BEISPIELS
				Im Zuge einer betriebswirtschaftlichen Analyse und Beratung werden bei zwei Firmen die Kostenverl�ufe in Abh�ngigkeit von der Produktionsmenge untersucht.
				
				Bei Firma A wird der Zusammenhang zwischen der monatlichen Produktionsmenge $x$ (in Mengeneinheiten [ME]) und den entstehenden Produktionskosten $K_A(x)$ (in Geldeinheiten [GE]) durch die Kostenfunktion $K_A$ mit $$K_A(x)=0,01x�-3x�+350x+20000$$ beschrieben. Firma A kann monatlich maximal 400 ME produzieren. In der untenstehenden Abbildung ist der Graph der Funktion $K_A$ im Intervall [0;400] dargestellt.
				\leer
				
				\psset{xunit=0.03cm,yunit=0.20cm,algebraic=true,dimen=middle,dotstyle=o,dotsize=5pt 0,linewidth=0.8pt,arrowsize=3pt 2,arrowinset=0.25}
\begin{pspicture*}(-52.99077542799506,-5.578723404255297)(441.7596374622275,34.54127659574461)
\psaxes[labelFontSize=\scriptstyle,xAxis=true,yAxis=true,labels=x,Dx=50.,Dy=4.,ticksize=-2pt 0,subticks=2]{->}(0,0)(0.,0.)(441.7596374622275,34.54127659574461)
\psplot[linewidth=1.2pt,plotpoints=200]{0}{441.7596374622275}{(0.01*x^(3.0)-3.0*x^(2.0)+350.0*x+20000.0)/10000.0}
\begin{scriptsize}
\rput[tl](226.22984894259397,9.221276595744664){$K_A$}
\rput[tl](5.514441087612724,34.014468085106316){$K_A(x)$ (in GE)}
\rput[tl](386.09707955689225,1.996595744680852){$x$ (in ME)}
\rput[tl](-31.647814702919874,4.40425531914893){40.000}
\rput[tl](-31.647814702919874,8.40425531914893){80.000}
\rput[tl](-35.647814702919874,12.40425531914893){120.000}
\rput[tl](-35.647814702919874,16.40425531914893){160.000}
\rput[tl](-35.647814702919874,20.40425531914893){200.000}
\rput[tl](-35.647814702919874,24.40425531914893){240.000}
\rput[tl](-35.647814702919874,28.40425531914893){280.000}
\rput[tl](-35.647814702919874,32.40425531914893){320.000}
\end{scriptsize}
\end{pspicture*}\leer

Bei Firma B wird der Zusammenhang zwischen der monatlichen Produktionsmenge $x$ (in ME) und den entstehenden Produktionskosten $K_B(x)$ (in GE) durch die Kostenfunktion $K_B$ mit $K_B(x)=0,5x�+100x+15000$ beschrieben. Firma B kann monatlich maximal 300 ME produzieren.
				
\subsection{Aufgabenstellung:}
\begin{enumerate}
	\item Untersuche, ob der Kostenverlauf bei Firma B progressiv oder degressiv ist! Begr�nde deine Antwort!
	
	Allgemein kann eine solche Kostenfunktion in Abh�ngigkeit von den produzierten Mengeneinheiten durch eine Polynomfunktion $f$ zweiten Grades mit $f(x)=ax�+bx+c (a,b,c\in\mathbb{R},a\neq 0)$ beschrieben werden.
	
	F�r welche Werte von $a$ liegt im sreng monoton wachsenden Bereich der Funktion ein progressiver bzw. ein degressiver Kostenverlauf vor? Begr�nde deine Antwort!
	
	\item die erste Ableitung einer Kostenfunktion bezeichnet man als \textit{Grenzkostenfunktion}. Diese beschreibt n�herungsweise die Kostensteigerung, wenn der Produktionsumfang vergr��ert wird. Berechne, um wie viel GE sich der Wert der Grenzkostenfunktion bei einem Produktionsumfang von $x=50$ ME vom tats�chlichen Zuwachs der Kosten bei Firma A unterscheidet, wenn der Produktionsumfang von 50 ME auf 51 ME erh�ht wird!
	
	F�r die vorliegende Kostenfunktion gilt die Aussage: "`Die Funktionswerte der Grenzkostenfunktion sind immer positiv."' Interpretiere diese Aussage im Hinblick auf den Verlauf!
	
	\item F�r die Festlegung des Produktionsplans ist es erforderlich, die durchschnittlichen Kosten pro erzeugter ME in Abh�ngigkeit von der Produktionsmenge zu kennen. Die St�ckkostenfunktion gibt den durchschnittlichen Preis pro erzeugter ME an.
	
	Ermittle die St�ckkostenfunktion $\overline{K}_B(x)$ bei Firma B! Gib an, bei welcher Produktionsmenge die durchschnittlichen St�ckkosten bei Firma B am kleinsten sind!
				\end{enumerate}\leer
				
				
\antwort{\subsection{L�sungserwartung:}
\begin{enumerate}
	\item $K_B(x)=0,5x�+100x+15000$
	
	$K_B'(x)=x+100$
	
	$K_B''(x)=1>0$
	
	Da die zweite Ableitung positiv ist, ist die Funktion linksgekr�mmt. Es liegt progressives Wachstum vor. 
	
	\textit{Andere richtige Begr�ndungen (z.B. anhand des Graphen) sind auch zul�ssig.}
	
	$f(x)=ax�+bx+c$
	
	Wenn $a>0$ ist, ist der Graph der Kostenfunktion linksgekr�mmt. Es liegt progressives Wachstum vor.
	
	Wenn $a<0$ ist, ist der Graph der Kostenfunktion rechtsgekr�mmt. Es liegt degressives Wachstum vor.
	
	\item Grenzkostenfunktion $K_A'(x)=0,03x�-6x+350$
	
	$K_A'(50)=125$
	
	$K_A(51)-K_A(50)=31\,373,51-31\,250=123,51$
	
	Der Wert der Grenzkostenfunktion bei einem Produktionsumfang von $x=50$ ME unterscheidet sich vom tats�chlichen Zuwachs der Kosten bei Firma A um 1,49 GE.
	
	Da die Kostenfunktion $K(x)$ im angegebenen Bereich monoton steigend ist, gilt $K'(x)>0 \rightarrow$ die Funktionswerte der Grenzkostenfunktion (=Ableitungsfunktion der Kostenfunktion) sind also immer positiv.
	
	\item $K_B(x)=0,5x�+100x+15\,000$
	
	$\overline{K}_B(x)=\frac{K_B(x)}{x}$
	
	$\overline{K}_B(x)=0,5x+100+\frac{15\,000}{x}$
	
	$\overline{K}_B'(x)=0,5-\frac{15\,000}{x�}$
	
	$\overline{K}_B'(x)=0 \rightarrow 0,5x�=15\,000 \rightarrow x=\sqrt{30\,000}$
	
	$x\approx 173,2$
	
	Bei einer Produktion von ca. 173 Mengeneinheiten sind die durchschnittlichen St�ckkosten bei Firma B am kleinsten.
\end{enumerate}}
\end{langesbeispiel}