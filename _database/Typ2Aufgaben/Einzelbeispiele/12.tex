\section{12 - MAT - AN 1.3, AN 3.3, AG 2.3 - Kostenfunktion - BIFIE Aufgabensammlung}

\begin{langesbeispiel} \item[0] %PUNKTE DES BEISPIELS
Im Zuge einer betriebswirtschaftlichen Analyse und Beratung werden bei zwei Firmen die Kostenverläufe in Abhängigkeit von der Produktionsmenge untersucht.
				
				Bei Firma A wird der Zusammenhang zwischen der monatlichen Produktionsmenge $x$ (in Mengeneinheiten [ME]) und den entstehenden Produktionskosten $K_A(x)$ (in Geldeinheiten [GE]) durch die Kostenfunktion $K_A$ mit $$K_A(x)=0,01x^3-3x^2+350x+20000$$ beschrieben. Firma A kann monatlich maximal 400 ME produzieren. In der untenstehenden Abbildung ist der Graph der Funktion $K_A$ im Intervall [0;400] dargestellt.\vspace{0,2cm}
				
				\psset{xunit=0.03cm,yunit=0.20cm,algebraic=true,dimen=middle,dotstyle=o,dotsize=5pt 0,linewidth=0.8pt,arrowsize=3pt 2,arrowinset=0.25}
\begin{pspicture*}(-52.99077542799506,-5.578723404255297)(441.7596374622275,34.54127659574461)
\psaxes[labelFontSize=\scriptstyle,xAxis=true,yAxis=true,labels=x,Dx=50.,Dy=4.,ticksize=-2pt 0,subticks=0]{->}(0,0)(0.,0.)(441.7596374622275,34.54127659574461)
\psplot[linewidth=1.2pt,plotpoints=200]{0}{441.7596374622275}{(0.01*x^(3.0)-3.0*x^(2.0)+350.0*x+20000.0)/10000.0}
\begin{scriptsize}
\rput[tl](226.22984894259397,9.221276595744664){$K_A$}
\rput[tl](5.514441087612724,34.014468085106316){$K_A(x)$ (in GE)}
\rput[tl](386.09707955689225,1.996595744680852){$x$ (in ME)}
\rput[tl](-31.647814702919874,4.40425531914893){40.000}
\rput[tl](-31.647814702919874,8.40425531914893){80.000}
\rput[tl](-35.647814702919874,12.40425531914893){120.000}
\rput[tl](-35.647814702919874,16.40425531914893){160.000}
\rput[tl](-35.647814702919874,20.40425531914893){200.000}
\rput[tl](-35.647814702919874,24.40425531914893){240.000}
\rput[tl](-35.647814702919874,28.40425531914893){280.000}
\rput[tl](-35.647814702919874,32.40425531914893){320.000}
\end{scriptsize}
\end{pspicture*}

Bei Firma B wird der Zusammenhang zwischen der monatlichen Produktionsmenge $x$ (in ME) und den entstehenden Produktionskosten $K_B(x)$ (in GE) durch die Kostenfunktion $K_B$ mit $K_B(x)=0,5x^2+100x+15000$ beschrieben. Firma B kann monatlich maximal 300 ME produzieren.%Aufgabentext

\begin{aufgabenstellung}
\item %Aufgabentext

\Subitem{Untersuche, ob der Kostenverlauf bei Firma B progressiv oder degressiv ist! Begründe deine Antwort.} %Unterpunkt1

Allgemein kann eine solche Kostenfunktion in Abhängigkeit von den produzierten Mengeneinheiten durch eine Polynomfunktion $f$ zweiten Grades mit $f(x)=ax^2+bx+c (a,b,c\in\mathbb{R},a\neq 0)$ beschrieben werden.

\Subitem{Für welche Werte von $a$ liegt im streng monoton wachsenden Bereich der Funktion ein progressiver bzw. ein degressiver Kostenverlauf vor? Begründe deine Antwort.} %Unterpunkt2

\item Die erste Ableitung einer Kostenfunktion bezeichnet man als \textit{Grenzkostenfunktion}. Diese beschreibt näherungsweise die Kostensteigerung, wenn der Produktionsumfang vergrößert wird.

\Subitem{Berechne, um wie viel GE sich der Wert der Grenzkostenfunktion bei einem Produktionsumfang von $x=50$ ME vom tatsächlichen Zuwachs der Kosten bei Firma A unterscheidet, wenn der Produktionsumfang von 50 ME auf 51 ME erhöht wird.}

\Subitem{Für die vorliegende Kostenfunktion gilt die Aussage: "`Die Funktionswerte der Grenzkostenfunktion sind immer positiv."' Interpretiere diese Aussage im Hinblick auf den Verlauf!}

\item Für die Festlegung des Produktionsplans ist es erforderlich, die durchschnittlichen Kosten pro erzeugter ME in Abhängigkeit von der Produktionsmenge zu kennen. Die Stückkostenfunktion gibt den durchschnittlichen Preis pro erzeugter ME an.

\Subitem{Ermittle die Stückkostenfunktion $\overline{K}_B(x)$ bei Firma B.}

\Subitem{Gib an, bei welcher Produktionsmenge die durchschnittlichen Stückkosten bei Firma B am kleinsten sind.}

\end{aufgabenstellung}

\begin{loesung}
\item \subsection{Lösungserwartung:} 

\Subitem{$K_B(x)=0,5x^2+100x+15000$
	
	$K_B'(x)=x+100$
	
	$K_B''(x)=1>0$
	
	Da die zweite Ableitung positiv ist, ist die Funktion linksgekrümmt. Es liegt progressives Wachstum vor. } %Lösung von Unterpunkt1
\Subitem{$f(x)=ax^2+bx+c$
	
	Wenn $a>0$ ist, ist der Graph der Kostenfunktion linksgekrümmt. Es liegt progressives Wachstum vor.
	
	Wenn $a<0$ ist, ist der Graph der Kostenfunktion rechtsgekrümmt. Es liegt degressives Wachstum vor.} %%Lösung von Unterpunkt2

\setcounter{subitemcounter}{0}
\subsection{Lösungsschlüssel:}
 
\Subitem{Ein Punkt für eine richtige Begründung. Andere richtige Begründungen (z.B. anhand des Graphen) sind auch zulässig.} %Lösungschlüssel von Unterpunkt1
\Subitem{Ein Punkt für eine richtige Begründung.} %Lösungschlüssel von Unterpunkt2

\item \subsection{Lösungserwartung:} 

\Subitem{Grenzkostenfunktion $K_A'(x)=0,03x^2-6x+350$
	
	$K_A'(50)=125$
	
	$K_A(51)-K_A(50)=31\,373,51-31\,250=123,51$
	
	Der Wert der Grenzkostenfunktion bei einem Produktionsumfang von $x=50$ ME unterscheidet sich vom tatsächlichen Zuwachs der Kosten bei Firma A um 1,49 GE.} %Lösung von Unterpunkt1
\Subitem{Da die Kostenfunktion $K(x)$ im angegebenen Bereich monoton steigend ist, gilt $K'(x)>0 \rightarrow$ die Funktionswerte der Grenzkostenfunktion (=Ableitungsfunktion der Kostenfunktion) sind also immer positiv.} %%Lösung von Unterpunkt2

\setcounter{subitemcounter}{0}
\subsection{Lösungsschlüssel:}
 
\Subitem{Ein Punkt für die richtige Angabe der Geldeinheiten.} %Lösungschlüssel von Unterpunkt1
\Subitem{Ein Punkt für die richtige Interpretation.} %Lösungschlüssel von Unterpunkt2

\item \subsection{Lösungserwartung:} 

\Subitem{$K_B(x)=0,5x^2+100x+15\,000$
	
	$\overline{K}_B(x)=\frac{K_B(x)}{x}$
	
	$\overline{K}_B(x)=0,5x+100+\frac{15\,000}{x}$} %Lösung von Unterpunkt1
\Subitem{$\overline{K}_B'(x)=0,5-\frac{15\,000}{x^2}$
	
	$\overline{K}_B'(x)=0 \rightarrow 0,5x^2=15\,000 \rightarrow x=\sqrt{30\,000}$
	
	$x\approx 173,2$
	
	Bei einer Produktion von ca. 173 Mengeneinheiten sind die durchschnittlichen Stückkosten bei Firma B am kleinsten.} %%Lösung von Unterpunkt2

\setcounter{subitemcounter}{0}
\subsection{Lösungsschlüssel:}
 
\Subitem{Ein Punkt für die Stückkostenfunktion.} %Lösungschlüssel von Unterpunkt1
\Subitem{Ein Punkt für die richtige Produktionsmenge.} %Lösungschlüssel von Unterpunkt2

\end{loesung}

\end{langesbeispiel}