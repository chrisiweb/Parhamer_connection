\section{27 - MAT - AN 1.1, AN 1.3, FA 2.2, WS 1.2, WS 1.3, WS 1.1 - Kartoffeln in Österreich - BIFIE Aufgabensammlung}

\begin{langesbeispiel} \item[0] %PUNKTE DES BEISPIELS
				Die Kartoffel ist weltweit eines der wichtigsten Nahrungsmittel.
				
				Die nachstehende Grafik zeigt die Entwicklung der Kartoffelerzeugung in Österreich vom Jahr 2000 bis zum Jahr 2011.
				
				\begin{center}
					\textbf{Kartoffelerzeugung}
					
					\resizebox{0.8\linewidth}{!}{\psset{xunit=1.0cm,yunit=0.7cm,algebraic=true,dimen=middle,dotstyle=o,dotsize=5pt 0,linewidth=0.8pt,arrowsize=3pt 2,arrowinset=0.25}
\begin{pspicture*}(-1.366338441507557,-1.4071090537819535)(11.835465721609575,9.72757238201069)
\multips(0,0)(0,1.0){12}{\psline[linestyle=dashed,linecap=1,dash=1.5pt 1.5pt,linewidth=0.4pt,linecolor=lightgray]{c-c}(0,0)(11.835465721609575,0)}
\multips(0,0)(1.0,0){13}{\psline[linestyle=dashed,linecap=1,dash=1.5pt 1.5pt,linewidth=0.4pt,linecolor=lightgray]{c-c}(0,0)(0,9.72757238201069)}
\psaxes[labelFontSize=\scriptstyle,xAxis=true,yAxis=true,yLabels={,100, 200, 300, 400, 500, 600, 700, 800, 900},Ox=2000,Dx=1.,Dy=1.,ticksize=-2pt 0,subticks=2]{->}(0,0)(0.,0.)(11.835465721609575,9.72757238201069)
\psline[linewidth=1.2pt](0.,6.95)(1.,6.95)
\psline[linewidth=1.2pt](1.,6.95)(2.,6.84)
\psline[linewidth=1.2pt](2.,6.84)(3.,5.6)
\psline[linewidth=1.2pt](3.,5.6)(4.,6.93)
\psline[linewidth=1.2pt](4.,6.93)(5.,7.63)
\psline[linewidth=1.2pt](5.,7.63)(6.,6.55)
\psline[linewidth=1.2pt](6.,6.55)(7.,6.69)
\psline[linewidth=1.2pt](7.,6.69)(8.,7.57)
\psline[linewidth=1.2pt](8.,7.57)(9.,7.22)
\psline[linewidth=1.2pt](9.,7.22)(10.,6.72)
\psline[linewidth=1.2pt](10.,6.72)(11.,8.16)
\begin{scriptsize}
\rput[tl](-1,6){$\rotatebox{90}{\text{Menge in 1000 Tonnen}}$}
\rput[tl](0.02236339286879133,7.3){695}
\rput[tl](0.75,6.8){695}
\rput[tl](1.75,7.3){684}
\rput[tl](2.75,5.5){560}
\rput[tl](3.75,7.4){693}
\rput[tl](4.75,8){763}
\rput[tl](5.75,6.45){655}
\rput[tl](6.75,6.6){669}
\rput[tl](7.75,7.9){757}
\rput[tl](8.75,7.6){722}
\rput[tl](9.75,6.6){672}
\rput[tl](10.75,8.55){816}
\rput[tl](5.151545584806671,-0.8116715438465181){Jahr}
\end{scriptsize}
\end{pspicture*}}
				\end{center}\leer
				
				In der nachstehenden Abbildung werden Kartoffelexporte und -importe für den gleichen Zeitraum einander gegenübergestellt.
				
				\begin{center}
					\textbf{Außenhandel}
					
					\resizebox{0.8\linewidth}{!}{\newrgbcolor{qqzzff}{0. 0.6 1.}
\psset{xunit=1.0cm,yunit=0.3cm,algebraic=true,dimen=middle,dotstyle=o,dotsize=5pt 0,linewidth=0.8pt,arrowsize=3pt 2,arrowinset=0.25}
\begin{pspicture*}(-1.38,-3.449354157872534)(12.4,28.08759814267654)
\multips(0,-0)(0,5.0){7}{\psline[linestyle=dashed,linecap=1,dash=1.5pt 1.5pt,linewidth=0.4pt,linecolor=lightgray]{c-c}(0,0)(12.4,0)}
\multips(0,0)(1.0,0){14}{\psline[linestyle=dashed,linecap=1,dash=1.5pt 1.5pt,linewidth=0.4pt,linecolor=lightgray]{c-c}(0,0)(0,28.08759814267654)}
\psaxes[labelFontSize=\scriptstyle,xAxis=true,yAxis=true,yLabels={,50, 100, 150, 200, 250}, Ox=2000,Dx=1.,Dy=5.,ticksize=-2pt 0,subticks=2]{->}(0,0)(0.,0.)(12.4,28.08759814267654)
\psline[linewidth=1.2pt,linecolor=qqzzff](0.,8.4)(1.,9.1)
\psline[linewidth=1.2pt,linecolor=qqzzff](1.,9.1)(2.,12.9)
\psline[linewidth=1.2pt,linecolor=qqzzff](2.,12.9)(3.,13.5)
\psline[linewidth=1.2pt,linecolor=qqzzff](3.,13.5)(4.,11.9)
\psline[linewidth=1.2pt,linecolor=qqzzff](4.,11.9)(5.,12.8)
\psline[linewidth=1.2pt,linecolor=qqzzff](5.,12.8)(6.,16.4)
\psline[linewidth=1.2pt,linecolor=qqzzff](6.,16.4)(7.,17.7)
\psline[linewidth=1.2pt,linecolor=qqzzff](7.,17.7)(8.,15.4)
\psline[linewidth=1.2pt,linecolor=qqzzff](8.,15.4)(9.,18.2)
\psline[linewidth=1.2pt,linecolor=qqzzff](9.,18.2)(10.,19.8)
\psline[linewidth=1.2pt,linecolor=qqzzff](10.,19.8)(11.,17.2)
\psline[linewidth=1.2pt,linestyle=dashed,dash=7pt 7pt,linecolor=red](0.,7.5)(1.,6.6)
\psline[linewidth=1.2pt,linestyle=dashed,dash=7pt 7pt,linecolor=red](1.,6.6)(2.,9.)
\psline[linewidth=1.2pt,linestyle=dashed,dash=7pt 7pt,linecolor=red](2.,9.)(3.,8.2)
\psline[linewidth=1.2pt,linestyle=dashed,dash=7pt 7pt,linecolor=red](3.,8.2)(4.,10.2)
\psline[linewidth=1.2pt,linestyle=dashed,dash=7pt 7pt,linecolor=red](4.,10.2)(5.,14.8)
\psline[linewidth=1.2pt,linestyle=dashed,dash=7pt 7pt,linecolor=red](5.,14.8)(6.,13.2)
\psline[linewidth=1.2pt,linestyle=dashed,dash=7pt 7pt,linecolor=red](6.,13.2)(7.,13.2)
\psline[linewidth=1.2pt,linestyle=dashed,dash=7pt 7pt,linecolor=red](7.,13.2)(8.,17.1)
\psline[linewidth=1.2pt,linestyle=dashed,dash=7pt 7pt,linecolor=red](8.,17.1)(9.,17.2)
\psline[linewidth=1.2pt,linestyle=dashed,dash=7pt 7pt,linecolor=red](9.,17.2)(10.,16.7)
\psline[linewidth=1.2pt,linestyle=dashed,dash=7pt 7pt,linecolor=red](10.,16.7)(11.,20.8)
\psline[linewidth=1.2pt,linecolor=qqzzff](1.,24.)(2.,24.)
\psline[linewidth=1.2pt,linestyle=dashed,dash=7pt 7pt,linecolor=red](1.,22.)(2.,22.)
\begin{scriptsize}
\rput[tl](6.1,-1.8478682988602764){Jahr}
\rput[tl](0.1,9.5){\qqzzff{84}}
\rput[tl](0.85,10.5){\qqzzff{91}}
\rput[tl](1.8,14.1){\qqzzff{129}}
\rput[tl](2.8,14.6){\qqzzff{135}}
\rput[tl](3.8,13.25){\qqzzff{119}}
\rput[tl](4.8,12.5){\qqzzff{128}}
\rput[tl](5.8,17.65){\qqzzff{164}}
\rput[tl](6.8,18.7){\qqzzff{177}}
\rput[tl](7.8,15){\qqzzff{154}}
\rput[tl](8.8,19.45){\qqzzff{182}}
\rput[tl](9.8,20.8){\qqzzff{198}}
\rput[tl](10.8,16.9){\qqzzff{172}}
\rput[tl](0.1,6.9){\red{75}}
\rput[tl](0.85,6.3){\red{66}}
\rput[tl](1.85,10.1){\red{90}}
\rput[tl](2.85,8){\red{82}}
\rput[tl](3.8,9.7){\red{102}}
\rput[tl](4.8,16){\red{148}}
\rput[tl](5.8,12.9){\red{132}}
\rput[tl](6.8,12.9){\red{132}}
\rput[tl](7.8,18.2){\red{171}}
\rput[tl](8.8,16.8){\red{172}}
\rput[tl](9.8,16.5){\red{167}}
\rput[tl](10.8,21.8){\red{208}}
\rput[tl](2.4,24.5){\qqzzff{Kartoffelimporte}}
\rput[tl](2.4,22.5){\red{Kartoffelexporte}}
\rput[tl](-1.08,20.08016884761525){$\rotatebox{90}{\text{Menge in 1000 Tonnen}}$}
\end{scriptsize}
\end{pspicture*}}
					\end{center}

\subsection{Aufgabenstellung:}
\begin{enumerate}
	\item Entnehmen Sie der entsprechenden Graphik, zwischen welchen (aufeinanderfolgenden)
Jahren die absolute Zunahme (in Tonnen) und die relative Zunahme (in Prozent) der Erzeugung im Vergleich zum Vorjahr jeweils am größten war! Gib die entsprechenden
Werte an!

Im vorliegenden Fall fand die größte relative Zunahme der Erzeugung in einem anderen
Zeitintervall statt als die größte absolute Zunahme.\\
Gib eine mathematische Begründung an, warum die größte relative Zunahme und
die größte absolute Zunahme einer Größe oder eines Prozesses nicht im gleichen Zeitintervall stattfinden müssen!

\item Berechne und interpretiere den Ausdruck $\frac{E_{2011}-E_{2000}}{11}$, wobei $E_\text{Jahr}$ die Exportmenge in einem Kalenderjahr angibt! Gib bei der Interpretation auch die entsprechende Einheit an.

Die Exportentwicklung von 2000 bis 2011 soll durch eine lineare Funktion $f$ approximiert
werden, wobei die Variable t die Anzahl der seit 2000 vergangenen Jahre sein soll. Die
Funktionswerte für die Jahre 2000 und 2011 sollen dabei mit den in der Graphik angeführten Werten übereinstimmen. Gib eine Gleichung dieser Funktion $f$ an!

\item Stelle in der nachstehenden Abbildung die Differenz "`Export minus Import"' der Mengen an Kartoffeln für die Jahre 2003 bis 2009 in einem Säulendiagramm dar!
	
	\begin{center}
					\textbf{Saldo Export-Import}
					
					\resizebox{0.8\linewidth}{!}{\psset{xunit=1.4cm,yunit=0.05cm,algebraic=true,dimen=middle,dotstyle=o,dotsize=5pt 0,linewidth=0.8pt,arrowsize=3pt 2,arrowinset=0.25}
\begin{pspicture*}(-1,-69.33333333333154)(7.151969309462931,66.39999999999834)
\multips(0,-60)(0,4.0){34}{\psline[linestyle=dashed,linecap=1,dash=1.5pt 1.5pt,linewidth=0.4pt,linecolor=lightgray]{c-c}(0,0)(7.151969309462931,0)}
\multips(0,0)(100.0,0){1}{\psline[linestyle=dashed,linecap=1,dash=1.5pt 1.5pt,linewidth=0.4pt,linecolor=lightgray]{c-c}(0,-60)(0,66.39999999999834)}
\psaxes[labelFontSize=\scriptstyle,xAxis=true,yAxis=true,labels=y,Dx=1.,Dy=20.,ticksize=-2pt 0,subticks=2]{->}(0,0)(0.,-60)(7.151969309462931,66.39999999999834)
\antwort{\pspolygon[linecolor=blue,fillcolor=blue,fillstyle=solid,opacity=0.5](0.2,0.)(0.2,-53.)(0.8,-53.)(0.8,0.)
\pspolygon[linecolor=blue,fillcolor=blue,fillstyle=solid,opacity=0.5](1.2,0.)(1.2,-17.)(1.8,-17.)(1.8,0.)
\pspolygon[linecolor=blue,fillcolor=blue,fillstyle=solid,opacity=0.5](2.2,0.)(2.2,20.)(2.8,20.)(2.8,0.)
\pspolygon[linecolor=blue,fillcolor=blue,fillstyle=solid,opacity=0.5](3.2,0.)(3.2,-32.)(3.8,-32.)(3.8,0.)
\pspolygon[linecolor=blue,fillcolor=blue,fillstyle=solid,opacity=0.5](4.2,0.)(4.2,-45.)(4.8,-45.)(4.8,0.)
\pspolygon[linecolor=blue,fillcolor=blue,fillstyle=solid,opacity=0.5](5.2,0.)(5.2,17.)(5.8,17.)(5.8,0.)
\pspolygon[linecolor=blue,fillcolor=blue,fillstyle=solid,opacity=0.5](6.2,0.)(6.2,-10.)(6.8,-10.)(6.8,0.)}
\begin{scriptsize}
\rput[tl](0.3,-3.1999999999998883){2003}
\rput[tl](1.3,-3.1999999999998883){2004}
\rput[tl](2.3,-3.1999999999998883){2005}
\rput[tl](3.3,-3.1999999999998883){2006}
\rput[tl](4.3,-3.1999999999998883){2007}
\rput[tl](5.3,-3.1999999999998883){2008}
\rput[tl](6.3,-3.1999999999998883){2009}
\rput[tl](3,-63){Jahr}
\rput[tl](-0.8,26.93333333333268){$\rotatebox{90}{\text{Saldo in 1000 Tonnen}}$}
\end{scriptsize}
\end{pspicture*}}
\end{center}

Berechne das arithmetische Mittel dieser Differenz für die genannten Jahre!

\item Ein Index ist eine statistische Kennziffer, um die Entwicklung von Größen im Zeitverlauf darzustellen. Oft wird der Ausgangswert mit dem Basiswert 100 versehen. Ein Index von 120 bedeutet beispielsweise, dass eine Größe seit dem Basiszeitpunkt um 20\,\% gestiegen ist.

Die nachstehende Graphik zeigt die Entwicklung der in Österreich verzehrten Kartoffelmenge (Nahrungsverbrauch) bezogen auf das Jahr 2000.

\begin{center}
					\textbf{Nahrungsverbrauch}
					
					\resizebox{0.8\linewidth}{!}{\psset{xunit=1.6cm,yunit=0.8cm,algebraic=true,dimen=middle,dotstyle=o,dotsize=5pt 0,linewidth=0.8pt,arrowsize=3pt 2,arrowinset=0.25}
\begin{pspicture*}(-0.8674067054466462,-1.1184701870029385)(6.267692353635041,8.521136170252973)
\multips(0,0)(0,1.0){10}{\psline[linestyle=dashed,linecap=1,dash=1.5pt 1.5pt,linewidth=0.4pt,linecolor=lightgray]{c-c}(0,0)(7.267692353635041,0)}
\multips(0,0)(100.0,0){1}{\psline[linestyle=dashed,linecap=1,dash=1.5pt 1.5pt,linewidth=0.4pt,linecolor=lightgray]{c-c}(0,0)(0,8.521136170252973)}
\psaxes[labelFontSize=\scriptstyle,xAxis=true,yAxis=true,labels=y,yLabels={,\text{94,0}, \text{96,0},\text{98,0},\text{100,0},\text{102,0},\text{104,0},\text{106,0},\text{108,0},\text{110,0}},Dx=1.,Dy=1.,ticksize=-2pt 0,subticks=2]{->}(0,0)(0.,0.)(6.267692353635041,8.521136170252973)
\pspolygon[linecolor=blue,fillcolor=blue,fillstyle=solid,opacity=1.0](0.25,0.)(0.25,3.)(0.5,3.)(0.5,0.)
\pspolygon[linecolor=red,fillcolor=red,fillstyle=solid,opacity=1.0](0.75,0.)(0.5,0.)(0.5,3.)(0.75,3.)
\pspolygon[linecolor=blue,fillcolor=blue,fillstyle=solid,opacity=1.0](1.25,0.)(1.25,5.1)(1.5,5.1)(1.5,0.)
\pspolygon[linecolor=red,fillcolor=red,fillstyle=solid,opacity=1.0](1.5,5.15)(1.75,5.15)(1.75,0.)(1.5,0.)
\pspolygon[linecolor=blue,fillcolor=blue,fillstyle=solid,opacity=1.0](2.25,0.)(2.25,4.)(2.5,4.)(2.5,0.)
\pspolygon[linecolor=red,fillcolor=red,fillstyle=solid,opacity=1.0](2.5,0.)(2.5,3.4)(2.75,3.4)(2.75,0.)
\pspolygon[linecolor=blue,fillcolor=blue,fillstyle=solid,opacity=1.0](3.25,0.)(3.25,3.9)(3.5,3.9)(3.5,0.)
\pspolygon[linecolor=red,fillcolor=red,fillstyle=solid,opacity=1.0](3.5,0.)(3.5,2.8)(3.75,2.8)(3.75,0.)
\pspolygon[linecolor=blue,fillcolor=blue,fillstyle=solid,opacity=1.0](4.25,0.)(4.25,7.2)(4.5,7.2)(4.5,0.)
\pspolygon[linecolor=red,fillcolor=red,fillstyle=solid,opacity=1.0](4.5,0.)(4.5,5.6)(4.75,5.6)(4.75,0.)
\pspolygon[linecolor=blue,fillcolor=blue,fillstyle=solid,opacity=1.0](5.25,0.)(5.25,6.05)(5.5,6.05)(5.5,0.)
\pspolygon[linecolor=red,fillcolor=red,fillstyle=solid,opacity=1.0](5.5,0.)(5.5,4.2)(5.75,4.2)(5.75,0.)
\pspolygon[linecolor=blue,fillcolor=blue,fillstyle=solid,opacity=1.0](0.25,8.)(0.25,7.5)(0.5,7.5)(0.5,8.)
\pspolygon[linecolor=red,fillcolor=red,fillstyle=solid,opacity=1.0](0.25,7.)(0.25,6.5)(0.5,6.5)(0.5,7.)
\begin{scriptsize}
\rput[tl](0.63,7.85){Nahrungsverbrauch}
\rput[tl](0.63,6.85){Nahrungsverbrauch pro Kopf}
\rput[tl](-0.8,6.949255369364486){$\rotatebox{90}{\text{Index bezogen auf das Jahr 2000}}$}
\rput[tl](3,-0.7){Jahr}
\rput[tl](0.3,-0.15){2000}
\rput[tl](1.3,-0.15){2002}
\rput[tl](2.3,-0.15){2004}
\rput[tl](3.3,-0.15){2006}
\rput[tl](4.3,-0.15){2008}
\rput[tl](5.3,-0.15){2010}
\end{scriptsize}
\end{pspicture*}}
\end{center}

Geben Sie jeweils ein Jahr an, in dem die Einwohnerzahl in Österreich höher bzw. niedriger
war als im Jahr 2000! Begründe deine Antwort!

Zeichne in die nachstehende Graphik zwei mögliche Säulen für ein Jahr, in dem der
absolute Nahrungsverbrauch niedriger und die Bevölkerungszahl höher war als im
Jahr 2000!

\begin{center}
					\textbf{Nahrungsverbrauch}
					
					\resizebox{0.8\linewidth}{!}{\psset{xunit=5cm,yunit=0.8cm,algebraic=true,dimen=middle,dotstyle=o,dotsize=5pt 0,linewidth=0.8pt,arrowsize=3pt 2,arrowinset=0.25}
\begin{pspicture*}(-0.35,-1.1184701870029385)(2.267692353635041,11.521136170252973)
\multips(0,0)(0,1.0){12}{\psline[linestyle=dashed,linecap=1,dash=1.5pt 1.5pt,linewidth=0.4pt,linecolor=lightgray]{c-c}(0,0)(7.267692353635041,0)}
\multips(0,0)(100.0,0){1}{\psline[linestyle=dashed,linecap=1,dash=1.5pt 1.5pt,linewidth=0.4pt,linecolor=lightgray]{c-c}(0,0)(0,8.521136170252973)}
\psaxes[labelFontSize=\scriptstyle,xAxis=true,yAxis=true,labels=y,yLabels={,\text{0,0}, \text{20,0},\text{40,0},\text{60,0},\text{80,0},\text{100,0},\text{120,0},\text{140,0},\text{160,0},\text{180,0},\text{200,0}},Dx=1.,Dy=1.,ticksize=-2pt 0]{->}(0,0)(0.,0.)(2.267692353635041,11.521136170252973)
\pspolygon[linecolor=blue,fillcolor=blue,fillstyle=solid,opacity=1.0](0.25,0.)(0.25,3.)(0.5,3.)(0.5,0.)
\pspolygon[linecolor=red,fillcolor=red,fillstyle=solid,opacity=1.0](0.75,0.)(0.5,0.)(0.5,3.)(0.75,3.)

\pspolygon[linecolor=blue,fillcolor=blue,fillstyle=solid,opacity=1.0](0.25,11.)(0.25,10.5)(0.35,10.5)(0.35,11.)
\pspolygon[linecolor=red,fillcolor=red,fillstyle=solid,opacity=1.0](0.25,10.)(0.25,9.5)(0.35,9.5)(0.35,10.)
\begin{scriptsize}
\rput[tl](0.4,10.85){Nahrungsverbrauch}
\rput[tl](0.4,9.85){Nahrungsverbrauch pro Kopf}
\rput[tl](-0.25,7.6){$\rotatebox{90}{\text{Index bezogen auf das Jahr 2000}}$}
\rput[tl](1,-0.4){Jahr}
\rput[tl](0.44,-0.15){2000}
\rput[tl](1.44,-0.15){2002}
\end{scriptsize}
\end{pspicture*}}
\end{center}
						\end{enumerate}\leer
				
\antwort{\subsection{Lösungserwartung:}
\begin{enumerate}
	\item absolute Zunahme zwischen 2003 und 2004: 133 000 Tonnen\\
absolute Zunahme zwischen 2010 und 2011: 144 000 Tonnen\\
Die größte absolute Zunahme war im Zeitintervall von 2010 bis 2011.\\
relative Zunahme zwischen 2003 und 2004: 23,75\,\%\\
Lösungsintervall in Prozent: $[23; 24]$\\
relative Zunahme zwischen 2010 und 2011: ca. 21,43\,\%\\
Lösungsintervall in Prozent: $[21; 22]$
Die größte relative Zunahme war zwischen 2003 und 2004.\\
Da für die Berechnung der relativen Zunahme einer Größe auch der Bezugswert entscheidend ist, müssen größte absolute Zunahme und größte relative Zunahme einer Größe oder
eines Prozesses nicht im gleichen Zeitintervall stattfinden. Äquivalente Formulierungen sind
ebenfalls als richtig zu werten.

\item $\frac{E_{2011}-E_{2000}}{11}\approx 12\,000$ Tonnen pro Jahr.

Die durchschnittliche Zunahme des österreichischen Kartoffelexports beträgt
ca. 12 000 Tonnen pro Jahr.\\
Lösungsintervall: $[12 000; 12 100]$
Die richtige Einheit muss in der Interpretation vorhanden sein.\\
$f$ mit $f(t)=75000+12000\cdot t$ oder $f(t)=75+12\cdot t$ oder $f(t)=\frac{133}{11}\cdot t+75$
Jede jährliche Zunahme aus dem oben angeführten Lösungsintervall muss akzeptiert werden.

\item Lösung Grafik siehe oben

Genauigkeit der Säulenlängen: Toleranzbereich $\pm 5\,000$ Tonnen

arithmetisches Mittel: $\frac{-53-17+20-32-45+17-10}{7}\approx -17$

Lösungsintervall: bei Berechnung in 1\,000 Tonnen: $[-18; -16]$; bei Berechnung in Tonnen:
$[-18\,000; -16\,000]$
	
\item \begin{itemize}
	\item niedrigere Einwohnerzahl (als im Jahr 2000) in den Jahren: 2002 (prozentuelle Veränderung des Nahrungsverbrauchs kleiner als jene des Nahrungsverbrauchs pro Kopf)
\item höhere Einwohnerzahl (als im Jahr 2000) in den Jahren: 2004, 2006, 2008 und 2010
(prozentuelle Veränderung des Nahrungsverbrauchs größer als jene des Nahrungsverbrauchs pro Kopf)
\end{itemize}
Die Angabe eines Jahres ist hier ausreichend.

Die Säule für den gesamten Nahrungsverbrauch muss niedriger sein als die Säule für
100\% im Jahr 2000. Die Säule für den Nahrungsverbrauch pro Kopf muss bei einer steigenden Bevölkerungszahl niedriger als die Säule für den gesamten Nahrungsverbrauch
sein.
			\end{enumerate}}
		\end{langesbeispiel}