\section{11 - MAT - FA 1.6, FA 1.7, FA 2.1, AN 3.3 - Erlös und Gewinn - BIFIE Aufgabensammlung}

\begin{langesbeispiel} \item[0] %PUNKTE DES BEISPIELS
Eine Digital-Spiegelreflexkamera wird zu einem Stückpreis von \EUR{1.320} angeboten.
				
				Ein Produktionsbetrieb kann monatlich maximal 1.800 Stück dieser Kamera produzieren. Es wird dabei angenommen, dass der Verkaufspreis unabhängig von der verkauften Stückzahl $x$ konstant gehalten wird und alle produzierten Kameras auch verkauft werden. Die Funktion $K$ mit $$K(x)=0,00077x^3-0,693x^2+396x+317900$$ beschreibt die Gesamtkosten $K$ für die Produktion in Abhängigkeit von der produzierten Stückzahl $x$.
				
				Die Graphen der Kostenfunktion $K$ und der Erlösfunktion $E$ sind in der nachstehenden Grafik dargestellt.\vspace{0,3cm}
				
				\newrgbcolor{qqwuqq}{0. 0.39215686274509803 0.}
\psset{xunit=0.005cm,yunit=0.2cm,algebraic=true,dimen=middle,dotstyle=o,dotsize=5pt 0,linewidth=0.8pt,arrowsize=3pt 2,arrowinset=0.25}
\begin{pspicture*}(-270.3960407283936,-13.15313488549706)(2061.9493903703883,34.12528078617641)
\multips(0,-10)(0,5.0){9}{\psline[linestyle=dashed,linecap=1,dash=1.5pt 1.5pt,linewidth=0.4pt,linecolor=gray]{c-c}(0,0)(2061.9493903703883,0)}
\multips(0,0)(200.0,0){13}{\psline[linestyle=dashed,linecap=1,dash=1.5pt 1.5pt,linewidth=0.4pt,linecolor=gray]{c-c}(0,-13.15313488549706)(0,34.12528078617641)}
\psaxes[labelFontSize=\scriptstyle,showorigin=false,xAxis=true,yAxis=true,labels=x,Dx=200.,Dy=5,ticksize=-2pt 0,subticks=0]{->}(0,0)(-260.3960407283936,-13.15313488549706)(2061.9493903703883,34.12528078617641)
\psplot[linewidth=1.2pt,plotpoints=200]{0}{1800}{(0.00077*x^(3.0)-0.693*x^(2.0)+396.0*x+317900.0)/100000.0}
\psplot[linewidth=1.2pt,plotpoints=200]{0}{1800}{(1320.0*x)/100000.0}
\antwort{\begin{scriptsize}
\rput[tl](1430.175329649479,2.9012993992141887){G}
\rput[tl](1046.3798839928404,7.146242854811816){MAX}
\end{scriptsize}
\psline[linewidth=1.2pt,linestyle=dashed,dash=5pt 1.5pt](1000.,-20.613229035746011)(1000.,34.66518663592747)
\psline[linewidth=1.6pt](1000.,13.2)(1000.,7.909)
\psplot[linewidth=1.2pt,linestyle=dashed,dash=5pt 1.5pt,plotpoints=200]{0}{1800}{(-7.7E-4*x^(3.0)+0.693*x^(2.0)+924.0*x-317900.0)/100000.0}
\psline[linewidth=1.6pt](1000.,5.291)(1000.,0.)}
\begin{scriptsize}
\rput[tl](80.92939221574203,5.5607334126604195){K}
\rput[tl](582.1195382202467,10.000601385299875){E}
\rput[tl](-220.06549445931726,5.6){500.000}
\rput[tl](-265.06549445931726,10.6){1.000.000}
\rput[tl](-265.06549445931726,15.6){1.500.000}
\rput[tl](-265.06549445931726,20.6){2.000.000}
\rput[tl](-265.06549445931726,25.6){2.500.000}
\rput[tl](-265.06549445931726,30.6){3.000.000}
\rput[tl](-230.06549445931726,-4.6){-500.000}
\rput[tl](-275.06549445931726,-9.6){-1.000.000}
\psdots[dotsize=4pt 0,dotstyle=*](299.22,3.9497)
\rput[bl](270.8540791227122,4.804205086206228){$A$}
\psdots[dotsize=4pt 0,dotstyle=*](1512.82,19.9692)
\rput[bl](1484.2618010283552,20.905714745369643){$B$}
\rput[tl](1800.175329649479,1.4012993992141887){$x$ in Stk.}
\rput[tl](50.175329649479,33){$y$ in \euro}
\end{scriptsize}
\end{pspicture*}%Aufgabentext

\begin{aufgabenstellung}
\item %Aufgabentext

\Subitem{Zeichne in der obigen Abbildung den Graphen der Gewinnfunktion $G$ ein} %Unterpunkt1

Eine Stückpreisänderung wurde vorgenommen und hat bewirkt, dass der Break-even-Point bei einer geringeren Stückzahl erreicht wird.

\Subitem{Gib an, wie der Stückpreis verändert wurde und welchen Einfluss diese Veränderung auf die Lage der Nullstellen der Gewinnfunktion $G$ und den Gewinnbereich hat.} %Unterpunkt2

\item %Aufgabentext

\Subitem{Erstelle die Gleichung der Gewinnfunktion $G$.} %Unterpunkt1
\Subitem{Berechne diejenige Stückzahl, bei der der Gewinn maximal wird.} %Unterpunkt2

\item In der nachstehenden Grafik wurde die Erlösfunktion so abgeändert, dass die Graphen der Kostenfunktion $K$ und der Erlösfunktion $E_{neu}$ einander im Punkt $T$ berühren.\vspace{0,2cm}

\newrgbcolor{qqwuqq}{0. 0.39215686274509803 0.}
\psset{xunit=0.005cm,yunit=0.2cm,algebraic=true,dimen=middle,dotstyle=o,dotsize=5pt 0,linewidth=0.8pt,arrowsize=3pt 2,arrowinset=0.25}
\begin{pspicture*}(-270.3960407283936,-3.15313488549706)(2061.9493903703883,34.12528078617641)
\multips(0,-10)(0,5.0){9}{\psline[linestyle=dashed,linecap=1,dash=1.5pt 1.5pt,linewidth=0.4pt,linecolor=gray]{c-c}(0,0)(2061.9493903703883,0)}
\multips(0,0)(200.0,0){13}{\psline[linestyle=dashed,linecap=1,dash=1.5pt 1.5pt,linewidth=0.4pt,linecolor=gray]{c-c}(0,-13.15313488549706)(0,34.12528078617641)}
\psaxes[labelFontSize=\scriptstyle,showorigin=false,xAxis=true,yAxis=true,labels=x,Dx=200.,Dy=5,ticksize=-2pt 0,subticks=0]{->}(0,0)(-260.3960407283936,-13.15313488549706)(2061.9493903703883,34.12528078617641)
\psplot[linewidth=1.2pt,plotpoints=200]{0}{1800}{(0.00077*x^(3.0)-0.693*x^(2.0)+396.0*x+317900.0)/100000.0}
\psplot{0}{3041.788116004077}{(--0.08711120000000072--0.00720324*x)/1.}
\psdots[dotsize=4pt 0,dotstyle=*](783.7943961192058,5.732970345901708)
\begin{scriptsize}
\rput[tl](80.92939221574203,5.5607334126604195){$K$}
\rput[tl](1450.1195382202467,10.000601385299875){$E_{neu}$}
\rput[tl](-220.06549445931726,5.6){500.000}
\rput[tl](-265.06549445931726,10.6){1.000.000}
\rput[tl](-265.06549445931726,15.6){1.500.000}
\rput[tl](-265.06549445931726,20.6){2.000.000}
\rput[tl](-265.06549445931726,25.6){2.500.000}
\rput[tl](-265.06549445931726,30.6){3.000.000}
\rput[tl](1800.175329649479,1.4012993992141887){$x$ in Stk.}
\rput[tl](50.175329649479,33){$y$ in \euro}
\rput[tl](700.1195382202467,4.400601385299875){$T=(x_T/y_T)$}
\end{scriptsize}
\end{pspicture*}%Aufgabentext

\Subitem{Bestimme die Gleichung der Erlösfunktion $E_{neu}$.} %Unterpunkt1
\Subitem{Interpretiere die Koordinaten des Punktes $T$ im gegebenen Kontext und erkläre, welche Auswirkung die Änderung der Erlösfunktion auf den Gewinnbereich hat.} %Unterpunkt2

\end{aufgabenstellung}

\begin{loesung}
\item \subsection{Lösungserwartung:} 

\Subitem{Graph der Gewinnfunktion: siehe oben.} %Lösung von Unterpunkt1
\Subitem{Der Stückpreis muss erhöht werden. Die Nullstellen liegen weiter auseinander, das heißt, der Gewinnbereich wird größer.} %%Lösung von Unterpunkt2

\setcounter{subitemcounter}{0}
\subsection{Lösungsschlüssel:}
 
\Subitem{Ein Punkt für den richtigen Gewinngraphen.} %Lösungschlüssel von Unterpunkt1
\Subitem{Ein Punkt für die richtige Stückpreisveränderung.} %Lösungschlüssel von Unterpunkt2

\item \subsection{Lösungserwartung:} 

\Subitem{$G(x)=E(x)-K(x)$

$G(x)=1320x-(0,00077x^3-0,693x^2+396x+317900)$

$G(x)=-0,00077x^3+0,693x^2+924x-317900$} %Lösung von Unterpunkt1
\Subitem{Bedingung für maximalen Gewinn:

$G'(x)=0 \rightarrow G'(x)=-0,00231x^2+1386x+924$

$-0,00231x^2+1,386x+924=0 \Rightarrow x_{1,2}=\dfrac{-1,386\pm\sqrt{1,386^2+4\cdot 0,00231\cdot 924}}{-0,00462} \rightarrow (x_1=-400); x_2=1000$

Der maximale Gewinn wird bei einer Stückzahl von 1000 erzielt.} %%Lösung von Unterpunkt2

\setcounter{subitemcounter}{0}
\subsection{Lösungsschlüssel:}
 
\Subitem{Ein Punkt für die richtige Gleichung.} %Lösungschlüssel von Unterpunkt1
\Subitem{Ein Punkt für die richtige Stückzahl.} %Lösungschlüssel von Unterpunkt2

\item \subsection{Lösungserwartung:} 

\Subitem{Die Gleichung der Erlösfunktion $E_{neu}$ lautet:

$E_{neu}(x)=\frac{y_T}{x_T}\cdot x$} %Lösung von Unterpunkt1
\Subitem{Nur bei der Produktionsmenge von $x_T$ Stück wird genau kostendeckend produziert. Kosten und Erlös betragen je \EUR{$y_T$}. Bei dieser Produktionsmenge ist es nicht möglich, mit Gewinn zu produzieren.} %%Lösung von Unterpunkt2

\setcounter{subitemcounter}{0}
\subsection{Lösungsschlüssel:}
 
\Subitem{Ein Punkt für die richtige Gleichung.} %Lösungschlüssel von Unterpunkt1
\Subitem{Ein Punkt für die richtige Interpretation.} %Lösungschlüssel von Unterpunkt2

\end{loesung}

\end{langesbeispiel}