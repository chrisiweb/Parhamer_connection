\section{11 - MAT - FA 1.6, FA 1.7, FA 2.1, AN 3.3 - Erl�s und Gewinn - BIFIE Aufgabensammlung}

\begin{langesbeispiel} \item[0] %PUNKTE DES BEISPIELS
				Eine Digital-Spiegelreflexkamera wird zu einem St�ckpreis von \EUR{1.320} angeboten.
				
				Ein Produktionsbetrieb kann monatlich maximal 1.800 St�ck dieser Kamera produzieren. Es wird dabei angenommen, dass der Verkaufspreis unabh�ngig von der verkauften St�ckzahl $x$ konstant gehalten wird und alle produzierten Kameras auch verkauft werden. Die Funktion $K$ mit $$K(x)=0,00077x�-0,693x�+396x+317900$$ beschreibt die Gesamtkosten $K$ f�r die Produktion in Abh�ngigkeit von der produzierten St�ckzahl $x$.
				
				Die Graphen der Kostenfunktion $K$ und der Erl�sfunktion $E$ sind in der nachstehenden Grafik dargestellt.
				\leer
				
				\newrgbcolor{qqwuqq}{0. 0.39215686274509803 0.}
\psset{xunit=0.005cm,yunit=0.2cm,algebraic=true,dimen=middle,dotstyle=o,dotsize=5pt 0,linewidth=0.8pt,arrowsize=3pt 2,arrowinset=0.25}
\begin{pspicture*}(-270.3960407283936,-13.15313488549706)(2061.9493903703883,34.12528078617641)
\multips(0,-10)(0,5.0){9}{\psline[linestyle=dashed,linecap=1,dash=1.5pt 1.5pt,linewidth=0.4pt,linecolor=lightgray]{c-c}(-260.3960407283936,0)(2061.9493903703883,0)}
\multips(-200,0)(200.0,0){13}{\psline[linestyle=dashed,linecap=1,dash=1.5pt 1.5pt,linewidth=0.4pt,linecolor=lightgray]{c-c}(0,-13.15313488549706)(0,34.12528078617641)}
\psaxes[labelFontSize=\scriptstyle,xAxis=true,yAxis=true,labels=x,Dx=200.,Dy=5,ticksize=-2pt 0,subticks=2]{->}(0,0)(-260.3960407283936,-13.15313488549706)(2061.9493903703883,34.12528078617641)
\psplot[linewidth=1.2pt,plotpoints=200]{0}{1800}{(0.00077*x^(3.0)-0.693*x^(2.0)+396.0*x+317900.0)/100000.0}
\psplot[linewidth=1.2pt,plotpoints=200]{0}{1800}{(1320.0*x)/100000.0}
\begin{scriptsize}
\rput[tl](80.92939221574203,5.5607334126604195){K}
\rput[tl](582.1195382202467,10.000601385299875){E}
\rput[tl](-220.06549445931726,5.6){500.000}
\rput[tl](-265.06549445931726,10.6){1.000.000}
\rput[tl](-265.06549445931726,15.6){1.500.000}
\rput[tl](-265.06549445931726,20.6){2.000.000}
\rput[tl](-265.06549445931726,25.6){2.500.000}
\rput[tl](-265.06549445931726,30.6){3.000.000}
\rput[tl](-230.06549445931726,-5.6){-500.000}
\rput[tl](-275.06549445931726,-10.6){-1.000.000}
\psdots[dotsize=4pt 0,dotstyle=*](299.22,3.9497)
\rput[bl](270.8540791227122,4.804205086206228){$A$}
\psdots[dotsize=4pt 0,dotstyle=*](1512.82,19.9692)
\rput[bl](1484.2618010283552,20.905714745369643){$B$}
\rput[tl](1800.175329649479,1.4012993992141887){$x$ in Stk.}
\rput[tl](50.175329649479,33){$y$ in \euro}
\end{scriptsize}
\end{pspicture*}
				
\subsection{Aufgabenstellung:}
\begin{enumerate}
	\item Zeichne in der obigen Abbildung den Graphen der Gewinnfunktion $G$ ein!
	
	Eine St�ckpreis�nderung wurde vorgenommen und hat bewirkt, dass der Break-even-Point bei einer geringeren St�ckzahl erreicht wird. Gib an, wie der St�ckpreis ver�ndert wurde und welchen Einfluss diese Ver�nderung auf die Lage der Nullstellen der Gewinnfunktion $G$ und den Gewinnbereich hat!
	
\item Erstelle die Gleichung der Gewinnfunktion $G$!

Berechne diejenige St�ckzahl, bei der der Gewinn maximal wird!

\item In der nachstehenden Grafik wurde die Erl�sfunktion so abge�ndert, dass die Graphen der Kostenfunktion $K$ und der Erl�sfunktion $E_{neu}$ einander im Punkt $T$ ber�hren. Bestimme die Gleichung der Erl�sfunktion $E_{neu}$!
\leer

\newrgbcolor{qqwuqq}{0. 0.39215686274509803 0.}
\psset{xunit=0.005cm,yunit=0.2cm,algebraic=true,dimen=middle,dotstyle=o,dotsize=5pt 0,linewidth=0.8pt,arrowsize=3pt 2,arrowinset=0.25}
\begin{pspicture*}(-270.3960407283936,-3.15313488549706)(2061.9493903703883,34.12528078617641)
\multips(0,-10)(0,5.0){9}{\psline[linestyle=dashed,linecap=1,dash=1.5pt 1.5pt,linewidth=0.4pt,linecolor=lightgray]{c-c}(-260.3960407283936,0)(2061.9493903703883,0)}
\multips(-200,0)(200.0,0){13}{\psline[linestyle=dashed,linecap=1,dash=1.5pt 1.5pt,linewidth=0.4pt,linecolor=lightgray]{c-c}(0,-13.15313488549706)(0,34.12528078617641)}
\psaxes[labelFontSize=\scriptstyle,xAxis=true,yAxis=true,labels=x,Dx=200.,Dy=5,ticksize=-2pt 0,subticks=2]{->}(0,0)(-260.3960407283936,-13.15313488549706)(2061.9493903703883,34.12528078617641)
\psplot[linewidth=1.2pt,plotpoints=200]{0}{1800}{(0.00077*x^(3.0)-0.693*x^(2.0)+396.0*x+317900.0)/100000.0}
\psplot{0}{3041.788116004077}{(--0.08711120000000072--0.00720324*x)/1.}
\psdots[dotsize=4pt 0,dotstyle=*](783.7943961192058,5.732970345901708)
\begin{scriptsize}
\rput[tl](80.92939221574203,5.5607334126604195){$K$}
\rput[tl](1450.1195382202467,10.000601385299875){$E_{neu}$}
\rput[tl](-220.06549445931726,5.6){500.000}
\rput[tl](-265.06549445931726,10.6){1.000.000}
\rput[tl](-265.06549445931726,15.6){1.500.000}
\rput[tl](-265.06549445931726,20.6){2.000.000}
\rput[tl](-265.06549445931726,25.6){2.500.000}
\rput[tl](-265.06549445931726,30.6){3.000.000}
\rput[tl](1800.175329649479,1.4012993992141887){$x$ in Stk.}
\rput[tl](50.175329649479,33){$y$ in \euro}
\rput[tl](700.1195382202467,4.400601385299875){$T=(x_T/y_T)$}
\end{scriptsize}
\end{pspicture*}
				\end{enumerate}\leer
				
				Interpretiere die Koordinaten des Punktes $T$ im gegebenen Kontext und erkl�re, welche Auswirkung die �nderung der Erl�sfunktion auf den Gewinnbereich hat!
				
\antwort{\subsection{L�sungserwartung:}
\begin{enumerate}
	\item Graph der Gewinnfunktion:
	
	\newrgbcolor{qqwuqq}{0. 0.39215686274509803 0.}
\psset{xunit=0.005cm,yunit=0.2cm,algebraic=true,dimen=middle,dotstyle=o,dotsize=5pt 0,linewidth=0.8pt,arrowsize=3pt 2,arrowinset=0.25}
\begin{pspicture*}(-270.3960407283936,-13.15313488549706)(2061.9493903703883,34.12528078617641)
\multips(0,-10)(0,5.0){9}{\psline[linestyle=dashed,linecap=1,dash=1.5pt 1.5pt,linewidth=0.4pt,linecolor=lightgray]{c-c}(-260.3960407283936,0)(2061.9493903703883,0)}
\multips(-200,0)(200.0,0){13}{\psline[linestyle=dashed,linecap=1,dash=1.5pt 1.5pt,linewidth=0.4pt,linecolor=lightgray]{c-c}(0,-13.15313488549706)(0,34.12528078617641)}
\psaxes[labelFontSize=\scriptstyle,xAxis=true,yAxis=true,labels=x,Dx=200.,Dy=5,ticksize=-2pt 0,subticks=2]{->}(0,0)(-260.3960407283936,-13.15313488549706)(2061.9493903703883,34.12528078617641)
\psplot[linewidth=1.2pt,linecolor=qqwuqq,plotpoints=200]{-499.3960407283936}{1800}{0.0}
\psplot[linewidth=1.2pt,plotpoints=200]{0}{1800}{(0.00077*x^(3.0)-0.693*x^(2.0)+396.0*x+317900.0)/100000.0}
\psplot[linewidth=1.2pt,plotpoints=200]{0}{1800}{(1320.0*x)/100000.0}
\psline[linewidth=1.2pt,linestyle=dashed,dash=5pt 1.5pt](1000.,-20.613229035746011)(1000.,34.66518663592747)
\psline[linewidth=1.6pt](1000.,13.2)(1000.,7.909)
\psline(1000.,13.2)(1000.,7.909)
\psplot[linewidth=1.2pt,linestyle=dashed,dash=5pt 1.5pt,plotpoints=200]{0}{1800}{(-7.7E-4*x^(3.0)+0.693*x^(2.0)+924.0*x-317900.0)/100000.0}
\psline[linewidth=1.6pt](1000.,5.291)(1000.,0.)
\begin{scriptsize}
\rput[tl](1046.3798839928404,7.146242854811816){MAX}
\rput[tl](80.92939221574203,5.5607334126604195){K}
\rput[tl](582.1195382202467,10.000601385299875){E}
\rput[tl](1430.175329649479,2.9012993992141887){G}
\rput[tl](-220.06549445931726,5.6){500.000}
\rput[tl](-265.06549445931726,10.6){1.000.000}
\rput[tl](-265.06549445931726,15.6){1.500.000}
\rput[tl](-265.06549445931726,20.6){2.000.000}
\rput[tl](-265.06549445931726,25.6){2.500.000}
\rput[tl](-265.06549445931726,30.6){3.000.000}
\rput[tl](-230.06549445931726,-5.6){-500.000}
\rput[tl](-275.06549445931726,-10.6){-1.000.000}
\psdots[dotsize=4pt 0,dotstyle=*](299.22,3.9497)
\rput[bl](270.8540791227122,4.804205086206228){$A$}
\psdots[dotsize=4pt 0,dotstyle=*](1512.82,19.9692)
\rput[bl](1484.2618010283552,20.905714745369643){$B$}
\rput[tl](1800.175329649479,1.4012993992141887){$x$ in Stk.}
\rput[tl](50.175329649479,33){$y$ in \euro}
\end{scriptsize}
\end{pspicture*}
\leer

Der St�ckpreis muss erh�ht werden. Die Nullstellen liegen weiter auseinander, das hei�t, der Gewinnbereich wird gr��er.

\item Gewinnfunktion:

$G(x)=E(x)-K(x)$

$G(x)=1320x-(0,00077x�-0,693x�+396x+317900)$

$G(x)=-0,00077x�+0,693x�+924x-317900$

Bedingung f�r maximalen Gewinn:

$G'(x)=0 \rightarrow G'(x)=-0,00231x�+1386x+924$

$-0,00231x�+1,386x+924=0 \Rightarrow x_{1,2}=\dfrac{-1,386\pm\sqrt{1,386�+4\cdot 0,00231\cdot 924}}{-0,00462} \rightarrow (x_1=-400); x_2=1000$

Der maximale Gewinn wird bei einer St�ckzahl von 1000 erzielt.

\item Die Gleichung der Erl�sfunktion $E_{neu}$ lautet:

$E_{neu}(x)=\frac{y_T}{x_T}\cdot x$

Nur bei der Produktionsmenge von $x_T$ St�ck wird genau kostendeckend produziert. Kosten und Erl�s betragen je \EUR{$y_T$}. Bei dieser Produktionsmenge ist es nicht m�glich, mit Gewinn zu produzieren.
\end{enumerate}}
\end{langesbeispiel}