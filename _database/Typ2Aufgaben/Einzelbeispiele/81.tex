\section{81 - MAT - AG 2.1, AN 3.3, AG 2.3, FA 1.4 - Schwimmzonen - Matura NT 1 16/17}


\begin{langesbeispiel} \item[6] %PUNKTE DES BEISPIELS

Wegen der großen Anzahl an Motorbooten, Jetskis etc. hat man an einigen Stränden spezielle Schwimmzonen eingerichtet.

Alle in dieser Aufgabe beschriebenen Schwimmzonen sind mit je zwei Bojen und einem 180 Meter langem Seil an einem nahezu geraden Strand angelegt.

\begin{center}
	\resizebox{0.5\linewidth}{!}{
\psset{xunit=1.0cm,yunit=1.0cm,algebraic=true,dimen=middle,dotstyle=o,dotsize=5pt 0,linewidth=1.6pt,arrowsize=3pt 2,arrowinset=0.25}
\begin{pspicture*}(2.176363636363639,0.1)(9.736363636363635,5.28)
\psplot[linewidth=2.pt]{2.176363636363639}{9.736363636363635}{(--6.-0.*x)/6.}
\psline[linewidth=2.pt,linestyle=dashed,dash=1pt 2pt 5pt 2pt ](3.,1.)(5.,4.)
\psline[linewidth=2.pt,linestyle=dashed,dash=1pt 2pt 5pt 2pt ](5.,4.)(7.,4.)
\psline[linewidth=2.pt,linestyle=dashed,dash=1pt 2pt 5pt 2pt ](7.,4.)(9.,1.)
\rput[tl](4.7,2.7){Schwimmzone}
\rput[tl](6.616363636363636,4.6){Boje 2}
\rput[tl](4.536363636363638,4.6){Boje 1}
\rput[tl](5.516363636363637,0.68){Strand}
\begin{scriptsize}
\psdots[dotstyle=*](5.,4.)
\psdots[dotstyle=*](7.,4.)
\end{scriptsize}
\end{pspicture*}}
\end{center}

\subsection{Aufgabenstellung:}
\begin{enumerate}
	\item Gegeben ist eine rechteckige Schwimmzone ($x$ in Metern).
	\begin{center}
		\resizebox{0.5\linewidth}{!}{\psset{xunit=1.0cm,yunit=1.0cm,algebraic=true,dimen=middle,dotstyle=o,dotsize=5pt 0,linewidth=1.6pt,arrowsize=3pt 2,arrowinset=0.25}
\begin{pspicture*}(2.176363636363639,0.1)(10.776363636363628,5.28)
\psplot[linewidth=2.pt]{2.176363636363639}{10.776363636363628}{(--5.-0.*x)/5.}
\psline[linewidth=2.pt,linestyle=dashed,dash=1pt 2pt 5pt 2pt ](4.,1.)(4.,4.)
\psline[linewidth=2.pt,linestyle=dashed,dash=1pt 2pt 5pt 2pt ](4.,4.)(9.,4.)
\psline[linewidth=2.pt,linestyle=dashed,dash=1pt 2pt 5pt 2pt ](9.,4.)(9.,1.)
\rput[tl](8.676363636363632,4.62){Boje 2}
\rput[tl](3.456363636363638,4.62){Boje 1}
\rput[tl](6.276363636363635,0.76){Strand}
\rput[tl](9.2,2.68){x}
\rput[tl](3.6363636363636376,2.68){x}
\begin{scriptsize}
\psdots[dotstyle=*](4.,4.)
\psdots[dotstyle=*](9.,4.)
\end{scriptsize}
\end{pspicture*}}
	\end{center}
	
	\fbox{A} Zeige, dass für den Flächeninhalt $A(x)$ einer derartigen Schwimmzone die Gleichung $A(x)=180\cdot x-2\cdot x^2$ gilt!
	
	Ermittle die Länge, die Breite und den Flächeninhalt derjenigen Schwimmzone, die den größten Flächeninhalt aufweist!
	
	Länge = \antwort[\rule{3cm}{0.3pt}]{90}\,m
	
	Breite = \antwort[\rule{3cm}{0.3pt}]{45}\,m
	
	Flächeninhalt = \antwort[\rule{3cm}{0.3pt}]{4\,050}\,m$^2$
	
	\item Gegeben sind trapezförmige Schwimmzonen ($x$ und $h$ in Metern).
	
	\begin{center}
		\resizebox{0.5\linewidth}{!}{\psset{xunit=1.0cm,yunit=1.0cm,algebraic=true,dimen=middle,dotstyle=o,dotsize=5pt 0,linewidth=1.6pt,arrowsize=3pt 2,arrowinset=0.25}
\begin{pspicture*}(1.9043636363636391,-0.348)(10.220363636363631,5.35)
\psplot[linewidth=2.pt]{1.9043636363636391}{10.220363636363631}{(--6.-0.*x)/6.}
\psline[linewidth=2.pt,linestyle=dashed,dash=1pt 3pt 5pt 3pt ](3.,1.)(5.,4.)
\psline[linewidth=2.pt,linestyle=dashed,dash=1pt 3pt 5pt 3pt ](5.,4.)(7.,4.)
\psline[linewidth=2.pt,linestyle=dashed,dash=1pt 3pt 5pt 3pt ](7.,4.)(9.,1.)
\rput[tl](6.612363636363636,4.5){Boje 2}
\rput[tl](4.544363636363637,4.5){Boje 1}
\rput[tl](5.512363636363636,0.708){Strand}
\psline[linewidth=2.pt,linestyle=dashed,dash=3pt 3pt](5.,1.)(5.,4.)
\psline[linewidth=2.pt,linestyle=dashed,dash=3pt 3pt](7.,1.)(7.,4.)
\parametricplot{1.5707963267948966}{3.141592653589793}{0.66*cos(t)+7.|0.66*sin(t)+1.}
\psellipse*[linewidth=2.pt,fillcolor=black,fillstyle=solid,opacity=1](6.725476190833458,1.2745238091665416)(0.05,0.05)
\parametricplot{1.5707963267948966}{3.141592653589793}{0.66*cos(t)+5.|0.66*sin(t)+1.}
\psellipse*[linewidth=2.pt,fillcolor=black,fillstyle=solid,opacity=1](4.725476190833458,1.2745238091665416)(0.05,0.05)
\rput[tl](8.086363636363634,3.04){x}
\rput[tl](3.730363636363638,2.996){x}
\rput[tl](6.6,2.93){h}
\rput[tl](5.3,2.952){h}
\begin{scriptsize}
\psdots[dotstyle=*](5.,4.)
\psdots[dotstyle=*](7.,4.)
\end{scriptsize}
\end{pspicture*}}
	\end{center}

Um den Flächeninhalt einer solchen trapezförmigen Schwimmzone berechnen zu können, kann die Formel $A(x,h)=h\cdot(180-2\cdot x+\sqrt{x^2-h^2})$ herangezogen werden.

Gib alle Werte an, die $x$ annehmen darf, wenn $h$ 40\,m lang ist!

Gib alle Werte an, die $h$ annehmen darf, wenn $x$ 50\,m lang ist!
	\item Gegeben sind trapezförmige Schwimmzonen, bei denen alle drei Seilabschnitte gleich lang sind ($x$ und $h$ in Metern).
	
	\begin{center}
		\resizebox{0.5\linewidth}{!}{\psset{xunit=1.0cm,yunit=1.0cm,algebraic=true,dimen=middle,dotstyle=o,dotsize=5pt 0,linewidth=1.6pt,arrowsize=3pt 2,arrowinset=0.25}
\begin{pspicture*}(1.9043636363636391,-0.348)(10.220363636363636,5.35)
\psplot[linewidth=2.pt]{1.9043636363636391}{10.220363636363636}{(--6.-0.*x)/6.}
\psline[linewidth=2.pt,linestyle=dashed,dash=1pt 3pt 5pt 3pt ](3.,1.)(5.,4.)
\psline[linewidth=2.pt,linestyle=dashed,dash=1pt 3pt 5pt 3pt ](5.,4.)(7.,4.)
\psline[linewidth=2.pt,linestyle=dashed,dash=1pt 3pt 5pt 3pt ](7.,4.)(9.,1.)
\rput[tl](6.612363636363638,4.734){Boje 2}
\rput[tl](4.544363636363639,4.734){Boje 1}
\rput[tl](5.512363636363639,0.708){Strand}
\psline[linewidth=2.pt,linestyle=dashed,dash=3pt 3pt](5.,1.)(5.,4.)
\psline[linewidth=2.pt,linestyle=dashed,dash=3pt 3pt](7.,1.)(7.,4.)
\parametricplot{1.5707963267948966}{3.141592653589793}{0.66*cos(t)+7.|0.66*sin(t)+1.}
\psellipse*[linewidth=2.pt,fillcolor=black,fillstyle=solid,opacity=1](6.725476190833458,1.2745238091665418)(0.05,0.05)
\parametricplot{1.5707963267948966}{3.141592653589793}{0.66*cos(t)+5.|0.66*sin(t)+1.}
\psellipse*[linewidth=2.pt,fillcolor=black,fillstyle=solid,opacity=1](4.725476190833458,1.2745238091665418)(0.05,0.05)
\rput[tl](8.086363636363638,3.04){x}
\rput[tl](3.730363636363639,2.996){x}
\rput[tl](6.7003636363636385,2.93){h}
\rput[tl](5.292363636363639,2.952){h}
\parametricplot{-2.1587989303424644}{-1.570796326794897}{1.32*cos(t)+5.|1.32*sin(t)+4.}
\begin{scriptsize}
\psdots[dotstyle=*](5.,4.)
\psdots[dotstyle=*](7.,4.)
\rput[bl](4.632363636363639,2.996){$\alpha$}
\end{scriptsize}
\end{pspicture*}}
	\end{center}
	
	Der Flächeninhalt $A(\alpha)$ einer derartigen Schwimmzone kann in Abhängigkeit vom Winkel $\alpha$ beschrieben werden ($A(\alpha)$ in m$^2$, $\alpha$ in Grad).
	
	Stelle eine Formel auf, mit deren Hilfe der Flächeninhalt einer solchen Schwimmzone in Abhängigkeit vom Winkel $\alpha$ berechnet werden kann!\leer
	
	$A(\alpha)=$ \antwort[\rule{3cm}{0.3pt}]{$3\,600\cdot\cos(\alpha)+1\,800\cdot\sin(2\cdot\alpha)$ oder $3\,600\cdot\cos(\alpha)\cdot\sin(\alpha)$}
	
	In der nachstehenden Abbildung sind die Werte der Flächeninhalte für den jeweiligen Winkel $\alpha$ dargestellt.
	
	\begin{center}
		\resizebox{0.8\linewidth}{!}{\psset{xunit=0.1cm,yunit=0.001cm,algebraic=true,dimen=middle,dotstyle=o,dotsize=5pt 0,linewidth=1.6pt,arrowsize=3pt 2,arrowinset=0.25}
\begin{pspicture*}(-14,-977.903157117887)(94.39374356174048,5716.259447779984)
\multips(0,0)(0,1000.0){7}{\psline[linestyle=dashed,linecap=1,dash=1.5pt 1.5pt,linewidth=0.4pt,linecolor=lightgray]{c-c}(0,0)(94.39374356174048,0)}
\multips(0,0)(10.0,0){11}{\psline[linestyle=dashed,linecap=1,dash=1.5pt 1.5pt,linewidth=0.4pt,linecolor=lightgray]{c-c}(0,0)(0,5716.259447779984)}
\psaxes[labelFontSize=\scriptstyle,xAxis=true,yAxis=true,Dx=10.,Dy=1000.,ticksize=-2pt 0,subticks=2]{->}(0,0)(0.,0.)(94.39374356174048,5716.259447779984)
\psplot[linewidth=2.pt,plotpoints=200]{0}{90}{-1.29*(x-30.0)^(2.0)+4700.0}
\rput[tl](1.3504923728944642,5732.707021747792){Flächeninhalt in m$^2$}
\rput[tl](60,-500){Winkel $\alpha$ in Grad}
\end{pspicture*}}
	\end{center}
	
	Es soll eine Schwimmzone mit größtmöglichem Flächeninhalt angelegt werden. Berechne untr Zuhilfenahme der obigen Abbildung diejenige Länge, die sich dabei für den Strandabschnitt, von dem aus man die Schwimmzone betreten kann, ergibt!
	
	\end{enumerate}

\antwort{
\begin{enumerate}
	\item \subsection{Lösungserwartung:} 

$A(x)=x\cdot(180-2\cdot x)=180\cdot x-2\cdot x^2$\leer

Mögliche Vorgehensweise:

$A'(x)=180-4\cdot x=0\Rightarrow x=45$

$(A''(45)=-4<0 \Rightarrow x=45$

Aus $A''(x)<0$ für alle $x$ folgt, dass $A$ rechtsgekrümmt ist und das lokale Maximum daher ein globales Maximum ist.)

Länge, Breite und Flächeninhalt siehe oben

\subsection{Lösungsschlüssel:}
\begin{itemize}
	\item Ein Ausgleichspunkt für einen korrekten Nachweis. Andere korrekte Nachweise sind ebenfalls als richtig zu werten.
	\item Ein Punkt für die Angabe aller drei richtigen Werte. Der Nachweis für das Vorliegen eines Maximums ist nicht erforderlich.
\end{itemize}

	\item \subsection{Lösungserwartung:}

alle Werte, die $x$ bei $h=40$\,m annehmen darf: $x\in[40;90]$

alle Werte, die $h$ bei $x=50$\,m annehmen darf: $h\in[0;50]$

\subsection{Lösungsschlüssel:}
\begin{itemize}
	\item Ein Punkt für die Angabe eines korrekten Intervalls für $x$.
	
	Andere Schreibweisen des Intervalls (offen oder halboffen) sowie korrekte formale oder verbale Beschreibung sind ebenfalls als richtig zu werten.
	
	\item Ein Punkt für die Angabe eines korrekten Intervalls für $h$.
	
	Andere Schreibweisen des Intervalls (offen oder halboffen) sowie korrekte formale oder verbale Beschreibungen sind ebenfalls als richtig zu werten.
\end{itemize}

\item \subsection{Lösungserwartung:}

Lösung $A(\alpha)$ siehe oben!\leer

Mögliche Vorgehensweise:
größtmöglicher Flächeninhalt: bei $\alpha=30^\circ$

$30+60+30=120$

Der Strandabschnitt ist dabei 120\, m lang.

\subsection{Lösungsschlüssel:}
\begin{itemize}
	\item Ein Punkt für eine korrekte Formel. Andere korrekte Formeln sind ebenfalls als richtig zu werten.
	\item Ein Punkt für die richtige Lösung, wobei die Einheit "`m"' nicht angeführt sein muss. Toleranzintervall: $[115\,\text{m};125\,\text{m}]$
\end{itemize}
\end{enumerate}}
		\end{langesbeispiel}