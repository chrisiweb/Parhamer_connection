\section{16 - MAT - FA 1.6, FA 2.3, FA 1.5, FA 2.2, AN 3.3, AN 1.3 - Produktionskosten - BIFIE Aufgabensammlung}

\begin{langesbeispiel} \item[0] %PUNKTE DES BEISPIELS
				Die Produktionskosten eines Betriebes setzen sich aus Fixkosten und variablen Kosten zusammen und können durch eine Kostenfunktion beschrieben werden. Fixkosten fallen auf jeden Fall an und sind unabhängig von der produzierten Menge. Variable Kosten hingegen nehmen mit steigender Produktionsmenge zu.
				
Die Kostenkehre ist jene Produktionsmenge, ab der die variablen Kosten immer stärker steigen, in diesem Fall spricht man von einem progressiven Kostenverlauf. Vor der Kostenkehre ist der Kostenverlauf degressiv, das heißt, die Kosten steigen bei zunehmender Produktionsmenge immer schwächer.
 
Der Verkaufserlös ist das Produkt aus der verkauften Stückzahl und dem Verkaufspreis pro Stück.

Die untenstehende Abbildung zeigt die Graphen der Kostenfunktion $K$ und der Erlösfunktion $E$ des Betriebes, wobei $x$ die Anzahl der produzierten und verkauften Mengeneinheiten (ME) pro Tag ist. 1 ME entspricht einer Verpackungseinheit von 100 Stück. Pro Tag können höchstens 110 ME produziert werden.

Der Gewinn ist die Differenz aus Erlös und Produktionskosten.\leer

\psset{xunit=0.1cm,yunit=0.001cm,algebraic=true,dimen=middle,dotstyle=o,dotsize=5pt 0,linewidth=0.8pt,arrowsize=3pt 2,arrowinset=0.25}
\begin{pspicture*}(-8.279911803646898,-615.7327314023055)(116.1512155298207,8875.51813193647)
\multips(0,0)(0,1000.0){10}{\psline[linestyle=dashed,linecap=1,dash=1.5pt 1.5pt,linewidth=0.4pt,linecolor=black!60]{c-c}(0,0)(116.1512155298207,0)}
\multips(0,0)(10.0,0){13}{\psline[linestyle=dashed,linecap=1,dash=1.5pt 1.5pt,linewidth=0.4pt,linecolor=black!60]{c-c}(0,0)(0,8875.51813193647)}
\psaxes[labelFontSize=\scriptstyle,xAxis=true,yAxis=true,Dx=10.,Dy=1000.,ticksize=-2pt 0,subticks=2]{->}(0,0)(-7.879911803646898,-615.7327314023055)(116.1512155298207,8875.51813193647)
\psplot{0.}{110}{(-0.--3000.*x)/50.}
\psplot[linewidth=1.2pt,plotpoints=200]{0}{116.1512155298207}{-3.30988455988456E-5*x^(4.0)+0.01733225108225108*x^(3.0)-1.3169282106782108*x^(2.0)+56.46915584415584*x+500.0}
\antwort{
\psplot{-7.879911803646898}{116.1512155298207}{(-56844.17013105299--3000.*x)/50.}
\psline(62.99349367592369,0.)(62.99349367592369,2642.726217934361)
\psline[linestyle=dashed,dash=4pt 2pt](62.99349367592369,2642.726217934361)(62.993493675923695,3779.609620555421)
\psdots[dotsize=3pt 0,dotstyle=*](62.99349367592369,2642.726217934361)
\psdots[dotsize=3pt 0,dotstyle=*](62.993493675923695,3779.609620555421)
\psdots[dotsize=3pt 0,dotstyle=*](62.99349367592369,0.)}
\begin{scriptsize}
\rput[tl](99.67365030436004,400.8332437014949){x in ME}
\rput[tl](1.7070352366211292,8458.736656899637){y in Euro}
\rput[tl](67.61729613846381,4631.924931561448){E}
\rput[tl](94.28099259420927,7000.001494270723){K}
\antwort{\rput[tl](94.28099259420927,4400.001494270723){t}}
\end{scriptsize}
\end{pspicture*}

				
\subsection{Aufgabenstellung:}
\begin{enumerate}
	\item Ermittle anhand der obigen Abbildung den Gewinnbereich, das sind jene Stückzahlen (1\,ME = 100 Stück), für die der Betrieb Gewinn erzielt!
	
	Beschreibe, wie sich eine Senkung des Verkaufspreises auf den Verlauf des Graphen der Erlösfunktion $E$ auswirkt und wie sich dadurch der Gewinnbereich verändert!
	
	\item Bestimme anhand der Abbildung die Fixkosten und den Verkaufspreis pro ME möglichst genau!
	
	\item Welche der nachstehenden Aussagen treffen für die in der Grafik abgebildeten Produktionskosten zu? Kreuze die beiden zutreffenden Aussagen an!\leer
	
	\multiplechoice[5]{  %Anzahl der Antwortmoeglichkeiten, Standard: 5
					L1={Bei degressivem Kostenverlauf gilt: $K'(x)<0$.},   %1. Antwortmoeglichkeit 
					L2={Bei progressivem Kostenverlauf gilt: $K''(x)>0$.},   %2. Antwortmoeglichkeit
					L3={Bei der Kostenkehre gilt: $K'(x)=0$.},   %3. Antwortmoeglichkeit
					L4={Für alle $x$ aus dem Definitionsbereich [0 ME; 110 ME] gilt: \mbox{$K'(x)>0$}.},   %4. Antwortmoeglichkeit
					L5={Es gilt: K'(50)>K'(90).},	 %5. Antwortmoeglichkeit
					L6={},	 %6. Antwortmoeglichkeit
					L7={},	 %7. Antwortmoeglichkeit
					L8={},	 %8. Antwortmoeglichkeit
					L9={},	 %9. Antwortmoeglichkeit
					%% LOESUNG: %%
					A1=2,  % 1. Antwort
					A2=4,	 % 2. Antwort
					A3=0,  % 3. Antwort
					A4=0,  % 4. Antwort
					A5=0,  % 5. Antwort
					}\leer
					
					Erkläre ausführlich, was die 1. und die 2. Ableitung der Kostenfunktion an einer bestimmten Stelle über den Verlauf des Graphen von $K$ an dieser Stelle aussagen!
					
					\item Deute die Beziehung $K'(x)=E'(x)$ geometrisch und ermittle anhand der nachstehenden Abbildung jene Produktionsmenge $x_1$ fpr die dies zutrifft! Begründe, warum der erzielte Gewinn bei dieser Produktionsmenge $x_1$ am größten ist!\leer
					
					\psset{xunit=0.1cm,yunit=0.001cm,algebraic=true,dimen=middle,dotstyle=o,dotsize=5pt 0,linewidth=0.8pt,arrowsize=3pt 2,arrowinset=0.25}
\begin{pspicture*}(-8.279911803646898,-615.7327314023055)(116.1512155298207,8875.51813193647)
\multips(0,0)(0,1000.0){10}{\psline[linestyle=dashed,linecap=1,dash=1.5pt 1.5pt,linewidth=0.4pt,linecolor=lightgray]{c-c}(0,0)(116.1512155298207,0)}
\multips(0,0)(10.0,0){13}{\psline[linestyle=dashed,linecap=1,dash=1.5pt 1.5pt,linewidth=0.4pt,linecolor=lightgray]{c-c}(0,0)(0,8875.51813193647)}
\psaxes[labelFontSize=\scriptstyle,xAxis=true,yAxis=true,Dx=10.,Dy=1000.,ticksize=-2pt 0,subticks=2]{->}(0,0)(-7.879911803646898,-615.7327314023055)(116.1512155298207,8875.51813193647)
\psplot{0.}{110}{(-0.--3000.*x)/50.}
\psplot[linewidth=1.2pt,plotpoints=200]{0}{116.1512155298207}{-3.30988455988456E-5*x^(4.0)+0.01733225108225108*x^(3.0)-1.3169282106782108*x^(2.0)+56.46915584415584*x+500.0}
\antwort{
\psplot{-7.879911803646898}{116.1512155298207}{(-56844.17013105299--3000.*x)/50.}
\psline(62.99349367592369,0.)(62.99349367592369,2642.726217934361)
\psline[linestyle=dashed,dash=4pt 2pt](62.99349367592369,2642.726217934361)(62.993493675923695,3779.609620555421)
\psdots[dotsize=3pt 0,dotstyle=*](62.99349367592369,2642.726217934361)
\psdots[dotsize=3pt 0,dotstyle=*](62.993493675923695,3779.609620555421)
\psdots[dotsize=3pt 0,dotstyle=*](62.99349367592369,0.)}
\begin{scriptsize}
\rput[tl](99.67365030436004,400.8332437014949){x in ME}
\rput[tl](1.7070352366211292,8458.736656899637){y in Euro}
\rput[tl](67.61729613846381,4631.924931561448){E}
\rput[tl](94.28099259420927,7000.001494270723){K}
\antwort{\rput[tl](94.28099259420927,4400.001494270723){t}}
\end{scriptsize}
\end{pspicture*}	
					\end{enumerate}\leer
				
\antwort{\subsection{Lösungserwartung:}
\begin{enumerate}
	\item Die Antwort ist als richtig zu werten, wenn \textbf{beide} Grenzen des Grenzbereichs richtig angegeben sind, z.B.: \textit{Bei einer Produktion von 2\,000 bis 9\,000 Stück wird Gewinn erzielt} (Toleranz bei Gewinngrenzen: $\pm 100$ Stück).
	
	Weiters muss eine richtige Interpretation angeführt sein, wie sich eine Senkung des Verkaufspreises auf den Gewinnbereich auswirkt, z.B.: \textit{Bei einer Senkung des Verkaufspreises verläuft der Graph von $E$ flacher, wodurch der Gewinnbereich kleiner ("`schmäler"') wird.}
	
	Als richtig zu werten ist auch die Antwort, dass bei einer starken Senkung des Verkaufspreises bei allen Produktionsmengen Verlust erzielt wird.
	
	\item \textbf{Fixkosten:} 500 Euro (Toleranz: $\pm 100$ Euro)
	
	\textbf{Verkaufspreis} pro ME: $\frac{3000}{50}=60$ Euro (Toleranz: $\pm 5$ Euro)
	
	Falls der Verkaufspreis durch ein "`zu kleines"' Steigungsdreieck sehr ungenau abgelesen wird (z.B.: 50 Euro), so ist das Ergebnis als falsch zu werten.
	
	\item Lösung Multiple Choice siehe oben
	
	Zudem muss eine Erklärung angegeben sein, z.B.:
	
	\textit{$K'(x)$ beschreibt die Steigung der Kostenfunktion (oder: Steigung der Tangente) an der Stelle $x$ (bei Produktion von $x$ ME).}
	
	\textit{$K''(x)$ beschreibt die Änderung der Steigung, also das Krümmungsverhalten der Kostenfunktion an der Stelle $x$.}
	
	\textit{Im degressiven Bereich ist der Graph von $K$ rechtsgekrümmt und im progressiven Bereich ist der Graph von $K$ linksgekrümmt.}
	
	Auch folgende bzw. alle anderen inhaltlich richtigen Formulierungen sind als richtig zu werten:
	
	\textit{$K'$ beschreibt das Monotonieverhalten von $K$, d.h. falls $K'(x)>0$ ist, steigt $K$ an der Stelle $x$.}
	
	\textit{$K''$ beschreibt das Monotonieverhalten von $K'$, d.h. falls $K''(x)>0$ ist, steigt $K'$ an der Stelle $x$ (d.h., die Kostensteigerung nimmt zu).}
	
	Anmerkung: Aus der Antwort muss jedenfalls ersichtlich sein, welche geometrische Bedetuung $K'$ und $K''$ besitzen, der Begriff \textit{Monotonieverhalten} alleine ist nicht ausreichend.
	
	\item Die Antwort ist als richtig zu werten, wenn die richtige geometrische Deutung angegeben ist und $x_1$ bestimmt ist (falls $x_1$ nur eingezeichnet ist, der Wert aber nicht angegeben ist, so ist dies auch als richtig zu werten), z. B.: Geometrisch bedeutet dies, dass der Graph von $K$ und der Graph von $E$ an dieser Stelle die gleiche Steigung besitzen.
	
Oder: Die Tangente $t$ an den Graphen von $K$ verläuft parallel zum Graphen von $E$.
Dies ist bei ca. 63 ME der Fall (Toleranz: $\pm 3$ ME).

Zudem muss die Interpretation angegeben sein, dass an der gesuchten Stelle $G'(x)=0$ gilt und somit $G(x_1)$ der maximale Gewinn ist, z. B.: \textit{Wegen der Beziehung $G(x)=E(x)-K(x)$ gilt: $G'(x)=E'(x)-K'(x)$.}

\textit{Somit gilt: $G'(x_1)=E'(x_1)-K'(x_1)=0$ und $G(x_1)$ ist daher der maximale Gewinn.}

(Anmerkung: Der Nachweis des Maximums (Monotoniewechsel von $G$ an der Stelle $x_1$ ist nicht erforderlich.)

Auch die geometrische Begründung, dass der vertikale Abstand zwischen Erlös- und Kostenkurve an der Stelle $x_1$ am größten ist, ist als richtig zu werten, falls dieser Abstand (strichlierte Linie) richtig eingezeichnet ist.

Lösung der Graphik: siehe oben!
		\end{enumerate}}
\end{langesbeispiel}