\section{28 - MAT - AG 2.1, FA 2.1, FA 4.3, AN 1.1, AN 1.3, AN 3.3, AN 4.3 - Saturn-V-Rakete - BIFIE Aufgabensammlung}

\begin{langesbeispiel} \item[0] %PUNKTE DES BEISPIELS
				Eine Mehrstufenrakete besteht aus mehreren, oft übereinander montierten "`Raketenstufen"'. Jede Raketenstufe ist eine separate Rakete mit Treibstoffvorrat und Raketentriebwerk. Leere Treibstofftanks und nicht mehr benötigte Triebwerke werden abgeworfen. Auf diese Weise werden höhere Geschwindigkeiten und somit höhere Umlaufbahnen als mit einstufigen Raketen erreicht.\\
Die Familie der Saturn-Raketen gehört zu den leistungsstärksten Trägersystemen der Raumfahrt, die jemals gebaut wurden. Sie wurden im Rahmen des Apollo-Programms für die US-
amerikanische Raumfahrtbehörde NASA entwickelt. Die Saturn V ist die größte jemals gebaute
Rakete. Mithilfe dieser dreistufigen Rakete konnten in den Jahren 1969 bis 1972 insgesamt
12 Personen auf den Mond gebracht werden. 1973 beförderte eine Saturn V die US-amerikanische Raumstation Skylab in eine Erdumlaufbahn in 435 km Höhe.\\
Eine Saturn V hatte die Startmasse $m_0=2,9\cdot 10^6$ kg. Innerhalb von 160 $s$ nach dem Start
wurden die $2,24\cdot 10^6$ kg Treibstoff der ersten Stufe gleichmäßig verbrannt. Diese ersten 160 $s$ werden als Brenndauer der ersten Stufe bezeichnet. Die Geschwindigkeit $v(t)$ (in m/s) einer Saturn V kann $t$ Sekunden nach dem Start während der Brenndauer der ersten Stufe näherungsweise durch die Funktion $v$ mit

\begin{center}
	$v(t)=0,0000000283\cdot t^5-0,00000734\cdot t^4+0,000872\cdot t^3-0,00275\cdot t^2+2,27\cdot t$\end{center}
	
	beschrieben werden.


\subsection{Aufgabenstellung:}
\begin{enumerate}
	\item Berechne die Beschleunigung einer Saturn V beim Start und am Ende der Brenndauer der ersten Stufe!
	
Gib an, ob die Beschleunigung der Rakete nach der halben Brenndauer der ersten
Stufe kleiner oder größer als die mittlere Beschleunigung (= mittlere Änderungsrate der
Geschwindigkeit) während der ersten 160 Sekunden des Flugs ist! Begründe deine Antwort anhand des Graphen der Geschwindigkeitsfunktion!

\item Berechnen Sie die Länge des Weges, den eine Saturn V 160 $s$ nach dem Start zurückgelegt hat!

Begründe, warum in dieser Aufgabenstellung der zurückgelegte Weg nicht mit der
Formel "`Weg = Geschwindigkeit mal Zeit"' berechnet werden kann!
	
\item Berechne denjenigen Zeitpunkt $t_1$ für den gilt: $v(t_1)=\frac{v(0)-v(160)}{2}$.\\
Interpretiere $t_1$ und $v(t_1)$ im gegebenen Kontext!

\item Beschreibe die Abhängigkeit der Treibstoffmasse $m_T$ (in Tonnen) der Saturn V von der Flugzeit $t$ während der Brenndauer der ersten Stufe durch eine Funktionsgleichung!

Gib die prozentuelle Abnahme der Gesamtmasse einer Saturn V für diesen Zeitraum an!

\item Nach dem Gravitationsgesetz wirkt auf eine im Abstand $r$ vom Erdmittelpunkt befindliche Masse $m$ die Gravitationskraft $F=G\cdot\frac{m\cdot M}{r^2}$, wobei $G$ die Gravitationskonstante und $M$ die Masse der Erde ist.

Deute das bestimmte Integral $\int^{r_2}_{r_1}{F(r)}$\,d$r$ im Hinblick auf die Beförderung der Raumstation Skylab in die Erdumlaufbahn und beschreibe, welche Werte dabei für die Grenzen $r_1$ und $r_2$ einzusetzen sind!

Begründe anhand der Formel für die Gravitationskraft, um welchen Faktor sich das bestimmte Integral $\int^{r_2}_{r_1}{F(r)}$\,d$r$ ändert, wenn ein Objekt mit einem Zehntel der Masse von Skylab in eine Umlaufbahn derselben Höhe gebracht wird.
	
						\end{enumerate}\leer
				
\antwort{\subsection{Lösungserwartung:}
\begin{enumerate}
	\item $a(0)=v'(0)=2,27$\,m/$\text{s}^2$\\
	$a(160)=v'(160)=40,83$\,m/$\text{s}^2$
	
	\meinlr[0.15]{Bestimmt man die zur Sekante parallele Tangente, so liegt die Stelle des zugehörigen Berührpunktes rechts von $t=80$. Aus der Linkskrümmung der Funktion $v$ folgt
daher, dass die Beschleunigung nach 80 Sekunden kleiner als die mittlere Beschleunigung im Intervall $[0\,s;160\,s]$ ist.

Weitere mögliche Begründung:\\
Die mittlere Beschleunigung (= Steigung der Sekante) in $[0; 160]$ ist größer als die Momentanbeschleunigung (= Steigung der Tangente) bei $t=80$.}{\resizebox{0.8\linewidth}{!}{\psset{xunit=0.02cm,yunit=0.0025cm,algebraic=true,dimen=middle,dotstyle=o,dotsize=5pt 0,linewidth=0.8pt,arrowsize=3pt 2,arrowinset=0.25}
\begin{pspicture*}(-43.045291005289336,-279.80766475314863)(210.57291005290267,2222.0020436279533)
\psaxes[labelFontSize=\scriptstyle,xAxis=true,yAxis=true,Dx=50.,Dy=400.,ticksize=-2pt 0,subticks=2]{->}(0,0)(0.,0.)(210.57291005290267,2222.0020436279533)
\psplot[linewidth=1.2pt,plotpoints=200]{0}{210.57291005290267}{2.83E-8*x^(5.0)-7.34E-6*x^(4.0)+8.72E-4*x^(3.0)-0.00275*x^(2.0)+2.27*x}
\psline(0.,0.)(159.46696207090903,2000.)
\psline[linestyle=dashed,dash=3pt 3pt](159.46696207090903,0.)(159.46696207090903,2000.)
\begin{scriptsize}
\rput[tl](13.378941798941455,2128.73282204357){v(t) in m/s}
\rput[tl](164.6191534391477,-190.2963719862192){t in s}
\end{scriptsize}
\end{pspicture*}}}\leer
	
\item $s(160)=\int^{160}_0{v(t)}\,$d$t\approx 93\,371$

zurückgelegter Weg nach $160\,s$:$\,93\,371$

$s=v\cdot t$ gilt nur bei konstanter Geschwindigkeit. Die Geschwindigkeit der Saturn V ändert
sich allerdings mit der Zeit.

\item $v(0)=0\,m/s;\hspace*{0.5cm} v(160)\approx 2\,022\,m/s$\\
$v(t_1)=1\,011\Rightarrow t_1\approx 125\,s$
	
Die Geschwindigkeit ist nach $125\,s$ halb so groß wie nach $160\,s$.

\item $m_T(t)=2\,240-14\cdot t$\\
$\frac{2,24}{2,9}\approx 0,77$\\
Die Gesamtmasse hat um $77\,\%$ abgenommen.

\item Das Ergebnis gibt die Arbeit an, die nötig ist, um die Raumstation Skylab in die entsprechende Erdumlaufbahn zu bringen.\\
$r_1$ ist der Erdradius, $r_2$ ist die Summe aus Erdradius und Höhe der Umlaufbahn.

Die Gravitationskraft und somit auch die Arbeit sind direkt proportional zur Masse des Objekts. Die erforderliche Arbeit ist daher nur ein Zehntel des Vergleichswertes.
			\end{enumerate}
			
			\subsection{Lösungsschlüssel:}
\begin{enumerate}
\item Ein Punkt für die richtige Berechnung der beiden Beschleunigungswerte.\\
Toleranzintervall für $a(0):\,[2,2\,m/s^2; 2,3\,m/s^2]$\\
Toleranzintervall für $a(160):\,[40\,m/s^2; 42\,m/s^2]$\\
Ein Punkt für eine sinngemäß richtige Begründung laut Lösungserwartung.

\item Ein Punkt für die richtige Berechnung des zurückgelegten Weges.\\
Toleranzintervall: $[93\,000\,m; 94\,000\,m]$\\
Ein Punkt für eine sinngemäß richtige Begründung laut Lösungserwartung.

\item Ein Punkt für die richtige Berechnung des Zeitpunkts $t_1$.\\
Toleranzintervall: $[124\,s; 126\,s]$\\
Ein Punkt für eine sinngemäß richtige Deutung der beiden Werte laut Lösungserwartung.

\item Ein Punkt für die Angabe einer richtigen Funktionsgleichung.\\
Äquivalente Schreibweisen sind als richtig zu werten.\\
Ein Punkt für die Angabe des richtigen Prozentsatzes.\\
Toleranzintervall: $[77\,\%; 78\,\%]$

\item Ein Punkt für die richtige Deutung des bestimmten Integrals und die richtige Beschreibung
der Werte der beiden Grenzen.\\
Ein Punkt für eine richtige Begründung, um welchen Faktor sich das Ergebnis ändert.\\
Die direkte Proportionalität zwischen Masse und Gravitationskraft muss dabei sinngemäß
erwähnt werden.
\end{enumerate}}
		\end{langesbeispiel}