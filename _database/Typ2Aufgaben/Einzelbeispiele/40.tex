\section{40 - MAT - WS 2.1, WS2.2, WS 2.3, WS 3.2, WS 3.3  - Lottozahlen - Matura 2013/14 1. Nebentermin}

\begin{langesbeispiel} \item[0] %PUNKTE DES BEISPIELS
				Beim österreichischen Zahlenlotto sind 45 Kugeln mit den Zahlen von 1 bis 45 beschriftet. Bei einer Lottoziehung werden zufällig und ohne Zurücklegen 6 der 45 Kugeln aus der "`Lottotrommel"' entnommen. Die Wahrscheinlichkeit, dass eine bestimmte Zahl im Rahmen einer Lottoziehung (6 aus 45) gezogen wird, beträgt $\frac{6}{45}$.
				
				Ein Zufallsexperiment habe genau zwei Ausgänge: Ein Ereignis $A$ tritt mit einer gewissen Wahrscheinlichkeit ein oder es tritt nicht ein. Das empirische Gesetz der großen Zahlen besagt nun Folgendes: Bei einer hinreichend großen Anzahl von Durchführungen dieses Experiments stabilisieren sich die relativen Häufigkeiten $h_r(A)$ bei einem Wert, der der Wahrscheinlichkeit $P(A)$ für das Ereignis $A$ entspricht.

Abbildung 1 zeigt die absoluten Ziehungshäufigkeiten der Zahlen 1 bis 45 bei den 104 Ziehungen im Kalenderjahr 2010. 

Abbildung 2 zeigt die absoluten Ziehungshäufigkeiten der Zahlen 1 bis 45 bei 2 056 Ziehungen vom 1.1.1986 bis zum 27.11.2011.

Abbildung 1:
				\begin{center}\resizebox{0.8\linewidth}{!}{\includegraphics{../_database/Bilder/Bild40-1.eps}}\end{center}
				
Abbildung 2:
\begin{center}\resizebox{0.8\linewidth}{!}{\includegraphics{../_database/Bilder/Bild40-2.eps}}\end{center}

\subsection{Aufgabenstellung:}
\begin{enumerate}
	\item \fbox{A} Kreuze die beiden zutreffenden Aussagen an!
	
	Stelle eine der falschen Aussagen zur Ziehungswahrscheinlichkeit richtig!
	\multiplechoice[5]{  %Anzahl der Antwortmoeglichkeiten, Standard: 5
					L1={Die Ziehung der Gewinnzahlen $3, 12, 19, 25, 36, 41$ bei einer Lottoziehung ist gleich wahrscheinlich wie die Ziehung der Gewinnzahlen $1, 2, 3, 4, 5, 6$.},   %1. Antwortmoeglichkeit 
					L2={Eine Zahl, die bei einer Lottoziehung gezogen wurde, wird bei der darauffolgenden Lottoziehung mit einer Wahrscheinlichkeit kleiner als $\frac{6}{45}$ erneut gezogen. },   %2. Antwortmoeglichkeit
					L3={Im Kalenderjahr 2010 war die Wahrscheinlichkeit, die Zahl 8 zu ziehen, bei manchen Ziehungen kleiner als $\frac{6}{45}$.},   %3. Antwortmoeglichkeit
					L4={Die Wahrscheinlichkeit, dass die Zahl 17 als erste Zahl gezogen wird, beträgt bei jeder Ziehung $\frac{1}{45}$.},   %4. Antwortmoeglichkeit
					L5={Die Wahrscheinlichkeit, dass die Zahl 32 bei einer Ziehung als zweite Zahl gezogen wird, beträgt $\frac{1}{44}$.},	 %5. Antwortmoeglichkeit
					L6={},	 %6. Antwortmoeglichkeit
					L7={},	 %7. Antwortmoeglichkeit
					L8={},	 %8. Antwortmoeglichkeit
					L9={},	 %9. Antwortmoeglichkeit
					%% LOESUNG: %%
					A1=1,  % 1. Antwort
					A2=4,	 % 2. Antwort
					A3=0,  % 3. Antwort
					A4=0,  % 4. Antwort
					A5=0,  % 5. Antwort
					}
					
		\item Ermittle die relative Ziehungshäufigkeit der Zahl 10 im Kalenderjahr 2010!
		
		Zeige, dass die Ziehungshäufigkeiten der Zahl 10 in den Abbildungen 1 und 2 mit dem empirischen Gesetz der großen Zahlen im Einklang stehen!
		
		\item Überprüfe anhand von Abbildung 2, bei welchen Zahlen die absolute Ziehungshäufigkeit bei den $2\,056$ Ziehungen um mehr als das Doppelte der Standardabweichung vom Erwartungswert abweicht!
		
		Gib an, welche Verteilung du für die Berechnungen verwendet hast, und begründe deine Entscheidung.
						\end{enumerate}\leer
				
\antwort{
\begin{enumerate}
	\item \subsection{Lösungserwartung:} 
	
	\subsubsection{Richtigstellung:}
	\begin{itemize}
		\item  Eine Zahl, die bei einer Lottoziehung gezogen wurde, wird bei der darauffolgenden Lottoziehung mit einer Wahrscheinlichkeit von $\frac{6}{45}$ erneut gezogen.
		\item Im Kalenderjahr 2010 war die Wahrscheinlichkeit, die Zahl 8 zu ziehen, bei jeder Ziehung gleich $\frac{6}{45}$.
		\item Die Wahrscheinlichkeit, dass die Zahl 32 bei einer Ziehung als zweite Zahl gezogen wird, beträgt $\frac{1}{45}$.
	\end{itemize}
	 	
	\subsection{Lösungsschlüssel:}
	\begin{itemize}
		\item  Ein Ausgleichspunkt ist nur dann zu geben, wenn genau zwei Aussagen angekreuzt sind und beide Kreuze richtig gesetzt sind.
		\item Ein Punkt für eine (sinngemäß) korrekte Richtigstellung einer der drei falschen Aussagen.
	\end{itemize}
	
	\item \subsection{Lösungserwartung:}
			
		Die relative Ziehungshäufigkeit der Zahl 10 im Kalenderjahr 2010 beträgt $\frac{19}{104}\approx 0,183$.
		
		Bei $2\,056$ Ziehungen hat sich die relative Häufigkeit $\left(\frac{270}{2\,056}\approx 0,131\right)$ der theoretischen Ziehungswahrscheinlichkeit von $\frac{6}{45}\approx 0,133$ im Vergleich zu den 104 Ziehungen des Kalenderjahres 2010 angenähert.
		
	\subsection{Lösungsschlüssel:}
	
\begin{itemize}
	\item  Ein Punkt für das korrekte Bestimmen der relativen Ziehungshäufigkeit, wobei das Ergebnis in Bruch-, Dezimal- oder Prozentschreibweise angegeben werden kann.
	\item  Ein Punkt für eine (sinngemäß) korrekte Erklärung, warum die Häufigkeiten in den Abbildungen 1 und 2 mit dem empirischen Gesetz der großen Zahlen für die relative Ziehungshäufigkeit der Zahl 10 im Einklang stehen.
\end{itemize}

\item \subsection{Lösungserwartung:}
			
		$\mu=2\,056\cdot\frac{6}{45}\approx 274$ \hspace{1cm} $\sigma=\sqrt{2\,056\cdot\frac{6}{45}\cdot\frac{39}{45}}\approx 15$
		
		$\mu-2\sigma\approx 243$
		
		$\mu+2\sigma\approx 305$
		
		Bei allen Zahlen, die höchstens 243-mal oder mindestens 305-mal gezogen wurden, weicht die Ziehungshäufigkeit um mehr als 2 $\sigma$ vom Erwartungswert ab. Dies trifft auf die Zahlen 39, 42 und 43 zu.
		
		Es wurde die Binomialverteilung verwendet, da es um Anzahlen geht ("`absolute Ziehungshäufigkeit"'), es nur zwei mögliche Ausgänge bei einer Lottoziehung gibt (eine bestimmte Zahl wurde gezogen oder sie wurde nicht gezogen) und weil die Ziehungswahrscheinlichkeit von Ziehung zu Ziehung gleich bleibt.
		
	\subsection{Lösungsschlüssel:}
	
\begin{itemize}
	\item  Ein Punkt für das Ermitteln der Zahlen $39,42$ und 43.
	\item  Ein Punkt für eine (sinngemäß) korrekte Begründung, warum die Binomialverteilung verwendet werden darf.

\end{itemize}
\end{enumerate}}
		\end{langesbeispiel}