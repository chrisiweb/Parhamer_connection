\section{136 - K7 - WM - Silbernitrat - VerSie}

\begin{langesbeispiel} \item[8] %PUNKTE DES BEISPIELS
Silbernitrat mit einem Reinheitsgrad von 99,99\,\% wird zu einem Preis von 25\,\euro pro Gramm angeboten.

Die Funktion $K$ mit $K(x)=0,01x^3-0,5x^2+17x+100$ beschreibt die Gesamtkosten für die Produktion in Abhängigkeit von der produzierten Mengeneinheit $x$ mit $K(x)$ in \euro und $x$ in Gramm.

Die untenstehende Abbildung zeigt den Graph der Kostenfunktion $K$ des herstellenden Betriebes.

\begin{center}
\psset{xunit=0.2cm,yunit=0.005cm,algebraic=true,dimen=middle,dotstyle=o,dotsize=5pt 0,linewidth=1.6pt,arrowsize=3pt 2,arrowinset=0.25}
\begin{pspicture*}(-4,-96)(69.64659574468095,1898.136645962731)
\multips(0,0)(0,100.0){19}{\psline[linestyle=dashed,linecap=1,dash=1.5pt 1.5pt,linewidth=0.4pt,linecolor=gray]{c-c}(0,0)(69.64659574468095,0)}
\multips(0,0)(5.0,0){15}{\psline[linestyle=dashed,linecap=1,dash=1.5pt 1.5pt,linewidth=0.4pt,linecolor=gray]{c-c}(0,0)(0,1898.136645962731)}
\psaxes[labelFontSize=\scriptstyle,xAxis=true,yAxis=true,Dx=5.,Dy=200.,ticksize=-2pt 0,subticks=0]{->}(0,0)(0.,0.)(69.64659574468095,1898.136645962731)[$x$,140] [$K(x)$,-40]
\psplot[linewidth=2.pt,plotpoints=200]{-0}{69.64659574468095}{0.01*x^(3.0)-0.5*x^(2.0)+17.0*x+100.0}
\rput[tl](58.027021276595825,1291.9254658385084){$K$}
\antwort{\psplot[linewidth=2.pt,plotpoints=200]{0}{69.64659574468095}{25.0*x}
\psplot[linewidth=2.pt,plotpoints=200]{-2.62148936170214}{69.64659574468095}{25.0*x-(0.01*x^(3.0)-0.5*x^(2.0)+17.0*x+100.0)}
\rput[tl](43.573404255319204,1230){$E$}
\rput[tl](50.587659574468155,370){$G$}}
\end{pspicture*}
\end{center}%Aufgabentext

\begin{aufgabenstellung}
\item Aus der Angabe lässt sich ablesen, dass die Preisfunktion $p$ die Gleichung\\ 
	$p(x)=25$ besitzt.%Aufgabentext

\Subitem{Stelle die Gleichung der Erlösfunktion $E$ auf.\vspace{0,3cm}

$E(x)=\,\antwort[\rule{5cm}{0.3pt}]{25x}$} %Unterpunkt1
\Subitem{Zeichne den Graphen der Erlösfunktion in das obige Koordinatensystem.} %Unterpunkt2

\item %Aufgabentext

\Subitem{Die Gewinnfunktion ist die Differenz aus Erlös und Kostenfunktion. Skizziere die Gewinnfunktion in der obigen Grafik.} %Unterpunkt1
\Subitem{Lies den Absatzbereich aus der Grafik ab, bei dem mit Gewinn verkauft wird.\vspace{0,3cm}
	
	untere Gewinnschranke: \antwort[\rule{5cm}{0.3pt}]{8,64}\vspace{0,3cm}
	
	obere Gewinnschranke: \antwort[\rule{5cm}{0.3pt}]{60,49}} %Unterpunkt2

\item %Aufgabentext

\Subitem{Zeige, dass die Gewinnfunktion die Gleichung\\ 
$G(x)=-0,01x^3+0,5x^2+8x-100$ besitzt.} %Unterpunkt1
\Subitem{Berechne jene Absatzmenge, bei der der Gewinn maximal ist. Gib an wie groß der Gewinn in diesem Fall ist.} %Unterpunkt2

\item Die erste Ableitung der Kostenfunktion wird als Grenzkostenfunktion bezeichnet.%Aufgabentext

\ASubitem{Bestimme den Wert der Grenzkostenfunktion für ein Produktion von 30 Stück.} %Unterpunkt1
\Subitem{Was bedeutet der Wert im gegebenen Kontext?} %Unterpunkt2

\end{aufgabenstellung}

\begin{loesung}
\item \subsection{Lösungserwartung:} 

\Subitem{Gleichung der Erlösfunktion: siehe oben.} %Lösung von Unterpunkt1
\Subitem{Graph der Erlösfunktion: siehe oben.} %%Lösung von Unterpunkt2

\setcounter{subitemcounter}{0}
\subsection{Lösungsschlüssel:}
 
\Subitem{Ein Punkt für die richtige Funktionsgleichung.} %Lösungschlüssel von Unterpunkt1
\Subitem{Ein Punkt für den richtigen Graphen.} %Lösungschlüssel von Unterpunkt2

\item \subsection{Lösungserwartung:} 

\Subitem{Graph der Gewinnfunktion: siehe oben.} %Lösung von Unterpunkt1
\Subitem{Gewinnschranken: siehe oben.} %%Lösung von Unterpunkt2

\setcounter{subitemcounter}{0}
\subsection{Lösungsschlüssel:}
 
\Subitem{Ein Punkt für den Graphen der Gewinnfunktion.} %Lösungschlüssel von Unterpunkt1
\Subitem{Ein Punkt für die richtigen Schranken.

		Toleranzintervall untere Schranke: $[6;10]$\\
		Toleranzintervall obere Schranke: $[58;62]$} %Lösungschlüssel von Unterpunkt2

\item \subsection{Lösungserwartung:} 

\Subitem{ $G(x)=E(x)-K(x)=25\cdot x-(0,01\cdot x^3-0,5\cdot x^2+17\cdot x+100)$\\
	 $G(x)=-0,01x^3+0,5x^2+8x-100$} %Lösung von Unterpunkt1
\Subitem{$G'(x)=0 \Rightarrow (x_1\approx -6,65)\text{ und }x_2=40$
	 
	 $G(40)=380$
	 
	 Bei einer Absatzmenge von 40\,ME ist der Gewinn mit 380\,GE maximal.} %%Lösung von Unterpunkt2

\setcounter{subitemcounter}{0}
\subsection{Lösungsschlüssel:}
 
\Subitem{Ein Punkt für die richtige Berechnung der Gewinnfunktion.} %Lösungschlüssel von Unterpunkt1
\Subitem{Ein Punkt für die richtige Berechnung des Maximal-Gewinn.} %Lösungschlüssel von Unterpunkt2

\item \subsection{Lösungserwartung:} 

\Subitem{$K'(x)=0,03x^2-x+17 \Rightarrow K'(30)=14$
	
	Die Grenzkosten bei einer Produktion von 30\,ME betragen 14\,GE.} %Lösung von Unterpunkt1
\Subitem{Die momentane Änderungsrate der Kostenfunktion bei einer Produktion von 30\,ME beträgt 14\,GE.} %%Lösung von Unterpunkt2

\setcounter{subitemcounter}{0}
\subsection{Lösungsschlüssel:}
 
\Subitem{Ein Punkt für die richtige Berechnung der Grenzkosten.} %Lösungschlüssel von Unterpunkt1
\Subitem{Ein Punkt für die richtige Interpretation.} %Lösungschlüssel von Unterpunkt2

\end{loesung}

\end{langesbeispiel}