\section{73 - MAT - AG 2.1, FA 2.5, WS 2.2, WS 3.2, WS 4.1 - Buccolam - Matura 2016/17 Haupttermin}

\begin{langesbeispiel} \item[0] %PUNKTE DES BEISPIELS
	
Buccolam ist ein flüssiges Arzneimittel zur Behandlung akuter, länger anhaltender Krampfanfälle bei Personen, die mindestens drei Monate alt und jünger als 18 Jahre sind (im Folgenden "`Kinder"'). Es enthält als Wirkstoff Midazolam, ein stark wirksames Beruhigungsmittel. Im Rahmen einer klinischen Studie wurde Buccolam 440 Kindern mit Krampfanfällen verabreicht. Bei 22 Kindern traten dabei als Nebenwirkung Übelkeit und Erbrechen auf. Bei 308 Kindern verschwanden sichtbare Zeichen der Krampfanfälle innerhalb von 10 Minuten nach Verabreichung des Medikaments.

\subsection{Aufgabenstellung:}
\begin{enumerate}
	\item Es gibt vier Arten von Buccolam-Spritzen mit der dem jeweiligen Altersbereich entsprechenden Midazolam-Dosis:
	
	\begin{center}
		\begin{tabular}{|l|l|l|}\hline
		\cellcolor[gray]{0.9}Altersbereich&\cellcolor[gray]{0.9}Midazolam-Dosis&\cellcolor[gray]{0.9}Farbe des Etiketts\\ \hline
		bis $<1$ Jahr&2,5\,mg&Gelb\\ \hline
		1 Jahr bis $<5$ Jahre&5\,mg&Blau\\ \hline
		5 Jahre bis $<10$ Jahre&7,5\,mg&Violett\\ \hline
		10 Jahre bis $<18$ Jahre&10\,mg&Orange\\ \hline		
		\end{tabular}
	\end{center}
	
	\begin{scriptsize}\begin{singlespace}Datenquelle: http://www.ema.europa.eu/docs/de\_DE/document\_library/EPAR\_-\_Product\_ Information/human/002267/WC500112310.pdf [02.12.2016].
 \end{singlespace}\end{scriptsize}
	
	Diese Spritzen beinhalten je nach Altersbereich eine Lösung mit der entsprechenden Midazolam-Dosis. Zum Beispiel beinhalten die Spritzen mit gelbem Etikett eine Lösung mit einem Volumen von 0,5 ml. 
	
	Allgemein besteht zwischen dem Volumen $V$ (in ml) einer Lösung und der Midazolam-Dosis $D$ (in mg) ein direkt proportionaler Zusammenhang.\leer
	
 Beschreiben den Zusammenhang zwischen dem Volumen $V$ einer Lösung und der Midazolam-Dosis $D$ mithilfe einer Gleichung!\leer

 Gib an, ob zwischen dem Alter (in Jahren) der Patientin/des Patienten und der zu verabreichenden Midazolam-Dosis ein linearer Zusammenhang besteht, und begründe deine Entscheidung anhand der in der obigen Tabelle angegebenen Daten!\leer

\item Die relative Häufigkeit $H$ von Nebenwirkungen nach Verabreichung eines Medikaments wird folgendermaßen klassifiziert:

\begin{center}
	\begin{tabular}{|l|l|}\hline
	häufig&$0,01\leq H<0,1$\\ \hline
	gelegentlich&$0,001\leq H<0,01$\\ \hline
	selten&$0,0001\leq H<0,001$\\ \hline
	sehr selten&$H<0,0001$\\ \hline	
	\end{tabular}
\end{center}
\begin{scriptsize}\begin{singlespace}Datenquelle: https://www.vfa.de/de/patienten/patientenratgeber/ratgeber031.html [02.12.2016] (adaptiert).\end{singlespace}\end{scriptsize}

\fbox{A} Gib an, wie die relative Häufigkeit von Nebenwirkungen der Art "`Übelkeit und Erbrechen"' bei der Verabreichung von Buccolam gemäß der in der Einleitung erwähnten klinischen Studie klassifiziert werden müsste!\leer

 In der Packungsbeilage von Buccolam wird die Häufigkeit der Nebenwirkung "`Hautausschlag"' mit "`gelegentlich"' angegeben. 

Die Zufallsvariable $X$ beschreibt, bei wie vielen von den 440 im Rahmen der Studie mit Buccolam behandelten Kindern die Nebenwirkung "`Hautausschlag"' auftritt, und kann als binomialverteilte Zufallsvariable mit dem Parameter $p=0,01$ sowie dem Erwartungswert $\mu$ und der Standardabweichung $\sigma$ angenommen werden. 

Gib an, bei wie vielen Kindern in der erwähnten Studie die Nebenwirkung "`Hautausschlag"' auftreten darf, damit die Anzahl der davon betroffenen Kinder im Intervall $[\mu-\sigma;\mu+\sigma]$ liegt!

\item Der tatsächliche Anteil derjenigen Patientinnen/Patienten, bei denen sichtbare Zeichen der Krampfanfälle innerhalb von 10 Minuten nach der Medikamentenverabreichung verschwinden, wird mit $p$ bezeichnet. 

Ermittle für $p$ anhand der in der Einleitung angegebenen Daten der klinischen Studie ein symmetrisches Konfidenzintervall mit dem Konfidenzniveau $\gamma=0,95$! \leer

 In einer anderen Studie zur Wirksamkeit von Buccolam wurden $n_1$ Kinder untersucht. Die Ergebnisse führten mit derselben Methodik zu dem symmetrischen Konfidenzintervall  $[0,67; 0,73]$ mit dem Konfidenzniveau $\gamma_1$. 

Begründe, warum die Werte  $n_1<400$  und  $\gamma_1=0,99$  nicht die Grundlage zur Berechnung dieses Konfidenzintervalls gewesen sein können

\end{enumerate}

\antwort{
\begin{enumerate}
	\item \subsection{Lösungserwartung:} 

$V(D)=0,2\cdot D$

Zwischen dem Alter (in Jahren) der Patientin/des Patienten und der zu verabreichenden Midazolam-Dosis besteht kein linearer Zusammenhang.\leer

Mögliche Begründung:  Bei einem linearen Zusammenhang würden z. B. 3-jährige Kinder eine niedrigere Dosis als 4-jährige Kinder erhalten. Laut Tabelle ist dies nicht der Fall.

	\subsection{Lösungsschlüssel:}
	\begin{itemize}
		\item Ein Punkt für eine korrekte Gleichung. Äquivalente Gleichungen sind als richtig zu werten.
		\item Ein Punkt für eine richtige Entscheidung und eine korrekte Begründung. Andere korrekte Begründungen (z. B. mithilfe einer Skizze oder mit einem Hinweis auf das Vorliegen einer unstetigen Funktion) sind ebenfalls als richtig zu werten.
	\end{itemize}
	
	\item \subsection{Lösungserwartung:}
	
	Da bei 22 von 44 Kindern die Nebenwirkung "`Übelkeit und Erbrechen"' auftrat, beträgt die relative Häufigkeit $\frac{22}{440}=0,05$.
	
	Wegen $0,01\leq 0,05<0,1$ würde sich für die Nebenwirkung "`Übelkeit und Erbrechen"' die Klassifizierung "`häufig"' ergeben.\leer
	
	Mögliche Vorgehensweise:
	
	$\mu=4,4$ $\sigma\approx 2,09 \Rightarrow [\mu-\sigma;\mu+\sigma]\approx [2,31;6,49]$
	
	Die Nebenwirkung "`Hautausschlag"' muss demnach bei mindestens drei und darf bei höchstens sechs Kindern der erwähnten Studie auftreten.
	
	\subsection{Lösungsschlüssel:}
	
\begin{itemize}
	\item Ein Ausgleichspunkt für eine korrekte Klassifizierung.   
	\item Ein Punkt für die richtige Lösung. 
\end{itemize}

\item \subsection{Lösungserwartung:}
	
	$n=440, h=0,7$
	
	$0,7\,\pm\,1,96\sqrt{\frac{0,7\cdot 0,3}{440}}\approx 0,7\,\pm\,0,04 \Rightarrow [0,66;0,74]$\leer
	
	Die Werte $n_1<400$ und $\gamma_1=0,99$ würden zu einem wesentlich breiteren Konfidenzintervall führen und können daher nicht die Grundlage zur Berechnung gewesen sein.	
	\subsection{Lösungsschlüssel:}
	
\begin{itemize}
	\item  Ein Punkt für ein korrektes Intervall. Andere Schreibweisen des Ergebnisses (als Bruch oder in Prozent) sind ebenfalls als richtig zu werten. 
	
	Toleranzintervall für den unteren Wert: $[0,65; 0,66]$ 
	
	Toleranzintervall für den oberen Wert: $[0,74; 0,75]$ 
	
	Die Aufgabe ist auch dann als richtig gelöst zu werten, wenn bei korrektem Ansatz das Ergebnis aufgrund eines Rechenfehlers nicht richtig ist. 
	\item Ein Punkt für eine (sinngemäß) korrekte Begründung.
\end{itemize}

\end{enumerate}}
		\end{langesbeispiel}