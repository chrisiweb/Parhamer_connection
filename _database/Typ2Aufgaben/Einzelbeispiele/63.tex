\section{63 - MAT - FA 5.1, AN 4.3, AN 1.3, FA 2.5, FA 2.3 - Bevölkerungswachstum in den USA - Matura 2015/16 1. Nebentermin}

\begin{langesbeispiel} \item[0] %PUNKTE DES BEISPIELS
	
Die erste Volkszählung in den USA fand im Jahre 1790 statt. Seit diesem Zeitpunkt werden Volkszählungen im Abstand von zehn Jahren abgehalten. Zwischen den Volkszählungen wird die Zahl der Einwohner/innen durch die Meldeämter ermittelt.

Nachstehend wird ein Überblick über die Bevölkerungsentwicklung in den USA im Zeitraum von 1790 bis 1890 (Tabelle) bzw. 2003 bis 2013 (Grafik) gegeben.

Tabelle: Bevölkerungsentwicklung in den USA von 1790 bis 1890

\meinlr{\begin{tabular}{|c|c|}\hline
\cellcolor[gray]{0.9}Jahr&\cellcolor[gray]{0.9}Einwohnerzahl in Millionen\\ \hline
1790&3,9\\ \hline
1800&5,2\\ \hline
1810&7,2\\ \hline
1820&9,6\\ \hline
1839&12,9\\ \hline
1840&17,1\\ \hline
\end{tabular}}{\begin{tabular}{|c|c|}\hline
\cellcolor[gray]{0.9}Jahr&\cellcolor[gray]{0.9}Einwohnerzahl in Millionen\\ \hline
1850&23,2\\ \hline
1860&31,4\\ \hline
1870&38,6\\ \hline
1880&49,3\\ \hline
1890&49,3\\ \hline
\end{tabular}}

\begin{scriptsize}Quelle: Keller, G. (2011). Mathematik in den Life Sciences. Stuttgart: Ulmer. S. 55.\end{scriptsize}

Grafik: Bevölkerungsentwicklung in den USA von 2003 bis 2013

\begin{center}
	\resizebox{0.8\linewidth}{!}{\newrgbcolor{wqwqwq}{0.3764705882352941 0.3764705882352941 0.3764705882352941}
\psset{xunit=0.5cm,yunit=0.015cm,algebraic=true,dimen=middle,dotstyle=o,dotsize=4pt 0,linewidth=0.8pt,arrowsize=3pt 2,arrowinset=0.25}
\begin{pspicture*}(-3,-44.766323609317894)(22.771337825622386,467.0247388732074)
\multips(0,0)(0,100.0){6}{\psline[linestyle=dashed,linecap=1,dash=1.5pt 1.5pt,linewidth=0.4pt,linecolor=lightgray]{c-c}(0,0)(22.771337825622386,0)}
\multips(0,0)(100.0,0){1}{\psline[linestyle=dashed,linecap=1,dash=1.5pt 1.5pt,linewidth=0.4pt,linecolor=lightgray]{c-c}(0,0)(0,467.0247388732074)}
\psaxes[labelFontSize=\scriptstyle,xAxis=true,yAxis=true,labels=y,Dx=2.,Dy=100.,ticksize=-2pt 0,subticks=0]{->}(0,0)(0.,0.)(22.771337825622386,467.0247388732074)
\pspolygon[linewidth=0.8pt,linecolor=wqwqwq,fillcolor=wqwqwq,fillstyle=solid,opacity=0.89](0.5,0.)(0.5,290.73)(1.5,290.73)(1.5,0.)
\pspolygon[linewidth=0.8pt,linecolor=wqwqwq,fillcolor=wqwqwq,fillstyle=solid,opacity=0.89](2.5,0.)(2.5,293.39)(3.5,293.39)(3.5,0.)
\pspolygon[linewidth=0.8pt,linecolor=wqwqwq,fillcolor=wqwqwq,fillstyle=solid,opacity=0.89](4.5,0.)(4.5,296.12)(5.5,296.12)(5.5,0.)
\pspolygon[linewidth=0.8pt,linecolor=wqwqwq,fillcolor=wqwqwq,fillstyle=solid,opacity=0.89](6.5,0.)(6.5,298.93)(7.5,298.93)(7.5,0.)
\pspolygon[linewidth=0.8pt,linecolor=wqwqwq,fillcolor=wqwqwq,fillstyle=solid,opacity=0.89](8.5,0.)(8.5,301.9)(9.5,301.9)(9.5,0.)
\pspolygon[linewidth=0.8pt,linecolor=wqwqwq,fillcolor=wqwqwq,fillstyle=solid,opacity=0.89](10.5,0.)(10.5,304.72)(11.5,304.72)(11.5,0.)
\pspolygon[linewidth=0.8pt,linecolor=wqwqwq,fillcolor=wqwqwq,fillstyle=solid,opacity=0.89](12.5,0.)(12.5,307.37)(13.5,307.37)(13.5,0.)
\pspolygon[linewidth=0.8pt,linecolor=wqwqwq,fillcolor=wqwqwq,fillstyle=solid,opacity=0.89](14.5,0.)(14.5,309.73)(15.5,309.73)(15.5,0.)
\pspolygon[linewidth=0.8pt,linecolor=wqwqwq,fillcolor=wqwqwq,fillstyle=solid,opacity=0.89](16.5,0.)(16.5,311.94)(17.5,311.94)(17.5,0.)
\pspolygon[linewidth=0.8pt,linecolor=wqwqwq,fillcolor=wqwqwq,fillstyle=solid,opacity=0.89](18.5,0.)(18.5,314.18)(19.5,314.18)(19.5,0.)
\pspolygon[linewidth=0.8pt,linecolor=wqwqwq,fillcolor=wqwqwq,fillstyle=solid,opacity=0.89](20.5,0.)(20.5,316.85)(21.5,316.85)(21.5,0.)
\begin{scriptsize}
\rput[tl](0.4,-12){2003}
\rput[tl](2.4,-12){2004}
\rput[tl](4.4,-12){2005}
\rput[tl](6.4,-12){2006}
\rput[tl](8.4,-12){2007}
\rput[tl](10.4,-12){2008}
\rput[tl](12.4,-12){2009}
\rput[tl](14.4,-12){2010}
\rput[tl](16.4,-12){2011}
\rput[tl](18.4,-12){2012}
\rput[tl](20.4,-12){2013}
\end{scriptsize}
\rput[tl](-3,380){$\rotatebox{90}{\text{EinwohnerInnen in Millionen}}$}
\begin{tiny}
\rput[tl](0.35,305.73){290,73}
\rput[tl](2.35,308.39){293,39}
\rput[tl](4.35,311.12){296,12}
\rput[tl](6.35,313.93){298,93}
\rput[tl](8.35,316.90){301,90}
\rput[tl](10.35,319.72){304,72}
\rput[tl](12.35,322.37){307,37}
\rput[tl](14.35,324.73){309,73}
\rput[tl](16.35,326.94){311,94}
\rput[tl](18.35,329.18){314,18}
\rput[tl](20.35,331.85){316,85}
\end{tiny}
\end{pspicture*}}
\end{center}

\begin{scriptsize}\begin{singlespace}Datenquelle: http://de.statista.com/statistik/daten/studie/19320/umfrage/gesamtbevoelkerung-der-usa/ [19.09.2013] (adaptiert).\end{singlespace}\end{scriptsize}

Für den Zeitraum von 1790 bis 1890 kann die Entwicklung der Zahl der Einwohner/innen der USA näherungsweise durch eine Exponentialfunktion $B$ mit  $B(t)=B_0\cdot a^t$  beschrieben werden. Dabei gibt $t$ die Zeit in Jahren, die seit 1790 vergangen sind, an. $B(t)$ wird in Millionen Einwohner/innen angegeben.



\subsection{Aufgabenstellung:}
\begin{enumerate}
	\item Ermittle eine Gleichung der Funktion $B$ unter Verwendung der Daten aus den beiden Jahren 1790 und 1890!
	
	Interpretiere das bestimmte Integral $\int^{50}_0{B'(t)}$d$t$ im gegebenen Zusammenhang!
	
	\item Die erste Ableitung der Funktion $B$ ist gegeben durch $B'(t)=B_0\cdot\ln(a)\cdot a^t$.
	
	Gib $t^*$ so an, das $B'(t^*)=B_0\cdot\ln(a)$ gilt!
	
	Interpretiere $B'(t^*)$ im Zusammenhang mit dem Bevölkerungswachstum in den USA!
	
	\item \fbox{A} Begründe, warum die Bevölkerungsentwicklung in den USA im Zeitraum von 2003 bis 2013 näherungsweise durch eine lineare Funktion $N$ mit $N(t)=k\cdot t+d$ beschrieben werden kann (dabei gibt $t$ die Zeit in Jahren, die seit 2003 vergangen sind, an)!
	
	Interpretiere die Bedeutung des Parameters $k$ dieser linearen Funktion! Eine Berechnung des Parameters $k$ ist nicht erforderlich.	
\end{enumerate}
\antwort{
\begin{enumerate}
	\item \subsection{Lösungserwartung:} 
	
Mögliche Berechnung:

$B_0=3,9$ und $B(100)=62,9 \Rightarrow 62,9=3,9\cdot a^{100} \Rightarrow a=\sqrt[100]{\frac{62,9}{3,9}} \Rightarrow a\approx 1,0282$

$B(t)=3,9\cdot 1,0282^t$

Das bestimmte Integral $\int^{50}_0{B'(t)}$d$t$ gibt denjenigen Wert näherungsweise an, um den die Einwohnerzahl von 1790 bis 1840 gewachsen ist.

	\subsection{Lösungsschlüssel:}
	\begin{itemize}
		\item Ein Punkt für eine korrekte Funktionsgleichung. Äquivalente Gleichungen sind als richtig zu werten.  
		
		Toleranzintervall für $a: [1,028; 1,029]$
		\item Ein Punkt für eine (sinngemäß) korrekte Interpretation.
	\end{itemize}
	
	\item \subsection{Lösungserwartung:}
			
	$t^*=0$
	
	$B_0\cdot\ln(a)$ ist die Wachstumsgeschwindigkeit der Bevölkerung (momentane Änderungsrate der Einwohnerzahl) zum Zeitpunkt $t=0$ in Millionen Einwohner/innen pro Jahr.

	\subsection{Lösungsschlüssel:}
	
\begin{itemize}
	\item Ein Punkt für die richtige Lösung. 
	\item Ein Punkt für eine (sinngemäß) korrekte Interpretation.
\end{itemize}

\item \subsection{Lösungserwartung:}
			
Mögliche Begründung:

Im Zeitraum von 2003 bis 2013 ist die (absolute) Zunahme der Bevölkerung pro Jahr annähernd konstant.

Der Parameter $k$ entspricht der (durchschnittlichen) Zunahme der Bevölkerung pro Jahr.

	\subsection{Lösungsschlüssel:}
	
\begin{itemize}
	\item Ein Ausgleichspunkt für eine (sinngemäß) richtige Begründung.
	\item Ein Punkt für die (sinngemäß) richtige Interpretation des Parameters $k$.
\end{itemize}
\end{enumerate}}
		\end{langesbeispiel}