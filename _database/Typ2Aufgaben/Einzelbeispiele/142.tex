\section{142 - MAT - AG 2.1, AG 2.2, FA 1.7, FA 5.2, AN 1.4 - Wachstumsprozesse - Matura 2019/20 1. HT}

\begin{langesbeispiel}\item[6] %PUNKTE DER AUFGABE
Im Folgenden werden Wachstumsmodelle betrachtet.

Die nachstehende Differenzengleichung beschreibt ein Wachstum.

$N_{t+1}-N_t=r\cdot(S-N_t)$

$N_t\ldots$ Bestand zum Zeitpunkt $t$\\
$r\ldots$ Wachstumskonstante, $r\in\mathbb{R}^+$\\
$S\ldots$ (obere) Kapazitätsgrenze%Aufgabentext

\begin{aufgabenstellung}
\item Auf einem Kreuzfahrtschiff mit 2\,000 Passagieren erkranken ab dem Zeitpunkt $t=0$, zu dem noch kein Passagier erkrankt ist, jeden Tag 5\,\% der noch nicht erkrankten Passagiere.\\
Dabei ist $N_t$ die Anzahl der erkrankten Passagiere zum Zeitpunkt $t$ mit $t$ in Tagen.%Aufgabentext

\Subitem{Gib eine Differenzengleichung für $N_{t+1}$ an.} %Unterpunkt1
\Subitem{Ermittle, nach wie vielen Tagen erstmals mehr als 25\,\% der Passagiere erkrankt sind.} %Unterpunkt2

\item Die Differenzengleichung $N_{t+1}-N_t=r\cdot(S-N_t)$ lässt sich in der Form $N_{t+1}=a\cdot N_t+b$ mit $a,b\in\mathbb{R}$ darstellen.

\Subitem{Drücke $r$ und $S$ durch $a$ und $b$ aus.\leer

$r=\,\antwort[\rule{5cm}{0.3pt}]{1-a}$\leer

$S=\,\antwort[\rule{5cm}{0.3pt}]{\frac{b}{r}=\frac{b}{1-a}}$}

Zur Entwicklung eines neuen Impfstoffs wird das Wachstum einer Bakterienkultur in einer Petrischale untersucht.

In der nachstehenden Tabelle ist der Inhalt $N_t$ (in cm$^2$) derjenigen Fläche angeführt, die von der Bakterienkultur zum Zeitpunkt $t$ (in h) bedeckt wird.

\begin{center}
\begin{tabular}{|c|c|}\hline
\cellcolor[gray]{0.9}$t$ in h&\cellcolor[gray]{0.9}$N_t$ in cm$^2$\\ \hline
0&5,00\\ \hline
1&9,80\\ \hline
2&14,41\\ \hline
\end{tabular}
\end{center}

\Subitem{Ermittle $a$ und $b$ mithilfe der in der obigen Tabelle angegebenen Werte.}

\item Ein Pharmaunternehmen bringt einen neuen Impfstoff auf den Markt. In der ersten Woche nach der Markteinführung haben bereits 15\,000 Personen den Impfstoff gekauft.

Die Anzahl $f(t)$ derjenigen Personen, die den Impfstoff innerhalb von $t$ Wochen nach der Markteinführung gekauft haben, lässt sich modellhaft durch die Funktion $f$ mit\\
$f(t)=1\,000\,000\cdot(1-e^{-k\cdot t})$ beschreiben ($k\in\mathbb{R}^+$).

\ASubitem{Berechne $k$.}

\Subitem{Ermittle denjenigen Zeitpunkt $t_0$, zu dem erstmals 500\,000 Personen diesen Impfstoff gekauft haben.}

\end{aufgabenstellung}

\begin{loesung}
\item \subsection{Lösungserwartung:} 

\Subitem{$N_{t+1}-N_t=0,05\cdot(2\,000-N_t)$ mit $N_0=0$} %Lösung von Unterpunkt1
\Subitem{mögliche Vorgehensweise:

$N_{t+1}=N_t+0,05\cdot(2\,000-N_t)=0,95\cdot N_t+100$\\
$\Rightarrow $ Für $N_0=0$ ist $N_6>500$.

Nach 6 Tagen sind erstmals mehr als 25\,\% der Passagiere erkrankt.} %%Lösung von Unterpunkt2

\setcounter{subitemcounter}{0}
\subsection{Lösungsschlüssel:}
 
\Subitem{Ein Punkt für eine richtige Differenzengleichung, wobei "`$N_0=0$"' nicht angegeben sein muss. Äquivalente Gleichungen sind als richtig zu werten.} %Lösungschlüssel von Unterpunkt1
\Subitem{Ein Punkt für die richtige Lösung.\\
Toleranzintervall: $[5;6]$} %Lösungschlüssel von Unterpunkt2

\item \subsection{Lösungserwartung:} 

\Subitem{$N_{t+1}=(1-r)\cdot N_t+r\cdot S \Rightarrow a=1-r$ und $b=r\cdot S$

$r=1-a$

$S=\frac{b}{r}=\frac{b}{1-a}$} %Lösung von Unterpunkt1
\Subitem{mögliche Vorgehensweise:

I: $9,8=5\cdot a+b$\\
II: $14,41=9,8\cdot a+b$

$a=0,960\ldots\approx 0,96$ und $b=4,997\ldots\approx 5,00$} %%Lösung von Unterpunkt2

\setcounter{subitemcounter}{0}
\subsection{Lösungsschlüssel:}
 
\Subitem{Ein Punkt für die beiden richtigen Lösungen. Andere Schreibweisen der Lösungen sind ebenfalls als richtig zu werten.} %Lösungschlüssel von Unterpunkt1
\Subitem{Ein Punkt für die Angabe der beiden richtigen Werte.} %Lösungschlüssel von Unterpunkt2

\item \subsection{Lösungserwartung:} 

\Subitem{$15\,000=1\,000\,000\cdot(1-e^{-k\cdot 1}) \Rightarrow k=0,01511\ldots \Rightarrow k\approx 0,0151$ pro Woche} %Lösung von Unterpunkt1
\Subitem{$500\,000=1\,000\,000\cdot(1-e^{-k\cdot t_0}) \Rightarrow t_0=45,8\ldots \Rightarrow t_0\approx 46$\,Wochen} %%Lösung von Unterpunkt2

\setcounter{subitemcounter}{0}
\subsection{Lösungsschlüssel:}
 
\Subitem{Ein Ausgleichspunkt für die richtige Lösung, wobei die Einheit "`pro Woche"' nicht angeführt sein muss.} %Lösungschlüssel von Unterpunkt1
\Subitem{Ein Punkt für die richtige Lösung, wobei die Einheit "`Woche"' nicht angegeben sein muss.\\
(Die Lösung kann je nach Rundung von $k$ von der angegebenen Lösung abweichen.} %Lösungschlüssel von Unterpunkt2

\end{loesung}

\antwort{GK/Themen: AG 2.1, AG 2.2, FA 1.7, FA 5.2, AN 1.4}
\end{langesbeispiel}