\section{30 - MAT - AG 2.1, FA 4.1, FA 4.3, FA 5.1, FA 5.6, AN 1.1, AN 1.4, WS 1.3 - Waldbewirtschaftung - BIFIE Aufgabensammlung}

\begin{langesbeispiel} \item[0] %PUNKTE DES BEISPIELS
				Der Holzbestand eines durchschnittlichen Fichtenwaldes in Österreich beträgt $350\,m^3$ pro Hektar Waldfläche. Pro Jahr ist mit einem durchschnittlichen Zuwachs von $3,3\,\%$ zu rechnen. Bei einer nachhaltigen Bewirtschaftung, wie sie in Österreich vorgeschrieben ist, soll der Holzbestand des Waldes gleich bleiben oder leicht zunehmen.
				
Der nachstehenden Grafik kann die Entwicklung des Holzpreises bei Fichtenholz im Zeitraum von 1995 bis 2011 entnommen werden.

\begin{center}
\subsection{Holzpreis bei Fichtenholz in \euro$/m^3$}
\resizebox{1\linewidth}{!}{\psset{xunit=2.3cm,yunit=0.25cm,algebraic=true,dimen=middle,dotstyle=o,dotsize=5pt 0,linewidth=0.8pt,arrowsize=3pt 2,arrowinset=0.25}
\begin{pspicture*}(-0.82,-3.166666666666629)(9.52,49.66666666666655)
\multips(0,0)(0,2.0){27}{\psline[linestyle=dashed,linecap=1,dash=1.5pt 1.5pt,linewidth=0.4pt,linecolor=lightgray]{c-c}(0,0)(9.52,0)}
\multips(0,0)(100.0,0){1}{\psline[linestyle=dashed,linecap=1,dash=1.5pt 1.5pt,linewidth=0.4pt,linecolor=lightgray]{c-c}(0,0)(0,49.66666666666655)}
\psaxes[labelFontSize=\scriptstyle,xAxis=true,yAxis=true,labels=y,Oy=50,Dy=2.,ticksize=-2pt 0,subticks=2]{->}(0,0)(0.,0.)(9.52,49.66666666666655)
\rput[tl](0.06,48.749999999999886){\euro$/m^3$}
\rput[tl](8.42,2.){Jahr}
\psline[linewidth=1.2pt](0.5,26.)(1.,14.)
\psline[linewidth=1.2pt](1.,14.)(1.5,23.)
\psline[linewidth=1.2pt](1.5,23.)(2.,28.)
\psline[linewidth=1.2pt](2.,28.)(2.5,29.)
\psline[linewidth=1.2pt](2.5,29.)(3.,23.)
\psline[linewidth=1.2pt](3.,23.)(3.5,23.)
\psline[linewidth=1.2pt](3.5,23.)(4.,24.)
\psline[linewidth=1.2pt](4.,24.)(4.5,16.)
\psline[linewidth=1.2pt](4.5,16.)(5.,18.)
\psline[linewidth=1.2pt](5.,18.)(5.5,19.)
\psline[linewidth=1.2pt](5.5,19.)(6.,26.)
\psline[linewidth=1.2pt](6.,26.)(6.5,26.)
\psline[linewidth=1.2pt](6.5,26.)(7.,19.)
\psline[linewidth=1.2pt](7.,19.)(7.5,20.)
\psline[linewidth=1.2pt](7.5,20.)(8.,39.)
\psline[linewidth=1.2pt](8.,39.)(8.5,44.)
\begin{scriptsize}
\psdots[dotsize=4pt 0,dotstyle=square*,dotangle=45](0.5,26.)
\psdots[dotsize=4pt 0,dotstyle=square*,dotangle=45](1.,14.)
\psdots[dotsize=4pt 0,dotstyle=square*,dotangle=45](1.5,23.)
\psdots[dotsize=4pt 0,dotstyle=square*,dotangle=45](2.,28.)
\psdots[dotsize=4pt 0,dotstyle=square*,dotangle=45](2.5,29.)
\psdots[dotsize=4pt 0,dotstyle=square*,dotangle=45](3.,23.)
\psdots[dotsize=4pt 0,dotstyle=square*,dotangle=45](3.5,23.)
\psdots[dotsize=4pt 0,dotstyle=square*,dotangle=45](4.,24.)
\psdots[dotsize=4pt 0,dotstyle=square*,dotangle=45](4.5,16.)
\psdots[dotsize=4pt 0,dotstyle=square*,dotangle=45](5.,18.)
\psdots[dotsize=4pt 0,dotstyle=square*,dotangle=45](5.5,19.)
\psdots[dotsize=4pt 0,dotstyle=square*,dotangle=45](6.,26.)
\psdots[dotsize=4pt 0,dotstyle=square*,dotangle=45](6.5,26.)
\psdots[dotsize=4pt 0,dotstyle=square*,dotangle=45](7.,19.)
\psdots[dotsize=4pt 0,dotstyle=square*,dotangle=45](7.5,20.)
\psdots[dotsize=4pt 0,dotstyle=square*,dotangle=45](8.,39.)
\psdots[dotsize=4pt 0,dotstyle=square*,dotangle=45](8.5,44.)
\rput[tl](-0.18,-1){1994}
\rput[tl](0.82,-1){1996}
\rput[tl](1.82,-1){1998}
\rput[tl](2.82,-1){2000}
\rput[tl](3.82,-1){2002}
\rput[tl](4.82,-1){2004}
\rput[tl](5.82,-1){2006}
\rput[tl](6.82,-1){2008}
\rput[tl](7.82,-1){2010}
\rput[tl](8.82,-1){2012}
\end{scriptsize}
\end{pspicture*}}
\end{center}

\subsection{Aufgabenstellung:}
\begin{enumerate}
	\item Bestimme das maximale Holzvolumen (in $m^3/ha$), das bei einer nachhaltigen Bewirtschaftung pro Jahr geschlägert werden darf!
	
Berechne, um wie viel Prozent der Holzbestand eines durchschnittlichen Fichtenwaldes innerhalb von 10 Jahren zunimmt, unter der Annahme, dass keinerlei Schlägerungen vorgenommen werden, alle anderen genannten Rahmenbedingungen jedoch unverändert bleiben!

\item Der Holzbestand eines 20 ha großen Fichtenwaldes wird in einem Zeitraum von 15 Jahren jährlich jeweils am Ende des Jahres (nachdem der jährliche Zuwachs abgeschlossen ist) um 10 $m^3$ pro Hektar (also um 200 $m^3$) verringert.

Ermitteln Sie den Holzbestand des Fichtenwaldes nach Ablauf von 15 Jahren!

Gib an, ob bei dieser Art der Bewirtschaftung der Holzbestand des Fichtenwaldes trotz Schlägerung exponentiell zunimmt, und begründe deine Entscheidung!
	
\item Ermittel für den Zeitraum 2003 bis 2011 die empirische Standardabweichung des Holzpreises entsprechend der Formel

$$\sigma=\sqrt{\frac{1}{n-1}\cdot\sum^n_{i=1}({x_i-\overline{x})^2}}$$

Dabei werden mit $x_i$ die Beobachtungswerte und mit $\overline{x}$ das arithmetische Mittel der Beobachtungswerte bezeichnet. Lies die dazu notwendigen Daten aus der Grafik ab!

Begründen Sie anhand der Grafik, warum die empirische Standardabweichung des
Holzpreises für den Zeitraum 1998 bis 2004 kleiner ist als die empirische Standardabweichung für den Zeitraum 2005 bis 2011!

\item Die Entwicklung des Holzpreises soll für den Zeitraum von 2009 bis 2011 durch eine Funktion $P$ mit $P(t)=a\cdot t^2+b\cdot t+c$ mit $a,b,c\in\mathbb{R}$ und $a\neq 0$ modelliert werden. Der Holzpreis $P(t)$ wird in \euro$/m^3$ angegeben, die Zeitrechnung beginnt mit dem Jahr 2009 und erfolgt in der Einheit "`Jahre"'.

Führe die Modellierung auf Basis der Daten für die Jahre 2009, 2010 und 2011 durch und begründe, warum der Parameter $a$ negativ sein muss!

Ermittle eine Prognose für den in der Grafik nicht angegebenen Holzpreis für das Jahr 2012 mithilfe dieser Modellfunktion!

\item Bestimme für den Zeitraum von 1995 bis 2011 die absoluten Holzpreisänderungen aufeinanderfolgender Jahre!

Gib dasjenige Intervall [Jahr 1; Jahr 2] an, in dem sich der Holzpreis prozentuell am stärksten ändert!
						\end{enumerate}\leer
				
\antwort{\subsection{Lösungserwartung:}
\begin{enumerate}
	\item $350\cdot 0,033=11,55$\\
	Jährlich dürfen bei einer nachhaltigen Bewirtschaftung maximal 11,55 $m^3$/ha Holz geschlägert werden.
	
	$1,033^{10}\approx 1,384$\\
	Unter der Annahme, dass keine Schlägerungen erfolgen, nimmt der Holzbestand innerhalb von zehn Jahren um ca. 38,4\,\% zu.
	
	\item Tabelle:
	
	\begin{tabular}{|c|c|}\hline
	Jahr&Holzbestand in $m^3$\\ \hline
	0&7\,000\\ \hline
	1&7\,031\\ \hline
	2&7\,063,023\\ \hline
	3&7\,096,102759\\ \hline
4&7\,130,27415\\ \hline
5&7\,165,573197\\ \hline
6&7\,202,037112\\ \hline
7&7\,239,704337\\ \hline
8&7\,278,61458\\ \hline
9&7\,318,808861\\ \hline
10&7\,360,329554\\ \hline
11&7\,403,220429\\ \hline
12&7\,447,526703\\ \hline
13&7\,493,295085\\ \hline
14&7\,540,573822\\ \hline
15&7\,589,412759\\ \hline
	\end{tabular}\leer
	
	Wenn der Holzbestand eines 20 ha großen Fichtenwaldes jährlich jeweils am Ende des Jahres um 10 $m^3$ pro Hektar verringert wird, beträgt er nach Ablauf von 15 Jahren ca. 7 589,41 $m^3$.
	
Bei dieser Art der Bewirtschaftung nimmt der Holzbestand nicht exponentiell zu, da das jährliche prozentuelle Wachstum nicht konstant ist.

\item Die empirische Standardabweichung beträgt ca. 9,91 \euro$/m^3$.

Im Zeitraum von 1998 bis 2004 ist die empirische Standardabweichung des Holzpreises kleiner als im Zeitraum von 2005 bis 2011, da die Schwankungen der Werte des Holzpreises im Zeitraum von 1998 bis 2004 geringer sind.

\item $P(t)=-7\cdot t^2+26\cdot t+70$

\begin{center}
	\subsection{Holzpreis in \euro$/m^3$}
	
	\resizebox{0.8\linewidth}{!}{\psset{xunit=5cm,yunit=0.09cm,algebraic=true,dimen=middle,dotstyle=o,dotsize=5pt 0,linewidth=0.8pt,arrowsize=3pt 2,arrowinset=0.25}
\begin{pspicture*}(-0.2473684210526314,-12.595999999999828)(2.9368421052631564,109.31528571428521)
\multips(0,0)(0,10.0){13}{\psline[linestyle=dashed,linecap=1,dash=1.5pt 1.5pt,linewidth=0.4pt,linecolor=lightgray]{c-c}(0,0)(2.9368421052631564,0)}
\multips(0,0)(10.0,0){1}{\psline[linestyle=dashed,linecap=1,dash=1.5pt 1.5pt,linewidth=0.4pt,linecolor=lightgray]{c-c}(0,0)(0,109.31528571428521)}
\psaxes[labelFontSize=\scriptstyle,xAxis=true,yAxis=true,Dx=0.5,Dy=10.,ticksize=-2pt 0,subticks=2]{->}(0,0)(0.,0.)(2.9368421052631564,109.31528571428521)
\psplot[linewidth=1.2pt,plotpoints=200]{0}{2}{-7.0*x^(2.0)+26.0*x+70.0}
\rput[tl](1.2052631578947361,-5.9){Jahre ab 2009}
\rput[tl](0.10526315789473682,104.36685714285666){$y=-7x^2+26x+70$}
\rput[tl](0.32631578947368406,95.19349999999955){$R^2=1$}
\begin{scriptsize}
\psdots[dotsize=4pt 0,dotstyle=square*,dotangle=45](0.,70.)
\psdots[dotsize=4pt 0,dotstyle=square*,dotangle=45](1.,89.)
\psdots[dotsize=4pt 0,dotstyle=square*,dotangle=45](2.,94.)
\end{scriptsize}
\end{pspicture*}}
\end{center}\leer

Der Wert des Parameters $a$ muss negativ sein, weil der Graph der Modellfunktion eine nach unten geöffnete Parabel ist.

Prognosewert für das Jahr 2012:\\
$P(3)=85\,$\euro$/m^3$

\item Tabelle:

\begin{tabular}{|c|c|c|c|}\hline
Jahr&Holzpreis in \euro$/m^3$&absolute Änderungen&prozentuelle Änderungen\\ \hline
1995&76& & \\ \hline
1996&64&-12&-15,79\\ \hline
1997&73&9&14,06\\ \hline
1998&78&5&6,85\\ \hline
1999&79&1&1,28\\ \hline
2000&73&-6&-7,59\\ \hline
2001&73&0&0\\ \hline
2002&74&1&1,37\\ \hline
2003&66&-8&-10,81\\ \hline
2004&68&2&3,03\\ \hline
2005&69&1&1,47\\ \hline
2006&76&7&10,14\\ \hline
2007&76&0&0\\ \hline
2008&69&-7&-9,21\\ \hline
2009&70&1&1,45\\ \hline
2010&89&19&27,14\\ \hline
2011&94&5&5,62\\ \hline
\end{tabular}\leer

Im Zeitraum $[2009; 2010]$ ändert sich der Holzpreis prozentuell am stärksten.
	\end{enumerate}}
		\end{langesbeispiel}