\section{43 - MAT - AN 1.1, FA 2.2, FA 2.5, AN 1.3, WS 2.2, WS 2.3 - Verkehrsunfälle - Matura 2013/14 2. Nebentermin}

\begin{langesbeispiel} \item[0] %PUNKTE DES BEISPIELS
				 Die Verkehrsunfallstatistik in Österreich umfasst grundsätzlich alle Unfälle, die sich auf Österreichs Straßen mit öffentlichem Verkehr ereignen und bei denen Personen verletzt oder getötet werden.
				
Die bei Straßenverkehrsunfällen Verletzten und Getöteten werden unter dem Begriff Verunglückte zusammengefasst.

				Einige der erhobenen Daten werden nachstehend in einer Tabelle und in zwei Grafiken angeführt.
				
				\begin{center}
				\begin{tabular}{|c|>{\centering\arraybackslash}p{5cm}|>{\centering\arraybackslash}p{4cm}|}\hline
				Jahr&Anzahl der Verkehrsunfälle mit Personenschaden&Kraftfahrzeugbestand zu Jahresende\\ \hline
				1961&42\,653&1\,426\,043\\ \hline
				1971&52\,763&2\,336\,520\\ \hline
				1981&46\,690&3\,494\,065\\ \hline
				1991&46\,013&4\,341\,042\\ \hline
				2001&43\,073&5\,684\,244\\ \hline
				2011&35\,129&6\,195\,207\\ \hline
				\end{tabular}
				\end{center}
				
	\begin{footnotesize}\begin{singlespace}
	\begin{center}
	Anzahl der bei Verkehrsunfällen Getöteten
	\end{center}
	\end{singlespace}\end{footnotesize}
	
	
	\begin{center}
	\resizebox{0.9\linewidth}{!}{\psset{xunit=1.0cm,yunit=0.002cm,algebraic=true,dimen=middle,dotstyle=o,dotsize=4pt 0,linewidth=0.8pt,arrowsize=3pt 2,arrowinset=0.25}
	\begin{pspicture*}(-1.8981818181818189,-396.9696969696982)(11.12,3336.36363636364)
\multips(0,0)(0,500.0){8}{\psline[linestyle=dashed,linecap=1,dash=1.5pt 1.5pt,linewidth=0.4pt,linecolor=lightgray]{c-c}(0,0)(11.12,0)}
\multips(0,0)(100.0,0){1}{\psline[linestyle=dashed,linecap=1,dash=1.5pt 1.5pt,linewidth=0.4pt,linecolor=lightgray]{c-c}(0,0)(0,3336.36363636364)}
\psaxes[labelFontSize=\scriptstyle,xAxis=true,yAxis=true,labels=y,Dx=2.,Dy=500.,ticksize=-2pt 0,subticks=2]{->}(0,0)(0.,0.)(11.12,3336.36363636364)
\psline[linewidth=1.2pt](0.,1640.)(2.,2782.)
\psline[linewidth=1.2pt](2.,2782.)(4.,1898.)
\psline[linewidth=1.2pt](4.,1898.)(6.,1551.)
\psline[linewidth=1.2pt](6.,1551.)(8.,958.)
\psline[linewidth=1.2pt](8.,958.)(10.,523.)
\rput[tl](-1.5163636363636368,1718.1818181818198){$\rotatebox{90}{\text{Anzahl}}$}
\rput[tl](10.19272727272728,230){Jahr}
\begin{scriptsize}
\rput[tl](0.2,1650){$1\,640$}
\rput[tl](2.2,2850){$2\,782$}
\rput[tl](4.2,2008){$1\,898$}
\rput[tl](6.2,1630){$1\,551$}
\rput[tl](8.2,1050){958}
\rput[tl](10.2,640){523}
\rput[tl](-0.3,-80){1961}
\rput[tl](1.7,-80){1971}
\rput[tl](3.7,-80){1981}
\rput[tl](5.7,-80){1991}
\rput[tl](7.7,-80){2001}
\rput[tl](9.7,-80){2011}
\psdots[dotstyle=x](0.,1640.)
\psdots[dotstyle=x](2.,2782.)
\psdots[dotstyle=x](4.,1898.)
\psdots[dotstyle=x](6.,1551.)
\psdots[dotstyle=x](8.,958.)
\psdots[dotstyle=x](10.,523.)
\end{scriptsize}
\end{pspicture*}}
\end{center}
\newpage

	\begin{footnotesize}\begin{singlespace}
	\begin{center}
	Prozentueller Anteil der Getöteten an der Gesamtzahl der bei Verkehrsunfällen verunglückten Personen
	\end{center}
	\end{singlespace}\end{footnotesize}
	
	\begin{center}\resizebox{0.8\linewidth}{!}{\psset{xunit=2.0cm,yunit=2.0cm,algebraic=true,dimen=middle,dotstyle=o,dotsize=4pt 0,linewidth=0.8pt,arrowsize=3pt 2,arrowinset=0.25}
\begin{pspicture*}(-1,-0.8328656126482282)(7.2,4.423853754940712)
\multips(0,0)(0,0.5){11}{\psline[linestyle=dashed,linecap=1,dash=1.5pt 1.5pt,linewidth=0.4pt,linecolor=lightgray]{c-c}(0,0)(7.7927272727272765,0)}
\multips(0,0)(100.0,0){1}{\psline[linestyle=dashed,linecap=1,dash=1.5pt 1.5pt,linewidth=0.4pt,linecolor=lightgray]{c-c}(0,0)(0,4.423853754940712)}
\psaxes[labelFontSize=\scriptstyle,xAxis=true,yAxis=true,labels=y,Dx=1.,Dy=0.5,ticksize=-2pt 0,subticks=2]{->}(0,0)(0.,0.)(7.2,4.423853754940712)
\pspolygon[linewidth=1.2pt,fillcolor=black,fillstyle=solid,opacity=0.6](0.5,0.)(1.,0.)(1.,2.8)(0.5,2.8)
\pspolygon[linewidth=1.2pt,fillcolor=black,fillstyle=solid,opacity=0.6](1.5,0.)(2.,0.)(2.,3.7)(1.5,3.7)
\pspolygon[linewidth=1.2pt,fillcolor=black,fillstyle=solid,opacity=0.6](2.5,0.)(3.,0.)(3.,3.)(2.5,3.)
\pspolygon[linewidth=1.2pt,fillcolor=black,fillstyle=solid,opacity=0.6](3.5,0.)(4.,0.)(4.,2.5)(3.5,2.5)
\pspolygon[linewidth=1.2pt,fillcolor=black,fillstyle=solid,opacity=0.6](4.5,0.)(5.,0.)(5.,1.7)(4.5,1.7)
\pspolygon[linewidth=1.2pt,fillcolor=black,fillstyle=solid,opacity=0.6](5.5,0.)(6.,0.)(6.,1.1)(5.5,1.1)
\rput[tl](6.610909090909094,0.2){Jahr}
\rput[tl](-0.7,3){$\rotatebox{90}{\text{Anteil (in Prozent)}}$}
\begin{scriptsize}
\rput[tl](0.65,2.95){2,8}
\rput[tl](1.65,3.85){3,7}
\rput[tl](2.65,3.15){3,0}
\rput[tl](3.65,2.65){2,5}
\rput[tl](4.65,1.85){1,7}
\rput[tl](5.65,1.25){1,1}
\rput[tl](0.6,-0.1){1961}
\rput[tl](1.6,-0.1){1971}
\rput[tl](2.6,-0.1){1981}
\rput[tl](3.6,-0.1){1991}
\rput[tl](4.6,-0.1){2001}
\rput[tl](5.6,-0.1){2011}
\end{scriptsize}
\end{pspicture*}}\end{center}
	
\subsection{Aufgabenstellung:}
\begin{enumerate}
	\item Entnimm der entsprechenden Grafik, in welchem Zeitintervall die absolute und die relative Abnahme (in Prozent) der bei Verkehrsunfällen getöteten Personen jeweils am größten waren, und gib die entsprechenden Werte an!
	
	Im vorliegenden Fall fand die größte relative Abnahme der Anzahl der bei Verkehrsunfällen Getöteten in einem anderen Zeitintervall statt als die größte absolute Abnahme. Gib eine mathematische Begründung an, warum die größte relative Abnahme und die größte absolute Abnahme einer Größe oder eines Prozesses nicht im gleichen Zeitintervall stattfinden müssen!

\item Die Entwicklung des prozentuellen Anteils der Getöteten gemessen an der Gesamtzahl der bei Verkehrsunfällen verunglückten Personen kann für den Zeitraum von Beginn des Jahres 1971 bis Ende 2011 durch eine lineare Funktion $f$ angenähert werden, wobei die Variable $t$ die Anzahl der seit Ende 1970 vergangenen Jahre bezeichnet. 

Ermittle eine Gleichung dieser Funktion $f$ auf Basis der Daten aus der entsprechenden Grafik im Zeitraum von Beginn des Jahres 1971 bis Ende 2011!

Gib den theoretisch größtmöglichen Zeitraum an, für den diese Funktion $f$ ein unter der Annahme eines gleichbleibenden Trends geeignetes Modell darstellt!

\item Im Jahr 1976 wurde in Österreich die Gurtenpflicht eingeführt. Seit diesem Zeitpunkt ist man dazu verpflichtet, auf den vorderen Sitzen eines PKW oder Kombis den Sicherheitsgurt anzulegen. Durch die Einführung der Gurtenpflicht kam es zu einer deutlichen Abschwächung der Unfallfolgen.

Berechne auf Basis der Tabellenwerte für die Jahre 1971 und 1981 die durchschnittliche jährliche Abnahme der Anzahl der Unfälle mit Personenschaden! 

Ein "`Gurtenmuffel"' behauptet, dass es auch schon vor der Einführung der Gurtenpflicht im Zeitraum zwischen 1961 und 1971 zu einer relativen Abnahme der Verkehrsunfälle mit  Personenschaden kam. Ermittle mithilfe des vorhandenen Datenmaterials Zahlen, die seine Aussage untermauern, und präzisieren Sie diese Aussage!

	\item Die nachstehende Tabelle enthält Daten über Verunglückte im Jahr 2001.
	
	\begin{center}
		\begin{tabular}{|l|c|c|}\hline
		Verkehrsart&Anzahl der Verletzten&Anzahl der Getöteten\\ \hline
		einspuriges KFZ&8\,605&85\\ \hline
		PKW&24\,853&290\\ \hline
		sonstige&11\,567&148\\ \hline
		\end{tabular}
	\end{center}
	
	Jemand ist im Jahr 2011 bei einem Verkehrsunfall verunglückt.
	
	\fbox{A} Gib die relative Häufigkeit als Schätzwert der Wahrscheinlichkeit, dass diese Person mit einem einspurigen KFZ oder einem PKW unterwegs war und den Unfall nicht überlebt hat, an!
	
	Interpretiere die mit $\frac{24\,853}{25\,143}\approx 0,99$ angegebene Wahrscheinlichkeit im vorliegenden Zusammenhang!
						\end{enumerate}\leer
				
\antwort{
\begin{enumerate}
	\item \subsection{Lösungserwartung:} 
	
		Die größte absolute Abnahme fand im Zeitintervall von 1971 bis 1981 statt (-884), die größte relative Abnahme war in den Jahren von 2001 bis 2011 (-0,454 bzw. -45,4\,\%).
		
		Da für die Berechnung der relativen Abnahme einer Größe auch der Bezugswert entscheidend ist, müssen größte absolute Abnahme und größte relative Abnahme einer Größe oder eines Prozesses nicht im gleichen Zeitintervall stattfinden.
	 	
	\subsection{Lösungsschlüssel:}
	\begin{itemize}
		\item  Ein Punkt wird für die korrekten Zeitintervalle und die richtigen Abnahmewerte vergeben. Toleranzintervall für relative Abnahme: $[-0,46; -0,45]$ bzw. $[-46\,\%; -45\,\%]$; die Vorzeichen müssen nicht angegeben sein
		\item   Ein Punkt wird für eine (sinngemäß) richtige verbale Begründung vergeben. Dabei kann die Begründung auch anhand konkreter Zahlen erfolgen.
	\end{itemize}
	
	\item \subsection{Lösungserwartung:}
			
		$f(t)=-0,065t+3,7$
		
		Diese Funktion kann höchstens 57 Jahre, also bis zum Beginn des Jahres 2028, zur Modellbildung herangezogen werden.
		
	\subsection{Lösungsschlüssel:}
	
\begin{itemize}
	\item   Ein Punkt wird für die Angabe eines korrekten Funktionsterms vergeben. (Der Punkt kann auch vergeben werden, wenn eine andere Variable als $t$ verwendet wird.)  Toleranzintervall für die ersten Parameter: $[-0,08; -0,05]$.
	\item   Ein Punkt wird für die Angabe der entsprechenden Zeitspanne und/oder des entsprechenden Jahres vergeben. Toleranzintervalle: $[51 \text{ Jahre}; 70 \text{ Jahre}]$, $[2022; 2042]$.
 
\end{itemize}

\item \subsection{Lösungserwartung:}
			Die Anzahl der Unfälle mit Personenschäden nahm durchschnittlich um 607,3 pro Jahr ab.
			
			Anzahl der Unfälle mit Personenschaden pro tausend KFZ:
			
			\begin{itemize}
				\item 1961: 30 (berechneter Wert liegt bei $\approx 29,9$)
				\item 1971: 23 (berechneter Wert liegt bei $\approx 22,5$)
			\end{itemize}
			
			Bezogen auf die Anzahl der zugelassenen KFZ hat die Anzahl der Unfälle mit Personenschaden also tatsächlich abgenommen.
		
	\subsection{Lösungsschlüssel:}
	
\begin{itemize}
	\item   Ein Punkt wird für die korrekte Angabe der durchschnittlichen jährlichen Abnahme vergeben. Toleranzintervall: $[600; 610]$.
	\item   Ein Punkt wird für das Heranziehen des entsprechenden Datenmaterials und eine korrekte Berechnung vergeben. Die Aussage kann auch anhand der relativen Werte präzisiert werden. 
\end{itemize}

\item \subsection{Lösungserwartung:}
			\begin{tabular}{|l|c|c|c|}\hline
			Verkehrsart&Anzahl der Verletzten&Anzahl der Getöteten&\cellcolor[gray]{0.9}Summe\\ \hline
			einspuriges KFZ&8\,605&85&\cellcolor[gray]{0.9}8\,690\\ \hline
			PKW&24\,853&290&\cellcolor[gray]{0.9}25\,143\\ \hline
			sonstiges&11\,567&148&\cellcolor[gray]{0.9}11\,715\\ \hline
			\cellcolor[gray]{0.9}Summe&\cellcolor[gray]{0.9}45\,025&\cellcolor[gray]{0.9}523&\cellcolor[gray]{0.9}45\,548\\ \hline			
			\end{tabular}
			
			$\frac{(85+290)}{45\,548}\approx 0,008$
			
			Die gesuchte Wahrscheinlichkeit beträgt ca. 0,8\,\%.
			
			Die Wahrscheinlichkeit, den Unfall zu überleben, wenn man mit einem PKW verunglückt, beträgt 99\,\%.
	\subsection{Lösungsschlüssel:}
	
\begin{itemize}
	\item Ein Ausgleichspunkt wird für die richtige Angabe der Wahrscheinlichkeit vergeben.  Toleranzintervall: $[0,008; 0,0083]$ bzw. $[0,8\,\%; 0,83\,\%]$.
	\item    Ein Punkt wird für eine (sinngemäß) korrekte Interpretation vergeben.
\end{itemize}
\end{enumerate}}
		\end{langesbeispiel}