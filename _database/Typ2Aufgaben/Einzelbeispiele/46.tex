\section{46 - MAT - FA 5.5, AN 1.3, AN 1.4 - Altersbestimmung - Matura 2014/15 Haupttermin}

\begin{langesbeispiel} \item[0] %PUNKTE DES BEISPIELS
				 Die Radiokohlenstoffdatierung, auch $^{14}$C-Methode genannt, ist ein Verfahren zur Altersbestimmung von kohlenstoffhaltigen Materialien. Das Verfahren beruht darauf, dass in abgestorbenen Organismen die Menge an gebundenen radioaktiven $^{14}$C-Atomen gemäß dem Zerfallsgesetz exponentiell abnimmt, während der Anteil an $^{12}$C-Atomen gleich bleibt. Lebende Organismen sind von diesem Effekt nicht betroffen, da sie ständig neuen Kohlenstoff aus der Umwelt aufnehmen, sodass der $^{14}$C-Anteil nahezu konstant bleibt und somit auch das Verhältnis zwischen $^{14}$C und $^{12}$C.
				
			Die Anzahl der noch vorhandenen $^{14}$C-Atome in einem abgestorbenen Organismus wird durch die Funktion $N$ beschrieben. Für die Anzahl $N(t)$ der $^{14}$C-Atome $t$ Jahr nach dem Tod des Organismus gilt daher näherungsweise die Gleichung $N(t)=N_0\cdot e^{-\lambda\cdot t}$, wobei $N_0$ die Anzahl der $^{14}$C-Atome zum Zeitpunkt $t=0$ angibt und die Zerfallskonstante für $^{14}$C den Wert $\lambda=1,21\cdot 10^{-4}$ pro Jahr hat.
			
			Eine frische Probe enthält pro Billion ($10^{12}$) Kohlenstoffatomen nur ein $^{14}$C-Atom. Die Nachweisgrenze von $^{14}$C liegt bei einem Atom pro Billiarde ($10^{15}$)Kohlenstoffatomen (also einem Tausendstel der frischen Probe).


\subsection{Aufgabenstellung:}
\begin{enumerate}
	\item \fbox{A} Berechne die Halbwertszeit von $^{14}$C!
	
	Zeige, dass nach zehn Halbwertszeiten die Nachweisgrenze von $^{14}$C unterschritten ist!

\item  Im Jahr 1991 wurde in den Ötztaler Alpen von Wanderern die Gletschermumie "`Ötzi"' entdeckt. Die $^{14}$C-Methode ergab, dass bereits $47\,\%\,\pm\,0,5\,\%$ der ursprünglich vorhandenen $^{14}$C-Atome zerfallen waren (d. h., das Messverfahren hat einen Fehler von $\pm\,0,5\,\%$ der in der frischen Probe vorhandenen Anzahl an $^{14}$C-Atomen). 

Berechne ein Intervall für das Alter der Gletschermumie zum Zeitpunkt ihres Auffindens!

Angenommen, Ötzi wäre nicht im Jahr $t_1 = 1991$, sondern zu einem späteren Zeitpunkt $t_2$ gefunden worden. Gib an, welche Auswirkung auf die Breite des für das Alter der Gletschermumie ermittelten Intervalls dies hat (den gleichen Messfehler vorausgesetzt)! Begründe deine Aussage anhand der unten abgebildeten Grafik!
 
\begin{center}
\resizebox{0.8\linewidth}{!}{\psset{xunit=1.0cm,yunit=1.0cm,algebraic=true,dimen=middle,dotstyle=o,dotsize=4pt 0,linewidth=0.8pt,arrowsize=3pt 2,arrowinset=0.25}
\begin{pspicture*}(-0.62,-0.88)(11.74,7.88)
\psaxes[labelFontSize=\scriptstyle,xAxis=true,yAxis=true,labels=none,Dx=1.,Dy=1.,ticksize=0pt 0,subticks=0]{->}(0,0)(0.,0.)(11.74,7.88)
\psplot[linewidth=1.2pt,plotpoints=200]{0}{11.740000000000004}{6.06*2.7182^(-0.36*x)}
\psline[linewidth=1.2pt,linestyle=dashed,dash=2pt 2pt](2.,3.)(2.,0.)
\psline[linewidth=1.2pt,linestyle=dashed,dash=2pt 2pt](5.,1.)(5.,0.)
\rput[tl](0.15,7.4){N(t)}
\rput[tl](10.9,-0.1){t}
\rput[tl](1.36,4.56){N}
\rput[tl](1.96,-0.1){$t_1$}
\rput[tl](4.9,-0.1){$t_2$}
\end{pspicture*}}
\end{center}

\item $N(t)$ beschreibt die Anzahl der $^{14}$C-Atome zum Zeitpunkt $t$.

Interpretiere $N'(t)$ im Hinblick auf den radioaktiven Zerfallsprozess!\leer

Nach den Gesetzmäßigkeiten des radioaktiven Zerfalls zerfällt pro Zeiteinheit ein konstanter Prozentsatz $p$ der vorhandenen Menge an $^{14}$C-Atomen.

Welche der folgenden Differenzengleichungen beschreibt diese Gesetzmäßigkeit? Kreuze die zutreffende Differenzengleichung an!

\multiplechoice[6]{  %Anzahl der Antwortmoeglichkeiten, Standard: 5
				L1={$N(t+1)-N(t)=p$},   %1. Antwortmoeglichkeit 
				L2={$N(t+1)-N(t)=-p$},   %2. Antwortmoeglichkeit
				L3={$N(t+1)-N(t)=p\cdot t$},   %3. Antwortmoeglichkeit
				L4={$N(t+1)-N(t)=-p\cdot t$},   %4. Antwortmoeglichkeit
				L5={$N(t+1)-N(t)=p\cdot N(t)$},	 %5. Antwortmoeglichkeit
				L6={$N(t+1)-N(t)=-p\cdot N(t)$},	 %6. Antwortmoeglichkeit
				L7={},	 %7. Antwortmoeglichkeit
				L8={},	 %8. Antwortmoeglichkeit
				L9={},	 %9. Antwortmoeglichkeit
				%% LOESUNG: %%
				A1=6,  % 1. Antwort
				A2=0,	 % 2. Antwort
				A3=0,  % 3. Antwort
				A4=0,  % 4. Antwort
				A5=0,  % 5. Antwort
				}
						\end{enumerate}\leer
				
\antwort{
\begin{enumerate}
	\item \subsection{Lösungserwartung:} 
	
		$\tau=\frac{\ln(2)}{\lambda}\approx 5\,728$
		
		Die Halbwertszeit von $^{14}$C beträgt ca. 5\,728 Jahre.
		
		Mögliche Überprüfungen:
		
		$\left(\frac{1}{2}\right)^10\approx 0,00098<\frac{1}{1\,000}$
		
		oder:
		
		$N_0\cdot e^{-\lambda\cdot 5\,728\cdot 10}\approx 0,00098\cdot N_0<\frac{N_0}{1\,000}$
		
		Das bedeutet, dass die Nachweisgrenze von $^{14}$C nach 10 Halbwertszeiten unterschritten ist.
	 	
	\subsection{Lösungsschlüssel:}
	\begin{itemize}
		\item  Ein Ausgleichspunkt für die richtige Lösung, wobei die Einheit \textit{Jahre} nicht angeführt werden muss.
		
		Toleranzintervall: $[5\,727;5\,730]$
		\item  Ein Punkt für einen korrekten Nachweis. Jeder korrekte Nachweis, der zeigt, dass nach 10 Halbwertszeiten die Nachweisgrenze von $^{14}$C unterschritten ist, ist ebenfalls als richtig zu werten.
	\end{itemize}
	
	\item \subsection{Lösungserwartung:}
			
		$0,535\cdot N_0=N_0\cdot e^{-\lambda\cdot t_1} \Rightarrow t_1=\frac{\ln(0,535)}{-\lambda}\approx 5\,169$
		
		$0,525\cdot N_0=N_0\cdot e^{-\lambda\cdot t_2} \Rightarrow t_2=\frac{\ln(0,525)}{-\lambda}\approx 5\,325$
		
		Das Alter der Mumie (in Jahren) lag zum Zeitpunkt ihres Auffindens im Intervall $[5\,169; 5\,325]$.
		
		Für große Werte von $t$ wird der Graph der Funktion $N$ flacher, d.h., einem Intervall konstanter Länge auf der $N(t)$-Achse entspricht ein größeres Intervall auf der $t$-Achse.

	\subsection{Lösungsschlüssel:}
	
\begin{itemize}
	\item   Ein punkt für ein korrektes Intervall.
	
	Toleranzintervall für $t_1:[5\,164;5\,174]$, für $t_2:[5\,320;5\,330]$
	\item Ein Punkt für eine (sinngemäß) korrekte Begründung.
\end{itemize}

\item \subsection{Lösungserwartung:}
	Mögliche Interpretationen:
	
	$N'(t)$ beschreibt die (momentane) Zerfallsgeschwindigkeit von $^{14}$C zum Zeitpunkt $t$.
	
	oder:
	
	$N'(t)$ beschreibt die (momentane) Änderungsrate (Abnahmerate) der Anzahl der $^{14}$C-Atome zum Zeitpunkt $t$.
	
	Lösung MC - siehe oben!

	\subsection{Lösungsschlüssel:}
	
\begin{itemize}
	\item Ein Punkt für eine (sinngemäß) korrekte Interpretation.
	\item Ein Punkt ist genau dann zu geben, wenn ausschließlich die laut Lösungserwartung richtige Differenzengleichung angekreuzt ist.
\end{itemize}

\end{enumerate}}
		\end{langesbeispiel}