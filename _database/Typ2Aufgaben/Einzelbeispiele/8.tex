\section{08 - MAT - AG 1.2, FA 1.2, FA 1.4, FA 1.7, FA 3.3 - Haber'sche Regel - BIFIE Aufgabensammlung}

\begin{langesbeispiel} \item[0] %PUNKTE DES BEISPIELS
Abhängig  von  der  Dosis  von  Giftgasen  und  der  Dauer  ihrer  Einwirkung  kann  es  zu  toxischen  Wirkungen bei lebenden Organismen kommen. Diesen Zusammenhang untersuchte der deutsche Chemiker Fritz Haber. Die nach ihm benannte Haber'sche Regel $c\cdot t=W$ (mit $W=$ konstant) beschreibt den Zusammenhang zwischen toxischen Wirkungen $W$ (in mg$\cdot$min$\cdot L^{-1}$ oder ppm$\cdot$min),  der  Einwirkzeit $t$ (in  min)  der  Verabreichung  und  der  Wirkkonzentration $c$ (in ppm oder mg$\cdot L^{-1}$) eines Giftstoffes.  
				
Die  toxische  Wirkung  kann  eine  Erkrankung  (beispielsweise  Krebs)  hervorrufen  oder  den  Tod  des diesem Gift ausgesetzten Lebewesens bedeuten. Nicht am Erbgut angreifende Gifte zeigenerst dann  eine  Wirkung $W$,  wenn  eine  für  das  Gift  spezifische  Konzentration  (Schwellenkonzentration $e$) erreicht  wird. Zum  Beispiel  hat  Kohlenmonoxid  keinen  schädlichen  Effekt,  wenn seine Konzentration unter einem Wert von $5\,$ppm liegt. Für Gifte mit einer Schwellenkonzentration $e$ wird die Haber'sche Regel abgewandelt dargestellt: $(c-e)\cdot t=W$\\
(mit $W=$ konstant). %Aufgabentext

\begin{aufgabenstellung}
\item Die Haber'sche Regel $c\cdot t=W$ (mit $W=$ konstant) kann als Funktion $c$ in Abhängigkeit von der Variablen $t$ geschrieben werden.%Aufgabentext

\Subitem{Kreuze die zutreffende Aussage an!
	
	\multiplechoice[6]{  %Anzahl der Antwortmoeglichkeiten, Standard: 5
					L1={Bei der Funktion $c$ handelt es sich um eine lineare Funktion $f$ vom Typ $f(x)=k\cdot x+d$ mit $k,d\in\mathbb{R}$.},   %1. Antwortmoeglichkeit 
					L2={Bei der Funktion $c$ handelt es sich um eine Potenzfunktion $f$ vom Typ $f(x)=a\cdot x^z$ mit $z\in\mathbb{Z}$.},   %2. Antwortmoeglichkeit
					L3={Bei der Funktion $c$ handelt es sich um eine Potenzfunktion $f$ vom Typ $f(x)=a\cdot x^n+b$ mit $a,b\in\mathbb{R}, n\in\mathbb{N}$.},   %3. Antwortmoeglichkeit
					L4={Bei der Funktion $c$ handelt es sich um eine Polynomfunktion $f$ vom Typ $f(x)=\sum^n_{i=0}{a_i\cdot x^i}$ mit $a\in\mathbb{R}, n\in\mathbb{N}$.},   %4. Antwortmoeglichkeit
					L5={Bei der Funktion $c$ handelt es sich um eine Exponentialfunktion $f$ vom Typ $f(x)=a\cdot b^x$ bzw. $f(x)=a\cdot e^{\lambda\cdot x}$ mit $a,b\in\mathbb{R^+}, \lambda\in\mathbb{R}$.},	 %5. Antwortmoeglichkeit
					L6={Bei der Funktion $c$ handelt es sich um eine konstante Funktion $f$ vom Typ $f(x)=a$ mit $a\in\mathbb{R}$.},	 %6. Antwortmoeglichkeit
					L7={},	 %7. Antwortmoeglichkeit
					L8={},	 %8. Antwortmoeglichkeit
					L9={},	 %9. Antwortmoeglichkeit
					%% LOESUNG: %%
					A1=2,  % 1. Antwort
					A2=0,	 % 2. Antwort
					A3=0,  % 3. Antwort
					A4=0,  % 4. Antwort
					A5=0,  % 5. Antwort
					}} %Unterpunkt1
					
Phosgen ist ein sehr giftiges Gas. Ein Lebewesen wird für eine Zeitdauer von 10 Minuten diesem Giftgas in einer Wirkkonzentration von $0,3\,$mg/L ausgesetzt.

\Subitem{Gib jene Wirkkonzentration in mg/L an, mit der in nur einer Minute die gleiche toxische Wirkung erreicht wird.} %Unterpunkt2

\item %Aufgabentext

\Subitem{Stelle den Zusammenhang zwischen der Wirkkonzentration $c$ (in mg/L) und der Einwirkzeit $t$ (in min) für eine Wirkung von $W=2\,\text{mg}\cdot\text{min}\cdot L^{-1}$ grafisch dar!\vspace{0,2cm}

\psset{xunit=1.0cm,yunit=1.0cm,algebraic=true,dimen=middle,dotstyle=o,dotsize=5pt 0,linewidth=0.8pt,arrowsize=3pt 2,arrowinset=0.25}
\begin{pspicture*}(-0.72,-1.)(13.12,8.76)
\multips(0,0)(0,1.0){10}{\psline[linestyle=dashed,linecap=1,dash=1.5pt 1.5pt,linewidth=0.4pt,linecolor=gray]{c-c}(0,0)(13.12,0)}
\multips(0,0)(1.0,0){14}{\psline[linestyle=dashed,linecap=1,dash=1.5pt 1.5pt,linewidth=0.4pt,linecolor=gray]{c-c}(0,0)(0,8.76)}
\psaxes[labelFontSize=\scriptstyle,xAxis=true,yAxis=true,Dx=1.,Dy=1.,ticksize=-2pt 0,subticks=0]{->}(0,0)(0.,0.)(13.12,8.76)
\begin{scriptsize}
\rput[tl](11.9,0.5){$t$ in $min$}
\rput[tl](0.5,8.5){$c(t)$ in $mg/L$}
\end{scriptsize}
\antwort{\psplot[linewidth=2.pt,plotpoints=200]{0.05}{13.119999999999985}{2.0/x}}
\end{pspicture*}} %Unterpunkt1
\Subitem{\lueckentext[-0.09]{
									text={Die Größen $c$ und $t$ in der Haber'sche Regel $c\cdot t=W$ (mit $W=$ konstant) sind zueinander \gap, weil \gap dieselbe Wirkung $W$ erzielt wird.}, 	%Lueckentext Luecke=\gap
									L1={direkt proportional}, 		%1.Moeglichkeit links  
									L2={indirekt proportional}, 		%2.Moeglichkeit links
									L3={weder indirekt noch direkt proportional}, 		%3.Moeglichkeit links
									R1={bei Erhöhung der Einwirkzeit $t$ auf das $n-$Fache und Reduktion der Konzentration $c$ auf den $n-$ten Teil}, 		%1.Moeglichkeit rechts 
									R2={bei Erhöhung der Einwirkzeit $t$ auf der $n-$Fache und beliebiger Konzentration $c$}, 		%2.Moeglichkeit rechts
									R3={bei Erhöhung der Einwirkzeit $t$ auf das $n-$Fache und Erhöhung der Konzentration $c$ auf den $n-$Fache}, 		%3.Moeglichkeit rechts
									%% LOESUNG: %%
									A1=2,   % Antwort links
									A2=1		% Antwort rechts 
									}} %Unterpunkt2

\item Ein nicht näher bezeichnetes Giftgas hat eine natürliche Schwellenkonzentration von $e=5$\,ppm. Bei einer Einwirkzeit von $10$\,min liegt die tödliche Dosis (letale Dosis) bei etwa $c=35$\,ppm. Die nachstehende Abbildung zeigt den Graphen der Konzentration $c$ in Abhängigkeit von der Einwirkzeit $t$ bei der letalen Dosis.
\leer

\psset{xunit=0.34cm,yunit=0.06cm,algebraic=true,dimen=middle,dotstyle=o,dotsize=5pt 0,linewidth=0.8pt,arrowsize=3pt 2,arrowinset=0.25}
\begin{pspicture*}(-1.9943034055727542,-7.349333333333778)(37.376718266253854,143.20533333333697)
\multips(0,0)(0,10.0){16}{\psline[linestyle=dashed,linecap=1,dash=1.5pt 1.5pt,linewidth=0.4pt,linecolor=gray]{c-c}(0,0)(39.976718266253854,0)}
\multips(0,0)(1.0,0){42}{\psline[linestyle=dashed,linecap=1,dash=1.5pt 1.5pt,linewidth=0.4pt,linecolor=gray]{c-c}(0,0)(0,143.20533333333697)}
\psaxes[labelFontSize=\scriptstyle,xAxis=true,yAxis=true,Dx=2.,Dy=10.,ticksize=-2pt 0,subticks=0]{->}(0,0)(0.,0.)(39.976718266253854,143.20533333333697)
\rput[tl](2.4130030959752317,143.7831111111145){$c(t)$ in ppm}
\rput[tl](32.01900928792568,6.4648888888887495){$t$ in min}
\psplot[linewidth=1.6pt,plotpoints=200]{5.439009326775952E-8}{39.976718266253854}{275.80534701715675*x^(-0.8927892607143718)}
\end{pspicture*}%Aufgabentext

\Subitem{Lies aus dem Graphen die letale Dosis des angegebenen Giftes für einen Menschen bei einer Einwirkzeit von 20 Minuten ab.} %Unterpunkt1
\Subitem{Interpretiere das Ergebnis im Vergleich zu den Angabewerten.} %Unterpunkt2

\item Die abgewandelte Haber'sche Regel $(c-e)\cdot t=W$ (mit $W=$ konstant) kann als eine Funktion $c$ in Abhängigkeit von der Einwirkzeit $t$ geschrieben werden.%Aufgabentext

\Subitem{Kreuze die zutreffende(n) Aussage(n) an!

\multiplechoice[5]{  %Anzahl der Antwortmoeglichkeiten, Standard: 5
				L1={Der Wert $e$ ist im Funktionsterm der Funktion $c$ eine additive Konstante.},   %1. Antwortmoeglichkeit 
				L2={Der Wert $e$ ist im Funktionsterm der Funktion $c$ eine multiplikative Konstante.},   %2. Antwortmoeglichkeit
				L3={Der Wert $e$ ist im Funktionsterm der Funktion $c$ eine von $t$ abhängige Variable.},   %3. Antwortmoeglichkeit
				L4={Der Wert $e$ ist im Funktionsterm der Funktion $c$ nicht mehr vorhanden, weil der Wert $e$ bei der Umformung wegfällt.},   %4. Antwortmoeglichkeit
				L5={Der Wert $e$ ist im Funktionsterm der Funktion $c$ immer gleich groß wie $W$.},	 %5. Antwortmoeglichkeit
				L6={},	 %6. Antwortmoeglichkeit
				L7={},	 %7. Antwortmoeglichkeit
				L8={},	 %8. Antwortmoeglichkeit
				L9={},	 %9. Antwortmoeglichkeit
				%% LOESUNG: %%
				A1=1,  % 1. Antwort
				A2=0,	 % 2. Antwort
				A3=0,  % 3. Antwort
				A4=0,  % 4. Antwort
				A5=0,  % 5. Antwort
				}} %Unterpunkt1
				
Die Haber'sche Regel ohne Schwellenkonzentration lautet $c\cdot t=W$, die Form mit Schwellenkonzentration $e$ und mit derselben biologischen Wirkungskonstante $W$ lautet $(c-e)\cdot t=W$.				
				
\Subitem{Woran erkennt man an der graphischen Darstellung beider Funktionen $c$ mit $c(t)$, um welche es sich handelt? Begründe.} %Unterpunkt2

\end{aufgabenstellung}

\begin{loesung}
\item \subsection{Lösungserwartung:} 

\Subitem{Die richtige Antwort ist 2.} %Lösung von Unterpunkt1
\Subitem{$W=c\cdot t=0,3\,\frac{mg}{L}\cdot 10\,min=3\,mg\cdot min\cdot L^{-1}$
	
	$\rightarrow c(t)=\frac{3}{t}\Rightarrow c(1)=3$
	
	Die Konzentration beträgt $3\,mg/L$.} %%Lösung von Unterpunkt2

\setcounter{subitemcounter}{0}
\subsection{Lösungsschlüssel:}
 
\Subitem{Ein Punkt für die richtige Antwort.} %Lösungschlüssel von Unterpunkt1
\Subitem{Ein Punkt für den richtigen Wert. Der Rechengang sowie die Einheiten sind für das Erreichen des Punktes nicht von Bedeutung.} %Lösungschlüssel von Unterpunkt2

\item \subsection{Lösungserwartung:} 

\Subitem{Graph: siehe oben.} %Lösung von Unterpunkt1
\Subitem{Richtige Antworten Lückentext: 2 und 1} %%Lösung von Unterpunkt2

\setcounter{subitemcounter}{0}
\subsection{Lösungsschlüssel:}
 
\Subitem{Ein Punkt für den richtigen Graphen.} %Lösungschlüssel von Unterpunkt1
\Subitem{Ein Punkt für die richtigen Antworten.} %Lösungschlüssel von Unterpunkt2

\item \subsection{Lösungserwartung:} 

\Subitem{Die letale Dosis beträgt für eine Einwirkzeit von 20 Minuten zirka $20$\,ppm.} %Lösung von Unterpunkt1
\Subitem{Verdoppelt sich die Einwirkzeit, so halbiert sich die letale Dosis hier nicht. Durch das Vorhandensein einer Schwellenkonzentration von $5$\,ppm liegt das Ergebnis höher als die Hälfte von $35$\,ppm.} %%Lösung von Unterpunkt2

\setcounter{subitemcounter}{0}
\subsection{Lösungsschlüssel:}
 
\Subitem{Ein Punkt für die korrekte Dosis.} %Lösungschlüssel von Unterpunkt1
\Subitem{Ein Punkt für eine korrekte Interpretation.} %Lösungschlüssel von Unterpunkt2

\item \subsection{Lösungserwartung:} 

\Subitem{Die richtige MC-Antwort ist 1.} %Lösung von Unterpunkt1
\Subitem{Der Wert $e$ ist im Funktionsterm der Funktion $c$ eine additive Konstante, dadurch wird der Graph der Funktion $c(t)=\frac{W}{t}$ entlang der y-Achse verschoben.
	
	Die Haber'sche Regel ohne Schwellenkonzentration lautet $c\cdot t=W$ und hat als Funktion $c$ mit $c(t)$ gesehen die beiden Achsen als Asymptoten.
	
	Die Haber'sche Regel mit Schwellenkonzentration $(c-e)\cdot t=W$ mit derselben biologischen Wirkungskonstante $W$ besitzt statt der x-Achse an der Stelle $y=e$ eine Asymptote. Der Graph der Funktion ist entlang der y-Achse um den Wert $e$ verschoben.} %%Lösung von Unterpunkt2

\setcounter{subitemcounter}{0}
\subsection{Lösungsschlüssel:}
 
\Subitem{Ein Punkt für die richtige MC-Antwort.} %Lösungschlüssel von Unterpunkt1
\Subitem{Ein Punkt für eine richtige Erklärung. Adäquate Antworten sind als richtig zu werten.} %Lösungschlüssel von Unterpunkt2

\end{loesung}

\end{langesbeispiel}