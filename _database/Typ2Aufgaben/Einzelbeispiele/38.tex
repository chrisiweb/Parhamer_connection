\section{38 - MAT - FA 5.3, FA 5.6, FA 5.5 - Bakterienkultur - Matura 2013/14 1. Nebentermin}

\begin{langesbeispiel} \item[0] %PUNKTE DES BEISPIELS
				Eine Petrischale hat die Form eines oben offenen, geraden Drehzylinders geringer Höhe.
				
				In einer Petrischale mit einem Durchmesser von 55 mm wird eine Bakterienkultur gezüchtet. Die von Bakterien bedeckte Fläche $A(t)$ in Abhängigkeit von der Zeit $t$ wird modellhaft durch $A(t)=3\cdot 1,05^t$ beschrieben. Dabei ist die Zeit $t$ in Stunden und die Fläche $A(t)$ in Quadratmillimetern angegeben.
				
\subsection{Aufgabenstellung:}
\begin{enumerate}
	\item \fbox{A} \lueckentext{
					text={Gemäß dem gegebenen Wachstumsprozess bedecken die Bakterien am Beginn eine Fläche von \gap, und diese Fläche nimmt pro Stunde um \gap zu.}, 	%Lueckentext Luecke=\gap
					L1={1,05\,mm$²$}, 		%1.Moeglichkeit links  
					L2={3\,mm$²$}, 		%2.Moeglichkeit links
					L3={$3\cdot 1,05$\,mm$²$}, 		%3.Moeglichkeit links
					R1={1,05\,\%}, 		%1.Moeglichkeit rechts 
					R2={3\,\%}, 		%2.Moeglichkeit rechts
					R3={5\,\%}, 		%3.Moeglichkeit rechts
					%% LOESUNG: %%
					A1=2,   % Antwort links
					A2=3		% Antwort rechts 
					}
					
					Beschreibe, an welche Grenzen das gegebene exponentielle Wachstumsmodell für die von Bakterienkultur bedeckte Fläche stößt!
					
\item Berechne, nach wie vielen Stunden sich die Fläche der Bakterienkultur verdoppelt hat!

Erkläre und begründe mithilfe der durchgeführten Rechnung oder allgemein, welche Auswirkung eine Änderung der anfangs von Bakterien bedeckten Fläche auf die Verdopplungszeit für diese Fläche hat!
						\end{enumerate}\leer
				
\antwort{
\begin{enumerate}
	\item \subsection{Lösungserwartung:} 
	
	Dem vorliegenden exponentiellen Wachstum bzgl. der Fläche der Bakterienkultur wird durch die Größe der Petrischale (ca $2\,376$\,mm$²$) eine Grenze gesetzt.
 
	 	
	\subsection{Lösungsschlüssel:}
	\begin{itemize}
		\item  Ein Ausgleichspunkt ist nur dann zu geben, wenn für beide Lücken ausschließlich der jeweils richtige Satzteil angekreuzt ist.
		\item  Ein Punkt für eine (sinngemäß) korrekte Beschreibung. Korrekt sind alle Antworten, die erläutern, dass kein unbeschränktes Wachstum innerhalb der Petrischale möglich ist.
	\end{itemize}
	
	\item \subsection{Lösungserwartung:}
		Verdopplungzeit: $t=\frac{\ln(2)}{\ln(1,05)}\approx 14,21$
		
		Die Verdoppelungszeit beträgt ca. 14 Stunden.
		
		Begründung:
		
		Bei der Berechnung der Verdoppelungszeit $t$ sieht man, dass das Ergebnis vom Anfangswert $A(0)=3$ unabhängig ist (er fällt durch Gleichungsumformungen weg), z.B.:
		
		$2\cdot 3=3\cdot 1,05^t$\hspace{1cm} $|:3$
		
		$2=1,05^t$ $\Rightarrow$ $t=\frac{\ln(2)}{\ln(1,05)}$ oder:
		
		Wenn in jeder Stunde gleich viele Prozent dazukommen, dann dauert es immer - also unabhängig von der Anfangsmenge - gleich lang, bis 100\,\% dazugekommen sind.
		
		
	\subsection{Lösungsschlüssel:}
	
\begin{itemize}
	\item Ein Punkt für eine korrekte Berechnung. Toleranzintervall: $[14; 15]$.
	\item  Ein Punkt für eine (sinngemäß) korrekte Begründung. Auch andere Berechnungen mit einer Variablen (z. B.: $a$ anstelle von 1,05, $A(0)$ anstelle von $3$) oder selbst anderer Bezeichnung für $A(t)$ (z. B.: $N(t)$), die die Unabhängigkeit von $t$ vom Anfangswert (hier $A(0)$) zeigen, gelten als richtig. 
\end{itemize}
\end{enumerate}}
		\end{langesbeispiel}