\section{99 - AG 2.1, AN 4.3, FA 5.2, FA 5.4, AN 1.2 - Kondensator - Matura 2. NT - 2017/18}

\begin{langesbeispiel} \item[4] %PUNKTE DES BEISPIELS
Ein Kondensator ist ein elektrisches Bauelement, mit dem elektrische Ladung und die daraus resultierende elektrische Energie gespeichert werden kann.\\
Eine einfache Form des Kondensators ist der sogenannte \textit{Plattenkondensator}. Es besteht aus zwei einander gegenüberliegenden elektrisch leitfähigen Flächen, die als \textit{Kondensatorplatten} bezeichnet werden.

Das Verhältnis zwischen der gespeicherten Ladung $Q$ und der an die Kondensatorplatten angelegten (Gleich-)Spannung $U$ wird als Kapazität $C$ bezeichnet.

Es gilt $C=\frac{Q}{U}$, wobei $C$ in der Einheit Farad angegeben wird.

\subsection{Aufgabenstellung:}
\begin{enumerate}
	\item Ein Kondensator mit einer bestimmten Kapazität $C$ wird bis zur Ladungsmenge $Q_1$ aufgeladen, die gemessene Spannung $U(Q_1)$ hat dann den Wert $U_1$.\\
	\fbox{A} Skizziere in der nachstehenden Abbildung die Spannung $U$ beim Ladevorgang am Kondensator in Abhängigkeit von der Ladung $Q$!
	\begin{center}
	\psset{xunit=1.0cm,yunit=1.0cm,algebraic=true,dimen=middle,dotstyle=o,dotsize=5pt 0,linewidth=1.6pt,arrowsize=3pt 2,arrowinset=0.25}
\begin{pspicture*}(-0.74,-0.86)(10.52,6.9)
\multips(0,0)(0,1.0){8}{\psline[linestyle=dashed,linecap=1,dash=1.5pt 1.5pt,linewidth=0.4pt,linecolor=darkgray]{c-c}(0,0)(10.52,0)}
\multips(0,0)(1.0,0){12}{\psline[linestyle=dashed,linecap=1,dash=1.5pt 1.5pt,linewidth=0.4pt,linecolor=darkgray]{c-c}(0,0)(0,6.9)}
\psaxes[labelFontSize=\scriptstyle,xAxis=true,yAxis=true,labels=none,Dx=1.,Dy=1.,ticksize=0pt 0,subticks=0]{->}(0,0)(0.,0.)(10.52,6.9)[$Q$,140] [$U(Q)$,-40]
\rput[tl](-0.28,0.28){0}
\rput[tl](0.06,-0.14){0}
\rput[tl](-0.54,5.14){$U_1$}
\rput[tl](6.9,-0.26){$Q_1$}
\antwort{\psline[linewidth=2.pt](0.,0.)(7.,5.)
\rput[tl](4.26,3.72){$U$}}
\begin{scriptsize}
\psdots[dotstyle=+](0.,5.)
\psdots[dotstyle=+](7.02,0.)
\end{scriptsize}
\end{pspicture*}
	\end{center}
	
	Die in diesem Kondensator gespeicherte Energie $W$ kann mithilfe der Formel $W=\displaystyle\int^{Q_1}_0U(Q)\,\text{d}Q$ berechnet werden.\\
	Gib eine Formel für die gespeicherte Energie $W$ in Abhängigkeit von $U_1$ und $C$ an!
	
	\item Bei einem Ladevorgang kann die Spannung zwischen den Kondensatorplatten als Funktion $U$ in Abhängigkeit von der Zeit $t$ durch $U(t)=U^*\cdot\left(1-e^{\frac{t}{\tau}}\right)$ beschrieben werden. Dabei ist $U^*>0$ die an den Kondensator angelegte Spannung und $\tau>0$ eine für den Ladevorgang charakteristische Konstante. Der Ladevorgang beginnt zum Zeitpunkt $t=0$.
	
	Die Zeit, nach der die Spannung $U(t)$ zwischen den Kondensatorplatten $99\,\%$ der angelegten Spannung $U^*$ beträgt, wird als \textit{Ladezeit} bezeichnet.\\
	Bestimme die Ladezeit eines Kondensators in Abhängigkeit von $\tau$!
	
	Gib eine Formel für die momentane Änderungsrate der Spannung zwischen den Kondensatorplatten in Abhängigkeit von $t$ an und zeige mithilfe dieser Formel, dass die Spannung während des Ladevorgangs ständig steigt!
\end{enumerate}

\antwort{
\begin{enumerate}
\item \subsection{Lösungserwartung:}

Grafik: siehe oben\leer

$W=\displaystyle\int^{Q_1}_0 U(Q)\,\text{d}Q=\frac{1}{2}\cdot U_1\cdot Q_1=\frac{1}{2}\cdot C\cdot U_1^2$

\subsection{Lösungsschlüssel:}
\begin{itemize}
\item Ein Ausgleichspunkt für eine richtige Skizze.
\item Ein Punkt für eine richtige Formel. Äquivalente Formeln sind als richtig zu werten.
\end{itemize}

\item \subsection{Lösungserwartung:}

Mögliche Vorgehensweise:\\
$0,99\cdot U^*=U^*\cdot\left(1-e^{-\frac{t}{\tau}}\right)$\\
$0,01=e^{-\frac{t}{\tau}}$

Ladezeit: $t=-\tau\cdot\ln(0,01)$ bzw. $t=\tau\cdot\ln(100)$\leer

Mögliche Vorgehensweise:\\
$U'(t)=\dfrac{e^{-\frac{t}{\tau}}\cdot U^*}{\tau}$\\
Es gilt: $U^*>0, \tau >0, e^{-\frac{t}{\tau}}>0 \Rightarrow U'(t)>0$ für alle $t\geq 0$.\\
Da $U'(t)>0$ für alle $t\geq 0$ gilt, ist $U$ während des Ladevorgangs streng monoton steigend.

\subsection{Lösungsschlüssel:}
\begin{itemize}
\item Ein Punkt für die richtige Lösung. Äquivalente Schreibweisen der Lösung sind als richtig zu werten.
\item Ein Punkt für eine richtige Formel und eine (sinngemäß) korrekte Begründung. Äquivalente Formeln sind als richtig zu werten.
\end{itemize}

\end{enumerate}}
\end{langesbeispiel}