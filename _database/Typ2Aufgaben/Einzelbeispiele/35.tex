\section{35 - MAT - AN 1.2, AN 2.1, AN 4.2, AN 4.3, FA 2.1, FA 2.2 - Sportwagen - Matura 2013/14 Haupttermin}

\begin{langesbeispiel} \item[0] %PUNKTE DES BEISPIELS
				Ein Sportwagen wird von 0 m/s auf 28 m/s ($\approx$ 100 km/h) in ca. 4 Sekunden beschleunigt. $v(t)$ beschreibt die Geschwindigkeit in Metern/Sekunde während des Beschleunigungsvorganges in Abhängigkeit von der Zeit $t$ in Sekunden. Die Geschwindigkeit lässt sich durch die Funktionsgleichung $v(t)=-0,5t^3+3,75t^2$ angeben.

\subsection{Aufgabenstellung:}
\begin{enumerate}
	\item \fbox{A}  Gib die Funktionsgleichung zur Berechnung der momentanen Beschleunigung $a(t)$ zum Zeitpunkt $t$ an!  Berechne die momentane Beschleunigung zum Zeitpunkt $t=2$!
	
	\item  Gib einen Ausdruck zur Berechnung des in den ersten 4 Sekunden zurückgelegten Weges an! Ermittle diesen Weg $s(4)$ (in Metern)!
	
	\item  Angenommen, dieser Sportwagen beschleunigt - anders als ursprünglich angegeben - gleichmäßig in 4 Sekunden von 0 m/s auf 28 m/s. Nun wird mit $v_1(t)$ die Geschwindigkeit des Sportwagens nach $t$ Sekunden bezeichnet.
	
	Gib an, welcher funktionale Zusammenhang zwischen $v_1$ und $t$ vorliegt! Ermittle die Funktionsgleichung für $v_1$!
						\end{enumerate}\leer
				
\antwort{
\begin{enumerate}
	\item \subsection{Lösungserwartung:} 
	
	$a(t)=v'(t)=-1,5\cdot t^2+7,5\cdot t$\\
	$a(2)=-1,5\cdot 2^2+7,5\cdot 2=0\Rightarrow a(2)=0\,m/s^2$
	
	Auch die Berechnung über den Differenzenquotienten mit korrektem Grenzwertübergang ist zulässig. 	
	\subsection{Lösungsschlüssel:}
	\begin{itemize}
		\item  Ein Ausgleichspunkt, wenn $a(t)$ als 1. Ableitung der Geschwindigkeitsfunktion korrekt bestimmt wurde.
		\item  Ein Punkt für die korrekte Berechnung des Ergebnisses. Sollte $a(t)$ im Ansatz richtig (aber fehlerhaft) aufgestellt worden sein, die Berechnung aber in weiterer Folge korrekt sein, dann ist dieser Punkt zu geben.
	\end{itemize}
	
	\item \subsection{Lösungserwartung:}
		$s(4)=\int^4_0{v(t)}$d$t=\int^4_0{(-0,5\cdot t^3+3,75\cdot t^2)}$d$t$
		
		$s(4)=\int^4_0{(-0,5\cdot t^3+3,75\cdot t^2)}$d$t=(-0,125\cdot t^4+1,25\cdot t^3)\big|^4_0=48\Rightarrow s(4)=48$
		
	\subsection{Lösungsschlüssel:}
	
\begin{itemize}
	\item Ein Punkt, wenn der Ansatz $s(4)=\int^4_0{v(t)}$d$t$ mit dem bestimmten Integral inklusive der richtigen Grenzen vorhanden ist
	\item  Ein Punkt für das richtige Ergebnis. Sollte das bestimmte Integral im Ansatz richtig (aber fehlerhaft) aufgestellt worden sein, die Berechnung aber in weiterer Folge korrekt sein, dann ist dieser Punkt zu geben.
\end{itemize}
	\item \subsection{Lösungserwartung:}
		Es liegt ein linearer funktionaler Zusammenhang vor.
		
		$v_1(t)=\frac{28}{4}\cdot t+0=7\cdot t$
		
	\subsection{Lösungsschlüssel:}
	
\begin{itemize}
	\item Ein Punkt, wenn erkannt wurde, dass ein linearer Zusammenhang vorliegt, und dieser angegeben wurde (entweder textlich oder auch in Form einer Funktionsgleichung). Dieser Punkt ist auch zu geben, wenn zwar ein linearer Zusammenhang erkannt wurde, aber die Funktionsgleichung falsch aufgestellt wurde.
	\item Ein Punkt, wenn die Funktionsgleichung mit den korrekten Parametern aufgestellt wurde.
\end{itemize}
\end{enumerate}}
		\end{langesbeispiel}