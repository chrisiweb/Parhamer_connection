\section{116 - K6 - PWLU - Schalldruckpegel - MatKon}

\begin{langesbeispiel} \item[6] %PUNKTE DES BEISPIELS
Der Schalldruckpegel ist eine logarithmische Größe zur Beschreibung der Stärke eines Schallereignisses. Er gehört zu den Schallfeldgrößen. Häufig wird der Schalldruckpegel, obwohl dann physikalisch nicht eindeutig, auch einfach Schallpegel genannt.

Der Schalldruckpegel $L_p$ beschreibt das logarithmierte Verhältnis des quadrierten Effektivwertes des Schalldrucks (Formelzeichen $\tilde{p}$ mit der Einheit Pa für Pascal) eines Schallereignisses zum Quadrat des Bezugswerts $p_0$. Das Ergebnis wird mit der Hilfsmaßeinheit Dezibel (dB) gekennzeichnet.

$$L_p=20\cdot\log_{10}\left(\dfrac{\tilde{p}}{p_0}\right)\,\text{dB}$$

Der Bezugswert für Luftschall wurde Anfang des 20. Jahrhunderts auf\\ 
$p_0=2\cdot 10^{-5}$\,Pa festgelegt. Für die Angabe eines Schalldruckpegels in Wasser und anderen Medien ist ein Bezugswert von $p_0=1\cdot 10^{-6}$\,Pa festgelegt. Als Pegelgröße kann der Schalldruckpegel sowohl positive (Schalldruck ist größer als Bezugswert) als auch negative (Schalldruck ist kleiner als Bezugswert) Werte annehmen.%Aufgabentext

\begin{aufgabenstellung}
\item  Ab einem Effektivwert des Schalldrucks von $\tilde{p}=20$\,Pa können Gehördschäden bei kurzfristiger Einwirkung entstehen.%Aufgabentext

\ASubitem{Berechne den dabei entstehenden Luftschalldruckpegel $L_p$.} %Unterpunkt1
\Subitem{Um wie viel Prozent größer wäre der Schalldruckpegel (im Vergleich zum eben berechneten Wert) wenn der Effektivwert des Schalldrucks $\tilde{p}=20$\,Pa unter Wasser gemessen worden wäre?} %Unterpunkt2

\item Mit Schmerzschwelle, bezeichnet man in der Akustik und in der Medizin die niedrigste Stärke eines Reizes, der vom Probanden als schmerzhaft empfunden wird. Durchschnittlich liegt die akustische Schmerzwelle bei einem Luftschalldruckpegel von ungefähr 134 Dezibel.%Aufgabentext

\Subitem{Gib eine Formel zur Berechnung des bei der Schmerzwelle erreichten Effektivwert des Schalldrucks ($\tilde{p}$) an. (Verwende dafür den Bezugswert für Luftschall)} %Unterpunkt1
\Subitem{Berechne jenen Effektivwert des Schalldrucks.} %Unterpunkt2

\item Für so genannte binaurale Tonaufnahmen werden Kunstköpfe verwendet. Von einem binauralen Schalldruckpegel spricht man, wenn aus den beiden Schalldruckpegeln des linken und des rechten Ohrs ein Gesamtpegel gebildet wird. Für diese Größe hat sich in der Psychoakustik die Bezeichnung BSPL (binaural sound pressure level) etabliert. Die Bildung des BSPL wird gemäß dem sogenannten 6-dB-Lautheits-Gesetz nach folgender Formel durchgeführt:
	$$\text{BSPL}=6\cdot\log_2\left(2^\frac{L_l}{6}+2^\frac{L_r}{6}\right)\,\text{dB}$$
	In dieser Formel stehen die Größen $L_l$ und $L_r$ für die Luftschalldruckpegel, die am linken bzw. am rechten Kunstkopfohr gemessen werden.%Aufgabentext

\Subitem{Berechne jeweils den Luftschalldruckpegel des linken bzw. des rechten Kunstkopfohrs wenn bei einer Messung für den Schalldruckpegel am linken Kunstkopfohr ein Pegel von 1,9\,Pa und am rechten Kunstkopfohr ein Pegel von 2,3 gemessen worden ist.} %Unterpunkt1
\Subitem{Berechne den entsprechenden BSPL.} %Unterpunkt2

\end{aufgabenstellung}

\begin{loesung}
\item \subsection{Lösungserwartung:} 

\Subitem{$L_p=20\cdot\log_{10}\left(\dfrac{20}{2\cdot 10^{-5}}\right)=120\,\text{dB}$

Der dabei entstehende Schalldruckpegel $L_p$ beträgt 120\,dB} %Lösung von Unterpunkt1
\Subitem{$L_p=20\cdot\log_{10}\left(\dfrac{20}{1\cdot 10^{-6}}\right)=146,02\,\text{dB}$

$\frac{146,02}{120}=1,21683$

Der Schalldruckpegel wäre um 21,68\,\% größer.} %%Lösung von Unterpunkt2

\setcounter{subitemcounter}{0}
\subsection{Lösungsschlüssel:}
 
\Subitem{Ein Punkt für die Berechnung des Schalldruckpegels.} %Lösungschlüssel von Unterpunkt1
\Subitem{Ein Punkt für die richtige Berechnung des Prozentwerts.} %Lösungschlüssel von Unterpunkt2

\item \subsection{Lösungserwartung:} 

\Subitem{$20\cdot\log_{10}\left(\dfrac{\tilde{p}}{2\cdot 10^{-5}}\right)=134$} %Lösung von Unterpunkt1
\Subitem{$20\cdot\log_{10}\left(\dfrac{\tilde{p}}{2\cdot 10^{-5}}\right)=134$\quad $|:20$
	
	$\log_{10}\left(\dfrac{\tilde{p}}{2\cdot 10^{-5}}\right)=6,7$
	
	$\dfrac{\tilde{p}}{2\cdot 10^{-5}}=10^{6,7}$
	
	$\tilde{p}=10^{6,7}\cdot2\cdot 10^{-5}=100,237$} %%Lösung von Unterpunkt2

\setcounter{subitemcounter}{0}
\subsection{Lösungsschlüssel:}
 
\Subitem{Ein Punkt für das richtige Einsetzen in die Formel.} %Lösungschlüssel von Unterpunkt1
\Subitem{Ein Punkt für die richtige Umformung/das richtige Ergebnis.} %Lösungschlüssel von Unterpunkt2

\item \subsection{Lösungserwartung:} 

\Subitem{$L_l=20\cdot\log_{10}\left(\dfrac{1,9}{2\cdot 10^{-5}}\right)=99,554$

$L_l=20\cdot\log_{10}\left(\dfrac{2,3}{2\cdot 10^{-5}}\right)=101,214$} %Lösung von Unterpunkt1
\Subitem{$\text{BSPL}=6\cdot\log_2\left(2^\frac{99,55}{6}+2^\frac{101,21}{6}\right)=6\cdot\log_2\left(98761,88+119639,42\right)=106,4197\,\text{dB}$} %%Lösung von Unterpunkt2

\setcounter{subitemcounter}{0}
\subsection{Lösungsschlüssel:}
 
\Subitem{Ein Punkt für die richtige Berechnung der beiden Schalldruckpegel.} %Lösungschlüssel von Unterpunkt1
\Subitem{Ein Punkt für die richtige Berechnung des BSPL.} %Lösungschlüssel von Unterpunkt2

\end{loesung}

\end{langesbeispiel}