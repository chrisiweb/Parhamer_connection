\section{25 - MAT - AN 4.3, AN 1.3, AN 3.3, FA 2.3, AG 2.3, AN 4.3 - Bewegung eines Fahrzeugs - BIFIE Aufgabensammlung}

\begin{langesbeispiel} \item[0] %PUNKTE DES BEISPIELS
				Im Folgenden wird die Bewegung eines Fahrzeugs beschrieben:
				
In den ersten f�nf Sekunden seiner Bewegung f�hrt es mit einer Momentangeschwindigkeit
(in m/s), die durch die Funktion $v$ mit $v(t)=-0,8t�+8t$ (mit $t$ in Sekunden) modelliert werden
kann. In den folgenden drei Sekunden sinkt seine Geschwindigkeit.\\
Ab der achten Sekunde bewegt es sich mit einer konstanten Geschwindigkeit von 15\,m/s.
Nach zehn Sekunden Fahrzeit erkennt der Lenker ein Hindernis in 90 m Entfernung und reagiert eine Sekunde sp�ter. Zu diesem Zeitpunkt beginnt er gleichm��ig zu bremsen und schafft
es, rechtzeitig beim Hindernis anzuhalten.

\subsection{Aufgabenstellung:}
\begin{enumerate}
	\item Interpretiere den Ausdruck $\int^5_0{v(t)}\,dt$ im Hinblick auf die Bewegung des Fahrzeugs!
	
	Gib die Bedeutung des Ausdrucks $\dfrac{\int^5_0{v(t)dt}-\int^2_0{v(t)dt}}{3}$ im vorliegenden Kontext an!
	
	\item Interpretiere den Wert $v'(3)$ im Zusammenhang mit der Bewegung des Fahrzeugs! Die Ableitungsfunktion $v'$ ist eine lineare Funktion.
	
	Bestimme ihren Anstieg und gib dessen Bedeutung im Hinblick auf die Bewegung des Fahrzeugs in den ersten f�nf Sekunden an!
	
	\item Ermittle, nach wie vielen Sekunden das Fahrzeug eine Momentangeschwindigkeit von 20\,m/s erreicht!
	
	Beschreibe (verbal und/oder mithilfe einer Skizze) den Geschwindigkeitsverlauf in den ersten f�nf Sekunden!
	
	\item Der Anhalteweg setzt sich aus dem Reaktionsweg und dem Bremsweg zusammen. Be-
rechnen Sie die Zeit, die vom Einsetzen der Bremswirkung elf Sekunden nach Beginn der
Bewegung bis zum Stillstand des Fahrzeugs verstreicht!

Stellen Sie den Geschwindigkeitsverlauf ab dem Zeitpunkt $t=10$ in der angegebenen Ab-
bildung graphisch dar und kennzeichnen Sie den Anhalteweg!\leer

\begin{center}
	\resizebox{1\linewidth}{!}{\psset{xunit=0.5cm,yunit=0.2cm,algebraic=true,dimen=middle,dotstyle=o,dotsize=5pt 0,linewidth=0.6pt,arrowsize=3pt 2,arrowinset=0.25}
\begin{pspicture*}(-1.201022408958831,-2.7014408142981874)(22.75521309768611,29.51944091763475)
\multips(0,0)(0,5.0){7}{\psline[linestyle=dashed,linecap=1,dash=1.5pt 1.5pt,linewidth=0.4pt,linecolor=lightgray]{c-c}(0,0)(22.75521309768611,0)}
\multips(0,0)(1.0,0){24}{\psline[linestyle=dashed,linecap=1,dash=1.5pt 1.5pt,linewidth=0.4pt,linecolor=lightgray]{c-c}(0,0)(0,29.51944091763475)}
\psaxes[labelFontSize=\scriptstyle,xAxis=true,yAxis=true,Dx=1.,Dy=5.,ticksize=-2pt 0]{->}(0,0)(0.,0.)(22.75521309768611,29.51944091763475)
\antwort{\pspolygon[fillcolor=black,fillstyle=solid,opacity=0.100](10.,0.)(10.,15.)(11.,15.)(21.,0.)}
\begin{scriptsize}
\rput[tl](0.30848745021420937,28.184321508853){$v(t)$ in m/s}
\rput[tl](21.117469166325495,1.6390368208288186){$t$ in s}
\end{scriptsize}
\end{pspicture*}}
\end{center}
						\end{enumerate}\leer
				
\antwort{\subsection{L�sungserwartung:}
\begin{enumerate}
	\item Das Integral $\int^5_0{v(t)}dt$ gibt die L�nge des Weges in Metern an, den das Fahrzeug in den ersten f�nf Sekunden seiner Bewegung zur�cklegt.
	
	Anmerkung: Die Antwort muss die Einheit m und das Zeitintervall beinhalten.
	
	Der Ausdruck $\dfrac{\int^5_0{v(t)dt}-\int^2_0{v(t)dt}}{3}$ gibt die durchschnittliche Geschwindigkeit in m/s des Fahrzeugs im Zeitintervall $[2;5]$ an.
	
Anmerkung: �quivalente Formulierungen sind zu akzeptieren.
	
	\item $v(t)=-0,8t�+8t$\\
	$v'(t)=-1,6t+8$\\
	$v'(3)=3,2$
	
	Die Beschleunigung 3 Sekunden nach dem Beginn der Bewegung betr�gt 3,2\,m/$\text{s}�$.
	
Anmerkung: Der Zeitpunkt und der Begriff "`Beschleunigung"' m�ssen angegeben werden.

Die Beschleunigung nimmt pro Sekunde um 1,6\,m/$\text{s}�$ ab.

Anmerkung: Die Einheit m/$\text{s}�$ muss angegeben werden.

\item $v(t)=-0,8t�+8t=20$\\
$-0,8t�+8t-20=0$ \hspace*{1cm} $t_1=t_2=5$

Die Geschwindigkeit steigt im gegebenen Zeitintervall an und erreicht nach 5 Sekunden ihr Maximum von 20 m/s.

Skizzen von Parabeln, die diese Aussage belegen, und �quivalente Formulierungen sind ebenfalls
als richtig zu werten.

Anmerkung: Die Beantwortung der ersten Frage kann auch auf anderem Wege erfolgen. Somit
kann daraus auch umgekehrt auf die Anzahl der L�sungen der quadratischen Gleichung, ohne diese zu l�sen, geschlossen werden.

\item $A=90=15\cdot 1+\frac{15\cdot t}{2}$

Der Bremsweg wird in 10 Sekunden zur�ckgelegt.

Graphik: siehe oben

Auch andere L�sungswege (z.B. mit Formeln aus der Physik) sind zu akzeptieren.
			\end{enumerate}}
		\end{langesbeispiel}