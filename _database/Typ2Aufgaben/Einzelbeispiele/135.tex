\section{135 - K7 - WS 2.2, WS 2.3, WS 2.4, WS 3.2, WS 3.3 - Qualitätsüberprüfung - VerSie}

\begin{langesbeispiel} \item[6] %PUNKTE DES BEISPIELS
In einem Paket mit 20 Kurbelwellen sind 3 fehlerhafte enthalten. Bei einer Qualitätsüberprüfung werden aus dieser Schachtel 5 Kurbelwellen entnommen. Die Zufallsvariable $X$ beschreibt die Anzahl der fehlerhaften Kurbelwellen in der Stichprobe.%Aufgabentext

\begin{aufgabenstellung}
\item %Aufgabentext

\ASubitem{Wie viele defekte Kurbelwellen erwartet man in der Stichprobe zu finden?} %Unterpunkt1
\Subitem{Wie groß ist die Wahrscheinlichkeit, dass genau eine defekte Kurbelwelle in der Stichprobe enthalten ist?} %Unterpunkt2

\item %Aufgabentext

\ASubitem{Was bedeutet der Ausdruck $1-\dfrac{\binom{17}{5}}{\binom{20}{5}}$ im gegebenen Kontext?} %Unterpunkt1
\Subitem{Berechne die Wahrscheinlichkeit, dass unter den 5 ausgewählten Kurbelwellen mindestens 2 defekte sind.} %Unterpunkt2

\item Eine Firma untersucht eine Charge von 200 Kurbelwellen. Wie groß ist die Wahrscheinlichkeit, dass weniger als 5 defekte Kurbelwellen enthalten sind wobei davon ausgegangen wird, dass der Anteil $p$ an defekten Produkten gleich $0,15$ beträgt.%Aufgabentext

\Subitem{Begründe warum hier eine Approximation mit Hilfe der Binomialverteilung möglich ist.} %Unterpunkt1
\Subitem{Berechne die gesuchte Wahrscheinlichkeit.} %Unterpunkt2

\end{aufgabenstellung}

\begin{loesung}
\item \subsection{Lösungserwartung:} 

\Subitem{Relative Häufigkeit der fehlerhaften Kurbelwellen: $\frac{3}{20}=0,15$
	
	Man kann bei 5 entnommenen Kurbelwellen mit $0,75$ kaputten Kurbelwellen rechnen.} %Lösung von Unterpunkt1
\Subitem{$P(X=1)=\binom{5}{1}\cdot \frac{3}{20}\cdot\frac{17}{19}\cdot\frac{16}{18}\cdot\frac{15}{17}\cdot\frac{14}{17}=0,43344$
	
	Die Wahrscheinlichkeit, dass genau eine defekte Kurbelwelle in der Stichprobe enthalten ist beträgt 43,34\,\%.} %%Lösung von Unterpunkt2

\setcounter{subitemcounter}{0}
\subsection{Lösungsschlüssel:}
 
\Subitem{Ein Punkt für die richtige Anzahl von erwartbaren defekten Kurbelwellen.} %Lösungschlüssel von Unterpunkt1
\Subitem{Ein Punkt für die korrekte Wahrscheinlichkeit.} %Lösungschlüssel von Unterpunkt2

\item \subsection{Lösungserwartung:} 

\Subitem{$\binom{17}{5}$ Anzahl der Möglichkeiten aus den 17 nicht-defekten Kurbelwellen 5 Kurbelwellen auszuwählen.\vspace{0,3cm}
	
	$\binom{20}{5}$ Anzahl der Möglichkeiten aus den 20 Kurbelwellen 5 Kurbelwellen auszuwählen.\vspace{0,3cm}
	
	$\dfrac{\binom{17}{5}}{\binom{20}{5}}=0,39912\approx 0,$ Wahrscheinlichkeit, dass keine einzige defekte Kurbelwelle ausgewählt worden ist.\vspace{0,3cm}
	
	$1-\dfrac{\binom{17}{5}}{\binom{20}{5}}$ Wahrscheinlichkeit, dass mindestens eine defekte Kurbelwelle ausgewählt worden ist.} %Lösung von Unterpunkt1
\Subitem{$P(X\geq 2)=1-P(X=0)-P(X=1)=1-0,39912-0,43344=0,16744$
	
	Die Wahrscheinlichkeit, dass mindestens 2 defekte Kurbelwellen in der Stichprobe enthalten sind beträgt 16,74\,\%.} %%Lösung von Unterpunkt2

\setcounter{subitemcounter}{0}
\subsection{Lösungsschlüssel:}
 
\Subitem{Ein Punkt für die richtige Interpretation des Ausdrucks.} %Lösungschlüssel von Unterpunkt1
\Subitem{Ein Punkt für die richtige Wahrscheinlichkeit.} %Lösungschlüssel von Unterpunkt2

\item \subsection{Lösungserwartung:} 

\Subitem{$P(X<5)=P(X\leq 5)=0,000000000540293$} %Lösung von Unterpunkt1
\Subitem{Da man mit einer gleichbleibenden Wahrscheinlichkeit von $0,15$ rechnen kann und man immer nur zwei Versuchsausgänge hat (defekt und nicht-defekt).} %%Lösung von Unterpunkt2

\setcounter{subitemcounter}{0}
\subsection{Lösungsschlüssel:}
 
\Subitem{Ein Punkt für die richtige Berechnung der Wahrscheinlichkeit.} %Lösungschlüssel von Unterpunkt1
\Subitem{Ein Punkt für die richtige Begründung.} %Lösungschlüssel von Unterpunkt2

\end{loesung}

\end{langesbeispiel}