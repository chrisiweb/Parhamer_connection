\section{123 - MAT - AN 1.1, AN 1.2, FA 5.2, WS 3.2 - E-Book - Matura 18/19 2. NT}

\begin{langesbeispiel} \item[6] %PUNKTE DES BEISPIELS
Ein Buch in digitaler Form wird als \textit{E-Book} (von engl. \textit{electronic book}) bezeichnet.

Die beiden folgenden auf Deutschland bezogenen Grafiken stellen Schätzwerte für die Entwicklung des Markts für E-Books dar:\leer

\psset{xunit=1cm,yunit=0.005cm,algebraic=true,dimen=middle,dotstyle=o,dotsize=5pt 0,linewidth=1.6pt,arrowsize=3pt 2,arrowinset=0.25}
\begin{pspicture*}(-1.7,-85.18518518518594)(12.5,1166.6666666666679)
\multips(0,0)(0,200.0){7}{\psline[linestyle=dashed,linecap=1,dash=1.5pt 1.5pt,linewidth=0.4pt,linecolor=lightgray]{c-c}(0,0)(12.5,0)}
\psaxes[labelFontSize=\scriptstyle,xAxis=true,labels=y,Dx=1.,Dy=200.,ticks=none]{-}(0,0)(0.,0.)(12.5,1100)
\pspolygon[linewidth=2.pt,fillcolor=black,fillstyle=solid,opacity=0.1](0.5,0.)(1.5,0.)(1.5,349.)(0.5,349.)
\pspolygon[linewidth=2.pt,fillcolor=black,fillstyle=solid,opacity=0.1](2.5,0.)(3.5,0.)(3.5,446.)(2.5,446.)
\pspolygon[linewidth=2.pt,fillcolor=black,fillstyle=solid,opacity=0.1](4.5,0.)(5.5,0.)(5.5,550.)(4.5,550.)
\pspolygon[linewidth=2.pt,fillcolor=black,fillstyle=solid,opacity=0.1](6.5,0.)(7.5,0.)(7.5,659.)(6.5,659.)
\pspolygon[linewidth=2.pt,fillcolor=black,fillstyle=solid,opacity=0.1](8.5,0.)(9.5,0.)(9.5,773.)(8.5,773.)
\pspolygon[linewidth=2.pt,fillcolor=black,fillstyle=solid,opacity=0.1](10.5,0.)(11.5,0.)(11.5,869.)(10.5,869.)
\rput[tl](1.5,1125.925925925927){Umsatz im Markt für E-Books}
\rput[tl](-1.5,829.6296296296302){\rotatebox{90}{in Millionen Euro}}
\psline[linewidth=2.pt](0.5,0.)(1.5,0.)
\psline[linewidth=2.pt](1.5,0.)(1.5,349.)
\psline[linewidth=2.pt](1.5,349.)(0.5,349.)
\psline[linewidth=2.pt](0.5,349.)(0.5,0.)
\psline[linewidth=2.pt](2.5,0.)(3.5,0.)
\psline[linewidth=2.pt](3.5,0.)(3.5,446.)
\psline[linewidth=2.pt](3.5,446.)(2.5,446.)
\psline[linewidth=2.pt](2.5,446.)(2.5,0.)
\psline[linewidth=2.pt](4.5,0.)(5.5,0.)
\psline[linewidth=2.pt](5.5,0.)(5.5,550.)
\psline[linewidth=2.pt](5.5,550.)(4.5,550.)
\psline[linewidth=2.pt](4.5,550.)(4.5,0.)
\psline[linewidth=2.pt](6.5,0.)(7.5,0.)
\psline[linewidth=2.pt](7.5,0.)(7.5,659.)
\psline[linewidth=2.pt](7.5,659.)(6.5,659.)
\psline[linewidth=2.pt](6.5,659.)(6.5,0.)
\psline[linewidth=2.pt](8.5,0.)(9.5,0.)
\psline[linewidth=2.pt](9.5,0.)(9.5,773.)
\psline[linewidth=2.pt](9.5,773.)(8.5,773.)
\psline[linewidth=2.pt](8.5,773.)(8.5,0.)
\psline[linewidth=2.pt](10.5,0.)(11.5,0.)
\psline[linewidth=2.pt](11.5,0.)(11.5,869.)
\psline[linewidth=2.pt](11.5,869.)(10.5,869.)
\psline[linewidth=2.pt](10.5,869.)(10.5,0.)
\rput[tl](0.6,-25.9){2015}
\rput[tl](2.6,-25.9){2016}
\rput[tl](4.6,-25.9){2017}
\rput[tl](6.6,-25.9){2018}
\rput[tl](8.6,-25.9){2019}
\rput[tl](10.6,-25.9){2020}
\rput[tl](0.7,420){349}
\rput[tl](2.7,520){446}
\rput[tl](4.7,620){550}
\rput[tl](6.7,730){659}
\rput[tl](8.7,850){773}
\rput[tl](10.7,940){869}
\end{pspicture*}\leer


\psset{xunit=1cm,yunit=0.55cm,algebraic=true,dimen=middle,dotstyle=o,dotsize=5pt 0,linewidth=1.6pt,arrowsize=3pt 2,arrowinset=0.25}
\begin{pspicture*}(-1.7,-1)(12.5,11.66)
\multips(0,0)(0,2.0){7}{\psline[linestyle=dashed,linecap=1,dash=1.5pt 1.5pt,linewidth=0.4pt,linecolor=gray]{c-c}(0,0)(12.5,0)}
\psaxes[labelFontSize=\scriptstyle,xAxis=true,labels=y,Dx=1.,Dy=2.,ticks=none]{-}(0,0)(0.,0.)(12.5,10.66)
\rput[tl](1.5,11.25){Nutzer im Markt für E-Books}
\rput[tl](-1.3,6){\rotatebox{90}{in Millionen}}
\rput[tl](0.6,-0.3){2015}
\rput[tl](2.6,-0.3){2016}
\rput[tl](4.6,-0.3){2017}
\rput[tl](6.6,-0.3){2018}
\rput[tl](8.6,-0.3){2019}
\rput[tl](10.6,-0.3){2020}
\psline[linewidth=2.pt](1.,7.)(3.,7.8)
\psline[linewidth=2.pt](3.,7.8)(5.,8.2)
\psline[linewidth=2.pt](5.,8.2)(7.,8.5)
\psline[linewidth=2.pt](7.,8.5)(9.,8.8)
\psline[linewidth=2.pt](9.,8.8)(11.,9.1)

\psdots[dotstyle=*](1.,7.)
\rput[bl](0.9,7.35){$7$}
\psdots[dotstyle=*](3.,7.8)
\rput[bl](2.7,8.15){$7,8$}
\psdots[dotstyle=*](5.,8.2)
\rput[bl](4.7,8.55){$8,2$}
\psdots[dotstyle=*](7.,8.5)
\rput[bl](6.7,8.85){$8,5$}
\psdots[dotstyle=*](11.,9.1)
\rput[bl](10.7,9.45){$9,1$}
\psdots[dotstyle=*](9.,8.8)
\rput[bl](8.7,9.15){$8,8$}
\end{pspicture*}

\begin{tiny}
Quelle: http://www.e-book-news.de/20-prozent-wachstum-pro-jahr-statista-sieht-deutschen-e-book-markt-im-aufwind/ [19.06.2019]\vspace{-0,2cm}
(adaptiert).
\end{tiny}%Aufgabentext

\begin{aufgabenstellung}
\item %Aufgabentext

\Subitem{Berechne für den geschätzten Umsatz pro Nutzer in Deutschland die absolute und die relative Änderung für den Zeitraum von 2015 bis 2020.\vspace{0,3cm}
	
	absolute Änderung: \euro \antwort[\rule{3cm}{0.3pt}]{$\approx 45,63$}\vspace{0,3cm}
	
	relative Änderung: \antwort[\rule{3cm}{0.3pt}]{$\approx 0,9155$}} %Unterpunkt1
\Subitem{Berechne den Differenzenquotient des geschätzten Umsatzes pro Nutzer in Deutschland für den Zeitraum von 2015 bis 2020.} %Unterpunkt2

\item Die geschätzte Steigerung des Umsatzes im Markt für E-Books von 349 Millionen Euro im Jahr 2015 auf 869 Millionen Euro im Jahr 2020 wird in der oben angeführten Quelle wie folgt beschrieben:
	
	"`20 Prozent Wachstum pro Jahr"'%Aufgabentext

\Subitem{Gib an, wie die Umsatzschätzung $U(2017)$ für das Jahr 2017 hätte lauten müssen, wenn der Umsatz ausgehend vom Schätzwert von 2015 tatsächlich jährlich um 20\,\% zugenommen hätte.\leer
	
	$U(2017)=$\,\antwort[\rule{3cm}{0.3pt}]{502,56}\,Millionen Euro} %Unterpunkt1
	
Jemand beschreibt die geschätzte Steigerung des Umsatzes im Markt für E-Books von 349 Millionen Euro im Jahr 2015 auf 869 Millionen Euro im Jahr 2020 wie folgt:
	
	"`$a$ Millionen Euro Wachstum pro Jahr"'	
	
\Subitem{Berechne $a$} %Unterpunkt2

\item Im Jahr 2015 betrug die Einwohnerzahl von Deutschland ungefähr 82,18\,Millionen, jene von Österreich ungefähr 8,58\,Millionen. Jemand stellt sich die folgende Frage: "`Wie groß ist die Anzahl der Personen aus Österreich, die im Jahr 2015 schon E-Book-Nutzer waren?"'%Aufgabentext

\Subitem{Beantworte diese Frage unter der Annahme, dass Österreich im Jahr 2015 den gleichen (geschätzten) Anteil an E-Book-Nutzern wie Deutschland hatte.\vspace{0,3cm}
	
	Anzahl:\,\antwort[\rule{3cm}{0.3pt}]{730\,835}\,Personen} %Unterpunkt1

Im Jahr 2020 werden 500 Personen aus Österreich zufällig ausgewählt. Die als binomialverteilte Zufallsvariable $X$ gibt die Anzahl der Personen aus dieser Auswahl an, die E-Book-Nutzer sind. Dabei wird die Wahrscheinlichkeit, dass eine Person E-Book-Nutzer ist, mit 12\,\% angenommen.	
	
\Subitem{Berechne die Wahrscheinlichkeit dafür, dass mindestens 50\,E-Book-Nutzer in dieser Auswahl sind.} %Unterpunkt2

\end{aufgabenstellung}

\begin{loesung}
\item \subsection{Lösungserwartung:} 

\Subitem{Umsatz pro Nutzer 2015: rund \EUR{49,86}\\
	Umsatz pro Nutzer 2020: rund \EUR{95,49}
	
	absolute Änderung: \EUR{45,63}\\
	relative Änderung: $0,9155$} %Lösung von Unterpunkt1
\Subitem{Differenzenquotient für das Intervall $[2015;2020]$: rund \EUR{9,13} pro Jahr} %%Lösung von Unterpunkt2

\setcounter{subitemcounter}{0}
\subsection{Lösungsschlüssel:}
 
\Subitem{Ein Punkt für die Angabe der beiden richtigen Werte. Andere Schreibweisen der Lösungen sind ebenfalls als richtig zu werten.

		Toleranzintervall für die absolute Änderung: $[44;47]$\\
		Toleranzintervall für die relative Änderung: $[0,88;0,95]$} %Lösungschlüssel von Unterpunkt1
\Subitem{Ein Punkt für die richtige Lösung.

		Toleranzintervall: $[8,90;9,40]$} %Lösungschlüssel von Unterpunkt2

\item \subsection{Lösungserwartung:} 

\Subitem{$U(2017)=U(2015)\cdot 1,2^2$\\
	$U(2017)=502,56$\,Millionen Euro} %Lösung von Unterpunkt1
\Subitem{$a=\frac{869-349}{5}=104$} %%Lösung von Unterpunkt2

\setcounter{subitemcounter}{0}
\subsection{Lösungsschlüssel:}
 
\Subitem{Ein Ausgleichspunkt für die richtige Lösung.

		Toleranzintervall: [502 Millionen Euro; 503 Millionen Euro]} %Lösungschlüssel von Unterpunkt1
\Subitem{Ein Punkt für die richtige Lösung.} %Lösungschlüssel von Unterpunkt2

\item \subsection{Lösungserwartung:} 

\Subitem{mögliche Vorgehensweise:\\
	$8,58\cdot\frac{7}{81,18}=0,7308347\ldots\approx0,730835$
	
	Anzahl: 730\,835 Personen} %Lösung von Unterpunkt1
\Subitem{mögliche Vorgehensweise:\\
	$n=500$; $p=12$\\
	$P(X\geq 50)=0,9287\ldots\approx 0,929$} %%Lösung von Unterpunkt2

\setcounter{subitemcounter}{0}
\subsection{Lösungsschlüssel:}
 
\Subitem{Ein Punkt für die richtige Lösung.

		Toleranzintervall: $[720\,000;780\,000]$\\} %Lösungschlüssel von Unterpunkt1
\Subitem{Ein Punkt für die richtige Lösung. Andere Schreibweisen der Lösung sind ebenfalls als richtig zu werten.

		Toleranzintervall: $[0,90;0,95]$} %Lösungschlüssel von Unterpunkt2

\end{loesung}

\end{langesbeispiel}