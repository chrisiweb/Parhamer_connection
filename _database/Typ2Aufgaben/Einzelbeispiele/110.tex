\section{110 - K7 - AG-L 5.1, WS 2.1, WS 2.2, WS 2.3 - Sport-Club - MatKon}

\begin{langesbeispiel} \item[6] %PUNKTE DES BEISPIELS
Der Wiener Sport-Club (oft kurz "`WSC"' oder einfach nur "`Sport-Club"' genannt) ist einer der ältesten Sportvereine Österreichs. Der Verein ging aus dem am 24. Februar 1883 gegründeten Wiener Cyclistenclub hervor.\\
Seit der Fusion mit der Wiener Sportvereinigung im Jahr 1907 gibt es auch die berühmte Fußballabteilung des Wiener Sport-Club und gehört damit zu den ältesten Fußballvereinen Österreichs. Mit drei österreichischen Meistertiteln und einem Cup-Sieg gehört der Wiener Sport-Club auch zu den erfolgreichsten Fußballvereinen des Landes.

Seit längerem ist der Klub schon auf der Suche nach einem Platz für ein größeres Stadium. Nun hat sich mit \textit{Soli-Saftl} (ein Getränkehersteller) endlich ein Sponsor gefunden, der dem Sportklub eine rechteckige Fläche mit einer Länge von 160\,m und eine breite von 116\,m verpachten würde.%Aufgabentext

\begin{aufgabenstellung}
\item Der \textit{WSC} möchte ein elliptisches Stadion auf dieser Fläche bauen lassen welches die Fläche maximal ausnützt. Angenommen man legt auf diese Rechtecksfläche ein Koordinatensystem dessen Ursprung direkt in der Mitte des Rechtecks liegt.%Aufgabentext

\ASubitem{Stelle die Ellipsengleichung auf, die den Grundriss des neuen Stadions darstellt.} %Unterpunkt1

Die Idee wäre, dass Spielfeld so zu legen, dass die beiden Torlinien jeweils durch die Brennpunkte der Ellipse verlaufen. Ein reguläres Fußballspielfeld ist 105 Meter lang und 68 Meter breit. 

\Subitem{Überprüfe rechnerisch (ohne Geogebra und durch Angabe aller Nebenrechnungen), ob das geplante Spielfeld des Sportklubs den Anforderungen eines regulären Fußballspielfelds entsprechen kann und begründe deine Antwort.} %Unterpunkt2

\item In der Saison 2018/19 absolvierte der Sportklub insgesamt 29 Spiele von denen sie 12 gewinnen konnten. Die nachstehende Tabelle zeigt die Häufigkeiten der Tordifferenzen aller in dieser Saison gewonnenen Spiele:
	
	\begin{center}
	\begin{tabular}{c|c}
	Tordifferenz&Häufigkeit\\ \hline
	$+1$&$33,3\,\%$\\
	$+2$&$33,3\,\%$\\
	$+3$&$16,7\,\%$\\
	$+4$&$8,3\,\%$\\
	$+5$&$8,3\,\%$
	\end{tabular}
	\end{center}%Aufgabentext

\Subitem{Berechne, basierend auf dieser Tabelle, welche Tordifferenz beim nächsten Sieg des Sportklubs zu erwarten wäre.} %Unterpunkt1
\Subitem{Erkläre in eigenen Worten, warum diese Prognose für die Saison 2019/20 nicht sehr aussagekräftig ist.} %Unterpunkt2

\item Erst kürzlich wurde erhoben, dass 76\,\% der Sportklubspieler "`rechtsfüßig"' sind und dadurch "`nur"' 24\,\% "`linksfüßig"'. Außerdem weiß man, dass 23\,\% der Spieler eine Brille tragen müssen.%Aufgabentext

\Subitem{Ein Spieler der Mannschaft wird per Zufallsprinzip zum Kapitän der Mannschaft ernannt. Deute den Ausdruck $0,76\cdot 0,23+0,24\cdot 0,77$ im gegebenen Kontext.} %Unterpunkt1
\Subitem{Gib das Gegenereignis zu einem "`linksfüßigen"' Brillenträger an! (Vorsicht: "`Nicht linksfüßiger Brillenträger"' ist zu wenig!).} %Unterpunkt2

\end{aufgabenstellung}

\begin{loesung}
\item \subsection{Lösungserwartung:} 

\Subitem{$2a=160 \Rightarrow a=80; 2b=116 \Rightarrow b=58$
	
	$3364x^2+6400y^2=21\,529\,600$} %Lösung von Unterpunkt1
\Subitem{$e=\sqrt{80^2-58^2}\approx 55,1$
	
	$y^2=3364-0,5256x^2 \Rightarrow y^2=3364-0,5256\cdot 55,1^2 \Rightarrow y=\pm 42,05$
	
	Das Spielfeld hätte also eine Breite von 110,2\,m und eine Höhe von 84,1\,m. Es wäre also genug Platz für ein reguläres Fußballfeld.} %%Lösung von Unterpunkt2

\setcounter{subitemcounter}{0}
\subsection{Lösungsschlüssel:}
 
\Subitem{Ein Punkt für die richtige Ellipsengleichung.} %Lösungschlüssel von Unterpunkt1
\Subitem{Ein Punkt für eine richtige Begründung.} %Lösungschlüssel von Unterpunkt2

\item \subsection{Lösungserwartung:} 

\Subitem{$\mu=1\cdot 0,333+2\cdot 0,333+3\cdot 0,167+4\cdot 0,083+5\cdot 0,083=2,247$} %Lösung von Unterpunkt1
\Subitem{Es handelt sich dabei um kein Zufallsexperiment. Der Stürmer könnte sich bis zur nächsten Saison verletzen, Sportklub könnte aufsteigen in einer stärkere oder absteigen in eine schwächere Liga und dort leichtere/schwerere Gegner vorfinden,...} %%Lösung von Unterpunkt2

\setcounter{subitemcounter}{0}
\subsection{Lösungsschlüssel:}
 
\Subitem{Ein Punkt für den richtigen Erwartungswert.} %Lösungschlüssel von Unterpunkt1
\Subitem{Ein Punkt für eine richtige Begründung.} %Lösungschlüssel von Unterpunkt2

\item \subsection{Lösungserwartung:} 

\Subitem{Der Ausdruck gibt die Wahrscheinlichkeit an, dass der Kapitän der Mannschaft entweder ein rechtsfüßiger Brillenträger ist oder ein linksfüßiger ohne Brille.} %Lösung von Unterpunkt1
\Subitem{Das Gegenereignis wäre rechtsfüßig oder linksfüßig ohne Brille.} %%Lösung von Unterpunkt2

\setcounter{subitemcounter}{0}
\subsection{Lösungsschlüssel:}
 
\Subitem{Ein Punkt für die richtige Interpretation.} %Lösungschlüssel von Unterpunkt1
\Subitem{Ein Punkt für das richtige Gegenereignis.} %Lösungschlüssel von Unterpunkt2

\end{loesung}

\end{langesbeispiel}