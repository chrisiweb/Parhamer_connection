\section{97 - K8 - SWS - WS 3.4 - Düngesäcke - BIFIE BHS}

\begin{langesbeispiel} \item[11] %PUNKTE DES BEISPIELS
\textbf{Düngesäcke}

Mehrere Maschinen füllen Säcke mit Dünger ab. Als Füllmenge sind laut Aufdruck 25 kg vorgesehen.

\begin{enumerate}
\item
Langfristige Uberprüfungen einer bestimmten Maschine haben ergeben, dass die tatsächliche Füllmenge der Säcke mit dem Erwartungswert $\mu = 24,8$ kg und der Standardabweichung $\sigma = 0,2$ kg normalverteilt ist.


\begin{enumerate}
\item[-]
Berechne die Wahrscheinlichkeit, dass ein zufällig ausgewählter Sack eine geringere Füllmenge als vorgesehen aufweist.(1)

\item[-]
Beschreibe, wie man den Prozentsatz dieser Wahrscheinlichkeit anhand einer Skizze ohne genaue Berechnung abschätzen kann.(1)

\antwort{P(x$\leq24.6)=0,157$
Die Wahrscheinlichkeit beträt rund $84,3\%$}

\antwort{Die Differenz zw. der Sollfüllmenge und dem Erwartungswert entspricht der Standardabweichung. Der Bereich für $x< 25$ besteht daher aus der Sigmaumgebung ($68\%$) und der Hälfte des außerhalb der Sigma Umgebung liegenden Bereiches, das sind ca. $16\%$}
\end{enumerate}

\item 
In der nachstehenden Abbildung sind 2 Wahrscheinlichkeitsdichtefunktionen zu sehen.

\newrgbcolor{rctzbb}{0.10980392156862745 0.2235294117647059 0.7333333333333333}
\psset{xunit=7cm,yunit=2cm,algebraic=true,dimen=middle,dotstyle=o,dotsize=3pt 0,linewidth=0.8pt,arrowsize=3pt 2,arrowinset=0.25}
\begin{pspicture*}(24.15,-0.22081676036979597)(26.2,2.393653682408596)
\psaxes[labelFontSize=\scriptstyle,xAxis=true,yAxis=true,Dx=0.2,Dy=0.5,ticksize=-2pt 0,subticks=2]{->}(0,0)(24.15,-0.22081676036979597)(26.1,2.393653682408596)
\psplot[linewidth=1pt,linecolor=rctzbb,plotpoints=200]{24.15}{26.05}{EXP((-(x-24.8)^(2.0))/(0.2^(2.0)*2.0))/(abs(0.2)*sqrt(3.141592653589793*2.0))}
\psplot[linewidth=1pt,linecolor=rctzbb,plotpoints=200,linestyle=dashed,dash=6pt 1pt]{24.15}{26.05}{EXP((-(x-25.2)^(2.0))/(0.2^(2.0)*2.0))/(abs(0.2)*sqrt(3.141592653589793*2.0))}
\psline(24.8,0.)(24.8,2.1)
\begin{scriptsize}
\rput[tl](24.2,1.133401545086242){Dichtefunktion 1}
\rput[tl](25.45,1.1389289667411646){Dichtefunktion 2}
\end{scriptsize}
\end{pspicture*}


\begin{enumerate}

\item[-]
Beschreibe, welcher Parameter verändert wurde und wie das zu sehen ist.
(1)
\item[-]
Beschreibe, woran zu sehen ist, dass der andere Parameter nicht verändert wurde.(1)

\antwort{Der Erwartungswert ist größer geworden, weil die 2. Kurve den Hochpunkt weiter rechts hat. Die Standardabweichung ist gleich geblieben, weil die Kurve außer der Verschiebung gleich geblieben ist.}
\end{enumerate}

\item
Bei einer bestimmten Abfüllmaschine erhält man durch Beobachtung folgende Daten: Die Füllmenge ist normalverteilt mit dem Erwartungswert $\mu=24.8$kg.
Wie groß darf bei dieser Produktion die Standardabweichung höchstens sein, wenn $95\%$ der Säcke im Inhalt höchstens um 0.5 kg vom Erwartungswert abweichen sollen? (2)

\antwort{$\sigma=0.26$}



\item
Bei einer bestimmten Abfüllmaschine kann die Füllmenge als normalverteilte Zufallsgröße mit einer Standardabweichung von 0,94 kg und einem Erwartungswert von 25 kg angenommen werden.
Laut Betriebsvorschrift müssen Säcke mit mehr als $\pm \frac{1}{2}$kg Abweichung vom Erwartungswert nachkorrigiert werden.

\begin{enumerate}

\item[-]
Berechne die Wahrscheinlichkeit, dass die Füllmenge eines zufällig ausgewählten Sackes nachkorrigiert werden muss. (1)
\item[-]
Nach einer Untersuchung wird beschlossen, dass $97 \%$ aller Säcke (symmetrisch um den Erwartungswert) als korrekt abgefüllt betrachtet werden sollen.

Berechne, welche Abweichungen vom Erwartungswert nun bei der Abfüllung toleriert werden. (2)

\antwort{Die Wahrscheinlichkeit, dass nachkorrigiert werden muss beträgt rund $59,5\%$}

\antwort{$22,96\leq x \leq 27,04$}


\end{enumerate}

\item[\fbox{\LARGE A} (e)]
Bei der Abfüllung gehen erfahrungsgemäß $8\%$ der Düngesäcke kaputt.
Berechne die Wahrscheinlichkeit, dass von 800 Säcken höchstens 50 kaputt gehen? (2)

\antwort{P=0.034}

\end{enumerate}
\end{langesbeispiel}