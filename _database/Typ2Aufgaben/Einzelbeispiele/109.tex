\section{109 - K7 - AG 3.4, AG-L 5.1, AG-L 5.2, WS 3.1 - Cool-Conic - MatKon}

\begin{langesbeispiel} \item[6] %PUNKTE DES BEISPIELS
Das neue Wiener Start-Up \textit{Cool-Conic} versucht weltweit Fuß zu fassen und hat sich dafür von einem bekannten Grafik-Büro folgendes Logo machen lassen:
	
	\begin{center}
	\resizebox{0.3\linewidth}{!}{\newrgbcolor{xdxdff}{0.49019607843137253 0.49019607843137253 1}
\newrgbcolor{ccqqqq}{0.8 0 0}
\psset{xunit=1cm,yunit=1cm,algebraic=true,dimen=middle,dotstyle=o,dotsize=5pt 0,linewidth=1.6pt,arrowsize=3pt 2,arrowinset=0.25}
\begin{pspicture*}(-3.817272727272728,-3.4018181818181863)(3.91,3.7981818181818214)
%\psaxes[labelFontSize=\scriptstyle,xAxis=true,yAxis=true,labels=none,Dx=1,Dy=1,ticksize=-2pt 0,subticks=2]{->}(0,0)(-3.817272727272728,-3.4018181818181863)(3.91,3.7981818181818214)[x,140] [y,-40]
\pscustom[linewidth=0.8pt,linecolor=blue,fillcolor=blue,fillstyle=solid,opacity=0.1]{\psplot{-2.4}{0}{sqrt(-x^(2)+9)}\lineto(0,3)\psplot{0}{-2.4}{0.5*x+3}\lineto(-2.4,1.8)\closepath}
\pscustom[linewidth=0.8pt,linecolor=blue,fillcolor=blue,fillstyle=solid,opacity=0.1]{\psplot{0}{2.4}{sqrt(-x^(2)+9)}\lineto(2.4,1.8)\psplot{2.4}{0}{-0.5*x+3}\lineto(0,3)\closepath}
\pscircle[linewidth=2pt](0,0){3}
\rput{0}(0,0){\parametricplot[linewidth=2pt]{-0.99}{0.99}{3*(1+t^2)/(1-t^2)|5.196152422706632*2*t/(1-t^2)}}
\rput{0}(0,0){\parametricplot[linewidth=2pt]{-0.99}{0.99}{3*(-1-t^2)/(1-t^2)|5.196152422706632*(-2)*t/(1-t^2)}}
\psplot[linewidth=2pt]{-2.4}{0}{(--18--3*x)/6}
\psplot[linewidth=2pt]{0}{2.4}{(-18--3*x)/-6}
\rput{0}(0,0){\psplot[linewidth=2pt]{-1}{1}{x^2/2/1}}
\end{pspicture*}}
	\end{center}
	
	Du arbeitest schon seit längerem bei der Konkurrenzfirma \textit{Copycat-Conic} und dein Chef verlangt nun von dir das oben abgebildete Logo so gut wie möglich nachzubauen. Um dir eine Übersicht zu verschaffen legst du im ersten Schritt ein Koordinatensystem in die Mitte des Logos und ergänzt ein paar Beschriftungen:
	
	\begin{center}
	\resizebox{0.4\linewidth}{!}{\newrgbcolor{xdxdff}{0.49019607843137253 0.49019607843137253 1}
\newrgbcolor{ccqqqq}{0.8 0 0}
\psset{xunit=1cm,yunit=1cm,algebraic=true,dimen=middle,dotstyle=o,dotsize=5pt 0,linewidth=1.6pt,arrowsize=3pt 2,arrowinset=0.25}
\begin{pspicture*}(-3.817272727272728,-3.4018181818181863)(3.91,3.7981818181818214)
\psaxes[labelFontSize=\scriptstyle,xAxis=true,yAxis=true,labels=none,Dx=1,Dy=1,ticksize=0pt 0,subticks=0]{->}(0,0)(-3.817272727272728,-3.4018181818181863)(3.91,3.7981818181818214)[x,140] [y,-40]
\pscustom[linewidth=0.8pt,linecolor=blue,fillcolor=blue,fillstyle=solid,opacity=0.1]{\psplot{-2.4}{0}{sqrt(-x^(2)+9)}\lineto(0,3)\psplot{0}{-2.4}{0.5*x+3}\lineto(-2.4,1.8)\closepath}
\pscustom[linewidth=0.8pt,linecolor=blue,fillcolor=blue,fillstyle=solid,opacity=0.1]{\psplot{0}{2.4}{sqrt(-x^(2)+9)}\lineto(2.4,1.8)\psplot{2.4}{0}{-0.5*x+3}\lineto(0,3)\closepath}
\pscircle[linewidth=2pt](0,0){3}
\rput{0}(0,0){\parametricplot[linewidth=2pt]{-0.99}{0.99}{3*(1+t^2)/(1-t^2)|5.196152422706632*2*t/(1-t^2)}}
\rput{0}(0,0){\parametricplot[linewidth=2pt]{-0.99}{0.99}{3*(-1-t^2)/(1-t^2)|5.196152422706632*(-2)*t/(1-t^2)}}
\psplot[linewidth=2pt]{-2.4}{0}{(--18--3*x)/6}
\psplot[linewidth=2pt]{0}{2.4}{(-18--3*x)/-6}
\rput{0}(0,0){\psplot[linewidth=2pt]{-1}{1}{x^2/2/1}}

\rput[tl](-1.354634612421705,2.2){$z$}
\rput[tl](1.010741999146064,2.2){$z_2$}
\psdots[dotsize=4pt 0,dotstyle=*](1,0.5)
\rput[bl](1.149881799826521,0.3568598224078179){$P$}
\rput[bl](-0.4549881799826521,3.2){$Q$}
\begin{scriptsize}
\psdots[dotsize=6pt 0,dotstyle=*](0,3)
\end{scriptsize}
\end{pspicture*}}
	\end{center}
	
	 Von einem Informanten hat dein Chef die Information, dass der äußere Teil des Logos eine Hyperbel in 1. Hauptlage sein soll mit folgender Gleichung:
	$$-3x^2+y^2=-27$$.%Aufgabentext

\begin{aufgabenstellung}
\item %Aufgabentext

\ASubitem{Zwischen den zwei Hyperbelästen befindet sich ein Kreis. Stelle die Kreisgleichung auf.} %Unterpunkt1
\Subitem{Laut deinen Bemessungen sollten die Koordinaten des Punkts $P=(1\mid 0,5)$ sein. Stelle damit die entsprechende Parabelgleichung auf.} %Unterpunkt2

\item %Aufgabentext

\Subitem{Die Strecke $z_2$ liegt auf einer Geraden, die durch den Punkt $Q$ sowie dem rechten Brennpunkt der Hyperbel geht. Stelle diese Gerade in Parameterform auf.} %Unterpunkt1
\Subitem{Berechne die Länge $z=z_2$ ohne dabei Geogebra zu verwenden (schreibe alle notwendigen Rechenschritte auf).} %Unterpunkt2

\item Du hast es geschafft! Du hast das Logo exakt kopiert, nur dein Chef ist misstrauisch. Zur Sicherheit lässt er 1000 Personen auf der Straße befragen ob sie einen Unterschied zwischen dem Original und deinem Logo erkennen können. Dabei kamen folgende Antworten: 3\,\% sahen keine Ähnlichkeit, 12\,\% sahen eine gewisse Ähnlichkeit, 42\,\% sahen eine starke Ähnlichkeit und 43\,\% konnten deine Fälschung nicht vom Original unterscheiden.
	
	Legt man den Antworten eine Skala zu Grunde (0 - keine Ähnlichkeit, 1 - gewisse Ähnlichkeit, 2 - starke Ähnlichkeit, 3 - ident), dann ergab sich bei den Antworten ein Erwartungswert $\mu=2,25$.%Aufgabentext

\Subitem{Berechne die dazu gehörende Standardabweichung.} %Unterpunkt1

Noch vor der Befragung wollte dich ein Kollege zu einer Wette herausfordern. Die Wette wäre wie folgt verlaufen: Man hätte zufällig eine Person auf der Straße befragt. Wenn diese Person keinen Unterschied zwischen deinem Logo und dem Original erkannt hätte, hättest du von deinem Kollegen 60 Euro bekommen. Hätte die Person einen Fehler erkannt hätte er von dir 45 Euro bekommen.

\Subitem{Du hast die Wette dankend abgelehnt, war das ein Fehler? Begründe mit Hilfe des Erwartungswerts.} %Unterpunkt2

\end{aufgabenstellung}

\begin{loesung}
\item \subsection{Lösungserwartung:} 

\Subitem{$\frac{x^2}{9}-\frac{y^2}{27}=1 \Rightarrow a^2=9, b^2=27$
	
	Der Radius des Kreises ist deshalb $r=3$ und der Mittelpunkt ist $M=(0\mid 0)$, daher: $k\!:x^2+y^2=9$} %Lösung von Unterpunkt1
\Subitem{$x^2=2py \Rightarrow 1=p \Rightarrow x^2=2y$} %%Lösung von Unterpunkt2

\setcounter{subitemcounter}{0}
\subsection{Lösungsschlüssel:}
 
\Subitem{Ein Punkt für die richtige Kreisgleichung.} %Lösungschlüssel von Unterpunkt1
\Subitem{Ein Punkt für die richtige Parabelgleichung.} %Lösungschlüssel von Unterpunkt2

\item \subsection{Lösungserwartung:} 

\Subitem{$Q=(0\mid 3)$, $e=\sqrt{9+27}=6 \Rightarrow F=(6\mid 0)$
	
	$g\!:X=\Vek{0}{3}{}+t\cdot\Vek{6}{-3}{}$} %Lösung von Unterpunkt1
\Subitem{$g\cap k: (0+6t)^2+(3-3t)^2=9 \Rightarrow 36t^2+9-18t+9t^2=9 \Rightarrow t_1=0, t_2=0.4$
	
	$S_1=(0\mid 3), S_2=(2,4\mid 1,8) \Rightarrow |\vec{S_1S_2}|=\sqrt{2,4^2+1,2^2}\approx 2,68$} %%Lösung von Unterpunkt2

\setcounter{subitemcounter}{0}
\subsection{Lösungsschlüssel:}
 
\Subitem{Ein Punkt für die richtige Parameterform.} %Lösungschlüssel von Unterpunkt1
\Subitem{Ein Punkt für die richtige Länge.} %Lösungschlüssel von Unterpunkt2

\item \subsection{Lösungserwartung:} 

\Subitem{$V(X)=(0-2,25)^2\cdot 0,03+(1-2,25)^2\cdot 0,12+(2-2,25)^2\cdot 0,42+(3-2,25)^2\cdot 0,43\approx 0,635625 \Rightarrow s=\sqrt{0,635625}\approx 0,79726$} %Lösung von Unterpunkt1
\Subitem{$\mu=60\cdot 0,43-45\cdot 0,57=0,15$
	
	Du hättest die Wette annehmen sollen.} %%Lösung von Unterpunkt2

\setcounter{subitemcounter}{0}
\subsection{Lösungsschlüssel:}
 
\Subitem{Ein Punkt für die richtige Standardabweichung.} %Lösungschlüssel von Unterpunkt1
\Subitem{Ein Punkt für eine richtige Begründung.} %Lösungschlüssel von Unterpunkt2

\end{loesung}

\end{langesbeispiel}