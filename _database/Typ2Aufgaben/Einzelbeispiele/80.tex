\section{80 - MAT - AN 1.3, FA 5.1, FA 5.2, FA 5.4, FA 5.5 - Aktivität und Altersbestimmung - Matura NT 1 16/17}


\begin{langesbeispiel} \item[6] %PUNKTE DES BEISPIELS

Beim Zerfall eines radioaktiven Stoffes nimmt die Anzahl der noch nicht zerfallenen Atomkerne exponentiell ab und lässt sich näherungsweise durch eine Funktion $N$ mit $N(t)=N_0\cdot e^{-\lambda\cdot t}$ beschreiben. Dabei ist $N_0$ die Anzahl der Atomkerne zum Zeitpunkt $t=0, N(t)$ die Anzahl der noch nicht zerfallenen Atomkerne zum Zeitpunkt $t\geq 0$ und $\lambda$ die sogenannte Zerfallskonstante.

Die Aktivität $A(t)$ ist der Absolutbetrag der momentanen Änderungsrate der Funktion $N$ zum Zeitpunkt $t$. Sie wird in Becquerel (Bq) gemessen. Eine Aktivität von 1\,Bq entspricht einem radioaktiven Zerfall pro Sekunde.

Bei radioaktiven Stoffen nimmt die Aktivität ebenfalls exponentiell ab und kann durch eine Funktion $A$ mit $A(t)=A_0\cdot e^{-\lambda\cdot t}$ modelliert werden. Dabei ist $A_0$ die Aktivität zum Zeitpunkt $t=0$ und $A(t)$ die Aktivität zum Zeitpunkt $t\geq 0$.

\subsection{Aufgabenstellung:}
\begin{enumerate}
	\item Gib eine Formel an, mit der die Anzahl der Atomkerne $N_0$ aus der gemessenen Aktivität $A_0$ berechnet werden kann!\leer
	
	Eine Probe von $^{238}$U (Uran-238) hat zum Zeitpunkt $t=0$ eine Aktivität von 17\,Bq. Die Zerfallskonstante von $^{238}$U hat den Wert $\lambda\approx 4,92\cdot 10^{-18}$ pro Sekunde.
	
	Bestimme die Anzahl der $^{238}$U-Atomkerne zum Zeitpunkt $t=0$ in der Probe!\leer
	
	\item Mithilfe des Anteils des in einer Probe enthaltenen Kohlenstoffisotops $^{14}$C kann das Alter der Probe ermittelt werden. Durch den Stoffwechsel hat sich zwischen der Bildung und dem radioaktiven Zerfall des Isotops sowohl in der Atmosphäre als auch in lebenden Organismen eine Gleichgewichtskonzentration von $^{14}$C bzw. eine Aktivität von ca. 0,267\,Bq pro Gramm Kohlenstoff eingestellt. Mit dem Absterben eines Organismus (z.B. eines Baumes) endet die Aufnahme von $^{14}$C. Der $^{14}$C-Anteil nimmt ab diesem Zeitpunkt exponentiell (mit der Zerfallskonstante $\lambda\approx 1,21\cdot 10^{-4}$ pro Jahr) ab und damit nimmt auch die Aktivität exponentiell ab.\leer
	
	Ein Fundstück aus Holz hat einen Kohlenstoffanteil von 25 Gramm und eine Aktivität von ca. 4\,Bq. Gib an, vor wie vielen Jahren dieses Holz abgestorben ist!\leer
	
	Gib an, ob zum Zeitpunkt des Fundes mehr oder weniger als die Hälfte des ursprünglich vorhandenen $^{14}$C-Atomkerne zerfallen ist, und begründe deine 			Entscheidung!\leer

	\item Die Funktion $N$ kann auch in der Form $N(t)=N_0\cdot 0,5^{\frac{t}{c}}$ mit $c\in\mathbb{R}^+$ angegeben werden.\leer
	
	\fbox{A} Gib an, welcher Zusammenhang zweischen der Konstanten $c$ und der Halbwertszeit $\tau$ eines radioaktiven Stoffes besteht!\leer
	
	In der unten stehenden Abbildung ist der Graph einer Funktion $N$ mit $N(t)=N_0\cdot 0,5^{\frac{t}{c}}$ mit $c\in\mathbb{R}^+$ dargestellt.
	
	Zeichne den Verlauf des Graphen einer Funktion $N_{\text{neu}}$ mit $N_{\text{neu}}(t)=N_0\cdot 0,5^{\frac{t}{c_{\text{neu}}}}$ mit $c_{\text{neu}}\in\mathbb{R}^+$ in dieser Abbildung ein, wenn $c_{\text{neu}}<c$ gelten soll!
		
		\begin{center}
			\resizebox{0.7\linewidth}{!}{\psset{xunit=1.0cm,yunit=1.0cm,algebraic=true,dimen=middle,dotstyle=o,dotsize=5pt 0,linewidth=1.6pt,arrowsize=3pt 2,arrowinset=0.25}
\begin{pspicture*}(-0.74,-0.44)(9.42,6.58)
\psaxes[labelFontSize=\scriptstyle,xAxis=true,yAxis=true,labels=none,Dx=1.,Dy=1.,ticksize=0pt 0,subticks=2]{->}(0,0)(0.,0.)(9.42,6.58)[t,140] [\text{$N(t)$, $N_{neu}(t)$},-40]
\psplot[linewidth=2.pt,plotpoints=200]{0}{9.419999999999995}{5.0*0.5^(x/2.5)}
\antwort{\psplot[linewidth=2.pt,plotpoints=200]{0}{9.419999999999995}{5.0*0.5^(x/1.8)}}
\rput[tl](3.48,2.42){$N$}
\antwort{\rput[tl](2.2,1.52){$N_{\text{neu}}$}}
\rput[tl](-0.8,5.2){$N_0$}
\end{pspicture*}}
		\end{center}
	\end{enumerate}

\antwort{
\begin{enumerate}
	\item \subsection{Lösungserwartung:} 

Mögliche Vorgehensweise:

$A(t)=|N'(t)|=\lambda\cdot N_0\cdot e^{-\lambda\cdot t}$

$A_0=\lambda\cdot N_0$\leer

$N_0=\frac{A_0}{\lambda}=\frac{17}{4,92\cdot 10^{-18}}\approx 3,46\cdot 10^{18}$

Zum Zeitpunkt $t=0$ befinden sich ca. $3,46\cdot 10^{18}$ Atomkerne von $^238$U in der Probe.

\subsection{Lösungsschlüssel:}
\begin{itemize}
	\item Ein Punkt für eine korrekte Formel. Äquivalente Formeln sind als richtig zu werten.
	\item Ein Punkt für die richtige Lösung. Toleranzintervall: $[3,4\cdot 10^{18}\,\text{Atomkerne}; 3,5\cdot 10^{18}\,\text{Atomkerne}]$
\end{itemize}

	\item \subsection{Lösungserwartung:}

Mögliche Vorgehensweise:

lebender Organismus: $A_0=25\cdot 0,267=6,675$\,Bq für 25\,g Kohlenstoff

$4=6,675\cdot e^{-1,21\cdot 10^{-4}\cdot t}$

$t\approx 4\,232$ Jahre\leer

Mögliche Vorgehensweise:

$\tau$ ... Halbwertszeit

$\frac{N_0}{2}=N_0\cdot e^{-\lambda\cdot\tau}$

$\ln(2)=\lambda\cdot\tau$

$\tau=\frac{\ln(2)}{\lambda}\approx 5\,730$\leer

Zum Zeitpunkt des Fundes sind weniger als die Hälfte der ursprünglich vorhandenen Atomkerne zerfallen, da die Halbwertszeit von $^14$C ca. 5\,730 Jahre beträgt, das Holz aber erst vor ca. 4\,232 Jahren abgestorben ist.

\subsection{Lösungsschlüssel:}
\begin{itemize}
	\item Ein Punkt für die richtige Lösung. Toleranzintervall: $[4\,225\,\text{Jahre};2\,240\,\text{Jahre}]$
	
	Die Aufgabe ist auch dann als richtig gelöst zu werten, wenn bei korrektem Ansatz das Ergebnis aufgrund eines Rechenfehlers nicht richtig ist.
	\item Ein Punkt für eine (sinngemäß) korrekte Begründung dafür, dass weniger als die Hälfte der ursprünglich vorhandenen $^14$C-Atomkerne zerfallen sind. Andere korrekte Begründungen (z.B. über das Absinken der Aktivität) sind ebenfalls als richtig zu werten.
\end{itemize}

\item \subsection{Lösungserwartung:}

Mögliche Vorgehensweise:

$\frac{1}{2}=\frac{N(\tau)}{N_0}=\frac{N_0\cdot 0,5^{\frac{\tau}{c}}}{N_0}=0,5^{\frac{\tau}{c}}\Leftrightarrow\frac{\tau}{c}=1\Leftrightarrow\tau=c$

Die Konstante $c$ entspricht der Halbwertszeit eines radioaktiven Stoffes.

Lösung Abbildung: siehe oben!

\subsection{Lösungsschlüssel:}
\begin{itemize}
	\item Ein Ausgleichspunkt für die Angabe des richtigen Zusammenhangs.
	\item Ein Punkt für das Einzeichnen eines korrekten Verlaufs des Graphen einer möglichen Funktion $N_{\text{neu}}$. Der skizzierte Graph muss den Punkt $(0|N_0)$ enthalten, zwischen dem Graphen der Funktion $N$ und der Zeitachse liegen und als Graph einer (streng) monoton fallenden linksgekrümmten Funktion erkennbar sein.
\end{itemize}
\end{enumerate}}
		\end{langesbeispiel}