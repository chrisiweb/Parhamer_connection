\section{68 - MAT - WS 1.2, WS 1.3, WS 1.4 - Nettomonatseinkommen - Matura 2015/16 2. Nebentermin}

\begin{langesbeispiel} \item[0] %PUNKTE DES BEISPIELS
	
Das Nettomonatseinkommen erwerbstätiger Personen hängt von sozioökonomischen Faktoren wie Alter, Staatsangehörigkeit, Schulbildung, Beschäftigungsausmaß und beruflicher Stellung ab. Die nachstehende Tabelle zeigt Daten zu den Nettomonatseinkommen unselbständig Erwerbstätiger in Österreich im Jahresdurchschnitt 2010 in Abhängigkeit von sozioökonomischen Faktoren. Alle folgenden Aufgabenstellungen beziehen sich auf diese Daten des Jahres 2010.\leer

\begin{tiny}
\begin{tabular}{|C{1.4cm}|C{1.4cm}|C{1.4cm}|C{1.4cm}|C{1.4cm}|C{1.4cm}|C{1.4cm}|C{1.4cm}|}\hline
\multirow{3}{1.1cm}{Merkmale}&\multicolumn{1}{C{1.1cm}|}{\multirow{2}{1.1cm}{unselbst- ständig Erwerbstätige}}&\multicolumn{1}{C{1.1cm}|}{\multirow{2}{1.1cm}{arithme- tisches Mittel}}&\multirow{2}{1.4cm}{\centering$10\,\%$}&\multicolumn{3}{c|}{Quartile}&\multirow{2}{1.1cm}{\centering$90\,\%$}\\ \cline{5-7}
&&&&$25\,\%$&$50\,\%$ (Median)&$75\,\%$& \\ \cline{2-8}
&\multicolumn{1}{C{1.4cm}|}{\mbox{in 1000} \mbox{Personen}}&in Euro&\multicolumn{5}{c|}{verdienen weniger als oder gleich viel wie ... Euro} \\ \hline
\end{tabular}

\begin{tabular}{p{1.4cm}C{1.4cm}C{1.4cm}C{1.4cm}C{1.4cm}C{1.4cm}C{1.4cm}C{1.4cm}}
\textbf{Insgesamt}&3\,407,9&1\,872,7&665,0&1\,188,0&1\,707&2\,303,0&3\,122,0\\\leer

\textbf{Alter}&&&&&&&\\
15-19 Jahre&173,5&799,4&399,0&531,0&730,0&1\,020,0&1\,315,0\\
20-29 Jahre&705,1&1\,487,0&598,0&1\,114,0&1\,506,0&1\,843,0&2\,175,0\\
30-39 Jahre&803,1&1\,885,7&770,0&1\,252,0&1\,778,0&2\,306,0&2\,997,0\\
40-49 Jahre&1\,020,4&2\,086,1&863,0&1\,338,0&1\,892,0&2\,556,0&3\,442,0\\
50-59 Jahre&632,8&2\,205,0&893,0&1\,394,0&1\,977,0&2\,779,0&3\,710,0\\
60+ Jahre&73,0&2\,144,7&258,0&420,0&1\,681,0&3\,254,0&4\,808,0\\
&&&&&&&\\
\multicolumn{3}{l}{\textbf{Höchste abgeschlossene Schulbildung}}&&&&&\\
Pflichtschule&523,4&1\,183,0&439&677,0&1\,104,0&1\,564,0&1\,985,0\\
Lehre&1\,385,2&1\,789,3&833,0&1\,303,0&1\,724,0&2\,143,0&2\,707,0\\
BMS&454,4&1\,777,1&733,0&1\,199,0&1\,677,0&2\,231,0&2\,824,0\\
Höhere Schule&557,2&2\,061,6&590,0&1\,218,0&1\,824,0&2\,624,0&3\,678,0\\
Universität&487,7&2\,723,4&1\,157,0&1\,758,0&2\,480,0&3\,376,0&4\,567,0\\
&&&&&&&\\
\multicolumn{3}{l}{\textbf{Berufliche Stellung}}&&&&&\\
Lehrlinge&134,2&775,3&466,0&551,0&705,0&930,0&1\,167,0\\
Angestellte(r)&1\,800,3&2\,018,1&705,0&1\,222,0&1\,771,0&2\,489,0&3\,550,0\\
Arbeiter(in)&1\,030,9&1\,539,3&627,0&1\,135,0&1\,554,0&1\,922,0&2\,274,0\\
Beamte und Vertragsbedienstete&442,5&2\,391,4&1\,377,0&1\,800,0&2\,295,0&2\,848,0&3\,492,0
\end{tabular}
\begin{singlespace}Datenquelle: Statistik Austria (Hrsg.) (2012). Arbeitsmarktstatistik. Jahresergebnisse 2011. Mikrozensus-Arbeitskräfteerhebung. 
 Wien: Statistik Austria. S. 81 (adaptiert)\end{singlespace}
\end{tiny}

\subsection{Aufgabenstellung:}
\begin{enumerate}
	\item Zeichne in der nachstehenden Grafik ein Diagramm, das die Medianeinkommen der 20- bis 59-Jährigen darstellt! Verwende dafür die auf die Hunderterstelle gerundeten Medianeinkommen.
	
	\begin{center}
		\resizebox{0.8\linewidth}{!}{\newrgbcolor{uququq}{0.25098039215686274 0.25098039215686274 0.25098039215686274}
\psset{xunit=2.0cm,yunit=0.0001cm,algebraic=true,dimen=middle,dotstyle=o,dotsize=4pt 0,linewidth=0.8pt,arrowsize=3pt 2,arrowinset=0.25}
\begin{pspicture*}(-0.7735499280941546,-6798.3471820682125)(4.846644372997255,67316.8397144404)
\multips(0,0)(0,10000.0){8}{\psline[linestyle=dashed,linecap=1,dash=1.5pt 1.5pt,linewidth=0.4pt,linecolor=lightgray]{c-c}(0,0)(4.846644372997255,0)}
\multips(0,0)(1,0){6}{\psline[linestyle=dashed,linecap=1,dash=1.5pt 1.5pt,linewidth=0.4pt,linecolor=lightgray]{c-c}(0,0)(0,67316.8397144404)}
\psaxes[labelFontSize=\scriptstyle,xAxis=true,yAxis=true,labels=none,Dx=1.,Dy=10000.,ticksize=-2pt 0,subticks=2]{->}(0,0)(0.,0.)(4.846644372997255,67316.8397144404)
\begin{scriptsize}
\rput[tl](4.438293735441856,-1568.8084472780674){Alter}
\rput[tl](0.25,-1000){$20-29$}
\rput[tl](1.25,-1000){$30-39$}
\rput[tl](2.25,-1000){$40-49$}
\rput[tl](3.25,-1000){$50-59$}
\rput[tl](-0.4,1000){$1\,000$}
\rput[tl](-0.4,11000){$1\,200$}
\rput[tl](-0.4,21000){$1\,400$}
\rput[tl](-0.4,31000){$1\,600$}
\rput[tl](-0.4,41000){$1\,800$}
\rput[tl](-0.4,51000){$2\,000$}
\rput[tl](-0.4,61000){$2\,200$}
\rput[tl](0.18285024933822672,66054.53726121518){Euro}
\end{scriptsize}
\end{pspicture*}}
	\end{center}
	
	Ist es anhand der Daten in der gegebenen Tabelle möglich, die Nettomonatseinkommen der 20- bis 29-Jährigen und der 30- bis 39-Jährigen in Boxplots (Kastenschaubildern) gegenüberzustellen? Begründe deine Antwort!
	
	\item Jemand hat das arithmetische Mittel aller Nettomonatseinkommen anhand der arithmetischen Mittel der sechs Altersklassen folgendermaßen berechnet:
	
	$\frac{799,4+1\,487,0+1\,885,7+2\,086,1+2\,205,0+2\,144,7}{6}\approx 1\,768,0$
	
	In der gegebenen Tabelle ist allerdings für das arithmetische Mittel aller Einkommen der Wert 1 872,7 angegeben. 
	
	Begründe, warum die oben angeführte Rechnung nicht das richtige Ergebnis liefert, und gib den richtigen Ansatz für die Berechnung an! \leer
	
	Bei der Altersklasse 60+ ist das arithmetische Mittel der Nettomonatseinkommen deutlich (um fast \EUR{500}) größer als das Medianeinkommen dieser Altersklasse. Gib eine daraus ableitbare Schlussfolgerung im Hinblick auf sehr niedrige bzw. sehr hohe Nettomonatseinkommen in dieser Altersklasse an!
	
	\item \fbox{A} Gib die Werte des 1. und des 3. Quartils der Nettomonatseinkommen der unselbstständig Erwerbstätigen mit Pflichtschulabschluss als höchste abgeschlossene Schulbildung an!
	
	1. Quartil: \rule{3cm}{0.3pt}
	
	3. Quartil: \rule{3cm}{0.3pt}\leer
	
	Der Interquartilsabstand ist die Differenz von 3. und 1. Quartil. 
	
	Ein Experte behauptet: "`Mit zunehmender höchster abgeschlossener Schulbildung, die über einen Pflichtschulabschluss hinausgeht, nimmt auch der Interquartilsabstand der Nettomonatseinkommen zu."' 
	
	Verifiziere oder widerlege diese Behauptung und verwende dazu die Daten in der gegebenen Tabelle!
 
	\item Die Daten in der gegebenen Tabelle zeigen, dass ungefähr 53\,\% der unselbständig Erwerbstätigen Angestellte und ungefähr 30\,\% Arbeiter/innen sind. 
	
	In einem Kommentar zum Arbeitsmarktbericht ist zu lesen: "`Der relative Anteil der Angestellten ist um ungefähr 23\,\% höher als der relative Anteil der Arbeiter/innen."' Ist diese Aussage richtig? Begründe deine Antwort! 
	
	Überprüfe folgende Aussagen über Nettomonatseinkommen anhand der Daten in der gegebenen Tabelle! Kreuze die beiden zutreffenden Aussagen an!\leer
	
	\multiplechoice[5]{  %Anzahl der Antwortmoeglichkeiten, Standard: 5
					L1={Angestellte verdienen im Durchschnitt um über \EUR{500} mehr als Arbeiter/innen.},   %1. Antwortmoeglichkeit 
					L2={Höchstens ein Viertel der Arbeiter/innen verdient mehr als \EUR{1.922}. },   %2. Antwortmoeglichkeit
					L3={Die Spannweite des Nettomonatseinkommens kann anhand der Daten in der Tabelle nicht exakt angegeben werden.},   %3. Antwortmoeglichkeit
					L4={Drei Viertel der Lehrlinge verdienen mindestens \EUR{930}.},   %4. Antwortmoeglichkeit
					L5={Genau die Hälfte der Beamten und Vertragsbediensteten verdient exakt \EUR{1.800}.},	 %5. Antwortmoeglichkeit
					L6={},	 %6. Antwortmoeglichkeit
					L7={},	 %7. Antwortmoeglichkeit
					L8={},	 %8. Antwortmoeglichkeit
					L9={},	 %9. Antwortmoeglichkeit
					%% LOESUNG: %%
					A1=2,  % 1. Antwort
					A2=3,	 % 2. Antwort
					A3=0,  % 3. Antwort
					A4=0,  % 4. Antwort
					A5=0,  % 5. Antwort
					}

\end{enumerate}

\antwort{
\begin{enumerate}
	\item \subsection{Lösungserwartung:} 

Mögliches Diagramm:

\begin{center}
		\resizebox{0.8\linewidth}{!}{\newrgbcolor{uququq}{0.25098039215686274 0.25098039215686274 0.25098039215686274}
\psset{xunit=2.0cm,yunit=0.0001cm,algebraic=true,dimen=middle,dotstyle=o,dotsize=4pt 0,linewidth=0.8pt,arrowsize=3pt 2,arrowinset=0.25}
\begin{pspicture*}(-0.7735499280941546,-6798.3471820682125)(4.846644372997255,67316.8397144404)
\multips(0,0)(0,10000.0){8}{\psline[linestyle=dashed,linecap=1,dash=1.5pt 1.5pt,linewidth=0.4pt,linecolor=lightgray]{c-c}(0,0)(4.846644372997255,0)}
\multips(0,0)(1,0){6}{\psline[linestyle=dashed,linecap=1,dash=1.5pt 1.5pt,linewidth=0.4pt,linecolor=lightgray]{c-c}(0,0)(0,67316.8397144404)}
\psaxes[labelFontSize=\scriptstyle,xAxis=true,yAxis=true,labels=none,Dx=1.,Dy=10000.,ticksize=-2pt 0,subticks=2]{->}(0,0)(0.,0.)(4.846644372997255,67316.8397144404)
\pspolygon[linewidth=0.8pt,linecolor=uququq,fillcolor=uququq,fillstyle=solid,opacity=0.69](0.,0.)(0.,25000.)(1.,25000.)(1.,0.)
\pspolygon[linewidth=0.8pt,linecolor=uququq,fillcolor=uququq,fillstyle=solid,opacity=0.69](1.,0.)(1.,40000.)(2.,40000.)(2.,0.)
\pspolygon[linewidth=0.8pt,linecolor=uququq,fillcolor=uququq,fillstyle=solid,opacity=0.69](2.,0.)(2.,45000.)(3.,45000.)(3.,0.)
\pspolygon[linewidth=0.8pt,linecolor=uququq,fillcolor=uququq,fillstyle=solid,opacity=0.69](3.,0.)(3.,50000.)(4.,50000.)(4.,0.)
\begin{scriptsize}
\rput[tl](4.438293735441856,-1568.8084472780674){Alter}
\rput[tl](0.25,-1000){$20-29$}
\rput[tl](1.25,-1000){$30-39$}
\rput[tl](2.25,-1000){$40-49$}
\rput[tl](3.25,-1000){$50-59$}
\rput[tl](-0.4,1000){$1\,000$}
\rput[tl](-0.4,11000){$1\,200$}
\rput[tl](-0.4,21000){$1\,400$}
\rput[tl](-0.4,31000){$1\,600$}
\rput[tl](-0.4,41000){$1\,800$}
\rput[tl](-0.4,51000){$2\,000$}
\rput[tl](-0.4,61000){$2\,200$}
\rput[tl](0.18285024933822672,66054.53726121518){Euro}
\end{scriptsize}
\end{pspicture*}}
	\end{center}
 
Die Gegenüberstellung der Nettomonatseinkommen in Boxplots (Kastenschaubildern) ist anhand der gegebenen Daten nicht möglich, da die niedrigsten und die höchsten Nettomonatseinkommen (Minimum und Maximum) in der Tabelle nicht angegeben sind.

	\subsection{Lösungsschlüssel:}
	\begin{itemize}
		\item Ein Punkt für ein korrektes Diagramm.
		\item Ein Punkt für eine (sinngemäß) richtige Begründung. 
	\end{itemize}
	
	\item \subsection{Lösungserwartung:}
			
Mögliche Begründung:

Die angeführte Rechnung ist falsch, da die Anzahl der Erwerbstätigen in den einzelnen Altersklassen nicht berücksichtigt ist.

Ein richtiger Ansatz lautet:

$\frac{799,4\cdot 173,5+1\,487\cdot 705,1+1\,885,7\cdot 803,1+2\,086,1\cdot 1\,020,4+2\,205\cdot 632,8+2\,144,7\cdot 73}{3\,407,9}$\leer

Mögliche Begründung:

In der Altersklasse 60+ weichen die sehr hohen  Nettomonatseinkommen viel stärker vom Medianeinkommen ab als die sehr niedrigen Einkommen.
	
	\subsection{Lösungsschlüssel:}
	
\begin{itemize}
	\item Ein Punkt für eine (sinngemäß) richtige Begründung und einen korrekten Ansatz.
	\item Ein Punkt für eine (sinngemäß) richtige Begründung. 
\end{itemize}

\item \subsection{Lösungserwartung:}
			
1. Quartil: \EUR{677,0}

3. Quartil: \EUR{1.564,0}\leer

Die Behauptung ist richtig, wie die folgenden Interquartilsabstände zeigen:

Lehrabschluss: \EUR{840}

BMS-Abschluss: \EUR{1.032}

Abschluss einer höheren Schule: \EUR{1.406}

Universitätsabschluss: \EUR{1.618}
	
	\subsection{Lösungsschlüssel:}
	
\begin{itemize}
	\item Ein Ausgleichspunkt für die Angabe beider korrekten Werte.
	\item Ein Punkt für eine (sinngemäß) richtige Begründung. 
\end{itemize}

\item \subsection{Lösungserwartung:}
			
Die Aussage ist nicht richtig.

Mögliche Begründungen:

Für diesen Vergleich muss der relative Anteil (in Prozent) der Arbeiter/innen als Grundwert verwendet werden.

oder:

In der Aussage wurde ein relativer Zuwachs (in Prozent) mit einem Zuwachs von Prozentpunkten verwechselt.
	
	\subsection{Lösungsschlüssel:}
	
\begin{itemize}
	\item Ein Punkt für die Angabe, dass die Aussage nicht richtig ist, und eine (sinngemäß) richtige Begründung. Eine richtige Berechnung des relativen Anteils (ca. 75\,\% mehr Angestellte) ist auch als richtig zu werten. 
	\item Ein Punkt ist genau dann zu geben, wenn ausschließlich die beiden laut Lösungserwartung richtigen Aussagen angekreuzt sind.
\end{itemize}

\end{enumerate}}
		\end{langesbeispiel}