\section{89 - MAT - AG 2.3, AN 1.2, AN 4.3, AN 4.2 - Eingenschaften einer Polynomfunktion dritten Grades - Matura 2017/18}

\begin{langesbeispiel} \item[4] %PUNKTE DES BEISPIELS
			Gegeben ist eine Polynomfunktion dritten Grades $f$ mit der Funktionsgleichung $f(x)=a\cdot x^3+b\cdot x$, wobei die Koeffizienten $a,b\in\mathbb{R}\backslash\{0\}$ sind.

\subsection{Aufgabenstellung:}
\begin{enumerate}
	\item Begr�nde, warum die Funktion $f$ genau drei verschiedene reelle Nullstellen hat, wenn die Koeffizienten $a$ und $b$ unterschiedliche Vorzeichen haben!
	
	\fbox{A} Die Steigung der Tangente an den Graphen von $f$ an der Stelle $x=0$ entspricht dem Wert des Koeffizienten $b$. Begr�nde, warum diese Aussage wahr ist!
	
	\item Gib eine Beziehung zwischen den Koeffizienten $a$ und $b$ an, sodass $\displaystyle\int^1_0f(x)$d$x=0$ gilt!
	
	Begr�nde, warum aus der Annahme $\displaystyle\int^1_0f(x)$d$x=0$ folgt, dass $f$ eine Nullstelle im Intervall $(0;1)$ hat, und skizziere einen m�glichen Graphen einer solchen Funktion $f$ im nachstehenden Koordinatensystem.\leer
	
	\begin{center}
		\resizebox{0.6\linewidth}{!}{\newrgbcolor{uququq}{0.25098039215686274 0.25098039215686274 0.25098039215686274}
\psset{xunit=1.7cm,yunit=1.7cm,algebraic=true,dimen=middle,dotstyle=o,dotsize=5pt 0,linewidth=1.6pt,arrowsize=3pt 2,arrowinset=0.25}
\begin{pspicture*}(-2.331917201924157,-4.3427023216928)(2.3963904714973214,2.291781743631572)
\multips(0,-4)(0,1.0){7}{\psline[linestyle=dashed,linecap=1,dash=1.5pt 1.5pt,linewidth=0.4pt,linecolor=darkgray]{c-c}(-2.331917201924157,0)(2.3963904714973214,0)}
\multips(-2,0)(1.0,0){5}{\psline[linestyle=dashed,linecap=1,dash=1.5pt 1.5pt,linewidth=0.4pt,linecolor=darkgray]{c-c}(0,-4.3427023216928)(0,2.291781743631572)}
\psaxes[labelFontSize=\scriptstyle,xAxis=true,yAxis=true,Dx=1.,Dy=1.,ticksize=-4pt 0,subticks=2]{->}(0,0)(-2.331917201924157,-4.3427023216928)(2.3963904714973214,2.291781743631572)[x,140] [f(x),-40]
\antwort{\pscustom[linewidth=0.8pt,linecolor=uququq,fillcolor=uququq,fillstyle=solid,opacity=0.5]{\psplot{0.}{1.}{-4.3478260869565215*x^(3.0)-2.6289784566965458E-51*x^(2.0)+2.4456521739130435*x}\lineto(1.,0)\lineto(0.,0)\closepath}
\psplot[linewidth=2.pt,plotpoints=200]{-2.331917201924157}{2.3963904714973214}{-4.3478260869565215*x^(3.0)-2.6289784566965458E-51*x^(2.0)+2.4456521739130435*x}
\begin{scriptsize}
\rput[bl](-0.8218294109361456,1.6151303837353055){$f$}
\end{scriptsize}}
\end{pspicture*}}
	\end{center}

	\end{enumerate}
	
	\antwort{
\begin{enumerate}
	\item \subsection{L�sungserwartung:} 

M�gliche Begr�ndung:

Berechnung der Nullstellen: $a\cdot x^3+b\cdot x=x\cdot (a\cdot x^2+b)=0$\\
Eine Nullstelle ist daher $x_1=0$.

Berechnung weiterer Nullstellen: $a\cdot x^2+b=0\Rightarrow x^2=-\frac{b}{a}$

Wenn die Koeffizienten $a$ und $b$ unterschiedliche Vorzeichen haben, dann gilt: $-\frac{b}{a}>0$.

Damit hat diese Gleichung zwei verschiedene reelle L�sungen und die Funktion $f$ hat insgesamt drei verschiedene Nullstellen.\leer

M�gliche Begr�ndung:

Der Wert der Steigung der Tangente an den Graphen von $f$ an einer Stelle $x$ entspricht dem Wert $f'(x)$.

$f'(x)=3\cdot a\cdot x^2+b \Rightarrow f'(0)=b$

	\item \subsection{L�sungserwartung:} 
	
	M�gliche Vorgehensweise:
	
	$\displaystyle\int^1_0(a\cdot x^3+b\cdot x)$d$x=\left(a\cdot\frac{x^4}{4}+b\cdot\frac{x^2}{2}\right)\big|^1_0=0 \Rightarrow a=-2\cdot b$\leer
	
	M�gliche Begr�ndung:
	
	Das bestimmte Integral liefert die Summe der orientierten Fl�cheninhalte, die vom Graphen von $f$ und von der x-Achse begrenzt werden. H�tte $f$ keine Nullstelle im Intervall $(0;1)$, dann w�rde der Graph von $f$ in diesem Intervall entweder zur G�nze oberhalb der x-Achse (mit $f(x)>0$ f�r alle $x\in(0;1)$) oder zur G�nze unterhalb der x-Achse (mit $f(x)<0$ f�r alle $x\in(0;1)$) verlaufen. Somit w�re das bestimmte Integral von $f$ im Intervall $(0;1)$ entweder gr��er oder kleiner null, aber keinesfalls gleich null.
	
	F�r einen m�glichen Graphen von $f$ siehe oben!
\end{enumerate}}
	
	\end{langesbeispiel}