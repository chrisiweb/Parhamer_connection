\section{09 - MAT - AG 2.3, FA 1.4, FA 1.6, FA 1.7, FA 2.3 - Gewinnfunktion - BIFIE Aufgabensammlung}

\begin{langesbeispiel} \item[0] %PUNKTE DES BEISPIELS
In einem Unternehmen werden die Entwicklungen der Kosten $K$ und des Erlöses $E$ in Geldeinheiten (GE) bei variabler Menge $x$ in Mengeneinheiten (ME) beobachtet. Als Modellfunktionen werden die Erlösfunktion $E$ mit\\ 
$E(x)=-0,05\cdot x^2+1,5\cdot x$ und eine Kostenfunktion $K$ mit $K(x)=0,3\cdot x+5,4$ angewendet. Alle produzierten Mengeneinheiten werden vom Unternehmen abgesetzt.%Aufgabentext

\begin{aufgabenstellung}
\item %Aufgabentext

\Subitem{Berechne die Koordinaten der Schnittpunkte der Funktionsgraphen von $E$ und $K$.} %Unterpunkt1
\Subitem{Beschreibe, welche Informationen die Koordinaten dieser Schnittpunkte für den Gewinn des Unternehmens liefern.} %Unterpunkt2

\item

\Subitem{Zeichne den Graphen der Gewinnfunktion $G$ in die untenstehende Abbildung ein.

\psset{xunit=0.3cm,yunit=0.6cm,algebraic=true,dimen=middle,dotstyle=o,dotsize=5pt 0,linewidth=0.8pt,arrowsize=3pt 2,arrowinset=0.25}
\begin{pspicture*}(-3.6642696629213383,-6.136216216216208)(41.063370786516806,13.667027027027025)
\multips(0,-6)(0,1.0){40}{\psline[linestyle=dashed,linecap=1,dash=1.5pt 1.5pt,linewidth=0.4pt,linecolor=gray]{c-c}(0,0)(51.063370786516806,0)}
\multips(0,0)(5.0,0){21}{\psline[linestyle=dashed,linecap=1,dash=1.5pt 1.5pt,linewidth=0.4pt,linecolor=gray]{c-c}(0,-6)(0,13.667027027027025)}
\psaxes[labelFontSize=\scriptstyle,xAxis=true,yAxis=true,showorigin=false,Dx=5.,Dy=1.,ticksize=-2pt 0,subticks=0]{->}(0,0)(-3.6642696629213383,-6.136216216216208)(51.063370786516806,13.667027027027025)
\begin{scriptsize}
\rput[tl](0.6501123595505673,13.39081081081081){Geldeinheiten (GE)}
\rput[tl](30.38741573033705,0.774054054054059){Mengeneinheiten (ME)}
\end{scriptsize}
\psplot[linewidth=2.pt,linestyle=dashed,dash=3pt 3pt,plotpoints=200]{-0}{30}{-0.05*x^(2.0)+1.5*x}
\rput[tl](23.404044943820207,8.94864864864865){E}
\psplot[linewidth=2.pt,,plotpoints=200]{0}{30}{0.3*x+5.4}
\rput[tl](23.285842696629196,12.016756756756756){K}
\antwort{\psplot[linewidth=2.pt,,plotpoints=200]{0}{22}{-0.05*x^(2.0)+1.2*x-5.4}
\rput[tl](16.439550561797738,1.6502702702702747){G}
\psline[linewidth=2.pt](15.,11.25)(15.,9.9)}
\end{pspicture*}}

\Subitem{Markiere in der Abbildung den Gewinn im Erlösmaximum.}

\item

\Subitem{Berechne den zu erwartenden Gewinn, wenn 13 Mengeneinheiten produziert und abgesetzt werden.}

Bei der gegebenen Kostenfunktion $K$ gibt der Wert 5,4 die Fixkosten an.

\Subitem{Im folgenden werden Aussagen getroffen, die ausschließlich die Änderungen der Fixkosten in Betracht ziehen. Kreuze die für den gegebenen Sachverhalt zutreffende(n) Aussage(n) an!
\vspace{0,2cm}

\multiplechoice[5]{  %Anzahl der Antwortmoeglichkeiten, Standard: 5
				L1={Eine Senkung der Fixkosten bewirkt eine breitere Gewinnzone, d.h., der Abstand zwischen den beiden Nullstellen der Gewinnfunktion wird größer.},   %1. Antwortmoeglichkeit 
				L2={Eine Veränderung der Fixkosten hat keine Auswirkung auf diejenigen Stückzahl, bei der der höchste Gewinn erzielt wird.},   %2. Antwortmoeglichkeit
				L3={Eine Erhöhung der Fixkosten steigert die Höhe des maximalen Gewinns.},   %3. Antwortmoeglichkeit
				L4={Eine Veränderung der Fixkosten hat keine Auswirkung auf die Höhe des maximalen Gewinns.},   %4. Antwortmoeglichkeit
				L5={Eine Senkung der Fixkosten führt zu einer Erhöhung des Gewinns.},	 %5. Antwortmoeglichkeit
				L6={},	 %6. Antwortmoeglichkeit
				L7={},	 %7. Antwortmoeglichkeit
				L8={},	 %8. Antwortmoeglichkeit
				L9={},	 %9. Antwortmoeglichkeit
				%% LOESUNG: %%
				A1=1,  % 1. Antwort
				A2=2,	 % 2. Antwort
				A3=5,  % 3. Antwort
				A4=0,  % 4. Antwort
				A5=0,  % 5. Antwort
				}}

\end{aufgabenstellung}

\begin{loesung}
\item \subsection{Lösungserwartung:} 

\Subitem{$-0,05\cdot x^2+1,5\cdot x=0,3\cdot x+5,4$
	
	$-0,05\cdot x^2+1,2\cdot x-5,4=0$
	
	Die Lösung der quadratischen Gleichung führt zu den Lösungen $x_1=6$ und $x_2=18 \rightarrow S_1\,(6/7,2)$ und $S_2\,(18/10,8)$.} %Lösung von Unterpunkt1
\Subitem{Mögliche Interpretationen: 
	
	Für die Mengen $x_1$ und $x_2$ sind Erlös und Kosten jeweils gleich groß, der Gewinn ist daher null.
	
	Für die Stückzahlen $x_1$ und $x_2$ wird kein Gewinn erzielt.
	
	Für den Stückzahlbereich $(x_1;x_2)$ wird ein Gewinn erzielt.} %%Lösung von Unterpunkt2

\setcounter{subitemcounter}{0}
\subsection{Lösungsschlüssel:}
 
\Subitem{Ein Punkt für die richtigen Koordinaten.} %Lösungschlüssel von Unterpunkt1
\Subitem{Ein Punkt für eine adäquate Interpretation.} %Lösungschlüssel von Unterpunkt2

\item \subsection{Lösungserwartung:} 

\Subitem{Lösung: siehe Grafik.} %Lösung von Unterpunkt1
\Subitem{Lösung: siehe Grafik.} %%Lösung von Unterpunkt2

\setcounter{subitemcounter}{0}
\subsection{Lösungsschlüssel:}
 
\Subitem{Ein Punkt für den richtigen Graphen der Gewinnfunktion.} %Lösungschlüssel von Unterpunkt1
\Subitem{Ein Punkt für die richtige Gewinnmarkierung.} %Lösungschlüssel von Unterpunkt2

\item \subsection{Lösungserwartung:} 

\Subitem{$G(x)=-0,05\cdot x^2+1,2\cdot x-5,4$
	
	$G(13)=1,75$ GE} %Lösung von Unterpunkt1
\Subitem{MC-Antworten: 1,2,5} %%Lösung von Unterpunkt2

\setcounter{subitemcounter}{0}
\subsection{Lösungsschlüssel:}
 
\Subitem{Ein Punkt für den zu erwartenden Gewinn.} %Lösungschlüssel von Unterpunkt1
\Subitem{Ein Punkt für die richtigen MC-Antworten.} %Lösungschlüssel von Unterpunkt2

\end{loesung}

\end{langesbeispiel}