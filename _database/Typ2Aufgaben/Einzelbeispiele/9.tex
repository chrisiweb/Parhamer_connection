\section{09 - MAT - AG 2.3, FA 1.4, FA 1.6, FA 1.7, FA 2.3 - Gewinnfunktion - BIFIE Aufgabensammlung}

\begin{langesbeispiel} \item[0] %PUNKTE DES BEISPIELS
				In einem Unternehmen werden die Entwicklungen der Kosten $K$ und des Erl�ses $E$ in Geldeinheiten $(GE)$ bei variabler Menge $x$ in Mengeneinheiten (ME) beobachtet. Als Modellfunktionen werden die Erl�sfunktion $E$ mit \mbox{$E(x)=-0,05\cdot x�+1,5\cdot x$} und eine Kostenfunktion $K$ mit $K(x)=0,3\cdot x+5,4$ angewendet. Alle produzierten Mengeneinheiten werden vom Unternehmen abgesetzt.
				
\subsection{Aufgabenstellung:}
\begin{enumerate}
	\item Berechne die Koordinaten der Schnittpunkte der Funktionsgraphen von $E$ und $K$! Beschreibe, welche Informationen die Koordinaten dieser Schnittpunkte f�r den Gewinn des Unternehmens liefern!
	
	\item Zeichne den Graphen der Gewinnfunktion $G$ in die untenstehende Abbildung ein! Markiere in der Abbildung den Gewinn im Erl�smaximum!
	\leer
	
	\psset{xunit=0.3cm,yunit=0.6cm,algebraic=true,dimen=middle,dotstyle=o,dotsize=5pt 0,linewidth=0.8pt,arrowsize=3pt 2,arrowinset=0.25}
\begin{pspicture*}(-3.6642696629213383,-6.136216216216208)(41.063370786516806,13.667027027027025)
\multips(0,-6)(0,1.0){40}{\psline[linestyle=dashed,linecap=1,dash=1.5pt 1.5pt,linewidth=0.4pt,linecolor=lightgray]{c-c}(0,0)(51.063370786516806,0)}
\multips(0,0)(5.0,0){21}{\psline[linestyle=dashed,linecap=1,dash=1.5pt 1.5pt,linewidth=0.4pt,linecolor=lightgray]{c-c}(0,-6)(0,13.667027027027025)}
\psaxes[labelFontSize=\scriptstyle,xAxis=true,yAxis=true,Dx=5.,Dy=1.,ticksize=-2pt 0,subticks=2]{->}(0,0)(-3.6642696629213383,-6.136216216216208)(51.063370786516806,13.667027027027025)
\begin{scriptsize}
\rput[tl](0.6501123595505673,13.39081081081081){Geldeinheiten (GE)}
\rput[tl](30.38741573033705,0.774054054054059){Mengeneinheiten (ME)}
\end{scriptsize}
\psplot[linewidth=2.pt,linestyle=dashed,dash=3pt 3pt,plotpoints=200]{-0}{30}{-0.05*x^(2.0)+1.5*x}
\rput[tl](23.404044943820207,8.94864864864865){E}
\psplot[linewidth=2.pt,,plotpoints=200]{0}{30}{0.3*x+5.4}
\rput[tl](23.285842696629196,12.016756756756756){K}
\antwort{\psplot[linewidth=2.pt,,plotpoints=200]{0}{22}{-0.05*x^(2.0)+1.2*x-5.4}
\rput[tl](16.439550561797738,1.6502702702702747){G}}
\end{pspicture*}

\item Berechne den zu erwartenden Gewinn, wenn 13 Mengeneinheiten produziert und abgesetzt werden!

Bei der gegebenen Kostenfunktion $K$ gibt der Wert 5,4 die Fixkosten an. Im folgenden werden Aussagen getroffen, die ausschlie�lich die �nderungen der Fixkosten in Betracht ziehen. Kreuze die f�r den gegebenen Sachverhalt zutreffende(n) Aussage(n) an!
\leer

\multiplechoice[5]{  %Anzahl der Antwortmoeglichkeiten, Standard: 5
				L1={Eine Senkung der Fixkosten bewirkt eine breitere Gewinnzone, d.h., der Abstand zwischen den beiden Nullstellen der Gewinnfunktion wird gr��er.},   %1. Antwortmoeglichkeit 
				L2={Eine Ver�nderung der Fixkosten hat keine Auswirkung auf diejenigen St�ckzahl, bei der der h�chste Gewinn erzielt wird.},   %2. Antwortmoeglichkeit
				L3={Eine Erh�hung der Fixkosten steigert die H�he des maximalen Gewinns.},   %3. Antwortmoeglichkeit
				L4={Eine Ver�nderung der Fixkosten hat keine Auswirkung auf die H�he des maximalen Gewinns.},   %4. Antwortmoeglichkeit
				L5={Eine Senkung der Fixkosten f�hrt zu einer Erh�hung des Gewinns.},	 %5. Antwortmoeglichkeit
				L6={},	 %6. Antwortmoeglichkeit
				L7={},	 %7. Antwortmoeglichkeit
				L8={},	 %8. Antwortmoeglichkeit
				L9={},	 %9. Antwortmoeglichkeit
				%% LOESUNG: %%
				A1=1,  % 1. Antwort
				A2=2,	 % 2. Antwort
				A3=5,  % 3. Antwort
				A4=0,  % 4. Antwort
				A5=0,  % 5. Antwort
				}
	
				\end{enumerate}
\antwort{\subsection{L�sungserwartung:}
\begin{enumerate}
	\item $-0,05\cdot x�+1,5\cdot x=0,3\cdot x+5,4$
	
	$-0,05\cdot x�+1,2\cdot x-5,4=0$
	
	Die L�sung der quadratischen Gleichung f�hrt zu den L�sungen $x_1=6$ und $x_2=18 \rightarrow S_1\,(6/7,2)$ und $S_2\,(18/10,8)$.
	
	Reflexion beispielsweise: 
	
	F�r die Mengen $x_1$ und $x_2$ sind Erl�s und Kosten jeweils gleich gro�, der Gewinn ist daher null.
	
	F�r die St�ckzahlen $x_1$ und $x_2$ wird kein Gewinn erzielt.
	
	F�r den St�ckzahlbereich $(x_1;x_2)$ wird ein Gewinn erzielt.
	
	\item Eine der m�glichen Markierungen f�r den Gewinn reicht in der L�sung aus. L�sungsvorschlag siehe Grafik oben.
	
	$G(x)$ einzeichnen, Markierung des Gewinns
	
	\item $G(x)=-0,05\cdot x�+1,2\cdot x-5,4$
	
	$G(13)=1,75$ GE (Geldeinheiten)
	
	L�sungen Multiple Choice: siehe oben
	\end{enumerate}}
\end{langesbeispiel}