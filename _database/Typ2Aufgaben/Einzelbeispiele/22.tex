\section{22 - MAT - AN 1.3, FA 1.5, FA 1.8, WS 2.3, WS 3.2 - Zehnkampf - BIFIE Aufgabensammlung}

\begin{langesbeispiel} \item[0] %PUNKTE DES BEISPIELS
				Die "`K�nigsdisziplin"' der Leichtathletik ist bei den M�nnern der Zehnkampf. Dabei erh�lt jeder Athlet in jeder der 10 Disziplinen Punkte, die f�r jede Disziplin nach einer eigenen Formel errechnet werden. F�r den Weitsprung gilt die Formel $P=0,14354\cdot(x-220)^{1,4}$. Dabei ist $x$ die Sprungweite in cm und $P$ die Punktezahl (auf Ganze gerundet).
				
Im Bewerb sind 3 Spr�nge erlaubt. Gewertet wird der weiteste fehlerfreie Sprung. Als Fehlversuch gilt in erster Linie das �bertreten beim Absprungbalken. Dies passiert in ca. 1 von 20 Versuchen. Der Weltrekord im Weitsprung liegt bei 895 cm. Der Weltrekord im Zehnkampf wurde von Roman Sebrle 2001 beim Leichtathletikmeeting in G�tzis aufgestellt und liegt bei 9\,026 Punkten. Seine Weitsprungleistung betrug dabei 811 cm. 

\subsection{Aufgabenstellung:}
\begin{enumerate}
	\item Berechne, wie viele Punkte Roman Sebrle mehr erhalten h�tte, wenn er die Weltrekordweite gesprungen w�re!
	
	Begr�nde mit der Formel, warum erst Spr�nge ab 220 cm einen Punktwert ergeben.
	
	\item Eine Sprungleistungssteigerung um 84 cm bringt nicht von jedem Ausgangswert den gleichen durchschnittlichen Punktezuwachs (in Punkten/cm). Zeige das f�r die Intervalle $[500\,\text{cm}; 584\,\text{cm}]$ und $[811\,\text{cm};895\,\text{cm}]$ durch Rechnung!
	
	Begr�nde mithilfe der untenstehenden Graphik, warum ein absolut gleicher Weitenzuwachs f�r gr��ere Ausgangswerte mehr Punkte bringt als f�r kleinere Ausgangswerte!
	
	\begin{center}
		\resizebox{0.8\linewidth}{!}{\psset{xunit=0.0066cm,yunit=0.005cm,algebraic=true,dimen=middle,dotstyle=o,dotsize=5pt 0,linewidth=0.8pt,arrowsize=3pt 2,arrowinset=0.25}
\begin{pspicture*}(-120.06849315068379,-110.34782608693672)(1323.2876712328716,1773.7391304344762)
\multips(0,0)(0,200.0){10}{\psline[linestyle=dashed,linecap=1,dash=1.5pt 1.5pt,linewidth=0.4pt,linecolor=lightgray]{c-c}(0,0)(1323.2876712328716,0)}
\multips(0,0)(200.0,0){8}{\psline[linestyle=dashed,linecap=1,dash=1.5pt 1.5pt,linewidth=0.4pt,linecolor=lightgray]{c-c}(0,0)(0,1773.7391304344762)}
\psaxes[labelFontSize=\scriptstyle,xAxis=true,yAxis=true,xlabelFactor={\text{\,cm}},Dx=200.,Dy=200.,ticksize=-2pt 0,subticks=2]{->}(0,0)(0.,0.)(1323.2876712328716,1773.7391304344762)
\psplot[linewidth=1.2pt,plotpoints=200]{220.0000000684934}{1323.2876712328716}{0.14354*(x-220.0)^(1.4)}
\psline[linestyle=dashed,dash=4pt 2pt](0.,1312.194532507742)(895.,1312.194532507742)
\psline[linestyle=dashed,dash=4pt 2pt](895.,1312.194532507742)(895.,0.)
\psline[linestyle=dashed,dash=4pt 2pt](811.,0.)(811.,1089.4201785316532)
\psline[linestyle=dashed,dash=4pt 2pt](811.,1089.4201785316532)(0.,1089.4201785316532)
\psline[linestyle=dashed,dash=4pt 2pt](584.,0.)(584.,552.7323443269263)
\psline[linestyle=dashed,dash=4pt 2pt](584.,552.7323443269263)(0.,552.7323443269263)
\psline[linestyle=dashed,dash=4pt 2pt](500.,0.)(500.,382.8196181903659)
\psline[linestyle=dashed,dash=4pt 2pt](500.,382.8196181903659)(0.,382.8196181903659)
\begin{scriptsize}
\rput[tl](30.35616438356212,1700.1739130431847){Punkte}
\rput[tl](1000.8356164383522,80.60869565215339){Sprungweite}
\psdots[dotsize=3pt 0,dotstyle=*](220.,0.)
\rput[bl](210.95890410958873,28.608695652169658){$A$}
\end{scriptsize}
\end{pspicture*}}
	\end{center}
	
	\item Berechne die Wahrscheinlichkeit, dass ein Athlet die Weitsprungpunkte bei seinem Zehnkampf ohne Fehlversuch erh�lt!
	
	Durch bessere Trainingsmethoden kann dieser Wahrscheinlichkeitswert erh�ht werden, indem die Fehlerquote von $1:20$ gesenkt wird, etwa auf $1:n$.
	
	Wenn unter $n$ Spr�ngen nur ein Fehlversuch dabei ist, ergibt sich eine Erfolgsquote von $\frac{n-1}{n}$. Begr�nde damit, warum die oben genannte Wahrscheinlichkeit nie 1 sein kann!
	
						\end{enumerate}\leer
				
\antwort{\subsection{L�sungserwartung:}
\begin{enumerate}
	\item $f(895)\approx 1\,312$, $f(811)\approx 1\,089$. Er h�tte um 223 Punkte mehr erzielt.
	
	Die Basis (der Radikand) wird erst ab $220\,\text{cm}\geq 0$. (oder eine sinngem��ge Formulierung.
	\item $\frac{f(584)-f(500)}{584-500}$ bzw. $\frac{f(895)-f(811)}{895-811}$
	
	im ersten Intervall: ca. 2,02 Punkte/cm
	
	im zweiten Intervall: ca. 2,65 Punkte/cm
	
	Begr�ndung: $f$ ist streng monoton wachsend und steig im zweiten Intervall schneller. (Jede sinngem�� formulierte Antwort ist richtig.)
	
	\item $X$ ... Anzahl der Fehlversuche
	
	$p=\frac{1}{20}$, $q=p-1=\frac{19}{20}$
	
		$P(X=0)=...=\left(\frac{19}{20}\right)^3=0,857$, d.h. mit einer Wahrscheinlichkeit von 85,7\,\%.
	
	Begr�ndung: Da der Z�hler immer kleiner als der Nenner ist, ist $\frac{n-1}{n}<1$. Daher muss auch die 3. Potenz $<1$ sein.
	
	Oder:
	
	Sobald die Wahrscheinlichkeit f�r einen Fehlversuch gr��er als 0 ist, muss die Wahrscheinlichkeit, dass 3 Spr�nge ohne Fehlversuch gelingen, kleiner als 1 sein. (Sinngem��e Argumentation m�glich!)
		\end{enumerate}}
		\end{langesbeispiel}