\section{22 - MAT - AN 1.3, FA 1.5, FA 1.8, WS 2.3, WS 3.2 - Zehnkampf - BIFIE Aufgabensammlung}

\begin{langesbeispiel} \item[0] %PUNKTE DES BEISPIELS
				Die "`Königsdisziplin"' der Leichtathletik ist bei den Männern der Zehnkampf. Dabei erhält jeder Athlet in jeder der 10 Disziplinen Punkte, die für jede Disziplin nach einer eigenen Formel errechnet werden. Für den Weitsprung gilt die Formel $P=0,14354\cdot(x-220)^{1,4}$. Dabei ist $x$ die Sprungweite in cm und $P$ die Punktezahl (auf Ganze gerundet).
				
Im Bewerb sind 3 Sprünge erlaubt. Gewertet wird der weiteste fehlerfreie Sprung. Als Fehlversuch gilt in erster Linie das Übertreten beim Absprungbalken. Dies passiert in ca. 1 von 20 Versuchen. Der Weltrekord im Weitsprung liegt bei 895 cm. Der Weltrekord im Zehnkampf wurde von Roman Sebrle 2001 beim Leichtathletikmeeting in Götzis aufgestellt und liegt bei 9\,026 Punkten. Seine Weitsprungleistung betrug dabei 811 cm. 

\subsection{Aufgabenstellung:}
\begin{enumerate}
	\item Berechne, wie viele Punkte Roman Sebrle mehr erhalten hätte, wenn er die Weltrekordweite gesprungen wäre!
	
	Begründe mit der Formel, warum erst Sprünge ab 220 cm einen Punktwert ergeben.
	
	\item Eine Sprungleistungssteigerung um 84 cm bringt nicht von jedem Ausgangswert den gleichen durchschnittlichen Punktezuwachs (in Punkten/cm). Zeige das für die Intervalle $[500\,\text{cm}; 584\,\text{cm}]$ und $[811\,\text{cm};895\,\text{cm}]$ durch Rechnung!
	
	Begründe mithilfe der untenstehenden Graphik, warum ein absolut gleicher Weitenzuwachs für größere Ausgangswerte mehr Punkte bringt als für kleinere Ausgangswerte!
	
	\begin{center}
		\resizebox{0.8\linewidth}{!}{\psset{xunit=0.0066cm,yunit=0.005cm,algebraic=true,dimen=middle,dotstyle=o,dotsize=5pt 0,linewidth=0.8pt,arrowsize=3pt 2,arrowinset=0.25}
\begin{pspicture*}(-120.06849315068379,-110.34782608693672)(1323.2876712328716,1773.7391304344762)
\multips(0,0)(0,200.0){10}{\psline[linestyle=dashed,linecap=1,dash=1.5pt 1.5pt,linewidth=0.4pt,linecolor=lightgray]{c-c}(0,0)(1323.2876712328716,0)}
\multips(0,0)(200.0,0){8}{\psline[linestyle=dashed,linecap=1,dash=1.5pt 1.5pt,linewidth=0.4pt,linecolor=lightgray]{c-c}(0,0)(0,1773.7391304344762)}
\psaxes[labelFontSize=\scriptstyle,xAxis=true,yAxis=true,xlabelFactor={\text{\,cm}},Dx=200.,Dy=200.,ticksize=-2pt 0,subticks=2]{->}(0,0)(0.,0.)(1323.2876712328716,1773.7391304344762)
\psplot[linewidth=1.2pt,plotpoints=200]{220.0000000684934}{1323.2876712328716}{0.14354*(x-220.0)^(1.4)}
\psline[linestyle=dashed,dash=4pt 2pt](0.,1312.194532507742)(895.,1312.194532507742)
\psline[linestyle=dashed,dash=4pt 2pt](895.,1312.194532507742)(895.,0.)
\psline[linestyle=dashed,dash=4pt 2pt](811.,0.)(811.,1089.4201785316532)
\psline[linestyle=dashed,dash=4pt 2pt](811.,1089.4201785316532)(0.,1089.4201785316532)
\psline[linestyle=dashed,dash=4pt 2pt](584.,0.)(584.,552.7323443269263)
\psline[linestyle=dashed,dash=4pt 2pt](584.,552.7323443269263)(0.,552.7323443269263)
\psline[linestyle=dashed,dash=4pt 2pt](500.,0.)(500.,382.8196181903659)
\psline[linestyle=dashed,dash=4pt 2pt](500.,382.8196181903659)(0.,382.8196181903659)
\begin{scriptsize}
\rput[tl](30.35616438356212,1700.1739130431847){Punkte}
\rput[tl](1000.8356164383522,80.60869565215339){Sprungweite}
\psdots[dotsize=3pt 0,dotstyle=*](220.,0.)
\rput[bl](210.95890410958873,28.608695652169658){$A$}
\end{scriptsize}
\end{pspicture*}}
	\end{center}
	
	\item Berechne die Wahrscheinlichkeit, dass ein Athlet die Weitsprungpunkte bei seinem Zehnkampf ohne Fehlversuch erhält!
	
	Durch bessere Trainingsmethoden kann dieser Wahrscheinlichkeitswert erhöht werden, indem die Fehlerquote von $1:20$ gesenkt wird, etwa auf $1:n$.
	
	Wenn unter $n$ Sprüngen nur ein Fehlversuch dabei ist, ergibt sich eine Erfolgsquote von $\frac{n-1}{n}$. Begründe damit, warum die oben genannte Wahrscheinlichkeit nie 1 sein kann!
	
						\end{enumerate}\leer
				
\antwort{\subsection{Lösungserwartung:}
\begin{enumerate}
	\item $f(895)\approx 1\,312$, $f(811)\approx 1\,089$. Er hätte um 223 Punkte mehr erzielt.
	
	Die Basis (der Radikand) wird erst ab $220\,\text{cm}\geq 0$. (oder eine sinngemäßge Formulierung.
	\item $\frac{f(584)-f(500)}{584-500}$ bzw. $\frac{f(895)-f(811)}{895-811}$
	
	im ersten Intervall: ca. 2,02 Punkte/cm
	
	im zweiten Intervall: ca. 2,65 Punkte/cm
	
	Begründung: $f$ ist streng monoton wachsend und steig im zweiten Intervall schneller. (Jede sinngemäß formulierte Antwort ist richtig.)
	
	\item $X$ ... Anzahl der Fehlversuche
	
	$p=\frac{1}{20}$, $q=p-1=\frac{19}{20}$
	
		$P(X=0)=...=\left(\frac{19}{20}\right)^3=0,857$, d.h. mit einer Wahrscheinlichkeit von 85,7\,\%.
	
	Begründung: Da der Zähler immer kleiner als der Nenner ist, ist $\frac{n-1}{n}<1$. Daher muss auch die 3. Potenz $<1$ sein.
	
	Oder:
	
	Sobald die Wahrscheinlichkeit für einen Fehlversuch größer als 0 ist, muss die Wahrscheinlichkeit, dass 3 Sprünge ohne Fehlversuch gelingen, kleiner als 1 sein. (Sinngemäße Argumentation möglich!)
		\end{enumerate}}
		\end{langesbeispiel}