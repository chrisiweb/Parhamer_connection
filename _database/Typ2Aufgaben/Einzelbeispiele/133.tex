\section{133 - K6 - WS 2.3, WS-L 2.5 - Zoll - VerSie}

\begin{langesbeispiel} \item[4] %PUNKTE DES BEISPIELS
Am Zollübergang schmuggelt durchschnittlich 1 von tausend Personen Zigaretten. Bei der Zollbehörde arbeitet Fido seit 3 Jahren als Spürhund. Er ist auf das Aufspüren von Zigaretten spezialisiert und bellt mit einer Wahrscheinlichkeit von $0,95$, wenn er die Inhaltsstoffe von Zigaretten riecht. Aus früheren Fällen weiß sein Besitzer aber, dass er auch in 5\,\% der Fälle bellt, wenn keine Zigaretten versteckt sind.%Aufgabentext

\begin{aufgabenstellung}
\item %Aufgabentext

\Subitem{Zeichne ein Baumdiagramm das diesen Sachverhalt darstellt.} %Unterpunkt1
\Subitem{Markiere den Idealfall (Fido bellt und es sind tatsächlich Zigaretten versteckt).} %Unterpunkt2

\item %Aufgabentext

\Subitem{Berechne die Wahrscheinlichkeit, dass der Hund bei einer beliebig ausgesuchten Personen anschlägt (bellt).} %Unterpunkt1
\Subitem{Fido bellt bei einem ankommenden Grenzgänger. Wie sicher kann der Zollbeamter sein, dass der Grenzgänger ein Zigarettenschmuggler ist?} %Unterpunkt2

\end{aufgabenstellung}

\begin{loesung}
\item \subsection{Lösungserwartung:} 

\Subitem{\begin{center}
\resizebox{0.8\linewidth}{!}{\Huge
\psset{xunit=1.0cm,yunit=1.0cm,algebraic=true,dimen=middle,dotstyle=o,dotsize=5pt 0,linewidth=0.8pt,arrowsize=3pt 2,arrowinset=0.25}
\begin{pspicture*}(-16.570738647589398,5.491882395363719)(12.343075289784963,16.418622163615545)
\rput[tl](-4.1,15){\fbox{Start}}
\rput[tl](-11.6,11.5){\fbox{Schmuggler}}
\rput[tl](2.2,11.5){\fbox{kein Schmuggler}}
\rput[tl](-15.5,7.6){\fbox{Fido bellt}}
\rput[tl](-7.8,7.6){\fbox{Fido bellt nicht}}
\rput[tl](-1.3,7.6){\fbox{Fido bellt}}
\rput[tl](5.909684973977872,7.6){\fbox{Fido bellt nicht}}
\psline{->}(-4.524455870295381,14.04519098444609)(-10.044781720361687,12.)
\psline{->}(-1.6136440467856739,14.04519098444609)(4.,12.)
\psline{->}(-12.047784891058932,10.328308194426004)(-14.,8.)
\psline{->}(-7.927866617783654,10.373089914787691)(-6.,8.)
\psline{->}(1.8793301414259747,10.283526474064315)(0.,8.)
\psline{->}(6.088811855424628,10.283526474064315)(8.,8.)
\rput[tl](-8.19655693995378,14.089972704807778){0,001}
\rput[tl](0.6702236916604013,14.04519098444609){0,999}
\rput[tl](-14.510779510951759,9.9700544315325){0,95}
\rput[tl](-6.7187601680180835,9.835709270447436){0,05}
\rput[tl](-0.44931931738179337,9.9700544315325){0,05}
\rput[tl](7.253136584828506,9.79092755008575){0,95}
\end{pspicture*}}
\normalsize
\end{center}} %Lösung von Unterpunkt1
\setcounter{subitemcounter}{1}
\Subitem{Der Idealfall ist der ganz linke Pfad.} %%Lösung von Unterpunkt2

\setcounter{subitemcounter}{0}
\subsection{Lösungsschlüssel:}
 
\Subitem{Ein Punkt für das richtige Einzeichnen des Wahrscheinlichkeitsbaums.} %Lösungschlüssel von Unterpunkt1
\Subitem{Ein Punkt für das richtige Einzeichnen des Pfades.} %Lösungschlüssel von Unterpunkt2

\item \subsection{Lösungserwartung:} 

\Subitem{$P(\text{Fido bellt})=0,001\cdot 0,95+0,999\cdot 0,05=0,0509$
	
	Die Wahrscheinlichkeit beträgt 5,09\,\%.} %Lösung von Unterpunkt1
\Subitem{$P(\text{Schmuggler}\mid \text{Fido bellt})=\dfrac{0,001\cdot 0,95}{0,001\cdot 0,95+0,999\cdot 0,95}=0,01866404715128$} %%Lösung von Unterpunkt2

\setcounter{subitemcounter}{0}
\subsection{Lösungsschlüssel:}
 
\Subitem{Ein Punkt für die richtige Wahrscheinlichkeit.} %Lösungschlüssel von Unterpunkt1
\Subitem{Ein Punkt für die korrekte Berechnung der Wahrscheinlichkeit.} %Lösungschlüssel von Unterpunkt2

\end{loesung}

\end{langesbeispiel}