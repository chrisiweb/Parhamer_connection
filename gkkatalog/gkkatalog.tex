
\documentclass[a4paper,12pt]{article}
\usepackage{geometry}
\geometry{a4paper,left=18mm,right=18mm, top=2cm, bottom=2cm}
 

\usepackage{lmodern}
\usepackage[T1]{fontenc}
\usepackage[utf8]{inputenc}
\usepackage[ngerman]{babel}
\usepackage{srdp-mathematik} % solution_on/off, random, info_on/off
\usepackage{bookmark}
\setcounter{tocdepth}{3}

\pagestyle{plain} %PAGESTYLE: empty, plain
\onehalfspacing %Zeilenabstand
\setcounter{secnumdepth}{-1} % keine Nummerierung der Überschriften
%
%
%%%%%%%%%%%%%%%%%%%%%%%%%%%%%%%%%%%%%%%%%%%%%%%%%%%%%%%%%%%%%%%%%
%%%%%%%%%%%%%%%%%%%%%% DOKUMENT - ANFANG %%%%%%%%%%%%%%%%%%%%%%%%
%%%%%%%%%%%%%%%%%%%%%%%%%%%%%%%%%%%%%%%%%%%%%%%%%%%%%%%%%%%%%%%%%
%
%

\begin{document}
\section{Grundkompetenzkatalog} 

\subsection{Algebra und Geometrie (AG)}

\subsubsection{Grundbegriffe der Algebra (Reifeprüfung)}

\begin{tabular}{cp{0.85\linewidth}}
AG 1.1 & Wissen über die Zahlenmengen $\mathbb{N}, \mathbb{Z}, \mathbb{Q}, \mathbb{R}, \mathbb{C}$ verständig einsetzen können \\

AG 1.2 &Wissen über algebraische Begriffe angemessen einsetzen können: Variable, Terme, Formeln, (Un-)Gleichungen, Gleichungssysteme; Äquivalenz, Umformungen, Lösbarkeit\\
\end{tabular}

%AG 1.1 -- Wissen über die Zahlenmengen $\mathbb{N}, \mathbb{Z}, \mathbb{Q}, \mathbb{R}, \mathbb{C}$ verständig einsetzen können 
%
%AG 1.2 -- Wissen über algebraische Begriffe angemessen einsetzen können: Variable, Terme, Formeln, (Un-)Gleichungen, Gleichungssysteme; Äquivalenz, Umformungen, Lösbarkeit



\subsubsection{Grundbegriffe der Algebra (Lehrplan)}
\begin{em}
\begin{tabular}{cp{0.85\linewidth}}
AG 1.3-L & Mit Aussagen und Mengen umgehen können \\

AG 1.4-L & Zahlen in einem nichtdekadischen Zahlensystem darstellen können \\

AG 1.5-L & Komplexe Zahlen in der Gaußschen Zahlenebene darstellen und mit komplexen Zahlen rechnen können \\
\end{tabular}
\end{em}

\subsubsection{(Un-)Gleichungen und Gleichungssysteme (Reifeprüfung)}

\begin{tabular}{cp{0.85\linewidth}}
AG 2.1 &Einfache Terme und Formeln aufstellen, umformen und im Kontext deuten können\\

AG 2.2 & Lineare Gleichungen aufstellen, interpretieren, umformen/lösen und die Lösung im Kontext deuten können \\

AG 2.3 & Quadratische Gleichungen in einer Variablen umformen/lösen, über Lösungsfälle Bescheid wissen, Lösungen und Lösungsfälle (auch geometrisch) deuten können \\

AG 2.4 & Lineare Ungleichungen aufstellen, interpretieren, umformen/lösen, Lösungen (auch geometrisch) deuten können\\

AG 2.5 & Lineare Gleichungssysteme in zwei Variablen aufstellen, interpretieren, umformen/lösen, über Lösungsfälle Bescheid wissen, Lösungen und Lösungsfälle (auch geometrisch) deuten können \\
\end{tabular}


\subsubsection{(Un-)Gleichungen und Gleichungssysteme (Lehrplan)}
\begin{em}
\begin{tabular}{cp{0.85\linewidth}}
AG 2.6-L & Den Satz von Vieta kennen und anwenden können\\

AG 2.7-L & Lineare Gleichungssysteme in drei Variablen lösen können \\

AG 2.8-L & Den Fundamentalsatz der Algebra kennen und seine Bedeutung bei der Zahlenbereichserweiterung von $\mathbb{R}$ auf $\mathbb{C}$ erläutern können\\
\end{tabular}
\end{em}

\subsubsection{Vektoren und analytische Geometrie (Reifeprüfung)}
\begin{tabular}{cp{0.85\linewidth}}
AG 3.1 & Vektoren als Zahlentupel verständig einsetzen und im Kontext deuten können\\

AG 3.2 & Vektoren geometrisch (als Punkte bzw. Pfeile) deuten und verständig einsetzen können\\

AG 3.3 & Definition der Rechenoperationen mit Vektoren (Addition, Multiplikation mit einem Skalar, Skalarmultiplikation) kennen, Rechenoperationen verständig einsetzen und (auch geometrisch) deuten können\\

AG 3.4 & Geraden durch (Parameter-)Gleichungen in $\mathbb{R}^2$ und $\mathbb{R}^3$ angeben können; Geradengleichungen interpretieren können; Lagebeziehungen (zwischen Geraden und zwischen Punkt und Gerade) analysieren, Schnittpunkte ermitteln können\\

AG 3.5 & Normalvektoren in $\mathbb{R}^2$ aufstellen, verständig einsetzen und interpretieren können\\
\end{tabular}

\subsubsection{Vektoren und analytische Geometrie (Lehrplan)}
\begin{em}
\begin{tabular}{cp{0.85\linewidth}}
AG 3.6-L & Die geometrische Bedeutung des Skalarprodukts kennen und den Winkel zwischen zwei Vektoren ermitteln können\\

AG 3.7-L & Einheitsvektoren ermitteln, verständig einsetzen und interpretieren können\\

AG 3.8-L & Definition des vektoriellen Produkts und seine geometrische Bedeutung kennen\\

AG 3.9-L & Wissen, wodurch Ebenen festgelegt sind; Ebenen in Parameter- und Normalvektordarstellung aufstellen können\\
\end{tabular}
\end{em}

\subsubsection{Trigonometrie (Reifeprüfung)}
\begin{tabular}{cp{0.85\linewidth}}
AG 4.1 & Definitionen von Sinus, Cosinus, Tangens im rechtwinkligen Dreieck kennen und zur Auflösung rechtwinkliger Dreiecke einsetzen können\\

AG 4.2 & Definitionen von Sinus, Cosinus für Winkel größer als 90\degre\ kennen und einsetzen können\\
\end{tabular}

\subsubsection{Trigonometrie (Lehrplan)}
\begin{em}
\begin{tabular}{cp{0.85\linewidth}}
AG 4.3-L & Einfache Berechnungen an allgemeinen Dreiecken, an Figuren und Körpern (auch mittels Sinus- und Cosinussatz) durchführen können\\

AG 4.4-L & Polarkoordinaten kennen und einsetzen können\\
\end{tabular}
\end{em}

\subsubsection{Nichtlineare analytische Geometrie (Lehrplan)}
\begin{em}
\begin{tabular}{cp{0.85\linewidth}}
AG 5.1-L &Kegelschnitte in der Ebene durch Gleichungen beschreiben können; aus einer Kreisgleichung Mittelpunkt und Radius bestimmen können\\
AG 5.2-L & Die gegenseitige Lage von Kegelschnitt und Gerade ermitteln können\\
AG 5.3-L & Kugeln durch Gleichungen beschreiben können \\
\end{tabular}
\end{em}

\newpage

\subsection{Funktionale Abhängigkeiten (FA)}

\subsubsection{Funktionsbegriff, reelle Funktionen, Darstellungsformen und Eigenschaften (Reifeprüfung)}

\begin{tabular}{cp{0.85\linewidth}}
FA 1.1 & Für gegebene Zusammenhänge entscheiden können, ob man sie als Funktionen betrachten kann\\
FA 1.2 & Formeln als Darstellung von Funktionen interpretieren und den Funktionstyp zuordnen können \\
FA 1.3 & Zwischen tabellarischen und grafischen Darstellungen funktionaler Zusammenhänge wechseln können \\
FA 1.4 & Aus Tabellen, Graphen und Gleichungen von Funktionen Werte(paare) ermitteln und im Kontext deuten können\\
FA 1.5 & Eigenschaften von Funktionen erkennen, benennen, im Kontext deuten und zum Erstellen von Funktionsgraphen einsetzen können:Monotonie, Monotoniewechsel (lokale Extrema), Wendepunkte,Periodizität, Achsensymmetrie, asymptotisches Verhalten, Schnittpunkte mit den Achsen \\
FA 1.6 & Schnittpunkte zweier Funktionsgraphen grafisch und rechnerisch ermitteln und im Kontext interpretieren können \\
FA 1.7 & Funktionen als mathematische Modelle verstehen und damit verständig arbeiten können \\
FA 1.8 & Durch Gleichungen (Formeln) gegebene Funktionen mit mehreren Veränderlichen im Kontext deuten können, Funktionswerte ermitteln können \\
FA 1.9 & Einen Überblick über die wichtigsten (unten angeführten) Typen mathematischer Funktionen geben, ihre Eigenschaften vergleichen können \\
\end{tabular}


\subsubsection{Lineare Funktion $f(x) = k\cdot x +d$ (Reifeprüfung)}

\begin{tabular}{cp{0.85\linewidth}}
FA 2.1 & Verbal, tabellarisch, grafisch oder durch eine Gleichung (Formel) gegebene lineare Zusammenhänge als lineare Funktionen erkennen bzw. betrachten können; zwischen diesen Darstellungsformen wechseln können \\
FA 2.2 & Aus Tabellen, Graphen und Gleichungen linearer Funktionen Werte(paare) sowie die Parameter $k$ und $d$ ermitteln und im Kontext deuten können \\
FA 2.3 & Die Wirkung der Parameter $k$ und $d$ kennen und die Parameter in unterschiedlichen Kontexten deuten können\\
FA 2.4 & Charakteristische Eigenschaften von lineare Funktionen kennen und im Kontext deuten können:

$f(x+1)=f(x)+k; \frac{f(x_2-f(x_1}{x_2-x_1}=k=f'(x)$ \\
FA 2.5 & Die Angemessenheit einer Beschreibung mittels linearer Funktion bewerten können\\
FA 2.6 & Direkte Proportionalität als lineare Funktion vom Typ $f(x) = k\cdot x$ beschreiben können\\
\end{tabular}

\subsubsection{Potenzfunktionen $f(x)=a \cdot x^z + b, z\in \mathbb{Z}$ oder $f(x)=a\cdot x^{\frac{1}{2}}+b$ (Reifeprüfung)}

\begin{tabular}{cp{0.85\linewidth}}
FA 3.1 & Verbal, tabellarisch, grafisch oder durch eine Gleichung (Formel) gegebene Zusammenhänge dieser Art als entsprechende Potenzfunktionen erkennen bzw. betrachten können; zwischen diesen Darstellungsformen wechseln können \\
FA 3.2 & Aus Tabellen, Graphen und Gleichungen von Potenzfunktionen Werte(paare) sowie die Parameter $a$ und $b$ ermitteln und im Kontext deuten können \\
FA 3.3 & Die Wirkung der Parameter $a$ und $b$ bei Potenzfunktionen kennen und die Parameter im Kontext deuten können\\
FA 3.4 & Indirekte Proportionalität als Potenzfunktion vom Typ $f(x)=\frac{a}{x}$ beschreiben können
\end{tabular}

\subsubsection{Polynomfunktionen $f(x)=\sum_{i=0}^n a_i \cdot x^i$ mit $n \in \mathbb{N}$ (Reifeprüfung)}

\begin{tabular}{cp{0.85\linewidth}}
FA 4.1 & Typische Verläufe von Graphen in Abhängigkeit vom Grad der Polynomfunktion (er)kennen \\
FA 4.2 & Zwischen tabellarischen und grafischen Darstellungen von Zusammenhängen dieser Art wechseln können \\
FA 4.3 & Aus Tabellen, Graphen und Gleichungen von Polynomfunktionen Funktionswerte, aus Tabellen und Graphen sowie aus einer quadratischen Funktionsgleichung Argumentwerte ermitteln können\\
FA 4.4 & Den Zusammenhang zwischen dem Grad der Polynomfunktion und der Anzahl der Null-, Extrem- und Wendestellen wissen\\
\end{tabular}


\subsubsection{Exponentialfunktion $f(x)=a\cdot b^x$ bzw. $f(x)=a \cdot e^{\lambda \cdot x}$ mit $a,b \in \mathbb{R}^+, \lambda \in \mathbb{R}$ (Reifeprüfung)}

\begin{tabular}{cp{0.85\linewidth}}
FA 5.1 & Verbal, tabellarisch, grafisch oder durch eine Gleichung (Formel) gegebene exponentielle Zusammenhänge als Exponentialfunktion erkennen bzw. betrachten können; zwischen diesen Darstellungsformen wechseln können\\
FA 5.2 & Aus Tabellen, Graphen und Gleichungen von Exponentialfunktionen Werte(paare) ermitteln und im Kontext deuten können\\
FA 5.3 & Die Wirkung der Parameter $a$ und $b$ (bzw. $e^\lambda$) kennen und die Parameter in unterschiedlichen Kontexten deuten können\\
FA 5.4 & Charakteristische Eigenschaften ($f(x+1)=b\cdot f(x); (e^x)'=e^x$) von Exponentialfunktionen kennen und im Kontext deuten können \\
FA 5.5 & Die Begriffe "`Halbwertszeit"' und "`Verdoppelungszeit"' kennen, die entsprechenden Werte berechnen und im Kontext deuten können\\
FA 5.6 & Die Angemessenheit einer Beschreibung mittels Exponentialfunktion bewerten können \\
\end{tabular}


\subsubsection{Sinusfunktion, Cosinusfunktion (Reifeprüfung)}
\begin{tabular}{cp{0.85\linewidth}}
FA 6.1 & Grafisch oder durch eine Gleichung (Formel) gegebene Zusammenhänge der Art $f(x) = a\cdot \sin(b\cdot x)$ als Allgemeine Sinusfunktion erkennen bzw.betrachten können; zwischen diesen Darstellungsformen wechseln können\\
FA 6.2 & Aus Graphen und Gleichungen von allgemeinen Sinusfunktionen Werte(paare) ermitteln und im Kontext deuten können\\
FA 6.3 & Die Wirkung der Parameter $a$ und $b$ bei Winkelfunktionen kennen und die Parameter im Kontext deuten können\\
FA 6.4 & Periodizität als charakteristische Eigenschaft kennen und im Kontext deuten können\\
FA 6.5 & Wissen, dass $\cos(x)=\sin(x+\frac{\pi}{2})$\\
FA 6.6 & Wissen, dass gilt: $\sin(x)'=\cos(x)$ und $\cos(x)'=-\sin(x)$
\end{tabular}

\subsubsection{Folgen (Lehrplan)}
\begin{em}
\begin{tabular}{cp{0.85\linewidth}}
FA 7.1-L & Zahlenfolgen (insbesondere arithmetische und geometrische Folgen) durch explizite und rekursive Bildungsgesetze beschreiben und graphisch darstellen können\\
FA 7.2-L & Zahlenfolgen als Funktionen über $\mathbb{N}$ bzw. $\mathbb{N}^*$ auffassen können, insbesondere arithmetische Folgen als lineare Funktionen und geometrische Folgen als Exponentialfunktionen \\
FA 7.3-L & Definitionen monotoner und beschränkter Folgen kennen und anwenden können\\
FA 7.4-L & Grenzwerte von einfachen Folgen ermitteln können\\
\end{tabular}

\end{em}

\subsubsection{Reihen (Lehrplan)}

\begin{em}
\begin{tabular}{cp{0.85\linewidth}}
FA 8.1-L & Endliche arithmetische und geometrische Reihen kennen und ihre Summen berechnen können\\
FA 8.2-L & Den Begriff der Summe einer unendlichen Reihe definieren können\\
FA 8.3-L & Summen konvergenter geometrischer Reihen berechnen können\\
FA 8.4-L & Folgen und Reihen zur Beschreibung diskreter Prozesse in anwendungsorientierten Bereichen einsetzen können\\
\end{tabular}

\end{em}

\newpage

\subsection{Analysis (AN)}

\subsubsection{Änderungsmaße (Reifeprüfung)}
\begin{tabular}{cp{0.85\linewidth}}
AN 1.1 & Absolute und relative (prozentuelle) Änderungsmaße unterscheiden und angemessen verwenden können\\
AN 1.2 & Den Zusammenhang Differenzenquotient (mittlere Änderungsrate – Differentialquotient ("`momentane"' Änderungsrate) auf der Grundlage eines intuitiven Grenzwertbegriffes kennen und damit (verbal und auch in formaler Schreibweise) auch kontextbezogen anwenden können\\
AN 1.3 & Den Differenzen- und Differentialquotienten in verschiedenen Kontexten deuten und entsprechende Sachverhalte durch den Differenzen- bzw. Differentialquotienten beschreiben können\\
AN 1.4 & Das systemdynamische Verhalten von Größen durch Differenzengleichungen beschreiben bzw. diese im Kontext deuten können \\
\end{tabular}

\subsubsection{Änderungsmaße (Lehrplan)}
\begin{em}
\begin{tabular}{cp{0.85\linewidth}}
AN 1.5-L & Einfache Differentialgleichungen, insbesondere $f'(x)= k\cdot f(x)$, lösen können\\
\end{tabular}
\end{em}


\subsubsection{Regeln für das Differenzieren (Reifeprüfung)}
\begin{tabular}{cp{0.85\linewidth}}
AN 2.1 & Einfache Regeln des Differenzierens kennen und anwenden können: Potenzregel, Summenregel, Regeln für $k\cdot f(x)'$ und $f(k\cdot x)'$\\
\end{tabular}


\subsubsection{Regeln für das Differenzieren (Lehrplan)}
\begin{em}
\begin{tabular}{cp{0.85\linewidth}}
AN 2.2-L & Kettenregel kennen und anwenden können
\end{tabular}
\end{em}


\subsubsection{Ableitungsfunktion/Stammfunktion (Reifeprüfung)}
\begin{tabular}{cp{0.85\linewidth}}
AN 3.1 & Den Begriff Ableitungsfunktion/Stammfunktion kennen und zur Beschreibung von Funktionen einsetzen können\\
AN 3.2 & Den Zusammenhang zwischen Funktion und Ableitungsfunktion (bzw. Funktion und Stammfunktion) in deren grafischer Darstellung erkennen und beschreiben können\\
AN 3.3 & Eigenschaften von Funktionen mithilfe der Ableitung(sfunktion) beschreiben können: Monotonie, lokale Extrema, Links- und Rechtskrümmung, Wendestellen\\  
\end{tabular}


\subsubsection{Ableitungsfunktion/Stammfunktion (Lehrplan)}

\begin{em}
\begin{tabular}{cp{0.85\linewidth}}
AN 3.4-L &  Zielfunktionen in einer Variablen für Optimierungsaufgaben (Extremwertaufgaben) aufstellen und globale Extremstellen ermitteln können\\
\end{tabular}
\end{em}

\subsubsection{Summation und Integral (Reifeprüfung)}
\begin{tabular}{cp{0.85\linewidth}}
AN 4.1 & Den Begriff des bestimmten Integrals als Grenzwert einer Summe von Produkten deuten und beschreiben können\\
AN 4.2 & Einfache Regeln des Integrierens kennen und anwenden können: Potenzregel, Summenregel, Regeln für $\int k\cdot f(x)\dx$, $\int f(k\cdot x)\dx$; bestimmte Integrale von Polynomfunktionen ermitteln können \\
AN 4.3 & Das bestimmte Integral in verschiedenen Kontexten deuten und entsprechende Sachverhalte durch Integrale beschreiben können\\
\end{tabular}


\newpage

\subsection{Wahrscheinlichkeit und Statistik (WS)}

\subsubsection{Beschreibende Statistik (Reifeprüfung)}
\begin{tabular}{cp{0.85\linewidth}}
WS 1.1 & Werte aus tabellarischen und elementaren grafischen Darstellungen ablesen (bzw. zusammengesetzte Werte ermitteln) und im jeweiligen Kontext angemessen interpretieren können\\
WS 1.2 & Tabellen und einfache statistische Grafiken erstellen, zwischen Darstellungsformen wechseln können\\
WS 1.3 & Statistische Kennzahlen (absolute und relative Häufigkeiten; arithmetisches Mittel, Median, Modus; Quartile; Spannweite, empirische Varianz/Standardabweichung) im jeweiligen Kontext interpretieren können; die angeführten Kennzahlen für einfache Datensätze ermitteln können\\
WS 1.4 & Definition und wichtige Eigenschaften des arithmetischen Mittels und des Medians angeben und nutzen, Quartile ermitteln und interpretieren können, die Entscheidung für die Verwendung einer bestimmten Kennzahl begründen können\\
\end{tabular}


\subsubsection{Wahrscheinlichkeitsrechnung -- Grundbegriffe (Reifeprüfung)}
\begin{tabular}{cp{0.85\linewidth}}
WS 2.1 & Grundraum und Ereignisse in angemessenen Situationen verbal bzw. formal angeben können\\
WS 2.2 & Relative Häufigkeit als Schätzwert von Wahrscheinlichkeit verwenden und anwenden können\\
WS 2.3 & Wahrscheinlichkeit unter der Verwendung der Laplace-Annahme (Laplace Wahrscheinlichkeit) berechnen und interpretieren können, Additionsregel und Multiplikationsregel anwenden und interpretieren können\\
WS 2.4 & Binomialkoeffizient berechnen und interpretieren können\\
\end{tabular}


\subsubsection{Wahrscheinlichkeitsrechnung -- Grundbegriffe (Lehrplan)}
\begin{em}
\begin{tabular}{cp{0.85\linewidth}}
WS 2.5-L & Bedingte Wahrscheinlichkeiten kennen, berechnen und interpretieren können\\
WS 2.6-L & Entscheiden können, ob ein Ereignis von einem anderen Ereignis abhängt oder von diesem unabhängig ist\\
\end{tabular}
\end{em}


\subsubsection{Wahrscheinlichkeitsverteilung(en) (Reifeprüfung)}
\begin{tabular}{cp{0.85\linewidth}}
WS 3.1 & Die Begriffe Zufallsvariable, (Wahrscheinlichkeits-)Verteilung, Erwartungswert und Standardabweichung verständig deuten und einsetzen können\\
WS 3.2 & Binomialverteilung als Modell einer diskreten Verteilung kennen -- Erwartungswert sowie Varianz/Standardabweichung binomialverteilter Zufallsgrößen ermitteln können, Wahrscheinlichkeitsverteilung binomialverteilter Zufallsgrößen angeben können, Arbeiten mit der Binomialverteilung in anwendungsorientierten Bereichen\\
WS 3.3 & Situationen erkennen und beschreiben können, in denen mit Binomialverteilung modelliert werden kann\\
WS 3.4 & Normalapproximation der Binomialverteilung interpretieren und anwenden können\\
\end{tabular}

\subsubsection{Wahrscheinlichkeitsverteilung(en) (Lehrplan)}
\begin{em}
\begin{tabular}{cp{0.85\linewidth}}
WS 3.5-L & Mit der Normalverteilung, auch in anwendungsorientierten Bereichen, arbeiten können\\
\end{tabular}
\end{em}


\subsubsection{Schließende/Beurteilende Statistik (Reifeprüfung)}

\begin{tabular}{cp{0.85\linewidth}}
WS 4.1 & Konfidenzintervalle als Schätzung für eine Wahrscheinlichkeit oder einen unbekannten Anteil $p$ interpretieren (frequentistische Deutung) und verwenden können, Berechnungen auf Basis der Binomialverteilung oder einer durch die Normalverteilung approximierten Binomialverteilung durchführen können\\
\end{tabular}

\subsubsection{Schließende/Beurteilende Statistik (Lehrplan)}
\begin{em}
\begin{tabular}{cp{0.85\linewidth}}
WS 4.2-L & Einfache Anteilstests durchführen können und ihr Ergebnis erläutern können\\
\end{tabular}
\end{em}

\end{document}